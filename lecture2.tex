\section{Lecture 2: Basic Mathematics II}

\subsection{Summation Operator}

\begin{itemize}
    \item Consider a set $X = \{x_1, x_2, ..., x_n\}$
    \begin{itemize}
        \item $\sum\limits_{i = 1}^{n} x_i = x_1 + x_2 + ... + x_n$
        \begin{itemize}
            \item ``The sum of $x_i$, over the range from $i=l$ through $i = n$.''
        \end{itemize}
    \end{itemize}
    \item Let $Y = \{y_1, y_2, ..., y_n\}$ 
    \begin{itemize}
        \item $\sum\limits_{i=1}^n x_iy_i = x_1y_1 + x_2y_2 + ... + x_ny_n$ 
        \begin{itemize}
            \item Here, we sum the outcome of $x_iy_i$ $n$ times.
        \end{itemize}
    \end{itemize}
    \item But, what if there is more than one summation operator and subscript?
    \vspace{-1em}
    \begin{align*}
        \Sigma_{j=1}^m \Sigma_{i=1}^n x_iy_j 
        & = (y_1x_1 + y_1x_2 + ... + y_1x_n)   \\
        & + (y_2x_1 + y_2x_2 + ... + y_2x_n) \\
        & + ... + (y_mx_1 + y_mx_2 + ... + y_mx_n)
    \end{align*}
    We have $m$ parentheses, with $n$ $x$'s in each parenthesis. We end up summing each combination of $x_i$ and $y_j$.
    \item What if there is a constant, $c$? Drawing upon the \emph{distributive property}:
    \begin{itemize}
        \item $\sum\limits_{i = 1}^{n} cx_i = cx_1 + cx_2 + ... + cx_n = c\sum\limits_{i = 1}^{n} x_i$
    \end{itemize}
    \item The \emph{associative property} can also be applied to summation:
    \begin{itemize}
        \item $\sum\limits_{i=1}^n (x_i + y_i) = (x_1 + y_1) + (x_2 + y_2)
            + ... + (x_n + y_n) = \sum\limits_{i=1}^n x_i + \sum\limits_{i=1}^n y_i$
    \end{itemize}
\end{itemize}

\subsection{Product Operator}

\begin{itemize}
    \item $X = \{x_1, x_2, ..., x_n\}$ and $Y = \{y_1, y_2, ..., y_n\}$ 
    \item $\prod\limits_{i=1}^n x_i = x_1 \cdot x_2 \cdot ... \cdot x_n$
    \item $\prod\limits_{i=1}^n (x_i + y_i) = (x_1 + y_1)(x_2 + y_2) ...
        (x_n + y_n)$
    \begin{itemize}
        \item We can't split $(x_i + y_i)$ like in summation. Instead, we multiply $(x_i + y_i)$ repeatedly.
    \end{itemize}
    \item What about a constant? 
    \begin{itemize}
        \item $\prod\limits_{i = 1}^{n} cx_i = (cx_1)(cx_2) \cdot ... \cdot
            (cx_n) = c^n\prod\limits_{i = 1}^{n} x_i$
    \end{itemize}
    We can move $c$ to the front, but we have to exponentiate it to $c^n$.
\end{itemize}

\subsection{Factorials, Permutations, and Combinations}

Most of our quantitative coursework is about modeling probabilities, where: 

\begin{equation*}
    \text{probability} = \frac{\# \text{occurences}}{\# \text{possibilities}}
\end{equation*}

\noindent For both the numerator and denominator we are dealing with \emph{counting} the number of relevant outcomes. Factorials, permutations, and combinations are foundational concepts when it comes to counting. 

\begin{itemize}
    \item Some useful illustrations/properties:
    \begin{itemize}
        \item $0! = 1$ 
        \item $x! = x \cdot (x-1) \cdot (x-2) \cdot ... \cdot 0!$
        \item $2! = 2 \cdot 1$
        \item $3! = 3 \cdot 2 \cdot 1$
        \item $10! = 10 \cdot 9 \cdot 8 \cdot 7 \cdot 6 \cdot ... \cdot 1$ 
        \item If $n$ objects in boxes of size $x$ where \emph{order matters}, then the number of \emph{permutations} is:
        \begin{equation*}
            \frac{n!}{(n - x)}
        \end{equation*}
        \item If order does not matter, then the number of \emph{combinations} is:
        \begin{equation*}
            \frac{n!}{(n - x)! x!} = n C x = {n \choose x }
        \end{equation*}
        \begin{itemize}
            \item Pronounced: `$n$ choose $x$'
        \end{itemize}
    \end{itemize}
\end{itemize}

\subsection{Solving equations, inequalities, and for roots}

\begin{itemize}
    \item Solving an equation example:
    \begin{align*}
        3x + 4y + 8 & = 0 \\
        4y & = - (3x + 8) \\
        y & = \frac{- (3x + 8)}{4} \\
        y & = -\frac{3}{4}x - 2
    \end{align*}
    
    \noindent We can also write this answer as:
    
    \begin{equation*}
        \{(x,y) \in \mathbb{R} | y = - \frac{3}{4} x - 2 \}
    \end{equation*}
    
    \item Solving an inequality ($x > $, $x \geq y$, $y < x$, $y \leq x$) example:
    \begin{align*}
        -4y & > 2x + 12 \\
        y & < -\frac{2x}{4} - \frac{12}{4} \\
        y & < \frac{x}{2} - 3
    \end{align*}

    \noindent Note that dividing by a negative flips the sign

    \item Solving for a quadratic
    \begin{itemize}
        \item Quadratic formula:
        \begin{align*}
            & ax^2 + bx + c \\
            & x \in \frac{-b \pm \sqrt{b^2 - 4ac}}{2a}
        \end{align*}
        \noindent Example:
        \begin{align*}
            & 1.4x^2 + 3.7x + 1.1 = 0 \\
            & a = 1.4, b = 3.7, c = 1.1 \\
            & x = \frac{-3.7 \pm \sqrt{3.7^2 -
                4 \times 1.4 \times 1.1}}{2.8} \\
            & x = - 0.341 \text{ or } x = -2.301 
        \end{align*}
    \end{itemize}
\end{itemize}

\subsection{Logarithms}

\begin{itemize}
    \item If $y = a^x$, then we can rewrite as $\text{log}_a y = x$
    \item Examples: 
    \begin{itemize}
        \item $\text{log}_e e = 1$, because $e^1 = e$ 
        \item Solving for $x$: 
        \begin{itemize}
            \item If $8 = 2^x$, then $\text{log}_2 8 = x = 3$
        \end{itemize}
    \end{itemize}
    \item \textbf{Rules}
    \begin{itemize}
        \item $\text{log}(m + n) = \text{log}(m) + \text{log}(n)$ 
        \item $\text{log}(\frac{m}{n}) = \text{log}(m) - \text{log}(n)$
        \item $\text{log}(b^a) = a \text{log} b$
        \item $\text{log}b^a = (\text{log}_be)(\text{log}_ea)$ 
        \item $\text{log}b^a = \frac{1}{\text{log}_ab}$
    \end{itemize}
    \item \textbf{Visual}
    
    \resizebox{4in}{!}{
        % Created by tikzDevice version 0.12 on 2019-06-02 23:07:08
% !TEX encoding = UTF-8 Unicode
\begin{tikzpicture}[x=1pt,y=1pt]
\definecolor{fillColor}{RGB}{255,255,255}
\path[use as bounding box,fill=fillColor,fill opacity=0.00] (0,0) rectangle (289.08,289.08);
\begin{scope}
\path[clip] (  0.00,  0.00) rectangle (289.08,289.08);
\definecolor{drawColor}{RGB}{255,255,255}
\definecolor{fillColor}{RGB}{255,255,255}

\path[draw=drawColor,line width= 0.6pt,line join=round,line cap=round,fill=fillColor] (  0.00,  0.00) rectangle (289.08,289.08);
\end{scope}
\begin{scope}
\path[clip] ( 16.80, 20.52) rectangle (283.58,283.58);
\definecolor{fillColor}{RGB}{255,255,255}

\path[fill=fillColor] ( 16.80, 20.52) rectangle (283.58,283.58);
\definecolor{drawColor}{RGB}{0,0,0}

\path[draw=drawColor,line width= 0.6pt,line join=round] (119.88, 32.48) --
	(119.92, 32.81) --
	(119.96, 33.14) --
	(120.00, 33.47) --
	(120.04, 33.79) --
	(120.08, 34.12) --
	(120.12, 34.44) --
	(120.16, 34.77) --
	(120.20, 35.09) --
	(120.24, 35.41) --
	(120.28, 35.73) --
	(120.32, 36.04) --
	(120.36, 36.36) --
	(120.40, 36.68) --
	(120.44, 36.99) --
	(120.48, 37.31) --
	(120.52, 37.62) --
	(120.56, 37.93) --
	(120.60, 38.24) --
	(120.64, 38.55) --
	(120.69, 38.85) --
	(120.73, 39.16) --
	(120.77, 39.47) --
	(120.81, 39.77) --
	(120.85, 40.07) --
	(120.89, 40.37) --
	(120.93, 40.67) --
	(120.97, 40.97) --
	(121.01, 41.27) --
	(121.05, 41.57) --
	(121.09, 41.87) --
	(121.13, 42.16) --
	(121.17, 42.46) --
	(121.21, 42.75) --
	(121.25, 43.04) --
	(121.29, 43.33) --
	(121.33, 43.62) --
	(121.37, 43.91) --
	(121.41, 44.20) --
	(121.45, 44.49) --
	(121.49, 44.77) --
	(121.53, 45.06) --
	(121.57, 45.34) --
	(121.61, 45.63) --
	(121.66, 45.91) --
	(121.70, 46.19) --
	(121.74, 46.47) --
	(121.78, 46.75) --
	(121.82, 47.03) --
	(121.86, 47.31) --
	(121.90, 47.58) --
	(121.94, 47.86) --
	(121.98, 48.13) --
	(122.02, 48.41) --
	(122.06, 48.68) --
	(122.10, 48.95) --
	(122.14, 49.22) --
	(122.18, 49.49) --
	(122.22, 49.76) --
	(122.26, 50.03) --
	(122.30, 50.30) --
	(122.34, 50.57) --
	(122.38, 50.83) --
	(122.42, 51.10) --
	(122.46, 51.36) --
	(122.50, 51.63) --
	(122.54, 51.89) --
	(122.58, 52.15) --
	(122.63, 52.41) --
	(122.67, 52.67) --
	(122.71, 52.93) --
	(122.75, 53.19) --
	(122.79, 53.45) --
	(122.83, 53.70) --
	(122.87, 53.96) --
	(122.91, 54.22) --
	(122.95, 54.47) --
	(122.99, 54.72) --
	(123.03, 54.98) --
	(123.07, 55.23) --
	(123.11, 55.48) --
	(123.15, 55.73) --
	(123.19, 55.98) --
	(123.23, 56.23) --
	(123.27, 56.48) --
	(123.31, 56.73) --
	(123.35, 56.97) --
	(123.39, 57.22) --
	(123.43, 57.47) --
	(123.47, 57.71) --
	(123.51, 57.95) --
	(123.55, 58.20) --
	(123.60, 58.44) --
	(123.64, 58.68) --
	(123.68, 58.92) --
	(123.72, 59.16) --
	(123.76, 59.40) --
	(123.80, 59.64) --
	(123.84, 59.88) --
	(123.88, 60.12) --
	(123.92, 60.36) --
	(123.96, 60.59) --
	(124.00, 60.83) --
	(124.04, 61.06) --
	(124.08, 61.30) --
	(124.12, 61.53) --
	(124.16, 61.76) --
	(124.20, 62.00) --
	(124.24, 62.23) --
	(124.28, 62.46) --
	(124.32, 62.69) --
	(124.36, 62.92) --
	(124.40, 63.15) --
	(124.44, 63.38) --
	(124.48, 63.61) --
	(124.53, 63.83) --
	(124.57, 64.06) --
	(124.61, 64.29) --
	(124.65, 64.51) --
	(124.69, 64.74) --
	(124.73, 64.96) --
	(124.77, 65.18) --
	(124.81, 65.41) --
	(124.85, 65.63) --
	(124.89, 65.85) --
	(124.93, 66.07) --
	(124.97, 66.29) --
	(125.01, 66.51) --
	(125.05, 66.73) --
	(125.09, 66.95) --
	(125.13, 67.17) --
	(125.17, 67.39) --
	(125.21, 67.60) --
	(125.25, 67.82) --
	(125.29, 68.04) --
	(125.33, 68.25) --
	(125.37, 68.47) --
	(125.41, 68.68) --
	(125.45, 68.90) --
	(125.50, 69.11) --
	(125.54, 69.32) --
	(125.58, 69.53) --
	(125.62, 69.75) --
	(125.66, 69.96) --
	(125.70, 70.17) --
	(125.74, 70.38) --
	(125.78, 70.59) --
	(125.82, 70.80) --
	(125.86, 71.00) --
	(125.90, 71.21) --
	(125.94, 71.42) --
	(125.98, 71.63) --
	(126.02, 71.83) --
	(126.06, 72.04) --
	(126.10, 72.24) --
	(126.14, 72.45) --
	(126.18, 72.65) --
	(126.22, 72.86) --
	(126.26, 73.06) --
	(126.30, 73.26) --
	(126.34, 73.47) --
	(126.38, 73.67) --
	(126.42, 73.87) --
	(126.47, 74.07) --
	(126.51, 74.27) --
	(126.55, 74.47) --
	(126.59, 74.67) --
	(126.63, 74.87) --
	(126.67, 75.07) --
	(126.71, 75.26) --
	(126.75, 75.46) --
	(126.79, 75.66) --
	(126.83, 75.86) --
	(126.87, 76.05) --
	(126.91, 76.25) --
	(126.95, 76.44) --
	(126.99, 76.64) --
	(127.03, 76.83) --
	(127.07, 77.03) --
	(127.11, 77.22) --
	(127.15, 77.41) --
	(127.19, 77.60) --
	(127.23, 77.80) --
	(127.27, 77.99) --
	(127.31, 78.18) --
	(127.35, 78.37) --
	(127.39, 78.56) --
	(127.44, 78.75) --
	(127.48, 78.94) --
	(127.52, 79.13) --
	(127.56, 79.32) --
	(127.60, 79.50) --
	(127.64, 79.69) --
	(127.68, 79.88) --
	(127.72, 80.07) --
	(127.76, 80.25) --
	(127.80, 80.44) --
	(127.84, 80.62) --
	(127.88, 80.81) --
	(127.92, 80.99) --
	(127.96, 81.18) --
	(128.00, 81.36) --
	(128.04, 81.55) --
	(128.08, 81.73) --
	(128.12, 81.91) --
	(128.16, 82.09) --
	(128.20, 82.28) --
	(128.24, 82.46) --
	(128.28, 82.64) --
	(128.32, 82.82) --
	(128.36, 83.00) --
	(128.41, 83.18) --
	(128.45, 83.36) --
	(128.49, 83.54) --
	(128.53, 83.72) --
	(128.57, 83.90) --
	(128.61, 84.07) --
	(128.65, 84.25) --
	(128.69, 84.43) --
	(128.73, 84.61) --
	(128.77, 84.78) --
	(128.81, 84.96) --
	(128.85, 85.13) --
	(128.89, 85.31) --
	(128.93, 85.48) --
	(128.97, 85.66) --
	(129.01, 85.83) --
	(129.05, 86.01) --
	(129.09, 86.18) --
	(129.13, 86.35) --
	(129.17, 86.53) --
	(129.21, 86.70) --
	(129.25, 86.87) --
	(129.29, 87.04) --
	(129.34, 87.21) --
	(129.38, 87.38) --
	(129.42, 87.56) --
	(129.46, 87.73) --
	(129.50, 87.90) --
	(129.54, 88.07) --
	(129.58, 88.23) --
	(129.62, 88.40) --
	(129.66, 88.57) --
	(129.70, 88.74) --
	(129.74, 88.91) --
	(129.78, 89.08) --
	(129.82, 89.24) --
	(129.86, 89.41) --
	(129.90, 89.58) --
	(129.94, 89.74) --
	(129.98, 89.91) --
	(130.02, 90.07) --
	(130.06, 90.24) --
	(130.10, 90.40) --
	(130.14, 90.57) --
	(130.18, 90.73) --
	(130.22, 90.90) --
	(130.26, 91.06) --
	(130.31, 91.22) --
	(130.35, 91.39) --
	(130.39, 91.55) --
	(130.43, 91.71) --
	(130.47, 91.87) --
	(130.51, 92.04) --
	(130.55, 92.20) --
	(130.59, 92.36) --
	(130.63, 92.52) --
	(130.67, 92.68) --
	(130.71, 92.84) --
	(130.75, 93.00) --
	(130.79, 93.16) --
	(130.83, 93.32) --
	(130.87, 93.48) --
	(130.91, 93.63) --
	(130.95, 93.79) --
	(130.99, 93.95) --
	(131.03, 94.11) --
	(131.07, 94.27) --
	(131.11, 94.42) --
	(131.15, 94.58) --
	(131.19, 94.74) --
	(131.23, 94.89) --
	(131.28, 95.05) --
	(131.32, 95.20) --
	(131.36, 95.36) --
	(131.40, 95.51) --
	(131.44, 95.67) --
	(131.48, 95.82) --
	(131.52, 95.98) --
	(131.56, 96.13) --
	(131.60, 96.28) --
	(131.64, 96.44) --
	(131.68, 96.59) --
	(131.72, 96.74) --
	(131.76, 96.90) --
	(131.80, 97.05) --
	(131.84, 97.20) --
	(131.88, 97.35) --
	(131.92, 97.50) --
	(131.96, 97.65) --
	(132.00, 97.81) --
	(132.04, 97.96) --
	(132.08, 98.11) --
	(132.12, 98.26) --
	(132.16, 98.41) --
	(132.20, 98.56) --
	(132.25, 98.70) --
	(132.29, 98.85) --
	(132.33, 99.00) --
	(132.37, 99.15) --
	(132.41, 99.30) --
	(132.45, 99.45) --
	(132.49, 99.59) --
	(132.53, 99.74) --
	(132.57, 99.89) --
	(132.61,100.03) --
	(132.65,100.18) --
	(132.69,100.33) --
	(132.73,100.47) --
	(132.77,100.62) --
	(132.81,100.76) --
	(132.85,100.91) --
	(132.89,101.05) --
	(132.93,101.20) --
	(132.97,101.34) --
	(133.01,101.49) --
	(133.05,101.63) --
	(133.09,101.78) --
	(133.13,101.92) --
	(133.18,102.06) --
	(133.22,102.21) --
	(133.26,102.35) --
	(133.30,102.49) --
	(133.34,102.63) --
	(133.38,102.78) --
	(133.42,102.92) --
	(133.46,103.06) --
	(133.50,103.20) --
	(133.54,103.34) --
	(133.58,103.48) --
	(133.62,103.62) --
	(133.66,103.76) --
	(133.70,103.90) --
	(133.74,104.04) --
	(133.78,104.18) --
	(133.82,104.32) --
	(133.86,104.46) --
	(133.90,104.60) --
	(133.94,104.74) --
	(133.98,104.88) --
	(134.02,105.01) --
	(134.06,105.15) --
	(134.10,105.29) --
	(134.15,105.43) --
	(134.19,105.57) --
	(134.23,105.70) --
	(134.27,105.84) --
	(134.31,105.98) --
	(134.35,106.11) --
	(134.39,106.25) --
	(134.43,106.38) --
	(134.47,106.52) --
	(134.51,106.66) --
	(134.55,106.79) --
	(134.59,106.93) --
	(134.63,107.06) --
	(134.67,107.20) --
	(134.71,107.33) --
	(134.75,107.46) --
	(134.79,107.60) --
	(134.83,107.73) --
	(134.87,107.87) --
	(134.91,108.00) --
	(134.95,108.13) --
	(134.99,108.26) --
	(135.03,108.40) --
	(135.07,108.53) --
	(135.12,108.66) --
	(135.16,108.79) --
	(135.20,108.93) --
	(135.24,109.06) --
	(135.28,109.19) --
	(135.32,109.32) --
	(135.36,109.45) --
	(135.40,109.58) --
	(135.44,109.71) --
	(135.48,109.84) --
	(135.52,109.97) --
	(135.56,110.10) --
	(135.60,110.23) --
	(135.64,110.36) --
	(135.68,110.49) --
	(135.72,110.62) --
	(135.76,110.75) --
	(135.80,110.88) --
	(135.84,111.01) --
	(135.88,111.14) --
	(135.92,111.26) --
	(135.96,111.39) --
	(136.00,111.52) --
	(136.04,111.65) --
	(136.09,111.77) --
	(136.13,111.90) --
	(136.17,112.03) --
	(136.21,112.15) --
	(136.25,112.28) --
	(136.29,112.41) --
	(136.33,112.53) --
	(136.37,112.66) --
	(136.41,112.79) --
	(136.45,112.91) --
	(136.49,113.04) --
	(136.53,113.16) --
	(136.57,113.29) --
	(136.61,113.41) --
	(136.65,113.54) --
	(136.69,113.66) --
	(136.73,113.79) --
	(136.77,113.91) --
	(136.81,114.03) --
	(136.85,114.16) --
	(136.89,114.28) --
	(136.93,114.40) --
	(136.97,114.53) --
	(137.01,114.65) --
	(137.06,114.77) --
	(137.10,114.90) --
	(137.14,115.02) --
	(137.18,115.14) --
	(137.22,115.26) --
	(137.26,115.39) --
	(137.30,115.51) --
	(137.34,115.63) --
	(137.38,115.75) --
	(137.42,115.87) --
	(137.46,115.99) --
	(137.50,116.11) --
	(137.54,116.23) --
	(137.58,116.35) --
	(137.62,116.47) --
	(137.66,116.59) --
	(137.70,116.71) --
	(137.74,116.83) --
	(137.78,116.95) --
	(137.82,117.07) --
	(137.86,117.19) --
	(137.90,117.31) --
	(137.94,117.43) --
	(137.99,117.55) --
	(138.03,117.67) --
	(138.07,117.79) --
	(138.11,117.91) --
	(138.15,118.02) --
	(138.19,118.14) --
	(138.23,118.26) --
	(138.27,118.38) --
	(138.31,118.49) --
	(138.35,118.61) --
	(138.39,118.73) --
	(138.43,118.85) --
	(138.47,118.96) --
	(138.51,119.08) --
	(138.55,119.20) --
	(138.59,119.31) --
	(138.63,119.43) --
	(138.67,119.54) --
	(138.71,119.66) --
	(138.75,119.78) --
	(138.79,119.89) --
	(138.83,120.01) --
	(138.87,120.12) --
	(138.91,120.24) --
	(138.96,120.35) --
	(139.00,120.47) --
	(139.04,120.58) --
	(139.08,120.69) --
	(139.12,120.81) --
	(139.16,120.92) --
	(139.20,121.04) --
	(139.24,121.15) --
	(139.28,121.26) --
	(139.32,121.38) --
	(139.36,121.49) --
	(139.40,121.60) --
	(139.44,121.72) --
	(139.48,121.83) --
	(139.52,121.94) --
	(139.56,122.05) --
	(139.60,122.17) --
	(139.64,122.28) --
	(139.68,122.39) --
	(139.72,122.50) --
	(139.76,122.61) --
	(139.80,122.73) --
	(139.84,122.84) --
	(139.88,122.95) --
	(139.93,123.06) --
	(139.97,123.17) --
	(140.01,123.28) --
	(140.05,123.39) --
	(140.09,123.50) --
	(140.13,123.61) --
	(140.17,123.72) --
	(140.21,123.83) --
	(140.25,123.94) --
	(140.29,124.05) --
	(140.33,124.16) --
	(140.37,124.27) --
	(140.41,124.38) --
	(140.45,124.49) --
	(140.49,124.60) --
	(140.53,124.71) --
	(140.57,124.82) --
	(140.61,124.93) --
	(140.65,125.04) --
	(140.69,125.14) --
	(140.73,125.25) --
	(140.77,125.36) --
	(140.81,125.47) --
	(140.85,125.58) --
	(140.90,125.68) --
	(140.94,125.79) --
	(140.98,125.90) --
	(141.02,126.01) --
	(141.06,126.11) --
	(141.10,126.22) --
	(141.14,126.33) --
	(141.18,126.43) --
	(141.22,126.54) --
	(141.26,126.65) --
	(141.30,126.75) --
	(141.34,126.86) --
	(141.38,126.97) --
	(141.42,127.07) --
	(141.46,127.18) --
	(141.50,127.28) --
	(141.54,127.39) --
	(141.58,127.49) --
	(141.62,127.60) --
	(141.66,127.70) --
	(141.70,127.81) --
	(141.74,127.91) --
	(141.78,128.02) --
	(141.83,128.12) --
	(141.87,128.23) --
	(141.91,128.33) --
	(141.95,128.44) --
	(141.99,128.54) --
	(142.03,128.64) --
	(142.07,128.75) --
	(142.11,128.85) --
	(142.15,128.95) --
	(142.19,129.06) --
	(142.23,129.16) --
	(142.27,129.26) --
	(142.31,129.37) --
	(142.35,129.47) --
	(142.39,129.57) --
	(142.43,129.68) --
	(142.47,129.78) --
	(142.51,129.88) --
	(142.55,129.98) --
	(142.59,130.08) --
	(142.63,130.19) --
	(142.67,130.29) --
	(142.71,130.39) --
	(142.75,130.49) --
	(142.80,130.59) --
	(142.84,130.69) --
	(142.88,130.80) --
	(142.92,130.90) --
	(142.96,131.00) --
	(143.00,131.10) --
	(143.04,131.20) --
	(143.08,131.30) --
	(143.12,131.40) --
	(143.16,131.50) --
	(143.20,131.60) --
	(143.24,131.70) --
	(143.28,131.80) --
	(143.32,131.90) --
	(143.36,132.00) --
	(143.40,132.10) --
	(143.44,132.20) --
	(143.48,132.30) --
	(143.52,132.40) --
	(143.56,132.50) --
	(143.60,132.60) --
	(143.64,132.70) --
	(143.68,132.79) --
	(143.72,132.89) --
	(143.77,132.99) --
	(143.81,133.09) --
	(143.85,133.19) --
	(143.89,133.29) --
	(143.93,133.38) --
	(143.97,133.48) --
	(144.01,133.58) --
	(144.05,133.68) --
	(144.09,133.78) --
	(144.13,133.87) --
	(144.17,133.97) --
	(144.21,134.07) --
	(144.25,134.17) --
	(144.29,134.26) --
	(144.33,134.36) --
	(144.37,134.46) --
	(144.41,134.55) --
	(144.45,134.65) --
	(144.49,134.75) --
	(144.53,134.84) --
	(144.57,134.94) --
	(144.61,135.04) --
	(144.65,135.13) --
	(144.69,135.23) --
	(144.74,135.32) --
	(144.78,135.42) --
	(144.82,135.51) --
	(144.86,135.61) --
	(144.90,135.71) --
	(144.94,135.80) --
	(144.98,135.90) --
	(145.02,135.99) --
	(145.06,136.09) --
	(145.10,136.18) --
	(145.14,136.28) --
	(145.18,136.37) --
	(145.22,136.46) --
	(145.26,136.56) --
	(145.30,136.65) --
	(145.34,136.75) --
	(145.38,136.84) --
	(145.42,136.94) --
	(145.46,137.03) --
	(145.50,137.12) --
	(145.54,137.22) --
	(145.58,137.31) --
	(145.62,137.40) --
	(145.66,137.50) --
	(145.71,137.59) --
	(145.75,137.68) --
	(145.79,137.78) --
	(145.83,137.87) --
	(145.87,137.96) --
	(145.91,138.06) --
	(145.95,138.15) --
	(145.99,138.24) --
	(146.03,138.33) --
	(146.07,138.43) --
	(146.11,138.52) --
	(146.15,138.61) --
	(146.19,138.70) --
	(146.23,138.79) --
	(146.27,138.89) --
	(146.31,138.98) --
	(146.35,139.07) --
	(146.39,139.16) --
	(146.43,139.25) --
	(146.47,139.34) --
	(146.51,139.43) --
	(146.55,139.53) --
	(146.59,139.62) --
	(146.64,139.71) --
	(146.68,139.80) --
	(146.72,139.89) --
	(146.76,139.98) --
	(146.80,140.07) --
	(146.84,140.16) --
	(146.88,140.25) --
	(146.92,140.34) --
	(146.96,140.43) --
	(147.00,140.52) --
	(147.04,140.61) --
	(147.08,140.70) --
	(147.12,140.79) --
	(147.16,140.88) --
	(147.20,140.97) --
	(147.24,141.06) --
	(147.28,141.15) --
	(147.32,141.24) --
	(147.36,141.33) --
	(147.40,141.42) --
	(147.44,141.50) --
	(147.48,141.59) --
	(147.52,141.68) --
	(147.56,141.77) --
	(147.61,141.86) --
	(147.65,141.95) --
	(147.69,142.04) --
	(147.73,142.12) --
	(147.77,142.21) --
	(147.81,142.30) --
	(147.85,142.39) --
	(147.89,142.48) --
	(147.93,142.56) --
	(147.97,142.65) --
	(148.01,142.74) --
	(148.05,142.83) --
	(148.09,142.91) --
	(148.13,143.00) --
	(148.17,143.09) --
	(148.21,143.18) --
	(148.25,143.26) --
	(148.29,143.35) --
	(148.33,143.44) --
	(148.37,143.52) --
	(148.41,143.61) --
	(148.45,143.70) --
	(148.49,143.78) --
	(148.53,143.87) --
	(148.58,143.96) --
	(148.62,144.04) --
	(148.66,144.13) --
	(148.70,144.22) --
	(148.74,144.30) --
	(148.78,144.39) --
	(148.82,144.47) --
	(148.86,144.56) --
	(148.90,144.64) --
	(148.94,144.73) --
	(148.98,144.82) --
	(149.02,144.90) --
	(149.06,144.99) --
	(149.10,145.07) --
	(149.14,145.16) --
	(149.18,145.24) --
	(149.22,145.33) --
	(149.26,145.41) --
	(149.30,145.50) --
	(149.34,145.58) --
	(149.38,145.67) --
	(149.42,145.75) --
	(149.46,145.83) --
	(149.50,145.92) --
	(149.55,146.00) --
	(149.59,146.09) --
	(149.63,146.17) --
	(149.67,146.26) --
	(149.71,146.34) --
	(149.75,146.42) --
	(149.79,146.51) --
	(149.83,146.59) --
	(149.87,146.67) --
	(149.91,146.76) --
	(149.95,146.84) --
	(149.99,146.92) --
	(150.03,147.01) --
	(150.07,147.09) --
	(150.11,147.17) --
	(150.15,147.26) --
	(150.19,147.34) --
	(150.23,147.42) --
	(150.27,147.50) --
	(150.31,147.59) --
	(150.35,147.67) --
	(150.39,147.75) --
	(150.43,147.84) --
	(150.48,147.92) --
	(150.52,148.00) --
	(150.56,148.08) --
	(150.60,148.16) --
	(150.64,148.25) --
	(150.68,148.33) --
	(150.72,148.41) --
	(150.76,148.49) --
	(150.80,148.57) --
	(150.84,148.65) --
	(150.88,148.74) --
	(150.92,148.82) --
	(150.96,148.90) --
	(151.00,148.98) --
	(151.04,149.06) --
	(151.08,149.14) --
	(151.12,149.22) --
	(151.16,149.30) --
	(151.20,149.39) --
	(151.24,149.47) --
	(151.28,149.55) --
	(151.32,149.63) --
	(151.36,149.71) --
	(151.40,149.79) --
	(151.45,149.87) --
	(151.49,149.95) --
	(151.53,150.03) --
	(151.57,150.11) --
	(151.61,150.19) --
	(151.65,150.27) --
	(151.69,150.35) --
	(151.73,150.43) --
	(151.77,150.51) --
	(151.81,150.59) --
	(151.85,150.67) --
	(151.89,150.75) --
	(151.93,150.83) --
	(151.97,150.91) --
	(152.01,150.99) --
	(152.05,151.07) --
	(152.09,151.14) --
	(152.13,151.22) --
	(152.17,151.30) --
	(152.21,151.38) --
	(152.25,151.46) --
	(152.29,151.54) --
	(152.33,151.62) --
	(152.37,151.70) --
	(152.42,151.78) --
	(152.46,151.85) --
	(152.50,151.93) --
	(152.54,152.01) --
	(152.58,152.09) --
	(152.62,152.17) --
	(152.66,152.25) --
	(152.70,152.32) --
	(152.74,152.40) --
	(152.78,152.48) --
	(152.82,152.56) --
	(152.86,152.63) --
	(152.90,152.71) --
	(152.94,152.79) --
	(152.98,152.87) --
	(153.02,152.95) --
	(153.06,153.02) --
	(153.10,153.10) --
	(153.14,153.18) --
	(153.18,153.25) --
	(153.22,153.33) --
	(153.26,153.41) --
	(153.30,153.49) --
	(153.34,153.56) --
	(153.39,153.64) --
	(153.43,153.72) --
	(153.47,153.79) --
	(153.51,153.87) --
	(153.55,153.95) --
	(153.59,154.02) --
	(153.63,154.10) --
	(153.67,154.18) --
	(153.71,154.25) --
	(153.75,154.33) --
	(153.79,154.40) --
	(153.83,154.48) --
	(153.87,154.56) --
	(153.91,154.63) --
	(153.95,154.71) --
	(153.99,154.78) --
	(154.03,154.86) --
	(154.07,154.93) --
	(154.11,155.01) --
	(154.15,155.09) --
	(154.19,155.16) --
	(154.23,155.24) --
	(154.27,155.31) --
	(154.31,155.39) --
	(154.36,155.46) --
	(154.40,155.54) --
	(154.44,155.61) --
	(154.48,155.69) --
	(154.52,155.76) --
	(154.56,155.84) --
	(154.60,155.91) --
	(154.64,155.99) --
	(154.68,156.06) --
	(154.72,156.14) --
	(154.76,156.21) --
	(154.80,156.28) --
	(154.84,156.36) --
	(154.88,156.43) --
	(154.92,156.51) --
	(154.96,156.58) --
	(155.00,156.66) --
	(155.04,156.73) --
	(155.08,156.80) --
	(155.12,156.88) --
	(155.16,156.95) --
	(155.20,157.02) --
	(155.24,157.10) --
	(155.29,157.17) --
	(155.33,157.25) --
	(155.37,157.32) --
	(155.41,157.39) --
	(155.45,157.47) --
	(155.49,157.54) --
	(155.53,157.61) --
	(155.57,157.69) --
	(155.61,157.76) --
	(155.65,157.83) --
	(155.69,157.90) --
	(155.73,157.98) --
	(155.77,158.05) --
	(155.81,158.12) --
	(155.85,158.20) --
	(155.89,158.27) --
	(155.93,158.34) --
	(155.97,158.41) --
	(156.01,158.49) --
	(156.05,158.56) --
	(156.09,158.63) --
	(156.13,158.70) --
	(156.17,158.78) --
	(156.21,158.85) --
	(156.26,158.92) --
	(156.30,158.99) --
	(156.34,159.06) --
	(156.38,159.14) --
	(156.42,159.21) --
	(156.46,159.28) --
	(156.50,159.35) --
	(156.54,159.42) --
	(156.58,159.49) --
	(156.62,159.57) --
	(156.66,159.64) --
	(156.70,159.71) --
	(156.74,159.78) --
	(156.78,159.85) --
	(156.82,159.92) --
	(156.86,159.99) --
	(156.90,160.06) --
	(156.94,160.14) --
	(156.98,160.21) --
	(157.02,160.28) --
	(157.06,160.35) --
	(157.10,160.42) --
	(157.14,160.49) --
	(157.18,160.56) --
	(157.23,160.63) --
	(157.27,160.70) --
	(157.31,160.77) --
	(157.35,160.84) --
	(157.39,160.91) --
	(157.43,160.98) --
	(157.47,161.05) --
	(157.51,161.12) --
	(157.55,161.19) --
	(157.59,161.26) --
	(157.63,161.33) --
	(157.67,161.40) --
	(157.71,161.47) --
	(157.75,161.54) --
	(157.79,161.61) --
	(157.83,161.68) --
	(157.87,161.75) --
	(157.91,161.82) --
	(157.95,161.89) --
	(157.99,161.96) --
	(158.03,162.03) --
	(158.07,162.10) --
	(158.11,162.17) --
	(158.15,162.24) --
	(158.20,162.31) --
	(158.24,162.38) --
	(158.28,162.45) --
	(158.32,162.51) --
	(158.36,162.58) --
	(158.40,162.65) --
	(158.44,162.72) --
	(158.48,162.79) --
	(158.52,162.86) --
	(158.56,162.93) --
	(158.60,163.00) --
	(158.64,163.06) --
	(158.68,163.13) --
	(158.72,163.20) --
	(158.76,163.27) --
	(158.80,163.34) --
	(158.84,163.41) --
	(158.88,163.48) --
	(158.92,163.54) --
	(158.96,163.61) --
	(159.00,163.68) --
	(159.04,163.75) --
	(159.08,163.82) --
	(159.13,163.88) --
	(159.17,163.95) --
	(159.21,164.02) --
	(159.25,164.09) --
	(159.29,164.15) --
	(159.33,164.22) --
	(159.37,164.29) --
	(159.41,164.36) --
	(159.45,164.42) --
	(159.49,164.49) --
	(159.53,164.56) --
	(159.57,164.63) --
	(159.61,164.69) --
	(159.65,164.76) --
	(159.69,164.83) --
	(159.73,164.89) --
	(159.77,164.96) --
	(159.81,165.03) --
	(159.85,165.10) --
	(159.89,165.16) --
	(159.93,165.23) --
	(159.97,165.30) --
	(160.01,165.36) --
	(160.05,165.43) --
	(160.10,165.50) --
	(160.14,165.56) --
	(160.18,165.63) --
	(160.22,165.70) --
	(160.26,165.76) --
	(160.30,165.83) --
	(160.34,165.89) --
	(160.38,165.96) --
	(160.42,166.03) --
	(160.46,166.09) --
	(160.50,166.16) --
	(160.54,166.22) --
	(160.58,166.29) --
	(160.62,166.36) --
	(160.66,166.42) --
	(160.70,166.49) --
	(160.74,166.55) --
	(160.78,166.62) --
	(160.82,166.69) --
	(160.86,166.75) --
	(160.90,166.82) --
	(160.94,166.88) --
	(160.98,166.95) --
	(161.02,167.01) --
	(161.07,167.08) --
	(161.11,167.14) --
	(161.15,167.21) --
	(161.19,167.27) --
	(161.23,167.34) --
	(161.27,167.40) --
	(161.31,167.47) --
	(161.35,167.53) --
	(161.39,167.60) --
	(161.43,167.66) --
	(161.47,167.73) --
	(161.51,167.79) --
	(161.55,167.86) --
	(161.59,167.92) --
	(161.63,167.99) --
	(161.67,168.05) --
	(161.71,168.12) --
	(161.75,168.18) --
	(161.79,168.25) --
	(161.83,168.31) --
	(161.87,168.37) --
	(161.91,168.44) --
	(161.95,168.50) --
	(161.99,168.57) --
	(162.04,168.63) --
	(162.08,168.69) --
	(162.12,168.76) --
	(162.16,168.82) --
	(162.20,168.89) --
	(162.24,168.95) --
	(162.28,169.01) --
	(162.32,169.08) --
	(162.36,169.14) --
	(162.40,169.21) --
	(162.44,169.27) --
	(162.48,169.33) --
	(162.52,169.40) --
	(162.56,169.46) --
	(162.60,169.52) --
	(162.64,169.59) --
	(162.68,169.65) --
	(162.72,169.71) --
	(162.76,169.78) --
	(162.80,169.84) --
	(162.84,169.90) --
	(162.88,169.97) --
	(162.92,170.03) --
	(162.96,170.09) --
	(163.01,170.15) --
	(163.05,170.22) --
	(163.09,170.28) --
	(163.13,170.34) --
	(163.17,170.41) --
	(163.21,170.47) --
	(163.25,170.53) --
	(163.29,170.59) --
	(163.33,170.66) --
	(163.37,170.72) --
	(163.41,170.78) --
	(163.45,170.84) --
	(163.49,170.91) --
	(163.53,170.97) --
	(163.57,171.03) --
	(163.61,171.09) --
	(163.65,171.15) --
	(163.69,171.22) --
	(163.73,171.28) --
	(163.77,171.34) --
	(163.81,171.40) --
	(163.85,171.46) --
	(163.89,171.53) --
	(163.94,171.59) --
	(163.98,171.65) --
	(164.02,171.71) --
	(164.06,171.77) --
	(164.10,171.84) --
	(164.14,171.90) --
	(164.18,171.96) --
	(164.22,172.02) --
	(164.26,172.08) --
	(164.30,172.14) --
	(164.34,172.20) --
	(164.38,172.27) --
	(164.42,172.33) --
	(164.46,172.39) --
	(164.50,172.45) --
	(164.54,172.51) --
	(164.58,172.57) --
	(164.62,172.63) --
	(164.66,172.69) --
	(164.70,172.76) --
	(164.74,172.82) --
	(164.78,172.88) --
	(164.82,172.94) --
	(164.86,173.00) --
	(164.91,173.06) --
	(164.95,173.12) --
	(164.99,173.18) --
	(165.03,173.24) --
	(165.07,173.30) --
	(165.11,173.36) --
	(165.15,173.42) --
	(165.19,173.48) --
	(165.23,173.54) --
	(165.27,173.60) --
	(165.31,173.66) --
	(165.35,173.73) --
	(165.39,173.79) --
	(165.43,173.85) --
	(165.47,173.91) --
	(165.51,173.97) --
	(165.55,174.03) --
	(165.59,174.09) --
	(165.63,174.15) --
	(165.67,174.21) --
	(165.71,174.27) --
	(165.75,174.33) --
	(165.79,174.39) --
	(165.83,174.45) --
	(165.88,174.50) --
	(165.92,174.56) --
	(165.96,174.62) --
	(166.00,174.68) --
	(166.04,174.74) --
	(166.08,174.80) --
	(166.12,174.86) --
	(166.16,174.92) --
	(166.20,174.98) --
	(166.24,175.04) --
	(166.28,175.10) --
	(166.32,175.16) --
	(166.36,175.22) --
	(166.40,175.28) --
	(166.44,175.34) --
	(166.48,175.40) --
	(166.52,175.45) --
	(166.56,175.51) --
	(166.60,175.57) --
	(166.64,175.63) --
	(166.68,175.69) --
	(166.72,175.75) --
	(166.76,175.81) --
	(166.80,175.87) --
	(166.85,175.93) --
	(166.89,175.98) --
	(166.93,176.04) --
	(166.97,176.10) --
	(167.01,176.16) --
	(167.05,176.22) --
	(167.09,176.28) --
	(167.13,176.33) --
	(167.17,176.39) --
	(167.21,176.45) --
	(167.25,176.51) --
	(167.29,176.57) --
	(167.33,176.63) --
	(167.37,176.68) --
	(167.41,176.74) --
	(167.45,176.80) --
	(167.49,176.86) --
	(167.53,176.92) --
	(167.57,176.97) --
	(167.61,177.03) --
	(167.65,177.09) --
	(167.69,177.15) --
	(167.73,177.21) --
	(167.78,177.26) --
	(167.82,177.32) --
	(167.86,177.38) --
	(167.90,177.44) --
	(167.94,177.49) --
	(167.98,177.55) --
	(168.02,177.61) --
	(168.06,177.67) --
	(168.10,177.72) --
	(168.14,177.78) --
	(168.18,177.84) --
	(168.22,177.90) --
	(168.26,177.95) --
	(168.30,178.01) --
	(168.34,178.07) --
	(168.38,178.13) --
	(168.42,178.18) --
	(168.46,178.24) --
	(168.50,178.30) --
	(168.54,178.35) --
	(168.58,178.41) --
	(168.62,178.47) --
	(168.66,178.52) --
	(168.70,178.58) --
	(168.75,178.64) --
	(168.79,178.69) --
	(168.83,178.75) --
	(168.87,178.81) --
	(168.91,178.87) --
	(168.95,178.92) --
	(168.99,178.98) --
	(169.03,179.03) --
	(169.07,179.09) --
	(169.11,179.15) --
	(169.15,179.20) --
	(169.19,179.26) --
	(169.23,179.32) --
	(169.27,179.37) --
	(169.31,179.43) --
	(169.35,179.49) --
	(169.39,179.54) --
	(169.43,179.60) --
	(169.47,179.65) --
	(169.51,179.71) --
	(169.55,179.77) --
	(169.59,179.82) --
	(169.63,179.88) --
	(169.67,179.93) --
	(169.72,179.99) --
	(169.76,180.05) --
	(169.80,180.10) --
	(169.84,180.16) --
	(169.88,180.21) --
	(169.92,180.27) --
	(169.96,180.32) --
	(170.00,180.38) --
	(170.04,180.44) --
	(170.08,180.49) --
	(170.12,180.55) --
	(170.16,180.60) --
	(170.20,180.66) --
	(170.24,180.71) --
	(170.28,180.77) --
	(170.32,180.82) --
	(170.36,180.88) --
	(170.40,180.93) --
	(170.44,180.99) --
	(170.48,181.04) --
	(170.52,181.10) --
	(170.56,181.16) --
	(170.60,181.21) --
	(170.64,181.27) --
	(170.69,181.32) --
	(170.73,181.38) --
	(170.77,181.43) --
	(170.81,181.48) --
	(170.85,181.54) --
	(170.89,181.59) --
	(170.93,181.65) --
	(170.97,181.70) --
	(171.01,181.76) --
	(171.05,181.81) --
	(171.09,181.87) --
	(171.13,181.92) --
	(171.17,181.98) --
	(171.21,182.03) --
	(171.25,182.09) --
	(171.29,182.14) --
	(171.33,182.20) --
	(171.37,182.25) --
	(171.41,182.30) --
	(171.45,182.36) --
	(171.49,182.41) --
	(171.53,182.47) --
	(171.57,182.52) --
	(171.61,182.58) --
	(171.66,182.63) --
	(171.70,182.68) --
	(171.74,182.74) --
	(171.78,182.79) --
	(171.82,182.85) --
	(171.86,182.90) --
	(171.90,182.95) --
	(171.94,183.01) --
	(171.98,183.06) --
	(172.02,183.11) --
	(172.06,183.17) --
	(172.10,183.22) --
	(172.14,183.28) --
	(172.18,183.33) --
	(172.22,183.38) --
	(172.26,183.44) --
	(172.30,183.49) --
	(172.34,183.54) --
	(172.38,183.60) --
	(172.42,183.65) --
	(172.46,183.70) --
	(172.50,183.76) --
	(172.54,183.81) --
	(172.59,183.86) --
	(172.63,183.92) --
	(172.67,183.97) --
	(172.71,184.02) --
	(172.75,184.08) --
	(172.79,184.13) --
	(172.83,184.18) --
	(172.87,184.24) --
	(172.91,184.29) --
	(172.95,184.34) --
	(172.99,184.40) --
	(173.03,184.45) --
	(173.07,184.50) --
	(173.11,184.56) --
	(173.15,184.61) --
	(173.19,184.66) --
	(173.23,184.71) --
	(173.27,184.77) --
	(173.31,184.82) --
	(173.35,184.87) --
	(173.39,184.92) --
	(173.43,184.98) --
	(173.47,185.03) --
	(173.51,185.08) --
	(173.56,185.13) --
	(173.60,185.19) --
	(173.64,185.24) --
	(173.68,185.29) --
	(173.72,185.34) --
	(173.76,185.40) --
	(173.80,185.45) --
	(173.84,185.50) --
	(173.88,185.55) --
	(173.92,185.61) --
	(173.96,185.66) --
	(174.00,185.71) --
	(174.04,185.76) --
	(174.08,185.81) --
	(174.12,185.87) --
	(174.16,185.92) --
	(174.20,185.97) --
	(174.24,186.02) --
	(174.28,186.07) --
	(174.32,186.13) --
	(174.36,186.18) --
	(174.40,186.23) --
	(174.44,186.28) --
	(174.48,186.33) --
	(174.53,186.39) --
	(174.57,186.44) --
	(174.61,186.49) --
	(174.65,186.54) --
	(174.69,186.59) --
	(174.73,186.64) --
	(174.77,186.70) --
	(174.81,186.75) --
	(174.85,186.80) --
	(174.89,186.85) --
	(174.93,186.90) --
	(174.97,186.95) --
	(175.01,187.00) --
	(175.05,187.05) --
	(175.09,187.11) --
	(175.13,187.16) --
	(175.17,187.21) --
	(175.21,187.26) --
	(175.25,187.31) --
	(175.29,187.36) --
	(175.33,187.41) --
	(175.37,187.46) --
	(175.41,187.52) --
	(175.45,187.57) --
	(175.50,187.62) --
	(175.54,187.67) --
	(175.58,187.72) --
	(175.62,187.77) --
	(175.66,187.82) --
	(175.70,187.87) --
	(175.74,187.92) --
	(175.78,187.97) --
	(175.82,188.02) --
	(175.86,188.07) --
	(175.90,188.13) --
	(175.94,188.18) --
	(175.98,188.23) --
	(176.02,188.28) --
	(176.06,188.33) --
	(176.10,188.38) --
	(176.14,188.43) --
	(176.18,188.48) --
	(176.22,188.53) --
	(176.26,188.58) --
	(176.30,188.63) --
	(176.34,188.68) --
	(176.38,188.73) --
	(176.43,188.78) --
	(176.47,188.83) --
	(176.51,188.88) --
	(176.55,188.93) --
	(176.59,188.98) --
	(176.63,189.03) --
	(176.67,189.08) --
	(176.71,189.13) --
	(176.75,189.18) --
	(176.79,189.23) --
	(176.83,189.28) --
	(176.87,189.33) --
	(176.91,189.38) --
	(176.95,189.43) --
	(176.99,189.48) --
	(177.03,189.53) --
	(177.07,189.58) --
	(177.11,189.63) --
	(177.15,189.68) --
	(177.19,189.73) --
	(177.23,189.78) --
	(177.27,189.83) --
	(177.31,189.88) --
	(177.35,189.93) --
	(177.40,189.98) --
	(177.44,190.03) --
	(177.48,190.08) --
	(177.52,190.13) --
	(177.56,190.18) --
	(177.60,190.23) --
	(177.64,190.28) --
	(177.68,190.32) --
	(177.72,190.37) --
	(177.76,190.42) --
	(177.80,190.47) --
	(177.84,190.52) --
	(177.88,190.57) --
	(177.92,190.62) --
	(177.96,190.67) --
	(178.00,190.72) --
	(178.04,190.77) --
	(178.08,190.82) --
	(178.12,190.87) --
	(178.16,190.91) --
	(178.20,190.96) --
	(178.24,191.01) --
	(178.28,191.06) --
	(178.32,191.11) --
	(178.37,191.16) --
	(178.41,191.21) --
	(178.45,191.26) --
	(178.49,191.30) --
	(178.53,191.35) --
	(178.57,191.40) --
	(178.61,191.45) --
	(178.65,191.50) --
	(178.69,191.55) --
	(178.73,191.60) --
	(178.77,191.65) --
	(178.81,191.69) --
	(178.85,191.74) --
	(178.89,191.79) --
	(178.93,191.84) --
	(178.97,191.89) --
	(179.01,191.94) --
	(179.05,191.98) --
	(179.09,192.03) --
	(179.13,192.08) --
	(179.17,192.13) --
	(179.21,192.18) --
	(179.25,192.23) --
	(179.29,192.27) --
	(179.34,192.32) --
	(179.38,192.37) --
	(179.42,192.42) --
	(179.46,192.47) --
	(179.50,192.51) --
	(179.54,192.56) --
	(179.58,192.61) --
	(179.62,192.66) --
	(179.66,192.71) --
	(179.70,192.75) --
	(179.74,192.80) --
	(179.78,192.85) --
	(179.82,192.90) --
	(179.86,192.95) --
	(179.90,192.99) --
	(179.94,193.04) --
	(179.98,193.09) --
	(180.02,193.14) --
	(180.06,193.18) --
	(180.10,193.23) --
	(180.14,193.28) --
	(180.18,193.33) --
	(180.22,193.37) --
	(180.26,193.42) --
	(180.31,193.47) --
	(180.35,193.52) --
	(180.39,193.56) --
	(180.43,193.61) --
	(180.47,193.66) --
	(180.51,193.71) --
	(180.55,193.75) --
	(180.59,193.80) --
	(180.63,193.85) --
	(180.67,193.90) --
	(180.71,193.94) --
	(180.75,193.99) --
	(180.79,194.04) --
	(180.83,194.08) --
	(180.87,194.13) --
	(180.91,194.18) --
	(180.95,194.23) --
	(180.99,194.27) --
	(181.03,194.32) --
	(181.07,194.37) --
	(181.11,194.41) --
	(181.15,194.46) --
	(181.19,194.51) --
	(181.24,194.55) --
	(181.28,194.60) --
	(181.32,194.65) --
	(181.36,194.70) --
	(181.40,194.74) --
	(181.44,194.79) --
	(181.48,194.84) --
	(181.52,194.88) --
	(181.56,194.93) --
	(181.60,194.98) --
	(181.64,195.02) --
	(181.68,195.07) --
	(181.72,195.12) --
	(181.76,195.16) --
	(181.80,195.21) --
	(181.84,195.25) --
	(181.88,195.30) --
	(181.92,195.35) --
	(181.96,195.39) --
	(182.00,195.44) --
	(182.04,195.49) --
	(182.08,195.53) --
	(182.12,195.58) --
	(182.16,195.63) --
	(182.21,195.67) --
	(182.25,195.72) --
	(182.29,195.76) --
	(182.33,195.81) --
	(182.37,195.86) --
	(182.41,195.90) --
	(182.45,195.95) --
	(182.49,195.99) --
	(182.53,196.04) --
	(182.57,196.09) --
	(182.61,196.13) --
	(182.65,196.18) --
	(182.69,196.22) --
	(182.73,196.27) --
	(182.77,196.32) --
	(182.81,196.36) --
	(182.85,196.41) --
	(182.89,196.45) --
	(182.93,196.50) --
	(182.97,196.55) --
	(183.01,196.59) --
	(183.05,196.64) --
	(183.09,196.68) --
	(183.13,196.73) --
	(183.18,196.77) --
	(183.22,196.82) --
	(183.26,196.87) --
	(183.30,196.91) --
	(183.34,196.96) --
	(183.38,197.00) --
	(183.42,197.05) --
	(183.46,197.09) --
	(183.50,197.14) --
	(183.54,197.18) --
	(183.58,197.23) --
	(183.62,197.27) --
	(183.66,197.32) --
	(183.70,197.37) --
	(183.74,197.41) --
	(183.78,197.46) --
	(183.82,197.50) --
	(183.86,197.55) --
	(183.90,197.59) --
	(183.94,197.64) --
	(183.98,197.68) --
	(184.02,197.73) --
	(184.06,197.77) --
	(184.10,197.82) --
	(184.15,197.86) --
	(184.19,197.91) --
	(184.23,197.95) --
	(184.27,198.00) --
	(184.31,198.04) --
	(184.35,198.09) --
	(184.39,198.13) --
	(184.43,198.18) --
	(184.47,198.22) --
	(184.51,198.27) --
	(184.55,198.31) --
	(184.59,198.36) --
	(184.63,198.40) --
	(184.67,198.45) --
	(184.71,198.49) --
	(184.75,198.53) --
	(184.79,198.58) --
	(184.83,198.62) --
	(184.87,198.67) --
	(184.91,198.71) --
	(184.95,198.76) --
	(184.99,198.80) --
	(185.03,198.85) --
	(185.08,198.89) --
	(185.12,198.94) --
	(185.16,198.98) --
	(185.20,199.02) --
	(185.24,199.07) --
	(185.28,199.11) --
	(185.32,199.16) --
	(185.36,199.20) --
	(185.40,199.25) --
	(185.44,199.29) --
	(185.48,199.33) --
	(185.52,199.38) --
	(185.56,199.42) --
	(185.60,199.47) --
	(185.64,199.51) --
	(185.68,199.56) --
	(185.72,199.60) --
	(185.76,199.64) --
	(185.80,199.69) --
	(185.84,199.73) --
	(185.88,199.78) --
	(185.92,199.82) --
	(185.96,199.86) --
	(186.00,199.91) --
	(186.05,199.95) --
	(186.09,200.00) --
	(186.13,200.04) --
	(186.17,200.08) --
	(186.21,200.13) --
	(186.25,200.17) --
	(186.29,200.21) --
	(186.33,200.26) --
	(186.37,200.30) --
	(186.41,200.35) --
	(186.45,200.39) --
	(186.49,200.43) --
	(186.53,200.48) --
	(186.57,200.52) --
	(186.61,200.56) --
	(186.65,200.61) --
	(186.69,200.65) --
	(186.73,200.69) --
	(186.77,200.74) --
	(186.81,200.78) --
	(186.85,200.83) --
	(186.89,200.87) --
	(186.93,200.91) --
	(186.97,200.96) --
	(187.02,201.00) --
	(187.06,201.04) --
	(187.10,201.09) --
	(187.14,201.13) --
	(187.18,201.17) --
	(187.22,201.22) --
	(187.26,201.26) --
	(187.30,201.30) --
	(187.34,201.34) --
	(187.38,201.39) --
	(187.42,201.43) --
	(187.46,201.47) --
	(187.50,201.52) --
	(187.54,201.56) --
	(187.58,201.60) --
	(187.62,201.65) --
	(187.66,201.69) --
	(187.70,201.73) --
	(187.74,201.78) --
	(187.78,201.82) --
	(187.82,201.86) --
	(187.86,201.90) --
	(187.90,201.95) --
	(187.94,201.99) --
	(187.99,202.03) --
	(188.03,202.08) --
	(188.07,202.12) --
	(188.11,202.16) --
	(188.15,202.20) --
	(188.19,202.25) --
	(188.23,202.29) --
	(188.27,202.33) --
	(188.31,202.37) --
	(188.35,202.42) --
	(188.39,202.46) --
	(188.43,202.50) --
	(188.47,202.55) --
	(188.51,202.59) --
	(188.55,202.63) --
	(188.59,202.67) --
	(188.63,202.72) --
	(188.67,202.76) --
	(188.71,202.80) --
	(188.75,202.84) --
	(188.79,202.88) --
	(188.83,202.93) --
	(188.87,202.97) --
	(188.92,203.01) --
	(188.96,203.05) --
	(189.00,203.10) --
	(189.04,203.14) --
	(189.08,203.18) --
	(189.12,203.22) --
	(189.16,203.27) --
	(189.20,203.31) --
	(189.24,203.35) --
	(189.28,203.39) --
	(189.32,203.43) --
	(189.36,203.48) --
	(189.40,203.52) --
	(189.44,203.56) --
	(189.48,203.60) --
	(189.52,203.64) --
	(189.56,203.69) --
	(189.60,203.73) --
	(189.64,203.77) --
	(189.68,203.81) --
	(189.72,203.85) --
	(189.76,203.90) --
	(189.80,203.94) --
	(189.84,203.98) --
	(189.89,204.02) --
	(189.93,204.06) --
	(189.97,204.10) --
	(190.01,204.15) --
	(190.05,204.19) --
	(190.09,204.23) --
	(190.13,204.27) --
	(190.17,204.31) --
	(190.21,204.36) --
	(190.25,204.40) --
	(190.29,204.44) --
	(190.33,204.48) --
	(190.37,204.52) --
	(190.41,204.56) --
	(190.45,204.60) --
	(190.49,204.65) --
	(190.53,204.69) --
	(190.57,204.73) --
	(190.61,204.77) --
	(190.65,204.81) --
	(190.69,204.85) --
	(190.73,204.89) --
	(190.77,204.94) --
	(190.81,204.98) --
	(190.86,205.02) --
	(190.90,205.06) --
	(190.94,205.10) --
	(190.98,205.14) --
	(191.02,205.18) --
	(191.06,205.22) --
	(191.10,205.27) --
	(191.14,205.31) --
	(191.18,205.35) --
	(191.22,205.39) --
	(191.26,205.43) --
	(191.30,205.47) --
	(191.34,205.51) --
	(191.38,205.55) --
	(191.42,205.59) --
	(191.46,205.64) --
	(191.50,205.68) --
	(191.54,205.72) --
	(191.58,205.76) --
	(191.62,205.80) --
	(191.66,205.84) --
	(191.70,205.88) --
	(191.74,205.92) --
	(191.78,205.96) --
	(191.83,206.00) --
	(191.87,206.04) --
	(191.91,206.09) --
	(191.95,206.13) --
	(191.99,206.17) --
	(192.03,206.21) --
	(192.07,206.25) --
	(192.11,206.29) --
	(192.15,206.33) --
	(192.19,206.37) --
	(192.23,206.41) --
	(192.27,206.45) --
	(192.31,206.49) --
	(192.35,206.53) --
	(192.39,206.57) --
	(192.43,206.61) --
	(192.47,206.65) --
	(192.51,206.70) --
	(192.55,206.74) --
	(192.59,206.78) --
	(192.63,206.82) --
	(192.67,206.86) --
	(192.71,206.90) --
	(192.75,206.94) --
	(192.80,206.98) --
	(192.84,207.02) --
	(192.88,207.06) --
	(192.92,207.10) --
	(192.96,207.14) --
	(193.00,207.18) --
	(193.04,207.22) --
	(193.08,207.26) --
	(193.12,207.30) --
	(193.16,207.34) --
	(193.20,207.38) --
	(193.24,207.42) --
	(193.28,207.46) --
	(193.32,207.50) --
	(193.36,207.54) --
	(193.40,207.58) --
	(193.44,207.62) --
	(193.48,207.66) --
	(193.52,207.70) --
	(193.56,207.74) --
	(193.60,207.78) --
	(193.64,207.82) --
	(193.68,207.86) --
	(193.73,207.90) --
	(193.77,207.94) --
	(193.81,207.98) --
	(193.85,208.02) --
	(193.89,208.06) --
	(193.93,208.10) --
	(193.97,208.14) --
	(194.01,208.18) --
	(194.05,208.22) --
	(194.09,208.26) --
	(194.13,208.30) --
	(194.17,208.34) --
	(194.21,208.38) --
	(194.25,208.42) --
	(194.29,208.46) --
	(194.33,208.50) --
	(194.37,208.54) --
	(194.41,208.58) --
	(194.45,208.62) --
	(194.49,208.66) --
	(194.53,208.69) --
	(194.57,208.73) --
	(194.61,208.77) --
	(194.65,208.81) --
	(194.70,208.85) --
	(194.74,208.89) --
	(194.78,208.93) --
	(194.82,208.97) --
	(194.86,209.01) --
	(194.90,209.05) --
	(194.94,209.09) --
	(194.98,209.13) --
	(195.02,209.17) --
	(195.06,209.21) --
	(195.10,209.25) --
	(195.14,209.29) --
	(195.18,209.32) --
	(195.22,209.36) --
	(195.26,209.40) --
	(195.30,209.44) --
	(195.34,209.48) --
	(195.38,209.52) --
	(195.42,209.56) --
	(195.46,209.60) --
	(195.50,209.64) --
	(195.54,209.68) --
	(195.58,209.72) --
	(195.62,209.75) --
	(195.67,209.79) --
	(195.71,209.83) --
	(195.75,209.87) --
	(195.79,209.91) --
	(195.83,209.95) --
	(195.87,209.99) --
	(195.91,210.03) --
	(195.95,210.07) --
	(195.99,210.10) --
	(196.03,210.14) --
	(196.07,210.18) --
	(196.11,210.22) --
	(196.15,210.26) --
	(196.19,210.30) --
	(196.23,210.34) --
	(196.27,210.38) --
	(196.31,210.42) --
	(196.35,210.45) --
	(196.39,210.49) --
	(196.43,210.53) --
	(196.47,210.57) --
	(196.51,210.61) --
	(196.55,210.65) --
	(196.59,210.69) --
	(196.64,210.72) --
	(196.68,210.76) --
	(196.72,210.80) --
	(196.76,210.84) --
	(196.80,210.88) --
	(196.84,210.92) --
	(196.88,210.96) --
	(196.92,210.99) --
	(196.96,211.03) --
	(197.00,211.07) --
	(197.04,211.11) --
	(197.08,211.15) --
	(197.12,211.19) --
	(197.16,211.22) --
	(197.20,211.26) --
	(197.24,211.30) --
	(197.28,211.34) --
	(197.32,211.38) --
	(197.36,211.42) --
	(197.40,211.45) --
	(197.44,211.49) --
	(197.48,211.53) --
	(197.52,211.57) --
	(197.57,211.61) --
	(197.61,211.64) --
	(197.65,211.68) --
	(197.69,211.72) --
	(197.73,211.76) --
	(197.77,211.80) --
	(197.81,211.83) --
	(197.85,211.87) --
	(197.89,211.91) --
	(197.93,211.95) --
	(197.97,211.99) --
	(198.01,212.02) --
	(198.05,212.06) --
	(198.09,212.10) --
	(198.13,212.14) --
	(198.17,212.18) --
	(198.21,212.21) --
	(198.25,212.25) --
	(198.29,212.29) --
	(198.33,212.33) --
	(198.37,212.37) --
	(198.41,212.40) --
	(198.45,212.44) --
	(198.49,212.48) --
	(198.54,212.52) --
	(198.58,212.55) --
	(198.62,212.59) --
	(198.66,212.63) --
	(198.70,212.67) --
	(198.74,212.71) --
	(198.78,212.74) --
	(198.82,212.78) --
	(198.86,212.82) --
	(198.90,212.86) --
	(198.94,212.89) --
	(198.98,212.93) --
	(199.02,212.97) --
	(199.06,213.01) --
	(199.10,213.04) --
	(199.14,213.08) --
	(199.18,213.12) --
	(199.22,213.16) --
	(199.26,213.19) --
	(199.30,213.23) --
	(199.34,213.27) --
	(199.38,213.31) --
	(199.42,213.34) --
	(199.46,213.38) --
	(199.51,213.42) --
	(199.55,213.45) --
	(199.59,213.49) --
	(199.63,213.53) --
	(199.67,213.57) --
	(199.71,213.60) --
	(199.75,213.64) --
	(199.79,213.68) --
	(199.83,213.72) --
	(199.87,213.75) --
	(199.91,213.79) --
	(199.95,213.83) --
	(199.99,213.86) --
	(200.03,213.90) --
	(200.07,213.94) --
	(200.11,213.98) --
	(200.15,214.01) --
	(200.19,214.05) --
	(200.23,214.09) --
	(200.27,214.12) --
	(200.31,214.16) --
	(200.35,214.20) --
	(200.39,214.23) --
	(200.43,214.27) --
	(200.48,214.31) --
	(200.52,214.35) --
	(200.56,214.38) --
	(200.60,214.42) --
	(200.64,214.46) --
	(200.68,214.49) --
	(200.72,214.53) --
	(200.76,214.57) --
	(200.80,214.60) --
	(200.84,214.64) --
	(200.88,214.68) --
	(200.92,214.71) --
	(200.96,214.75) --
	(201.00,214.79) --
	(201.04,214.82) --
	(201.08,214.86) --
	(201.12,214.90) --
	(201.16,214.93) --
	(201.20,214.97) --
	(201.24,215.01) --
	(201.28,215.04) --
	(201.32,215.08) --
	(201.36,215.12) --
	(201.40,215.15) --
	(201.45,215.19) --
	(201.49,215.23) --
	(201.53,215.26) --
	(201.57,215.30) --
	(201.61,215.34) --
	(201.65,215.37) --
	(201.69,215.41) --
	(201.73,215.45) --
	(201.77,215.48) --
	(201.81,215.52) --
	(201.85,215.55) --
	(201.89,215.59) --
	(201.93,215.63) --
	(201.97,215.66) --
	(202.01,215.70) --
	(202.05,215.74) --
	(202.09,215.77) --
	(202.13,215.81) --
	(202.17,215.84) --
	(202.21,215.88) --
	(202.25,215.92) --
	(202.29,215.95) --
	(202.33,215.99) --
	(202.38,216.03) --
	(202.42,216.06) --
	(202.46,216.10) --
	(202.50,216.13) --
	(202.54,216.17) --
	(202.58,216.21) --
	(202.62,216.24) --
	(202.66,216.28) --
	(202.70,216.31) --
	(202.74,216.35) --
	(202.78,216.39) --
	(202.82,216.42) --
	(202.86,216.46) --
	(202.90,216.49) --
	(202.94,216.53) --
	(202.98,216.57) --
	(203.02,216.60) --
	(203.06,216.64) --
	(203.10,216.67) --
	(203.14,216.71) --
	(203.18,216.75) --
	(203.22,216.78) --
	(203.26,216.82) --
	(203.30,216.85) --
	(203.35,216.89) --
	(203.39,216.92) --
	(203.43,216.96) --
	(203.47,217.00) --
	(203.51,217.03) --
	(203.55,217.07) --
	(203.59,217.10) --
	(203.63,217.14) --
	(203.67,217.17) --
	(203.71,217.21) --
	(203.75,217.25) --
	(203.79,217.28) --
	(203.83,217.32) --
	(203.87,217.35) --
	(203.91,217.39) --
	(203.95,217.42) --
	(203.99,217.46) --
	(204.03,217.50) --
	(204.07,217.53) --
	(204.11,217.57) --
	(204.15,217.60) --
	(204.19,217.64) --
	(204.23,217.67) --
	(204.27,217.71) --
	(204.32,217.74) --
	(204.36,217.78) --
	(204.40,217.81) --
	(204.44,217.85) --
	(204.48,217.89) --
	(204.52,217.92) --
	(204.56,217.96) --
	(204.60,217.99) --
	(204.64,218.03) --
	(204.68,218.06) --
	(204.72,218.10) --
	(204.76,218.13) --
	(204.80,218.17) --
	(204.84,218.20) --
	(204.88,218.24) --
	(204.92,218.27) --
	(204.96,218.31) --
	(205.00,218.34) --
	(205.04,218.38) --
	(205.08,218.41) --
	(205.12,218.45) --
	(205.16,218.48) --
	(205.20,218.52) --
	(205.24,218.55) --
	(205.29,218.59) --
	(205.33,218.62) --
	(205.37,218.66) --
	(205.41,218.69) --
	(205.45,218.73) --
	(205.49,218.76) --
	(205.53,218.80) --
	(205.57,218.83) --
	(205.61,218.87) --
	(205.65,218.90) --
	(205.69,218.94) --
	(205.73,218.97) --
	(205.77,219.01) --
	(205.81,219.04) --
	(205.85,219.08) --
	(205.89,219.11) --
	(205.93,219.15) --
	(205.97,219.18) --
	(206.01,219.22) --
	(206.05,219.25) --
	(206.09,219.29) --
	(206.13,219.32) --
	(206.17,219.36) --
	(206.22,219.39) --
	(206.26,219.43) --
	(206.30,219.46) --
	(206.34,219.50) --
	(206.38,219.53) --
	(206.42,219.57) --
	(206.46,219.60) --
	(206.50,219.63) --
	(206.54,219.67) --
	(206.58,219.70) --
	(206.62,219.74) --
	(206.66,219.77) --
	(206.70,219.81) --
	(206.74,219.84) --
	(206.78,219.88) --
	(206.82,219.91) --
	(206.86,219.95) --
	(206.90,219.98) --
	(206.94,220.01) --
	(206.98,220.05) --
	(207.02,220.08) --
	(207.06,220.12) --
	(207.10,220.15) --
	(207.14,220.19) --
	(207.19,220.22) --
	(207.23,220.26) --
	(207.27,220.29) --
	(207.31,220.32) --
	(207.35,220.36) --
	(207.39,220.39) --
	(207.43,220.43) --
	(207.47,220.46) --
	(207.51,220.50) --
	(207.55,220.53) --
	(207.59,220.56) --
	(207.63,220.60) --
	(207.67,220.63) --
	(207.71,220.67) --
	(207.75,220.70) --
	(207.79,220.74) --
	(207.83,220.77) --
	(207.87,220.80) --
	(207.91,220.84) --
	(207.95,220.87) --
	(207.99,220.91) --
	(208.03,220.94) --
	(208.07,220.97) --
	(208.11,221.01) --
	(208.16,221.04) --
	(208.20,221.08) --
	(208.24,221.11) --
	(208.28,221.14) --
	(208.32,221.18) --
	(208.36,221.21) --
	(208.40,221.25) --
	(208.44,221.28) --
	(208.48,221.31) --
	(208.52,221.35) --
	(208.56,221.38) --
	(208.60,221.42) --
	(208.64,221.45) --
	(208.68,221.48) --
	(208.72,221.52) --
	(208.76,221.55) --
	(208.80,221.59) --
	(208.84,221.62) --
	(208.88,221.65) --
	(208.92,221.69) --
	(208.96,221.72) --
	(209.00,221.75) --
	(209.04,221.79) --
	(209.08,221.82) --
	(209.13,221.86) --
	(209.17,221.89) --
	(209.21,221.92) --
	(209.25,221.96) --
	(209.29,221.99) --
	(209.33,222.02) --
	(209.37,222.06) --
	(209.41,222.09) --
	(209.45,222.12) --
	(209.49,222.16) --
	(209.53,222.19) --
	(209.57,222.23) --
	(209.61,222.26) --
	(209.65,222.29) --
	(209.69,222.33) --
	(209.73,222.36) --
	(209.77,222.39) --
	(209.81,222.43) --
	(209.85,222.46) --
	(209.89,222.49) --
	(209.93,222.53) --
	(209.97,222.56) --
	(210.01,222.59) --
	(210.05,222.63) --
	(210.10,222.66) --
	(210.14,222.69) --
	(210.18,222.73) --
	(210.22,222.76) --
	(210.26,222.79) --
	(210.30,222.83) --
	(210.34,222.86) --
	(210.38,222.89) --
	(210.42,222.93) --
	(210.46,222.96) --
	(210.50,222.99) --
	(210.54,223.03) --
	(210.58,223.06) --
	(210.62,223.09) --
	(210.66,223.13) --
	(210.70,223.16) --
	(210.74,223.19) --
	(210.78,223.23) --
	(210.82,223.26) --
	(210.86,223.29) --
	(210.90,223.33) --
	(210.94,223.36) --
	(210.98,223.39) --
	(211.03,223.42) --
	(211.07,223.46) --
	(211.11,223.49) --
	(211.15,223.52) --
	(211.19,223.56) --
	(211.23,223.59) --
	(211.27,223.62) --
	(211.31,223.66) --
	(211.35,223.69) --
	(211.39,223.72) --
	(211.43,223.75) --
	(211.47,223.79) --
	(211.51,223.82) --
	(211.55,223.85) --
	(211.59,223.89) --
	(211.63,223.92) --
	(211.67,223.95) --
	(211.71,223.99) --
	(211.75,224.02) --
	(211.79,224.05) --
	(211.83,224.08) --
	(211.87,224.12) --
	(211.91,224.15) --
	(211.95,224.18) --
	(212.00,224.21) --
	(212.04,224.25) --
	(212.08,224.28) --
	(212.12,224.31) --
	(212.16,224.35) --
	(212.20,224.38) --
	(212.24,224.41) --
	(212.28,224.44) --
	(212.32,224.48) --
	(212.36,224.51) --
	(212.40,224.54) --
	(212.44,224.57) --
	(212.48,224.61) --
	(212.52,224.64) --
	(212.56,224.67) --
	(212.60,224.70) --
	(212.64,224.74) --
	(212.68,224.77) --
	(212.72,224.80) --
	(212.76,224.83) --
	(212.80,224.87) --
	(212.84,224.90) --
	(212.88,224.93) --
	(212.92,224.96) --
	(212.97,225.00) --
	(213.01,225.03) --
	(213.05,225.06) --
	(213.09,225.09) --
	(213.13,225.13) --
	(213.17,225.16) --
	(213.21,225.19) --
	(213.25,225.22) --
	(213.29,225.26) --
	(213.33,225.29) --
	(213.37,225.32) --
	(213.41,225.35) --
	(213.45,225.39) --
	(213.49,225.42) --
	(213.53,225.45) --
	(213.57,225.48) --
	(213.61,225.51) --
	(213.65,225.55) --
	(213.69,225.58) --
	(213.73,225.61) --
	(213.77,225.64) --
	(213.81,225.68) --
	(213.85,225.71) --
	(213.89,225.74) --
	(213.94,225.77) --
	(213.98,225.80) --
	(214.02,225.84) --
	(214.06,225.87) --
	(214.10,225.90) --
	(214.14,225.93) --
	(214.18,225.97) --
	(214.22,226.00) --
	(214.26,226.03) --
	(214.30,226.06) --
	(214.34,226.09) --
	(214.38,226.13) --
	(214.42,226.16) --
	(214.46,226.19) --
	(214.50,226.22) --
	(214.54,226.25) --
	(214.58,226.29) --
	(214.62,226.32) --
	(214.66,226.35) --
	(214.70,226.38) --
	(214.74,226.41) --
	(214.78,226.45) --
	(214.82,226.48) --
	(214.87,226.51) --
	(214.91,226.54) --
	(214.95,226.57) --
	(214.99,226.60) --
	(215.03,226.64) --
	(215.07,226.67) --
	(215.11,226.70) --
	(215.15,226.73) --
	(215.19,226.76) --
	(215.23,226.80) --
	(215.27,226.83) --
	(215.31,226.86) --
	(215.35,226.89) --
	(215.39,226.92) --
	(215.43,226.95) --
	(215.47,226.99) --
	(215.51,227.02) --
	(215.55,227.05) --
	(215.59,227.08) --
	(215.63,227.11) --
	(215.67,227.14) --
	(215.71,227.18) --
	(215.75,227.21) --
	(215.79,227.24) --
	(215.84,227.27) --
	(215.88,227.30) --
	(215.92,227.33) --
	(215.96,227.36) --
	(216.00,227.40) --
	(216.04,227.43) --
	(216.08,227.46) --
	(216.12,227.49) --
	(216.16,227.52) --
	(216.20,227.55) --
	(216.24,227.59) --
	(216.28,227.62) --
	(216.32,227.65) --
	(216.36,227.68) --
	(216.40,227.71) --
	(216.44,227.74) --
	(216.48,227.77) --
	(216.52,227.81) --
	(216.56,227.84) --
	(216.60,227.87) --
	(216.64,227.90) --
	(216.68,227.93) --
	(216.72,227.96) --
	(216.76,227.99) --
	(216.81,228.02) --
	(216.85,228.06) --
	(216.89,228.09) --
	(216.93,228.12) --
	(216.97,228.15) --
	(217.01,228.18) --
	(217.05,228.21) --
	(217.09,228.24) --
	(217.13,228.27) --
	(217.17,228.31) --
	(217.21,228.34) --
	(217.25,228.37) --
	(217.29,228.40) --
	(217.33,228.43) --
	(217.37,228.46) --
	(217.41,228.49) --
	(217.45,228.52) --
	(217.49,228.55) --
	(217.53,228.59) --
	(217.57,228.62) --
	(217.61,228.65) --
	(217.65,228.68) --
	(217.69,228.71) --
	(217.73,228.74) --
	(217.78,228.77) --
	(217.82,228.80) --
	(217.86,228.83) --
	(217.90,228.87) --
	(217.94,228.90) --
	(217.98,228.93) --
	(218.02,228.96) --
	(218.06,228.99) --
	(218.10,229.02) --
	(218.14,229.05) --
	(218.18,229.08) --
	(218.22,229.11) --
	(218.26,229.14) --
	(218.30,229.17) --
	(218.34,229.21) --
	(218.38,229.24) --
	(218.42,229.27) --
	(218.46,229.30) --
	(218.50,229.33) --
	(218.54,229.36) --
	(218.58,229.39) --
	(218.62,229.42) --
	(218.66,229.45) --
	(218.70,229.48) --
	(218.75,229.51) --
	(218.79,229.54) --
	(218.83,229.57) --
	(218.87,229.61) --
	(218.91,229.64) --
	(218.95,229.67) --
	(218.99,229.70) --
	(219.03,229.73) --
	(219.07,229.76) --
	(219.11,229.79) --
	(219.15,229.82) --
	(219.19,229.85) --
	(219.23,229.88) --
	(219.27,229.91) --
	(219.31,229.94) --
	(219.35,229.97) --
	(219.39,230.00) --
	(219.43,230.03) --
	(219.47,230.06) --
	(219.51,230.09) --
	(219.55,230.13) --
	(219.59,230.16) --
	(219.63,230.19) --
	(219.68,230.22) --
	(219.72,230.25) --
	(219.76,230.28) --
	(219.80,230.31) --
	(219.84,230.34) --
	(219.88,230.37) --
	(219.92,230.40) --
	(219.96,230.43) --
	(220.00,230.46) --
	(220.04,230.49) --
	(220.08,230.52) --
	(220.12,230.55) --
	(220.16,230.58) --
	(220.20,230.61) --
	(220.24,230.64) --
	(220.28,230.67) --
	(220.32,230.70) --
	(220.36,230.73) --
	(220.40,230.76) --
	(220.44,230.79) --
	(220.48,230.82) --
	(220.52,230.85) --
	(220.56,230.88) --
	(220.60,230.91) --
	(220.65,230.94) --
	(220.69,230.98) --
	(220.73,231.01) --
	(220.77,231.04) --
	(220.81,231.07) --
	(220.85,231.10) --
	(220.89,231.13) --
	(220.93,231.16) --
	(220.97,231.19) --
	(221.01,231.22) --
	(221.05,231.25) --
	(221.09,231.28) --
	(221.13,231.31) --
	(221.17,231.34) --
	(221.21,231.37) --
	(221.25,231.40) --
	(221.29,231.43) --
	(221.33,231.46) --
	(221.37,231.49) --
	(221.41,231.52) --
	(221.45,231.55) --
	(221.49,231.58) --
	(221.53,231.61) --
	(221.57,231.64) --
	(221.62,231.67) --
	(221.66,231.70) --
	(221.70,231.73) --
	(221.74,231.76) --
	(221.78,231.79) --
	(221.82,231.82) --
	(221.86,231.85) --
	(221.90,231.88) --
	(221.94,231.91) --
	(221.98,231.94) --
	(222.02,231.97) --
	(222.06,232.00) --
	(222.10,232.03) --
	(222.14,232.06) --
	(222.18,232.08) --
	(222.22,232.11) --
	(222.26,232.14) --
	(222.30,232.17) --
	(222.34,232.20) --
	(222.38,232.23) --
	(222.42,232.26) --
	(222.46,232.29) --
	(222.50,232.32) --
	(222.54,232.35) --
	(222.59,232.38) --
	(222.63,232.41) --
	(222.67,232.44) --
	(222.71,232.47) --
	(222.75,232.50) --
	(222.79,232.53) --
	(222.83,232.56) --
	(222.87,232.59) --
	(222.91,232.62) --
	(222.95,232.65) --
	(222.99,232.68) --
	(223.03,232.71) --
	(223.07,232.74) --
	(223.11,232.77) --
	(223.15,232.80) --
	(223.19,232.83) --
	(223.23,232.86) --
	(223.27,232.89) --
	(223.31,232.91) --
	(223.35,232.94) --
	(223.39,232.97) --
	(223.43,233.00) --
	(223.47,233.03) --
	(223.52,233.06) --
	(223.56,233.09) --
	(223.60,233.12) --
	(223.64,233.15) --
	(223.68,233.18) --
	(223.72,233.21) --
	(223.76,233.24) --
	(223.80,233.27) --
	(223.84,233.30) --
	(223.88,233.33) --
	(223.92,233.36) --
	(223.96,233.39) --
	(224.00,233.41) --
	(224.04,233.44) --
	(224.08,233.47) --
	(224.12,233.50) --
	(224.16,233.53) --
	(224.20,233.56) --
	(224.24,233.59) --
	(224.28,233.62) --
	(224.32,233.65) --
	(224.36,233.68) --
	(224.40,233.71) --
	(224.44,233.74) --
	(224.49,233.77) --
	(224.53,233.80) --
	(224.57,233.82) --
	(224.61,233.85) --
	(224.65,233.88) --
	(224.69,233.91) --
	(224.73,233.94) --
	(224.77,233.97) --
	(224.81,234.00) --
	(224.85,234.03) --
	(224.89,234.06) --
	(224.93,234.09) --
	(224.97,234.12) --
	(225.01,234.14) --
	(225.05,234.17) --
	(225.09,234.20) --
	(225.13,234.23) --
	(225.17,234.26) --
	(225.21,234.29) --
	(225.25,234.32) --
	(225.29,234.35) --
	(225.33,234.38) --
	(225.37,234.41) --
	(225.41,234.43) --
	(225.46,234.46) --
	(225.50,234.49) --
	(225.54,234.52) --
	(225.58,234.55) --
	(225.62,234.58) --
	(225.66,234.61) --
	(225.70,234.64) --
	(225.74,234.67) --
	(225.78,234.70) --
	(225.82,234.72) --
	(225.86,234.75) --
	(225.90,234.78) --
	(225.94,234.81) --
	(225.98,234.84) --
	(226.02,234.87) --
	(226.06,234.90) --
	(226.10,234.93) --
	(226.14,234.95) --
	(226.18,234.98) --
	(226.22,235.01) --
	(226.26,235.04) --
	(226.30,235.07) --
	(226.34,235.10) --
	(226.38,235.13) --
	(226.43,235.16) --
	(226.47,235.18) --
	(226.51,235.21) --
	(226.55,235.24) --
	(226.59,235.27) --
	(226.63,235.30) --
	(226.67,235.33) --
	(226.71,235.36) --
	(226.75,235.38) --
	(226.79,235.41) --
	(226.83,235.44) --
	(226.87,235.47) --
	(226.91,235.50) --
	(226.95,235.53) --
	(226.99,235.56) --
	(227.03,235.59) --
	(227.07,235.61) --
	(227.11,235.64) --
	(227.15,235.67) --
	(227.19,235.70) --
	(227.23,235.73) --
	(227.27,235.76) --
	(227.31,235.78) --
	(227.35,235.81) --
	(227.40,235.84) --
	(227.44,235.87) --
	(227.48,235.90) --
	(227.52,235.93) --
	(227.56,235.96) --
	(227.60,235.98) --
	(227.64,236.01) --
	(227.68,236.04) --
	(227.72,236.07) --
	(227.76,236.10) --
	(227.80,236.13) --
	(227.84,236.15) --
	(227.88,236.18) --
	(227.92,236.21) --
	(227.96,236.24) --
	(228.00,236.27) --
	(228.04,236.30) --
	(228.08,236.32) --
	(228.12,236.35) --
	(228.16,236.38) --
	(228.20,236.41) --
	(228.24,236.44) --
	(228.28,236.47) --
	(228.33,236.49) --
	(228.37,236.52) --
	(228.41,236.55) --
	(228.45,236.58) --
	(228.49,236.61) --
	(228.53,236.64) --
	(228.57,236.66) --
	(228.61,236.69) --
	(228.65,236.72) --
	(228.69,236.75) --
	(228.73,236.78) --
	(228.77,236.80) --
	(228.81,236.83) --
	(228.85,236.86) --
	(228.89,236.89) --
	(228.93,236.92) --
	(228.97,236.94) --
	(229.01,236.97) --
	(229.05,237.00) --
	(229.09,237.03) --
	(229.13,237.06) --
	(229.17,237.09) --
	(229.21,237.11) --
	(229.25,237.14) --
	(229.30,237.17) --
	(229.34,237.20) --
	(229.38,237.23) --
	(229.42,237.25) --
	(229.46,237.28) --
	(229.50,237.31) --
	(229.54,237.34) --
	(229.58,237.37) --
	(229.62,237.39) --
	(229.66,237.42) --
	(229.70,237.45) --
	(229.74,237.48) --
	(229.78,237.50) --
	(229.82,237.53) --
	(229.86,237.56) --
	(229.90,237.59) --
	(229.94,237.62) --
	(229.98,237.64) --
	(230.02,237.67) --
	(230.06,237.70) --
	(230.10,237.73) --
	(230.14,237.76) --
	(230.18,237.78) --
	(230.22,237.81) --
	(230.27,237.84) --
	(230.31,237.87) --
	(230.35,237.89) --
	(230.39,237.92) --
	(230.43,237.95) --
	(230.47,237.98) --
	(230.51,238.01) --
	(230.55,238.03) --
	(230.59,238.06) --
	(230.63,238.09) --
	(230.67,238.12) --
	(230.71,238.14) --
	(230.75,238.17) --
	(230.79,238.20) --
	(230.83,238.23) --
	(230.87,238.25) --
	(230.91,238.28) --
	(230.95,238.31) --
	(230.99,238.34) --
	(231.03,238.37) --
	(231.07,238.39) --
	(231.11,238.42) --
	(231.15,238.45) --
	(231.19,238.48) --
	(231.24,238.50) --
	(231.28,238.53) --
	(231.32,238.56) --
	(231.36,238.59) --
	(231.40,238.61) --
	(231.44,238.64) --
	(231.48,238.67) --
	(231.52,238.70) --
	(231.56,238.72) --
	(231.60,238.75) --
	(231.64,238.78) --
	(231.68,238.81) --
	(231.72,238.83) --
	(231.76,238.86) --
	(231.80,238.89) --
	(231.84,238.92) --
	(231.88,238.94) --
	(231.92,238.97) --
	(231.96,239.00) --
	(232.00,239.03) --
	(232.04,239.05) --
	(232.08,239.08) --
	(232.12,239.11) --
	(232.17,239.14) --
	(232.21,239.16) --
	(232.25,239.19) --
	(232.29,239.22) --
	(232.33,239.24) --
	(232.37,239.27) --
	(232.41,239.30) --
	(232.45,239.33) --
	(232.49,239.35) --
	(232.53,239.38) --
	(232.57,239.41) --
	(232.61,239.44) --
	(232.65,239.46) --
	(232.69,239.49) --
	(232.73,239.52) --
	(232.77,239.54) --
	(232.81,239.57) --
	(232.85,239.60) --
	(232.89,239.63) --
	(232.93,239.65) --
	(232.97,239.68) --
	(233.01,239.71) --
	(233.05,239.74) --
	(233.09,239.76) --
	(233.14,239.79) --
	(233.18,239.82) --
	(233.22,239.84) --
	(233.26,239.87) --
	(233.30,239.90) --
	(233.34,239.93) --
	(233.38,239.95) --
	(233.42,239.98) --
	(233.46,240.01) --
	(233.50,240.03) --
	(233.54,240.06) --
	(233.58,240.09) --
	(233.62,240.11) --
	(233.66,240.14) --
	(233.70,240.17) --
	(233.74,240.20) --
	(233.78,240.22) --
	(233.82,240.25) --
	(233.86,240.28) --
	(233.90,240.30) --
	(233.94,240.33) --
	(233.98,240.36) --
	(234.02,240.38) --
	(234.06,240.41) --
	(234.11,240.44) --
	(234.15,240.47) --
	(234.19,240.49) --
	(234.23,240.52) --
	(234.27,240.55) --
	(234.31,240.57) --
	(234.35,240.60) --
	(234.39,240.63) --
	(234.43,240.65) --
	(234.47,240.68) --
	(234.51,240.71) --
	(234.55,240.73) --
	(234.59,240.76) --
	(234.63,240.79) --
	(234.67,240.81) --
	(234.71,240.84) --
	(234.75,240.87) --
	(234.79,240.90) --
	(234.83,240.92) --
	(234.87,240.95) --
	(234.91,240.98) --
	(234.95,241.00) --
	(234.99,241.03) --
	(235.03,241.06) --
	(235.08,241.08) --
	(235.12,241.11) --
	(235.16,241.14) --
	(235.20,241.16) --
	(235.24,241.19) --
	(235.28,241.22) --
	(235.32,241.24) --
	(235.36,241.27) --
	(235.40,241.30) --
	(235.44,241.32) --
	(235.48,241.35) --
	(235.52,241.38) --
	(235.56,241.40) --
	(235.60,241.43) --
	(235.64,241.46) --
	(235.68,241.48) --
	(235.72,241.51) --
	(235.76,241.54) --
	(235.80,241.56) --
	(235.84,241.59) --
	(235.88,241.62) --
	(235.92,241.64) --
	(235.96,241.67) --
	(236.00,241.69) --
	(236.05,241.72) --
	(236.09,241.75) --
	(236.13,241.77) --
	(236.17,241.80) --
	(236.21,241.83) --
	(236.25,241.85) --
	(236.29,241.88) --
	(236.33,241.91) --
	(236.37,241.93) --
	(236.41,241.96) --
	(236.45,241.99) --
	(236.49,242.01) --
	(236.53,242.04) --
	(236.57,242.07) --
	(236.61,242.09) --
	(236.65,242.12) --
	(236.69,242.14) --
	(236.73,242.17) --
	(236.77,242.20) --
	(236.81,242.22) --
	(236.85,242.25) --
	(236.89,242.28) --
	(236.93,242.30) --
	(236.98,242.33) --
	(237.02,242.36) --
	(237.06,242.38) --
	(237.10,242.41) --
	(237.14,242.43) --
	(237.18,242.46) --
	(237.22,242.49) --
	(237.26,242.51) --
	(237.30,242.54) --
	(237.34,242.57) --
	(237.38,242.59) --
	(237.42,242.62) --
	(237.46,242.64) --
	(237.50,242.67) --
	(237.54,242.70) --
	(237.58,242.72) --
	(237.62,242.75) --
	(237.66,242.78) --
	(237.70,242.80) --
	(237.74,242.83) --
	(237.78,242.85) --
	(237.82,242.88) --
	(237.86,242.91) --
	(237.90,242.93) --
	(237.95,242.96) --
	(237.99,242.98) --
	(238.03,243.01) --
	(238.07,243.04) --
	(238.11,243.06) --
	(238.15,243.09) --
	(238.19,243.12) --
	(238.23,243.14) --
	(238.27,243.17) --
	(238.31,243.19) --
	(238.35,243.22) --
	(238.39,243.25) --
	(238.43,243.27) --
	(238.47,243.30) --
	(238.51,243.32) --
	(238.55,243.35) --
	(238.59,243.38) --
	(238.63,243.40) --
	(238.67,243.43) --
	(238.71,243.45) --
	(238.75,243.48) --
	(238.79,243.51) --
	(238.83,243.53) --
	(238.87,243.56) --
	(238.92,243.58) --
	(238.96,243.61) --
	(239.00,243.64) --
	(239.04,243.66) --
	(239.08,243.69) --
	(239.12,243.71) --
	(239.16,243.74) --
	(239.20,243.76) --
	(239.24,243.79) --
	(239.28,243.82) --
	(239.32,243.84) --
	(239.36,243.87) --
	(239.40,243.89) --
	(239.44,243.92) --
	(239.48,243.95) --
	(239.52,243.97) --
	(239.56,244.00) --
	(239.60,244.02) --
	(239.64,244.05) --
	(239.68,244.07) --
	(239.72,244.10) --
	(239.76,244.13) --
	(239.80,244.15) --
	(239.84,244.18) --
	(239.89,244.20) --
	(239.93,244.23) --
	(239.97,244.25) --
	(240.01,244.28) --
	(240.05,244.31) --
	(240.09,244.33) --
	(240.13,244.36) --
	(240.17,244.38) --
	(240.21,244.41) --
	(240.25,244.43) --
	(240.29,244.46) --
	(240.33,244.49) --
	(240.37,244.51) --
	(240.41,244.54) --
	(240.45,244.56) --
	(240.49,244.59) --
	(240.53,244.61) --
	(240.57,244.64) --
	(240.61,244.67) --
	(240.65,244.69) --
	(240.69,244.72) --
	(240.73,244.74) --
	(240.77,244.77) --
	(240.82,244.79) --
	(240.86,244.82) --
	(240.90,244.84) --
	(240.94,244.87) --
	(240.98,244.90) --
	(241.02,244.92) --
	(241.06,244.95) --
	(241.10,244.97) --
	(241.14,245.00) --
	(241.18,245.02) --
	(241.22,245.05) --
	(241.26,245.07) --
	(241.30,245.10) --
	(241.34,245.12) --
	(241.38,245.15) --
	(241.42,245.18) --
	(241.46,245.20) --
	(241.50,245.23) --
	(241.54,245.25) --
	(241.58,245.28) --
	(241.62,245.30) --
	(241.66,245.33) --
	(241.70,245.35) --
	(241.74,245.38) --
	(241.79,245.40) --
	(241.83,245.43) --
	(241.87,245.46) --
	(241.91,245.48) --
	(241.95,245.51) --
	(241.99,245.53) --
	(242.03,245.56) --
	(242.07,245.58) --
	(242.11,245.61) --
	(242.15,245.63) --
	(242.19,245.66) --
	(242.23,245.68) --
	(242.27,245.71) --
	(242.31,245.73) --
	(242.35,245.76) --
	(242.39,245.78) --
	(242.43,245.81) --
	(242.47,245.83) --
	(242.51,245.86) --
	(242.55,245.89) --
	(242.59,245.91) --
	(242.63,245.94) --
	(242.67,245.96) --
	(242.71,245.99) --
	(242.76,246.01) --
	(242.80,246.04) --
	(242.84,246.06) --
	(242.88,246.09) --
	(242.92,246.11) --
	(242.96,246.14) --
	(243.00,246.16) --
	(243.04,246.19) --
	(243.08,246.21) --
	(243.12,246.24) --
	(243.16,246.26) --
	(243.20,246.29) --
	(243.24,246.31) --
	(243.28,246.34) --
	(243.32,246.36) --
	(243.36,246.39) --
	(243.40,246.41) --
	(243.44,246.44) --
	(243.48,246.46) --
	(243.52,246.49) --
	(243.56,246.51) --
	(243.60,246.54) --
	(243.64,246.56) --
	(243.68,246.59) --
	(243.73,246.61) --
	(243.77,246.64) --
	(243.81,246.66) --
	(243.85,246.69) --
	(243.89,246.71) --
	(243.93,246.74) --
	(243.97,246.76) --
	(244.01,246.79) --
	(244.05,246.81) --
	(244.09,246.84) --
	(244.13,246.86) --
	(244.17,246.89) --
	(244.21,246.91) --
	(244.25,246.94) --
	(244.29,246.96) --
	(244.33,246.99) --
	(244.37,247.01) --
	(244.41,247.04) --
	(244.45,247.06) --
	(244.49,247.09) --
	(244.53,247.11) --
	(244.57,247.14) --
	(244.61,247.16) --
	(244.65,247.19) --
	(244.70,247.21) --
	(244.74,247.24) --
	(244.78,247.26) --
	(244.82,247.29) --
	(244.86,247.31) --
	(244.90,247.34) --
	(244.94,247.36) --
	(244.98,247.38) --
	(245.02,247.41) --
	(245.06,247.43) --
	(245.10,247.46) --
	(245.14,247.48) --
	(245.18,247.51) --
	(245.22,247.53) --
	(245.26,247.56) --
	(245.30,247.58) --
	(245.34,247.61) --
	(245.38,247.63) --
	(245.42,247.66) --
	(245.46,247.68) --
	(245.50,247.71) --
	(245.54,247.73) --
	(245.58,247.76) --
	(245.63,247.78) --
	(245.67,247.80) --
	(245.71,247.83) --
	(245.75,247.85) --
	(245.79,247.88) --
	(245.83,247.90) --
	(245.87,247.93) --
	(245.91,247.95) --
	(245.95,247.98) --
	(245.99,248.00) --
	(246.03,248.03) --
	(246.07,248.05) --
	(246.11,248.08) --
	(246.15,248.10) --
	(246.19,248.12) --
	(246.23,248.15) --
	(246.27,248.17) --
	(246.31,248.20) --
	(246.35,248.22) --
	(246.39,248.25) --
	(246.43,248.27) --
	(246.47,248.30) --
	(246.51,248.32) --
	(246.55,248.35) --
	(246.60,248.37) --
	(246.64,248.39) --
	(246.68,248.42) --
	(246.72,248.44) --
	(246.76,248.47) --
	(246.80,248.49) --
	(246.84,248.52) --
	(246.88,248.54) --
	(246.92,248.57) --
	(246.96,248.59) --
	(247.00,248.61) --
	(247.04,248.64) --
	(247.08,248.66) --
	(247.12,248.69) --
	(247.16,248.71) --
	(247.20,248.74) --
	(247.24,248.76) --
	(247.28,248.78) --
	(247.32,248.81) --
	(247.36,248.83) --
	(247.40,248.86) --
	(247.44,248.88) --
	(247.48,248.91) --
	(247.52,248.93) --
	(247.57,248.96) --
	(247.61,248.98) --
	(247.65,249.00) --
	(247.69,249.03) --
	(247.73,249.05) --
	(247.77,249.08) --
	(247.81,249.10) --
	(247.85,249.12) --
	(247.89,249.15) --
	(247.93,249.17) --
	(247.97,249.20) --
	(248.01,249.22) --
	(248.05,249.25) --
	(248.09,249.27) --
	(248.13,249.29) --
	(248.17,249.32) --
	(248.21,249.34) --
	(248.25,249.37) --
	(248.29,249.39) --
	(248.33,249.42) --
	(248.37,249.44) --
	(248.41,249.46) --
	(248.45,249.49) --
	(248.49,249.51) --
	(248.54,249.54) --
	(248.58,249.56) --
	(248.62,249.58) --
	(248.66,249.61) --
	(248.70,249.63) --
	(248.74,249.66) --
	(248.78,249.68) --
	(248.82,249.71) --
	(248.86,249.73) --
	(248.90,249.75) --
	(248.94,249.78) --
	(248.98,249.80) --
	(249.02,249.83) --
	(249.06,249.85) --
	(249.10,249.87) --
	(249.14,249.90) --
	(249.18,249.92) --
	(249.22,249.95) --
	(249.26,249.97) --
	(249.30,249.99) --
	(249.34,250.02) --
	(249.38,250.04) --
	(249.42,250.07) --
	(249.47,250.09) --
	(249.51,250.11) --
	(249.55,250.14) --
	(249.59,250.16) --
	(249.63,250.19) --
	(249.67,250.21) --
	(249.71,250.23) --
	(249.75,250.26) --
	(249.79,250.28) --
	(249.83,250.31) --
	(249.87,250.33) --
	(249.91,250.35) --
	(249.95,250.38) --
	(249.99,250.40) --
	(250.03,250.42) --
	(250.07,250.45) --
	(250.11,250.47) --
	(250.15,250.50) --
	(250.19,250.52) --
	(250.23,250.54) --
	(250.27,250.57) --
	(250.31,250.59) --
	(250.35,250.62) --
	(250.39,250.64) --
	(250.44,250.66) --
	(250.48,250.69) --
	(250.52,250.71) --
	(250.56,250.73) --
	(250.60,250.76) --
	(250.64,250.78) --
	(250.68,250.81) --
	(250.72,250.83) --
	(250.76,250.85) --
	(250.80,250.88) --
	(250.84,250.90) --
	(250.88,250.92) --
	(250.92,250.95) --
	(250.96,250.97) --
	(251.00,251.00) --
	(251.04,251.02) --
	(251.08,251.04) --
	(251.12,251.07) --
	(251.16,251.09) --
	(251.20,251.11) --
	(251.24,251.14) --
	(251.28,251.16) --
	(251.32,251.19) --
	(251.36,251.21) --
	(251.41,251.23) --
	(251.45,251.26) --
	(251.49,251.28) --
	(251.53,251.30) --
	(251.57,251.33) --
	(251.61,251.35) --
	(251.65,251.37) --
	(251.69,251.40) --
	(251.73,251.42) --
	(251.77,251.44) --
	(251.81,251.47) --
	(251.85,251.49) --
	(251.89,251.52) --
	(251.93,251.54) --
	(251.97,251.56) --
	(252.01,251.59) --
	(252.05,251.61) --
	(252.09,251.63) --
	(252.13,251.66) --
	(252.17,251.68) --
	(252.21,251.70) --
	(252.25,251.73) --
	(252.29,251.75) --
	(252.33,251.77) --
	(252.38,251.80) --
	(252.42,251.82) --
	(252.46,251.84) --
	(252.50,251.87) --
	(252.54,251.89) --
	(252.58,251.92) --
	(252.62,251.94) --
	(252.66,251.96) --
	(252.70,251.99) --
	(252.74,252.01) --
	(252.78,252.03) --
	(252.82,252.06) --
	(252.86,252.08) --
	(252.90,252.10) --
	(252.94,252.13) --
	(252.98,252.15) --
	(253.02,252.17) --
	(253.06,252.20) --
	(253.10,252.22) --
	(253.14,252.24) --
	(253.18,252.27) --
	(253.22,252.29) --
	(253.26,252.31) --
	(253.30,252.34) --
	(253.35,252.36) --
	(253.39,252.38) --
	(253.43,252.41) --
	(253.47,252.43) --
	(253.51,252.45) --
	(253.55,252.48) --
	(253.59,252.50) --
	(253.63,252.52) --
	(253.67,252.55) --
	(253.71,252.57) --
	(253.75,252.59) --
	(253.79,252.62) --
	(253.83,252.64) --
	(253.87,252.66) --
	(253.91,252.69) --
	(253.95,252.71) --
	(253.99,252.73) --
	(254.03,252.76) --
	(254.07,252.78) --
	(254.11,252.80) --
	(254.15,252.82) --
	(254.19,252.85) --
	(254.23,252.87) --
	(254.28,252.89) --
	(254.32,252.92) --
	(254.36,252.94) --
	(254.40,252.96) --
	(254.44,252.99) --
	(254.48,253.01) --
	(254.52,253.03) --
	(254.56,253.06) --
	(254.60,253.08) --
	(254.64,253.10) --
	(254.68,253.13) --
	(254.72,253.15) --
	(254.76,253.17) --
	(254.80,253.20) --
	(254.84,253.22) --
	(254.88,253.24) --
	(254.92,253.26) --
	(254.96,253.29) --
	(255.00,253.31) --
	(255.04,253.33) --
	(255.08,253.36) --
	(255.12,253.38) --
	(255.16,253.40) --
	(255.20,253.43) --
	(255.25,253.45) --
	(255.29,253.47) --
	(255.33,253.49) --
	(255.37,253.52) --
	(255.41,253.54) --
	(255.45,253.56) --
	(255.49,253.59) --
	(255.53,253.61) --
	(255.57,253.63) --
	(255.61,253.66) --
	(255.65,253.68) --
	(255.69,253.70) --
	(255.73,253.72) --
	(255.77,253.75) --
	(255.81,253.77) --
	(255.85,253.79) --
	(255.89,253.82) --
	(255.93,253.84) --
	(255.97,253.86) --
	(256.01,253.89) --
	(256.05,253.91) --
	(256.09,253.93) --
	(256.13,253.95) --
	(256.17,253.98) --
	(256.22,254.00) --
	(256.26,254.02) --
	(256.30,254.05) --
	(256.34,254.07) --
	(256.38,254.09) --
	(256.42,254.11) --
	(256.46,254.14) --
	(256.50,254.16) --
	(256.54,254.18) --
	(256.58,254.21) --
	(256.62,254.23) --
	(256.66,254.25) --
	(256.70,254.27) --
	(256.74,254.30) --
	(256.78,254.32) --
	(256.82,254.34) --
	(256.86,254.36) --
	(256.90,254.39) --
	(256.94,254.41) --
	(256.98,254.43) --
	(257.02,254.46) --
	(257.06,254.48) --
	(257.10,254.50) --
	(257.14,254.52) --
	(257.19,254.55) --
	(257.23,254.57) --
	(257.27,254.59) --
	(257.31,254.61) --
	(257.35,254.64) --
	(257.39,254.66) --
	(257.43,254.68) --
	(257.47,254.71) --
	(257.51,254.73) --
	(257.55,254.75) --
	(257.59,254.77) --
	(257.63,254.80) --
	(257.67,254.82) --
	(257.71,254.84) --
	(257.75,254.86) --
	(257.79,254.89) --
	(257.83,254.91) --
	(257.87,254.93) --
	(257.91,254.95) --
	(257.95,254.98) --
	(257.99,255.00) --
	(258.03,255.02) --
	(258.07,255.04) --
	(258.12,255.07) --
	(258.16,255.09) --
	(258.20,255.11) --
	(258.24,255.14) --
	(258.28,255.16) --
	(258.32,255.18) --
	(258.36,255.20) --
	(258.40,255.23) --
	(258.44,255.25) --
	(258.48,255.27) --
	(258.52,255.29) --
	(258.56,255.32) --
	(258.60,255.34) --
	(258.64,255.36) --
	(258.68,255.38) --
	(258.72,255.41) --
	(258.76,255.43) --
	(258.80,255.45) --
	(258.84,255.47) --
	(258.88,255.50) --
	(258.92,255.52) --
	(258.96,255.54) --
	(259.00,255.56) --
	(259.04,255.59) --
	(259.09,255.61) --
	(259.13,255.63) --
	(259.17,255.65) --
	(259.21,255.67) --
	(259.25,255.70) --
	(259.29,255.72) --
	(259.33,255.74) --
	(259.37,255.76) --
	(259.41,255.79) --
	(259.45,255.81) --
	(259.49,255.83) --
	(259.53,255.85) --
	(259.57,255.88) --
	(259.61,255.90) --
	(259.65,255.92) --
	(259.69,255.94) --
	(259.73,255.97) --
	(259.77,255.99) --
	(259.81,256.01) --
	(259.85,256.03) --
	(259.89,256.05) --
	(259.93,256.08) --
	(259.97,256.10) --
	(260.01,256.12) --
	(260.06,256.14) --
	(260.10,256.17) --
	(260.14,256.19) --
	(260.18,256.21) --
	(260.22,256.23) --
	(260.26,256.26) --
	(260.30,256.28) --
	(260.34,256.30) --
	(260.38,256.32) --
	(260.42,256.34) --
	(260.46,256.37) --
	(260.50,256.39) --
	(260.54,256.41) --
	(260.58,256.43) --
	(260.62,256.46) --
	(260.66,256.48) --
	(260.70,256.50) --
	(260.74,256.52) --
	(260.78,256.54) --
	(260.82,256.57) --
	(260.86,256.59) --
	(260.90,256.61) --
	(260.94,256.63) --
	(260.98,256.66) --
	(261.03,256.68) --
	(261.07,256.70) --
	(261.11,256.72) --
	(261.15,256.74) --
	(261.19,256.77) --
	(261.23,256.79) --
	(261.27,256.81) --
	(261.31,256.83) --
	(261.35,256.85) --
	(261.39,256.88) --
	(261.43,256.90) --
	(261.47,256.92) --
	(261.51,256.94) --
	(261.55,256.96) --
	(261.59,256.99) --
	(261.63,257.01) --
	(261.67,257.03) --
	(261.71,257.05) --
	(261.75,257.07) --
	(261.79,257.10) --
	(261.83,257.12) --
	(261.87,257.14) --
	(261.91,257.16) --
	(261.95,257.18) --
	(262.00,257.21) --
	(262.04,257.23) --
	(262.08,257.25) --
	(262.12,257.27) --
	(262.16,257.29) --
	(262.20,257.32) --
	(262.24,257.34) --
	(262.28,257.36) --
	(262.32,257.38) --
	(262.36,257.40) --
	(262.40,257.43) --
	(262.44,257.45) --
	(262.48,257.47) --
	(262.52,257.49) --
	(262.56,257.51) --
	(262.60,257.54) --
	(262.64,257.56) --
	(262.68,257.58) --
	(262.72,257.60) --
	(262.76,257.62) --
	(262.80,257.65) --
	(262.84,257.67) --
	(262.88,257.69) --
	(262.93,257.71) --
	(262.97,257.73) --
	(263.01,257.76) --
	(263.05,257.78) --
	(263.09,257.80) --
	(263.13,257.82) --
	(263.17,257.84) --
	(263.21,257.86) --
	(263.25,257.89) --
	(263.29,257.91) --
	(263.33,257.93) --
	(263.37,257.95) --
	(263.41,257.97) --
	(263.45,258.00) --
	(263.49,258.02) --
	(263.53,258.04) --
	(263.57,258.06) --
	(263.61,258.08) --
	(263.65,258.10) --
	(263.69,258.13) --
	(263.73,258.15) --
	(263.77,258.17) --
	(263.81,258.19) --
	(263.85,258.21) --
	(263.90,258.23) --
	(263.94,258.26) --
	(263.98,258.28) --
	(264.02,258.30) --
	(264.06,258.32) --
	(264.10,258.34) --
	(264.14,258.36) --
	(264.18,258.39) --
	(264.22,258.41) --
	(264.26,258.43) --
	(264.30,258.45) --
	(264.34,258.47) --
	(264.38,258.49) --
	(264.42,258.52) --
	(264.46,258.54) --
	(264.50,258.56) --
	(264.54,258.58) --
	(264.58,258.60) --
	(264.62,258.62) --
	(264.66,258.65) --
	(264.70,258.67) --
	(264.74,258.69) --
	(264.78,258.71) --
	(264.82,258.73) --
	(264.87,258.75) --
	(264.91,258.78) --
	(264.95,258.80) --
	(264.99,258.82) --
	(265.03,258.84) --
	(265.07,258.86) --
	(265.11,258.88) --
	(265.15,258.91) --
	(265.19,258.93) --
	(265.23,258.95) --
	(265.27,258.97) --
	(265.31,258.99) --
	(265.35,259.01) --
	(265.39,259.03) --
	(265.43,259.06) --
	(265.47,259.08) --
	(265.51,259.10) --
	(265.55,259.12) --
	(265.59,259.14) --
	(265.63,259.16) --
	(265.67,259.19) --
	(265.71,259.21) --
	(265.75,259.23) --
	(265.79,259.25) --
	(265.84,259.27) --
	(265.88,259.29) --
	(265.92,259.31) --
	(265.96,259.34) --
	(266.00,259.36) --
	(266.04,259.38) --
	(266.08,259.40) --
	(266.12,259.42) --
	(266.16,259.44) --
	(266.20,259.46) --
	(266.24,259.49) --
	(266.28,259.51) --
	(266.32,259.53) --
	(266.36,259.55) --
	(266.40,259.57) --
	(266.44,259.59) --
	(266.48,259.61) --
	(266.52,259.64) --
	(266.56,259.66) --
	(266.60,259.68) --
	(266.64,259.70) --
	(266.68,259.72) --
	(266.72,259.74) --
	(266.77,259.76) --
	(266.81,259.78) --
	(266.85,259.81) --
	(266.89,259.83) --
	(266.93,259.85) --
	(266.97,259.87) --
	(267.01,259.89) --
	(267.05,259.91) --
	(267.09,259.93) --
	(267.13,259.96) --
	(267.17,259.98) --
	(267.21,260.00) --
	(267.25,260.02) --
	(267.29,260.04) --
	(267.33,260.06) --
	(267.37,260.08) --
	(267.41,260.10) --
	(267.45,260.13) --
	(267.49,260.15) --
	(267.53,260.17) --
	(267.57,260.19) --
	(267.61,260.21) --
	(267.65,260.23) --
	(267.69,260.25) --
	(267.74,260.27) --
	(267.78,260.29) --
	(267.82,260.32) --
	(267.86,260.34) --
	(267.90,260.36) --
	(267.94,260.38) --
	(267.98,260.40) --
	(268.02,260.42) --
	(268.06,260.44) --
	(268.10,260.46) --
	(268.14,260.49) --
	(268.18,260.51) --
	(268.22,260.53) --
	(268.26,260.55) --
	(268.30,260.57) --
	(268.34,260.59) --
	(268.38,260.61) --
	(268.42,260.63) --
	(268.46,260.65) --
	(268.50,260.68) --
	(268.54,260.70) --
	(268.58,260.72) --
	(268.62,260.74) --
	(268.66,260.76) --
	(268.71,260.78) --
	(268.75,260.80) --
	(268.79,260.82) --
	(268.83,260.84) --
	(268.87,260.87) --
	(268.91,260.89) --
	(268.95,260.91) --
	(268.99,260.93) --
	(269.03,260.95) --
	(269.07,260.97) --
	(269.11,260.99) --
	(269.15,261.01) --
	(269.19,261.03) --
	(269.23,261.05) --
	(269.27,261.08) --
	(269.31,261.10) --
	(269.35,261.12) --
	(269.39,261.14) --
	(269.43,261.16) --
	(269.47,261.18) --
	(269.51,261.20) --
	(269.55,261.22) --
	(269.59,261.24) --
	(269.63,261.26) --
	(269.68,261.29) --
	(269.72,261.31) --
	(269.76,261.33) --
	(269.80,261.35) --
	(269.84,261.37) --
	(269.88,261.39) --
	(269.92,261.41) --
	(269.96,261.43) --
	(270.00,261.45) --
	(270.04,261.47) --
	(270.08,261.49) --
	(270.12,261.52) --
	(270.16,261.54) --
	(270.20,261.56) --
	(270.24,261.58) --
	(270.28,261.60) --
	(270.32,261.62) --
	(270.36,261.64) --
	(270.40,261.66) --
	(270.44,261.68) --
	(270.48,261.70) --
	(270.52,261.72) --
	(270.56,261.74) --
	(270.60,261.77) --
	(270.65,261.79) --
	(270.69,261.81) --
	(270.73,261.83) --
	(270.77,261.85) --
	(270.81,261.87) --
	(270.85,261.89) --
	(270.89,261.91) --
	(270.93,261.93) --
	(270.97,261.95) --
	(271.01,261.97) --
	(271.05,261.99) --
	(271.09,262.01) --
	(271.13,262.04) --
	(271.17,262.06) --
	(271.21,262.08) --
	(271.25,262.10) --
	(271.29,262.12) --
	(271.33,262.14) --
	(271.37,262.16) --
	(271.41,262.18) --
	(271.45,262.20);

\path[draw=drawColor,line width= 0.6pt,line join=round] ( 28.93,147.34) -- (271.45,147.34);

\path[draw=drawColor,line width= 0.6pt,line join=round] ( 28.93,147.34) -- (271.45,147.34);

\path[draw=drawColor,line width= 0.6pt,line join=round] ( 28.93,147.34) -- (271.45,147.34);

\path[draw=drawColor,line width= 0.6pt,line join=round] ( 28.93,147.34) -- (271.45,147.34);

\path[draw=drawColor,line width= 0.6pt,line join=round] ( 28.93,147.34) -- (271.45,147.34);

\path[draw=drawColor,line width= 0.6pt,line join=round] ( 28.93,147.34) -- (271.45,147.34);

\path[draw=drawColor,line width= 0.6pt,line join=round] ( 28.93,147.34) -- (271.45,147.34);

\path[draw=drawColor,line width= 0.6pt,line join=round] ( 28.93,147.34) -- (271.45,147.34);

\path[draw=drawColor,line width= 0.6pt,line join=round] ( 28.93,147.34) -- (271.45,147.34);

\path[draw=drawColor,line width= 0.6pt,line join=round] ( 28.93,147.34) -- (271.45,147.34);

\path[draw=drawColor,line width= 0.6pt,line join=round] ( 28.93,147.34) -- (271.45,147.34);

\path[draw=drawColor,line width= 0.6pt,line join=round] ( 28.93,147.34) -- (271.45,147.34);

\path[draw=drawColor,line width= 0.6pt,line join=round] ( 28.93,147.34) -- (271.45,147.34);

\path[draw=drawColor,line width= 0.6pt,line join=round] ( 28.93,147.34) -- (271.45,147.34);

\path[draw=drawColor,line width= 0.6pt,line join=round] ( 28.93,147.34) -- (271.45,147.34);

\path[draw=drawColor,line width= 0.6pt,line join=round] ( 28.93,147.34) -- (271.45,147.34);

\path[draw=drawColor,line width= 0.6pt,line join=round] ( 28.93,147.34) -- (271.45,147.34);

\path[draw=drawColor,line width= 0.6pt,line join=round] ( 28.93,147.34) -- (271.45,147.34);

\path[draw=drawColor,line width= 0.6pt,line join=round] ( 28.93,147.34) -- (271.45,147.34);

\path[draw=drawColor,line width= 0.6pt,line join=round] ( 28.93,147.34) -- (271.45,147.34);

\path[draw=drawColor,line width= 0.6pt,line join=round] ( 28.93,147.34) -- (271.45,147.34);

\path[draw=drawColor,line width= 0.6pt,line join=round] ( 28.93,147.34) -- (271.45,147.34);

\path[draw=drawColor,line width= 0.6pt,line join=round] ( 28.93,147.34) -- (271.45,147.34);

\path[draw=drawColor,line width= 0.6pt,line join=round] ( 28.93,147.34) -- (271.45,147.34);

\path[draw=drawColor,line width= 0.6pt,line join=round] ( 28.93,147.34) -- (271.45,147.34);

\path[draw=drawColor,line width= 0.6pt,line join=round] ( 28.93,147.34) -- (271.45,147.34);

\path[draw=drawColor,line width= 0.6pt,line join=round] ( 28.93,147.34) -- (271.45,147.34);

\path[draw=drawColor,line width= 0.6pt,line join=round] ( 28.93,147.34) -- (271.45,147.34);

\path[draw=drawColor,line width= 0.6pt,line join=round] ( 28.93,147.34) -- (271.45,147.34);

\path[draw=drawColor,line width= 0.6pt,line join=round] ( 28.93,147.34) -- (271.45,147.34);

\path[draw=drawColor,line width= 0.6pt,line join=round] ( 28.93,147.34) -- (271.45,147.34);

\path[draw=drawColor,line width= 0.6pt,line join=round] ( 28.93,147.34) -- (271.45,147.34);

\path[draw=drawColor,line width= 0.6pt,line join=round] ( 28.93,147.34) -- (271.45,147.34);

\path[draw=drawColor,line width= 0.6pt,line join=round] ( 28.93,147.34) -- (271.45,147.34);

\path[draw=drawColor,line width= 0.6pt,line join=round] ( 28.93,147.34) -- (271.45,147.34);

\path[draw=drawColor,line width= 0.6pt,line join=round] ( 28.93,147.34) -- (271.45,147.34);

\path[draw=drawColor,line width= 0.6pt,line join=round] ( 28.93,147.34) -- (271.45,147.34);

\path[draw=drawColor,line width= 0.6pt,line join=round] ( 28.93,147.34) -- (271.45,147.34);

\path[draw=drawColor,line width= 0.6pt,line join=round] ( 28.93,147.34) -- (271.45,147.34);

\path[draw=drawColor,line width= 0.6pt,line join=round] ( 28.93,147.34) -- (271.45,147.34);

\path[draw=drawColor,line width= 0.6pt,line join=round] ( 28.93,147.34) -- (271.45,147.34);

\path[draw=drawColor,line width= 0.6pt,line join=round] ( 28.93,147.34) -- (271.45,147.34);

\path[draw=drawColor,line width= 0.6pt,line join=round] ( 28.93,147.34) -- (271.45,147.34);

\path[draw=drawColor,line width= 0.6pt,line join=round] ( 28.93,147.34) -- (271.45,147.34);

\path[draw=drawColor,line width= 0.6pt,line join=round] ( 28.93,147.34) -- (271.45,147.34);

\path[draw=drawColor,line width= 0.6pt,line join=round] ( 28.93,147.34) -- (271.45,147.34);

\path[draw=drawColor,line width= 0.6pt,line join=round] ( 28.93,147.34) -- (271.45,147.34);

\path[draw=drawColor,line width= 0.6pt,line join=round] ( 28.93,147.34) -- (271.45,147.34);

\path[draw=drawColor,line width= 0.6pt,line join=round] ( 28.93,147.34) -- (271.45,147.34);

\path[draw=drawColor,line width= 0.6pt,line join=round] ( 28.93,147.34) -- (271.45,147.34);

\path[draw=drawColor,line width= 0.6pt,line join=round] ( 28.93,147.34) -- (271.45,147.34);

\path[draw=drawColor,line width= 0.6pt,line join=round] ( 28.93,147.34) -- (271.45,147.34);

\path[draw=drawColor,line width= 0.6pt,line join=round] ( 28.93,147.34) -- (271.45,147.34);

\path[draw=drawColor,line width= 0.6pt,line join=round] ( 28.93,147.34) -- (271.45,147.34);

\path[draw=drawColor,line width= 0.6pt,line join=round] ( 28.93,147.34) -- (271.45,147.34);

\path[draw=drawColor,line width= 0.6pt,line join=round] ( 28.93,147.34) -- (271.45,147.34);

\path[draw=drawColor,line width= 0.6pt,line join=round] ( 28.93,147.34) -- (271.45,147.34);

\path[draw=drawColor,line width= 0.6pt,line join=round] ( 28.93,147.34) -- (271.45,147.34);

\path[draw=drawColor,line width= 0.6pt,line join=round] ( 28.93,147.34) -- (271.45,147.34);

\path[draw=drawColor,line width= 0.6pt,line join=round] ( 28.93,147.34) -- (271.45,147.34);

\path[draw=drawColor,line width= 0.6pt,line join=round] ( 28.93,147.34) -- (271.45,147.34);

\path[draw=drawColor,line width= 0.6pt,line join=round] ( 28.93,147.34) -- (271.45,147.34);

\path[draw=drawColor,line width= 0.6pt,line join=round] ( 28.93,147.34) -- (271.45,147.34);

\path[draw=drawColor,line width= 0.6pt,line join=round] ( 28.93,147.34) -- (271.45,147.34);

\path[draw=drawColor,line width= 0.6pt,line join=round] ( 28.93,147.34) -- (271.45,147.34);

\path[draw=drawColor,line width= 0.6pt,line join=round] ( 28.93,147.34) -- (271.45,147.34);

\path[draw=drawColor,line width= 0.6pt,line join=round] ( 28.93,147.34) -- (271.45,147.34);

\path[draw=drawColor,line width= 0.6pt,line join=round] ( 28.93,147.34) -- (271.45,147.34);

\path[draw=drawColor,line width= 0.6pt,line join=round] ( 28.93,147.34) -- (271.45,147.34);

\path[draw=drawColor,line width= 0.6pt,line join=round] ( 28.93,147.34) -- (271.45,147.34);

\path[draw=drawColor,line width= 0.6pt,line join=round] ( 28.93,147.34) -- (271.45,147.34);

\path[draw=drawColor,line width= 0.6pt,line join=round] ( 28.93,147.34) -- (271.45,147.34);

\path[draw=drawColor,line width= 0.6pt,line join=round] ( 28.93,147.34) -- (271.45,147.34);

\path[draw=drawColor,line width= 0.6pt,line join=round] ( 28.93,147.34) -- (271.45,147.34);

\path[draw=drawColor,line width= 0.6pt,line join=round] ( 28.93,147.34) -- (271.45,147.34);

\path[draw=drawColor,line width= 0.6pt,line join=round] ( 28.93,147.34) -- (271.45,147.34);

\path[draw=drawColor,line width= 0.6pt,line join=round] ( 28.93,147.34) -- (271.45,147.34);

\path[draw=drawColor,line width= 0.6pt,line join=round] ( 28.93,147.34) -- (271.45,147.34);

\path[draw=drawColor,line width= 0.6pt,line join=round] ( 28.93,147.34) -- (271.45,147.34);

\path[draw=drawColor,line width= 0.6pt,line join=round] ( 28.93,147.34) -- (271.45,147.34);

\path[draw=drawColor,line width= 0.6pt,line join=round] ( 28.93,147.34) -- (271.45,147.34);

\path[draw=drawColor,line width= 0.6pt,line join=round] ( 28.93,147.34) -- (271.45,147.34);

\path[draw=drawColor,line width= 0.6pt,line join=round] ( 28.93,147.34) -- (271.45,147.34);

\path[draw=drawColor,line width= 0.6pt,line join=round] ( 28.93,147.34) -- (271.45,147.34);

\path[draw=drawColor,line width= 0.6pt,line join=round] ( 28.93,147.34) -- (271.45,147.34);

\path[draw=drawColor,line width= 0.6pt,line join=round] ( 28.93,147.34) -- (271.45,147.34);

\path[draw=drawColor,line width= 0.6pt,line join=round] ( 28.93,147.34) -- (271.45,147.34);

\path[draw=drawColor,line width= 0.6pt,line join=round] ( 28.93,147.34) -- (271.45,147.34);

\path[draw=drawColor,line width= 0.6pt,line join=round] ( 28.93,147.34) -- (271.45,147.34);

\path[draw=drawColor,line width= 0.6pt,line join=round] ( 28.93,147.34) -- (271.45,147.34);

\path[draw=drawColor,line width= 0.6pt,line join=round] ( 28.93,147.34) -- (271.45,147.34);

\path[draw=drawColor,line width= 0.6pt,line join=round] ( 28.93,147.34) -- (271.45,147.34);

\path[draw=drawColor,line width= 0.6pt,line join=round] ( 28.93,147.34) -- (271.45,147.34);

\path[draw=drawColor,line width= 0.6pt,line join=round] ( 28.93,147.34) -- (271.45,147.34);

\path[draw=drawColor,line width= 0.6pt,line join=round] ( 28.93,147.34) -- (271.45,147.34);

\path[draw=drawColor,line width= 0.6pt,line join=round] ( 28.93,147.34) -- (271.45,147.34);

\path[draw=drawColor,line width= 0.6pt,line join=round] ( 28.93,147.34) -- (271.45,147.34);

\path[draw=drawColor,line width= 0.6pt,line join=round] ( 28.93,147.34) -- (271.45,147.34);

\path[draw=drawColor,line width= 0.6pt,line join=round] ( 28.93,147.34) -- (271.45,147.34);

\path[draw=drawColor,line width= 0.6pt,line join=round] ( 28.93,147.34) -- (271.45,147.34);

\path[draw=drawColor,line width= 0.6pt,line join=round] ( 28.93,147.34) -- (271.45,147.34);

\path[draw=drawColor,line width= 0.6pt,line join=round] ( 28.93,147.34) -- (271.45,147.34);

\path[draw=drawColor,line width= 0.6pt,line join=round] ( 28.93,147.34) -- (271.45,147.34);

\path[draw=drawColor,line width= 0.6pt,line join=round] ( 28.93,147.34) -- (271.45,147.34);

\path[draw=drawColor,line width= 0.6pt,line join=round] ( 28.93,147.34) -- (271.45,147.34);

\path[draw=drawColor,line width= 0.6pt,line join=round] ( 28.93,147.34) -- (271.45,147.34);

\path[draw=drawColor,line width= 0.6pt,line join=round] ( 28.93,147.34) -- (271.45,147.34);

\path[draw=drawColor,line width= 0.6pt,line join=round] ( 28.93,147.34) -- (271.45,147.34);

\path[draw=drawColor,line width= 0.6pt,line join=round] ( 28.93,147.34) -- (271.45,147.34);

\path[draw=drawColor,line width= 0.6pt,line join=round] ( 28.93,147.34) -- (271.45,147.34);

\path[draw=drawColor,line width= 0.6pt,line join=round] ( 28.93,147.34) -- (271.45,147.34);

\path[draw=drawColor,line width= 0.6pt,line join=round] ( 28.93,147.34) -- (271.45,147.34);

\path[draw=drawColor,line width= 0.6pt,line join=round] ( 28.93,147.34) -- (271.45,147.34);

\path[draw=drawColor,line width= 0.6pt,line join=round] ( 28.93,147.34) -- (271.45,147.34);

\path[draw=drawColor,line width= 0.6pt,line join=round] ( 28.93,147.34) -- (271.45,147.34);

\path[draw=drawColor,line width= 0.6pt,line join=round] ( 28.93,147.34) -- (271.45,147.34);

\path[draw=drawColor,line width= 0.6pt,line join=round] ( 28.93,147.34) -- (271.45,147.34);

\path[draw=drawColor,line width= 0.6pt,line join=round] ( 28.93,147.34) -- (271.45,147.34);

\path[draw=drawColor,line width= 0.6pt,line join=round] ( 28.93,147.34) -- (271.45,147.34);

\path[draw=drawColor,line width= 0.6pt,line join=round] ( 28.93,147.34) -- (271.45,147.34);

\path[draw=drawColor,line width= 0.6pt,line join=round] ( 28.93,147.34) -- (271.45,147.34);

\path[draw=drawColor,line width= 0.6pt,line join=round] ( 28.93,147.34) -- (271.45,147.34);

\path[draw=drawColor,line width= 0.6pt,line join=round] ( 28.93,147.34) -- (271.45,147.34);

\path[draw=drawColor,line width= 0.6pt,line join=round] ( 28.93,147.34) -- (271.45,147.34);

\path[draw=drawColor,line width= 0.6pt,line join=round] ( 28.93,147.34) -- (271.45,147.34);

\path[draw=drawColor,line width= 0.6pt,line join=round] ( 28.93,147.34) -- (271.45,147.34);

\path[draw=drawColor,line width= 0.6pt,line join=round] ( 28.93,147.34) -- (271.45,147.34);

\path[draw=drawColor,line width= 0.6pt,line join=round] ( 28.93,147.34) -- (271.45,147.34);

\path[draw=drawColor,line width= 0.6pt,line join=round] ( 28.93,147.34) -- (271.45,147.34);

\path[draw=drawColor,line width= 0.6pt,line join=round] ( 28.93,147.34) -- (271.45,147.34);

\path[draw=drawColor,line width= 0.6pt,line join=round] ( 28.93,147.34) -- (271.45,147.34);

\path[draw=drawColor,line width= 0.6pt,line join=round] ( 28.93,147.34) -- (271.45,147.34);

\path[draw=drawColor,line width= 0.6pt,line join=round] ( 28.93,147.34) -- (271.45,147.34);

\path[draw=drawColor,line width= 0.6pt,line join=round] ( 28.93,147.34) -- (271.45,147.34);

\path[draw=drawColor,line width= 0.6pt,line join=round] ( 28.93,147.34) -- (271.45,147.34);

\path[draw=drawColor,line width= 0.6pt,line join=round] ( 28.93,147.34) -- (271.45,147.34);

\path[draw=drawColor,line width= 0.6pt,line join=round] ( 28.93,147.34) -- (271.45,147.34);

\path[draw=drawColor,line width= 0.6pt,line join=round] ( 28.93,147.34) -- (271.45,147.34);

\path[draw=drawColor,line width= 0.6pt,line join=round] ( 28.93,147.34) -- (271.45,147.34);

\path[draw=drawColor,line width= 0.6pt,line join=round] ( 28.93,147.34) -- (271.45,147.34);

\path[draw=drawColor,line width= 0.6pt,line join=round] ( 28.93,147.34) -- (271.45,147.34);

\path[draw=drawColor,line width= 0.6pt,line join=round] ( 28.93,147.34) -- (271.45,147.34);

\path[draw=drawColor,line width= 0.6pt,line join=round] ( 28.93,147.34) -- (271.45,147.34);

\path[draw=drawColor,line width= 0.6pt,line join=round] ( 28.93,147.34) -- (271.45,147.34);

\path[draw=drawColor,line width= 0.6pt,line join=round] ( 28.93,147.34) -- (271.45,147.34);

\path[draw=drawColor,line width= 0.6pt,line join=round] ( 28.93,147.34) -- (271.45,147.34);

\path[draw=drawColor,line width= 0.6pt,line join=round] ( 28.93,147.34) -- (271.45,147.34);

\path[draw=drawColor,line width= 0.6pt,line join=round] ( 28.93,147.34) -- (271.45,147.34);

\path[draw=drawColor,line width= 0.6pt,line join=round] ( 28.93,147.34) -- (271.45,147.34);

\path[draw=drawColor,line width= 0.6pt,line join=round] ( 28.93,147.34) -- (271.45,147.34);

\path[draw=drawColor,line width= 0.6pt,line join=round] ( 28.93,147.34) -- (271.45,147.34);

\path[draw=drawColor,line width= 0.6pt,line join=round] ( 28.93,147.34) -- (271.45,147.34);

\path[draw=drawColor,line width= 0.6pt,line join=round] ( 28.93,147.34) -- (271.45,147.34);

\path[draw=drawColor,line width= 0.6pt,line join=round] ( 28.93,147.34) -- (271.45,147.34);

\path[draw=drawColor,line width= 0.6pt,line join=round] ( 28.93,147.34) -- (271.45,147.34);

\path[draw=drawColor,line width= 0.6pt,line join=round] ( 28.93,147.34) -- (271.45,147.34);

\path[draw=drawColor,line width= 0.6pt,line join=round] ( 28.93,147.34) -- (271.45,147.34);

\path[draw=drawColor,line width= 0.6pt,line join=round] ( 28.93,147.34) -- (271.45,147.34);

\path[draw=drawColor,line width= 0.6pt,line join=round] ( 28.93,147.34) -- (271.45,147.34);

\path[draw=drawColor,line width= 0.6pt,line join=round] ( 28.93,147.34) -- (271.45,147.34);

\path[draw=drawColor,line width= 0.6pt,line join=round] ( 28.93,147.34) -- (271.45,147.34);

\path[draw=drawColor,line width= 0.6pt,line join=round] ( 28.93,147.34) -- (271.45,147.34);

\path[draw=drawColor,line width= 0.6pt,line join=round] ( 28.93,147.34) -- (271.45,147.34);

\path[draw=drawColor,line width= 0.6pt,line join=round] ( 28.93,147.34) -- (271.45,147.34);

\path[draw=drawColor,line width= 0.6pt,line join=round] ( 28.93,147.34) -- (271.45,147.34);

\path[draw=drawColor,line width= 0.6pt,line join=round] ( 28.93,147.34) -- (271.45,147.34);

\path[draw=drawColor,line width= 0.6pt,line join=round] ( 28.93,147.34) -- (271.45,147.34);

\path[draw=drawColor,line width= 0.6pt,line join=round] ( 28.93,147.34) -- (271.45,147.34);

\path[draw=drawColor,line width= 0.6pt,line join=round] ( 28.93,147.34) -- (271.45,147.34);

\path[draw=drawColor,line width= 0.6pt,line join=round] ( 28.93,147.34) -- (271.45,147.34);

\path[draw=drawColor,line width= 0.6pt,line join=round] ( 28.93,147.34) -- (271.45,147.34);

\path[draw=drawColor,line width= 0.6pt,line join=round] ( 28.93,147.34) -- (271.45,147.34);

\path[draw=drawColor,line width= 0.6pt,line join=round] ( 28.93,147.34) -- (271.45,147.34);

\path[draw=drawColor,line width= 0.6pt,line join=round] ( 28.93,147.34) -- (271.45,147.34);

\path[draw=drawColor,line width= 0.6pt,line join=round] ( 28.93,147.34) -- (271.45,147.34);

\path[draw=drawColor,line width= 0.6pt,line join=round] ( 28.93,147.34) -- (271.45,147.34);

\path[draw=drawColor,line width= 0.6pt,line join=round] ( 28.93,147.34) -- (271.45,147.34);

\path[draw=drawColor,line width= 0.6pt,line join=round] ( 28.93,147.34) -- (271.45,147.34);

\path[draw=drawColor,line width= 0.6pt,line join=round] ( 28.93,147.34) -- (271.45,147.34);

\path[draw=drawColor,line width= 0.6pt,line join=round] ( 28.93,147.34) -- (271.45,147.34);

\path[draw=drawColor,line width= 0.6pt,line join=round] ( 28.93,147.34) -- (271.45,147.34);

\path[draw=drawColor,line width= 0.6pt,line join=round] ( 28.93,147.34) -- (271.45,147.34);

\path[draw=drawColor,line width= 0.6pt,line join=round] ( 28.93,147.34) -- (271.45,147.34);

\path[draw=drawColor,line width= 0.6pt,line join=round] ( 28.93,147.34) -- (271.45,147.34);

\path[draw=drawColor,line width= 0.6pt,line join=round] ( 28.93,147.34) -- (271.45,147.34);

\path[draw=drawColor,line width= 0.6pt,line join=round] ( 28.93,147.34) -- (271.45,147.34);

\path[draw=drawColor,line width= 0.6pt,line join=round] ( 28.93,147.34) -- (271.45,147.34);

\path[draw=drawColor,line width= 0.6pt,line join=round] ( 28.93,147.34) -- (271.45,147.34);

\path[draw=drawColor,line width= 0.6pt,line join=round] ( 28.93,147.34) -- (271.45,147.34);

\path[draw=drawColor,line width= 0.6pt,line join=round] ( 28.93,147.34) -- (271.45,147.34);

\path[draw=drawColor,line width= 0.6pt,line join=round] ( 28.93,147.34) -- (271.45,147.34);

\path[draw=drawColor,line width= 0.6pt,line join=round] ( 28.93,147.34) -- (271.45,147.34);

\path[draw=drawColor,line width= 0.6pt,line join=round] ( 28.93,147.34) -- (271.45,147.34);

\path[draw=drawColor,line width= 0.6pt,line join=round] ( 28.93,147.34) -- (271.45,147.34);

\path[draw=drawColor,line width= 0.6pt,line join=round] ( 28.93,147.34) -- (271.45,147.34);

\path[draw=drawColor,line width= 0.6pt,line join=round] ( 28.93,147.34) -- (271.45,147.34);

\path[draw=drawColor,line width= 0.6pt,line join=round] ( 28.93,147.34) -- (271.45,147.34);

\path[draw=drawColor,line width= 0.6pt,line join=round] ( 28.93,147.34) -- (271.45,147.34);

\path[draw=drawColor,line width= 0.6pt,line join=round] ( 28.93,147.34) -- (271.45,147.34);

\path[draw=drawColor,line width= 0.6pt,line join=round] ( 28.93,147.34) -- (271.45,147.34);

\path[draw=drawColor,line width= 0.6pt,line join=round] ( 28.93,147.34) -- (271.45,147.34);

\path[draw=drawColor,line width= 0.6pt,line join=round] ( 28.93,147.34) -- (271.45,147.34);

\path[draw=drawColor,line width= 0.6pt,line join=round] ( 28.93,147.34) -- (271.45,147.34);

\path[draw=drawColor,line width= 0.6pt,line join=round] ( 28.93,147.34) -- (271.45,147.34);

\path[draw=drawColor,line width= 0.6pt,line join=round] ( 28.93,147.34) -- (271.45,147.34);

\path[draw=drawColor,line width= 0.6pt,line join=round] ( 28.93,147.34) -- (271.45,147.34);

\path[draw=drawColor,line width= 0.6pt,line join=round] ( 28.93,147.34) -- (271.45,147.34);

\path[draw=drawColor,line width= 0.6pt,line join=round] ( 28.93,147.34) -- (271.45,147.34);

\path[draw=drawColor,line width= 0.6pt,line join=round] ( 28.93,147.34) -- (271.45,147.34);

\path[draw=drawColor,line width= 0.6pt,line join=round] ( 28.93,147.34) -- (271.45,147.34);

\path[draw=drawColor,line width= 0.6pt,line join=round] ( 28.93,147.34) -- (271.45,147.34);

\path[draw=drawColor,line width= 0.6pt,line join=round] ( 28.93,147.34) -- (271.45,147.34);

\path[draw=drawColor,line width= 0.6pt,line join=round] ( 28.93,147.34) -- (271.45,147.34);

\path[draw=drawColor,line width= 0.6pt,line join=round] ( 28.93,147.34) -- (271.45,147.34);

\path[draw=drawColor,line width= 0.6pt,line join=round] ( 28.93,147.34) -- (271.45,147.34);

\path[draw=drawColor,line width= 0.6pt,line join=round] ( 28.93,147.34) -- (271.45,147.34);

\path[draw=drawColor,line width= 0.6pt,line join=round] ( 28.93,147.34) -- (271.45,147.34);

\path[draw=drawColor,line width= 0.6pt,line join=round] ( 28.93,147.34) -- (271.45,147.34);

\path[draw=drawColor,line width= 0.6pt,line join=round] ( 28.93,147.34) -- (271.45,147.34);

\path[draw=drawColor,line width= 0.6pt,line join=round] ( 28.93,147.34) -- (271.45,147.34);

\path[draw=drawColor,line width= 0.6pt,line join=round] ( 28.93,147.34) -- (271.45,147.34);

\path[draw=drawColor,line width= 0.6pt,line join=round] ( 28.93,147.34) -- (271.45,147.34);

\path[draw=drawColor,line width= 0.6pt,line join=round] ( 28.93,147.34) -- (271.45,147.34);

\path[draw=drawColor,line width= 0.6pt,line join=round] ( 28.93,147.34) -- (271.45,147.34);

\path[draw=drawColor,line width= 0.6pt,line join=round] ( 28.93,147.34) -- (271.45,147.34);

\path[draw=drawColor,line width= 0.6pt,line join=round] ( 28.93,147.34) -- (271.45,147.34);

\path[draw=drawColor,line width= 0.6pt,line join=round] ( 28.93,147.34) -- (271.45,147.34);

\path[draw=drawColor,line width= 0.6pt,line join=round] ( 28.93,147.34) -- (271.45,147.34);

\path[draw=drawColor,line width= 0.6pt,line join=round] ( 28.93,147.34) -- (271.45,147.34);

\path[draw=drawColor,line width= 0.6pt,line join=round] ( 28.93,147.34) -- (271.45,147.34);

\path[draw=drawColor,line width= 0.6pt,line join=round] ( 28.93,147.34) -- (271.45,147.34);

\path[draw=drawColor,line width= 0.6pt,line join=round] ( 28.93,147.34) -- (271.45,147.34);

\path[draw=drawColor,line width= 0.6pt,line join=round] ( 28.93,147.34) -- (271.45,147.34);

\path[draw=drawColor,line width= 0.6pt,line join=round] ( 28.93,147.34) -- (271.45,147.34);

\path[draw=drawColor,line width= 0.6pt,line join=round] ( 28.93,147.34) -- (271.45,147.34);

\path[draw=drawColor,line width= 0.6pt,line join=round] ( 28.93,147.34) -- (271.45,147.34);

\path[draw=drawColor,line width= 0.6pt,line join=round] ( 28.93,147.34) -- (271.45,147.34);

\path[draw=drawColor,line width= 0.6pt,line join=round] ( 28.93,147.34) -- (271.45,147.34);

\path[draw=drawColor,line width= 0.6pt,line join=round] ( 28.93,147.34) -- (271.45,147.34);

\path[draw=drawColor,line width= 0.6pt,line join=round] ( 28.93,147.34) -- (271.45,147.34);

\path[draw=drawColor,line width= 0.6pt,line join=round] ( 28.93,147.34) -- (271.45,147.34);

\path[draw=drawColor,line width= 0.6pt,line join=round] ( 28.93,147.34) -- (271.45,147.34);

\path[draw=drawColor,line width= 0.6pt,line join=round] ( 28.93,147.34) -- (271.45,147.34);

\path[draw=drawColor,line width= 0.6pt,line join=round] ( 28.93,147.34) -- (271.45,147.34);

\path[draw=drawColor,line width= 0.6pt,line join=round] ( 28.93,147.34) -- (271.45,147.34);

\path[draw=drawColor,line width= 0.6pt,line join=round] ( 28.93,147.34) -- (271.45,147.34);

\path[draw=drawColor,line width= 0.6pt,line join=round] ( 28.93,147.34) -- (271.45,147.34);

\path[draw=drawColor,line width= 0.6pt,line join=round] ( 28.93,147.34) -- (271.45,147.34);

\path[draw=drawColor,line width= 0.6pt,line join=round] ( 28.93,147.34) -- (271.45,147.34);

\path[draw=drawColor,line width= 0.6pt,line join=round] ( 28.93,147.34) -- (271.45,147.34);

\path[draw=drawColor,line width= 0.6pt,line join=round] ( 28.93,147.34) -- (271.45,147.34);

\path[draw=drawColor,line width= 0.6pt,line join=round] ( 28.93,147.34) -- (271.45,147.34);

\path[draw=drawColor,line width= 0.6pt,line join=round] ( 28.93,147.34) -- (271.45,147.34);

\path[draw=drawColor,line width= 0.6pt,line join=round] ( 28.93,147.34) -- (271.45,147.34);

\path[draw=drawColor,line width= 0.6pt,line join=round] ( 28.93,147.34) -- (271.45,147.34);

\path[draw=drawColor,line width= 0.6pt,line join=round] ( 28.93,147.34) -- (271.45,147.34);

\path[draw=drawColor,line width= 0.6pt,line join=round] ( 28.93,147.34) -- (271.45,147.34);

\path[draw=drawColor,line width= 0.6pt,line join=round] ( 28.93,147.34) -- (271.45,147.34);

\path[draw=drawColor,line width= 0.6pt,line join=round] ( 28.93,147.34) -- (271.45,147.34);

\path[draw=drawColor,line width= 0.6pt,line join=round] ( 28.93,147.34) -- (271.45,147.34);

\path[draw=drawColor,line width= 0.6pt,line join=round] ( 28.93,147.34) -- (271.45,147.34);

\path[draw=drawColor,line width= 0.6pt,line join=round] ( 28.93,147.34) -- (271.45,147.34);

\path[draw=drawColor,line width= 0.6pt,line join=round] ( 28.93,147.34) -- (271.45,147.34);

\path[draw=drawColor,line width= 0.6pt,line join=round] ( 28.93,147.34) -- (271.45,147.34);

\path[draw=drawColor,line width= 0.6pt,line join=round] ( 28.93,147.34) -- (271.45,147.34);

\path[draw=drawColor,line width= 0.6pt,line join=round] ( 28.93,147.34) -- (271.45,147.34);

\path[draw=drawColor,line width= 0.6pt,line join=round] ( 28.93,147.34) -- (271.45,147.34);

\path[draw=drawColor,line width= 0.6pt,line join=round] ( 28.93,147.34) -- (271.45,147.34);

\path[draw=drawColor,line width= 0.6pt,line join=round] ( 28.93,147.34) -- (271.45,147.34);

\path[draw=drawColor,line width= 0.6pt,line join=round] ( 28.93,147.34) -- (271.45,147.34);

\path[draw=drawColor,line width= 0.6pt,line join=round] ( 28.93,147.34) -- (271.45,147.34);

\path[draw=drawColor,line width= 0.6pt,line join=round] ( 28.93,147.34) -- (271.45,147.34);

\path[draw=drawColor,line width= 0.6pt,line join=round] ( 28.93,147.34) -- (271.45,147.34);

\path[draw=drawColor,line width= 0.6pt,line join=round] ( 28.93,147.34) -- (271.45,147.34);

\path[draw=drawColor,line width= 0.6pt,line join=round] ( 28.93,147.34) -- (271.45,147.34);

\path[draw=drawColor,line width= 0.6pt,line join=round] ( 28.93,147.34) -- (271.45,147.34);

\path[draw=drawColor,line width= 0.6pt,line join=round] ( 28.93,147.34) -- (271.45,147.34);

\path[draw=drawColor,line width= 0.6pt,line join=round] ( 28.93,147.34) -- (271.45,147.34);

\path[draw=drawColor,line width= 0.6pt,line join=round] ( 28.93,147.34) -- (271.45,147.34);

\path[draw=drawColor,line width= 0.6pt,line join=round] ( 28.93,147.34) -- (271.45,147.34);

\path[draw=drawColor,line width= 0.6pt,line join=round] ( 28.93,147.34) -- (271.45,147.34);

\path[draw=drawColor,line width= 0.6pt,line join=round] ( 28.93,147.34) -- (271.45,147.34);

\path[draw=drawColor,line width= 0.6pt,line join=round] ( 28.93,147.34) -- (271.45,147.34);

\path[draw=drawColor,line width= 0.6pt,line join=round] ( 28.93,147.34) -- (271.45,147.34);

\path[draw=drawColor,line width= 0.6pt,line join=round] ( 28.93,147.34) -- (271.45,147.34);

\path[draw=drawColor,line width= 0.6pt,line join=round] ( 28.93,147.34) -- (271.45,147.34);

\path[draw=drawColor,line width= 0.6pt,line join=round] ( 28.93,147.34) -- (271.45,147.34);

\path[draw=drawColor,line width= 0.6pt,line join=round] ( 28.93,147.34) -- (271.45,147.34);

\path[draw=drawColor,line width= 0.6pt,line join=round] ( 28.93,147.34) -- (271.45,147.34);

\path[draw=drawColor,line width= 0.6pt,line join=round] ( 28.93,147.34) -- (271.45,147.34);

\path[draw=drawColor,line width= 0.6pt,line join=round] ( 28.93,147.34) -- (271.45,147.34);

\path[draw=drawColor,line width= 0.6pt,line join=round] ( 28.93,147.34) -- (271.45,147.34);

\path[draw=drawColor,line width= 0.6pt,line join=round] ( 28.93,147.34) -- (271.45,147.34);

\path[draw=drawColor,line width= 0.6pt,line join=round] ( 28.93,147.34) -- (271.45,147.34);

\path[draw=drawColor,line width= 0.6pt,line join=round] ( 28.93,147.34) -- (271.45,147.34);

\path[draw=drawColor,line width= 0.6pt,line join=round] ( 28.93,147.34) -- (271.45,147.34);

\path[draw=drawColor,line width= 0.6pt,line join=round] ( 28.93,147.34) -- (271.45,147.34);

\path[draw=drawColor,line width= 0.6pt,line join=round] ( 28.93,147.34) -- (271.45,147.34);

\path[draw=drawColor,line width= 0.6pt,line join=round] ( 28.93,147.34) -- (271.45,147.34);

\path[draw=drawColor,line width= 0.6pt,line join=round] ( 28.93,147.34) -- (271.45,147.34);

\path[draw=drawColor,line width= 0.6pt,line join=round] ( 28.93,147.34) -- (271.45,147.34);

\path[draw=drawColor,line width= 0.6pt,line join=round] ( 28.93,147.34) -- (271.45,147.34);

\path[draw=drawColor,line width= 0.6pt,line join=round] ( 28.93,147.34) -- (271.45,147.34);

\path[draw=drawColor,line width= 0.6pt,line join=round] ( 28.93,147.34) -- (271.45,147.34);

\path[draw=drawColor,line width= 0.6pt,line join=round] ( 28.93,147.34) -- (271.45,147.34);

\path[draw=drawColor,line width= 0.6pt,line join=round] ( 28.93,147.34) -- (271.45,147.34);

\path[draw=drawColor,line width= 0.6pt,line join=round] ( 28.93,147.34) -- (271.45,147.34);

\path[draw=drawColor,line width= 0.6pt,line join=round] ( 28.93,147.34) -- (271.45,147.34);

\path[draw=drawColor,line width= 0.6pt,line join=round] ( 28.93,147.34) -- (271.45,147.34);

\path[draw=drawColor,line width= 0.6pt,line join=round] ( 28.93,147.34) -- (271.45,147.34);

\path[draw=drawColor,line width= 0.6pt,line join=round] ( 28.93,147.34) -- (271.45,147.34);

\path[draw=drawColor,line width= 0.6pt,line join=round] ( 28.93,147.34) -- (271.45,147.34);

\path[draw=drawColor,line width= 0.6pt,line join=round] ( 28.93,147.34) -- (271.45,147.34);

\path[draw=drawColor,line width= 0.6pt,line join=round] ( 28.93,147.34) -- (271.45,147.34);

\path[draw=drawColor,line width= 0.6pt,line join=round] ( 28.93,147.34) -- (271.45,147.34);

\path[draw=drawColor,line width= 0.6pt,line join=round] ( 28.93,147.34) -- (271.45,147.34);

\path[draw=drawColor,line width= 0.6pt,line join=round] ( 28.93,147.34) -- (271.45,147.34);

\path[draw=drawColor,line width= 0.6pt,line join=round] ( 28.93,147.34) -- (271.45,147.34);

\path[draw=drawColor,line width= 0.6pt,line join=round] ( 28.93,147.34) -- (271.45,147.34);

\path[draw=drawColor,line width= 0.6pt,line join=round] ( 28.93,147.34) -- (271.45,147.34);

\path[draw=drawColor,line width= 0.6pt,line join=round] ( 28.93,147.34) -- (271.45,147.34);

\path[draw=drawColor,line width= 0.6pt,line join=round] ( 28.93,147.34) -- (271.45,147.34);

\path[draw=drawColor,line width= 0.6pt,line join=round] ( 28.93,147.34) -- (271.45,147.34);

\path[draw=drawColor,line width= 0.6pt,line join=round] ( 28.93,147.34) -- (271.45,147.34);

\path[draw=drawColor,line width= 0.6pt,line join=round] ( 28.93,147.34) -- (271.45,147.34);

\path[draw=drawColor,line width= 0.6pt,line join=round] ( 28.93,147.34) -- (271.45,147.34);

\path[draw=drawColor,line width= 0.6pt,line join=round] ( 28.93,147.34) -- (271.45,147.34);

\path[draw=drawColor,line width= 0.6pt,line join=round] ( 28.93,147.34) -- (271.45,147.34);

\path[draw=drawColor,line width= 0.6pt,line join=round] ( 28.93,147.34) -- (271.45,147.34);

\path[draw=drawColor,line width= 0.6pt,line join=round] ( 28.93,147.34) -- (271.45,147.34);

\path[draw=drawColor,line width= 0.6pt,line join=round] ( 28.93,147.34) -- (271.45,147.34);

\path[draw=drawColor,line width= 0.6pt,line join=round] ( 28.93,147.34) -- (271.45,147.34);

\path[draw=drawColor,line width= 0.6pt,line join=round] ( 28.93,147.34) -- (271.45,147.34);

\path[draw=drawColor,line width= 0.6pt,line join=round] ( 28.93,147.34) -- (271.45,147.34);

\path[draw=drawColor,line width= 0.6pt,line join=round] ( 28.93,147.34) -- (271.45,147.34);

\path[draw=drawColor,line width= 0.6pt,line join=round] ( 28.93,147.34) -- (271.45,147.34);

\path[draw=drawColor,line width= 0.6pt,line join=round] ( 28.93,147.34) -- (271.45,147.34);

\path[draw=drawColor,line width= 0.6pt,line join=round] ( 28.93,147.34) -- (271.45,147.34);

\path[draw=drawColor,line width= 0.6pt,line join=round] ( 28.93,147.34) -- (271.45,147.34);

\path[draw=drawColor,line width= 0.6pt,line join=round] ( 28.93,147.34) -- (271.45,147.34);

\path[draw=drawColor,line width= 0.6pt,line join=round] ( 28.93,147.34) -- (271.45,147.34);

\path[draw=drawColor,line width= 0.6pt,line join=round] ( 28.93,147.34) -- (271.45,147.34);

\path[draw=drawColor,line width= 0.6pt,line join=round] ( 28.93,147.34) -- (271.45,147.34);

\path[draw=drawColor,line width= 0.6pt,line join=round] ( 28.93,147.34) -- (271.45,147.34);

\path[draw=drawColor,line width= 0.6pt,line join=round] ( 28.93,147.34) -- (271.45,147.34);

\path[draw=drawColor,line width= 0.6pt,line join=round] ( 28.93,147.34) -- (271.45,147.34);

\path[draw=drawColor,line width= 0.6pt,line join=round] ( 28.93,147.34) -- (271.45,147.34);

\path[draw=drawColor,line width= 0.6pt,line join=round] ( 28.93,147.34) -- (271.45,147.34);

\path[draw=drawColor,line width= 0.6pt,line join=round] ( 28.93,147.34) -- (271.45,147.34);

\path[draw=drawColor,line width= 0.6pt,line join=round] ( 28.93,147.34) -- (271.45,147.34);

\path[draw=drawColor,line width= 0.6pt,line join=round] ( 28.93,147.34) -- (271.45,147.34);

\path[draw=drawColor,line width= 0.6pt,line join=round] ( 28.93,147.34) -- (271.45,147.34);

\path[draw=drawColor,line width= 0.6pt,line join=round] ( 28.93,147.34) -- (271.45,147.34);

\path[draw=drawColor,line width= 0.6pt,line join=round] ( 28.93,147.34) -- (271.45,147.34);

\path[draw=drawColor,line width= 0.6pt,line join=round] ( 28.93,147.34) -- (271.45,147.34);

\path[draw=drawColor,line width= 0.6pt,line join=round] ( 28.93,147.34) -- (271.45,147.34);

\path[draw=drawColor,line width= 0.6pt,line join=round] ( 28.93,147.34) -- (271.45,147.34);

\path[draw=drawColor,line width= 0.6pt,line join=round] ( 28.93,147.34) -- (271.45,147.34);

\path[draw=drawColor,line width= 0.6pt,line join=round] ( 28.93,147.34) -- (271.45,147.34);

\path[draw=drawColor,line width= 0.6pt,line join=round] ( 28.93,147.34) -- (271.45,147.34);

\path[draw=drawColor,line width= 0.6pt,line join=round] ( 28.93,147.34) -- (271.45,147.34);

\path[draw=drawColor,line width= 0.6pt,line join=round] ( 28.93,147.34) -- (271.45,147.34);

\path[draw=drawColor,line width= 0.6pt,line join=round] ( 28.93,147.34) -- (271.45,147.34);

\path[draw=drawColor,line width= 0.6pt,line join=round] ( 28.93,147.34) -- (271.45,147.34);

\path[draw=drawColor,line width= 0.6pt,line join=round] ( 28.93,147.34) -- (271.45,147.34);

\path[draw=drawColor,line width= 0.6pt,line join=round] ( 28.93,147.34) -- (271.45,147.34);

\path[draw=drawColor,line width= 0.6pt,line join=round] ( 28.93,147.34) -- (271.45,147.34);

\path[draw=drawColor,line width= 0.6pt,line join=round] ( 28.93,147.34) -- (271.45,147.34);

\path[draw=drawColor,line width= 0.6pt,line join=round] ( 28.93,147.34) -- (271.45,147.34);

\path[draw=drawColor,line width= 0.6pt,line join=round] ( 28.93,147.34) -- (271.45,147.34);

\path[draw=drawColor,line width= 0.6pt,line join=round] ( 28.93,147.34) -- (271.45,147.34);

\path[draw=drawColor,line width= 0.6pt,line join=round] ( 28.93,147.34) -- (271.45,147.34);

\path[draw=drawColor,line width= 0.6pt,line join=round] ( 28.93,147.34) -- (271.45,147.34);

\path[draw=drawColor,line width= 0.6pt,line join=round] ( 28.93,147.34) -- (271.45,147.34);

\path[draw=drawColor,line width= 0.6pt,line join=round] ( 28.93,147.34) -- (271.45,147.34);

\path[draw=drawColor,line width= 0.6pt,line join=round] ( 28.93,147.34) -- (271.45,147.34);

\path[draw=drawColor,line width= 0.6pt,line join=round] ( 28.93,147.34) -- (271.45,147.34);

\path[draw=drawColor,line width= 0.6pt,line join=round] ( 28.93,147.34) -- (271.45,147.34);

\path[draw=drawColor,line width= 0.6pt,line join=round] ( 28.93,147.34) -- (271.45,147.34);

\path[draw=drawColor,line width= 0.6pt,line join=round] ( 28.93,147.34) -- (271.45,147.34);

\path[draw=drawColor,line width= 0.6pt,line join=round] ( 28.93,147.34) -- (271.45,147.34);

\path[draw=drawColor,line width= 0.6pt,line join=round] ( 28.93,147.34) -- (271.45,147.34);

\path[draw=drawColor,line width= 0.6pt,line join=round] ( 28.93,147.34) -- (271.45,147.34);

\path[draw=drawColor,line width= 0.6pt,line join=round] ( 28.93,147.34) -- (271.45,147.34);

\path[draw=drawColor,line width= 0.6pt,line join=round] ( 28.93,147.34) -- (271.45,147.34);

\path[draw=drawColor,line width= 0.6pt,line join=round] ( 28.93,147.34) -- (271.45,147.34);

\path[draw=drawColor,line width= 0.6pt,line join=round] ( 28.93,147.34) -- (271.45,147.34);

\path[draw=drawColor,line width= 0.6pt,line join=round] ( 28.93,147.34) -- (271.45,147.34);

\path[draw=drawColor,line width= 0.6pt,line join=round] ( 28.93,147.34) -- (271.45,147.34);

\path[draw=drawColor,line width= 0.6pt,line join=round] ( 28.93,147.34) -- (271.45,147.34);

\path[draw=drawColor,line width= 0.6pt,line join=round] ( 28.93,147.34) -- (271.45,147.34);

\path[draw=drawColor,line width= 0.6pt,line join=round] ( 28.93,147.34) -- (271.45,147.34);

\path[draw=drawColor,line width= 0.6pt,line join=round] ( 28.93,147.34) -- (271.45,147.34);

\path[draw=drawColor,line width= 0.6pt,line join=round] ( 28.93,147.34) -- (271.45,147.34);

\path[draw=drawColor,line width= 0.6pt,line join=round] ( 28.93,147.34) -- (271.45,147.34);

\path[draw=drawColor,line width= 0.6pt,line join=round] ( 28.93,147.34) -- (271.45,147.34);

\path[draw=drawColor,line width= 0.6pt,line join=round] ( 28.93,147.34) -- (271.45,147.34);

\path[draw=drawColor,line width= 0.6pt,line join=round] ( 28.93,147.34) -- (271.45,147.34);

\path[draw=drawColor,line width= 0.6pt,line join=round] ( 28.93,147.34) -- (271.45,147.34);

\path[draw=drawColor,line width= 0.6pt,line join=round] ( 28.93,147.34) -- (271.45,147.34);

\path[draw=drawColor,line width= 0.6pt,line join=round] ( 28.93,147.34) -- (271.45,147.34);

\path[draw=drawColor,line width= 0.6pt,line join=round] ( 28.93,147.34) -- (271.45,147.34);

\path[draw=drawColor,line width= 0.6pt,line join=round] ( 28.93,147.34) -- (271.45,147.34);

\path[draw=drawColor,line width= 0.6pt,line join=round] ( 28.93,147.34) -- (271.45,147.34);

\path[draw=drawColor,line width= 0.6pt,line join=round] ( 28.93,147.34) -- (271.45,147.34);

\path[draw=drawColor,line width= 0.6pt,line join=round] ( 28.93,147.34) -- (271.45,147.34);

\path[draw=drawColor,line width= 0.6pt,line join=round] ( 28.93,147.34) -- (271.45,147.34);

\path[draw=drawColor,line width= 0.6pt,line join=round] ( 28.93,147.34) -- (271.45,147.34);

\path[draw=drawColor,line width= 0.6pt,line join=round] ( 28.93,147.34) -- (271.45,147.34);

\path[draw=drawColor,line width= 0.6pt,line join=round] ( 28.93,147.34) -- (271.45,147.34);

\path[draw=drawColor,line width= 0.6pt,line join=round] ( 28.93,147.34) -- (271.45,147.34);

\path[draw=drawColor,line width= 0.6pt,line join=round] ( 28.93,147.34) -- (271.45,147.34);

\path[draw=drawColor,line width= 0.6pt,line join=round] ( 28.93,147.34) -- (271.45,147.34);

\path[draw=drawColor,line width= 0.6pt,line join=round] ( 28.93,147.34) -- (271.45,147.34);

\path[draw=drawColor,line width= 0.6pt,line join=round] ( 28.93,147.34) -- (271.45,147.34);

\path[draw=drawColor,line width= 0.6pt,line join=round] ( 28.93,147.34) -- (271.45,147.34);

\path[draw=drawColor,line width= 0.6pt,line join=round] ( 28.93,147.34) -- (271.45,147.34);

\path[draw=drawColor,line width= 0.6pt,line join=round] ( 28.93,147.34) -- (271.45,147.34);

\path[draw=drawColor,line width= 0.6pt,line join=round] ( 28.93,147.34) -- (271.45,147.34);

\path[draw=drawColor,line width= 0.6pt,line join=round] ( 28.93,147.34) -- (271.45,147.34);

\path[draw=drawColor,line width= 0.6pt,line join=round] ( 28.93,147.34) -- (271.45,147.34);

\path[draw=drawColor,line width= 0.6pt,line join=round] ( 28.93,147.34) -- (271.45,147.34);

\path[draw=drawColor,line width= 0.6pt,line join=round] ( 28.93,147.34) -- (271.45,147.34);

\path[draw=drawColor,line width= 0.6pt,line join=round] ( 28.93,147.34) -- (271.45,147.34);

\path[draw=drawColor,line width= 0.6pt,line join=round] ( 28.93,147.34) -- (271.45,147.34);

\path[draw=drawColor,line width= 0.6pt,line join=round] ( 28.93,147.34) -- (271.45,147.34);

\path[draw=drawColor,line width= 0.6pt,line join=round] ( 28.93,147.34) -- (271.45,147.34);

\path[draw=drawColor,line width= 0.6pt,line join=round] ( 28.93,147.34) -- (271.45,147.34);

\path[draw=drawColor,line width= 0.6pt,line join=round] ( 28.93,147.34) -- (271.45,147.34);

\path[draw=drawColor,line width= 0.6pt,line join=round] ( 28.93,147.34) -- (271.45,147.34);

\path[draw=drawColor,line width= 0.6pt,line join=round] ( 28.93,147.34) -- (271.45,147.34);

\path[draw=drawColor,line width= 0.6pt,line join=round] ( 28.93,147.34) -- (271.45,147.34);

\path[draw=drawColor,line width= 0.6pt,line join=round] ( 28.93,147.34) -- (271.45,147.34);

\path[draw=drawColor,line width= 0.6pt,line join=round] ( 28.93,147.34) -- (271.45,147.34);

\path[draw=drawColor,line width= 0.6pt,line join=round] ( 28.93,147.34) -- (271.45,147.34);

\path[draw=drawColor,line width= 0.6pt,line join=round] ( 28.93,147.34) -- (271.45,147.34);

\path[draw=drawColor,line width= 0.6pt,line join=round] ( 28.93,147.34) -- (271.45,147.34);

\path[draw=drawColor,line width= 0.6pt,line join=round] ( 28.93,147.34) -- (271.45,147.34);

\path[draw=drawColor,line width= 0.6pt,line join=round] ( 28.93,147.34) -- (271.45,147.34);

\path[draw=drawColor,line width= 0.6pt,line join=round] ( 28.93,147.34) -- (271.45,147.34);

\path[draw=drawColor,line width= 0.6pt,line join=round] ( 28.93,147.34) -- (271.45,147.34);

\path[draw=drawColor,line width= 0.6pt,line join=round] ( 28.93,147.34) -- (271.45,147.34);

\path[draw=drawColor,line width= 0.6pt,line join=round] ( 28.93,147.34) -- (271.45,147.34);

\path[draw=drawColor,line width= 0.6pt,line join=round] ( 28.93,147.34) -- (271.45,147.34);

\path[draw=drawColor,line width= 0.6pt,line join=round] ( 28.93,147.34) -- (271.45,147.34);

\path[draw=drawColor,line width= 0.6pt,line join=round] ( 28.93,147.34) -- (271.45,147.34);

\path[draw=drawColor,line width= 0.6pt,line join=round] ( 28.93,147.34) -- (271.45,147.34);

\path[draw=drawColor,line width= 0.6pt,line join=round] ( 28.93,147.34) -- (271.45,147.34);

\path[draw=drawColor,line width= 0.6pt,line join=round] ( 28.93,147.34) -- (271.45,147.34);

\path[draw=drawColor,line width= 0.6pt,line join=round] ( 28.93,147.34) -- (271.45,147.34);

\path[draw=drawColor,line width= 0.6pt,line join=round] ( 28.93,147.34) -- (271.45,147.34);

\path[draw=drawColor,line width= 0.6pt,line join=round] ( 28.93,147.34) -- (271.45,147.34);

\path[draw=drawColor,line width= 0.6pt,line join=round] ( 28.93,147.34) -- (271.45,147.34);

\path[draw=drawColor,line width= 0.6pt,line join=round] ( 28.93,147.34) -- (271.45,147.34);

\path[draw=drawColor,line width= 0.6pt,line join=round] ( 28.93,147.34) -- (271.45,147.34);

\path[draw=drawColor,line width= 0.6pt,line join=round] ( 28.93,147.34) -- (271.45,147.34);

\path[draw=drawColor,line width= 0.6pt,line join=round] ( 28.93,147.34) -- (271.45,147.34);

\path[draw=drawColor,line width= 0.6pt,line join=round] ( 28.93,147.34) -- (271.45,147.34);

\path[draw=drawColor,line width= 0.6pt,line join=round] ( 28.93,147.34) -- (271.45,147.34);

\path[draw=drawColor,line width= 0.6pt,line join=round] ( 28.93,147.34) -- (271.45,147.34);

\path[draw=drawColor,line width= 0.6pt,line join=round] ( 28.93,147.34) -- (271.45,147.34);

\path[draw=drawColor,line width= 0.6pt,line join=round] ( 28.93,147.34) -- (271.45,147.34);

\path[draw=drawColor,line width= 0.6pt,line join=round] ( 28.93,147.34) -- (271.45,147.34);

\path[draw=drawColor,line width= 0.6pt,line join=round] ( 28.93,147.34) -- (271.45,147.34);

\path[draw=drawColor,line width= 0.6pt,line join=round] ( 28.93,147.34) -- (271.45,147.34);

\path[draw=drawColor,line width= 0.6pt,line join=round] ( 28.93,147.34) -- (271.45,147.34);

\path[draw=drawColor,line width= 0.6pt,line join=round] ( 28.93,147.34) -- (271.45,147.34);

\path[draw=drawColor,line width= 0.6pt,line join=round] ( 28.93,147.34) -- (271.45,147.34);

\path[draw=drawColor,line width= 0.6pt,line join=round] ( 28.93,147.34) -- (271.45,147.34);

\path[draw=drawColor,line width= 0.6pt,line join=round] ( 28.93,147.34) -- (271.45,147.34);

\path[draw=drawColor,line width= 0.6pt,line join=round] ( 28.93,147.34) -- (271.45,147.34);

\path[draw=drawColor,line width= 0.6pt,line join=round] ( 28.93,147.34) -- (271.45,147.34);

\path[draw=drawColor,line width= 0.6pt,line join=round] ( 28.93,147.34) -- (271.45,147.34);

\path[draw=drawColor,line width= 0.6pt,line join=round] ( 28.93,147.34) -- (271.45,147.34);

\path[draw=drawColor,line width= 0.6pt,line join=round] ( 28.93,147.34) -- (271.45,147.34);

\path[draw=drawColor,line width= 0.6pt,line join=round] ( 28.93,147.34) -- (271.45,147.34);

\path[draw=drawColor,line width= 0.6pt,line join=round] ( 28.93,147.34) -- (271.45,147.34);

\path[draw=drawColor,line width= 0.6pt,line join=round] ( 28.93,147.34) -- (271.45,147.34);

\path[draw=drawColor,line width= 0.6pt,line join=round] ( 28.93,147.34) -- (271.45,147.34);

\path[draw=drawColor,line width= 0.6pt,line join=round] ( 28.93,147.34) -- (271.45,147.34);

\path[draw=drawColor,line width= 0.6pt,line join=round] ( 28.93,147.34) -- (271.45,147.34);

\path[draw=drawColor,line width= 0.6pt,line join=round] ( 28.93,147.34) -- (271.45,147.34);

\path[draw=drawColor,line width= 0.6pt,line join=round] ( 28.93,147.34) -- (271.45,147.34);

\path[draw=drawColor,line width= 0.6pt,line join=round] ( 28.93,147.34) -- (271.45,147.34);

\path[draw=drawColor,line width= 0.6pt,line join=round] ( 28.93,147.34) -- (271.45,147.34);

\path[draw=drawColor,line width= 0.6pt,line join=round] ( 28.93,147.34) -- (271.45,147.34);

\path[draw=drawColor,line width= 0.6pt,line join=round] ( 28.93,147.34) -- (271.45,147.34);

\path[draw=drawColor,line width= 0.6pt,line join=round] ( 28.93,147.34) -- (271.45,147.34);

\path[draw=drawColor,line width= 0.6pt,line join=round] ( 28.93,147.34) -- (271.45,147.34);

\path[draw=drawColor,line width= 0.6pt,line join=round] ( 28.93,147.34) -- (271.45,147.34);

\path[draw=drawColor,line width= 0.6pt,line join=round] ( 28.93,147.34) -- (271.45,147.34);

\path[draw=drawColor,line width= 0.6pt,line join=round] ( 28.93,147.34) -- (271.45,147.34);

\path[draw=drawColor,line width= 0.6pt,line join=round] ( 28.93,147.34) -- (271.45,147.34);

\path[draw=drawColor,line width= 0.6pt,line join=round] ( 28.93,147.34) -- (271.45,147.34);

\path[draw=drawColor,line width= 0.6pt,line join=round] ( 28.93,147.34) -- (271.45,147.34);

\path[draw=drawColor,line width= 0.6pt,line join=round] ( 28.93,147.34) -- (271.45,147.34);

\path[draw=drawColor,line width= 0.6pt,line join=round] ( 28.93,147.34) -- (271.45,147.34);

\path[draw=drawColor,line width= 0.6pt,line join=round] ( 28.93,147.34) -- (271.45,147.34);

\path[draw=drawColor,line width= 0.6pt,line join=round] ( 28.93,147.34) -- (271.45,147.34);

\path[draw=drawColor,line width= 0.6pt,line join=round] ( 28.93,147.34) -- (271.45,147.34);

\path[draw=drawColor,line width= 0.6pt,line join=round] ( 28.93,147.34) -- (271.45,147.34);

\path[draw=drawColor,line width= 0.6pt,line join=round] ( 28.93,147.34) -- (271.45,147.34);

\path[draw=drawColor,line width= 0.6pt,line join=round] ( 28.93,147.34) -- (271.45,147.34);

\path[draw=drawColor,line width= 0.6pt,line join=round] ( 28.93,147.34) -- (271.45,147.34);

\path[draw=drawColor,line width= 0.6pt,line join=round] ( 28.93,147.34) -- (271.45,147.34);

\path[draw=drawColor,line width= 0.6pt,line join=round] ( 28.93,147.34) -- (271.45,147.34);

\path[draw=drawColor,line width= 0.6pt,line join=round] ( 28.93,147.34) -- (271.45,147.34);

\path[draw=drawColor,line width= 0.6pt,line join=round] ( 28.93,147.34) -- (271.45,147.34);

\path[draw=drawColor,line width= 0.6pt,line join=round] ( 28.93,147.34) -- (271.45,147.34);

\path[draw=drawColor,line width= 0.6pt,line join=round] ( 28.93,147.34) -- (271.45,147.34);

\path[draw=drawColor,line width= 0.6pt,line join=round] ( 28.93,147.34) -- (271.45,147.34);

\path[draw=drawColor,line width= 0.6pt,line join=round] ( 28.93,147.34) -- (271.45,147.34);

\path[draw=drawColor,line width= 0.6pt,line join=round] ( 28.93,147.34) -- (271.45,147.34);

\path[draw=drawColor,line width= 0.6pt,line join=round] ( 28.93,147.34) -- (271.45,147.34);

\path[draw=drawColor,line width= 0.6pt,line join=round] ( 28.93,147.34) -- (271.45,147.34);

\path[draw=drawColor,line width= 0.6pt,line join=round] ( 28.93,147.34) -- (271.45,147.34);

\path[draw=drawColor,line width= 0.6pt,line join=round] ( 28.93,147.34) -- (271.45,147.34);

\path[draw=drawColor,line width= 0.6pt,line join=round] ( 28.93,147.34) -- (271.45,147.34);

\path[draw=drawColor,line width= 0.6pt,line join=round] ( 28.93,147.34) -- (271.45,147.34);

\path[draw=drawColor,line width= 0.6pt,line join=round] ( 28.93,147.34) -- (271.45,147.34);

\path[draw=drawColor,line width= 0.6pt,line join=round] ( 28.93,147.34) -- (271.45,147.34);

\path[draw=drawColor,line width= 0.6pt,line join=round] ( 28.93,147.34) -- (271.45,147.34);

\path[draw=drawColor,line width= 0.6pt,line join=round] ( 28.93,147.34) -- (271.45,147.34);

\path[draw=drawColor,line width= 0.6pt,line join=round] ( 28.93,147.34) -- (271.45,147.34);

\path[draw=drawColor,line width= 0.6pt,line join=round] ( 28.93,147.34) -- (271.45,147.34);

\path[draw=drawColor,line width= 0.6pt,line join=round] ( 28.93,147.34) -- (271.45,147.34);

\path[draw=drawColor,line width= 0.6pt,line join=round] ( 28.93,147.34) -- (271.45,147.34);

\path[draw=drawColor,line width= 0.6pt,line join=round] ( 28.93,147.34) -- (271.45,147.34);

\path[draw=drawColor,line width= 0.6pt,line join=round] ( 28.93,147.34) -- (271.45,147.34);

\path[draw=drawColor,line width= 0.6pt,line join=round] ( 28.93,147.34) -- (271.45,147.34);

\path[draw=drawColor,line width= 0.6pt,line join=round] ( 28.93,147.34) -- (271.45,147.34);

\path[draw=drawColor,line width= 0.6pt,line join=round] ( 28.93,147.34) -- (271.45,147.34);

\path[draw=drawColor,line width= 0.6pt,line join=round] ( 28.93,147.34) -- (271.45,147.34);

\path[draw=drawColor,line width= 0.6pt,line join=round] ( 28.93,147.34) -- (271.45,147.34);

\path[draw=drawColor,line width= 0.6pt,line join=round] ( 28.93,147.34) -- (271.45,147.34);

\path[draw=drawColor,line width= 0.6pt,line join=round] ( 28.93,147.34) -- (271.45,147.34);

\path[draw=drawColor,line width= 0.6pt,line join=round] ( 28.93,147.34) -- (271.45,147.34);

\path[draw=drawColor,line width= 0.6pt,line join=round] ( 28.93,147.34) -- (271.45,147.34);

\path[draw=drawColor,line width= 0.6pt,line join=round] ( 28.93,147.34) -- (271.45,147.34);

\path[draw=drawColor,line width= 0.6pt,line join=round] ( 28.93,147.34) -- (271.45,147.34);

\path[draw=drawColor,line width= 0.6pt,line join=round] ( 28.93,147.34) -- (271.45,147.34);

\path[draw=drawColor,line width= 0.6pt,line join=round] ( 28.93,147.34) -- (271.45,147.34);

\path[draw=drawColor,line width= 0.6pt,line join=round] ( 28.93,147.34) -- (271.45,147.34);

\path[draw=drawColor,line width= 0.6pt,line join=round] ( 28.93,147.34) -- (271.45,147.34);

\path[draw=drawColor,line width= 0.6pt,line join=round] ( 28.93,147.34) -- (271.45,147.34);

\path[draw=drawColor,line width= 0.6pt,line join=round] ( 28.93,147.34) -- (271.45,147.34);

\path[draw=drawColor,line width= 0.6pt,line join=round] ( 28.93,147.34) -- (271.45,147.34);

\path[draw=drawColor,line width= 0.6pt,line join=round] ( 28.93,147.34) -- (271.45,147.34);

\path[draw=drawColor,line width= 0.6pt,line join=round] ( 28.93,147.34) -- (271.45,147.34);

\path[draw=drawColor,line width= 0.6pt,line join=round] ( 28.93,147.34) -- (271.45,147.34);

\path[draw=drawColor,line width= 0.6pt,line join=round] ( 28.93,147.34) -- (271.45,147.34);

\path[draw=drawColor,line width= 0.6pt,line join=round] ( 28.93,147.34) -- (271.45,147.34);

\path[draw=drawColor,line width= 0.6pt,line join=round] ( 28.93,147.34) -- (271.45,147.34);

\path[draw=drawColor,line width= 0.6pt,line join=round] ( 28.93,147.34) -- (271.45,147.34);

\path[draw=drawColor,line width= 0.6pt,line join=round] ( 28.93,147.34) -- (271.45,147.34);

\path[draw=drawColor,line width= 0.6pt,line join=round] ( 28.93,147.34) -- (271.45,147.34);

\path[draw=drawColor,line width= 0.6pt,line join=round] ( 28.93,147.34) -- (271.45,147.34);

\path[draw=drawColor,line width= 0.6pt,line join=round] ( 28.93,147.34) -- (271.45,147.34);

\path[draw=drawColor,line width= 0.6pt,line join=round] ( 28.93,147.34) -- (271.45,147.34);

\path[draw=drawColor,line width= 0.6pt,line join=round] ( 28.93,147.34) -- (271.45,147.34);

\path[draw=drawColor,line width= 0.6pt,line join=round] ( 28.93,147.34) -- (271.45,147.34);

\path[draw=drawColor,line width= 0.6pt,line join=round] ( 28.93,147.34) -- (271.45,147.34);

\path[draw=drawColor,line width= 0.6pt,line join=round] ( 28.93,147.34) -- (271.45,147.34);

\path[draw=drawColor,line width= 0.6pt,line join=round] ( 28.93,147.34) -- (271.45,147.34);

\path[draw=drawColor,line width= 0.6pt,line join=round] ( 28.93,147.34) -- (271.45,147.34);

\path[draw=drawColor,line width= 0.6pt,line join=round] ( 28.93,147.34) -- (271.45,147.34);

\path[draw=drawColor,line width= 0.6pt,line join=round] ( 28.93,147.34) -- (271.45,147.34);

\path[draw=drawColor,line width= 0.6pt,line join=round] ( 28.93,147.34) -- (271.45,147.34);

\path[draw=drawColor,line width= 0.6pt,line join=round] ( 28.93,147.34) -- (271.45,147.34);

\path[draw=drawColor,line width= 0.6pt,line join=round] ( 28.93,147.34) -- (271.45,147.34);

\path[draw=drawColor,line width= 0.6pt,line join=round] ( 28.93,147.34) -- (271.45,147.34);

\path[draw=drawColor,line width= 0.6pt,line join=round] ( 28.93,147.34) -- (271.45,147.34);

\path[draw=drawColor,line width= 0.6pt,line join=round] ( 28.93,147.34) -- (271.45,147.34);

\path[draw=drawColor,line width= 0.6pt,line join=round] ( 28.93,147.34) -- (271.45,147.34);

\path[draw=drawColor,line width= 0.6pt,line join=round] ( 28.93,147.34) -- (271.45,147.34);

\path[draw=drawColor,line width= 0.6pt,line join=round] ( 28.93,147.34) -- (271.45,147.34);

\path[draw=drawColor,line width= 0.6pt,line join=round] ( 28.93,147.34) -- (271.45,147.34);

\path[draw=drawColor,line width= 0.6pt,line join=round] ( 28.93,147.34) -- (271.45,147.34);

\path[draw=drawColor,line width= 0.6pt,line join=round] ( 28.93,147.34) -- (271.45,147.34);

\path[draw=drawColor,line width= 0.6pt,line join=round] ( 28.93,147.34) -- (271.45,147.34);

\path[draw=drawColor,line width= 0.6pt,line join=round] ( 28.93,147.34) -- (271.45,147.34);

\path[draw=drawColor,line width= 0.6pt,line join=round] ( 28.93,147.34) -- (271.45,147.34);

\path[draw=drawColor,line width= 0.6pt,line join=round] ( 28.93,147.34) -- (271.45,147.34);

\path[draw=drawColor,line width= 0.6pt,line join=round] ( 28.93,147.34) -- (271.45,147.34);

\path[draw=drawColor,line width= 0.6pt,line join=round] ( 28.93,147.34) -- (271.45,147.34);

\path[draw=drawColor,line width= 0.6pt,line join=round] ( 28.93,147.34) -- (271.45,147.34);

\path[draw=drawColor,line width= 0.6pt,line join=round] ( 28.93,147.34) -- (271.45,147.34);

\path[draw=drawColor,line width= 0.6pt,line join=round] ( 28.93,147.34) -- (271.45,147.34);

\path[draw=drawColor,line width= 0.6pt,line join=round] ( 28.93,147.34) -- (271.45,147.34);

\path[draw=drawColor,line width= 0.6pt,line join=round] ( 28.93,147.34) -- (271.45,147.34);

\path[draw=drawColor,line width= 0.6pt,line join=round] ( 28.93,147.34) -- (271.45,147.34);

\path[draw=drawColor,line width= 0.6pt,line join=round] ( 28.93,147.34) -- (271.45,147.34);

\path[draw=drawColor,line width= 0.6pt,line join=round] ( 28.93,147.34) -- (271.45,147.34);

\path[draw=drawColor,line width= 0.6pt,line join=round] ( 28.93,147.34) -- (271.45,147.34);

\path[draw=drawColor,line width= 0.6pt,line join=round] ( 28.93,147.34) -- (271.45,147.34);

\path[draw=drawColor,line width= 0.6pt,line join=round] ( 28.93,147.34) -- (271.45,147.34);

\path[draw=drawColor,line width= 0.6pt,line join=round] ( 28.93,147.34) -- (271.45,147.34);

\path[draw=drawColor,line width= 0.6pt,line join=round] ( 28.93,147.34) -- (271.45,147.34);

\path[draw=drawColor,line width= 0.6pt,line join=round] ( 28.93,147.34) -- (271.45,147.34);

\path[draw=drawColor,line width= 0.6pt,line join=round] ( 28.93,147.34) -- (271.45,147.34);

\path[draw=drawColor,line width= 0.6pt,line join=round] ( 28.93,147.34) -- (271.45,147.34);

\path[draw=drawColor,line width= 0.6pt,line join=round] ( 28.93,147.34) -- (271.45,147.34);

\path[draw=drawColor,line width= 0.6pt,line join=round] ( 28.93,147.34) -- (271.45,147.34);

\path[draw=drawColor,line width= 0.6pt,line join=round] ( 28.93,147.34) -- (271.45,147.34);

\path[draw=drawColor,line width= 0.6pt,line join=round] ( 28.93,147.34) -- (271.45,147.34);

\path[draw=drawColor,line width= 0.6pt,line join=round] ( 28.93,147.34) -- (271.45,147.34);

\path[draw=drawColor,line width= 0.6pt,line join=round] ( 28.93,147.34) -- (271.45,147.34);

\path[draw=drawColor,line width= 0.6pt,line join=round] ( 28.93,147.34) -- (271.45,147.34);

\path[draw=drawColor,line width= 0.6pt,line join=round] ( 28.93,147.34) -- (271.45,147.34);

\path[draw=drawColor,line width= 0.6pt,line join=round] ( 28.93,147.34) -- (271.45,147.34);

\path[draw=drawColor,line width= 0.6pt,line join=round] ( 28.93,147.34) -- (271.45,147.34);

\path[draw=drawColor,line width= 0.6pt,line join=round] ( 28.93,147.34) -- (271.45,147.34);

\path[draw=drawColor,line width= 0.6pt,line join=round] ( 28.93,147.34) -- (271.45,147.34);

\path[draw=drawColor,line width= 0.6pt,line join=round] ( 28.93,147.34) -- (271.45,147.34);

\path[draw=drawColor,line width= 0.6pt,line join=round] ( 28.93,147.34) -- (271.45,147.34);

\path[draw=drawColor,line width= 0.6pt,line join=round] ( 28.93,147.34) -- (271.45,147.34);

\path[draw=drawColor,line width= 0.6pt,line join=round] ( 28.93,147.34) -- (271.45,147.34);

\path[draw=drawColor,line width= 0.6pt,line join=round] ( 28.93,147.34) -- (271.45,147.34);

\path[draw=drawColor,line width= 0.6pt,line join=round] ( 28.93,147.34) -- (271.45,147.34);

\path[draw=drawColor,line width= 0.6pt,line join=round] ( 28.93,147.34) -- (271.45,147.34);

\path[draw=drawColor,line width= 0.6pt,line join=round] ( 28.93,147.34) -- (271.45,147.34);

\path[draw=drawColor,line width= 0.6pt,line join=round] ( 28.93,147.34) -- (271.45,147.34);

\path[draw=drawColor,line width= 0.6pt,line join=round] ( 28.93,147.34) -- (271.45,147.34);

\path[draw=drawColor,line width= 0.6pt,line join=round] ( 28.93,147.34) -- (271.45,147.34);

\path[draw=drawColor,line width= 0.6pt,line join=round] ( 28.93,147.34) -- (271.45,147.34);

\path[draw=drawColor,line width= 0.6pt,line join=round] ( 28.93,147.34) -- (271.45,147.34);

\path[draw=drawColor,line width= 0.6pt,line join=round] ( 28.93,147.34) -- (271.45,147.34);

\path[draw=drawColor,line width= 0.6pt,line join=round] ( 28.93,147.34) -- (271.45,147.34);

\path[draw=drawColor,line width= 0.6pt,line join=round] ( 28.93,147.34) -- (271.45,147.34);

\path[draw=drawColor,line width= 0.6pt,line join=round] ( 28.93,147.34) -- (271.45,147.34);

\path[draw=drawColor,line width= 0.6pt,line join=round] ( 28.93,147.34) -- (271.45,147.34);

\path[draw=drawColor,line width= 0.6pt,line join=round] ( 28.93,147.34) -- (271.45,147.34);

\path[draw=drawColor,line width= 0.6pt,line join=round] ( 28.93,147.34) -- (271.45,147.34);

\path[draw=drawColor,line width= 0.6pt,line join=round] ( 28.93,147.34) -- (271.45,147.34);

\path[draw=drawColor,line width= 0.6pt,line join=round] ( 28.93,147.34) -- (271.45,147.34);

\path[draw=drawColor,line width= 0.6pt,line join=round] ( 28.93,147.34) -- (271.45,147.34);

\path[draw=drawColor,line width= 0.6pt,line join=round] ( 28.93,147.34) -- (271.45,147.34);

\path[draw=drawColor,line width= 0.6pt,line join=round] ( 28.93,147.34) -- (271.45,147.34);

\path[draw=drawColor,line width= 0.6pt,line join=round] ( 28.93,147.34) -- (271.45,147.34);

\path[draw=drawColor,line width= 0.6pt,line join=round] ( 28.93,147.34) -- (271.45,147.34);

\path[draw=drawColor,line width= 0.6pt,line join=round] ( 28.93,147.34) -- (271.45,147.34);

\path[draw=drawColor,line width= 0.6pt,line join=round] ( 28.93,147.34) -- (271.45,147.34);

\path[draw=drawColor,line width= 0.6pt,line join=round] ( 28.93,147.34) -- (271.45,147.34);

\path[draw=drawColor,line width= 0.6pt,line join=round] ( 28.93,147.34) -- (271.45,147.34);

\path[draw=drawColor,line width= 0.6pt,line join=round] ( 28.93,147.34) -- (271.45,147.34);

\path[draw=drawColor,line width= 0.6pt,line join=round] ( 28.93,147.34) -- (271.45,147.34);

\path[draw=drawColor,line width= 0.6pt,line join=round] ( 28.93,147.34) -- (271.45,147.34);

\path[draw=drawColor,line width= 0.6pt,line join=round] ( 28.93,147.34) -- (271.45,147.34);

\path[draw=drawColor,line width= 0.6pt,line join=round] ( 28.93,147.34) -- (271.45,147.34);

\path[draw=drawColor,line width= 0.6pt,line join=round] ( 28.93,147.34) -- (271.45,147.34);

\path[draw=drawColor,line width= 0.6pt,line join=round] ( 28.93,147.34) -- (271.45,147.34);

\path[draw=drawColor,line width= 0.6pt,line join=round] ( 28.93,147.34) -- (271.45,147.34);

\path[draw=drawColor,line width= 0.6pt,line join=round] ( 28.93,147.34) -- (271.45,147.34);

\path[draw=drawColor,line width= 0.6pt,line join=round] ( 28.93,147.34) -- (271.45,147.34);

\path[draw=drawColor,line width= 0.6pt,line join=round] ( 28.93,147.34) -- (271.45,147.34);

\path[draw=drawColor,line width= 0.6pt,line join=round] ( 28.93,147.34) -- (271.45,147.34);

\path[draw=drawColor,line width= 0.6pt,line join=round] ( 28.93,147.34) -- (271.45,147.34);

\path[draw=drawColor,line width= 0.6pt,line join=round] ( 28.93,147.34) -- (271.45,147.34);

\path[draw=drawColor,line width= 0.6pt,line join=round] ( 28.93,147.34) -- (271.45,147.34);

\path[draw=drawColor,line width= 0.6pt,line join=round] ( 28.93,147.34) -- (271.45,147.34);

\path[draw=drawColor,line width= 0.6pt,line join=round] ( 28.93,147.34) -- (271.45,147.34);

\path[draw=drawColor,line width= 0.6pt,line join=round] ( 28.93,147.34) -- (271.45,147.34);

\path[draw=drawColor,line width= 0.6pt,line join=round] ( 28.93,147.34) -- (271.45,147.34);

\path[draw=drawColor,line width= 0.6pt,line join=round] ( 28.93,147.34) -- (271.45,147.34);

\path[draw=drawColor,line width= 0.6pt,line join=round] ( 28.93,147.34) -- (271.45,147.34);

\path[draw=drawColor,line width= 0.6pt,line join=round] ( 28.93,147.34) -- (271.45,147.34);

\path[draw=drawColor,line width= 0.6pt,line join=round] ( 28.93,147.34) -- (271.45,147.34);

\path[draw=drawColor,line width= 0.6pt,line join=round] ( 28.93,147.34) -- (271.45,147.34);

\path[draw=drawColor,line width= 0.6pt,line join=round] ( 28.93,147.34) -- (271.45,147.34);

\path[draw=drawColor,line width= 0.6pt,line join=round] ( 28.93,147.34) -- (271.45,147.34);

\path[draw=drawColor,line width= 0.6pt,line join=round] ( 28.93,147.34) -- (271.45,147.34);

\path[draw=drawColor,line width= 0.6pt,line join=round] ( 28.93,147.34) -- (271.45,147.34);

\path[draw=drawColor,line width= 0.6pt,line join=round] ( 28.93,147.34) -- (271.45,147.34);

\path[draw=drawColor,line width= 0.6pt,line join=round] ( 28.93,147.34) -- (271.45,147.34);

\path[draw=drawColor,line width= 0.6pt,line join=round] ( 28.93,147.34) -- (271.45,147.34);

\path[draw=drawColor,line width= 0.6pt,line join=round] ( 28.93,147.34) -- (271.45,147.34);

\path[draw=drawColor,line width= 0.6pt,line join=round] ( 28.93,147.34) -- (271.45,147.34);

\path[draw=drawColor,line width= 0.6pt,line join=round] ( 28.93,147.34) -- (271.45,147.34);

\path[draw=drawColor,line width= 0.6pt,line join=round] ( 28.93,147.34) -- (271.45,147.34);

\path[draw=drawColor,line width= 0.6pt,line join=round] ( 28.93,147.34) -- (271.45,147.34);

\path[draw=drawColor,line width= 0.6pt,line join=round] ( 28.93,147.34) -- (271.45,147.34);

\path[draw=drawColor,line width= 0.6pt,line join=round] ( 28.93,147.34) -- (271.45,147.34);

\path[draw=drawColor,line width= 0.6pt,line join=round] ( 28.93,147.34) -- (271.45,147.34);

\path[draw=drawColor,line width= 0.6pt,line join=round] ( 28.93,147.34) -- (271.45,147.34);

\path[draw=drawColor,line width= 0.6pt,line join=round] ( 28.93,147.34) -- (271.45,147.34);

\path[draw=drawColor,line width= 0.6pt,line join=round] ( 28.93,147.34) -- (271.45,147.34);

\path[draw=drawColor,line width= 0.6pt,line join=round] ( 28.93,147.34) -- (271.45,147.34);

\path[draw=drawColor,line width= 0.6pt,line join=round] ( 28.93,147.34) -- (271.45,147.34);

\path[draw=drawColor,line width= 0.6pt,line join=round] ( 28.93,147.34) -- (271.45,147.34);

\path[draw=drawColor,line width= 0.6pt,line join=round] ( 28.93,147.34) -- (271.45,147.34);

\path[draw=drawColor,line width= 0.6pt,line join=round] ( 28.93,147.34) -- (271.45,147.34);

\path[draw=drawColor,line width= 0.6pt,line join=round] ( 28.93,147.34) -- (271.45,147.34);

\path[draw=drawColor,line width= 0.6pt,line join=round] ( 28.93,147.34) -- (271.45,147.34);

\path[draw=drawColor,line width= 0.6pt,line join=round] ( 28.93,147.34) -- (271.45,147.34);

\path[draw=drawColor,line width= 0.6pt,line join=round] ( 28.93,147.34) -- (271.45,147.34);

\path[draw=drawColor,line width= 0.6pt,line join=round] ( 28.93,147.34) -- (271.45,147.34);

\path[draw=drawColor,line width= 0.6pt,line join=round] ( 28.93,147.34) -- (271.45,147.34);

\path[draw=drawColor,line width= 0.6pt,line join=round] ( 28.93,147.34) -- (271.45,147.34);

\path[draw=drawColor,line width= 0.6pt,line join=round] ( 28.93,147.34) -- (271.45,147.34);

\path[draw=drawColor,line width= 0.6pt,line join=round] ( 28.93,147.34) -- (271.45,147.34);

\path[draw=drawColor,line width= 0.6pt,line join=round] ( 28.93,147.34) -- (271.45,147.34);

\path[draw=drawColor,line width= 0.6pt,line join=round] ( 28.93,147.34) -- (271.45,147.34);

\path[draw=drawColor,line width= 0.6pt,line join=round] ( 28.93,147.34) -- (271.45,147.34);

\path[draw=drawColor,line width= 0.6pt,line join=round] ( 28.93,147.34) -- (271.45,147.34);

\path[draw=drawColor,line width= 0.6pt,line join=round] ( 28.93,147.34) -- (271.45,147.34);

\path[draw=drawColor,line width= 0.6pt,line join=round] ( 28.93,147.34) -- (271.45,147.34);

\path[draw=drawColor,line width= 0.6pt,line join=round] ( 28.93,147.34) -- (271.45,147.34);

\path[draw=drawColor,line width= 0.6pt,line join=round] ( 28.93,147.34) -- (271.45,147.34);

\path[draw=drawColor,line width= 0.6pt,line join=round] ( 28.93,147.34) -- (271.45,147.34);

\path[draw=drawColor,line width= 0.6pt,line join=round] ( 28.93,147.34) -- (271.45,147.34);

\path[draw=drawColor,line width= 0.6pt,line join=round] ( 28.93,147.34) -- (271.45,147.34);

\path[draw=drawColor,line width= 0.6pt,line join=round] ( 28.93,147.34) -- (271.45,147.34);

\path[draw=drawColor,line width= 0.6pt,line join=round] ( 28.93,147.34) -- (271.45,147.34);

\path[draw=drawColor,line width= 0.6pt,line join=round] ( 28.93,147.34) -- (271.45,147.34);

\path[draw=drawColor,line width= 0.6pt,line join=round] ( 28.93,147.34) -- (271.45,147.34);

\path[draw=drawColor,line width= 0.6pt,line join=round] ( 28.93,147.34) -- (271.45,147.34);

\path[draw=drawColor,line width= 0.6pt,line join=round] ( 28.93,147.34) -- (271.45,147.34);

\path[draw=drawColor,line width= 0.6pt,line join=round] ( 28.93,147.34) -- (271.45,147.34);

\path[draw=drawColor,line width= 0.6pt,line join=round] ( 28.93,147.34) -- (271.45,147.34);

\path[draw=drawColor,line width= 0.6pt,line join=round] ( 28.93,147.34) -- (271.45,147.34);

\path[draw=drawColor,line width= 0.6pt,line join=round] ( 28.93,147.34) -- (271.45,147.34);

\path[draw=drawColor,line width= 0.6pt,line join=round] ( 28.93,147.34) -- (271.45,147.34);

\path[draw=drawColor,line width= 0.6pt,line join=round] ( 28.93,147.34) -- (271.45,147.34);

\path[draw=drawColor,line width= 0.6pt,line join=round] ( 28.93,147.34) -- (271.45,147.34);

\path[draw=drawColor,line width= 0.6pt,line join=round] ( 28.93,147.34) -- (271.45,147.34);

\path[draw=drawColor,line width= 0.6pt,line join=round] ( 28.93,147.34) -- (271.45,147.34);

\path[draw=drawColor,line width= 0.6pt,line join=round] ( 28.93,147.34) -- (271.45,147.34);

\path[draw=drawColor,line width= 0.6pt,line join=round] ( 28.93,147.34) -- (271.45,147.34);

\path[draw=drawColor,line width= 0.6pt,line join=round] ( 28.93,147.34) -- (271.45,147.34);

\path[draw=drawColor,line width= 0.6pt,line join=round] ( 28.93,147.34) -- (271.45,147.34);

\path[draw=drawColor,line width= 0.6pt,line join=round] ( 28.93,147.34) -- (271.45,147.34);

\path[draw=drawColor,line width= 0.6pt,line join=round] ( 28.93,147.34) -- (271.45,147.34);

\path[draw=drawColor,line width= 0.6pt,line join=round] ( 28.93,147.34) -- (271.45,147.34);

\path[draw=drawColor,line width= 0.6pt,line join=round] ( 28.93,147.34) -- (271.45,147.34);

\path[draw=drawColor,line width= 0.6pt,line join=round] ( 28.93,147.34) -- (271.45,147.34);

\path[draw=drawColor,line width= 0.6pt,line join=round] ( 28.93,147.34) -- (271.45,147.34);

\path[draw=drawColor,line width= 0.6pt,line join=round] ( 28.93,147.34) -- (271.45,147.34);

\path[draw=drawColor,line width= 0.6pt,line join=round] ( 28.93,147.34) -- (271.45,147.34);

\path[draw=drawColor,line width= 0.6pt,line join=round] ( 28.93,147.34) -- (271.45,147.34);

\path[draw=drawColor,line width= 0.6pt,line join=round] ( 28.93,147.34) -- (271.45,147.34);

\path[draw=drawColor,line width= 0.6pt,line join=round] ( 28.93,147.34) -- (271.45,147.34);

\path[draw=drawColor,line width= 0.6pt,line join=round] ( 28.93,147.34) -- (271.45,147.34);

\path[draw=drawColor,line width= 0.6pt,line join=round] ( 28.93,147.34) -- (271.45,147.34);

\path[draw=drawColor,line width= 0.6pt,line join=round] ( 28.93,147.34) -- (271.45,147.34);

\path[draw=drawColor,line width= 0.6pt,line join=round] ( 28.93,147.34) -- (271.45,147.34);

\path[draw=drawColor,line width= 0.6pt,line join=round] ( 28.93,147.34) -- (271.45,147.34);

\path[draw=drawColor,line width= 0.6pt,line join=round] ( 28.93,147.34) -- (271.45,147.34);

\path[draw=drawColor,line width= 0.6pt,line join=round] ( 28.93,147.34) -- (271.45,147.34);

\path[draw=drawColor,line width= 0.6pt,line join=round] ( 28.93,147.34) -- (271.45,147.34);

\path[draw=drawColor,line width= 0.6pt,line join=round] ( 28.93,147.34) -- (271.45,147.34);

\path[draw=drawColor,line width= 0.6pt,line join=round] ( 28.93,147.34) -- (271.45,147.34);

\path[draw=drawColor,line width= 0.6pt,line join=round] ( 28.93,147.34) -- (271.45,147.34);

\path[draw=drawColor,line width= 0.6pt,line join=round] ( 28.93,147.34) -- (271.45,147.34);

\path[draw=drawColor,line width= 0.6pt,line join=round] ( 28.93,147.34) -- (271.45,147.34);

\path[draw=drawColor,line width= 0.6pt,line join=round] ( 28.93,147.34) -- (271.45,147.34);

\path[draw=drawColor,line width= 0.6pt,line join=round] ( 28.93,147.34) -- (271.45,147.34);

\path[draw=drawColor,line width= 0.6pt,line join=round] ( 28.93,147.34) -- (271.45,147.34);

\path[draw=drawColor,line width= 0.6pt,line join=round] ( 28.93,147.34) -- (271.45,147.34);

\path[draw=drawColor,line width= 0.6pt,line join=round] ( 28.93,147.34) -- (271.45,147.34);

\path[draw=drawColor,line width= 0.6pt,line join=round] ( 28.93,147.34) -- (271.45,147.34);

\path[draw=drawColor,line width= 0.6pt,line join=round] ( 28.93,147.34) -- (271.45,147.34);

\path[draw=drawColor,line width= 0.6pt,line join=round] ( 28.93,147.34) -- (271.45,147.34);

\path[draw=drawColor,line width= 0.6pt,line join=round] ( 28.93,147.34) -- (271.45,147.34);

\path[draw=drawColor,line width= 0.6pt,line join=round] ( 28.93,147.34) -- (271.45,147.34);

\path[draw=drawColor,line width= 0.6pt,line join=round] ( 28.93,147.34) -- (271.45,147.34);

\path[draw=drawColor,line width= 0.6pt,line join=round] ( 28.93,147.34) -- (271.45,147.34);

\path[draw=drawColor,line width= 0.6pt,line join=round] ( 28.93,147.34) -- (271.45,147.34);

\path[draw=drawColor,line width= 0.6pt,line join=round] ( 28.93,147.34) -- (271.45,147.34);

\path[draw=drawColor,line width= 0.6pt,line join=round] ( 28.93,147.34) -- (271.45,147.34);

\path[draw=drawColor,line width= 0.6pt,line join=round] ( 28.93,147.34) -- (271.45,147.34);

\path[draw=drawColor,line width= 0.6pt,line join=round] ( 28.93,147.34) -- (271.45,147.34);

\path[draw=drawColor,line width= 0.6pt,line join=round] ( 28.93,147.34) -- (271.45,147.34);

\path[draw=drawColor,line width= 0.6pt,line join=round] ( 28.93,147.34) -- (271.45,147.34);

\path[draw=drawColor,line width= 0.6pt,line join=round] ( 28.93,147.34) -- (271.45,147.34);

\path[draw=drawColor,line width= 0.6pt,line join=round] ( 28.93,147.34) -- (271.45,147.34);

\path[draw=drawColor,line width= 0.6pt,line join=round] ( 28.93,147.34) -- (271.45,147.34);

\path[draw=drawColor,line width= 0.6pt,line join=round] ( 28.93,147.34) -- (271.45,147.34);

\path[draw=drawColor,line width= 0.6pt,line join=round] ( 28.93,147.34) -- (271.45,147.34);

\path[draw=drawColor,line width= 0.6pt,line join=round] ( 28.93,147.34) -- (271.45,147.34);

\path[draw=drawColor,line width= 0.6pt,line join=round] ( 28.93,147.34) -- (271.45,147.34);

\path[draw=drawColor,line width= 0.6pt,line join=round] ( 28.93,147.34) -- (271.45,147.34);

\path[draw=drawColor,line width= 0.6pt,line join=round] ( 28.93,147.34) -- (271.45,147.34);

\path[draw=drawColor,line width= 0.6pt,line join=round] ( 28.93,147.34) -- (271.45,147.34);

\path[draw=drawColor,line width= 0.6pt,line join=round] ( 28.93,147.34) -- (271.45,147.34);

\path[draw=drawColor,line width= 0.6pt,line join=round] ( 28.93,147.34) -- (271.45,147.34);

\path[draw=drawColor,line width= 0.6pt,line join=round] ( 28.93,147.34) -- (271.45,147.34);

\path[draw=drawColor,line width= 0.6pt,line join=round] ( 28.93,147.34) -- (271.45,147.34);

\path[draw=drawColor,line width= 0.6pt,line join=round] ( 28.93,147.34) -- (271.45,147.34);

\path[draw=drawColor,line width= 0.6pt,line join=round] ( 28.93,147.34) -- (271.45,147.34);

\path[draw=drawColor,line width= 0.6pt,line join=round] ( 28.93,147.34) -- (271.45,147.34);

\path[draw=drawColor,line width= 0.6pt,line join=round] ( 28.93,147.34) -- (271.45,147.34);

\path[draw=drawColor,line width= 0.6pt,line join=round] ( 28.93,147.34) -- (271.45,147.34);

\path[draw=drawColor,line width= 0.6pt,line join=round] ( 28.93,147.34) -- (271.45,147.34);

\path[draw=drawColor,line width= 0.6pt,line join=round] ( 28.93,147.34) -- (271.45,147.34);

\path[draw=drawColor,line width= 0.6pt,line join=round] ( 28.93,147.34) -- (271.45,147.34);

\path[draw=drawColor,line width= 0.6pt,line join=round] ( 28.93,147.34) -- (271.45,147.34);

\path[draw=drawColor,line width= 0.6pt,line join=round] ( 28.93,147.34) -- (271.45,147.34);

\path[draw=drawColor,line width= 0.6pt,line join=round] ( 28.93,147.34) -- (271.45,147.34);

\path[draw=drawColor,line width= 0.6pt,line join=round] ( 28.93,147.34) -- (271.45,147.34);

\path[draw=drawColor,line width= 0.6pt,line join=round] ( 28.93,147.34) -- (271.45,147.34);

\path[draw=drawColor,line width= 0.6pt,line join=round] ( 28.93,147.34) -- (271.45,147.34);

\path[draw=drawColor,line width= 0.6pt,line join=round] ( 28.93,147.34) -- (271.45,147.34);

\path[draw=drawColor,line width= 0.6pt,line join=round] ( 28.93,147.34) -- (271.45,147.34);

\path[draw=drawColor,line width= 0.6pt,line join=round] ( 28.93,147.34) -- (271.45,147.34);

\path[draw=drawColor,line width= 0.6pt,line join=round] ( 28.93,147.34) -- (271.45,147.34);

\path[draw=drawColor,line width= 0.6pt,line join=round] ( 28.93,147.34) -- (271.45,147.34);

\path[draw=drawColor,line width= 0.6pt,line join=round] ( 28.93,147.34) -- (271.45,147.34);

\path[draw=drawColor,line width= 0.6pt,line join=round] ( 28.93,147.34) -- (271.45,147.34);

\path[draw=drawColor,line width= 0.6pt,line join=round] ( 28.93,147.34) -- (271.45,147.34);

\path[draw=drawColor,line width= 0.6pt,line join=round] ( 28.93,147.34) -- (271.45,147.34);

\path[draw=drawColor,line width= 0.6pt,line join=round] ( 28.93,147.34) -- (271.45,147.34);

\path[draw=drawColor,line width= 0.6pt,line join=round] ( 28.93,147.34) -- (271.45,147.34);

\path[draw=drawColor,line width= 0.6pt,line join=round] ( 28.93,147.34) -- (271.45,147.34);

\path[draw=drawColor,line width= 0.6pt,line join=round] ( 28.93,147.34) -- (271.45,147.34);

\path[draw=drawColor,line width= 0.6pt,line join=round] ( 28.93,147.34) -- (271.45,147.34);

\path[draw=drawColor,line width= 0.6pt,line join=round] ( 28.93,147.34) -- (271.45,147.34);

\path[draw=drawColor,line width= 0.6pt,line join=round] ( 28.93,147.34) -- (271.45,147.34);

\path[draw=drawColor,line width= 0.6pt,line join=round] ( 28.93,147.34) -- (271.45,147.34);

\path[draw=drawColor,line width= 0.6pt,line join=round] ( 28.93,147.34) -- (271.45,147.34);

\path[draw=drawColor,line width= 0.6pt,line join=round] ( 28.93,147.34) -- (271.45,147.34);

\path[draw=drawColor,line width= 0.6pt,line join=round] ( 28.93,147.34) -- (271.45,147.34);

\path[draw=drawColor,line width= 0.6pt,line join=round] ( 28.93,147.34) -- (271.45,147.34);

\path[draw=drawColor,line width= 0.6pt,line join=round] ( 28.93,147.34) -- (271.45,147.34);

\path[draw=drawColor,line width= 0.6pt,line join=round] ( 28.93,147.34) -- (271.45,147.34);

\path[draw=drawColor,line width= 0.6pt,line join=round] ( 28.93,147.34) -- (271.45,147.34);

\path[draw=drawColor,line width= 0.6pt,line join=round] ( 28.93,147.34) -- (271.45,147.34);

\path[draw=drawColor,line width= 0.6pt,line join=round] ( 28.93,147.34) -- (271.45,147.34);

\path[draw=drawColor,line width= 0.6pt,line join=round] ( 28.93,147.34) -- (271.45,147.34);

\path[draw=drawColor,line width= 0.6pt,line join=round] ( 28.93,147.34) -- (271.45,147.34);

\path[draw=drawColor,line width= 0.6pt,line join=round] ( 28.93,147.34) -- (271.45,147.34);

\path[draw=drawColor,line width= 0.6pt,line join=round] ( 28.93,147.34) -- (271.45,147.34);

\path[draw=drawColor,line width= 0.6pt,line join=round] ( 28.93,147.34) -- (271.45,147.34);

\path[draw=drawColor,line width= 0.6pt,line join=round] ( 28.93,147.34) -- (271.45,147.34);

\path[draw=drawColor,line width= 0.6pt,line join=round] ( 28.93,147.34) -- (271.45,147.34);

\path[draw=drawColor,line width= 0.6pt,line join=round] ( 28.93,147.34) -- (271.45,147.34);

\path[draw=drawColor,line width= 0.6pt,line join=round] ( 28.93,147.34) -- (271.45,147.34);

\path[draw=drawColor,line width= 0.6pt,line join=round] ( 28.93,147.34) -- (271.45,147.34);

\path[draw=drawColor,line width= 0.6pt,line join=round] ( 28.93,147.34) -- (271.45,147.34);

\path[draw=drawColor,line width= 0.6pt,line join=round] ( 28.93,147.34) -- (271.45,147.34);

\path[draw=drawColor,line width= 0.6pt,line join=round] ( 28.93,147.34) -- (271.45,147.34);

\path[draw=drawColor,line width= 0.6pt,line join=round] ( 28.93,147.34) -- (271.45,147.34);

\path[draw=drawColor,line width= 0.6pt,line join=round] ( 28.93,147.34) -- (271.45,147.34);

\path[draw=drawColor,line width= 0.6pt,line join=round] ( 28.93,147.34) -- (271.45,147.34);

\path[draw=drawColor,line width= 0.6pt,line join=round] ( 28.93,147.34) -- (271.45,147.34);

\path[draw=drawColor,line width= 0.6pt,line join=round] ( 28.93,147.34) -- (271.45,147.34);

\path[draw=drawColor,line width= 0.6pt,line join=round] ( 28.93,147.34) -- (271.45,147.34);

\path[draw=drawColor,line width= 0.6pt,line join=round] ( 28.93,147.34) -- (271.45,147.34);

\path[draw=drawColor,line width= 0.6pt,line join=round] ( 28.93,147.34) -- (271.45,147.34);

\path[draw=drawColor,line width= 0.6pt,line join=round] ( 28.93,147.34) -- (271.45,147.34);

\path[draw=drawColor,line width= 0.6pt,line join=round] ( 28.93,147.34) -- (271.45,147.34);

\path[draw=drawColor,line width= 0.6pt,line join=round] ( 28.93,147.34) -- (271.45,147.34);

\path[draw=drawColor,line width= 0.6pt,line join=round] ( 28.93,147.34) -- (271.45,147.34);

\path[draw=drawColor,line width= 0.6pt,line join=round] ( 28.93,147.34) -- (271.45,147.34);

\path[draw=drawColor,line width= 0.6pt,line join=round] ( 28.93,147.34) -- (271.45,147.34);

\path[draw=drawColor,line width= 0.6pt,line join=round] ( 28.93,147.34) -- (271.45,147.34);

\path[draw=drawColor,line width= 0.6pt,line join=round] ( 28.93,147.34) -- (271.45,147.34);

\path[draw=drawColor,line width= 0.6pt,line join=round] ( 28.93,147.34) -- (271.45,147.34);

\path[draw=drawColor,line width= 0.6pt,line join=round] ( 28.93,147.34) -- (271.45,147.34);

\path[draw=drawColor,line width= 0.6pt,line join=round] ( 28.93,147.34) -- (271.45,147.34);

\path[draw=drawColor,line width= 0.6pt,line join=round] ( 28.93,147.34) -- (271.45,147.34);

\path[draw=drawColor,line width= 0.6pt,line join=round] ( 28.93,147.34) -- (271.45,147.34);

\path[draw=drawColor,line width= 0.6pt,line join=round] ( 28.93,147.34) -- (271.45,147.34);

\path[draw=drawColor,line width= 0.6pt,line join=round] ( 28.93,147.34) -- (271.45,147.34);

\path[draw=drawColor,line width= 0.6pt,line join=round] ( 28.93,147.34) -- (271.45,147.34);

\path[draw=drawColor,line width= 0.6pt,line join=round] ( 28.93,147.34) -- (271.45,147.34);

\path[draw=drawColor,line width= 0.6pt,line join=round] ( 28.93,147.34) -- (271.45,147.34);

\path[draw=drawColor,line width= 0.6pt,line join=round] ( 28.93,147.34) -- (271.45,147.34);

\path[draw=drawColor,line width= 0.6pt,line join=round] ( 28.93,147.34) -- (271.45,147.34);

\path[draw=drawColor,line width= 0.6pt,line join=round] ( 28.93,147.34) -- (271.45,147.34);

\path[draw=drawColor,line width= 0.6pt,line join=round] ( 28.93,147.34) -- (271.45,147.34);

\path[draw=drawColor,line width= 0.6pt,line join=round] ( 28.93,147.34) -- (271.45,147.34);

\path[draw=drawColor,line width= 0.6pt,line join=round] ( 28.93,147.34) -- (271.45,147.34);

\path[draw=drawColor,line width= 0.6pt,line join=round] ( 28.93,147.34) -- (271.45,147.34);

\path[draw=drawColor,line width= 0.6pt,line join=round] ( 28.93,147.34) -- (271.45,147.34);

\path[draw=drawColor,line width= 0.6pt,line join=round] ( 28.93,147.34) -- (271.45,147.34);

\path[draw=drawColor,line width= 0.6pt,line join=round] ( 28.93,147.34) -- (271.45,147.34);

\path[draw=drawColor,line width= 0.6pt,line join=round] ( 28.93,147.34) -- (271.45,147.34);

\path[draw=drawColor,line width= 0.6pt,line join=round] ( 28.93,147.34) -- (271.45,147.34);

\path[draw=drawColor,line width= 0.6pt,line join=round] ( 28.93,147.34) -- (271.45,147.34);

\path[draw=drawColor,line width= 0.6pt,line join=round] ( 28.93,147.34) -- (271.45,147.34);

\path[draw=drawColor,line width= 0.6pt,line join=round] ( 28.93,147.34) -- (271.45,147.34);

\path[draw=drawColor,line width= 0.6pt,line join=round] ( 28.93,147.34) -- (271.45,147.34);

\path[draw=drawColor,line width= 0.6pt,line join=round] ( 28.93,147.34) -- (271.45,147.34);

\path[draw=drawColor,line width= 0.6pt,line join=round] ( 28.93,147.34) -- (271.45,147.34);

\path[draw=drawColor,line width= 0.6pt,line join=round] ( 28.93,147.34) -- (271.45,147.34);

\path[draw=drawColor,line width= 0.6pt,line join=round] ( 28.93,147.34) -- (271.45,147.34);

\path[draw=drawColor,line width= 0.6pt,line join=round] ( 28.93,147.34) -- (271.45,147.34);

\path[draw=drawColor,line width= 0.6pt,line join=round] ( 28.93,147.34) -- (271.45,147.34);

\path[draw=drawColor,line width= 0.6pt,line join=round] ( 28.93,147.34) -- (271.45,147.34);

\path[draw=drawColor,line width= 0.6pt,line join=round] ( 28.93,147.34) -- (271.45,147.34);

\path[draw=drawColor,line width= 0.6pt,line join=round] ( 28.93,147.34) -- (271.45,147.34);

\path[draw=drawColor,line width= 0.6pt,line join=round] ( 28.93,147.34) -- (271.45,147.34);

\path[draw=drawColor,line width= 0.6pt,line join=round] ( 28.93,147.34) -- (271.45,147.34);

\path[draw=drawColor,line width= 0.6pt,line join=round] ( 28.93,147.34) -- (271.45,147.34);

\path[draw=drawColor,line width= 0.6pt,line join=round] ( 28.93,147.34) -- (271.45,147.34);

\path[draw=drawColor,line width= 0.6pt,line join=round] ( 28.93,147.34) -- (271.45,147.34);

\path[draw=drawColor,line width= 0.6pt,line join=round] ( 28.93,147.34) -- (271.45,147.34);

\path[draw=drawColor,line width= 0.6pt,line join=round] ( 28.93,147.34) -- (271.45,147.34);

\path[draw=drawColor,line width= 0.6pt,line join=round] ( 28.93,147.34) -- (271.45,147.34);

\path[draw=drawColor,line width= 0.6pt,line join=round] ( 28.93,147.34) -- (271.45,147.34);

\path[draw=drawColor,line width= 0.6pt,line join=round] ( 28.93,147.34) -- (271.45,147.34);

\path[draw=drawColor,line width= 0.6pt,line join=round] ( 28.93,147.34) -- (271.45,147.34);

\path[draw=drawColor,line width= 0.6pt,line join=round] ( 28.93,147.34) -- (271.45,147.34);

\path[draw=drawColor,line width= 0.6pt,line join=round] ( 28.93,147.34) -- (271.45,147.34);

\path[draw=drawColor,line width= 0.6pt,line join=round] ( 28.93,147.34) -- (271.45,147.34);

\path[draw=drawColor,line width= 0.6pt,line join=round] ( 28.93,147.34) -- (271.45,147.34);

\path[draw=drawColor,line width= 0.6pt,line join=round] ( 28.93,147.34) -- (271.45,147.34);

\path[draw=drawColor,line width= 0.6pt,line join=round] ( 28.93,147.34) -- (271.45,147.34);

\path[draw=drawColor,line width= 0.6pt,line join=round] ( 28.93,147.34) -- (271.45,147.34);

\path[draw=drawColor,line width= 0.6pt,line join=round] ( 28.93,147.34) -- (271.45,147.34);

\path[draw=drawColor,line width= 0.6pt,line join=round] ( 28.93,147.34) -- (271.45,147.34);

\path[draw=drawColor,line width= 0.6pt,line join=round] ( 28.93,147.34) -- (271.45,147.34);

\path[draw=drawColor,line width= 0.6pt,line join=round] ( 28.93,147.34) -- (271.45,147.34);

\path[draw=drawColor,line width= 0.6pt,line join=round] ( 28.93,147.34) -- (271.45,147.34);

\path[draw=drawColor,line width= 0.6pt,line join=round] ( 28.93,147.34) -- (271.45,147.34);

\path[draw=drawColor,line width= 0.6pt,line join=round] ( 28.93,147.34) -- (271.45,147.34);

\path[draw=drawColor,line width= 0.6pt,line join=round] ( 28.93,147.34) -- (271.45,147.34);

\path[draw=drawColor,line width= 0.6pt,line join=round] ( 28.93,147.34) -- (271.45,147.34);

\path[draw=drawColor,line width= 0.6pt,line join=round] ( 28.93,147.34) -- (271.45,147.34);

\path[draw=drawColor,line width= 0.6pt,line join=round] ( 28.93,147.34) -- (271.45,147.34);

\path[draw=drawColor,line width= 0.6pt,line join=round] ( 28.93,147.34) -- (271.45,147.34);

\path[draw=drawColor,line width= 0.6pt,line join=round] ( 28.93,147.34) -- (271.45,147.34);

\path[draw=drawColor,line width= 0.6pt,line join=round] ( 28.93,147.34) -- (271.45,147.34);

\path[draw=drawColor,line width= 0.6pt,line join=round] ( 28.93,147.34) -- (271.45,147.34);

\path[draw=drawColor,line width= 0.6pt,line join=round] ( 28.93,147.34) -- (271.45,147.34);

\path[draw=drawColor,line width= 0.6pt,line join=round] ( 28.93,147.34) -- (271.45,147.34);

\path[draw=drawColor,line width= 0.6pt,line join=round] ( 28.93,147.34) -- (271.45,147.34);

\path[draw=drawColor,line width= 0.6pt,line join=round] ( 28.93,147.34) -- (271.45,147.34);

\path[draw=drawColor,line width= 0.6pt,line join=round] ( 28.93,147.34) -- (271.45,147.34);

\path[draw=drawColor,line width= 0.6pt,line join=round] ( 28.93,147.34) -- (271.45,147.34);

\path[draw=drawColor,line width= 0.6pt,line join=round] ( 28.93,147.34) -- (271.45,147.34);

\path[draw=drawColor,line width= 0.6pt,line join=round] ( 28.93,147.34) -- (271.45,147.34);

\path[draw=drawColor,line width= 0.6pt,line join=round] ( 28.93,147.34) -- (271.45,147.34);

\path[draw=drawColor,line width= 0.6pt,line join=round] ( 28.93,147.34) -- (271.45,147.34);

\path[draw=drawColor,line width= 0.6pt,line join=round] ( 28.93,147.34) -- (271.45,147.34);

\path[draw=drawColor,line width= 0.6pt,line join=round] ( 28.93,147.34) -- (271.45,147.34);

\path[draw=drawColor,line width= 0.6pt,line join=round] ( 28.93,147.34) -- (271.45,147.34);

\path[draw=drawColor,line width= 0.6pt,line join=round] ( 28.93,147.34) -- (271.45,147.34);

\path[draw=drawColor,line width= 0.6pt,line join=round] ( 28.93,147.34) -- (271.45,147.34);

\path[draw=drawColor,line width= 0.6pt,line join=round] ( 28.93,147.34) -- (271.45,147.34);

\path[draw=drawColor,line width= 0.6pt,line join=round] ( 28.93,147.34) -- (271.45,147.34);

\path[draw=drawColor,line width= 0.6pt,line join=round] ( 28.93,147.34) -- (271.45,147.34);

\path[draw=drawColor,line width= 0.6pt,line join=round] ( 28.93,147.34) -- (271.45,147.34);

\path[draw=drawColor,line width= 0.6pt,line join=round] ( 28.93,147.34) -- (271.45,147.34);

\path[draw=drawColor,line width= 0.6pt,line join=round] ( 28.93,147.34) -- (271.45,147.34);

\path[draw=drawColor,line width= 0.6pt,line join=round] ( 28.93,147.34) -- (271.45,147.34);

\path[draw=drawColor,line width= 0.6pt,line join=round] ( 28.93,147.34) -- (271.45,147.34);

\path[draw=drawColor,line width= 0.6pt,line join=round] ( 28.93,147.34) -- (271.45,147.34);

\path[draw=drawColor,line width= 0.6pt,line join=round] ( 28.93,147.34) -- (271.45,147.34);

\path[draw=drawColor,line width= 0.6pt,line join=round] ( 28.93,147.34) -- (271.45,147.34);

\path[draw=drawColor,line width= 0.6pt,line join=round] ( 28.93,147.34) -- (271.45,147.34);

\path[draw=drawColor,line width= 0.6pt,line join=round] ( 28.93,147.34) -- (271.45,147.34);

\path[draw=drawColor,line width= 0.6pt,line join=round] ( 28.93,147.34) -- (271.45,147.34);

\path[draw=drawColor,line width= 0.6pt,line join=round] ( 28.93,147.34) -- (271.45,147.34);

\path[draw=drawColor,line width= 0.6pt,line join=round] ( 28.93,147.34) -- (271.45,147.34);

\path[draw=drawColor,line width= 0.6pt,line join=round] ( 28.93,147.34) -- (271.45,147.34);

\path[draw=drawColor,line width= 0.6pt,line join=round] ( 28.93,147.34) -- (271.45,147.34);

\path[draw=drawColor,line width= 0.6pt,line join=round] ( 28.93,147.34) -- (271.45,147.34);

\path[draw=drawColor,line width= 0.6pt,line join=round] ( 28.93,147.34) -- (271.45,147.34);

\path[draw=drawColor,line width= 0.6pt,line join=round] ( 28.93,147.34) -- (271.45,147.34);

\path[draw=drawColor,line width= 0.6pt,line join=round] ( 28.93,147.34) -- (271.45,147.34);

\path[draw=drawColor,line width= 0.6pt,line join=round] ( 28.93,147.34) -- (271.45,147.34);

\path[draw=drawColor,line width= 0.6pt,line join=round] ( 28.93,147.34) -- (271.45,147.34);

\path[draw=drawColor,line width= 0.6pt,line join=round] ( 28.93,147.34) -- (271.45,147.34);

\path[draw=drawColor,line width= 0.6pt,line join=round] ( 28.93,147.34) -- (271.45,147.34);

\path[draw=drawColor,line width= 0.6pt,line join=round] ( 28.93,147.34) -- (271.45,147.34);

\path[draw=drawColor,line width= 0.6pt,line join=round] ( 28.93,147.34) -- (271.45,147.34);

\path[draw=drawColor,line width= 0.6pt,line join=round] ( 28.93,147.34) -- (271.45,147.34);

\path[draw=drawColor,line width= 0.6pt,line join=round] ( 28.93,147.34) -- (271.45,147.34);

\path[draw=drawColor,line width= 0.6pt,line join=round] ( 28.93,147.34) -- (271.45,147.34);

\path[draw=drawColor,line width= 0.6pt,line join=round] ( 28.93,147.34) -- (271.45,147.34);

\path[draw=drawColor,line width= 0.6pt,line join=round] ( 28.93,147.34) -- (271.45,147.34);

\path[draw=drawColor,line width= 0.6pt,line join=round] ( 28.93,147.34) -- (271.45,147.34);

\path[draw=drawColor,line width= 0.6pt,line join=round] ( 28.93,147.34) -- (271.45,147.34);

\path[draw=drawColor,line width= 0.6pt,line join=round] ( 28.93,147.34) -- (271.45,147.34);

\path[draw=drawColor,line width= 0.6pt,line join=round] ( 28.93,147.34) -- (271.45,147.34);

\path[draw=drawColor,line width= 0.6pt,line join=round] ( 28.93,147.34) -- (271.45,147.34);

\path[draw=drawColor,line width= 0.6pt,line join=round] ( 28.93,147.34) -- (271.45,147.34);

\path[draw=drawColor,line width= 0.6pt,line join=round] ( 28.93,147.34) -- (271.45,147.34);

\path[draw=drawColor,line width= 0.6pt,line join=round] ( 28.93,147.34) -- (271.45,147.34);

\path[draw=drawColor,line width= 0.6pt,line join=round] ( 28.93,147.34) -- (271.45,147.34);

\path[draw=drawColor,line width= 0.6pt,line join=round] ( 28.93,147.34) -- (271.45,147.34);

\path[draw=drawColor,line width= 0.6pt,line join=round] ( 28.93,147.34) -- (271.45,147.34);

\path[draw=drawColor,line width= 0.6pt,line join=round] ( 28.93,147.34) -- (271.45,147.34);

\path[draw=drawColor,line width= 0.6pt,line join=round] ( 28.93,147.34) -- (271.45,147.34);

\path[draw=drawColor,line width= 0.6pt,line join=round] ( 28.93,147.34) -- (271.45,147.34);

\path[draw=drawColor,line width= 0.6pt,line join=round] ( 28.93,147.34) -- (271.45,147.34);

\path[draw=drawColor,line width= 0.6pt,line join=round] ( 28.93,147.34) -- (271.45,147.34);

\path[draw=drawColor,line width= 0.6pt,line join=round] ( 28.93,147.34) -- (271.45,147.34);

\path[draw=drawColor,line width= 0.6pt,line join=round] ( 28.93,147.34) -- (271.45,147.34);

\path[draw=drawColor,line width= 0.6pt,line join=round] ( 28.93,147.34) -- (271.45,147.34);

\path[draw=drawColor,line width= 0.6pt,line join=round] ( 28.93,147.34) -- (271.45,147.34);

\path[draw=drawColor,line width= 0.6pt,line join=round] ( 28.93,147.34) -- (271.45,147.34);

\path[draw=drawColor,line width= 0.6pt,line join=round] ( 28.93,147.34) -- (271.45,147.34);

\path[draw=drawColor,line width= 0.6pt,line join=round] ( 28.93,147.34) -- (271.45,147.34);

\path[draw=drawColor,line width= 0.6pt,line join=round] ( 28.93,147.34) -- (271.45,147.34);

\path[draw=drawColor,line width= 0.6pt,line join=round] ( 28.93,147.34) -- (271.45,147.34);

\path[draw=drawColor,line width= 0.6pt,line join=round] ( 28.93,147.34) -- (271.45,147.34);

\path[draw=drawColor,line width= 0.6pt,line join=round] ( 28.93,147.34) -- (271.45,147.34);

\path[draw=drawColor,line width= 0.6pt,line join=round] ( 28.93,147.34) -- (271.45,147.34);

\path[draw=drawColor,line width= 0.6pt,line join=round] ( 28.93,147.34) -- (271.45,147.34);

\path[draw=drawColor,line width= 0.6pt,line join=round] ( 28.93,147.34) -- (271.45,147.34);

\path[draw=drawColor,line width= 0.6pt,line join=round] ( 28.93,147.34) -- (271.45,147.34);

\path[draw=drawColor,line width= 0.6pt,line join=round] ( 28.93,147.34) -- (271.45,147.34);

\path[draw=drawColor,line width= 0.6pt,line join=round] ( 28.93,147.34) -- (271.45,147.34);

\path[draw=drawColor,line width= 0.6pt,line join=round] ( 28.93,147.34) -- (271.45,147.34);

\path[draw=drawColor,line width= 0.6pt,line join=round] ( 28.93,147.34) -- (271.45,147.34);

\path[draw=drawColor,line width= 0.6pt,line join=round] ( 28.93,147.34) -- (271.45,147.34);

\path[draw=drawColor,line width= 0.6pt,line join=round] ( 28.93,147.34) -- (271.45,147.34);

\path[draw=drawColor,line width= 0.6pt,line join=round] ( 28.93,147.34) -- (271.45,147.34);

\path[draw=drawColor,line width= 0.6pt,line join=round] ( 28.93,147.34) -- (271.45,147.34);

\path[draw=drawColor,line width= 0.6pt,line join=round] ( 28.93,147.34) -- (271.45,147.34);

\path[draw=drawColor,line width= 0.6pt,line join=round] ( 28.93,147.34) -- (271.45,147.34);

\path[draw=drawColor,line width= 0.6pt,line join=round] ( 28.93,147.34) -- (271.45,147.34);

\path[draw=drawColor,line width= 0.6pt,line join=round] ( 28.93,147.34) -- (271.45,147.34);

\path[draw=drawColor,line width= 0.6pt,line join=round] ( 28.93,147.34) -- (271.45,147.34);

\path[draw=drawColor,line width= 0.6pt,line join=round] ( 28.93,147.34) -- (271.45,147.34);

\path[draw=drawColor,line width= 0.6pt,line join=round] ( 28.93,147.34) -- (271.45,147.34);

\path[draw=drawColor,line width= 0.6pt,line join=round] ( 28.93,147.34) -- (271.45,147.34);

\path[draw=drawColor,line width= 0.6pt,line join=round] ( 28.93,147.34) -- (271.45,147.34);

\path[draw=drawColor,line width= 0.6pt,line join=round] ( 28.93,147.34) -- (271.45,147.34);

\path[draw=drawColor,line width= 0.6pt,line join=round] ( 28.93,147.34) -- (271.45,147.34);

\path[draw=drawColor,line width= 0.6pt,line join=round] ( 28.93,147.34) -- (271.45,147.34);

\path[draw=drawColor,line width= 0.6pt,line join=round] ( 28.93,147.34) -- (271.45,147.34);

\path[draw=drawColor,line width= 0.6pt,line join=round] ( 28.93,147.34) -- (271.45,147.34);

\path[draw=drawColor,line width= 0.6pt,line join=round] ( 28.93,147.34) -- (271.45,147.34);

\path[draw=drawColor,line width= 0.6pt,line join=round] ( 28.93,147.34) -- (271.45,147.34);

\path[draw=drawColor,line width= 0.6pt,line join=round] ( 28.93,147.34) -- (271.45,147.34);

\path[draw=drawColor,line width= 0.6pt,line join=round] ( 28.93,147.34) -- (271.45,147.34);

\path[draw=drawColor,line width= 0.6pt,line join=round] ( 28.93,147.34) -- (271.45,147.34);

\path[draw=drawColor,line width= 0.6pt,line join=round] ( 28.93,147.34) -- (271.45,147.34);

\path[draw=drawColor,line width= 0.6pt,line join=round] ( 28.93,147.34) -- (271.45,147.34);

\path[draw=drawColor,line width= 0.6pt,line join=round] ( 28.93,147.34) -- (271.45,147.34);

\path[draw=drawColor,line width= 0.6pt,line join=round] ( 28.93,147.34) -- (271.45,147.34);

\path[draw=drawColor,line width= 0.6pt,line join=round] ( 28.93,147.34) -- (271.45,147.34);

\path[draw=drawColor,line width= 0.6pt,line join=round] ( 28.93,147.34) -- (271.45,147.34);

\path[draw=drawColor,line width= 0.6pt,line join=round] ( 28.93,147.34) -- (271.45,147.34);

\path[draw=drawColor,line width= 0.6pt,line join=round] ( 28.93,147.34) -- (271.45,147.34);

\path[draw=drawColor,line width= 0.6pt,line join=round] ( 28.93,147.34) -- (271.45,147.34);

\path[draw=drawColor,line width= 0.6pt,line join=round] ( 28.93,147.34) -- (271.45,147.34);

\path[draw=drawColor,line width= 0.6pt,line join=round] ( 28.93,147.34) -- (271.45,147.34);

\path[draw=drawColor,line width= 0.6pt,line join=round] ( 28.93,147.34) -- (271.45,147.34);

\path[draw=drawColor,line width= 0.6pt,line join=round] ( 28.93,147.34) -- (271.45,147.34);

\path[draw=drawColor,line width= 0.6pt,line join=round] ( 28.93,147.34) -- (271.45,147.34);

\path[draw=drawColor,line width= 0.6pt,line join=round] ( 28.93,147.34) -- (271.45,147.34);

\path[draw=drawColor,line width= 0.6pt,line join=round] ( 28.93,147.34) -- (271.45,147.34);

\path[draw=drawColor,line width= 0.6pt,line join=round] ( 28.93,147.34) -- (271.45,147.34);

\path[draw=drawColor,line width= 0.6pt,line join=round] ( 28.93,147.34) -- (271.45,147.34);

\path[draw=drawColor,line width= 0.6pt,line join=round] ( 28.93,147.34) -- (271.45,147.34);

\path[draw=drawColor,line width= 0.6pt,line join=round] ( 28.93,147.34) -- (271.45,147.34);

\path[draw=drawColor,line width= 0.6pt,line join=round] ( 28.93,147.34) -- (271.45,147.34);

\path[draw=drawColor,line width= 0.6pt,line join=round] ( 28.93,147.34) -- (271.45,147.34);

\path[draw=drawColor,line width= 0.6pt,line join=round] ( 28.93,147.34) -- (271.45,147.34);

\path[draw=drawColor,line width= 0.6pt,line join=round] ( 28.93,147.34) -- (271.45,147.34);

\path[draw=drawColor,line width= 0.6pt,line join=round] ( 28.93,147.34) -- (271.45,147.34);

\path[draw=drawColor,line width= 0.6pt,line join=round] ( 28.93,147.34) -- (271.45,147.34);

\path[draw=drawColor,line width= 0.6pt,line join=round] ( 28.93,147.34) -- (271.45,147.34);

\path[draw=drawColor,line width= 0.6pt,line join=round] ( 28.93,147.34) -- (271.45,147.34);

\path[draw=drawColor,line width= 0.6pt,line join=round] ( 28.93,147.34) -- (271.45,147.34);

\path[draw=drawColor,line width= 0.6pt,line join=round] ( 28.93,147.34) -- (271.45,147.34);

\path[draw=drawColor,line width= 0.6pt,line join=round] ( 28.93,147.34) -- (271.45,147.34);

\path[draw=drawColor,line width= 0.6pt,line join=round] ( 28.93,147.34) -- (271.45,147.34);

\path[draw=drawColor,line width= 0.6pt,line join=round] ( 28.93,147.34) -- (271.45,147.34);

\path[draw=drawColor,line width= 0.6pt,line join=round] ( 28.93,147.34) -- (271.45,147.34);

\path[draw=drawColor,line width= 0.6pt,line join=round] ( 28.93,147.34) -- (271.45,147.34);

\path[draw=drawColor,line width= 0.6pt,line join=round] ( 28.93,147.34) -- (271.45,147.34);

\path[draw=drawColor,line width= 0.6pt,line join=round] ( 28.93,147.34) -- (271.45,147.34);

\path[draw=drawColor,line width= 0.6pt,line join=round] ( 28.93,147.34) -- (271.45,147.34);

\path[draw=drawColor,line width= 0.6pt,line join=round] ( 28.93,147.34) -- (271.45,147.34);

\path[draw=drawColor,line width= 0.6pt,line join=round] ( 28.93,147.34) -- (271.45,147.34);

\path[draw=drawColor,line width= 0.6pt,line join=round] ( 28.93,147.34) -- (271.45,147.34);

\path[draw=drawColor,line width= 0.6pt,line join=round] ( 28.93,147.34) -- (271.45,147.34);

\path[draw=drawColor,line width= 0.6pt,line join=round] ( 28.93,147.34) -- (271.45,147.34);

\path[draw=drawColor,line width= 0.6pt,line join=round] ( 28.93,147.34) -- (271.45,147.34);

\path[draw=drawColor,line width= 0.6pt,line join=round] ( 28.93,147.34) -- (271.45,147.34);

\path[draw=drawColor,line width= 0.6pt,line join=round] ( 28.93,147.34) -- (271.45,147.34);

\path[draw=drawColor,line width= 0.6pt,line join=round] ( 28.93,147.34) -- (271.45,147.34);

\path[draw=drawColor,line width= 0.6pt,line join=round] ( 28.93,147.34) -- (271.45,147.34);

\path[draw=drawColor,line width= 0.6pt,line join=round] ( 28.93,147.34) -- (271.45,147.34);

\path[draw=drawColor,line width= 0.6pt,line join=round] ( 28.93,147.34) -- (271.45,147.34);

\path[draw=drawColor,line width= 0.6pt,line join=round] ( 28.93,147.34) -- (271.45,147.34);

\path[draw=drawColor,line width= 0.6pt,line join=round] ( 28.93,147.34) -- (271.45,147.34);

\path[draw=drawColor,line width= 0.6pt,line join=round] ( 28.93,147.34) -- (271.45,147.34);

\path[draw=drawColor,line width= 0.6pt,line join=round] ( 28.93,147.34) -- (271.45,147.34);

\path[draw=drawColor,line width= 0.6pt,line join=round] ( 28.93,147.34) -- (271.45,147.34);

\path[draw=drawColor,line width= 0.6pt,line join=round] ( 28.93,147.34) -- (271.45,147.34);

\path[draw=drawColor,line width= 0.6pt,line join=round] ( 28.93,147.34) -- (271.45,147.34);

\path[draw=drawColor,line width= 0.6pt,line join=round] ( 28.93,147.34) -- (271.45,147.34);

\path[draw=drawColor,line width= 0.6pt,line join=round] ( 28.93,147.34) -- (271.45,147.34);

\path[draw=drawColor,line width= 0.6pt,line join=round] ( 28.93,147.34) -- (271.45,147.34);

\path[draw=drawColor,line width= 0.6pt,line join=round] ( 28.93,147.34) -- (271.45,147.34);

\path[draw=drawColor,line width= 0.6pt,line join=round] ( 28.93,147.34) -- (271.45,147.34);

\path[draw=drawColor,line width= 0.6pt,line join=round] ( 28.93,147.34) -- (271.45,147.34);

\path[draw=drawColor,line width= 0.6pt,line join=round] ( 28.93,147.34) -- (271.45,147.34);

\path[draw=drawColor,line width= 0.6pt,line join=round] ( 28.93,147.34) -- (271.45,147.34);

\path[draw=drawColor,line width= 0.6pt,line join=round] ( 28.93,147.34) -- (271.45,147.34);

\path[draw=drawColor,line width= 0.6pt,line join=round] ( 28.93,147.34) -- (271.45,147.34);

\path[draw=drawColor,line width= 0.6pt,line join=round] ( 28.93,147.34) -- (271.45,147.34);

\path[draw=drawColor,line width= 0.6pt,line join=round] ( 28.93,147.34) -- (271.45,147.34);

\path[draw=drawColor,line width= 0.6pt,line join=round] ( 28.93,147.34) -- (271.45,147.34);

\path[draw=drawColor,line width= 0.6pt,line join=round] ( 28.93,147.34) -- (271.45,147.34);

\path[draw=drawColor,line width= 0.6pt,line join=round] ( 28.93,147.34) -- (271.45,147.34);

\path[draw=drawColor,line width= 0.6pt,line join=round] ( 28.93,147.34) -- (271.45,147.34);

\path[draw=drawColor,line width= 0.6pt,line join=round] ( 28.93,147.34) -- (271.45,147.34);

\path[draw=drawColor,line width= 0.6pt,line join=round] ( 28.93,147.34) -- (271.45,147.34);

\path[draw=drawColor,line width= 0.6pt,line join=round] ( 28.93,147.34) -- (271.45,147.34);

\path[draw=drawColor,line width= 0.6pt,line join=round] ( 28.93,147.34) -- (271.45,147.34);

\path[draw=drawColor,line width= 0.6pt,line join=round] ( 28.93,147.34) -- (271.45,147.34);

\path[draw=drawColor,line width= 0.6pt,line join=round] ( 28.93,147.34) -- (271.45,147.34);

\path[draw=drawColor,line width= 0.6pt,line join=round] ( 28.93,147.34) -- (271.45,147.34);

\path[draw=drawColor,line width= 0.6pt,line join=round] ( 28.93,147.34) -- (271.45,147.34);

\path[draw=drawColor,line width= 0.6pt,line join=round] ( 28.93,147.34) -- (271.45,147.34);

\path[draw=drawColor,line width= 0.6pt,line join=round] ( 28.93,147.34) -- (271.45,147.34);

\path[draw=drawColor,line width= 0.6pt,line join=round] ( 28.93,147.34) -- (271.45,147.34);

\path[draw=drawColor,line width= 0.6pt,line join=round] ( 28.93,147.34) -- (271.45,147.34);

\path[draw=drawColor,line width= 0.6pt,line join=round] ( 28.93,147.34) -- (271.45,147.34);

\path[draw=drawColor,line width= 0.6pt,line join=round] ( 28.93,147.34) -- (271.45,147.34);

\path[draw=drawColor,line width= 0.6pt,line join=round] ( 28.93,147.34) -- (271.45,147.34);

\path[draw=drawColor,line width= 0.6pt,line join=round] ( 28.93,147.34) -- (271.45,147.34);

\path[draw=drawColor,line width= 0.6pt,line join=round] ( 28.93,147.34) -- (271.45,147.34);

\path[draw=drawColor,line width= 0.6pt,line join=round] ( 28.93,147.34) -- (271.45,147.34);

\path[draw=drawColor,line width= 0.6pt,line join=round] ( 28.93,147.34) -- (271.45,147.34);

\path[draw=drawColor,line width= 0.6pt,line join=round] ( 28.93,147.34) -- (271.45,147.34);

\path[draw=drawColor,line width= 0.6pt,line join=round] ( 28.93,147.34) -- (271.45,147.34);

\path[draw=drawColor,line width= 0.6pt,line join=round] ( 28.93,147.34) -- (271.45,147.34);

\path[draw=drawColor,line width= 0.6pt,line join=round] ( 28.93,147.34) -- (271.45,147.34);

\path[draw=drawColor,line width= 0.6pt,line join=round] ( 28.93,147.34) -- (271.45,147.34);

\path[draw=drawColor,line width= 0.6pt,line join=round] ( 28.93,147.34) -- (271.45,147.34);

\path[draw=drawColor,line width= 0.6pt,line join=round] ( 28.93,147.34) -- (271.45,147.34);

\path[draw=drawColor,line width= 0.6pt,line join=round] ( 28.93,147.34) -- (271.45,147.34);

\path[draw=drawColor,line width= 0.6pt,line join=round] ( 28.93,147.34) -- (271.45,147.34);

\path[draw=drawColor,line width= 0.6pt,line join=round] ( 28.93,147.34) -- (271.45,147.34);

\path[draw=drawColor,line width= 0.6pt,line join=round] ( 28.93,147.34) -- (271.45,147.34);

\path[draw=drawColor,line width= 0.6pt,line join=round] ( 28.93,147.34) -- (271.45,147.34);

\path[draw=drawColor,line width= 0.6pt,line join=round] ( 28.93,147.34) -- (271.45,147.34);

\path[draw=drawColor,line width= 0.6pt,line join=round] ( 28.93,147.34) -- (271.45,147.34);

\path[draw=drawColor,line width= 0.6pt,line join=round] ( 28.93,147.34) -- (271.45,147.34);

\path[draw=drawColor,line width= 0.6pt,line join=round] ( 28.93,147.34) -- (271.45,147.34);

\path[draw=drawColor,line width= 0.6pt,line join=round] ( 28.93,147.34) -- (271.45,147.34);

\path[draw=drawColor,line width= 0.6pt,line join=round] ( 28.93,147.34) -- (271.45,147.34);

\path[draw=drawColor,line width= 0.6pt,line join=round] ( 28.93,147.34) -- (271.45,147.34);

\path[draw=drawColor,line width= 0.6pt,line join=round] ( 28.93,147.34) -- (271.45,147.34);

\path[draw=drawColor,line width= 0.6pt,line join=round] ( 28.93,147.34) -- (271.45,147.34);

\path[draw=drawColor,line width= 0.6pt,line join=round] ( 28.93,147.34) -- (271.45,147.34);

\path[draw=drawColor,line width= 0.6pt,line join=round] ( 28.93,147.34) -- (271.45,147.34);

\path[draw=drawColor,line width= 0.6pt,line join=round] ( 28.93,147.34) -- (271.45,147.34);

\path[draw=drawColor,line width= 0.6pt,line join=round] ( 28.93,147.34) -- (271.45,147.34);

\path[draw=drawColor,line width= 0.6pt,line join=round] ( 28.93,147.34) -- (271.45,147.34);

\path[draw=drawColor,line width= 0.6pt,line join=round] ( 28.93,147.34) -- (271.45,147.34);

\path[draw=drawColor,line width= 0.6pt,line join=round] ( 28.93,147.34) -- (271.45,147.34);

\path[draw=drawColor,line width= 0.6pt,line join=round] ( 28.93,147.34) -- (271.45,147.34);

\path[draw=drawColor,line width= 0.6pt,line join=round] ( 28.93,147.34) -- (271.45,147.34);

\path[draw=drawColor,line width= 0.6pt,line join=round] ( 28.93,147.34) -- (271.45,147.34);

\path[draw=drawColor,line width= 0.6pt,line join=round] ( 28.93,147.34) -- (271.45,147.34);

\path[draw=drawColor,line width= 0.6pt,line join=round] ( 28.93,147.34) -- (271.45,147.34);

\path[draw=drawColor,line width= 0.6pt,line join=round] ( 28.93,147.34) -- (271.45,147.34);

\path[draw=drawColor,line width= 0.6pt,line join=round] ( 28.93,147.34) -- (271.45,147.34);

\path[draw=drawColor,line width= 0.6pt,line join=round] ( 28.93,147.34) -- (271.45,147.34);

\path[draw=drawColor,line width= 0.6pt,line join=round] ( 28.93,147.34) -- (271.45,147.34);

\path[draw=drawColor,line width= 0.6pt,line join=round] ( 28.93,147.34) -- (271.45,147.34);

\path[draw=drawColor,line width= 0.6pt,line join=round] ( 28.93,147.34) -- (271.45,147.34);

\path[draw=drawColor,line width= 0.6pt,line join=round] ( 28.93,147.34) -- (271.45,147.34);

\path[draw=drawColor,line width= 0.6pt,line join=round] ( 28.93,147.34) -- (271.45,147.34);

\path[draw=drawColor,line width= 0.6pt,line join=round] ( 28.93,147.34) -- (271.45,147.34);

\path[draw=drawColor,line width= 0.6pt,line join=round] ( 28.93,147.34) -- (271.45,147.34);

\path[draw=drawColor,line width= 0.6pt,line join=round] ( 28.93,147.34) -- (271.45,147.34);

\path[draw=drawColor,line width= 0.6pt,line join=round] ( 28.93,147.34) -- (271.45,147.34);

\path[draw=drawColor,line width= 0.6pt,line join=round] ( 28.93,147.34) -- (271.45,147.34);

\path[draw=drawColor,line width= 0.6pt,line join=round] ( 28.93,147.34) -- (271.45,147.34);

\path[draw=drawColor,line width= 0.6pt,line join=round] ( 28.93,147.34) -- (271.45,147.34);

\path[draw=drawColor,line width= 0.6pt,line join=round] ( 28.93,147.34) -- (271.45,147.34);

\path[draw=drawColor,line width= 0.6pt,line join=round] ( 28.93,147.34) -- (271.45,147.34);

\path[draw=drawColor,line width= 0.6pt,line join=round] ( 28.93,147.34) -- (271.45,147.34);

\path[draw=drawColor,line width= 0.6pt,line join=round] ( 28.93,147.34) -- (271.45,147.34);

\path[draw=drawColor,line width= 0.6pt,line join=round] ( 28.93,147.34) -- (271.45,147.34);

\path[draw=drawColor,line width= 0.6pt,line join=round] ( 28.93,147.34) -- (271.45,147.34);

\path[draw=drawColor,line width= 0.6pt,line join=round] ( 28.93,147.34) -- (271.45,147.34);

\path[draw=drawColor,line width= 0.6pt,line join=round] ( 28.93,147.34) -- (271.45,147.34);

\path[draw=drawColor,line width= 0.6pt,line join=round] ( 28.93,147.34) -- (271.45,147.34);

\path[draw=drawColor,line width= 0.6pt,line join=round] ( 28.93,147.34) -- (271.45,147.34);

\path[draw=drawColor,line width= 0.6pt,line join=round] ( 28.93,147.34) -- (271.45,147.34);

\path[draw=drawColor,line width= 0.6pt,line join=round] ( 28.93,147.34) -- (271.45,147.34);

\path[draw=drawColor,line width= 0.6pt,line join=round] ( 28.93,147.34) -- (271.45,147.34);

\path[draw=drawColor,line width= 0.6pt,line join=round] ( 28.93,147.34) -- (271.45,147.34);

\path[draw=drawColor,line width= 0.6pt,line join=round] ( 28.93,147.34) -- (271.45,147.34);

\path[draw=drawColor,line width= 0.6pt,line join=round] ( 28.93,147.34) -- (271.45,147.34);

\path[draw=drawColor,line width= 0.6pt,line join=round] ( 28.93,147.34) -- (271.45,147.34);

\path[draw=drawColor,line width= 0.6pt,line join=round] ( 28.93,147.34) -- (271.45,147.34);

\path[draw=drawColor,line width= 0.6pt,line join=round] ( 28.93,147.34) -- (271.45,147.34);

\path[draw=drawColor,line width= 0.6pt,line join=round] ( 28.93,147.34) -- (271.45,147.34);

\path[draw=drawColor,line width= 0.6pt,line join=round] ( 28.93,147.34) -- (271.45,147.34);

\path[draw=drawColor,line width= 0.6pt,line join=round] ( 28.93,147.34) -- (271.45,147.34);

\path[draw=drawColor,line width= 0.6pt,line join=round] ( 28.93,147.34) -- (271.45,147.34);

\path[draw=drawColor,line width= 0.6pt,line join=round] ( 28.93,147.34) -- (271.45,147.34);

\path[draw=drawColor,line width= 0.6pt,line join=round] ( 28.93,147.34) -- (271.45,147.34);

\path[draw=drawColor,line width= 0.6pt,line join=round] ( 28.93,147.34) -- (271.45,147.34);

\path[draw=drawColor,line width= 0.6pt,line join=round] ( 28.93,147.34) -- (271.45,147.34);

\path[draw=drawColor,line width= 0.6pt,line join=round] ( 28.93,147.34) -- (271.45,147.34);

\path[draw=drawColor,line width= 0.6pt,line join=round] ( 28.93,147.34) -- (271.45,147.34);

\path[draw=drawColor,line width= 0.6pt,line join=round] ( 28.93,147.34) -- (271.45,147.34);

\path[draw=drawColor,line width= 0.6pt,line join=round] ( 28.93,147.34) -- (271.45,147.34);

\path[draw=drawColor,line width= 0.6pt,line join=round] ( 28.93,147.34) -- (271.45,147.34);

\path[draw=drawColor,line width= 0.6pt,line join=round] ( 28.93,147.34) -- (271.45,147.34);

\path[draw=drawColor,line width= 0.6pt,line join=round] ( 28.93,147.34) -- (271.45,147.34);

\path[draw=drawColor,line width= 0.6pt,line join=round] ( 28.93,147.34) -- (271.45,147.34);

\path[draw=drawColor,line width= 0.6pt,line join=round] ( 28.93,147.34) -- (271.45,147.34);

\path[draw=drawColor,line width= 0.6pt,line join=round] ( 28.93,147.34) -- (271.45,147.34);

\path[draw=drawColor,line width= 0.6pt,line join=round] ( 28.93,147.34) -- (271.45,147.34);

\path[draw=drawColor,line width= 0.6pt,line join=round] ( 28.93,147.34) -- (271.45,147.34);

\path[draw=drawColor,line width= 0.6pt,line join=round] ( 28.93,147.34) -- (271.45,147.34);

\path[draw=drawColor,line width= 0.6pt,line join=round] ( 28.93,147.34) -- (271.45,147.34);

\path[draw=drawColor,line width= 0.6pt,line join=round] ( 28.93,147.34) -- (271.45,147.34);

\path[draw=drawColor,line width= 0.6pt,line join=round] ( 28.93,147.34) -- (271.45,147.34);

\path[draw=drawColor,line width= 0.6pt,line join=round] ( 28.93,147.34) -- (271.45,147.34);

\path[draw=drawColor,line width= 0.6pt,line join=round] ( 28.93,147.34) -- (271.45,147.34);

\path[draw=drawColor,line width= 0.6pt,line join=round] ( 28.93,147.34) -- (271.45,147.34);

\path[draw=drawColor,line width= 0.6pt,line join=round] ( 28.93,147.34) -- (271.45,147.34);

\path[draw=drawColor,line width= 0.6pt,line join=round] ( 28.93,147.34) -- (271.45,147.34);

\path[draw=drawColor,line width= 0.6pt,line join=round] ( 28.93,147.34) -- (271.45,147.34);

\path[draw=drawColor,line width= 0.6pt,line join=round] ( 28.93,147.34) -- (271.45,147.34);

\path[draw=drawColor,line width= 0.6pt,line join=round] ( 28.93,147.34) -- (271.45,147.34);

\path[draw=drawColor,line width= 0.6pt,line join=round] ( 28.93,147.34) -- (271.45,147.34);

\path[draw=drawColor,line width= 0.6pt,line join=round] ( 28.93,147.34) -- (271.45,147.34);

\path[draw=drawColor,line width= 0.6pt,line join=round] ( 28.93,147.34) -- (271.45,147.34);

\path[draw=drawColor,line width= 0.6pt,line join=round] ( 28.93,147.34) -- (271.45,147.34);

\path[draw=drawColor,line width= 0.6pt,line join=round] ( 28.93,147.34) -- (271.45,147.34);

\path[draw=drawColor,line width= 0.6pt,line join=round] ( 28.93,147.34) -- (271.45,147.34);

\path[draw=drawColor,line width= 0.6pt,line join=round] ( 28.93,147.34) -- (271.45,147.34);

\path[draw=drawColor,line width= 0.6pt,line join=round] ( 28.93,147.34) -- (271.45,147.34);

\path[draw=drawColor,line width= 0.6pt,line join=round] ( 28.93,147.34) -- (271.45,147.34);

\path[draw=drawColor,line width= 0.6pt,line join=round] ( 28.93,147.34) -- (271.45,147.34);

\path[draw=drawColor,line width= 0.6pt,line join=round] ( 28.93,147.34) -- (271.45,147.34);

\path[draw=drawColor,line width= 0.6pt,line join=round] ( 28.93,147.34) -- (271.45,147.34);

\path[draw=drawColor,line width= 0.6pt,line join=round] ( 28.93,147.34) -- (271.45,147.34);

\path[draw=drawColor,line width= 0.6pt,line join=round] ( 28.93,147.34) -- (271.45,147.34);

\path[draw=drawColor,line width= 0.6pt,line join=round] ( 28.93,147.34) -- (271.45,147.34);

\path[draw=drawColor,line width= 0.6pt,line join=round] ( 28.93,147.34) -- (271.45,147.34);

\path[draw=drawColor,line width= 0.6pt,line join=round] ( 28.93,147.34) -- (271.45,147.34);

\path[draw=drawColor,line width= 0.6pt,line join=round] ( 28.93,147.34) -- (271.45,147.34);

\path[draw=drawColor,line width= 0.6pt,line join=round] ( 28.93,147.34) -- (271.45,147.34);

\path[draw=drawColor,line width= 0.6pt,line join=round] ( 28.93,147.34) -- (271.45,147.34);

\path[draw=drawColor,line width= 0.6pt,line join=round] ( 28.93,147.34) -- (271.45,147.34);

\path[draw=drawColor,line width= 0.6pt,line join=round] ( 28.93,147.34) -- (271.45,147.34);

\path[draw=drawColor,line width= 0.6pt,line join=round] ( 28.93,147.34) -- (271.45,147.34);

\path[draw=drawColor,line width= 0.6pt,line join=round] ( 28.93,147.34) -- (271.45,147.34);

\path[draw=drawColor,line width= 0.6pt,line join=round] ( 28.93,147.34) -- (271.45,147.34);

\path[draw=drawColor,line width= 0.6pt,line join=round] ( 28.93,147.34) -- (271.45,147.34);

\path[draw=drawColor,line width= 0.6pt,line join=round] ( 28.93,147.34) -- (271.45,147.34);

\path[draw=drawColor,line width= 0.6pt,line join=round] ( 28.93,147.34) -- (271.45,147.34);

\path[draw=drawColor,line width= 0.6pt,line join=round] ( 28.93,147.34) -- (271.45,147.34);

\path[draw=drawColor,line width= 0.6pt,line join=round] ( 28.93,147.34) -- (271.45,147.34);

\path[draw=drawColor,line width= 0.6pt,line join=round] ( 28.93,147.34) -- (271.45,147.34);

\path[draw=drawColor,line width= 0.6pt,line join=round] ( 28.93,147.34) -- (271.45,147.34);

\path[draw=drawColor,line width= 0.6pt,line join=round] ( 28.93,147.34) -- (271.45,147.34);

\path[draw=drawColor,line width= 0.6pt,line join=round] ( 28.93,147.34) -- (271.45,147.34);

\path[draw=drawColor,line width= 0.6pt,line join=round] ( 28.93,147.34) -- (271.45,147.34);

\path[draw=drawColor,line width= 0.6pt,line join=round] ( 28.93,147.34) -- (271.45,147.34);

\path[draw=drawColor,line width= 0.6pt,line join=round] ( 28.93,147.34) -- (271.45,147.34);

\path[draw=drawColor,line width= 0.6pt,line join=round] ( 28.93,147.34) -- (271.45,147.34);

\path[draw=drawColor,line width= 0.6pt,line join=round] ( 28.93,147.34) -- (271.45,147.34);

\path[draw=drawColor,line width= 0.6pt,line join=round] ( 28.93,147.34) -- (271.45,147.34);

\path[draw=drawColor,line width= 0.6pt,line join=round] ( 28.93,147.34) -- (271.45,147.34);

\path[draw=drawColor,line width= 0.6pt,line join=round] ( 28.93,147.34) -- (271.45,147.34);

\path[draw=drawColor,line width= 0.6pt,line join=round] ( 28.93,147.34) -- (271.45,147.34);

\path[draw=drawColor,line width= 0.6pt,line join=round] ( 28.93,147.34) -- (271.45,147.34);

\path[draw=drawColor,line width= 0.6pt,line join=round] ( 28.93,147.34) -- (271.45,147.34);

\path[draw=drawColor,line width= 0.6pt,line join=round] ( 28.93,147.34) -- (271.45,147.34);

\path[draw=drawColor,line width= 0.6pt,line join=round] ( 28.93,147.34) -- (271.45,147.34);

\path[draw=drawColor,line width= 0.6pt,line join=round] ( 28.93,147.34) -- (271.45,147.34);

\path[draw=drawColor,line width= 0.6pt,line join=round] ( 28.93,147.34) -- (271.45,147.34);

\path[draw=drawColor,line width= 0.6pt,line join=round] ( 28.93,147.34) -- (271.45,147.34);

\path[draw=drawColor,line width= 0.6pt,line join=round] ( 28.93,147.34) -- (271.45,147.34);

\path[draw=drawColor,line width= 0.6pt,line join=round] ( 28.93,147.34) -- (271.45,147.34);

\path[draw=drawColor,line width= 0.6pt,line join=round] ( 28.93,147.34) -- (271.45,147.34);

\path[draw=drawColor,line width= 0.6pt,line join=round] ( 28.93,147.34) -- (271.45,147.34);

\path[draw=drawColor,line width= 0.6pt,line join=round] ( 28.93,147.34) -- (271.45,147.34);

\path[draw=drawColor,line width= 0.6pt,line join=round] ( 28.93,147.34) -- (271.45,147.34);

\path[draw=drawColor,line width= 0.6pt,line join=round] ( 28.93,147.34) -- (271.45,147.34);

\path[draw=drawColor,line width= 0.6pt,line join=round] ( 28.93,147.34) -- (271.45,147.34);

\path[draw=drawColor,line width= 0.6pt,line join=round] ( 28.93,147.34) -- (271.45,147.34);

\path[draw=drawColor,line width= 0.6pt,line join=round] ( 28.93,147.34) -- (271.45,147.34);

\path[draw=drawColor,line width= 0.6pt,line join=round] ( 28.93,147.34) -- (271.45,147.34);

\path[draw=drawColor,line width= 0.6pt,line join=round] ( 28.93,147.34) -- (271.45,147.34);

\path[draw=drawColor,line width= 0.6pt,line join=round] ( 28.93,147.34) -- (271.45,147.34);

\path[draw=drawColor,line width= 0.6pt,line join=round] ( 28.93,147.34) -- (271.45,147.34);

\path[draw=drawColor,line width= 0.6pt,line join=round] ( 28.93,147.34) -- (271.45,147.34);

\path[draw=drawColor,line width= 0.6pt,line join=round] ( 28.93,147.34) -- (271.45,147.34);

\path[draw=drawColor,line width= 0.6pt,line join=round] ( 28.93,147.34) -- (271.45,147.34);

\path[draw=drawColor,line width= 0.6pt,line join=round] ( 28.93,147.34) -- (271.45,147.34);

\path[draw=drawColor,line width= 0.6pt,line join=round] ( 28.93,147.34) -- (271.45,147.34);

\path[draw=drawColor,line width= 0.6pt,line join=round] ( 28.93,147.34) -- (271.45,147.34);

\path[draw=drawColor,line width= 0.6pt,line join=round] ( 28.93,147.34) -- (271.45,147.34);

\path[draw=drawColor,line width= 0.6pt,line join=round] ( 28.93,147.34) -- (271.45,147.34);

\path[draw=drawColor,line width= 0.6pt,line join=round] ( 28.93,147.34) -- (271.45,147.34);

\path[draw=drawColor,line width= 0.6pt,line join=round] ( 28.93,147.34) -- (271.45,147.34);

\path[draw=drawColor,line width= 0.6pt,line join=round] ( 28.93,147.34) -- (271.45,147.34);

\path[draw=drawColor,line width= 0.6pt,line join=round] ( 28.93,147.34) -- (271.45,147.34);

\path[draw=drawColor,line width= 0.6pt,line join=round] ( 28.93,147.34) -- (271.45,147.34);

\path[draw=drawColor,line width= 0.6pt,line join=round] ( 28.93,147.34) -- (271.45,147.34);

\path[draw=drawColor,line width= 0.6pt,line join=round] ( 28.93,147.34) -- (271.45,147.34);

\path[draw=drawColor,line width= 0.6pt,line join=round] ( 28.93,147.34) -- (271.45,147.34);

\path[draw=drawColor,line width= 0.6pt,line join=round] ( 28.93,147.34) -- (271.45,147.34);

\path[draw=drawColor,line width= 0.6pt,line join=round] ( 28.93,147.34) -- (271.45,147.34);

\path[draw=drawColor,line width= 0.6pt,line join=round] ( 28.93,147.34) -- (271.45,147.34);

\path[draw=drawColor,line width= 0.6pt,line join=round] ( 28.93,147.34) -- (271.45,147.34);

\path[draw=drawColor,line width= 0.6pt,line join=round] ( 28.93,147.34) -- (271.45,147.34);

\path[draw=drawColor,line width= 0.6pt,line join=round] ( 28.93,147.34) -- (271.45,147.34);

\path[draw=drawColor,line width= 0.6pt,line join=round] ( 28.93,147.34) -- (271.45,147.34);

\path[draw=drawColor,line width= 0.6pt,line join=round] ( 28.93,147.34) -- (271.45,147.34);

\path[draw=drawColor,line width= 0.6pt,line join=round] ( 28.93,147.34) -- (271.45,147.34);

\path[draw=drawColor,line width= 0.6pt,line join=round] ( 28.93,147.34) -- (271.45,147.34);

\path[draw=drawColor,line width= 0.6pt,line join=round] ( 28.93,147.34) -- (271.45,147.34);

\path[draw=drawColor,line width= 0.6pt,line join=round] ( 28.93,147.34) -- (271.45,147.34);

\path[draw=drawColor,line width= 0.6pt,line join=round] ( 28.93,147.34) -- (271.45,147.34);

\path[draw=drawColor,line width= 0.6pt,line join=round] ( 28.93,147.34) -- (271.45,147.34);

\path[draw=drawColor,line width= 0.6pt,line join=round] ( 28.93,147.34) -- (271.45,147.34);

\path[draw=drawColor,line width= 0.6pt,line join=round] ( 28.93,147.34) -- (271.45,147.34);

\path[draw=drawColor,line width= 0.6pt,line join=round] ( 28.93,147.34) -- (271.45,147.34);

\path[draw=drawColor,line width= 0.6pt,line join=round] ( 28.93,147.34) -- (271.45,147.34);

\path[draw=drawColor,line width= 0.6pt,line join=round] ( 28.93,147.34) -- (271.45,147.34);

\path[draw=drawColor,line width= 0.6pt,line join=round] ( 28.93,147.34) -- (271.45,147.34);

\path[draw=drawColor,line width= 0.6pt,line join=round] ( 28.93,147.34) -- (271.45,147.34);

\path[draw=drawColor,line width= 0.6pt,line join=round] ( 28.93,147.34) -- (271.45,147.34);

\path[draw=drawColor,line width= 0.6pt,line join=round] ( 28.93,147.34) -- (271.45,147.34);

\path[draw=drawColor,line width= 0.6pt,line join=round] ( 28.93,147.34) -- (271.45,147.34);

\path[draw=drawColor,line width= 0.6pt,line join=round] ( 28.93,147.34) -- (271.45,147.34);

\path[draw=drawColor,line width= 0.6pt,line join=round] ( 28.93,147.34) -- (271.45,147.34);

\path[draw=drawColor,line width= 0.6pt,line join=round] ( 28.93,147.34) -- (271.45,147.34);

\path[draw=drawColor,line width= 0.6pt,line join=round] ( 28.93,147.34) -- (271.45,147.34);

\path[draw=drawColor,line width= 0.6pt,line join=round] ( 28.93,147.34) -- (271.45,147.34);

\path[draw=drawColor,line width= 0.6pt,line join=round] ( 28.93,147.34) -- (271.45,147.34);

\path[draw=drawColor,line width= 0.6pt,line join=round] ( 28.93,147.34) -- (271.45,147.34);

\path[draw=drawColor,line width= 0.6pt,line join=round] ( 28.93,147.34) -- (271.45,147.34);

\path[draw=drawColor,line width= 0.6pt,line join=round] ( 28.93,147.34) -- (271.45,147.34);

\path[draw=drawColor,line width= 0.6pt,line join=round] ( 28.93,147.34) -- (271.45,147.34);

\path[draw=drawColor,line width= 0.6pt,line join=round] ( 28.93,147.34) -- (271.45,147.34);

\path[draw=drawColor,line width= 0.6pt,line join=round] ( 28.93,147.34) -- (271.45,147.34);

\path[draw=drawColor,line width= 0.6pt,line join=round] ( 28.93,147.34) -- (271.45,147.34);

\path[draw=drawColor,line width= 0.6pt,line join=round] ( 28.93,147.34) -- (271.45,147.34);

\path[draw=drawColor,line width= 0.6pt,line join=round] ( 28.93,147.34) -- (271.45,147.34);

\path[draw=drawColor,line width= 0.6pt,line join=round] ( 28.93,147.34) -- (271.45,147.34);

\path[draw=drawColor,line width= 0.6pt,line join=round] ( 28.93,147.34) -- (271.45,147.34);

\path[draw=drawColor,line width= 0.6pt,line join=round] ( 28.93,147.34) -- (271.45,147.34);

\path[draw=drawColor,line width= 0.6pt,line join=round] ( 28.93,147.34) -- (271.45,147.34);

\path[draw=drawColor,line width= 0.6pt,line join=round] ( 28.93,147.34) -- (271.45,147.34);

\path[draw=drawColor,line width= 0.6pt,line join=round] ( 28.93,147.34) -- (271.45,147.34);

\path[draw=drawColor,line width= 0.6pt,line join=round] ( 28.93,147.34) -- (271.45,147.34);

\path[draw=drawColor,line width= 0.6pt,line join=round] ( 28.93,147.34) -- (271.45,147.34);

\path[draw=drawColor,line width= 0.6pt,line join=round] ( 28.93,147.34) -- (271.45,147.34);

\path[draw=drawColor,line width= 0.6pt,line join=round] ( 28.93,147.34) -- (271.45,147.34);

\path[draw=drawColor,line width= 0.6pt,line join=round] ( 28.93,147.34) -- (271.45,147.34);

\path[draw=drawColor,line width= 0.6pt,line join=round] ( 28.93,147.34) -- (271.45,147.34);

\path[draw=drawColor,line width= 0.6pt,line join=round] ( 28.93,147.34) -- (271.45,147.34);

\path[draw=drawColor,line width= 0.6pt,line join=round] ( 28.93,147.34) -- (271.45,147.34);

\path[draw=drawColor,line width= 0.6pt,line join=round] ( 28.93,147.34) -- (271.45,147.34);

\path[draw=drawColor,line width= 0.6pt,line join=round] ( 28.93,147.34) -- (271.45,147.34);

\path[draw=drawColor,line width= 0.6pt,line join=round] ( 28.93,147.34) -- (271.45,147.34);

\path[draw=drawColor,line width= 0.6pt,line join=round] ( 28.93,147.34) -- (271.45,147.34);

\path[draw=drawColor,line width= 0.6pt,line join=round] ( 28.93,147.34) -- (271.45,147.34);

\path[draw=drawColor,line width= 0.6pt,line join=round] ( 28.93,147.34) -- (271.45,147.34);

\path[draw=drawColor,line width= 0.6pt,line join=round] ( 28.93,147.34) -- (271.45,147.34);

\path[draw=drawColor,line width= 0.6pt,line join=round] ( 28.93,147.34) -- (271.45,147.34);

\path[draw=drawColor,line width= 0.6pt,line join=round] ( 28.93,147.34) -- (271.45,147.34);

\path[draw=drawColor,line width= 0.6pt,line join=round] ( 28.93,147.34) -- (271.45,147.34);

\path[draw=drawColor,line width= 0.6pt,line join=round] ( 28.93,147.34) -- (271.45,147.34);

\path[draw=drawColor,line width= 0.6pt,line join=round] ( 28.93,147.34) -- (271.45,147.34);

\path[draw=drawColor,line width= 0.6pt,line join=round] ( 28.93,147.34) -- (271.45,147.34);

\path[draw=drawColor,line width= 0.6pt,line join=round] ( 28.93,147.34) -- (271.45,147.34);

\path[draw=drawColor,line width= 0.6pt,line join=round] ( 28.93,147.34) -- (271.45,147.34);

\path[draw=drawColor,line width= 0.6pt,line join=round] ( 28.93,147.34) -- (271.45,147.34);

\path[draw=drawColor,line width= 0.6pt,line join=round] ( 28.93,147.34) -- (271.45,147.34);

\path[draw=drawColor,line width= 0.6pt,line join=round] ( 28.93,147.34) -- (271.45,147.34);

\path[draw=drawColor,line width= 0.6pt,line join=round] ( 28.93,147.34) -- (271.45,147.34);

\path[draw=drawColor,line width= 0.6pt,line join=round] ( 28.93,147.34) -- (271.45,147.34);

\path[draw=drawColor,line width= 0.6pt,line join=round] ( 28.93,147.34) -- (271.45,147.34);

\path[draw=drawColor,line width= 0.6pt,line join=round] ( 28.93,147.34) -- (271.45,147.34);

\path[draw=drawColor,line width= 0.6pt,line join=round] ( 28.93,147.34) -- (271.45,147.34);

\path[draw=drawColor,line width= 0.6pt,line join=round] ( 28.93,147.34) -- (271.45,147.34);

\path[draw=drawColor,line width= 0.6pt,line join=round] ( 28.93,147.34) -- (271.45,147.34);

\path[draw=drawColor,line width= 0.6pt,line join=round] ( 28.93,147.34) -- (271.45,147.34);

\path[draw=drawColor,line width= 0.6pt,line join=round] ( 28.93,147.34) -- (271.45,147.34);

\path[draw=drawColor,line width= 0.6pt,line join=round] ( 28.93,147.34) -- (271.45,147.34);

\path[draw=drawColor,line width= 0.6pt,line join=round] ( 28.93,147.34) -- (271.45,147.34);

\path[draw=drawColor,line width= 0.6pt,line join=round] ( 28.93,147.34) -- (271.45,147.34);

\path[draw=drawColor,line width= 0.6pt,line join=round] ( 28.93,147.34) -- (271.45,147.34);

\path[draw=drawColor,line width= 0.6pt,line join=round] ( 28.93,147.34) -- (271.45,147.34);

\path[draw=drawColor,line width= 0.6pt,line join=round] ( 28.93,147.34) -- (271.45,147.34);

\path[draw=drawColor,line width= 0.6pt,line join=round] ( 28.93,147.34) -- (271.45,147.34);

\path[draw=drawColor,line width= 0.6pt,line join=round] ( 28.93,147.34) -- (271.45,147.34);

\path[draw=drawColor,line width= 0.6pt,line join=round] ( 28.93,147.34) -- (271.45,147.34);

\path[draw=drawColor,line width= 0.6pt,line join=round] ( 28.93,147.34) -- (271.45,147.34);

\path[draw=drawColor,line width= 0.6pt,line join=round] ( 28.93,147.34) -- (271.45,147.34);

\path[draw=drawColor,line width= 0.6pt,line join=round] ( 28.93,147.34) -- (271.45,147.34);

\path[draw=drawColor,line width= 0.6pt,line join=round] ( 28.93,147.34) -- (271.45,147.34);

\path[draw=drawColor,line width= 0.6pt,line join=round] ( 28.93,147.34) -- (271.45,147.34);

\path[draw=drawColor,line width= 0.6pt,line join=round] ( 28.93,147.34) -- (271.45,147.34);

\path[draw=drawColor,line width= 0.6pt,line join=round] ( 28.93,147.34) -- (271.45,147.34);

\path[draw=drawColor,line width= 0.6pt,line join=round] ( 28.93,147.34) -- (271.45,147.34);

\path[draw=drawColor,line width= 0.6pt,line join=round] ( 28.93,147.34) -- (271.45,147.34);

\path[draw=drawColor,line width= 0.6pt,line join=round] ( 28.93,147.34) -- (271.45,147.34);

\path[draw=drawColor,line width= 0.6pt,line join=round] ( 28.93,147.34) -- (271.45,147.34);

\path[draw=drawColor,line width= 0.6pt,line join=round] ( 28.93,147.34) -- (271.45,147.34);

\path[draw=drawColor,line width= 0.6pt,line join=round] ( 28.93,147.34) -- (271.45,147.34);

\path[draw=drawColor,line width= 0.6pt,line join=round] ( 28.93,147.34) -- (271.45,147.34);

\path[draw=drawColor,line width= 0.6pt,line join=round] ( 28.93,147.34) -- (271.45,147.34);

\path[draw=drawColor,line width= 0.6pt,line join=round] ( 28.93,147.34) -- (271.45,147.34);

\path[draw=drawColor,line width= 0.6pt,line join=round] ( 28.93,147.34) -- (271.45,147.34);

\path[draw=drawColor,line width= 0.6pt,line join=round] ( 28.93,147.34) -- (271.45,147.34);

\path[draw=drawColor,line width= 0.6pt,line join=round] ( 28.93,147.34) -- (271.45,147.34);

\path[draw=drawColor,line width= 0.6pt,line join=round] ( 28.93,147.34) -- (271.45,147.34);

\path[draw=drawColor,line width= 0.6pt,line join=round] ( 28.93,147.34) -- (271.45,147.34);

\path[draw=drawColor,line width= 0.6pt,line join=round] ( 28.93,147.34) -- (271.45,147.34);

\path[draw=drawColor,line width= 0.6pt,line join=round] ( 28.93,147.34) -- (271.45,147.34);

\path[draw=drawColor,line width= 0.6pt,line join=round] ( 28.93,147.34) -- (271.45,147.34);

\path[draw=drawColor,line width= 0.6pt,line join=round] ( 28.93,147.34) -- (271.45,147.34);

\path[draw=drawColor,line width= 0.6pt,line join=round] ( 28.93,147.34) -- (271.45,147.34);

\path[draw=drawColor,line width= 0.6pt,line join=round] ( 28.93,147.34) -- (271.45,147.34);

\path[draw=drawColor,line width= 0.6pt,line join=round] ( 28.93,147.34) -- (271.45,147.34);

\path[draw=drawColor,line width= 0.6pt,line join=round] ( 28.93,147.34) -- (271.45,147.34);

\path[draw=drawColor,line width= 0.6pt,line join=round] ( 28.93,147.34) -- (271.45,147.34);

\path[draw=drawColor,line width= 0.6pt,line join=round] ( 28.93,147.34) -- (271.45,147.34);

\path[draw=drawColor,line width= 0.6pt,line join=round] ( 28.93,147.34) -- (271.45,147.34);

\path[draw=drawColor,line width= 0.6pt,line join=round] ( 28.93,147.34) -- (271.45,147.34);

\path[draw=drawColor,line width= 0.6pt,line join=round] ( 28.93,147.34) -- (271.45,147.34);

\path[draw=drawColor,line width= 0.6pt,line join=round] ( 28.93,147.34) -- (271.45,147.34);

\path[draw=drawColor,line width= 0.6pt,line join=round] ( 28.93,147.34) -- (271.45,147.34);

\path[draw=drawColor,line width= 0.6pt,line join=round] ( 28.93,147.34) -- (271.45,147.34);

\path[draw=drawColor,line width= 0.6pt,line join=round] ( 28.93,147.34) -- (271.45,147.34);

\path[draw=drawColor,line width= 0.6pt,line join=round] ( 28.93,147.34) -- (271.45,147.34);

\path[draw=drawColor,line width= 0.6pt,line join=round] ( 28.93,147.34) -- (271.45,147.34);

\path[draw=drawColor,line width= 0.6pt,line join=round] ( 28.93,147.34) -- (271.45,147.34);

\path[draw=drawColor,line width= 0.6pt,line join=round] ( 28.93,147.34) -- (271.45,147.34);

\path[draw=drawColor,line width= 0.6pt,line join=round] ( 28.93,147.34) -- (271.45,147.34);

\path[draw=drawColor,line width= 0.6pt,line join=round] ( 28.93,147.34) -- (271.45,147.34);

\path[draw=drawColor,line width= 0.6pt,line join=round] ( 28.93,147.34) -- (271.45,147.34);

\path[draw=drawColor,line width= 0.6pt,line join=round] ( 28.93,147.34) -- (271.45,147.34);

\path[draw=drawColor,line width= 0.6pt,line join=round] ( 28.93,147.34) -- (271.45,147.34);

\path[draw=drawColor,line width= 0.6pt,line join=round] ( 28.93,147.34) -- (271.45,147.34);

\path[draw=drawColor,line width= 0.6pt,line join=round] ( 28.93,147.34) -- (271.45,147.34);

\path[draw=drawColor,line width= 0.6pt,line join=round] ( 28.93,147.34) -- (271.45,147.34);

\path[draw=drawColor,line width= 0.6pt,line join=round] ( 28.93,147.34) -- (271.45,147.34);

\path[draw=drawColor,line width= 0.6pt,line join=round] ( 28.93,147.34) -- (271.45,147.34);

\path[draw=drawColor,line width= 0.6pt,line join=round] ( 28.93,147.34) -- (271.45,147.34);

\path[draw=drawColor,line width= 0.6pt,line join=round] ( 28.93,147.34) -- (271.45,147.34);

\path[draw=drawColor,line width= 0.6pt,line join=round] ( 28.93,147.34) -- (271.45,147.34);

\path[draw=drawColor,line width= 0.6pt,line join=round] ( 28.93,147.34) -- (271.45,147.34);

\path[draw=drawColor,line width= 0.6pt,line join=round] ( 28.93,147.34) -- (271.45,147.34);

\path[draw=drawColor,line width= 0.6pt,line join=round] ( 28.93,147.34) -- (271.45,147.34);

\path[draw=drawColor,line width= 0.6pt,line join=round] ( 28.93,147.34) -- (271.45,147.34);

\path[draw=drawColor,line width= 0.6pt,line join=round] ( 28.93,147.34) -- (271.45,147.34);

\path[draw=drawColor,line width= 0.6pt,line join=round] ( 28.93,147.34) -- (271.45,147.34);

\path[draw=drawColor,line width= 0.6pt,line join=round] ( 28.93,147.34) -- (271.45,147.34);

\path[draw=drawColor,line width= 0.6pt,line join=round] ( 28.93,147.34) -- (271.45,147.34);

\path[draw=drawColor,line width= 0.6pt,line join=round] ( 28.93,147.34) -- (271.45,147.34);

\path[draw=drawColor,line width= 0.6pt,line join=round] ( 28.93,147.34) -- (271.45,147.34);

\path[draw=drawColor,line width= 0.6pt,line join=round] ( 28.93,147.34) -- (271.45,147.34);

\path[draw=drawColor,line width= 0.6pt,line join=round] ( 28.93,147.34) -- (271.45,147.34);

\path[draw=drawColor,line width= 0.6pt,line join=round] ( 28.93,147.34) -- (271.45,147.34);

\path[draw=drawColor,line width= 0.6pt,line join=round] ( 28.93,147.34) -- (271.45,147.34);

\path[draw=drawColor,line width= 0.6pt,line join=round] ( 28.93,147.34) -- (271.45,147.34);

\path[draw=drawColor,line width= 0.6pt,line join=round] ( 28.93,147.34) -- (271.45,147.34);

\path[draw=drawColor,line width= 0.6pt,line join=round] ( 28.93,147.34) -- (271.45,147.34);

\path[draw=drawColor,line width= 0.6pt,line join=round] ( 28.93,147.34) -- (271.45,147.34);

\path[draw=drawColor,line width= 0.6pt,line join=round] ( 28.93,147.34) -- (271.45,147.34);

\path[draw=drawColor,line width= 0.6pt,line join=round] ( 28.93,147.34) -- (271.45,147.34);

\path[draw=drawColor,line width= 0.6pt,line join=round] ( 28.93,147.34) -- (271.45,147.34);

\path[draw=drawColor,line width= 0.6pt,line join=round] ( 28.93,147.34) -- (271.45,147.34);

\path[draw=drawColor,line width= 0.6pt,line join=round] ( 28.93,147.34) -- (271.45,147.34);

\path[draw=drawColor,line width= 0.6pt,line join=round] ( 28.93,147.34) -- (271.45,147.34);

\path[draw=drawColor,line width= 0.6pt,line join=round] ( 28.93,147.34) -- (271.45,147.34);

\path[draw=drawColor,line width= 0.6pt,line join=round] ( 28.93,147.34) -- (271.45,147.34);

\path[draw=drawColor,line width= 0.6pt,line join=round] ( 28.93,147.34) -- (271.45,147.34);

\path[draw=drawColor,line width= 0.6pt,line join=round] ( 28.93,147.34) -- (271.45,147.34);

\path[draw=drawColor,line width= 0.6pt,line join=round] ( 28.93,147.34) -- (271.45,147.34);

\path[draw=drawColor,line width= 0.6pt,line join=round] ( 28.93,147.34) -- (271.45,147.34);

\path[draw=drawColor,line width= 0.6pt,line join=round] ( 28.93,147.34) -- (271.45,147.34);

\path[draw=drawColor,line width= 0.6pt,line join=round] ( 28.93,147.34) -- (271.45,147.34);

\path[draw=drawColor,line width= 0.6pt,line join=round] ( 28.93,147.34) -- (271.45,147.34);

\path[draw=drawColor,line width= 0.6pt,line join=round] ( 28.93,147.34) -- (271.45,147.34);

\path[draw=drawColor,line width= 0.6pt,line join=round] ( 28.93,147.34) -- (271.45,147.34);

\path[draw=drawColor,line width= 0.6pt,line join=round] ( 28.93,147.34) -- (271.45,147.34);

\path[draw=drawColor,line width= 0.6pt,line join=round] ( 28.93,147.34) -- (271.45,147.34);

\path[draw=drawColor,line width= 0.6pt,line join=round] ( 28.93,147.34) -- (271.45,147.34);

\path[draw=drawColor,line width= 0.6pt,line join=round] ( 28.93,147.34) -- (271.45,147.34);

\path[draw=drawColor,line width= 0.6pt,line join=round] ( 28.93,147.34) -- (271.45,147.34);

\path[draw=drawColor,line width= 0.6pt,line join=round] ( 28.93,147.34) -- (271.45,147.34);

\path[draw=drawColor,line width= 0.6pt,line join=round] ( 28.93,147.34) -- (271.45,147.34);

\path[draw=drawColor,line width= 0.6pt,line join=round] ( 28.93,147.34) -- (271.45,147.34);

\path[draw=drawColor,line width= 0.6pt,line join=round] ( 28.93,147.34) -- (271.45,147.34);

\path[draw=drawColor,line width= 0.6pt,line join=round] ( 28.93,147.34) -- (271.45,147.34);

\path[draw=drawColor,line width= 0.6pt,line join=round] ( 28.93,147.34) -- (271.45,147.34);

\path[draw=drawColor,line width= 0.6pt,line join=round] ( 28.93,147.34) -- (271.45,147.34);

\path[draw=drawColor,line width= 0.6pt,line join=round] ( 28.93,147.34) -- (271.45,147.34);

\path[draw=drawColor,line width= 0.6pt,line join=round] ( 28.93,147.34) -- (271.45,147.34);

\path[draw=drawColor,line width= 0.6pt,line join=round] ( 28.93,147.34) -- (271.45,147.34);

\path[draw=drawColor,line width= 0.6pt,line join=round] ( 28.93,147.34) -- (271.45,147.34);

\path[draw=drawColor,line width= 0.6pt,line join=round] ( 28.93,147.34) -- (271.45,147.34);

\path[draw=drawColor,line width= 0.6pt,line join=round] ( 28.93,147.34) -- (271.45,147.34);

\path[draw=drawColor,line width= 0.6pt,line join=round] ( 28.93,147.34) -- (271.45,147.34);

\path[draw=drawColor,line width= 0.6pt,line join=round] ( 28.93,147.34) -- (271.45,147.34);

\path[draw=drawColor,line width= 0.6pt,line join=round] ( 28.93,147.34) -- (271.45,147.34);

\path[draw=drawColor,line width= 0.6pt,line join=round] ( 28.93,147.34) -- (271.45,147.34);

\path[draw=drawColor,line width= 0.6pt,line join=round] ( 28.93,147.34) -- (271.45,147.34);

\path[draw=drawColor,line width= 0.6pt,line join=round] ( 28.93,147.34) -- (271.45,147.34);

\path[draw=drawColor,line width= 0.6pt,line join=round] ( 28.93,147.34) -- (271.45,147.34);

\path[draw=drawColor,line width= 0.6pt,line join=round] ( 28.93,147.34) -- (271.45,147.34);

\path[draw=drawColor,line width= 0.6pt,line join=round] ( 28.93,147.34) -- (271.45,147.34);

\path[draw=drawColor,line width= 0.6pt,line join=round] ( 28.93,147.34) -- (271.45,147.34);

\path[draw=drawColor,line width= 0.6pt,line join=round] ( 28.93,147.34) -- (271.45,147.34);

\path[draw=drawColor,line width= 0.6pt,line join=round] ( 28.93,147.34) -- (271.45,147.34);

\path[draw=drawColor,line width= 0.6pt,line join=round] ( 28.93,147.34) -- (271.45,147.34);

\path[draw=drawColor,line width= 0.6pt,line join=round] ( 28.93,147.34) -- (271.45,147.34);

\path[draw=drawColor,line width= 0.6pt,line join=round] ( 28.93,147.34) -- (271.45,147.34);

\path[draw=drawColor,line width= 0.6pt,line join=round] ( 28.93,147.34) -- (271.45,147.34);

\path[draw=drawColor,line width= 0.6pt,line join=round] ( 28.93,147.34) -- (271.45,147.34);

\path[draw=drawColor,line width= 0.6pt,line join=round] ( 28.93,147.34) -- (271.45,147.34);

\path[draw=drawColor,line width= 0.6pt,line join=round] ( 28.93,147.34) -- (271.45,147.34);

\path[draw=drawColor,line width= 0.6pt,line join=round] ( 28.93,147.34) -- (271.45,147.34);

\path[draw=drawColor,line width= 0.6pt,line join=round] ( 28.93,147.34) -- (271.45,147.34);

\path[draw=drawColor,line width= 0.6pt,line join=round] ( 28.93,147.34) -- (271.45,147.34);

\path[draw=drawColor,line width= 0.6pt,line join=round] ( 28.93,147.34) -- (271.45,147.34);

\path[draw=drawColor,line width= 0.6pt,line join=round] ( 28.93,147.34) -- (271.45,147.34);

\path[draw=drawColor,line width= 0.6pt,line join=round] ( 28.93,147.34) -- (271.45,147.34);

\path[draw=drawColor,line width= 0.6pt,line join=round] ( 28.93,147.34) -- (271.45,147.34);

\path[draw=drawColor,line width= 0.6pt,line join=round] ( 28.93,147.34) -- (271.45,147.34);

\path[draw=drawColor,line width= 0.6pt,line join=round] ( 28.93,147.34) -- (271.45,147.34);

\path[draw=drawColor,line width= 0.6pt,line join=round] ( 28.93,147.34) -- (271.45,147.34);

\path[draw=drawColor,line width= 0.6pt,line join=round] ( 28.93,147.34) -- (271.45,147.34);

\path[draw=drawColor,line width= 0.6pt,line join=round] ( 28.93,147.34) -- (271.45,147.34);

\path[draw=drawColor,line width= 0.6pt,line join=round] ( 28.93,147.34) -- (271.45,147.34);

\path[draw=drawColor,line width= 0.6pt,line join=round] ( 28.93,147.34) -- (271.45,147.34);

\path[draw=drawColor,line width= 0.6pt,line join=round] ( 28.93,147.34) -- (271.45,147.34);

\path[draw=drawColor,line width= 0.6pt,line join=round] ( 28.93,147.34) -- (271.45,147.34);

\path[draw=drawColor,line width= 0.6pt,line join=round] ( 28.93,147.34) -- (271.45,147.34);

\path[draw=drawColor,line width= 0.6pt,line join=round] ( 28.93,147.34) -- (271.45,147.34);

\path[draw=drawColor,line width= 0.6pt,line join=round] ( 28.93,147.34) -- (271.45,147.34);

\path[draw=drawColor,line width= 0.6pt,line join=round] ( 28.93,147.34) -- (271.45,147.34);

\path[draw=drawColor,line width= 0.6pt,line join=round] ( 28.93,147.34) -- (271.45,147.34);

\path[draw=drawColor,line width= 0.6pt,line join=round] ( 28.93,147.34) -- (271.45,147.34);

\path[draw=drawColor,line width= 0.6pt,line join=round] ( 28.93,147.34) -- (271.45,147.34);

\path[draw=drawColor,line width= 0.6pt,line join=round] ( 28.93,147.34) -- (271.45,147.34);

\path[draw=drawColor,line width= 0.6pt,line join=round] ( 28.93,147.34) -- (271.45,147.34);

\path[draw=drawColor,line width= 0.6pt,line join=round] ( 28.93,147.34) -- (271.45,147.34);

\path[draw=drawColor,line width= 0.6pt,line join=round] ( 28.93,147.34) -- (271.45,147.34);

\path[draw=drawColor,line width= 0.6pt,line join=round] ( 28.93,147.34) -- (271.45,147.34);

\path[draw=drawColor,line width= 0.6pt,line join=round] ( 28.93,147.34) -- (271.45,147.34);

\path[draw=drawColor,line width= 0.6pt,line join=round] ( 28.93,147.34) -- (271.45,147.34);

\path[draw=drawColor,line width= 0.6pt,line join=round] ( 28.93,147.34) -- (271.45,147.34);

\path[draw=drawColor,line width= 0.6pt,line join=round] ( 28.93,147.34) -- (271.45,147.34);

\path[draw=drawColor,line width= 0.6pt,line join=round] ( 28.93,147.34) -- (271.45,147.34);

\path[draw=drawColor,line width= 0.6pt,line join=round] ( 28.93,147.34) -- (271.45,147.34);

\path[draw=drawColor,line width= 0.6pt,line join=round] ( 28.93,147.34) -- (271.45,147.34);

\path[draw=drawColor,line width= 0.6pt,line join=round] ( 28.93,147.34) -- (271.45,147.34);

\path[draw=drawColor,line width= 0.6pt,line join=round] ( 28.93,147.34) -- (271.45,147.34);

\path[draw=drawColor,line width= 0.6pt,line join=round] ( 28.93,147.34) -- (271.45,147.34);

\path[draw=drawColor,line width= 0.6pt,line join=round] ( 28.93,147.34) -- (271.45,147.34);

\path[draw=drawColor,line width= 0.6pt,line join=round] ( 28.93,147.34) -- (271.45,147.34);

\path[draw=drawColor,line width= 0.6pt,line join=round] ( 28.93,147.34) -- (271.45,147.34);

\path[draw=drawColor,line width= 0.6pt,line join=round] ( 28.93,147.34) -- (271.45,147.34);

\path[draw=drawColor,line width= 0.6pt,line join=round] ( 28.93,147.34) -- (271.45,147.34);

\path[draw=drawColor,line width= 0.6pt,line join=round] ( 28.93,147.34) -- (271.45,147.34);

\path[draw=drawColor,line width= 0.6pt,line join=round] ( 28.93,147.34) -- (271.45,147.34);

\path[draw=drawColor,line width= 0.6pt,line join=round] ( 28.93,147.34) -- (271.45,147.34);

\path[draw=drawColor,line width= 0.6pt,line join=round] ( 28.93,147.34) -- (271.45,147.34);

\path[draw=drawColor,line width= 0.6pt,line join=round] ( 28.93,147.34) -- (271.45,147.34);

\path[draw=drawColor,line width= 0.6pt,line join=round] ( 28.93,147.34) -- (271.45,147.34);

\path[draw=drawColor,line width= 0.6pt,line join=round] ( 28.93,147.34) -- (271.45,147.34);

\path[draw=drawColor,line width= 0.6pt,line join=round] ( 28.93,147.34) -- (271.45,147.34);

\path[draw=drawColor,line width= 0.6pt,line join=round] ( 28.93,147.34) -- (271.45,147.34);

\path[draw=drawColor,line width= 0.6pt,line join=round] ( 28.93,147.34) -- (271.45,147.34);

\path[draw=drawColor,line width= 0.6pt,line join=round] ( 28.93,147.34) -- (271.45,147.34);

\path[draw=drawColor,line width= 0.6pt,line join=round] ( 28.93,147.34) -- (271.45,147.34);

\path[draw=drawColor,line width= 0.6pt,line join=round] ( 28.93,147.34) -- (271.45,147.34);

\path[draw=drawColor,line width= 0.6pt,line join=round] ( 28.93,147.34) -- (271.45,147.34);

\path[draw=drawColor,line width= 0.6pt,line join=round] ( 28.93,147.34) -- (271.45,147.34);

\path[draw=drawColor,line width= 0.6pt,line join=round] ( 28.93,147.34) -- (271.45,147.34);

\path[draw=drawColor,line width= 0.6pt,line join=round] ( 28.93,147.34) -- (271.45,147.34);

\path[draw=drawColor,line width= 0.6pt,line join=round] ( 28.93,147.34) -- (271.45,147.34);

\path[draw=drawColor,line width= 0.6pt,line join=round] ( 28.93,147.34) -- (271.45,147.34);

\path[draw=drawColor,line width= 0.6pt,line join=round] ( 28.93,147.34) -- (271.45,147.34);

\path[draw=drawColor,line width= 0.6pt,line join=round] ( 28.93,147.34) -- (271.45,147.34);

\path[draw=drawColor,line width= 0.6pt,line join=round] ( 28.93,147.34) -- (271.45,147.34);

\path[draw=drawColor,line width= 0.6pt,line join=round] ( 28.93,147.34) -- (271.45,147.34);

\path[draw=drawColor,line width= 0.6pt,line join=round] ( 28.93,147.34) -- (271.45,147.34);

\path[draw=drawColor,line width= 0.6pt,line join=round] ( 28.93,147.34) -- (271.45,147.34);

\path[draw=drawColor,line width= 0.6pt,line join=round] ( 28.93,147.34) -- (271.45,147.34);

\path[draw=drawColor,line width= 0.6pt,line join=round] ( 28.93,147.34) -- (271.45,147.34);

\path[draw=drawColor,line width= 0.6pt,line join=round] ( 28.93,147.34) -- (271.45,147.34);

\path[draw=drawColor,line width= 0.6pt,line join=round] ( 28.93,147.34) -- (271.45,147.34);

\path[draw=drawColor,line width= 0.6pt,line join=round] ( 28.93,147.34) -- (271.45,147.34);

\path[draw=drawColor,line width= 0.6pt,line join=round] ( 28.93,147.34) -- (271.45,147.34);

\path[draw=drawColor,line width= 0.6pt,line join=round] ( 28.93,147.34) -- (271.45,147.34);

\path[draw=drawColor,line width= 0.6pt,line join=round] ( 28.93,147.34) -- (271.45,147.34);

\path[draw=drawColor,line width= 0.6pt,line join=round] ( 28.93,147.34) -- (271.45,147.34);

\path[draw=drawColor,line width= 0.6pt,line join=round] ( 28.93,147.34) -- (271.45,147.34);

\path[draw=drawColor,line width= 0.6pt,line join=round] ( 28.93,147.34) -- (271.45,147.34);

\path[draw=drawColor,line width= 0.6pt,line join=round] ( 28.93,147.34) -- (271.45,147.34);

\path[draw=drawColor,line width= 0.6pt,line join=round] ( 28.93,147.34) -- (271.45,147.34);

\path[draw=drawColor,line width= 0.6pt,line join=round] ( 28.93,147.34) -- (271.45,147.34);

\path[draw=drawColor,line width= 0.6pt,line join=round] ( 28.93,147.34) -- (271.45,147.34);

\path[draw=drawColor,line width= 0.6pt,line join=round] ( 28.93,147.34) -- (271.45,147.34);

\path[draw=drawColor,line width= 0.6pt,line join=round] ( 28.93,147.34) -- (271.45,147.34);

\path[draw=drawColor,line width= 0.6pt,line join=round] ( 28.93,147.34) -- (271.45,147.34);

\path[draw=drawColor,line width= 0.6pt,line join=round] ( 28.93,147.34) -- (271.45,147.34);

\path[draw=drawColor,line width= 0.6pt,line join=round] ( 28.93,147.34) -- (271.45,147.34);

\path[draw=drawColor,line width= 0.6pt,line join=round] ( 28.93,147.34) -- (271.45,147.34);

\path[draw=drawColor,line width= 0.6pt,line join=round] ( 28.93,147.34) -- (271.45,147.34);

\path[draw=drawColor,line width= 0.6pt,line join=round] ( 28.93,147.34) -- (271.45,147.34);

\path[draw=drawColor,line width= 0.6pt,line join=round] ( 28.93,147.34) -- (271.45,147.34);

\path[draw=drawColor,line width= 0.6pt,line join=round] ( 28.93,147.34) -- (271.45,147.34);

\path[draw=drawColor,line width= 0.6pt,line join=round] ( 28.93,147.34) -- (271.45,147.34);

\path[draw=drawColor,line width= 0.6pt,line join=round] ( 28.93,147.34) -- (271.45,147.34);

\path[draw=drawColor,line width= 0.6pt,line join=round] ( 28.93,147.34) -- (271.45,147.34);

\path[draw=drawColor,line width= 0.6pt,line join=round] ( 28.93,147.34) -- (271.45,147.34);

\path[draw=drawColor,line width= 0.6pt,line join=round] ( 28.93,147.34) -- (271.45,147.34);

\path[draw=drawColor,line width= 0.6pt,line join=round] ( 28.93,147.34) -- (271.45,147.34);

\path[draw=drawColor,line width= 0.6pt,line join=round] ( 28.93,147.34) -- (271.45,147.34);

\path[draw=drawColor,line width= 0.6pt,line join=round] ( 28.93,147.34) -- (271.45,147.34);

\path[draw=drawColor,line width= 0.6pt,line join=round] ( 28.93,147.34) -- (271.45,147.34);

\path[draw=drawColor,line width= 0.6pt,line join=round] ( 28.93,147.34) -- (271.45,147.34);

\path[draw=drawColor,line width= 0.6pt,line join=round] ( 28.93,147.34) -- (271.45,147.34);

\path[draw=drawColor,line width= 0.6pt,line join=round] ( 28.93,147.34) -- (271.45,147.34);

\path[draw=drawColor,line width= 0.6pt,line join=round] ( 28.93,147.34) -- (271.45,147.34);

\path[draw=drawColor,line width= 0.6pt,line join=round] ( 28.93,147.34) -- (271.45,147.34);

\path[draw=drawColor,line width= 0.6pt,line join=round] ( 28.93,147.34) -- (271.45,147.34);

\path[draw=drawColor,line width= 0.6pt,line join=round] ( 28.93,147.34) -- (271.45,147.34);

\path[draw=drawColor,line width= 0.6pt,line join=round] ( 28.93,147.34) -- (271.45,147.34);

\path[draw=drawColor,line width= 0.6pt,line join=round] ( 28.93,147.34) -- (271.45,147.34);

\path[draw=drawColor,line width= 0.6pt,line join=round] ( 28.93,147.34) -- (271.45,147.34);

\path[draw=drawColor,line width= 0.6pt,line join=round] ( 28.93,147.34) -- (271.45,147.34);

\path[draw=drawColor,line width= 0.6pt,line join=round] ( 28.93,147.34) -- (271.45,147.34);

\path[draw=drawColor,line width= 0.6pt,line join=round] ( 28.93,147.34) -- (271.45,147.34);

\path[draw=drawColor,line width= 0.6pt,line join=round] ( 28.93,147.34) -- (271.45,147.34);

\path[draw=drawColor,line width= 0.6pt,line join=round] ( 28.93,147.34) -- (271.45,147.34);

\path[draw=drawColor,line width= 0.6pt,line join=round] ( 28.93,147.34) -- (271.45,147.34);

\path[draw=drawColor,line width= 0.6pt,line join=round] ( 28.93,147.34) -- (271.45,147.34);

\path[draw=drawColor,line width= 0.6pt,line join=round] ( 28.93,147.34) -- (271.45,147.34);

\path[draw=drawColor,line width= 0.6pt,line join=round] ( 28.93,147.34) -- (271.45,147.34);

\path[draw=drawColor,line width= 0.6pt,line join=round] ( 28.93,147.34) -- (271.45,147.34);

\path[draw=drawColor,line width= 0.6pt,line join=round] ( 28.93,147.34) -- (271.45,147.34);

\path[draw=drawColor,line width= 0.6pt,line join=round] ( 28.93,147.34) -- (271.45,147.34);

\path[draw=drawColor,line width= 0.6pt,line join=round] ( 28.93,147.34) -- (271.45,147.34);

\path[draw=drawColor,line width= 0.6pt,line join=round] ( 28.93,147.34) -- (271.45,147.34);

\path[draw=drawColor,line width= 0.6pt,line join=round] ( 28.93,147.34) -- (271.45,147.34);

\path[draw=drawColor,line width= 0.6pt,line join=round] ( 28.93,147.34) -- (271.45,147.34);

\path[draw=drawColor,line width= 0.6pt,line join=round] ( 28.93,147.34) -- (271.45,147.34);

\path[draw=drawColor,line width= 0.6pt,line join=round] ( 28.93,147.34) -- (271.45,147.34);

\path[draw=drawColor,line width= 0.6pt,line join=round] ( 28.93,147.34) -- (271.45,147.34);

\path[draw=drawColor,line width= 0.6pt,line join=round] ( 28.93,147.34) -- (271.45,147.34);

\path[draw=drawColor,line width= 0.6pt,line join=round] ( 28.93,147.34) -- (271.45,147.34);

\path[draw=drawColor,line width= 0.6pt,line join=round] ( 28.93,147.34) -- (271.45,147.34);

\path[draw=drawColor,line width= 0.6pt,line join=round] ( 28.93,147.34) -- (271.45,147.34);

\path[draw=drawColor,line width= 0.6pt,line join=round] ( 28.93,147.34) -- (271.45,147.34);

\path[draw=drawColor,line width= 0.6pt,line join=round] ( 28.93,147.34) -- (271.45,147.34);

\path[draw=drawColor,line width= 0.6pt,line join=round] ( 28.93,147.34) -- (271.45,147.34);

\path[draw=drawColor,line width= 0.6pt,line join=round] ( 28.93,147.34) -- (271.45,147.34);

\path[draw=drawColor,line width= 0.6pt,line join=round] ( 28.93,147.34) -- (271.45,147.34);

\path[draw=drawColor,line width= 0.6pt,line join=round] ( 28.93,147.34) -- (271.45,147.34);

\path[draw=drawColor,line width= 0.6pt,line join=round] ( 28.93,147.34) -- (271.45,147.34);

\path[draw=drawColor,line width= 0.6pt,line join=round] ( 28.93,147.34) -- (271.45,147.34);

\path[draw=drawColor,line width= 0.6pt,line join=round] ( 28.93,147.34) -- (271.45,147.34);

\path[draw=drawColor,line width= 0.6pt,line join=round] ( 28.93,147.34) -- (271.45,147.34);

\path[draw=drawColor,line width= 0.6pt,line join=round] ( 28.93,147.34) -- (271.45,147.34);

\path[draw=drawColor,line width= 0.6pt,line join=round] ( 28.93,147.34) -- (271.45,147.34);

\path[draw=drawColor,line width= 0.6pt,line join=round] ( 28.93,147.34) -- (271.45,147.34);

\path[draw=drawColor,line width= 0.6pt,line join=round] ( 28.93,147.34) -- (271.45,147.34);

\path[draw=drawColor,line width= 0.6pt,line join=round] ( 28.93,147.34) -- (271.45,147.34);

\path[draw=drawColor,line width= 0.6pt,line join=round] ( 28.93,147.34) -- (271.45,147.34);

\path[draw=drawColor,line width= 0.6pt,line join=round] ( 28.93,147.34) -- (271.45,147.34);

\path[draw=drawColor,line width= 0.6pt,line join=round] ( 28.93,147.34) -- (271.45,147.34);

\path[draw=drawColor,line width= 0.6pt,line join=round] ( 28.93,147.34) -- (271.45,147.34);

\path[draw=drawColor,line width= 0.6pt,line join=round] ( 28.93,147.34) -- (271.45,147.34);

\path[draw=drawColor,line width= 0.6pt,line join=round] ( 28.93,147.34) -- (271.45,147.34);

\path[draw=drawColor,line width= 0.6pt,line join=round] ( 28.93,147.34) -- (271.45,147.34);

\path[draw=drawColor,line width= 0.6pt,line join=round] ( 28.93,147.34) -- (271.45,147.34);

\path[draw=drawColor,line width= 0.6pt,line join=round] ( 28.93,147.34) -- (271.45,147.34);

\path[draw=drawColor,line width= 0.6pt,line join=round] ( 28.93,147.34) -- (271.45,147.34);

\path[draw=drawColor,line width= 0.6pt,line join=round] ( 28.93,147.34) -- (271.45,147.34);

\path[draw=drawColor,line width= 0.6pt,line join=round] ( 28.93,147.34) -- (271.45,147.34);

\path[draw=drawColor,line width= 0.6pt,line join=round] ( 28.93,147.34) -- (271.45,147.34);

\path[draw=drawColor,line width= 0.6pt,line join=round] ( 28.93,147.34) -- (271.45,147.34);

\path[draw=drawColor,line width= 0.6pt,line join=round] ( 28.93,147.34) -- (271.45,147.34);

\path[draw=drawColor,line width= 0.6pt,line join=round] ( 28.93,147.34) -- (271.45,147.34);

\path[draw=drawColor,line width= 0.6pt,line join=round] ( 28.93,147.34) -- (271.45,147.34);

\path[draw=drawColor,line width= 0.6pt,line join=round] ( 28.93,147.34) -- (271.45,147.34);

\path[draw=drawColor,line width= 0.6pt,line join=round] ( 28.93,147.34) -- (271.45,147.34);

\path[draw=drawColor,line width= 0.6pt,line join=round] ( 28.93,147.34) -- (271.45,147.34);

\path[draw=drawColor,line width= 0.6pt,line join=round] ( 28.93,147.34) -- (271.45,147.34);

\path[draw=drawColor,line width= 0.6pt,line join=round] ( 28.93,147.34) -- (271.45,147.34);

\path[draw=drawColor,line width= 0.6pt,line join=round] ( 28.93,147.34) -- (271.45,147.34);

\path[draw=drawColor,line width= 0.6pt,line join=round] ( 28.93,147.34) -- (271.45,147.34);

\path[draw=drawColor,line width= 0.6pt,line join=round] ( 28.93,147.34) -- (271.45,147.34);

\path[draw=drawColor,line width= 0.6pt,line join=round] ( 28.93,147.34) -- (271.45,147.34);

\path[draw=drawColor,line width= 0.6pt,line join=round] ( 28.93,147.34) -- (271.45,147.34);

\path[draw=drawColor,line width= 0.6pt,line join=round] ( 28.93,147.34) -- (271.45,147.34);

\path[draw=drawColor,line width= 0.6pt,line join=round] ( 28.93,147.34) -- (271.45,147.34);

\path[draw=drawColor,line width= 0.6pt,line join=round] ( 28.93,147.34) -- (271.45,147.34);

\path[draw=drawColor,line width= 0.6pt,line join=round] ( 28.93,147.34) -- (271.45,147.34);

\path[draw=drawColor,line width= 0.6pt,line join=round] ( 28.93,147.34) -- (271.45,147.34);

\path[draw=drawColor,line width= 0.6pt,line join=round] ( 28.93,147.34) -- (271.45,147.34);

\path[draw=drawColor,line width= 0.6pt,line join=round] ( 28.93,147.34) -- (271.45,147.34);

\path[draw=drawColor,line width= 0.6pt,line join=round] ( 28.93,147.34) -- (271.45,147.34);

\path[draw=drawColor,line width= 0.6pt,line join=round] ( 28.93,147.34) -- (271.45,147.34);

\path[draw=drawColor,line width= 0.6pt,line join=round] ( 28.93,147.34) -- (271.45,147.34);

\path[draw=drawColor,line width= 0.6pt,line join=round] ( 28.93,147.34) -- (271.45,147.34);

\path[draw=drawColor,line width= 0.6pt,line join=round] ( 28.93,147.34) -- (271.45,147.34);

\path[draw=drawColor,line width= 0.6pt,line join=round] ( 28.93,147.34) -- (271.45,147.34);

\path[draw=drawColor,line width= 0.6pt,line join=round] ( 28.93,147.34) -- (271.45,147.34);

\path[draw=drawColor,line width= 0.6pt,line join=round] ( 28.93,147.34) -- (271.45,147.34);

\path[draw=drawColor,line width= 0.6pt,line join=round] ( 28.93,147.34) -- (271.45,147.34);

\path[draw=drawColor,line width= 0.6pt,line join=round] ( 28.93,147.34) -- (271.45,147.34);

\path[draw=drawColor,line width= 0.6pt,line join=round] ( 28.93,147.34) -- (271.45,147.34);

\path[draw=drawColor,line width= 0.6pt,line join=round] ( 28.93,147.34) -- (271.45,147.34);

\path[draw=drawColor,line width= 0.6pt,line join=round] ( 28.93,147.34) -- (271.45,147.34);

\path[draw=drawColor,line width= 0.6pt,line join=round] ( 28.93,147.34) -- (271.45,147.34);

\path[draw=drawColor,line width= 0.6pt,line join=round] ( 28.93,147.34) -- (271.45,147.34);

\path[draw=drawColor,line width= 0.6pt,line join=round] ( 28.93,147.34) -- (271.45,147.34);

\path[draw=drawColor,line width= 0.6pt,line join=round] ( 28.93,147.34) -- (271.45,147.34);

\path[draw=drawColor,line width= 0.6pt,line join=round] ( 28.93,147.34) -- (271.45,147.34);

\path[draw=drawColor,line width= 0.6pt,line join=round] ( 28.93,147.34) -- (271.45,147.34);

\path[draw=drawColor,line width= 0.6pt,line join=round] ( 28.93,147.34) -- (271.45,147.34);

\path[draw=drawColor,line width= 0.6pt,line join=round] ( 28.93,147.34) -- (271.45,147.34);

\path[draw=drawColor,line width= 0.6pt,line join=round] ( 28.93,147.34) -- (271.45,147.34);

\path[draw=drawColor,line width= 0.6pt,line join=round] ( 28.93,147.34) -- (271.45,147.34);

\path[draw=drawColor,line width= 0.6pt,line join=round] ( 28.93,147.34) -- (271.45,147.34);

\path[draw=drawColor,line width= 0.6pt,line join=round] ( 28.93,147.34) -- (271.45,147.34);

\path[draw=drawColor,line width= 0.6pt,line join=round] ( 28.93,147.34) -- (271.45,147.34);

\path[draw=drawColor,line width= 0.6pt,line join=round] ( 28.93,147.34) -- (271.45,147.34);

\path[draw=drawColor,line width= 0.6pt,line join=round] ( 28.93,147.34) -- (271.45,147.34);

\path[draw=drawColor,line width= 0.6pt,line join=round] ( 28.93,147.34) -- (271.45,147.34);

\path[draw=drawColor,line width= 0.6pt,line join=round] ( 28.93,147.34) -- (271.45,147.34);

\path[draw=drawColor,line width= 0.6pt,line join=round] ( 28.93,147.34) -- (271.45,147.34);

\path[draw=drawColor,line width= 0.6pt,line join=round] ( 28.93,147.34) -- (271.45,147.34);

\path[draw=drawColor,line width= 0.6pt,line join=round] ( 28.93,147.34) -- (271.45,147.34);

\path[draw=drawColor,line width= 0.6pt,line join=round] ( 28.93,147.34) -- (271.45,147.34);

\path[draw=drawColor,line width= 0.6pt,line join=round] ( 28.93,147.34) -- (271.45,147.34);

\path[draw=drawColor,line width= 0.6pt,line join=round] ( 28.93,147.34) -- (271.45,147.34);

\path[draw=drawColor,line width= 0.6pt,line join=round] ( 28.93,147.34) -- (271.45,147.34);

\path[draw=drawColor,line width= 0.6pt,line join=round] ( 28.93,147.34) -- (271.45,147.34);

\path[draw=drawColor,line width= 0.6pt,line join=round] ( 28.93,147.34) -- (271.45,147.34);

\path[draw=drawColor,line width= 0.6pt,line join=round] ( 28.93,147.34) -- (271.45,147.34);

\path[draw=drawColor,line width= 0.6pt,line join=round] ( 28.93,147.34) -- (271.45,147.34);

\path[draw=drawColor,line width= 0.6pt,line join=round] ( 28.93,147.34) -- (271.45,147.34);

\path[draw=drawColor,line width= 0.6pt,line join=round] ( 28.93,147.34) -- (271.45,147.34);

\path[draw=drawColor,line width= 0.6pt,line join=round] ( 28.93,147.34) -- (271.45,147.34);

\path[draw=drawColor,line width= 0.6pt,line join=round] ( 28.93,147.34) -- (271.45,147.34);

\path[draw=drawColor,line width= 0.6pt,line join=round] ( 28.93,147.34) -- (271.45,147.34);

\path[draw=drawColor,line width= 0.6pt,line join=round] ( 28.93,147.34) -- (271.45,147.34);

\path[draw=drawColor,line width= 0.6pt,line join=round] ( 28.93,147.34) -- (271.45,147.34);

\path[draw=drawColor,line width= 0.6pt,line join=round] ( 28.93,147.34) -- (271.45,147.34);

\path[draw=drawColor,line width= 0.6pt,line join=round] ( 28.93,147.34) -- (271.45,147.34);

\path[draw=drawColor,line width= 0.6pt,line join=round] ( 28.93,147.34) -- (271.45,147.34);

\path[draw=drawColor,line width= 0.6pt,line join=round] ( 28.93,147.34) -- (271.45,147.34);

\path[draw=drawColor,line width= 0.6pt,line join=round] ( 28.93,147.34) -- (271.45,147.34);

\path[draw=drawColor,line width= 0.6pt,line join=round] ( 28.93,147.34) -- (271.45,147.34);

\path[draw=drawColor,line width= 0.6pt,line join=round] ( 28.93,147.34) -- (271.45,147.34);

\path[draw=drawColor,line width= 0.6pt,line join=round] ( 28.93,147.34) -- (271.45,147.34);

\path[draw=drawColor,line width= 0.6pt,line join=round] ( 28.93,147.34) -- (271.45,147.34);

\path[draw=drawColor,line width= 0.6pt,line join=round] ( 28.93,147.34) -- (271.45,147.34);

\path[draw=drawColor,line width= 0.6pt,line join=round] ( 28.93,147.34) -- (271.45,147.34);

\path[draw=drawColor,line width= 0.6pt,line join=round] ( 28.93,147.34) -- (271.45,147.34);

\path[draw=drawColor,line width= 0.6pt,line join=round] ( 28.93,147.34) -- (271.45,147.34);

\path[draw=drawColor,line width= 0.6pt,line join=round] ( 28.93,147.34) -- (271.45,147.34);

\path[draw=drawColor,line width= 0.6pt,line join=round] ( 28.93,147.34) -- (271.45,147.34);

\path[draw=drawColor,line width= 0.6pt,line join=round] ( 28.93,147.34) -- (271.45,147.34);

\path[draw=drawColor,line width= 0.6pt,line join=round] ( 28.93,147.34) -- (271.45,147.34);

\path[draw=drawColor,line width= 0.6pt,line join=round] ( 28.93,147.34) -- (271.45,147.34);

\path[draw=drawColor,line width= 0.6pt,line join=round] ( 28.93,147.34) -- (271.45,147.34);

\path[draw=drawColor,line width= 0.6pt,line join=round] ( 28.93,147.34) -- (271.45,147.34);

\path[draw=drawColor,line width= 0.6pt,line join=round] ( 28.93,147.34) -- (271.45,147.34);

\path[draw=drawColor,line width= 0.6pt,line join=round] ( 28.93,147.34) -- (271.45,147.34);

\path[draw=drawColor,line width= 0.6pt,line join=round] ( 28.93,147.34) -- (271.45,147.34);

\path[draw=drawColor,line width= 0.6pt,line join=round] ( 28.93,147.34) -- (271.45,147.34);

\path[draw=drawColor,line width= 0.6pt,line join=round] ( 28.93,147.34) -- (271.45,147.34);

\path[draw=drawColor,line width= 0.6pt,line join=round] ( 28.93,147.34) -- (271.45,147.34);

\path[draw=drawColor,line width= 0.6pt,line join=round] ( 28.93,147.34) -- (271.45,147.34);

\path[draw=drawColor,line width= 0.6pt,line join=round] ( 28.93,147.34) -- (271.45,147.34);

\path[draw=drawColor,line width= 0.6pt,line join=round] ( 28.93,147.34) -- (271.45,147.34);

\path[draw=drawColor,line width= 0.6pt,line join=round] ( 28.93,147.34) -- (271.45,147.34);

\path[draw=drawColor,line width= 0.6pt,line join=round] ( 28.93,147.34) -- (271.45,147.34);

\path[draw=drawColor,line width= 0.6pt,line join=round] ( 28.93,147.34) -- (271.45,147.34);

\path[draw=drawColor,line width= 0.6pt,line join=round] ( 28.93,147.34) -- (271.45,147.34);

\path[draw=drawColor,line width= 0.6pt,line join=round] ( 28.93,147.34) -- (271.45,147.34);

\path[draw=drawColor,line width= 0.6pt,line join=round] ( 28.93,147.34) -- (271.45,147.34);

\path[draw=drawColor,line width= 0.6pt,line join=round] ( 28.93,147.34) -- (271.45,147.34);

\path[draw=drawColor,line width= 0.6pt,line join=round] ( 28.93,147.34) -- (271.45,147.34);

\path[draw=drawColor,line width= 0.6pt,line join=round] ( 28.93,147.34) -- (271.45,147.34);

\path[draw=drawColor,line width= 0.6pt,line join=round] ( 28.93,147.34) -- (271.45,147.34);

\path[draw=drawColor,line width= 0.6pt,line join=round] ( 28.93,147.34) -- (271.45,147.34);

\path[draw=drawColor,line width= 0.6pt,line join=round] ( 28.93,147.34) -- (271.45,147.34);

\path[draw=drawColor,line width= 0.6pt,line join=round] ( 28.93,147.34) -- (271.45,147.34);

\path[draw=drawColor,line width= 0.6pt,line join=round] ( 28.93,147.34) -- (271.45,147.34);

\path[draw=drawColor,line width= 0.6pt,line join=round] ( 28.93,147.34) -- (271.45,147.34);

\path[draw=drawColor,line width= 0.6pt,line join=round] ( 28.93,147.34) -- (271.45,147.34);

\path[draw=drawColor,line width= 0.6pt,line join=round] ( 28.93,147.34) -- (271.45,147.34);

\path[draw=drawColor,line width= 0.6pt,line join=round] ( 28.93,147.34) -- (271.45,147.34);

\path[draw=drawColor,line width= 0.6pt,line join=round] ( 28.93,147.34) -- (271.45,147.34);

\path[draw=drawColor,line width= 0.6pt,line join=round] ( 28.93,147.34) -- (271.45,147.34);

\path[draw=drawColor,line width= 0.6pt,line join=round] ( 28.93,147.34) -- (271.45,147.34);

\path[draw=drawColor,line width= 0.6pt,line join=round] ( 28.93,147.34) -- (271.45,147.34);

\path[draw=drawColor,line width= 0.6pt,line join=round] ( 28.93,147.34) -- (271.45,147.34);

\path[draw=drawColor,line width= 0.6pt,line join=round] ( 28.93,147.34) -- (271.45,147.34);

\path[draw=drawColor,line width= 0.6pt,line join=round] ( 28.93,147.34) -- (271.45,147.34);

\path[draw=drawColor,line width= 0.6pt,line join=round] ( 28.93,147.34) -- (271.45,147.34);

\path[draw=drawColor,line width= 0.6pt,line join=round] ( 28.93,147.34) -- (271.45,147.34);

\path[draw=drawColor,line width= 0.6pt,line join=round] ( 28.93,147.34) -- (271.45,147.34);

\path[draw=drawColor,line width= 0.6pt,line join=round] ( 28.93,147.34) -- (271.45,147.34);

\path[draw=drawColor,line width= 0.6pt,line join=round] ( 28.93,147.34) -- (271.45,147.34);

\path[draw=drawColor,line width= 0.6pt,line join=round] ( 28.93,147.34) -- (271.45,147.34);

\path[draw=drawColor,line width= 0.6pt,line join=round] ( 28.93,147.34) -- (271.45,147.34);

\path[draw=drawColor,line width= 0.6pt,line join=round] ( 28.93,147.34) -- (271.45,147.34);

\path[draw=drawColor,line width= 0.6pt,line join=round] ( 28.93,147.34) -- (271.45,147.34);

\path[draw=drawColor,line width= 0.6pt,line join=round] ( 28.93,147.34) -- (271.45,147.34);

\path[draw=drawColor,line width= 0.6pt,line join=round] ( 28.93,147.34) -- (271.45,147.34);

\path[draw=drawColor,line width= 0.6pt,line join=round] ( 28.93,147.34) -- (271.45,147.34);

\path[draw=drawColor,line width= 0.6pt,line join=round] ( 28.93,147.34) -- (271.45,147.34);

\path[draw=drawColor,line width= 0.6pt,line join=round] ( 28.93,147.34) -- (271.45,147.34);

\path[draw=drawColor,line width= 0.6pt,line join=round] ( 28.93,147.34) -- (271.45,147.34);

\path[draw=drawColor,line width= 0.6pt,line join=round] ( 28.93,147.34) -- (271.45,147.34);

\path[draw=drawColor,line width= 0.6pt,line join=round] ( 28.93,147.34) -- (271.45,147.34);

\path[draw=drawColor,line width= 0.6pt,line join=round] ( 28.93,147.34) -- (271.45,147.34);

\path[draw=drawColor,line width= 0.6pt,line join=round] ( 28.93,147.34) -- (271.45,147.34);

\path[draw=drawColor,line width= 0.6pt,line join=round] ( 28.93,147.34) -- (271.45,147.34);

\path[draw=drawColor,line width= 0.6pt,line join=round] ( 28.93,147.34) -- (271.45,147.34);

\path[draw=drawColor,line width= 0.6pt,line join=round] ( 28.93,147.34) -- (271.45,147.34);

\path[draw=drawColor,line width= 0.6pt,line join=round] ( 28.93,147.34) -- (271.45,147.34);

\path[draw=drawColor,line width= 0.6pt,line join=round] ( 28.93,147.34) -- (271.45,147.34);

\path[draw=drawColor,line width= 0.6pt,line join=round] ( 28.93,147.34) -- (271.45,147.34);

\path[draw=drawColor,line width= 0.6pt,line join=round] ( 28.93,147.34) -- (271.45,147.34);

\path[draw=drawColor,line width= 0.6pt,line join=round] ( 28.93,147.34) -- (271.45,147.34);

\path[draw=drawColor,line width= 0.6pt,line join=round] ( 28.93,147.34) -- (271.45,147.34);

\path[draw=drawColor,line width= 0.6pt,line join=round] ( 28.93,147.34) -- (271.45,147.34);

\path[draw=drawColor,line width= 0.6pt,line join=round] ( 28.93,147.34) -- (271.45,147.34);

\path[draw=drawColor,line width= 0.6pt,line join=round] ( 28.93,147.34) -- (271.45,147.34);

\path[draw=drawColor,line width= 0.6pt,line join=round] ( 28.93,147.34) -- (271.45,147.34);

\path[draw=drawColor,line width= 0.6pt,line join=round] ( 28.93,147.34) -- (271.45,147.34);

\path[draw=drawColor,line width= 0.6pt,line join=round] ( 28.93,147.34) -- (271.45,147.34);

\path[draw=drawColor,line width= 0.6pt,line join=round] ( 28.93,147.34) -- (271.45,147.34);

\path[draw=drawColor,line width= 0.6pt,line join=round] ( 28.93,147.34) -- (271.45,147.34);

\path[draw=drawColor,line width= 0.6pt,line join=round] ( 28.93,147.34) -- (271.45,147.34);

\path[draw=drawColor,line width= 0.6pt,line join=round] ( 28.93,147.34) -- (271.45,147.34);

\path[draw=drawColor,line width= 0.6pt,line join=round] ( 28.93,147.34) -- (271.45,147.34);

\path[draw=drawColor,line width= 0.6pt,line join=round] ( 28.93,147.34) -- (271.45,147.34);

\path[draw=drawColor,line width= 0.6pt,line join=round] ( 28.93,147.34) -- (271.45,147.34);

\path[draw=drawColor,line width= 0.6pt,line join=round] ( 28.93,147.34) -- (271.45,147.34);

\path[draw=drawColor,line width= 0.6pt,line join=round] ( 28.93,147.34) -- (271.45,147.34);

\path[draw=drawColor,line width= 0.6pt,line join=round] ( 28.93,147.34) -- (271.45,147.34);

\path[draw=drawColor,line width= 0.6pt,line join=round] ( 28.93,147.34) -- (271.45,147.34);

\path[draw=drawColor,line width= 0.6pt,line join=round] ( 28.93,147.34) -- (271.45,147.34);

\path[draw=drawColor,line width= 0.6pt,line join=round] ( 28.93,147.34) -- (271.45,147.34);

\path[draw=drawColor,line width= 0.6pt,line join=round] ( 28.93,147.34) -- (271.45,147.34);

\path[draw=drawColor,line width= 0.6pt,line join=round] ( 28.93,147.34) -- (271.45,147.34);

\path[draw=drawColor,line width= 0.6pt,line join=round] ( 28.93,147.34) -- (271.45,147.34);

\path[draw=drawColor,line width= 0.6pt,line join=round] ( 28.93,147.34) -- (271.45,147.34);

\path[draw=drawColor,line width= 0.6pt,line join=round] ( 28.93,147.34) -- (271.45,147.34);

\path[draw=drawColor,line width= 0.6pt,line join=round] ( 28.93,147.34) -- (271.45,147.34);

\path[draw=drawColor,line width= 0.6pt,line join=round] ( 28.93,147.34) -- (271.45,147.34);

\path[draw=drawColor,line width= 0.6pt,line join=round] ( 28.93,147.34) -- (271.45,147.34);

\path[draw=drawColor,line width= 0.6pt,line join=round] ( 28.93,147.34) -- (271.45,147.34);

\path[draw=drawColor,line width= 0.6pt,line join=round] ( 28.93,147.34) -- (271.45,147.34);

\path[draw=drawColor,line width= 0.6pt,line join=round] ( 28.93,147.34) -- (271.45,147.34);

\path[draw=drawColor,line width= 0.6pt,line join=round] ( 28.93,147.34) -- (271.45,147.34);

\path[draw=drawColor,line width= 0.6pt,line join=round] ( 28.93,147.34) -- (271.45,147.34);

\path[draw=drawColor,line width= 0.6pt,line join=round] ( 28.93,147.34) -- (271.45,147.34);

\path[draw=drawColor,line width= 0.6pt,line join=round] ( 28.93,147.34) -- (271.45,147.34);

\path[draw=drawColor,line width= 0.6pt,line join=round] ( 28.93,147.34) -- (271.45,147.34);

\path[draw=drawColor,line width= 0.6pt,line join=round] ( 28.93,147.34) -- (271.45,147.34);

\path[draw=drawColor,line width= 0.6pt,line join=round] ( 28.93,147.34) -- (271.45,147.34);

\path[draw=drawColor,line width= 0.6pt,line join=round] ( 28.93,147.34) -- (271.45,147.34);

\path[draw=drawColor,line width= 0.6pt,line join=round] ( 28.93,147.34) -- (271.45,147.34);

\path[draw=drawColor,line width= 0.6pt,line join=round] ( 28.93,147.34) -- (271.45,147.34);

\path[draw=drawColor,line width= 0.6pt,line join=round] ( 28.93,147.34) -- (271.45,147.34);

\path[draw=drawColor,line width= 0.6pt,line join=round] ( 28.93,147.34) -- (271.45,147.34);

\path[draw=drawColor,line width= 0.6pt,line join=round] ( 28.93,147.34) -- (271.45,147.34);

\path[draw=drawColor,line width= 0.6pt,line join=round] ( 28.93,147.34) -- (271.45,147.34);

\path[draw=drawColor,line width= 0.6pt,line join=round] ( 28.93,147.34) -- (271.45,147.34);

\path[draw=drawColor,line width= 0.6pt,line join=round] ( 28.93,147.34) -- (271.45,147.34);

\path[draw=drawColor,line width= 0.6pt,line join=round] ( 28.93,147.34) -- (271.45,147.34);

\path[draw=drawColor,line width= 0.6pt,line join=round] ( 28.93,147.34) -- (271.45,147.34);

\path[draw=drawColor,line width= 0.6pt,line join=round] ( 28.93,147.34) -- (271.45,147.34);

\path[draw=drawColor,line width= 0.6pt,line join=round] ( 28.93,147.34) -- (271.45,147.34);

\path[draw=drawColor,line width= 0.6pt,line join=round] ( 28.93,147.34) -- (271.45,147.34);

\path[draw=drawColor,line width= 0.6pt,line join=round] ( 28.93,147.34) -- (271.45,147.34);

\path[draw=drawColor,line width= 0.6pt,line join=round] ( 28.93,147.34) -- (271.45,147.34);

\path[draw=drawColor,line width= 0.6pt,line join=round] ( 28.93,147.34) -- (271.45,147.34);

\path[draw=drawColor,line width= 0.6pt,line join=round] ( 28.93,147.34) -- (271.45,147.34);

\path[draw=drawColor,line width= 0.6pt,line join=round] ( 28.93,147.34) -- (271.45,147.34);

\path[draw=drawColor,line width= 0.6pt,line join=round] ( 28.93,147.34) -- (271.45,147.34);

\path[draw=drawColor,line width= 0.6pt,line join=round] ( 28.93,147.34) -- (271.45,147.34);

\path[draw=drawColor,line width= 0.6pt,line join=round] ( 28.93,147.34) -- (271.45,147.34);

\path[draw=drawColor,line width= 0.6pt,line join=round] ( 28.93,147.34) -- (271.45,147.34);

\path[draw=drawColor,line width= 0.6pt,line join=round] ( 28.93,147.34) -- (271.45,147.34);

\path[draw=drawColor,line width= 0.6pt,line join=round] ( 28.93,147.34) -- (271.45,147.34);

\path[draw=drawColor,line width= 0.6pt,line join=round] ( 28.93,147.34) -- (271.45,147.34);

\path[draw=drawColor,line width= 0.6pt,line join=round] ( 28.93,147.34) -- (271.45,147.34);

\path[draw=drawColor,line width= 0.6pt,line join=round] ( 28.93,147.34) -- (271.45,147.34);

\path[draw=drawColor,line width= 0.6pt,line join=round] ( 28.93,147.34) -- (271.45,147.34);

\path[draw=drawColor,line width= 0.6pt,line join=round] ( 28.93,147.34) -- (271.45,147.34);

\path[draw=drawColor,line width= 0.6pt,line join=round] ( 28.93,147.34) -- (271.45,147.34);

\path[draw=drawColor,line width= 0.6pt,line join=round] ( 28.93,147.34) -- (271.45,147.34);

\path[draw=drawColor,line width= 0.6pt,line join=round] ( 28.93,147.34) -- (271.45,147.34);

\path[draw=drawColor,line width= 0.6pt,line join=round] ( 28.93,147.34) -- (271.45,147.34);

\path[draw=drawColor,line width= 0.6pt,line join=round] ( 28.93,147.34) -- (271.45,147.34);

\path[draw=drawColor,line width= 0.6pt,line join=round] ( 28.93,147.34) -- (271.45,147.34);

\path[draw=drawColor,line width= 0.6pt,line join=round] ( 28.93,147.34) -- (271.45,147.34);

\path[draw=drawColor,line width= 0.6pt,line join=round] ( 28.93,147.34) -- (271.45,147.34);

\path[draw=drawColor,line width= 0.6pt,line join=round] ( 28.93,147.34) -- (271.45,147.34);

\path[draw=drawColor,line width= 0.6pt,line join=round] ( 28.93,147.34) -- (271.45,147.34);

\path[draw=drawColor,line width= 0.6pt,line join=round] ( 28.93,147.34) -- (271.45,147.34);

\path[draw=drawColor,line width= 0.6pt,line join=round] ( 28.93,147.34) -- (271.45,147.34);

\path[draw=drawColor,line width= 0.6pt,line join=round] ( 28.93,147.34) -- (271.45,147.34);

\path[draw=drawColor,line width= 0.6pt,line join=round] ( 28.93,147.34) -- (271.45,147.34);

\path[draw=drawColor,line width= 0.6pt,line join=round] ( 28.93,147.34) -- (271.45,147.34);

\path[draw=drawColor,line width= 0.6pt,line join=round] ( 28.93,147.34) -- (271.45,147.34);

\path[draw=drawColor,line width= 0.6pt,line join=round] ( 28.93,147.34) -- (271.45,147.34);

\path[draw=drawColor,line width= 0.6pt,line join=round] ( 28.93,147.34) -- (271.45,147.34);

\path[draw=drawColor,line width= 0.6pt,line join=round] ( 28.93,147.34) -- (271.45,147.34);

\path[draw=drawColor,line width= 0.6pt,line join=round] ( 28.93,147.34) -- (271.45,147.34);

\path[draw=drawColor,line width= 0.6pt,line join=round] ( 28.93,147.34) -- (271.45,147.34);

\path[draw=drawColor,line width= 0.6pt,line join=round] ( 28.93,147.34) -- (271.45,147.34);

\path[draw=drawColor,line width= 0.6pt,line join=round] ( 28.93,147.34) -- (271.45,147.34);

\path[draw=drawColor,line width= 0.6pt,line join=round] ( 28.93,147.34) -- (271.45,147.34);

\path[draw=drawColor,line width= 0.6pt,line join=round] ( 28.93,147.34) -- (271.45,147.34);

\path[draw=drawColor,line width= 0.6pt,line join=round] ( 28.93,147.34) -- (271.45,147.34);

\path[draw=drawColor,line width= 0.6pt,line join=round] ( 28.93,147.34) -- (271.45,147.34);

\path[draw=drawColor,line width= 0.6pt,line join=round] ( 28.93,147.34) -- (271.45,147.34);

\path[draw=drawColor,line width= 0.6pt,line join=round] ( 28.93,147.34) -- (271.45,147.34);

\path[draw=drawColor,line width= 0.6pt,line join=round] ( 28.93,147.34) -- (271.45,147.34);

\path[draw=drawColor,line width= 0.6pt,line join=round] ( 28.93,147.34) -- (271.45,147.34);

\path[draw=drawColor,line width= 0.6pt,line join=round] ( 28.93,147.34) -- (271.45,147.34);

\path[draw=drawColor,line width= 0.6pt,line join=round] ( 28.93,147.34) -- (271.45,147.34);

\path[draw=drawColor,line width= 0.6pt,line join=round] ( 28.93,147.34) -- (271.45,147.34);

\path[draw=drawColor,line width= 0.6pt,line join=round] ( 28.93,147.34) -- (271.45,147.34);

\path[draw=drawColor,line width= 0.6pt,line join=round] ( 28.93,147.34) -- (271.45,147.34);

\path[draw=drawColor,line width= 0.6pt,line join=round] ( 28.93,147.34) -- (271.45,147.34);

\path[draw=drawColor,line width= 0.6pt,line join=round] ( 28.93,147.34) -- (271.45,147.34);

\path[draw=drawColor,line width= 0.6pt,line join=round] ( 28.93,147.34) -- (271.45,147.34);

\path[draw=drawColor,line width= 0.6pt,line join=round] ( 28.93,147.34) -- (271.45,147.34);

\path[draw=drawColor,line width= 0.6pt,line join=round] ( 28.93,147.34) -- (271.45,147.34);

\path[draw=drawColor,line width= 0.6pt,line join=round] ( 28.93,147.34) -- (271.45,147.34);

\path[draw=drawColor,line width= 0.6pt,line join=round] ( 28.93,147.34) -- (271.45,147.34);

\path[draw=drawColor,line width= 0.6pt,line join=round] ( 28.93,147.34) -- (271.45,147.34);

\path[draw=drawColor,line width= 0.6pt,line join=round] ( 28.93,147.34) -- (271.45,147.34);

\path[draw=drawColor,line width= 0.6pt,line join=round] ( 28.93,147.34) -- (271.45,147.34);

\path[draw=drawColor,line width= 0.6pt,line join=round] ( 28.93,147.34) -- (271.45,147.34);

\path[draw=drawColor,line width= 0.6pt,line join=round] ( 28.93,147.34) -- (271.45,147.34);

\path[draw=drawColor,line width= 0.6pt,line join=round] ( 28.93,147.34) -- (271.45,147.34);

\path[draw=drawColor,line width= 0.6pt,line join=round] ( 28.93,147.34) -- (271.45,147.34);

\path[draw=drawColor,line width= 0.6pt,line join=round] ( 28.93,147.34) -- (271.45,147.34);

\path[draw=drawColor,line width= 0.6pt,line join=round] ( 28.93,147.34) -- (271.45,147.34);

\path[draw=drawColor,line width= 0.6pt,line join=round] ( 28.93,147.34) -- (271.45,147.34);

\path[draw=drawColor,line width= 0.6pt,line join=round] ( 28.93,147.34) -- (271.45,147.34);

\path[draw=drawColor,line width= 0.6pt,line join=round] ( 28.93,147.34) -- (271.45,147.34);

\path[draw=drawColor,line width= 0.6pt,line join=round] ( 28.93,147.34) -- (271.45,147.34);

\path[draw=drawColor,line width= 0.6pt,line join=round] ( 28.93,147.34) -- (271.45,147.34);

\path[draw=drawColor,line width= 0.6pt,line join=round] ( 28.93,147.34) -- (271.45,147.34);

\path[draw=drawColor,line width= 0.6pt,line join=round] ( 28.93,147.34) -- (271.45,147.34);

\path[draw=drawColor,line width= 0.6pt,line join=round] ( 28.93,147.34) -- (271.45,147.34);

\path[draw=drawColor,line width= 0.6pt,line join=round] ( 28.93,147.34) -- (271.45,147.34);

\path[draw=drawColor,line width= 0.6pt,line join=round] ( 28.93,147.34) -- (271.45,147.34);

\path[draw=drawColor,line width= 0.6pt,line join=round] ( 28.93,147.34) -- (271.45,147.34);

\path[draw=drawColor,line width= 0.6pt,line join=round] ( 28.93,147.34) -- (271.45,147.34);

\path[draw=drawColor,line width= 0.6pt,line join=round] ( 28.93,147.34) -- (271.45,147.34);

\path[draw=drawColor,line width= 0.6pt,line join=round] ( 28.93,147.34) -- (271.45,147.34);

\path[draw=drawColor,line width= 0.6pt,line join=round] ( 28.93,147.34) -- (271.45,147.34);

\path[draw=drawColor,line width= 0.6pt,line join=round] ( 28.93,147.34) -- (271.45,147.34);

\path[draw=drawColor,line width= 0.6pt,line join=round] ( 28.93,147.34) -- (271.45,147.34);

\path[draw=drawColor,line width= 0.6pt,line join=round] ( 28.93,147.34) -- (271.45,147.34);

\path[draw=drawColor,line width= 0.6pt,line join=round] ( 28.93,147.34) -- (271.45,147.34);

\path[draw=drawColor,line width= 0.6pt,line join=round] ( 28.93,147.34) -- (271.45,147.34);

\path[draw=drawColor,line width= 0.6pt,line join=round] ( 28.93,147.34) -- (271.45,147.34);

\path[draw=drawColor,line width= 0.6pt,line join=round] ( 28.93,147.34) -- (271.45,147.34);

\path[draw=drawColor,line width= 0.6pt,line join=round] ( 28.93,147.34) -- (271.45,147.34);

\path[draw=drawColor,line width= 0.6pt,line join=round] ( 28.93,147.34) -- (271.45,147.34);

\path[draw=drawColor,line width= 0.6pt,line join=round] ( 28.93,147.34) -- (271.45,147.34);

\path[draw=drawColor,line width= 0.6pt,line join=round] ( 28.93,147.34) -- (271.45,147.34);

\path[draw=drawColor,line width= 0.6pt,line join=round] ( 28.93,147.34) -- (271.45,147.34);

\path[draw=drawColor,line width= 0.6pt,line join=round] ( 28.93,147.34) -- (271.45,147.34);

\path[draw=drawColor,line width= 0.6pt,line join=round] ( 28.93,147.34) -- (271.45,147.34);

\path[draw=drawColor,line width= 0.6pt,line join=round] ( 28.93,147.34) -- (271.45,147.34);

\path[draw=drawColor,line width= 0.6pt,line join=round] ( 28.93,147.34) -- (271.45,147.34);

\path[draw=drawColor,line width= 0.6pt,line join=round] ( 28.93,147.34) -- (271.45,147.34);

\path[draw=drawColor,line width= 0.6pt,line join=round] ( 28.93,147.34) -- (271.45,147.34);

\path[draw=drawColor,line width= 0.6pt,line join=round] ( 28.93,147.34) -- (271.45,147.34);

\path[draw=drawColor,line width= 0.6pt,line join=round] ( 28.93,147.34) -- (271.45,147.34);

\path[draw=drawColor,line width= 0.6pt,line join=round] ( 28.93,147.34) -- (271.45,147.34);

\path[draw=drawColor,line width= 0.6pt,line join=round] ( 28.93,147.34) -- (271.45,147.34);

\path[draw=drawColor,line width= 0.6pt,line join=round] ( 28.93,147.34) -- (271.45,147.34);

\path[draw=drawColor,line width= 0.6pt,line join=round] ( 28.93,147.34) -- (271.45,147.34);

\path[draw=drawColor,line width= 0.6pt,line join=round] ( 28.93,147.34) -- (271.45,147.34);

\path[draw=drawColor,line width= 0.6pt,line join=round] ( 28.93,147.34) -- (271.45,147.34);

\path[draw=drawColor,line width= 0.6pt,line join=round] ( 28.93,147.34) -- (271.45,147.34);

\path[draw=drawColor,line width= 0.6pt,line join=round] ( 28.93,147.34) -- (271.45,147.34);

\path[draw=drawColor,line width= 0.6pt,line join=round] ( 28.93,147.34) -- (271.45,147.34);

\path[draw=drawColor,line width= 0.6pt,line join=round] ( 28.93,147.34) -- (271.45,147.34);

\path[draw=drawColor,line width= 0.6pt,line join=round] ( 28.93,147.34) -- (271.45,147.34);

\path[draw=drawColor,line width= 0.6pt,line join=round] ( 28.93,147.34) -- (271.45,147.34);

\path[draw=drawColor,line width= 0.6pt,line join=round] ( 28.93,147.34) -- (271.45,147.34);

\path[draw=drawColor,line width= 0.6pt,line join=round] ( 28.93,147.34) -- (271.45,147.34);

\path[draw=drawColor,line width= 0.6pt,line join=round] ( 28.93,147.34) -- (271.45,147.34);

\path[draw=drawColor,line width= 0.6pt,line join=round] ( 28.93,147.34) -- (271.45,147.34);

\path[draw=drawColor,line width= 0.6pt,line join=round] ( 28.93,147.34) -- (271.45,147.34);

\path[draw=drawColor,line width= 0.6pt,line join=round] ( 28.93,147.34) -- (271.45,147.34);

\path[draw=drawColor,line width= 0.6pt,line join=round] ( 28.93,147.34) -- (271.45,147.34);

\path[draw=drawColor,line width= 0.6pt,line join=round] ( 28.93,147.34) -- (271.45,147.34);

\path[draw=drawColor,line width= 0.6pt,line join=round] ( 28.93,147.34) -- (271.45,147.34);

\path[draw=drawColor,line width= 0.6pt,line join=round] ( 28.93,147.34) -- (271.45,147.34);

\path[draw=drawColor,line width= 0.6pt,line join=round] ( 28.93,147.34) -- (271.45,147.34);

\path[draw=drawColor,line width= 0.6pt,line join=round] ( 28.93,147.34) -- (271.45,147.34);

\path[draw=drawColor,line width= 0.6pt,line join=round] ( 28.93,147.34) -- (271.45,147.34);

\path[draw=drawColor,line width= 0.6pt,line join=round] ( 28.93,147.34) -- (271.45,147.34);

\path[draw=drawColor,line width= 0.6pt,line join=round] ( 28.93,147.34) -- (271.45,147.34);

\path[draw=drawColor,line width= 0.6pt,line join=round] ( 28.93,147.34) -- (271.45,147.34);

\path[draw=drawColor,line width= 0.6pt,line join=round] ( 28.93,147.34) -- (271.45,147.34);

\path[draw=drawColor,line width= 0.6pt,line join=round] ( 28.93,147.34) -- (271.45,147.34);

\path[draw=drawColor,line width= 0.6pt,line join=round] ( 28.93,147.34) -- (271.45,147.34);

\path[draw=drawColor,line width= 0.6pt,line join=round] ( 28.93,147.34) -- (271.45,147.34);

\path[draw=drawColor,line width= 0.6pt,line join=round] ( 28.93,147.34) -- (271.45,147.34);

\path[draw=drawColor,line width= 0.6pt,line join=round] ( 28.93,147.34) -- (271.45,147.34);

\path[draw=drawColor,line width= 0.6pt,line join=round] ( 28.93,147.34) -- (271.45,147.34);

\path[draw=drawColor,line width= 0.6pt,line join=round] ( 28.93,147.34) -- (271.45,147.34);

\path[draw=drawColor,line width= 0.6pt,line join=round] ( 28.93,147.34) -- (271.45,147.34);

\path[draw=drawColor,line width= 0.6pt,line join=round] ( 28.93,147.34) -- (271.45,147.34);

\path[draw=drawColor,line width= 0.6pt,line join=round] ( 28.93,147.34) -- (271.45,147.34);

\path[draw=drawColor,line width= 0.6pt,line join=round] ( 28.93,147.34) -- (271.45,147.34);

\path[draw=drawColor,line width= 0.6pt,line join=round] ( 28.93,147.34) -- (271.45,147.34);

\path[draw=drawColor,line width= 0.6pt,line join=round] ( 28.93,147.34) -- (271.45,147.34);

\path[draw=drawColor,line width= 0.6pt,line join=round] ( 28.93,147.34) -- (271.45,147.34);

\path[draw=drawColor,line width= 0.6pt,line join=round] ( 28.93,147.34) -- (271.45,147.34);

\path[draw=drawColor,line width= 0.6pt,line join=round] ( 28.93,147.34) -- (271.45,147.34);

\path[draw=drawColor,line width= 0.6pt,line join=round] ( 28.93,147.34) -- (271.45,147.34);

\path[draw=drawColor,line width= 0.6pt,line join=round] ( 28.93,147.34) -- (271.45,147.34);

\path[draw=drawColor,line width= 0.6pt,line join=round] ( 28.93,147.34) -- (271.45,147.34);

\path[draw=drawColor,line width= 0.6pt,line join=round] ( 28.93,147.34) -- (271.45,147.34);

\path[draw=drawColor,line width= 0.6pt,line join=round] ( 28.93,147.34) -- (271.45,147.34);

\path[draw=drawColor,line width= 0.6pt,line join=round] ( 28.93,147.34) -- (271.45,147.34);

\path[draw=drawColor,line width= 0.6pt,line join=round] ( 28.93,147.34) -- (271.45,147.34);

\path[draw=drawColor,line width= 0.6pt,line join=round] ( 28.93,147.34) -- (271.45,147.34);

\path[draw=drawColor,line width= 0.6pt,line join=round] ( 28.93,147.34) -- (271.45,147.34);

\path[draw=drawColor,line width= 0.6pt,line join=round] ( 28.93,147.34) -- (271.45,147.34);

\path[draw=drawColor,line width= 0.6pt,line join=round] ( 28.93,147.34) -- (271.45,147.34);

\path[draw=drawColor,line width= 0.6pt,line join=round] ( 28.93,147.34) -- (271.45,147.34);

\path[draw=drawColor,line width= 0.6pt,line join=round] ( 28.93,147.34) -- (271.45,147.34);

\path[draw=drawColor,line width= 0.6pt,line join=round] ( 28.93,147.34) -- (271.45,147.34);

\path[draw=drawColor,line width= 0.6pt,line join=round] ( 28.93,147.34) -- (271.45,147.34);

\path[draw=drawColor,line width= 0.6pt,line join=round] ( 28.93,147.34) -- (271.45,147.34);

\path[draw=drawColor,line width= 0.6pt,line join=round] ( 28.93,147.34) -- (271.45,147.34);

\path[draw=drawColor,line width= 0.6pt,line join=round] ( 28.93,147.34) -- (271.45,147.34);

\path[draw=drawColor,line width= 0.6pt,line join=round] ( 28.93,147.34) -- (271.45,147.34);

\path[draw=drawColor,line width= 0.6pt,line join=round] ( 28.93,147.34) -- (271.45,147.34);

\path[draw=drawColor,line width= 0.6pt,line join=round] ( 28.93,147.34) -- (271.45,147.34);

\path[draw=drawColor,line width= 0.6pt,line join=round] ( 28.93,147.34) -- (271.45,147.34);

\path[draw=drawColor,line width= 0.6pt,line join=round] ( 28.93,147.34) -- (271.45,147.34);

\path[draw=drawColor,line width= 0.6pt,line join=round] ( 28.93,147.34) -- (271.45,147.34);

\path[draw=drawColor,line width= 0.6pt,line join=round] ( 28.93,147.34) -- (271.45,147.34);

\path[draw=drawColor,line width= 0.6pt,line join=round] ( 28.93,147.34) -- (271.45,147.34);

\path[draw=drawColor,line width= 0.6pt,line join=round] ( 28.93,147.34) -- (271.45,147.34);

\path[draw=drawColor,line width= 0.6pt,line join=round] ( 28.93,147.34) -- (271.45,147.34);

\path[draw=drawColor,line width= 0.6pt,line join=round] ( 28.93,147.34) -- (271.45,147.34);

\path[draw=drawColor,line width= 0.6pt,line join=round] ( 28.93,147.34) -- (271.45,147.34);

\path[draw=drawColor,line width= 0.6pt,line join=round] ( 28.93,147.34) -- (271.45,147.34);

\path[draw=drawColor,line width= 0.6pt,line join=round] ( 28.93,147.34) -- (271.45,147.34);

\path[draw=drawColor,line width= 0.6pt,line join=round] ( 28.93,147.34) -- (271.45,147.34);

\path[draw=drawColor,line width= 0.6pt,line join=round] ( 28.93,147.34) -- (271.45,147.34);

\path[draw=drawColor,line width= 0.6pt,line join=round] ( 28.93,147.34) -- (271.45,147.34);

\path[draw=drawColor,line width= 0.6pt,line join=round] ( 28.93,147.34) -- (271.45,147.34);

\path[draw=drawColor,line width= 0.6pt,line join=round] ( 28.93,147.34) -- (271.45,147.34);

\path[draw=drawColor,line width= 0.6pt,line join=round] ( 28.93,147.34) -- (271.45,147.34);

\path[draw=drawColor,line width= 0.6pt,line join=round] ( 28.93,147.34) -- (271.45,147.34);

\path[draw=drawColor,line width= 0.6pt,line join=round] ( 28.93,147.34) -- (271.45,147.34);

\path[draw=drawColor,line width= 0.6pt,line join=round] ( 28.93,147.34) -- (271.45,147.34);

\path[draw=drawColor,line width= 0.6pt,line join=round] ( 28.93,147.34) -- (271.45,147.34);

\path[draw=drawColor,line width= 0.6pt,line join=round] ( 28.93,147.34) -- (271.45,147.34);

\path[draw=drawColor,line width= 0.6pt,line join=round] ( 28.93,147.34) -- (271.45,147.34);

\path[draw=drawColor,line width= 0.6pt,line join=round] ( 28.93,147.34) -- (271.45,147.34);

\path[draw=drawColor,line width= 0.6pt,line join=round] ( 28.93,147.34) -- (271.45,147.34);

\path[draw=drawColor,line width= 0.6pt,line join=round] ( 28.93,147.34) -- (271.45,147.34);

\path[draw=drawColor,line width= 0.6pt,line join=round] ( 28.93,147.34) -- (271.45,147.34);

\path[draw=drawColor,line width= 0.6pt,line join=round] ( 28.93,147.34) -- (271.45,147.34);

\path[draw=drawColor,line width= 0.6pt,line join=round] ( 28.93,147.34) -- (271.45,147.34);

\path[draw=drawColor,line width= 0.6pt,line join=round] ( 28.93,147.34) -- (271.45,147.34);

\path[draw=drawColor,line width= 0.6pt,line join=round] ( 28.93,147.34) -- (271.45,147.34);

\path[draw=drawColor,line width= 0.6pt,line join=round] ( 28.93,147.34) -- (271.45,147.34);

\path[draw=drawColor,line width= 0.6pt,line join=round] ( 28.93,147.34) -- (271.45,147.34);

\path[draw=drawColor,line width= 0.6pt,line join=round] ( 28.93,147.34) -- (271.45,147.34);

\path[draw=drawColor,line width= 0.6pt,line join=round] ( 28.93,147.34) -- (271.45,147.34);

\path[draw=drawColor,line width= 0.6pt,line join=round] ( 28.93,147.34) -- (271.45,147.34);

\path[draw=drawColor,line width= 0.6pt,line join=round] ( 28.93,147.34) -- (271.45,147.34);

\path[draw=drawColor,line width= 0.6pt,line join=round] ( 28.93,147.34) -- (271.45,147.34);

\path[draw=drawColor,line width= 0.6pt,line join=round] ( 28.93,147.34) -- (271.45,147.34);

\path[draw=drawColor,line width= 0.6pt,line join=round] ( 28.93,147.34) -- (271.45,147.34);

\path[draw=drawColor,line width= 0.6pt,line join=round] ( 28.93,147.34) -- (271.45,147.34);

\path[draw=drawColor,line width= 0.6pt,line join=round] ( 28.93,147.34) -- (271.45,147.34);

\path[draw=drawColor,line width= 0.6pt,line join=round] ( 28.93,147.34) -- (271.45,147.34);

\path[draw=drawColor,line width= 0.6pt,line join=round] ( 28.93,147.34) -- (271.45,147.34);

\path[draw=drawColor,line width= 0.6pt,line join=round] ( 28.93,147.34) -- (271.45,147.34);

\path[draw=drawColor,line width= 0.6pt,line join=round] ( 28.93,147.34) -- (271.45,147.34);

\path[draw=drawColor,line width= 0.6pt,line join=round] ( 28.93,147.34) -- (271.45,147.34);

\path[draw=drawColor,line width= 0.6pt,line join=round] ( 28.93,147.34) -- (271.45,147.34);

\path[draw=drawColor,line width= 0.6pt,line join=round] ( 28.93,147.34) -- (271.45,147.34);

\path[draw=drawColor,line width= 0.6pt,line join=round] ( 28.93,147.34) -- (271.45,147.34);

\path[draw=drawColor,line width= 0.6pt,line join=round] ( 28.93,147.34) -- (271.45,147.34);

\path[draw=drawColor,line width= 0.6pt,line join=round] ( 28.93,147.34) -- (271.45,147.34);

\path[draw=drawColor,line width= 0.6pt,line join=round] ( 28.93,147.34) -- (271.45,147.34);

\path[draw=drawColor,line width= 0.6pt,line join=round] ( 28.93,147.34) -- (271.45,147.34);

\path[draw=drawColor,line width= 0.6pt,line join=round] ( 28.93,147.34) -- (271.45,147.34);

\path[draw=drawColor,line width= 0.6pt,line join=round] ( 28.93,147.34) -- (271.45,147.34);

\path[draw=drawColor,line width= 0.6pt,line join=round] ( 28.93,147.34) -- (271.45,147.34);

\path[draw=drawColor,line width= 0.6pt,line join=round] ( 28.93,147.34) -- (271.45,147.34);

\path[draw=drawColor,line width= 0.6pt,line join=round] ( 28.93,147.34) -- (271.45,147.34);

\path[draw=drawColor,line width= 0.6pt,line join=round] ( 28.93,147.34) -- (271.45,147.34);

\path[draw=drawColor,line width= 0.6pt,line join=round] ( 28.93,147.34) -- (271.45,147.34);

\path[draw=drawColor,line width= 0.6pt,line join=round] ( 28.93,147.34) -- (271.45,147.34);

\path[draw=drawColor,line width= 0.6pt,line join=round] ( 28.93,147.34) -- (271.45,147.34);

\path[draw=drawColor,line width= 0.6pt,line join=round] ( 28.93,147.34) -- (271.45,147.34);

\path[draw=drawColor,line width= 0.6pt,line join=round] ( 28.93,147.34) -- (271.45,147.34);

\path[draw=drawColor,line width= 0.6pt,line join=round] ( 28.93,147.34) -- (271.45,147.34);

\path[draw=drawColor,line width= 0.6pt,line join=round] ( 28.93,147.34) -- (271.45,147.34);

\path[draw=drawColor,line width= 0.6pt,line join=round] ( 28.93,147.34) -- (271.45,147.34);

\path[draw=drawColor,line width= 0.6pt,line join=round] ( 28.93,147.34) -- (271.45,147.34);

\path[draw=drawColor,line width= 0.6pt,line join=round] ( 28.93,147.34) -- (271.45,147.34);

\path[draw=drawColor,line width= 0.6pt,line join=round] ( 28.93,147.34) -- (271.45,147.34);

\path[draw=drawColor,line width= 0.6pt,line join=round] ( 28.93,147.34) -- (271.45,147.34);

\path[draw=drawColor,line width= 0.6pt,line join=round] ( 28.93,147.34) -- (271.45,147.34);

\path[draw=drawColor,line width= 0.6pt,line join=round] ( 28.93,147.34) -- (271.45,147.34);

\path[draw=drawColor,line width= 0.6pt,line join=round] ( 28.93,147.34) -- (271.45,147.34);

\path[draw=drawColor,line width= 0.6pt,line join=round] ( 28.93,147.34) -- (271.45,147.34);

\path[draw=drawColor,line width= 0.6pt,line join=round] ( 28.93,147.34) -- (271.45,147.34);

\path[draw=drawColor,line width= 0.6pt,line join=round] ( 28.93,147.34) -- (271.45,147.34);

\path[draw=drawColor,line width= 0.6pt,line join=round] ( 28.93,147.34) -- (271.45,147.34);

\path[draw=drawColor,line width= 0.6pt,line join=round] ( 28.93,147.34) -- (271.45,147.34);

\path[draw=drawColor,line width= 0.6pt,line join=round] ( 28.93,147.34) -- (271.45,147.34);

\path[draw=drawColor,line width= 0.6pt,line join=round] ( 28.93,147.34) -- (271.45,147.34);

\path[draw=drawColor,line width= 0.6pt,line join=round] ( 28.93,147.34) -- (271.45,147.34);

\path[draw=drawColor,line width= 0.6pt,line join=round] ( 28.93,147.34) -- (271.45,147.34);

\path[draw=drawColor,line width= 0.6pt,line join=round] ( 28.93,147.34) -- (271.45,147.34);

\path[draw=drawColor,line width= 0.6pt,line join=round] ( 28.93,147.34) -- (271.45,147.34);

\path[draw=drawColor,line width= 0.6pt,line join=round] ( 28.93,147.34) -- (271.45,147.34);

\path[draw=drawColor,line width= 0.6pt,line join=round] ( 28.93,147.34) -- (271.45,147.34);

\path[draw=drawColor,line width= 0.6pt,line join=round] ( 28.93,147.34) -- (271.45,147.34);

\path[draw=drawColor,line width= 0.6pt,line join=round] ( 28.93,147.34) -- (271.45,147.34);

\path[draw=drawColor,line width= 0.6pt,line join=round] ( 28.93,147.34) -- (271.45,147.34);

\path[draw=drawColor,line width= 0.6pt,line join=round] ( 28.93,147.34) -- (271.45,147.34);

\path[draw=drawColor,line width= 0.6pt,line join=round] ( 28.93,147.34) -- (271.45,147.34);

\path[draw=drawColor,line width= 0.6pt,line join=round] ( 28.93,147.34) -- (271.45,147.34);

\path[draw=drawColor,line width= 0.6pt,line join=round] ( 28.93,147.34) -- (271.45,147.34);

\path[draw=drawColor,line width= 0.6pt,line join=round] ( 28.93,147.34) -- (271.45,147.34);

\path[draw=drawColor,line width= 0.6pt,line join=round] ( 28.93,147.34) -- (271.45,147.34);

\path[draw=drawColor,line width= 0.6pt,line join=round] ( 28.93,147.34) -- (271.45,147.34);

\path[draw=drawColor,line width= 0.6pt,line join=round] ( 28.93,147.34) -- (271.45,147.34);

\path[draw=drawColor,line width= 0.6pt,line join=round] ( 28.93,147.34) -- (271.45,147.34);

\path[draw=drawColor,line width= 0.6pt,line join=round] ( 28.93,147.34) -- (271.45,147.34);

\path[draw=drawColor,line width= 0.6pt,line join=round] ( 28.93,147.34) -- (271.45,147.34);

\path[draw=drawColor,line width= 0.6pt,line join=round] ( 28.93,147.34) -- (271.45,147.34);

\path[draw=drawColor,line width= 0.6pt,line join=round] ( 28.93,147.34) -- (271.45,147.34);

\path[draw=drawColor,line width= 0.6pt,line join=round] ( 28.93,147.34) -- (271.45,147.34);

\path[draw=drawColor,line width= 0.6pt,line join=round] ( 28.93,147.34) -- (271.45,147.34);

\path[draw=drawColor,line width= 0.6pt,line join=round] ( 28.93,147.34) -- (271.45,147.34);

\path[draw=drawColor,line width= 0.6pt,line join=round] ( 28.93,147.34) -- (271.45,147.34);

\path[draw=drawColor,line width= 0.6pt,line join=round] ( 28.93,147.34) -- (271.45,147.34);

\path[draw=drawColor,line width= 0.6pt,line join=round] ( 28.93,147.34) -- (271.45,147.34);

\path[draw=drawColor,line width= 0.6pt,line join=round] ( 28.93,147.34) -- (271.45,147.34);

\path[draw=drawColor,line width= 0.6pt,line join=round] ( 28.93,147.34) -- (271.45,147.34);

\path[draw=drawColor,line width= 0.6pt,line join=round] ( 28.93,147.34) -- (271.45,147.34);

\path[draw=drawColor,line width= 0.6pt,line join=round] ( 28.93,147.34) -- (271.45,147.34);

\path[draw=drawColor,line width= 0.6pt,line join=round] ( 28.93,147.34) -- (271.45,147.34);

\path[draw=drawColor,line width= 0.6pt,line join=round] ( 28.93,147.34) -- (271.45,147.34);

\path[draw=drawColor,line width= 0.6pt,line join=round] ( 28.93,147.34) -- (271.45,147.34);

\path[draw=drawColor,line width= 0.6pt,line join=round] ( 28.93,147.34) -- (271.45,147.34);

\path[draw=drawColor,line width= 0.6pt,line join=round] ( 28.93,147.34) -- (271.45,147.34);

\path[draw=drawColor,line width= 0.6pt,line join=round] ( 28.93,147.34) -- (271.45,147.34);

\path[draw=drawColor,line width= 0.6pt,line join=round] ( 28.93,147.34) -- (271.45,147.34);

\path[draw=drawColor,line width= 0.6pt,line join=round] ( 28.93,147.34) -- (271.45,147.34);

\path[draw=drawColor,line width= 0.6pt,line join=round] ( 28.93,147.34) -- (271.45,147.34);

\path[draw=drawColor,line width= 0.6pt,line join=round] ( 28.93,147.34) -- (271.45,147.34);

\path[draw=drawColor,line width= 0.6pt,line join=round] ( 28.93,147.34) -- (271.45,147.34);

\path[draw=drawColor,line width= 0.6pt,line join=round] ( 28.93,147.34) -- (271.45,147.34);

\path[draw=drawColor,line width= 0.6pt,line join=round] ( 28.93,147.34) -- (271.45,147.34);

\path[draw=drawColor,line width= 0.6pt,line join=round] ( 28.93,147.34) -- (271.45,147.34);

\path[draw=drawColor,line width= 0.6pt,line join=round] ( 28.93,147.34) -- (271.45,147.34);

\path[draw=drawColor,line width= 0.6pt,line join=round] ( 28.93,147.34) -- (271.45,147.34);

\path[draw=drawColor,line width= 0.6pt,line join=round] ( 28.93,147.34) -- (271.45,147.34);

\path[draw=drawColor,line width= 0.6pt,line join=round] ( 28.93,147.34) -- (271.45,147.34);

\path[draw=drawColor,line width= 0.6pt,line join=round] ( 28.93,147.34) -- (271.45,147.34);

\path[draw=drawColor,line width= 0.6pt,line join=round] ( 28.93,147.34) -- (271.45,147.34);

\path[draw=drawColor,line width= 0.6pt,line join=round] ( 28.93,147.34) -- (271.45,147.34);

\path[draw=drawColor,line width= 0.6pt,line join=round] ( 28.93,147.34) -- (271.45,147.34);

\path[draw=drawColor,line width= 0.6pt,line join=round] ( 28.93,147.34) -- (271.45,147.34);

\path[draw=drawColor,line width= 0.6pt,line join=round] ( 28.93,147.34) -- (271.45,147.34);

\path[draw=drawColor,line width= 0.6pt,line join=round] ( 28.93,147.34) -- (271.45,147.34);

\path[draw=drawColor,line width= 0.6pt,line join=round] ( 28.93,147.34) -- (271.45,147.34);

\path[draw=drawColor,line width= 0.6pt,line join=round] ( 28.93,147.34) -- (271.45,147.34);

\path[draw=drawColor,line width= 0.6pt,line join=round] ( 28.93,147.34) -- (271.45,147.34);

\path[draw=drawColor,line width= 0.6pt,line join=round] ( 28.93,147.34) -- (271.45,147.34);

\path[draw=drawColor,line width= 0.6pt,line join=round] ( 28.93,147.34) -- (271.45,147.34);

\path[draw=drawColor,line width= 0.6pt,line join=round] ( 28.93,147.34) -- (271.45,147.34);

\path[draw=drawColor,line width= 0.6pt,line join=round] ( 28.93,147.34) -- (271.45,147.34);

\path[draw=drawColor,line width= 0.6pt,line join=round] ( 28.93,147.34) -- (271.45,147.34);

\path[draw=drawColor,line width= 0.6pt,line join=round] ( 28.93,147.34) -- (271.45,147.34);

\path[draw=drawColor,line width= 0.6pt,line join=round] ( 28.93,147.34) -- (271.45,147.34);

\path[draw=drawColor,line width= 0.6pt,line join=round] ( 28.93,147.34) -- (271.45,147.34);

\path[draw=drawColor,line width= 0.6pt,line join=round] ( 28.93,147.34) -- (271.45,147.34);

\path[draw=drawColor,line width= 0.6pt,line join=round] ( 28.93,147.34) -- (271.45,147.34);

\path[draw=drawColor,line width= 0.6pt,line join=round] ( 28.93,147.34) -- (271.45,147.34);

\path[draw=drawColor,line width= 0.6pt,line join=round] ( 28.93,147.34) -- (271.45,147.34);

\path[draw=drawColor,line width= 0.6pt,line join=round] ( 28.93,147.34) -- (271.45,147.34);

\path[draw=drawColor,line width= 0.6pt,line join=round] ( 28.93,147.34) -- (271.45,147.34);

\path[draw=drawColor,line width= 0.6pt,line join=round] ( 28.93,147.34) -- (271.45,147.34);

\path[draw=drawColor,line width= 0.6pt,line join=round] ( 28.93,147.34) -- (271.45,147.34);

\path[draw=drawColor,line width= 0.6pt,line join=round] ( 28.93,147.34) -- (271.45,147.34);

\path[draw=drawColor,line width= 0.6pt,line join=round] ( 28.93,147.34) -- (271.45,147.34);

\path[draw=drawColor,line width= 0.6pt,line join=round] ( 28.93,147.34) -- (271.45,147.34);

\path[draw=drawColor,line width= 0.6pt,line join=round] ( 28.93,147.34) -- (271.45,147.34);

\path[draw=drawColor,line width= 0.6pt,line join=round] ( 28.93,147.34) -- (271.45,147.34);

\path[draw=drawColor,line width= 0.6pt,line join=round] ( 28.93,147.34) -- (271.45,147.34);

\path[draw=drawColor,line width= 0.6pt,line join=round] ( 28.93,147.34) -- (271.45,147.34);

\path[draw=drawColor,line width= 0.6pt,line join=round] ( 28.93,147.34) -- (271.45,147.34);

\path[draw=drawColor,line width= 0.6pt,line join=round] ( 28.93,147.34) -- (271.45,147.34);

\path[draw=drawColor,line width= 0.6pt,line join=round] ( 28.93,147.34) -- (271.45,147.34);

\path[draw=drawColor,line width= 0.6pt,line join=round] ( 28.93,147.34) -- (271.45,147.34);

\path[draw=drawColor,line width= 0.6pt,line join=round] ( 28.93,147.34) -- (271.45,147.34);

\path[draw=drawColor,line width= 0.6pt,line join=round] ( 28.93,147.34) -- (271.45,147.34);

\path[draw=drawColor,line width= 0.6pt,line join=round] ( 28.93,147.34) -- (271.45,147.34);

\path[draw=drawColor,line width= 0.6pt,line join=round] ( 28.93,147.34) -- (271.45,147.34);

\path[draw=drawColor,line width= 0.6pt,line join=round] ( 28.93,147.34) -- (271.45,147.34);

\path[draw=drawColor,line width= 0.6pt,line join=round] ( 28.93,147.34) -- (271.45,147.34);

\path[draw=drawColor,line width= 0.6pt,line join=round] ( 28.93,147.34) -- (271.45,147.34);

\path[draw=drawColor,line width= 0.6pt,line join=round] ( 28.93,147.34) -- (271.45,147.34);

\path[draw=drawColor,line width= 0.6pt,line join=round] ( 28.93,147.34) -- (271.45,147.34);

\path[draw=drawColor,line width= 0.6pt,line join=round] ( 28.93,147.34) -- (271.45,147.34);

\path[draw=drawColor,line width= 0.6pt,line join=round] ( 28.93,147.34) -- (271.45,147.34);

\path[draw=drawColor,line width= 0.6pt,line join=round] ( 28.93,147.34) -- (271.45,147.34);

\path[draw=drawColor,line width= 0.6pt,line join=round] ( 28.93,147.34) -- (271.45,147.34);

\path[draw=drawColor,line width= 0.6pt,line join=round] ( 28.93,147.34) -- (271.45,147.34);

\path[draw=drawColor,line width= 0.6pt,line join=round] ( 28.93,147.34) -- (271.45,147.34);

\path[draw=drawColor,line width= 0.6pt,line join=round] ( 28.93,147.34) -- (271.45,147.34);

\path[draw=drawColor,line width= 0.6pt,line join=round] ( 28.93,147.34) -- (271.45,147.34);

\path[draw=drawColor,line width= 0.6pt,line join=round] ( 28.93,147.34) -- (271.45,147.34);

\path[draw=drawColor,line width= 0.6pt,line join=round] ( 28.93,147.34) -- (271.45,147.34);

\path[draw=drawColor,line width= 0.6pt,line join=round] ( 28.93,147.34) -- (271.45,147.34);

\path[draw=drawColor,line width= 0.6pt,line join=round] ( 28.93,147.34) -- (271.45,147.34);

\path[draw=drawColor,line width= 0.6pt,line join=round] ( 28.93,147.34) -- (271.45,147.34);

\path[draw=drawColor,line width= 0.6pt,line join=round] ( 28.93,147.34) -- (271.45,147.34);

\path[draw=drawColor,line width= 0.6pt,line join=round] ( 28.93,147.34) -- (271.45,147.34);

\path[draw=drawColor,line width= 0.6pt,line join=round] ( 28.93,147.34) -- (271.45,147.34);

\path[draw=drawColor,line width= 0.6pt,line join=round] ( 28.93,147.34) -- (271.45,147.34);

\path[draw=drawColor,line width= 0.6pt,line join=round] ( 28.93,147.34) -- (271.45,147.34);

\path[draw=drawColor,line width= 0.6pt,line join=round] ( 28.93,147.34) -- (271.45,147.34);

\path[draw=drawColor,line width= 0.6pt,line join=round] ( 28.93,147.34) -- (271.45,147.34);

\path[draw=drawColor,line width= 0.6pt,line join=round] ( 28.93,147.34) -- (271.45,147.34);

\path[draw=drawColor,line width= 0.6pt,line join=round] ( 28.93,147.34) -- (271.45,147.34);

\path[draw=drawColor,line width= 0.6pt,line join=round] ( 28.93,147.34) -- (271.45,147.34);

\path[draw=drawColor,line width= 0.6pt,line join=round] ( 28.93,147.34) -- (271.45,147.34);

\path[draw=drawColor,line width= 0.6pt,line join=round] ( 28.93,147.34) -- (271.45,147.34);

\path[draw=drawColor,line width= 0.6pt,line join=round] ( 28.93,147.34) -- (271.45,147.34);

\path[draw=drawColor,line width= 0.6pt,line join=round] ( 28.93,147.34) -- (271.45,147.34);

\path[draw=drawColor,line width= 0.6pt,line join=round] ( 28.93,147.34) -- (271.45,147.34);

\path[draw=drawColor,line width= 0.6pt,line join=round] ( 28.93,147.34) -- (271.45,147.34);

\path[draw=drawColor,line width= 0.6pt,line join=round] ( 28.93,147.34) -- (271.45,147.34);

\path[draw=drawColor,line width= 0.6pt,line join=round] ( 28.93,147.34) -- (271.45,147.34);

\path[draw=drawColor,line width= 0.6pt,line join=round] ( 28.93,147.34) -- (271.45,147.34);

\path[draw=drawColor,line width= 0.6pt,line join=round] ( 28.93,147.34) -- (271.45,147.34);

\path[draw=drawColor,line width= 0.6pt,line join=round] ( 28.93,147.34) -- (271.45,147.34);

\path[draw=drawColor,line width= 0.6pt,line join=round] ( 28.93,147.34) -- (271.45,147.34);

\path[draw=drawColor,line width= 0.6pt,line join=round] ( 28.93,147.34) -- (271.45,147.34);

\path[draw=drawColor,line width= 0.6pt,line join=round] ( 28.93,147.34) -- (271.45,147.34);

\path[draw=drawColor,line width= 0.6pt,line join=round] ( 28.93,147.34) -- (271.45,147.34);

\path[draw=drawColor,line width= 0.6pt,line join=round] ( 28.93,147.34) -- (271.45,147.34);

\path[draw=drawColor,line width= 0.6pt,line join=round] ( 28.93,147.34) -- (271.45,147.34);

\path[draw=drawColor,line width= 0.6pt,line join=round] ( 28.93,147.34) -- (271.45,147.34);

\path[draw=drawColor,line width= 0.6pt,line join=round] ( 28.93,147.34) -- (271.45,147.34);

\path[draw=drawColor,line width= 0.6pt,line join=round] ( 28.93,147.34) -- (271.45,147.34);

\path[draw=drawColor,line width= 0.6pt,line join=round] ( 28.93,147.34) -- (271.45,147.34);

\path[draw=drawColor,line width= 0.6pt,line join=round] ( 28.93,147.34) -- (271.45,147.34);

\path[draw=drawColor,line width= 0.6pt,line join=round] ( 28.93,147.34) -- (271.45,147.34);

\path[draw=drawColor,line width= 0.6pt,line join=round] ( 28.93,147.34) -- (271.45,147.34);

\path[draw=drawColor,line width= 0.6pt,line join=round] ( 28.93,147.34) -- (271.45,147.34);

\path[draw=drawColor,line width= 0.6pt,line join=round] ( 28.93,147.34) -- (271.45,147.34);

\path[draw=drawColor,line width= 0.6pt,line join=round] ( 28.93,147.34) -- (271.45,147.34);

\path[draw=drawColor,line width= 0.6pt,line join=round] ( 28.93,147.34) -- (271.45,147.34);

\path[draw=drawColor,line width= 0.6pt,line join=round] ( 28.93,147.34) -- (271.45,147.34);

\path[draw=drawColor,line width= 0.6pt,line join=round] ( 28.93,147.34) -- (271.45,147.34);

\path[draw=drawColor,line width= 0.6pt,line join=round] ( 28.93,147.34) -- (271.45,147.34);

\path[draw=drawColor,line width= 0.6pt,line join=round] ( 28.93,147.34) -- (271.45,147.34);

\path[draw=drawColor,line width= 0.6pt,line join=round] ( 28.93,147.34) -- (271.45,147.34);

\path[draw=drawColor,line width= 0.6pt,line join=round] ( 28.93,147.34) -- (271.45,147.34);

\path[draw=drawColor,line width= 0.6pt,line join=round] ( 28.93,147.34) -- (271.45,147.34);

\path[draw=drawColor,line width= 0.6pt,line join=round] ( 28.93,147.34) -- (271.45,147.34);

\path[draw=drawColor,line width= 0.6pt,line join=round] ( 28.93,147.34) -- (271.45,147.34);

\path[draw=drawColor,line width= 0.6pt,line join=round] ( 28.93,147.34) -- (271.45,147.34);

\path[draw=drawColor,line width= 0.6pt,line join=round] ( 28.93,147.34) -- (271.45,147.34);

\path[draw=drawColor,line width= 0.6pt,line join=round] ( 28.93,147.34) -- (271.45,147.34);

\path[draw=drawColor,line width= 0.6pt,line join=round] ( 28.93,147.34) -- (271.45,147.34);

\path[draw=drawColor,line width= 0.6pt,line join=round] ( 28.93,147.34) -- (271.45,147.34);

\path[draw=drawColor,line width= 0.6pt,line join=round] ( 28.93,147.34) -- (271.45,147.34);

\path[draw=drawColor,line width= 0.6pt,line join=round] ( 28.93,147.34) -- (271.45,147.34);

\path[draw=drawColor,line width= 0.6pt,line join=round] ( 28.93,147.34) -- (271.45,147.34);

\path[draw=drawColor,line width= 0.6pt,line join=round] ( 28.93,147.34) -- (271.45,147.34);

\path[draw=drawColor,line width= 0.6pt,line join=round] ( 28.93,147.34) -- (271.45,147.34);

\path[draw=drawColor,line width= 0.6pt,line join=round] ( 28.93,147.34) -- (271.45,147.34);

\path[draw=drawColor,line width= 0.6pt,line join=round] ( 28.93,147.34) -- (271.45,147.34);

\path[draw=drawColor,line width= 0.6pt,line join=round] ( 28.93,147.34) -- (271.45,147.34);

\path[draw=drawColor,line width= 0.6pt,line join=round] ( 28.93,147.34) -- (271.45,147.34);

\path[draw=drawColor,line width= 0.6pt,line join=round] ( 28.93,147.34) -- (271.45,147.34);

\path[draw=drawColor,line width= 0.6pt,line join=round] ( 28.93,147.34) -- (271.45,147.34);

\path[draw=drawColor,line width= 0.6pt,line join=round] ( 28.93,147.34) -- (271.45,147.34);

\path[draw=drawColor,line width= 0.6pt,line join=round] ( 28.93,147.34) -- (271.45,147.34);

\path[draw=drawColor,line width= 0.6pt,line join=round] ( 28.93,147.34) -- (271.45,147.34);

\path[draw=drawColor,line width= 0.6pt,line join=round] ( 28.93,147.34) -- (271.45,147.34);

\path[draw=drawColor,line width= 0.6pt,line join=round] ( 28.93,147.34) -- (271.45,147.34);

\path[draw=drawColor,line width= 0.6pt,line join=round] ( 28.93,147.34) -- (271.45,147.34);

\path[draw=drawColor,line width= 0.6pt,line join=round] ( 28.93,147.34) -- (271.45,147.34);

\path[draw=drawColor,line width= 0.6pt,line join=round] ( 28.93,147.34) -- (271.45,147.34);

\path[draw=drawColor,line width= 0.6pt,line join=round] ( 28.93,147.34) -- (271.45,147.34);

\path[draw=drawColor,line width= 0.6pt,line join=round] ( 28.93,147.34) -- (271.45,147.34);

\path[draw=drawColor,line width= 0.6pt,line join=round] ( 28.93,147.34) -- (271.45,147.34);

\path[draw=drawColor,line width= 0.6pt,line join=round] ( 28.93,147.34) -- (271.45,147.34);

\path[draw=drawColor,line width= 0.6pt,line join=round] ( 28.93,147.34) -- (271.45,147.34);

\path[draw=drawColor,line width= 0.6pt,line join=round] ( 28.93,147.34) -- (271.45,147.34);

\path[draw=drawColor,line width= 0.6pt,line join=round] ( 28.93,147.34) -- (271.45,147.34);

\path[draw=drawColor,line width= 0.6pt,line join=round] ( 28.93,147.34) -- (271.45,147.34);

\path[draw=drawColor,line width= 0.6pt,line join=round] ( 28.93,147.34) -- (271.45,147.34);

\path[draw=drawColor,line width= 0.6pt,line join=round] ( 28.93,147.34) -- (271.45,147.34);

\path[draw=drawColor,line width= 0.6pt,line join=round] ( 28.93,147.34) -- (271.45,147.34);

\path[draw=drawColor,line width= 0.6pt,line join=round] ( 28.93,147.34) -- (271.45,147.34);

\path[draw=drawColor,line width= 0.6pt,line join=round] ( 28.93,147.34) -- (271.45,147.34);

\path[draw=drawColor,line width= 0.6pt,line join=round] ( 28.93,147.34) -- (271.45,147.34);

\path[draw=drawColor,line width= 0.6pt,line join=round] ( 28.93,147.34) -- (271.45,147.34);

\path[draw=drawColor,line width= 0.6pt,line join=round] ( 28.93,147.34) -- (271.45,147.34);

\path[draw=drawColor,line width= 0.6pt,line join=round] ( 28.93,147.34) -- (271.45,147.34);

\path[draw=drawColor,line width= 0.6pt,line join=round] ( 28.93,147.34) -- (271.45,147.34);

\path[draw=drawColor,line width= 0.6pt,line join=round] ( 28.93,147.34) -- (271.45,147.34);

\path[draw=drawColor,line width= 0.6pt,line join=round] ( 28.93,147.34) -- (271.45,147.34);

\path[draw=drawColor,line width= 0.6pt,line join=round] ( 28.93,147.34) -- (271.45,147.34);

\path[draw=drawColor,line width= 0.6pt,line join=round] ( 28.93,147.34) -- (271.45,147.34);

\path[draw=drawColor,line width= 0.6pt,line join=round] ( 28.93,147.34) -- (271.45,147.34);

\path[draw=drawColor,line width= 0.6pt,line join=round] ( 28.93,147.34) -- (271.45,147.34);

\path[draw=drawColor,line width= 0.6pt,line join=round] ( 28.93,147.34) -- (271.45,147.34);

\path[draw=drawColor,line width= 0.6pt,line join=round] ( 28.93,147.34) -- (271.45,147.34);

\path[draw=drawColor,line width= 0.6pt,line join=round] ( 28.93,147.34) -- (271.45,147.34);

\path[draw=drawColor,line width= 0.6pt,line join=round] ( 28.93,147.34) -- (271.45,147.34);

\path[draw=drawColor,line width= 0.6pt,line join=round] ( 28.93,147.34) -- (271.45,147.34);

\path[draw=drawColor,line width= 0.6pt,line join=round] ( 28.93,147.34) -- (271.45,147.34);

\path[draw=drawColor,line width= 0.6pt,line join=round] ( 28.93,147.34) -- (271.45,147.34);

\path[draw=drawColor,line width= 0.6pt,line join=round] ( 28.93,147.34) -- (271.45,147.34);

\path[draw=drawColor,line width= 0.6pt,line join=round] ( 28.93,147.34) -- (271.45,147.34);

\path[draw=drawColor,line width= 0.6pt,line join=round] ( 28.93,147.34) -- (271.45,147.34);

\path[draw=drawColor,line width= 0.6pt,line join=round] ( 28.93,147.34) -- (271.45,147.34);

\path[draw=drawColor,line width= 0.6pt,line join=round] ( 28.93,147.34) -- (271.45,147.34);

\path[draw=drawColor,line width= 0.6pt,line join=round] ( 28.93,147.34) -- (271.45,147.34);

\path[draw=drawColor,line width= 0.6pt,line join=round] ( 28.93,147.34) -- (271.45,147.34);

\path[draw=drawColor,line width= 0.6pt,line join=round] ( 28.93,147.34) -- (271.45,147.34);

\path[draw=drawColor,line width= 0.6pt,line join=round] ( 28.93,147.34) -- (271.45,147.34);

\path[draw=drawColor,line width= 0.6pt,line join=round] ( 28.93,147.34) -- (271.45,147.34);

\path[draw=drawColor,line width= 0.6pt,line join=round] ( 28.93,147.34) -- (271.45,147.34);

\path[draw=drawColor,line width= 0.6pt,line join=round] ( 28.93,147.34) -- (271.45,147.34);

\path[draw=drawColor,line width= 0.6pt,line join=round] ( 28.93,147.34) -- (271.45,147.34);

\path[draw=drawColor,line width= 0.6pt,line join=round] ( 28.93,147.34) -- (271.45,147.34);

\path[draw=drawColor,line width= 0.6pt,line join=round] ( 28.93,147.34) -- (271.45,147.34);

\path[draw=drawColor,line width= 0.6pt,line join=round] ( 28.93,147.34) -- (271.45,147.34);

\path[draw=drawColor,line width= 0.6pt,line join=round] ( 28.93,147.34) -- (271.45,147.34);

\path[draw=drawColor,line width= 0.6pt,line join=round] ( 28.93,147.34) -- (271.45,147.34);

\path[draw=drawColor,line width= 0.6pt,line join=round] ( 28.93,147.34) -- (271.45,147.34);

\path[draw=drawColor,line width= 0.6pt,line join=round] ( 28.93,147.34) -- (271.45,147.34);

\path[draw=drawColor,line width= 0.6pt,line join=round] ( 28.93,147.34) -- (271.45,147.34);

\path[draw=drawColor,line width= 0.6pt,line join=round] ( 28.93,147.34) -- (271.45,147.34);

\path[draw=drawColor,line width= 0.6pt,line join=round] ( 28.93,147.34) -- (271.45,147.34);

\path[draw=drawColor,line width= 0.6pt,line join=round] ( 28.93,147.34) -- (271.45,147.34);

\path[draw=drawColor,line width= 0.6pt,line join=round] ( 28.93,147.34) -- (271.45,147.34);

\path[draw=drawColor,line width= 0.6pt,line join=round] ( 28.93,147.34) -- (271.45,147.34);

\path[draw=drawColor,line width= 0.6pt,line join=round] ( 28.93,147.34) -- (271.45,147.34);

\path[draw=drawColor,line width= 0.6pt,line join=round] ( 28.93,147.34) -- (271.45,147.34);

\path[draw=drawColor,line width= 0.6pt,line join=round] ( 28.93,147.34) -- (271.45,147.34);

\path[draw=drawColor,line width= 0.6pt,line join=round] ( 28.93,147.34) -- (271.45,147.34);

\path[draw=drawColor,line width= 0.6pt,line join=round] ( 28.93,147.34) -- (271.45,147.34);

\path[draw=drawColor,line width= 0.6pt,line join=round] ( 28.93,147.34) -- (271.45,147.34);

\path[draw=drawColor,line width= 0.6pt,line join=round] ( 28.93,147.34) -- (271.45,147.34);

\path[draw=drawColor,line width= 0.6pt,line join=round] ( 28.93,147.34) -- (271.45,147.34);

\path[draw=drawColor,line width= 0.6pt,line join=round] ( 28.93,147.34) -- (271.45,147.34);

\path[draw=drawColor,line width= 0.6pt,line join=round] ( 28.93,147.34) -- (271.45,147.34);

\path[draw=drawColor,line width= 0.6pt,line join=round] ( 28.93,147.34) -- (271.45,147.34);

\path[draw=drawColor,line width= 0.6pt,line join=round] ( 28.93,147.34) -- (271.45,147.34);

\path[draw=drawColor,line width= 0.6pt,line join=round] ( 28.93,147.34) -- (271.45,147.34);

\path[draw=drawColor,line width= 0.6pt,line join=round] ( 28.93,147.34) -- (271.45,147.34);

\path[draw=drawColor,line width= 0.6pt,line join=round] ( 28.93,147.34) -- (271.45,147.34);

\path[draw=drawColor,line width= 0.6pt,line join=round] ( 28.93,147.34) -- (271.45,147.34);

\path[draw=drawColor,line width= 0.6pt,line join=round] ( 28.93,147.34) -- (271.45,147.34);

\path[draw=drawColor,line width= 0.6pt,line join=round] ( 28.93,147.34) -- (271.45,147.34);

\path[draw=drawColor,line width= 0.6pt,line join=round] ( 28.93,147.34) -- (271.45,147.34);

\path[draw=drawColor,line width= 0.6pt,line join=round] ( 28.93,147.34) -- (271.45,147.34);

\path[draw=drawColor,line width= 0.6pt,line join=round] ( 28.93,147.34) -- (271.45,147.34);

\path[draw=drawColor,line width= 0.6pt,line join=round] ( 28.93,147.34) -- (271.45,147.34);

\path[draw=drawColor,line width= 0.6pt,line join=round] ( 28.93,147.34) -- (271.45,147.34);

\path[draw=drawColor,line width= 0.6pt,line join=round] ( 28.93,147.34) -- (271.45,147.34);

\path[draw=drawColor,line width= 0.6pt,line join=round] ( 28.93,147.34) -- (271.45,147.34);

\path[draw=drawColor,line width= 0.6pt,line join=round] ( 28.93,147.34) -- (271.45,147.34);

\path[draw=drawColor,line width= 0.6pt,line join=round] ( 28.93,147.34) -- (271.45,147.34);

\path[draw=drawColor,line width= 0.6pt,line join=round] ( 28.93,147.34) -- (271.45,147.34);

\path[draw=drawColor,line width= 0.6pt,line join=round] ( 28.93,147.34) -- (271.45,147.34);

\path[draw=drawColor,line width= 0.6pt,line join=round] ( 28.93,147.34) -- (271.45,147.34);

\path[draw=drawColor,line width= 0.6pt,line join=round] ( 28.93,147.34) -- (271.45,147.34);

\path[draw=drawColor,line width= 0.6pt,line join=round] ( 28.93,147.34) -- (271.45,147.34);

\path[draw=drawColor,line width= 0.6pt,line join=round] ( 28.93,147.34) -- (271.45,147.34);

\path[draw=drawColor,line width= 0.6pt,line join=round] ( 28.93,147.34) -- (271.45,147.34);

\path[draw=drawColor,line width= 0.6pt,line join=round] ( 28.93,147.34) -- (271.45,147.34);

\path[draw=drawColor,line width= 0.6pt,line join=round] ( 28.93,147.34) -- (271.45,147.34);

\path[draw=drawColor,line width= 0.6pt,line join=round] ( 28.93,147.34) -- (271.45,147.34);

\path[draw=drawColor,line width= 0.6pt,line join=round] ( 28.93,147.34) -- (271.45,147.34);

\path[draw=drawColor,line width= 0.6pt,line join=round] ( 28.93,147.34) -- (271.45,147.34);

\path[draw=drawColor,line width= 0.6pt,line join=round] ( 28.93,147.34) -- (271.45,147.34);

\path[draw=drawColor,line width= 0.6pt,line join=round] ( 28.93,147.34) -- (271.45,147.34);

\path[draw=drawColor,line width= 0.6pt,line join=round] ( 28.93,147.34) -- (271.45,147.34);

\path[draw=drawColor,line width= 0.6pt,line join=round] ( 28.93,147.34) -- (271.45,147.34);

\path[draw=drawColor,line width= 0.6pt,line join=round] ( 28.93,147.34) -- (271.45,147.34);

\path[draw=drawColor,line width= 0.6pt,line join=round] ( 28.93,147.34) -- (271.45,147.34);

\path[draw=drawColor,line width= 0.6pt,line join=round] ( 28.93,147.34) -- (271.45,147.34);

\path[draw=drawColor,line width= 0.6pt,line join=round] ( 28.93,147.34) -- (271.45,147.34);

\path[draw=drawColor,line width= 0.6pt,line join=round] ( 28.93,147.34) -- (271.45,147.34);

\path[draw=drawColor,line width= 0.6pt,line join=round] ( 28.93,147.34) -- (271.45,147.34);

\path[draw=drawColor,line width= 0.6pt,line join=round] ( 28.93,147.34) -- (271.45,147.34);

\path[draw=drawColor,line width= 0.6pt,line join=round] ( 28.93,147.34) -- (271.45,147.34);

\path[draw=drawColor,line width= 0.6pt,line join=round] ( 28.93,147.34) -- (271.45,147.34);

\path[draw=drawColor,line width= 0.6pt,line join=round] ( 28.93,147.34) -- (271.45,147.34);

\path[draw=drawColor,line width= 0.6pt,line join=round] ( 28.93,147.34) -- (271.45,147.34);

\path[draw=drawColor,line width= 0.6pt,line join=round] ( 28.93,147.34) -- (271.45,147.34);

\path[draw=drawColor,line width= 0.6pt,line join=round] ( 28.93,147.34) -- (271.45,147.34);

\path[draw=drawColor,line width= 0.6pt,line join=round] ( 28.93,147.34) -- (271.45,147.34);

\path[draw=drawColor,line width= 0.6pt,line join=round] ( 28.93,147.34) -- (271.45,147.34);

\path[draw=drawColor,line width= 0.6pt,line join=round] ( 28.93,147.34) -- (271.45,147.34);

\path[draw=drawColor,line width= 0.6pt,line join=round] ( 28.93,147.34) -- (271.45,147.34);

\path[draw=drawColor,line width= 0.6pt,line join=round] ( 28.93,147.34) -- (271.45,147.34);

\path[draw=drawColor,line width= 0.6pt,line join=round] ( 28.93,147.34) -- (271.45,147.34);

\path[draw=drawColor,line width= 0.6pt,line join=round] ( 28.93,147.34) -- (271.45,147.34);

\path[draw=drawColor,line width= 0.6pt,line join=round] ( 28.93,147.34) -- (271.45,147.34);

\path[draw=drawColor,line width= 0.6pt,line join=round] ( 28.93,147.34) -- (271.45,147.34);

\path[draw=drawColor,line width= 0.6pt,line join=round] ( 28.93,147.34) -- (271.45,147.34);

\path[draw=drawColor,line width= 0.6pt,line join=round] ( 28.93,147.34) -- (271.45,147.34);

\path[draw=drawColor,line width= 0.6pt,line join=round] ( 28.93,147.34) -- (271.45,147.34);

\path[draw=drawColor,line width= 0.6pt,line join=round] ( 28.93,147.34) -- (271.45,147.34);

\path[draw=drawColor,line width= 0.6pt,line join=round] ( 28.93,147.34) -- (271.45,147.34);

\path[draw=drawColor,line width= 0.6pt,line join=round] ( 28.93,147.34) -- (271.45,147.34);

\path[draw=drawColor,line width= 0.6pt,line join=round] ( 28.93,147.34) -- (271.45,147.34);

\path[draw=drawColor,line width= 0.6pt,line join=round] ( 28.93,147.34) -- (271.45,147.34);

\path[draw=drawColor,line width= 0.6pt,line join=round] ( 28.93,147.34) -- (271.45,147.34);

\path[draw=drawColor,line width= 0.6pt,line join=round] ( 28.93,147.34) -- (271.45,147.34);

\path[draw=drawColor,line width= 0.6pt,line join=round] ( 28.93,147.34) -- (271.45,147.34);

\path[draw=drawColor,line width= 0.6pt,line join=round] ( 28.93,147.34) -- (271.45,147.34);

\path[draw=drawColor,line width= 0.6pt,line join=round] ( 28.93,147.34) -- (271.45,147.34);

\path[draw=drawColor,line width= 0.6pt,line join=round] ( 28.93,147.34) -- (271.45,147.34);

\path[draw=drawColor,line width= 0.6pt,line join=round] ( 28.93,147.34) -- (271.45,147.34);

\path[draw=drawColor,line width= 0.6pt,line join=round] ( 28.93,147.34) -- (271.45,147.34);

\path[draw=drawColor,line width= 0.6pt,line join=round] ( 28.93,147.34) -- (271.45,147.34);

\path[draw=drawColor,line width= 0.6pt,line join=round] ( 28.93,147.34) -- (271.45,147.34);

\path[draw=drawColor,line width= 0.6pt,line join=round] ( 28.93,147.34) -- (271.45,147.34);

\path[draw=drawColor,line width= 0.6pt,line join=round] ( 28.93,147.34) -- (271.45,147.34);

\path[draw=drawColor,line width= 0.6pt,line join=round] ( 28.93,147.34) -- (271.45,147.34);

\path[draw=drawColor,line width= 0.6pt,line join=round] ( 28.93,147.34) -- (271.45,147.34);

\path[draw=drawColor,line width= 0.6pt,line join=round] ( 28.93,147.34) -- (271.45,147.34);

\path[draw=drawColor,line width= 0.6pt,line join=round] ( 28.93,147.34) -- (271.45,147.34);

\path[draw=drawColor,line width= 0.6pt,line join=round] ( 28.93,147.34) -- (271.45,147.34);

\path[draw=drawColor,line width= 0.6pt,line join=round] ( 28.93,147.34) -- (271.45,147.34);

\path[draw=drawColor,line width= 0.6pt,line join=round] ( 28.93,147.34) -- (271.45,147.34);

\path[draw=drawColor,line width= 0.6pt,line join=round] ( 28.93,147.34) -- (271.45,147.34);

\path[draw=drawColor,line width= 0.6pt,line join=round] ( 28.93,147.34) -- (271.45,147.34);

\path[draw=drawColor,line width= 0.6pt,line join=round] ( 28.93,147.34) -- (271.45,147.34);

\path[draw=drawColor,line width= 0.6pt,line join=round] ( 28.93,147.34) -- (271.45,147.34);

\path[draw=drawColor,line width= 0.6pt,line join=round] ( 28.93,147.34) -- (271.45,147.34);

\path[draw=drawColor,line width= 0.6pt,line join=round] ( 28.93,147.34) -- (271.45,147.34);

\path[draw=drawColor,line width= 0.6pt,line join=round] ( 28.93,147.34) -- (271.45,147.34);

\path[draw=drawColor,line width= 0.6pt,line join=round] ( 28.93,147.34) -- (271.45,147.34);

\path[draw=drawColor,line width= 0.6pt,line join=round] ( 28.93,147.34) -- (271.45,147.34);

\path[draw=drawColor,line width= 0.6pt,line join=round] ( 28.93,147.34) -- (271.45,147.34);

\path[draw=drawColor,line width= 0.6pt,line join=round] ( 28.93,147.34) -- (271.45,147.34);

\path[draw=drawColor,line width= 0.6pt,line join=round] ( 28.93,147.34) -- (271.45,147.34);

\path[draw=drawColor,line width= 0.6pt,line join=round] ( 28.93,147.34) -- (271.45,147.34);

\path[draw=drawColor,line width= 0.6pt,line join=round] ( 28.93,147.34) -- (271.45,147.34);

\path[draw=drawColor,line width= 0.6pt,line join=round] ( 28.93,147.34) -- (271.45,147.34);

\path[draw=drawColor,line width= 0.6pt,line join=round] ( 28.93,147.34) -- (271.45,147.34);

\path[draw=drawColor,line width= 0.6pt,line join=round] ( 28.93,147.34) -- (271.45,147.34);

\path[draw=drawColor,line width= 0.6pt,line join=round] ( 28.93,147.34) -- (271.45,147.34);

\path[draw=drawColor,line width= 0.6pt,line join=round] ( 28.93,147.34) -- (271.45,147.34);

\path[draw=drawColor,line width= 0.6pt,line join=round] ( 28.93,147.34) -- (271.45,147.34);

\path[draw=drawColor,line width= 0.6pt,line join=round] ( 28.93,147.34) -- (271.45,147.34);

\path[draw=drawColor,line width= 0.6pt,line join=round] ( 28.93,147.34) -- (271.45,147.34);

\path[draw=drawColor,line width= 0.6pt,line join=round] ( 28.93,147.34) -- (271.45,147.34);

\path[draw=drawColor,line width= 0.6pt,line join=round] ( 28.93,147.34) -- (271.45,147.34);

\path[draw=drawColor,line width= 0.6pt,line join=round] ( 28.93,147.34) -- (271.45,147.34);

\path[draw=drawColor,line width= 0.6pt,line join=round] ( 28.93,147.34) -- (271.45,147.34);

\path[draw=drawColor,line width= 0.6pt,line join=round] ( 28.93,147.34) -- (271.45,147.34);

\path[draw=drawColor,line width= 0.6pt,line join=round] ( 28.93,147.34) -- (271.45,147.34);

\path[draw=drawColor,line width= 0.6pt,line join=round] ( 28.93,147.34) -- (271.45,147.34);

\path[draw=drawColor,line width= 0.6pt,line join=round] ( 28.93,147.34) -- (271.45,147.34);

\path[draw=drawColor,line width= 0.6pt,line join=round] ( 28.93,147.34) -- (271.45,147.34);

\path[draw=drawColor,line width= 0.6pt,line join=round] ( 28.93,147.34) -- (271.45,147.34);

\path[draw=drawColor,line width= 0.6pt,line join=round] ( 28.93,147.34) -- (271.45,147.34);

\path[draw=drawColor,line width= 0.6pt,line join=round] ( 28.93,147.34) -- (271.45,147.34);

\path[draw=drawColor,line width= 0.6pt,line join=round] ( 28.93,147.34) -- (271.45,147.34);

\path[draw=drawColor,line width= 0.6pt,line join=round] ( 28.93,147.34) -- (271.45,147.34);

\path[draw=drawColor,line width= 0.6pt,line join=round] ( 28.93,147.34) -- (271.45,147.34);

\path[draw=drawColor,line width= 0.6pt,line join=round] ( 28.93,147.34) -- (271.45,147.34);

\path[draw=drawColor,line width= 0.6pt,line join=round] ( 28.93,147.34) -- (271.45,147.34);

\path[draw=drawColor,line width= 0.6pt,line join=round] ( 28.93,147.34) -- (271.45,147.34);

\path[draw=drawColor,line width= 0.6pt,line join=round] ( 28.93,147.34) -- (271.45,147.34);

\path[draw=drawColor,line width= 0.6pt,line join=round] ( 28.93,147.34) -- (271.45,147.34);

\path[draw=drawColor,line width= 0.6pt,line join=round] ( 28.93,147.34) -- (271.45,147.34);

\path[draw=drawColor,line width= 0.6pt,line join=round] ( 28.93,147.34) -- (271.45,147.34);

\path[draw=drawColor,line width= 0.6pt,line join=round] ( 28.93,147.34) -- (271.45,147.34);

\path[draw=drawColor,line width= 0.6pt,line join=round] ( 28.93,147.34) -- (271.45,147.34);

\path[draw=drawColor,line width= 0.6pt,line join=round] ( 28.93,147.34) -- (271.45,147.34);

\path[draw=drawColor,line width= 0.6pt,line join=round] ( 28.93,147.34) -- (271.45,147.34);

\path[draw=drawColor,line width= 0.6pt,line join=round] ( 28.93,147.34) -- (271.45,147.34);

\path[draw=drawColor,line width= 0.6pt,line join=round] ( 28.93,147.34) -- (271.45,147.34);

\path[draw=drawColor,line width= 0.6pt,line join=round] ( 28.93,147.34) -- (271.45,147.34);

\path[draw=drawColor,line width= 0.6pt,line join=round] ( 28.93,147.34) -- (271.45,147.34);

\path[draw=drawColor,line width= 0.6pt,line join=round] ( 28.93,147.34) -- (271.45,147.34);

\path[draw=drawColor,line width= 0.6pt,line join=round] ( 28.93,147.34) -- (271.45,147.34);

\path[draw=drawColor,line width= 0.6pt,line join=round] ( 28.93,147.34) -- (271.45,147.34);

\path[draw=drawColor,line width= 0.6pt,line join=round] ( 28.93,147.34) -- (271.45,147.34);

\path[draw=drawColor,line width= 0.6pt,line join=round] ( 28.93,147.34) -- (271.45,147.34);

\path[draw=drawColor,line width= 0.6pt,line join=round] ( 28.93,147.34) -- (271.45,147.34);

\path[draw=drawColor,line width= 0.6pt,line join=round] ( 28.93,147.34) -- (271.45,147.34);

\path[draw=drawColor,line width= 0.6pt,line join=round] ( 28.93,147.34) -- (271.45,147.34);

\path[draw=drawColor,line width= 0.6pt,line join=round] ( 28.93,147.34) -- (271.45,147.34);

\path[draw=drawColor,line width= 0.6pt,line join=round] ( 28.93,147.34) -- (271.45,147.34);

\path[draw=drawColor,line width= 0.6pt,line join=round] ( 28.93,147.34) -- (271.45,147.34);

\path[draw=drawColor,line width= 0.6pt,line join=round] ( 28.93,147.34) -- (271.45,147.34);

\path[draw=drawColor,line width= 0.6pt,line join=round] ( 28.93,147.34) -- (271.45,147.34);

\path[draw=drawColor,line width= 0.6pt,line join=round] ( 28.93,147.34) -- (271.45,147.34);

\path[draw=drawColor,line width= 0.6pt,line join=round] ( 28.93,147.34) -- (271.45,147.34);

\path[draw=drawColor,line width= 0.6pt,line join=round] ( 28.93,147.34) -- (271.45,147.34);

\path[draw=drawColor,line width= 0.6pt,line join=round] ( 28.93,147.34) -- (271.45,147.34);

\path[draw=drawColor,line width= 0.6pt,line join=round] ( 28.93,147.34) -- (271.45,147.34);

\path[draw=drawColor,line width= 0.6pt,line join=round] ( 28.93,147.34) -- (271.45,147.34);

\path[draw=drawColor,line width= 0.6pt,line join=round] ( 28.93,147.34) -- (271.45,147.34);

\path[draw=drawColor,line width= 0.6pt,line join=round] ( 28.93,147.34) -- (271.45,147.34);

\path[draw=drawColor,line width= 0.6pt,line join=round] ( 28.93,147.34) -- (271.45,147.34);

\path[draw=drawColor,line width= 0.6pt,line join=round] ( 28.93,147.34) -- (271.45,147.34);

\path[draw=drawColor,line width= 0.6pt,line join=round] ( 28.93,147.34) -- (271.45,147.34);

\path[draw=drawColor,line width= 0.6pt,line join=round] ( 28.93,147.34) -- (271.45,147.34);

\path[draw=drawColor,line width= 0.6pt,line join=round] ( 28.93,147.34) -- (271.45,147.34);

\path[draw=drawColor,line width= 0.6pt,line join=round] ( 28.93,147.34) -- (271.45,147.34);

\path[draw=drawColor,line width= 0.6pt,line join=round] ( 28.93,147.34) -- (271.45,147.34);

\path[draw=drawColor,line width= 0.6pt,line join=round] ( 28.93,147.34) -- (271.45,147.34);

\path[draw=drawColor,line width= 0.6pt,line join=round] ( 28.93,147.34) -- (271.45,147.34);

\path[draw=drawColor,line width= 0.6pt,line join=round] ( 28.93,147.34) -- (271.45,147.34);

\path[draw=drawColor,line width= 0.6pt,line join=round] ( 28.93,147.34) -- (271.45,147.34);

\path[draw=drawColor,line width= 0.6pt,line join=round] ( 28.93,147.34) -- (271.45,147.34);

\path[draw=drawColor,line width= 0.6pt,line join=round] ( 28.93,147.34) -- (271.45,147.34);

\path[draw=drawColor,line width= 0.6pt,line join=round] ( 28.93,147.34) -- (271.45,147.34);

\path[draw=drawColor,line width= 0.6pt,line join=round] ( 28.93,147.34) -- (271.45,147.34);

\path[draw=drawColor,line width= 0.6pt,line join=round] ( 28.93,147.34) -- (271.45,147.34);

\path[draw=drawColor,line width= 0.6pt,line join=round] ( 28.93,147.34) -- (271.45,147.34);

\path[draw=drawColor,line width= 0.6pt,line join=round] ( 28.93,147.34) -- (271.45,147.34);

\path[draw=drawColor,line width= 0.6pt,line join=round] ( 28.93,147.34) -- (271.45,147.34);

\path[draw=drawColor,line width= 0.6pt,line join=round] ( 28.93,147.34) -- (271.45,147.34);

\path[draw=drawColor,line width= 0.6pt,line join=round] ( 28.93,147.34) -- (271.45,147.34);

\path[draw=drawColor,line width= 0.6pt,line join=round] ( 28.93,147.34) -- (271.45,147.34);

\path[draw=drawColor,line width= 0.6pt,line join=round] ( 28.93,147.34) -- (271.45,147.34);

\path[draw=drawColor,line width= 0.6pt,line join=round] ( 28.93,147.34) -- (271.45,147.34);

\path[draw=drawColor,line width= 0.6pt,line join=round] ( 28.93,147.34) -- (271.45,147.34);

\path[draw=drawColor,line width= 0.6pt,line join=round] ( 28.93,147.34) -- (271.45,147.34);

\path[draw=drawColor,line width= 0.6pt,line join=round] ( 28.93,147.34) -- (271.45,147.34);

\path[draw=drawColor,line width= 0.6pt,line join=round] ( 28.93,147.34) -- (271.45,147.34);

\path[draw=drawColor,line width= 0.6pt,line join=round] ( 28.93,147.34) -- (271.45,147.34);

\path[draw=drawColor,line width= 0.6pt,line join=round] ( 28.93,147.34) -- (271.45,147.34);

\path[draw=drawColor,line width= 0.6pt,line join=round] ( 28.93,147.34) -- (271.45,147.34);

\path[draw=drawColor,line width= 0.6pt,line join=round] ( 28.93,147.34) -- (271.45,147.34);

\path[draw=drawColor,line width= 0.6pt,line join=round] ( 28.93,147.34) -- (271.45,147.34);

\path[draw=drawColor,line width= 0.6pt,line join=round] ( 28.93,147.34) -- (271.45,147.34);

\path[draw=drawColor,line width= 0.6pt,line join=round] ( 28.93,147.34) -- (271.45,147.34);

\path[draw=drawColor,line width= 0.6pt,line join=round] ( 28.93,147.34) -- (271.45,147.34);

\path[draw=drawColor,line width= 0.6pt,line join=round] ( 28.93,147.34) -- (271.45,147.34);

\path[draw=drawColor,line width= 0.6pt,line join=round] ( 28.93,147.34) -- (271.45,147.34);

\path[draw=drawColor,line width= 0.6pt,line join=round] ( 28.93,147.34) -- (271.45,147.34);

\path[draw=drawColor,line width= 0.6pt,line join=round] ( 28.93,147.34) -- (271.45,147.34);

\path[draw=drawColor,line width= 0.6pt,line join=round] ( 28.93,147.34) -- (271.45,147.34);

\path[draw=drawColor,line width= 0.6pt,line join=round] ( 28.93,147.34) -- (271.45,147.34);

\path[draw=drawColor,line width= 0.6pt,line join=round] ( 28.93,147.34) -- (271.45,147.34);

\path[draw=drawColor,line width= 0.6pt,line join=round] ( 28.93,147.34) -- (271.45,147.34);

\path[draw=drawColor,line width= 0.6pt,line join=round] ( 28.93,147.34) -- (271.45,147.34);

\path[draw=drawColor,line width= 0.6pt,line join=round] ( 28.93,147.34) -- (271.45,147.34);

\path[draw=drawColor,line width= 0.6pt,line join=round] ( 28.93,147.34) -- (271.45,147.34);

\path[draw=drawColor,line width= 0.6pt,line join=round] ( 28.93,147.34) -- (271.45,147.34);

\path[draw=drawColor,line width= 0.6pt,line join=round] ( 28.93,147.34) -- (271.45,147.34);

\path[draw=drawColor,line width= 0.6pt,line join=round] ( 28.93,147.34) -- (271.45,147.34);

\path[draw=drawColor,line width= 0.6pt,line join=round] ( 28.93,147.34) -- (271.45,147.34);

\path[draw=drawColor,line width= 0.6pt,line join=round] ( 28.93,147.34) -- (271.45,147.34);

\path[draw=drawColor,line width= 0.6pt,line join=round] ( 28.93,147.34) -- (271.45,147.34);

\path[draw=drawColor,line width= 0.6pt,line join=round] ( 28.93,147.34) -- (271.45,147.34);

\path[draw=drawColor,line width= 0.6pt,line join=round] ( 28.93,147.34) -- (271.45,147.34);

\path[draw=drawColor,line width= 0.6pt,line join=round] ( 28.93,147.34) -- (271.45,147.34);

\path[draw=drawColor,line width= 0.6pt,line join=round] ( 28.93,147.34) -- (271.45,147.34);

\path[draw=drawColor,line width= 0.6pt,line join=round] ( 28.93,147.34) -- (271.45,147.34);

\path[draw=drawColor,line width= 0.6pt,line join=round] ( 28.93,147.34) -- (271.45,147.34);

\path[draw=drawColor,line width= 0.6pt,line join=round] ( 28.93,147.34) -- (271.45,147.34);

\path[draw=drawColor,line width= 0.6pt,line join=round] ( 28.93,147.34) -- (271.45,147.34);

\path[draw=drawColor,line width= 0.6pt,line join=round] ( 28.93,147.34) -- (271.45,147.34);

\path[draw=drawColor,line width= 0.6pt,line join=round] ( 28.93,147.34) -- (271.45,147.34);

\path[draw=drawColor,line width= 0.6pt,line join=round] ( 28.93,147.34) -- (271.45,147.34);

\path[draw=drawColor,line width= 0.6pt,line join=round] ( 28.93,147.34) -- (271.45,147.34);

\path[draw=drawColor,line width= 0.6pt,line join=round] ( 28.93,147.34) -- (271.45,147.34);

\path[draw=drawColor,line width= 0.6pt,line join=round] ( 28.93,147.34) -- (271.45,147.34);

\path[draw=drawColor,line width= 0.6pt,line join=round] ( 28.93,147.34) -- (271.45,147.34);

\path[draw=drawColor,line width= 0.6pt,line join=round] ( 28.93,147.34) -- (271.45,147.34);

\path[draw=drawColor,line width= 0.6pt,line join=round] ( 28.93,147.34) -- (271.45,147.34);

\path[draw=drawColor,line width= 0.6pt,line join=round] ( 28.93,147.34) -- (271.45,147.34);

\path[draw=drawColor,line width= 0.6pt,line join=round] ( 28.93,147.34) -- (271.45,147.34);

\path[draw=drawColor,line width= 0.6pt,line join=round] ( 28.93,147.34) -- (271.45,147.34);

\path[draw=drawColor,line width= 0.6pt,line join=round] ( 28.93,147.34) -- (271.45,147.34);

\path[draw=drawColor,line width= 0.6pt,line join=round] ( 28.93,147.34) -- (271.45,147.34);

\path[draw=drawColor,line width= 0.6pt,line join=round] ( 28.93,147.34) -- (271.45,147.34);

\path[draw=drawColor,line width= 0.6pt,line join=round] ( 28.93,147.34) -- (271.45,147.34);

\path[draw=drawColor,line width= 0.6pt,line join=round] ( 28.93,147.34) -- (271.45,147.34);

\path[draw=drawColor,line width= 0.6pt,line join=round] ( 28.93,147.34) -- (271.45,147.34);

\path[draw=drawColor,line width= 0.6pt,line join=round] ( 28.93,147.34) -- (271.45,147.34);

\path[draw=drawColor,line width= 0.6pt,line join=round] ( 28.93,147.34) -- (271.45,147.34);

\path[draw=drawColor,line width= 0.6pt,line join=round] ( 28.93,147.34) -- (271.45,147.34);

\path[draw=drawColor,line width= 0.6pt,line join=round] ( 28.93,147.34) -- (271.45,147.34);

\path[draw=drawColor,line width= 0.6pt,line join=round] ( 28.93,147.34) -- (271.45,147.34);

\path[draw=drawColor,line width= 0.6pt,line join=round] ( 28.93,147.34) -- (271.45,147.34);

\path[draw=drawColor,line width= 0.6pt,line join=round] ( 28.93,147.34) -- (271.45,147.34);

\path[draw=drawColor,line width= 0.6pt,line join=round] ( 28.93,147.34) -- (271.45,147.34);

\path[draw=drawColor,line width= 0.6pt,line join=round] ( 28.93,147.34) -- (271.45,147.34);

\path[draw=drawColor,line width= 0.6pt,line join=round] ( 28.93,147.34) -- (271.45,147.34);

\path[draw=drawColor,line width= 0.6pt,line join=round] ( 28.93,147.34) -- (271.45,147.34);

\path[draw=drawColor,line width= 0.6pt,line join=round] ( 28.93,147.34) -- (271.45,147.34);

\path[draw=drawColor,line width= 0.6pt,line join=round] ( 28.93,147.34) -- (271.45,147.34);

\path[draw=drawColor,line width= 0.6pt,line join=round] ( 28.93,147.34) -- (271.45,147.34);

\path[draw=drawColor,line width= 0.6pt,line join=round] ( 28.93,147.34) -- (271.45,147.34);

\path[draw=drawColor,line width= 0.6pt,line join=round] ( 28.93,147.34) -- (271.45,147.34);

\path[draw=drawColor,line width= 0.6pt,line join=round] ( 28.93,147.34) -- (271.45,147.34);

\path[draw=drawColor,line width= 0.6pt,line join=round] ( 28.93,147.34) -- (271.45,147.34);

\path[draw=drawColor,line width= 0.6pt,line join=round] ( 28.93,147.34) -- (271.45,147.34);

\path[draw=drawColor,line width= 0.6pt,line join=round] ( 28.93,147.34) -- (271.45,147.34);

\path[draw=drawColor,line width= 0.6pt,line join=round] ( 28.93,147.34) -- (271.45,147.34);

\path[draw=drawColor,line width= 0.6pt,line join=round] ( 28.93,147.34) -- (271.45,147.34);

\path[draw=drawColor,line width= 0.6pt,line join=round] ( 28.93,147.34) -- (271.45,147.34);

\path[draw=drawColor,line width= 0.6pt,line join=round] ( 28.93,147.34) -- (271.45,147.34);

\path[draw=drawColor,line width= 0.6pt,line join=round] ( 28.93,147.34) -- (271.45,147.34);

\path[draw=drawColor,line width= 0.6pt,line join=round] ( 28.93,147.34) -- (271.45,147.34);

\path[draw=drawColor,line width= 0.6pt,line join=round] ( 28.93,147.34) -- (271.45,147.34);

\path[draw=drawColor,line width= 0.6pt,line join=round] ( 28.93,147.34) -- (271.45,147.34);

\path[draw=drawColor,line width= 0.6pt,line join=round] ( 28.93,147.34) -- (271.45,147.34);

\path[draw=drawColor,line width= 0.6pt,line join=round] ( 28.93,147.34) -- (271.45,147.34);

\path[draw=drawColor,line width= 0.6pt,line join=round] ( 28.93,147.34) -- (271.45,147.34);

\path[draw=drawColor,line width= 0.6pt,line join=round] ( 28.93,147.34) -- (271.45,147.34);

\path[draw=drawColor,line width= 0.6pt,line join=round] ( 28.93,147.34) -- (271.45,147.34);

\path[draw=drawColor,line width= 0.6pt,line join=round] ( 28.93,147.34) -- (271.45,147.34);

\path[draw=drawColor,line width= 0.6pt,line join=round] ( 28.93,147.34) -- (271.45,147.34);

\path[draw=drawColor,line width= 0.6pt,line join=round] ( 28.93,147.34) -- (271.45,147.34);

\path[draw=drawColor,line width= 0.6pt,line join=round] ( 28.93,147.34) -- (271.45,147.34);

\path[draw=drawColor,line width= 0.6pt,line join=round] ( 28.93,147.34) -- (271.45,147.34);

\path[draw=drawColor,line width= 0.6pt,line join=round] ( 28.93,147.34) -- (271.45,147.34);

\path[draw=drawColor,line width= 0.6pt,line join=round] ( 28.93,147.34) -- (271.45,147.34);

\path[draw=drawColor,line width= 0.6pt,line join=round] ( 28.93,147.34) -- (271.45,147.34);

\path[draw=drawColor,line width= 0.6pt,line join=round] ( 28.93,147.34) -- (271.45,147.34);

\path[draw=drawColor,line width= 0.6pt,line join=round] ( 28.93,147.34) -- (271.45,147.34);

\path[draw=drawColor,line width= 0.6pt,line join=round] ( 28.93,147.34) -- (271.45,147.34);

\path[draw=drawColor,line width= 0.6pt,line join=round] ( 28.93,147.34) -- (271.45,147.34);

\path[draw=drawColor,line width= 0.6pt,line join=round] ( 28.93,147.34) -- (271.45,147.34);

\path[draw=drawColor,line width= 0.6pt,line join=round] ( 28.93,147.34) -- (271.45,147.34);

\path[draw=drawColor,line width= 0.6pt,line join=round] ( 28.93,147.34) -- (271.45,147.34);

\path[draw=drawColor,line width= 0.6pt,line join=round] ( 28.93,147.34) -- (271.45,147.34);

\path[draw=drawColor,line width= 0.6pt,line join=round] ( 28.93,147.34) -- (271.45,147.34);

\path[draw=drawColor,line width= 0.6pt,line join=round] ( 28.93,147.34) -- (271.45,147.34);

\path[draw=drawColor,line width= 0.6pt,line join=round] ( 28.93,147.34) -- (271.45,147.34);

\path[draw=drawColor,line width= 0.6pt,line join=round] ( 28.93,147.34) -- (271.45,147.34);

\path[draw=drawColor,line width= 0.6pt,line join=round] ( 28.93,147.34) -- (271.45,147.34);

\path[draw=drawColor,line width= 0.6pt,line join=round] ( 28.93,147.34) -- (271.45,147.34);

\path[draw=drawColor,line width= 0.6pt,line join=round] ( 28.93,147.34) -- (271.45,147.34);

\path[draw=drawColor,line width= 0.6pt,line join=round] ( 28.93,147.34) -- (271.45,147.34);

\path[draw=drawColor,line width= 0.6pt,line join=round] ( 28.93,147.34) -- (271.45,147.34);

\path[draw=drawColor,line width= 0.6pt,line join=round] ( 28.93,147.34) -- (271.45,147.34);

\path[draw=drawColor,line width= 0.6pt,line join=round] ( 28.93,147.34) -- (271.45,147.34);

\path[draw=drawColor,line width= 0.6pt,line join=round] ( 28.93,147.34) -- (271.45,147.34);

\path[draw=drawColor,line width= 0.6pt,line join=round] ( 28.93,147.34) -- (271.45,147.34);

\path[draw=drawColor,line width= 0.6pt,line join=round] ( 28.93,147.34) -- (271.45,147.34);

\path[draw=drawColor,line width= 0.6pt,line join=round] ( 28.93,147.34) -- (271.45,147.34);

\path[draw=drawColor,line width= 0.6pt,line join=round] ( 28.93,147.34) -- (271.45,147.34);

\path[draw=drawColor,line width= 0.6pt,line join=round] ( 28.93,147.34) -- (271.45,147.34);

\path[draw=drawColor,line width= 0.6pt,line join=round] ( 28.93,147.34) -- (271.45,147.34);

\path[draw=drawColor,line width= 0.6pt,line join=round] ( 28.93,147.34) -- (271.45,147.34);

\path[draw=drawColor,line width= 0.6pt,line join=round] ( 28.93,147.34) -- (271.45,147.34);

\path[draw=drawColor,line width= 0.6pt,line join=round] ( 28.93,147.34) -- (271.45,147.34);

\path[draw=drawColor,line width= 0.6pt,line join=round] ( 28.93,147.34) -- (271.45,147.34);

\path[draw=drawColor,line width= 0.6pt,line join=round] ( 28.93,147.34) -- (271.45,147.34);

\path[draw=drawColor,line width= 0.6pt,line join=round] ( 28.93,147.34) -- (271.45,147.34);

\path[draw=drawColor,line width= 0.6pt,line join=round] ( 28.93,147.34) -- (271.45,147.34);

\path[draw=drawColor,line width= 0.6pt,line join=round] ( 28.93,147.34) -- (271.45,147.34);

\path[draw=drawColor,line width= 0.6pt,line join=round] ( 28.93,147.34) -- (271.45,147.34);

\path[draw=drawColor,line width= 0.6pt,line join=round] ( 28.93,147.34) -- (271.45,147.34);

\path[draw=drawColor,line width= 0.6pt,line join=round] ( 28.93,147.34) -- (271.45,147.34);

\path[draw=drawColor,line width= 0.6pt,line join=round] ( 28.93,147.34) -- (271.45,147.34);

\path[draw=drawColor,line width= 0.6pt,line join=round] ( 28.93,147.34) -- (271.45,147.34);

\path[draw=drawColor,line width= 0.6pt,line join=round] ( 28.93,147.34) -- (271.45,147.34);

\path[draw=drawColor,line width= 0.6pt,line join=round] ( 28.93,147.34) -- (271.45,147.34);

\path[draw=drawColor,line width= 0.6pt,line join=round] ( 28.93,147.34) -- (271.45,147.34);

\path[draw=drawColor,line width= 0.6pt,line join=round] ( 28.93,147.34) -- (271.45,147.34);

\path[draw=drawColor,line width= 0.6pt,line join=round] ( 28.93,147.34) -- (271.45,147.34);

\path[draw=drawColor,line width= 0.6pt,line join=round] ( 28.93,147.34) -- (271.45,147.34);

\path[draw=drawColor,line width= 0.6pt,line join=round] ( 28.93,147.34) -- (271.45,147.34);

\path[draw=drawColor,line width= 0.6pt,line join=round] ( 28.93,147.34) -- (271.45,147.34);

\path[draw=drawColor,line width= 0.6pt,line join=round] ( 28.93,147.34) -- (271.45,147.34);

\path[draw=drawColor,line width= 0.6pt,line join=round] ( 28.93,147.34) -- (271.45,147.34);

\path[draw=drawColor,line width= 0.6pt,line join=round] ( 28.93,147.34) -- (271.45,147.34);

\path[draw=drawColor,line width= 0.6pt,line join=round] ( 28.93,147.34) -- (271.45,147.34);

\path[draw=drawColor,line width= 0.6pt,line join=round] ( 28.93,147.34) -- (271.45,147.34);

\path[draw=drawColor,line width= 0.6pt,line join=round] ( 28.93,147.34) -- (271.45,147.34);

\path[draw=drawColor,line width= 0.6pt,line join=round] ( 28.93,147.34) -- (271.45,147.34);

\path[draw=drawColor,line width= 0.6pt,line join=round] ( 28.93,147.34) -- (271.45,147.34);

\path[draw=drawColor,line width= 0.6pt,line join=round] ( 28.93,147.34) -- (271.45,147.34);

\path[draw=drawColor,line width= 0.6pt,line join=round] ( 28.93,147.34) -- (271.45,147.34);

\path[draw=drawColor,line width= 0.6pt,line join=round] ( 28.93,147.34) -- (271.45,147.34);

\path[draw=drawColor,line width= 0.6pt,line join=round] ( 28.93,147.34) -- (271.45,147.34);

\path[draw=drawColor,line width= 0.6pt,line join=round] ( 28.93,147.34) -- (271.45,147.34);

\path[draw=drawColor,line width= 0.6pt,line join=round] ( 28.93,147.34) -- (271.45,147.34);

\path[draw=drawColor,line width= 0.6pt,line join=round] ( 28.93,147.34) -- (271.45,147.34);

\path[draw=drawColor,line width= 0.6pt,line join=round] ( 28.93,147.34) -- (271.45,147.34);

\path[draw=drawColor,line width= 0.6pt,line join=round] ( 28.93,147.34) -- (271.45,147.34);

\path[draw=drawColor,line width= 0.6pt,line join=round] ( 28.93,147.34) -- (271.45,147.34);

\path[draw=drawColor,line width= 0.6pt,line join=round] ( 28.93,147.34) -- (271.45,147.34);

\path[draw=drawColor,line width= 0.6pt,line join=round] ( 28.93,147.34) -- (271.45,147.34);

\path[draw=drawColor,line width= 0.6pt,line join=round] ( 28.93,147.34) -- (271.45,147.34);

\path[draw=drawColor,line width= 0.6pt,line join=round] ( 28.93,147.34) -- (271.45,147.34);

\path[draw=drawColor,line width= 0.6pt,line join=round] ( 28.93,147.34) -- (271.45,147.34);

\path[draw=drawColor,line width= 0.6pt,line join=round] ( 28.93,147.34) -- (271.45,147.34);

\path[draw=drawColor,line width= 0.6pt,line join=round] ( 28.93,147.34) -- (271.45,147.34);

\path[draw=drawColor,line width= 0.6pt,line join=round] ( 28.93,147.34) -- (271.45,147.34);

\path[draw=drawColor,line width= 0.6pt,line join=round] ( 28.93,147.34) -- (271.45,147.34);

\path[draw=drawColor,line width= 0.6pt,line join=round] ( 28.93,147.34) -- (271.45,147.34);

\path[draw=drawColor,line width= 0.6pt,line join=round] ( 28.93,147.34) -- (271.45,147.34);

\path[draw=drawColor,line width= 0.6pt,line join=round] ( 28.93,147.34) -- (271.45,147.34);

\path[draw=drawColor,line width= 0.6pt,line join=round] ( 28.93,147.34) -- (271.45,147.34);

\path[draw=drawColor,line width= 0.6pt,line join=round] ( 28.93,147.34) -- (271.45,147.34);

\path[draw=drawColor,line width= 0.6pt,line join=round] ( 28.93,147.34) -- (271.45,147.34);

\path[draw=drawColor,line width= 0.6pt,line join=round] ( 28.93,147.34) -- (271.45,147.34);

\path[draw=drawColor,line width= 0.6pt,line join=round] ( 28.93,147.34) -- (271.45,147.34);

\path[draw=drawColor,line width= 0.6pt,line join=round] ( 28.93,147.34) -- (271.45,147.34);

\path[draw=drawColor,line width= 0.6pt,line join=round] ( 28.93,147.34) -- (271.45,147.34);

\path[draw=drawColor,line width= 0.6pt,line join=round] ( 28.93,147.34) -- (271.45,147.34);

\path[draw=drawColor,line width= 0.6pt,line join=round] ( 28.93,147.34) -- (271.45,147.34);

\path[draw=drawColor,line width= 0.6pt,line join=round] ( 28.93,147.34) -- (271.45,147.34);

\path[draw=drawColor,line width= 0.6pt,line join=round] ( 28.93,147.34) -- (271.45,147.34);

\path[draw=drawColor,line width= 0.6pt,line join=round] ( 28.93,147.34) -- (271.45,147.34);

\path[draw=drawColor,line width= 0.6pt,line join=round] ( 28.93,147.34) -- (271.45,147.34);

\path[draw=drawColor,line width= 0.6pt,line join=round] ( 28.93,147.34) -- (271.45,147.34);

\path[draw=drawColor,line width= 0.6pt,line join=round] ( 28.93,147.34) -- (271.45,147.34);

\path[draw=drawColor,line width= 0.6pt,line join=round] ( 28.93,147.34) -- (271.45,147.34);

\path[draw=drawColor,line width= 0.6pt,line join=round] ( 28.93,147.34) -- (271.45,147.34);

\path[draw=drawColor,line width= 0.6pt,line join=round] ( 28.93,147.34) -- (271.45,147.34);

\path[draw=drawColor,line width= 0.6pt,line join=round] ( 28.93,147.34) -- (271.45,147.34);

\path[draw=drawColor,line width= 0.6pt,line join=round] ( 28.93,147.34) -- (271.45,147.34);

\path[draw=drawColor,line width= 0.6pt,line join=round] ( 28.93,147.34) -- (271.45,147.34);

\path[draw=drawColor,line width= 0.6pt,line join=round] ( 28.93,147.34) -- (271.45,147.34);

\path[draw=drawColor,line width= 0.6pt,line join=round] ( 28.93,147.34) -- (271.45,147.34);

\path[draw=drawColor,line width= 0.6pt,line join=round] ( 28.93,147.34) -- (271.45,147.34);

\path[draw=drawColor,line width= 0.6pt,line join=round] ( 28.93,147.34) -- (271.45,147.34);

\path[draw=drawColor,line width= 0.6pt,line join=round] ( 28.93,147.34) -- (271.45,147.34);

\path[draw=drawColor,line width= 0.6pt,line join=round] ( 28.93,147.34) -- (271.45,147.34);

\path[draw=drawColor,line width= 0.6pt,line join=round] ( 28.93,147.34) -- (271.45,147.34);

\path[draw=drawColor,line width= 0.6pt,line join=round] ( 28.93,147.34) -- (271.45,147.34);

\path[draw=drawColor,line width= 0.6pt,line join=round] ( 28.93,147.34) -- (271.45,147.34);

\path[draw=drawColor,line width= 0.6pt,line join=round] ( 28.93,147.34) -- (271.45,147.34);

\path[draw=drawColor,line width= 0.6pt,line join=round] ( 28.93,147.34) -- (271.45,147.34);

\path[draw=drawColor,line width= 0.6pt,line join=round] ( 28.93,147.34) -- (271.45,147.34);

\path[draw=drawColor,line width= 0.6pt,line join=round] ( 28.93,147.34) -- (271.45,147.34);

\path[draw=drawColor,line width= 0.6pt,line join=round] ( 28.93,147.34) -- (271.45,147.34);

\path[draw=drawColor,line width= 0.6pt,line join=round] ( 28.93,147.34) -- (271.45,147.34);

\path[draw=drawColor,line width= 0.6pt,line join=round] ( 28.93,147.34) -- (271.45,147.34);

\path[draw=drawColor,line width= 0.6pt,line join=round] ( 28.93,147.34) -- (271.45,147.34);

\path[draw=drawColor,line width= 0.6pt,line join=round] ( 28.93,147.34) -- (271.45,147.34);

\path[draw=drawColor,line width= 0.6pt,line join=round] ( 28.93,147.34) -- (271.45,147.34);

\path[draw=drawColor,line width= 0.6pt,line join=round] ( 28.93,147.34) -- (271.45,147.34);

\path[draw=drawColor,line width= 0.6pt,line join=round] ( 28.93,147.34) -- (271.45,147.34);

\path[draw=drawColor,line width= 0.6pt,line join=round] ( 28.93,147.34) -- (271.45,147.34);

\path[draw=drawColor,line width= 0.6pt,line join=round] ( 28.93,147.34) -- (271.45,147.34);

\path[draw=drawColor,line width= 0.6pt,line join=round] ( 28.93,147.34) -- (271.45,147.34);

\path[draw=drawColor,line width= 0.6pt,line join=round] ( 28.93,147.34) -- (271.45,147.34);

\path[draw=drawColor,line width= 0.6pt,line join=round] ( 28.93,147.34) -- (271.45,147.34);

\path[draw=drawColor,line width= 0.6pt,line join=round] ( 28.93,147.34) -- (271.45,147.34);

\path[draw=drawColor,line width= 0.6pt,line join=round] ( 28.93,147.34) -- (271.45,147.34);

\path[draw=drawColor,line width= 0.6pt,line join=round] ( 28.93,147.34) -- (271.45,147.34);

\path[draw=drawColor,line width= 0.6pt,line join=round] ( 28.93,147.34) -- (271.45,147.34);

\path[draw=drawColor,line width= 0.6pt,line join=round] ( 28.93,147.34) -- (271.45,147.34);

\path[draw=drawColor,line width= 0.6pt,line join=round] ( 28.93,147.34) -- (271.45,147.34);

\path[draw=drawColor,line width= 0.6pt,line join=round] ( 28.93,147.34) -- (271.45,147.34);

\path[draw=drawColor,line width= 0.6pt,line join=round] ( 28.93,147.34) -- (271.45,147.34);

\path[draw=drawColor,line width= 0.6pt,line join=round] ( 28.93,147.34) -- (271.45,147.34);

\path[draw=drawColor,line width= 0.6pt,line join=round] ( 28.93,147.34) -- (271.45,147.34);

\path[draw=drawColor,line width= 0.6pt,line join=round] ( 28.93,147.34) -- (271.45,147.34);

\path[draw=drawColor,line width= 0.6pt,line join=round] ( 28.93,147.34) -- (271.45,147.34);

\path[draw=drawColor,line width= 0.6pt,line join=round] ( 28.93,147.34) -- (271.45,147.34);

\path[draw=drawColor,line width= 0.6pt,line join=round] ( 28.93,147.34) -- (271.45,147.34);

\path[draw=drawColor,line width= 0.6pt,line join=round] ( 28.93,147.34) -- (271.45,147.34);

\path[draw=drawColor,line width= 0.6pt,line join=round] ( 28.93,147.34) -- (271.45,147.34);

\path[draw=drawColor,line width= 0.6pt,line join=round] ( 28.93,147.34) -- (271.45,147.34);

\path[draw=drawColor,line width= 0.6pt,line join=round] ( 28.93,147.34) -- (271.45,147.34);

\path[draw=drawColor,line width= 0.6pt,line join=round] ( 28.93,147.34) -- (271.45,147.34);

\path[draw=drawColor,line width= 0.6pt,line join=round] ( 28.93,147.34) -- (271.45,147.34);

\path[draw=drawColor,line width= 0.6pt,line join=round] ( 28.93,147.34) -- (271.45,147.34);

\path[draw=drawColor,line width= 0.6pt,line join=round] ( 28.93,147.34) -- (271.45,147.34);

\path[draw=drawColor,line width= 0.6pt,line join=round] ( 28.93,147.34) -- (271.45,147.34);

\path[draw=drawColor,line width= 0.6pt,line join=round] ( 28.93,147.34) -- (271.45,147.34);

\path[draw=drawColor,line width= 0.6pt,line join=round] ( 28.93,147.34) -- (271.45,147.34);

\path[draw=drawColor,line width= 0.6pt,line join=round] ( 28.93,147.34) -- (271.45,147.34);

\path[draw=drawColor,line width= 0.6pt,line join=round] ( 28.93,147.34) -- (271.45,147.34);

\path[draw=drawColor,line width= 0.6pt,line join=round] ( 28.93,147.34) -- (271.45,147.34);

\path[draw=drawColor,line width= 0.6pt,line join=round] ( 28.93,147.34) -- (271.45,147.34);

\path[draw=drawColor,line width= 0.6pt,line join=round] ( 28.93,147.34) -- (271.45,147.34);

\path[draw=drawColor,line width= 0.6pt,line join=round] ( 28.93,147.34) -- (271.45,147.34);

\path[draw=drawColor,line width= 0.6pt,line join=round] ( 28.93,147.34) -- (271.45,147.34);

\path[draw=drawColor,line width= 0.6pt,line join=round] ( 28.93,147.34) -- (271.45,147.34);

\path[draw=drawColor,line width= 0.6pt,line join=round] ( 28.93,147.34) -- (271.45,147.34);

\path[draw=drawColor,line width= 0.6pt,line join=round] ( 28.93,147.34) -- (271.45,147.34);

\path[draw=drawColor,line width= 0.6pt,line join=round] ( 28.93,147.34) -- (271.45,147.34);

\path[draw=drawColor,line width= 0.6pt,line join=round] ( 28.93,147.34) -- (271.45,147.34);

\path[draw=drawColor,line width= 0.6pt,line join=round] ( 28.93,147.34) -- (271.45,147.34);

\path[draw=drawColor,line width= 0.6pt,line join=round] ( 28.93,147.34) -- (271.45,147.34);

\path[draw=drawColor,line width= 0.6pt,line join=round] ( 28.93,147.34) -- (271.45,147.34);

\path[draw=drawColor,line width= 0.6pt,line join=round] ( 28.93,147.34) -- (271.45,147.34);

\path[draw=drawColor,line width= 0.6pt,line join=round] ( 28.93,147.34) -- (271.45,147.34);

\path[draw=drawColor,line width= 0.6pt,line join=round] ( 28.93,147.34) -- (271.45,147.34);

\path[draw=drawColor,line width= 0.6pt,line join=round] ( 28.93,147.34) -- (271.45,147.34);

\path[draw=drawColor,line width= 0.6pt,line join=round] ( 28.93,147.34) -- (271.45,147.34);

\path[draw=drawColor,line width= 0.6pt,line join=round] ( 28.93,147.34) -- (271.45,147.34);

\path[draw=drawColor,line width= 0.6pt,line join=round] ( 28.93,147.34) -- (271.45,147.34);

\path[draw=drawColor,line width= 0.6pt,line join=round] ( 28.93,147.34) -- (271.45,147.34);

\path[draw=drawColor,line width= 0.6pt,line join=round] ( 28.93,147.34) -- (271.45,147.34);

\path[draw=drawColor,line width= 0.6pt,line join=round] ( 28.93,147.34) -- (271.45,147.34);

\path[draw=drawColor,line width= 0.6pt,line join=round] ( 28.93,147.34) -- (271.45,147.34);

\path[draw=drawColor,line width= 0.6pt,line join=round] ( 28.93,147.34) -- (271.45,147.34);

\path[draw=drawColor,line width= 0.6pt,line join=round] ( 28.93,147.34) -- (271.45,147.34);

\path[draw=drawColor,line width= 0.6pt,line join=round] ( 28.93,147.34) -- (271.45,147.34);

\path[draw=drawColor,line width= 0.6pt,line join=round] ( 28.93,147.34) -- (271.45,147.34);

\path[draw=drawColor,line width= 0.6pt,line join=round] ( 28.93,147.34) -- (271.45,147.34);

\path[draw=drawColor,line width= 0.6pt,line join=round] ( 28.93,147.34) -- (271.45,147.34);

\path[draw=drawColor,line width= 0.6pt,line join=round] ( 28.93,147.34) -- (271.45,147.34);

\path[draw=drawColor,line width= 0.6pt,line join=round] ( 28.93,147.34) -- (271.45,147.34);

\path[draw=drawColor,line width= 0.6pt,line join=round] ( 28.93,147.34) -- (271.45,147.34);

\path[draw=drawColor,line width= 0.6pt,line join=round] ( 28.93,147.34) -- (271.45,147.34);

\path[draw=drawColor,line width= 0.6pt,line join=round] ( 28.93,147.34) -- (271.45,147.34);

\path[draw=drawColor,line width= 0.6pt,line join=round] ( 28.93,147.34) -- (271.45,147.34);

\path[draw=drawColor,line width= 0.6pt,line join=round] ( 28.93,147.34) -- (271.45,147.34);

\path[draw=drawColor,line width= 0.6pt,line join=round] ( 28.93,147.34) -- (271.45,147.34);

\path[draw=drawColor,line width= 0.6pt,line join=round] ( 28.93,147.34) -- (271.45,147.34);

\path[draw=drawColor,line width= 0.6pt,line join=round] ( 28.93,147.34) -- (271.45,147.34);

\path[draw=drawColor,line width= 0.6pt,line join=round] ( 28.93,147.34) -- (271.45,147.34);

\path[draw=drawColor,line width= 0.6pt,line join=round] ( 28.93,147.34) -- (271.45,147.34);

\path[draw=drawColor,line width= 0.6pt,line join=round] ( 28.93,147.34) -- (271.45,147.34);

\path[draw=drawColor,line width= 0.6pt,line join=round] ( 28.93,147.34) -- (271.45,147.34);

\path[draw=drawColor,line width= 0.6pt,line join=round] ( 28.93,147.34) -- (271.45,147.34);

\path[draw=drawColor,line width= 0.6pt,line join=round] ( 28.93,147.34) -- (271.45,147.34);

\path[draw=drawColor,line width= 0.6pt,line join=round] ( 28.93,147.34) -- (271.45,147.34);

\path[draw=drawColor,line width= 0.6pt,line join=round] ( 28.93,147.34) -- (271.45,147.34);

\path[draw=drawColor,line width= 0.6pt,line join=round] ( 28.93,147.34) -- (271.45,147.34);

\path[draw=drawColor,line width= 0.6pt,line join=round] ( 28.93,147.34) -- (271.45,147.34);

\path[draw=drawColor,line width= 0.6pt,line join=round] ( 28.93,147.34) -- (271.45,147.34);

\path[draw=drawColor,line width= 0.6pt,line join=round] ( 28.93,147.34) -- (271.45,147.34);

\path[draw=drawColor,line width= 0.6pt,line join=round] ( 28.93,147.34) -- (271.45,147.34);

\path[draw=drawColor,line width= 0.6pt,line join=round] ( 28.93,147.34) -- (271.45,147.34);

\path[draw=drawColor,line width= 0.6pt,line join=round] ( 28.93,147.34) -- (271.45,147.34);

\path[draw=drawColor,line width= 0.6pt,line join=round] ( 28.93,147.34) -- (271.45,147.34);

\path[draw=drawColor,line width= 0.6pt,line join=round] ( 28.93,147.34) -- (271.45,147.34);

\path[draw=drawColor,line width= 0.6pt,line join=round] ( 28.93,147.34) -- (271.45,147.34);

\path[draw=drawColor,line width= 0.6pt,line join=round] ( 28.93,147.34) -- (271.45,147.34);

\path[draw=drawColor,line width= 0.6pt,line join=round] ( 28.93,147.34) -- (271.45,147.34);

\path[draw=drawColor,line width= 0.6pt,line join=round] ( 28.93,147.34) -- (271.45,147.34);

\path[draw=drawColor,line width= 0.6pt,line join=round] ( 28.93,147.34) -- (271.45,147.34);

\path[draw=drawColor,line width= 0.6pt,line join=round] ( 28.93,147.34) -- (271.45,147.34);

\path[draw=drawColor,line width= 0.6pt,line join=round] ( 28.93,147.34) -- (271.45,147.34);

\path[draw=drawColor,line width= 0.6pt,line join=round] ( 28.93,147.34) -- (271.45,147.34);

\path[draw=drawColor,line width= 0.6pt,line join=round] ( 28.93,147.34) -- (271.45,147.34);

\path[draw=drawColor,line width= 0.6pt,line join=round] ( 28.93,147.34) -- (271.45,147.34);

\path[draw=drawColor,line width= 0.6pt,line join=round] ( 28.93,147.34) -- (271.45,147.34);

\path[draw=drawColor,line width= 0.6pt,line join=round] ( 28.93,147.34) -- (271.45,147.34);

\path[draw=drawColor,line width= 0.6pt,line join=round] ( 28.93,147.34) -- (271.45,147.34);

\path[draw=drawColor,line width= 0.6pt,line join=round] ( 28.93,147.34) -- (271.45,147.34);

\path[draw=drawColor,line width= 0.6pt,line join=round] ( 28.93,147.34) -- (271.45,147.34);

\path[draw=drawColor,line width= 0.6pt,line join=round] ( 28.93,147.34) -- (271.45,147.34);

\path[draw=drawColor,line width= 0.6pt,line join=round] ( 28.93,147.34) -- (271.45,147.34);

\path[draw=drawColor,line width= 0.6pt,line join=round] ( 28.93,147.34) -- (271.45,147.34);

\path[draw=drawColor,line width= 0.6pt,line join=round] ( 28.93,147.34) -- (271.45,147.34);

\path[draw=drawColor,line width= 0.6pt,line join=round] ( 28.93,147.34) -- (271.45,147.34);

\path[draw=drawColor,line width= 0.6pt,line join=round] ( 28.93,147.34) -- (271.45,147.34);

\path[draw=drawColor,line width= 0.6pt,line join=round] ( 28.93,147.34) -- (271.45,147.34);

\path[draw=drawColor,line width= 0.6pt,line join=round] ( 28.93,147.34) -- (271.45,147.34);

\path[draw=drawColor,line width= 0.6pt,line join=round] ( 28.93,147.34) -- (271.45,147.34);

\path[draw=drawColor,line width= 0.6pt,line join=round] ( 28.93,147.34) -- (271.45,147.34);

\path[draw=drawColor,line width= 0.6pt,line join=round] ( 28.93,147.34) -- (271.45,147.34);

\path[draw=drawColor,line width= 0.6pt,line join=round] ( 28.93,147.34) -- (271.45,147.34);

\path[draw=drawColor,line width= 0.6pt,line join=round] ( 28.93,147.34) -- (271.45,147.34);

\path[draw=drawColor,line width= 0.6pt,line join=round] ( 28.93,147.34) -- (271.45,147.34);

\path[draw=drawColor,line width= 0.6pt,line join=round] ( 28.93,147.34) -- (271.45,147.34);

\path[draw=drawColor,line width= 0.6pt,line join=round] ( 28.93,147.34) -- (271.45,147.34);

\path[draw=drawColor,line width= 0.6pt,line join=round] ( 28.93,147.34) -- (271.45,147.34);

\path[draw=drawColor,line width= 0.6pt,line join=round] ( 28.93,147.34) -- (271.45,147.34);

\path[draw=drawColor,line width= 0.6pt,line join=round] ( 28.93,147.34) -- (271.45,147.34);

\path[draw=drawColor,line width= 0.6pt,line join=round] ( 28.93,147.34) -- (271.45,147.34);

\path[draw=drawColor,line width= 0.6pt,line join=round] ( 28.93,147.34) -- (271.45,147.34);

\path[draw=drawColor,line width= 0.6pt,line join=round] ( 28.93,147.34) -- (271.45,147.34);

\path[draw=drawColor,line width= 0.6pt,line join=round] ( 28.93,147.34) -- (271.45,147.34);

\path[draw=drawColor,line width= 0.6pt,line join=round] ( 28.93,147.34) -- (271.45,147.34);

\path[draw=drawColor,line width= 0.6pt,line join=round] ( 28.93,147.34) -- (271.45,147.34);

\path[draw=drawColor,line width= 0.6pt,line join=round] ( 28.93,147.34) -- (271.45,147.34);

\path[draw=drawColor,line width= 0.6pt,line join=round] ( 28.93,147.34) -- (271.45,147.34);

\path[draw=drawColor,line width= 0.6pt,line join=round] ( 28.93,147.34) -- (271.45,147.34);

\path[draw=drawColor,line width= 0.6pt,line join=round] ( 28.93,147.34) -- (271.45,147.34);

\path[draw=drawColor,line width= 0.6pt,line join=round] ( 28.93,147.34) -- (271.45,147.34);

\path[draw=drawColor,line width= 0.6pt,line join=round] ( 28.93,147.34) -- (271.45,147.34);

\path[draw=drawColor,line width= 0.6pt,line join=round] ( 28.93,147.34) -- (271.45,147.34);

\path[draw=drawColor,line width= 0.6pt,line join=round] ( 28.93,147.34) -- (271.45,147.34);

\path[draw=drawColor,line width= 0.6pt,line join=round] ( 28.93,147.34) -- (271.45,147.34);

\path[draw=drawColor,line width= 0.6pt,line join=round] ( 28.93,147.34) -- (271.45,147.34);

\path[draw=drawColor,line width= 0.6pt,line join=round] ( 28.93,147.34) -- (271.45,147.34);

\path[draw=drawColor,line width= 0.6pt,line join=round] ( 28.93,147.34) -- (271.45,147.34);

\path[draw=drawColor,line width= 0.6pt,line join=round] ( 28.93,147.34) -- (271.45,147.34);

\path[draw=drawColor,line width= 0.6pt,line join=round] ( 28.93,147.34) -- (271.45,147.34);

\path[draw=drawColor,line width= 0.6pt,line join=round] ( 28.93,147.34) -- (271.45,147.34);

\path[draw=drawColor,line width= 0.6pt,line join=round] ( 28.93,147.34) -- (271.45,147.34);

\path[draw=drawColor,line width= 0.6pt,line join=round] ( 28.93,147.34) -- (271.45,147.34);

\path[draw=drawColor,line width= 0.6pt,line join=round] ( 28.93,147.34) -- (271.45,147.34);

\path[draw=drawColor,line width= 0.6pt,line join=round] ( 28.93,147.34) -- (271.45,147.34);

\path[draw=drawColor,line width= 0.6pt,line join=round] ( 28.93,147.34) -- (271.45,147.34);

\path[draw=drawColor,line width= 0.6pt,line join=round] ( 28.93,147.34) -- (271.45,147.34);

\path[draw=drawColor,line width= 0.6pt,line join=round] ( 28.93,147.34) -- (271.45,147.34);

\path[draw=drawColor,line width= 0.6pt,line join=round] ( 28.93,147.34) -- (271.45,147.34);

\path[draw=drawColor,line width= 0.6pt,line join=round] ( 28.93,147.34) -- (271.45,147.34);

\path[draw=drawColor,line width= 0.6pt,line join=round] ( 28.93,147.34) -- (271.45,147.34);

\path[draw=drawColor,line width= 0.6pt,line join=round] ( 28.93,147.34) -- (271.45,147.34);

\path[draw=drawColor,line width= 0.6pt,line join=round] ( 28.93,147.34) -- (271.45,147.34);

\path[draw=drawColor,line width= 0.6pt,line join=round] ( 28.93,147.34) -- (271.45,147.34);

\path[draw=drawColor,line width= 0.6pt,line join=round] ( 28.93,147.34) -- (271.45,147.34);

\path[draw=drawColor,line width= 0.6pt,line join=round] ( 28.93,147.34) -- (271.45,147.34);

\path[draw=drawColor,line width= 0.6pt,line join=round] ( 28.93,147.34) -- (271.45,147.34);

\path[draw=drawColor,line width= 0.6pt,line join=round] ( 28.93,147.34) -- (271.45,147.34);

\path[draw=drawColor,line width= 0.6pt,line join=round] ( 28.93,147.34) -- (271.45,147.34);

\path[draw=drawColor,line width= 0.6pt,line join=round] ( 28.93,147.34) -- (271.45,147.34);

\path[draw=drawColor,line width= 0.6pt,line join=round] ( 28.93,147.34) -- (271.45,147.34);

\path[draw=drawColor,line width= 0.6pt,line join=round] ( 28.93,147.34) -- (271.45,147.34);

\path[draw=drawColor,line width= 0.6pt,line join=round] ( 28.93,147.34) -- (271.45,147.34);

\path[draw=drawColor,line width= 0.6pt,line join=round] ( 28.93,147.34) -- (271.45,147.34);

\path[draw=drawColor,line width= 0.6pt,line join=round] ( 28.93,147.34) -- (271.45,147.34);

\path[draw=drawColor,line width= 0.6pt,line join=round] ( 28.93,147.34) -- (271.45,147.34);

\path[draw=drawColor,line width= 0.6pt,line join=round] ( 28.93,147.34) -- (271.45,147.34);

\path[draw=drawColor,line width= 0.6pt,line join=round] ( 28.93,147.34) -- (271.45,147.34);

\path[draw=drawColor,line width= 0.6pt,line join=round] ( 28.93,147.34) -- (271.45,147.34);

\path[draw=drawColor,line width= 0.6pt,line join=round] ( 28.93,147.34) -- (271.45,147.34);

\path[draw=drawColor,line width= 0.6pt,line join=round] ( 28.93,147.34) -- (271.45,147.34);

\path[draw=drawColor,line width= 0.6pt,line join=round] ( 28.93,147.34) -- (271.45,147.34);

\path[draw=drawColor,line width= 0.6pt,line join=round] ( 28.93,147.34) -- (271.45,147.34);

\path[draw=drawColor,line width= 0.6pt,line join=round] ( 28.93,147.34) -- (271.45,147.34);

\path[draw=drawColor,line width= 0.6pt,line join=round] ( 28.93,147.34) -- (271.45,147.34);

\path[draw=drawColor,line width= 0.6pt,line join=round] ( 28.93,147.34) -- (271.45,147.34);

\path[draw=drawColor,line width= 0.6pt,line join=round] ( 28.93,147.34) -- (271.45,147.34);

\path[draw=drawColor,line width= 0.6pt,line join=round] ( 28.93,147.34) -- (271.45,147.34);

\path[draw=drawColor,line width= 0.6pt,line join=round] ( 28.93,147.34) -- (271.45,147.34);

\path[draw=drawColor,line width= 0.6pt,line join=round] ( 28.93,147.34) -- (271.45,147.34);

\path[draw=drawColor,line width= 0.6pt,line join=round] ( 28.93,147.34) -- (271.45,147.34);

\path[draw=drawColor,line width= 0.6pt,line join=round] ( 28.93,147.34) -- (271.45,147.34);

\path[draw=drawColor,line width= 0.6pt,line join=round] ( 28.93,147.34) -- (271.45,147.34);

\path[draw=drawColor,line width= 0.6pt,line join=round] ( 28.93,147.34) -- (271.45,147.34);

\path[draw=drawColor,line width= 0.6pt,line join=round] ( 28.93,147.34) -- (271.45,147.34);

\path[draw=drawColor,line width= 0.6pt,line join=round] ( 28.93,147.34) -- (271.45,147.34);

\path[draw=drawColor,line width= 0.6pt,line join=round] ( 28.93,147.34) -- (271.45,147.34);

\path[draw=drawColor,line width= 0.6pt,line join=round] ( 28.93,147.34) -- (271.45,147.34);

\path[draw=drawColor,line width= 0.6pt,line join=round] ( 28.93,147.34) -- (271.45,147.34);

\path[draw=drawColor,line width= 0.6pt,line join=round] ( 28.93,147.34) -- (271.45,147.34);

\path[draw=drawColor,line width= 0.6pt,line join=round] ( 28.93,147.34) -- (271.45,147.34);

\path[draw=drawColor,line width= 0.6pt,line join=round] ( 28.93,147.34) -- (271.45,147.34);

\path[draw=drawColor,line width= 0.6pt,line join=round] ( 28.93,147.34) -- (271.45,147.34);

\path[draw=drawColor,line width= 0.6pt,line join=round] ( 28.93,147.34) -- (271.45,147.34);

\path[draw=drawColor,line width= 0.6pt,line join=round] ( 28.93,147.34) -- (271.45,147.34);

\path[draw=drawColor,line width= 0.6pt,line join=round] ( 28.93,147.34) -- (271.45,147.34);

\path[draw=drawColor,line width= 0.6pt,line join=round] ( 28.93,147.34) -- (271.45,147.34);

\path[draw=drawColor,line width= 0.6pt,line join=round] ( 28.93,147.34) -- (271.45,147.34);

\path[draw=drawColor,line width= 0.6pt,line join=round] ( 28.93,147.34) -- (271.45,147.34);

\path[draw=drawColor,line width= 0.6pt,line join=round] ( 28.93,147.34) -- (271.45,147.34);

\path[draw=drawColor,line width= 0.6pt,line join=round] ( 28.93,147.34) -- (271.45,147.34);

\path[draw=drawColor,line width= 0.6pt,line join=round] ( 28.93,147.34) -- (271.45,147.34);

\path[draw=drawColor,line width= 0.6pt,line join=round] ( 28.93,147.34) -- (271.45,147.34);

\path[draw=drawColor,line width= 0.6pt,line join=round] ( 28.93,147.34) -- (271.45,147.34);

\path[draw=drawColor,line width= 0.6pt,line join=round] ( 28.93,147.34) -- (271.45,147.34);

\path[draw=drawColor,line width= 0.6pt,line join=round] ( 28.93,147.34) -- (271.45,147.34);

\path[draw=drawColor,line width= 0.6pt,line join=round] ( 28.93,147.34) -- (271.45,147.34);

\path[draw=drawColor,line width= 0.6pt,line join=round] ( 28.93,147.34) -- (271.45,147.34);

\path[draw=drawColor,line width= 0.6pt,line join=round] ( 28.93,147.34) -- (271.45,147.34);

\path[draw=drawColor,line width= 0.6pt,line join=round] ( 28.93,147.34) -- (271.45,147.34);

\path[draw=drawColor,line width= 0.6pt,line join=round] ( 28.93,147.34) -- (271.45,147.34);

\path[draw=drawColor,line width= 0.6pt,line join=round] ( 28.93,147.34) -- (271.45,147.34);

\path[draw=drawColor,line width= 0.6pt,line join=round] ( 28.93,147.34) -- (271.45,147.34);

\path[draw=drawColor,line width= 0.6pt,line join=round] ( 28.93,147.34) -- (271.45,147.34);

\path[draw=drawColor,line width= 0.6pt,line join=round] ( 28.93,147.34) -- (271.45,147.34);

\path[draw=drawColor,line width= 0.6pt,line join=round] ( 28.93,147.34) -- (271.45,147.34);

\path[draw=drawColor,line width= 0.6pt,line join=round] ( 28.93,147.34) -- (271.45,147.34);

\path[draw=drawColor,line width= 0.6pt,line join=round] ( 28.93,147.34) -- (271.45,147.34);

\path[draw=drawColor,line width= 0.6pt,line join=round] ( 28.93,147.34) -- (271.45,147.34);

\path[draw=drawColor,line width= 0.6pt,line join=round] ( 28.93,147.34) -- (271.45,147.34);

\path[draw=drawColor,line width= 0.6pt,line join=round] ( 28.93,147.34) -- (271.45,147.34);

\path[draw=drawColor,line width= 0.6pt,line join=round] ( 28.93,147.34) -- (271.45,147.34);

\path[draw=drawColor,line width= 0.6pt,line join=round] ( 28.93,147.34) -- (271.45,147.34);

\path[draw=drawColor,line width= 0.6pt,line join=round] ( 28.93,147.34) -- (271.45,147.34);

\path[draw=drawColor,line width= 0.6pt,line join=round] ( 28.93,147.34) -- (271.45,147.34);

\path[draw=drawColor,line width= 0.6pt,line join=round] ( 28.93,147.34) -- (271.45,147.34);

\path[draw=drawColor,line width= 0.6pt,line join=round] ( 28.93,147.34) -- (271.45,147.34);

\path[draw=drawColor,line width= 0.6pt,line join=round] ( 28.93,147.34) -- (271.45,147.34);

\path[draw=drawColor,line width= 0.6pt,line join=round] ( 28.93,147.34) -- (271.45,147.34);

\path[draw=drawColor,line width= 0.6pt,line join=round] ( 28.93,147.34) -- (271.45,147.34);

\path[draw=drawColor,line width= 0.6pt,line join=round] ( 28.93,147.34) -- (271.45,147.34);

\path[draw=drawColor,line width= 0.6pt,line join=round] ( 28.93,147.34) -- (271.45,147.34);

\path[draw=drawColor,line width= 0.6pt,line join=round] ( 28.93,147.34) -- (271.45,147.34);

\path[draw=drawColor,line width= 0.6pt,line join=round] ( 28.93,147.34) -- (271.45,147.34);

\path[draw=drawColor,line width= 0.6pt,line join=round] ( 28.93,147.34) -- (271.45,147.34);

\path[draw=drawColor,line width= 0.6pt,line join=round] ( 28.93,147.34) -- (271.45,147.34);

\path[draw=drawColor,line width= 0.6pt,line join=round] ( 28.93,147.34) -- (271.45,147.34);

\path[draw=drawColor,line width= 0.6pt,line join=round] ( 28.93,147.34) -- (271.45,147.34);

\path[draw=drawColor,line width= 0.6pt,line join=round] ( 28.93,147.34) -- (271.45,147.34);

\path[draw=drawColor,line width= 0.6pt,line join=round] ( 28.93,147.34) -- (271.45,147.34);

\path[draw=drawColor,line width= 0.6pt,line join=round] ( 28.93,147.34) -- (271.45,147.34);

\path[draw=drawColor,line width= 0.6pt,line join=round] ( 28.93,147.34) -- (271.45,147.34);

\path[draw=drawColor,line width= 0.6pt,line join=round] ( 28.93,147.34) -- (271.45,147.34);

\path[draw=drawColor,line width= 0.6pt,line join=round] ( 28.93,147.34) -- (271.45,147.34);

\path[draw=drawColor,line width= 0.6pt,line join=round] ( 28.93,147.34) -- (271.45,147.34);

\path[draw=drawColor,line width= 0.6pt,line join=round] ( 28.93,147.34) -- (271.45,147.34);

\path[draw=drawColor,line width= 0.6pt,line join=round] ( 28.93,147.34) -- (271.45,147.34);

\path[draw=drawColor,line width= 0.6pt,line join=round] ( 28.93,147.34) -- (271.45,147.34);

\path[draw=drawColor,line width= 0.6pt,line join=round] ( 28.93,147.34) -- (271.45,147.34);

\path[draw=drawColor,line width= 0.6pt,line join=round] ( 28.93,147.34) -- (271.45,147.34);

\path[draw=drawColor,line width= 0.6pt,line join=round] ( 28.93,147.34) -- (271.45,147.34);

\path[draw=drawColor,line width= 0.6pt,line join=round] ( 28.93,147.34) -- (271.45,147.34);

\path[draw=drawColor,line width= 0.6pt,line join=round] ( 28.93,147.34) -- (271.45,147.34);

\path[draw=drawColor,line width= 0.6pt,line join=round] ( 28.93,147.34) -- (271.45,147.34);

\path[draw=drawColor,line width= 0.6pt,line join=round] ( 28.93,147.34) -- (271.45,147.34);

\path[draw=drawColor,line width= 0.6pt,line join=round] ( 28.93,147.34) -- (271.45,147.34);

\path[draw=drawColor,line width= 0.6pt,line join=round] ( 28.93,147.34) -- (271.45,147.34);

\path[draw=drawColor,line width= 0.6pt,line join=round] ( 28.93,147.34) -- (271.45,147.34);

\path[draw=drawColor,line width= 0.6pt,line join=round] ( 28.93,147.34) -- (271.45,147.34);

\path[draw=drawColor,line width= 0.6pt,line join=round] ( 28.93,147.34) -- (271.45,147.34);

\path[draw=drawColor,line width= 0.6pt,line join=round] ( 28.93,147.34) -- (271.45,147.34);

\path[draw=drawColor,line width= 0.6pt,line join=round] ( 28.93,147.34) -- (271.45,147.34);

\path[draw=drawColor,line width= 0.6pt,line join=round] ( 28.93,147.34) -- (271.45,147.34);

\path[draw=drawColor,line width= 0.6pt,line join=round] ( 28.93,147.34) -- (271.45,147.34);

\path[draw=drawColor,line width= 0.6pt,line join=round] ( 28.93,147.34) -- (271.45,147.34);

\path[draw=drawColor,line width= 0.6pt,line join=round] ( 28.93,147.34) -- (271.45,147.34);

\path[draw=drawColor,line width= 0.6pt,line join=round] ( 28.93,147.34) -- (271.45,147.34);

\path[draw=drawColor,line width= 0.6pt,line join=round] ( 28.93,147.34) -- (271.45,147.34);

\path[draw=drawColor,line width= 0.6pt,line join=round] ( 28.93,147.34) -- (271.45,147.34);

\path[draw=drawColor,line width= 0.6pt,line join=round] ( 28.93,147.34) -- (271.45,147.34);

\path[draw=drawColor,line width= 0.6pt,line join=round] ( 28.93,147.34) -- (271.45,147.34);

\path[draw=drawColor,line width= 0.6pt,line join=round] ( 28.93,147.34) -- (271.45,147.34);

\path[draw=drawColor,line width= 0.6pt,line join=round] ( 28.93,147.34) -- (271.45,147.34);

\path[draw=drawColor,line width= 0.6pt,line join=round] ( 28.93,147.34) -- (271.45,147.34);

\path[draw=drawColor,line width= 0.6pt,line join=round] ( 28.93,147.34) -- (271.45,147.34);

\path[draw=drawColor,line width= 0.6pt,line join=round] ( 28.93,147.34) -- (271.45,147.34);

\path[draw=drawColor,line width= 0.6pt,line join=round] ( 28.93,147.34) -- (271.45,147.34);

\path[draw=drawColor,line width= 0.6pt,line join=round] ( 28.93,147.34) -- (271.45,147.34);

\path[draw=drawColor,line width= 0.6pt,line join=round] ( 28.93,147.34) -- (271.45,147.34);

\path[draw=drawColor,line width= 0.6pt,line join=round] ( 28.93,147.34) -- (271.45,147.34);

\path[draw=drawColor,line width= 0.6pt,line join=round] ( 28.93,147.34) -- (271.45,147.34);

\path[draw=drawColor,line width= 0.6pt,line join=round] ( 28.93,147.34) -- (271.45,147.34);

\path[draw=drawColor,line width= 0.6pt,line join=round] ( 28.93,147.34) -- (271.45,147.34);

\path[draw=drawColor,line width= 0.6pt,line join=round] ( 28.93,147.34) -- (271.45,147.34);

\path[draw=drawColor,line width= 0.6pt,line join=round] ( 28.93,147.34) -- (271.45,147.34);

\path[draw=drawColor,line width= 0.6pt,line join=round] ( 28.93,147.34) -- (271.45,147.34);

\path[draw=drawColor,line width= 0.6pt,line join=round] ( 28.93,147.34) -- (271.45,147.34);

\path[draw=drawColor,line width= 0.6pt,line join=round] ( 28.93,147.34) -- (271.45,147.34);

\path[draw=drawColor,line width= 0.6pt,line join=round] ( 28.93,147.34) -- (271.45,147.34);

\path[draw=drawColor,line width= 0.6pt,line join=round] ( 28.93,147.34) -- (271.45,147.34);

\path[draw=drawColor,line width= 0.6pt,line join=round] ( 28.93,147.34) -- (271.45,147.34);

\path[draw=drawColor,line width= 0.6pt,line join=round] ( 28.93,147.34) -- (271.45,147.34);

\path[draw=drawColor,line width= 0.6pt,line join=round] ( 28.93,147.34) -- (271.45,147.34);

\path[draw=drawColor,line width= 0.6pt,line join=round] ( 28.93,147.34) -- (271.45,147.34);

\path[draw=drawColor,line width= 0.6pt,line join=round] ( 28.93,147.34) -- (271.45,147.34);

\path[draw=drawColor,line width= 0.6pt,line join=round] ( 28.93,147.34) -- (271.45,147.34);

\path[draw=drawColor,line width= 0.6pt,line join=round] ( 28.93,147.34) -- (271.45,147.34);

\path[draw=drawColor,line width= 0.6pt,line join=round] ( 28.93,147.34) -- (271.45,147.34);

\path[draw=drawColor,line width= 0.6pt,line join=round] ( 28.93,147.34) -- (271.45,147.34);

\path[draw=drawColor,line width= 0.6pt,line join=round] ( 28.93,147.34) -- (271.45,147.34);

\path[draw=drawColor,line width= 0.6pt,line join=round] ( 28.93,147.34) -- (271.45,147.34);

\path[draw=drawColor,line width= 0.6pt,line join=round] ( 28.93,147.34) -- (271.45,147.34);

\path[draw=drawColor,line width= 0.6pt,line join=round] ( 28.93,147.34) -- (271.45,147.34);

\path[draw=drawColor,line width= 0.6pt,line join=round] ( 28.93,147.34) -- (271.45,147.34);

\path[draw=drawColor,line width= 0.6pt,line join=round] ( 28.93,147.34) -- (271.45,147.34);

\path[draw=drawColor,line width= 0.6pt,line join=round] ( 28.93,147.34) -- (271.45,147.34);

\path[draw=drawColor,line width= 0.6pt,line join=round] ( 28.93,147.34) -- (271.45,147.34);

\path[draw=drawColor,line width= 0.6pt,line join=round] ( 28.93,147.34) -- (271.45,147.34);

\path[draw=drawColor,line width= 0.6pt,line join=round] ( 28.93,147.34) -- (271.45,147.34);

\path[draw=drawColor,line width= 0.6pt,line join=round] ( 28.93,147.34) -- (271.45,147.34);

\path[draw=drawColor,line width= 0.6pt,line join=round] ( 28.93,147.34) -- (271.45,147.34);

\path[draw=drawColor,line width= 0.6pt,line join=round] ( 28.93,147.34) -- (271.45,147.34);

\path[draw=drawColor,line width= 0.6pt,line join=round] ( 28.93,147.34) -- (271.45,147.34);

\path[draw=drawColor,line width= 0.6pt,line join=round] ( 28.93,147.34) -- (271.45,147.34);

\path[draw=drawColor,line width= 0.6pt,line join=round] ( 28.93,147.34) -- (271.45,147.34);

\path[draw=drawColor,line width= 0.6pt,line join=round] ( 28.93,147.34) -- (271.45,147.34);

\path[draw=drawColor,line width= 0.6pt,line join=round] ( 28.93,147.34) -- (271.45,147.34);

\path[draw=drawColor,line width= 0.6pt,line join=round] ( 28.93,147.34) -- (271.45,147.34);

\path[draw=drawColor,line width= 0.6pt,line join=round] ( 28.93,147.34) -- (271.45,147.34);

\path[draw=drawColor,line width= 0.6pt,line join=round] ( 28.93,147.34) -- (271.45,147.34);

\path[draw=drawColor,line width= 0.6pt,line join=round] ( 28.93,147.34) -- (271.45,147.34);

\path[draw=drawColor,line width= 0.6pt,line join=round] ( 28.93,147.34) -- (271.45,147.34);

\path[draw=drawColor,line width= 0.6pt,line join=round] ( 28.93,147.34) -- (271.45,147.34);

\path[draw=drawColor,line width= 0.6pt,line join=round] ( 28.93,147.34) -- (271.45,147.34);

\path[draw=drawColor,line width= 0.6pt,line join=round] ( 28.93,147.34) -- (271.45,147.34);

\path[draw=drawColor,line width= 0.6pt,line join=round] ( 28.93,147.34) -- (271.45,147.34);

\path[draw=drawColor,line width= 0.6pt,line join=round] ( 28.93,147.34) -- (271.45,147.34);

\path[draw=drawColor,line width= 0.6pt,line join=round] ( 28.93,147.34) -- (271.45,147.34);

\path[draw=drawColor,line width= 0.6pt,line join=round] ( 28.93,147.34) -- (271.45,147.34);

\path[draw=drawColor,line width= 0.6pt,line join=round] ( 28.93,147.34) -- (271.45,147.34);

\path[draw=drawColor,line width= 0.6pt,line join=round] ( 28.93,147.34) -- (271.45,147.34);

\path[draw=drawColor,line width= 0.6pt,line join=round] ( 28.93,147.34) -- (271.45,147.34);

\path[draw=drawColor,line width= 0.6pt,line join=round] ( 28.93,147.34) -- (271.45,147.34);

\path[draw=drawColor,line width= 0.6pt,line join=round] ( 28.93,147.34) -- (271.45,147.34);

\path[draw=drawColor,line width= 0.6pt,line join=round] ( 28.93,147.34) -- (271.45,147.34);

\path[draw=drawColor,line width= 0.6pt,line join=round] ( 28.93,147.34) -- (271.45,147.34);

\path[draw=drawColor,line width= 0.6pt,line join=round] ( 28.93,147.34) -- (271.45,147.34);

\path[draw=drawColor,line width= 0.6pt,line join=round] ( 28.93,147.34) -- (271.45,147.34);

\path[draw=drawColor,line width= 0.6pt,line join=round] ( 28.93,147.34) -- (271.45,147.34);

\path[draw=drawColor,line width= 0.6pt,line join=round] ( 28.93,147.34) -- (271.45,147.34);

\path[draw=drawColor,line width= 0.6pt,line join=round] ( 28.93,147.34) -- (271.45,147.34);

\path[draw=drawColor,line width= 0.6pt,line join=round] ( 28.93,147.34) -- (271.45,147.34);

\path[draw=drawColor,line width= 0.6pt,line join=round] ( 28.93,147.34) -- (271.45,147.34);

\path[draw=drawColor,line width= 0.6pt,line join=round] ( 28.93,147.34) -- (271.45,147.34);

\path[draw=drawColor,line width= 0.6pt,line join=round] ( 28.93,147.34) -- (271.45,147.34);

\path[draw=drawColor,line width= 0.6pt,line join=round] ( 28.93,147.34) -- (271.45,147.34);

\path[draw=drawColor,line width= 0.6pt,line join=round] ( 28.93,147.34) -- (271.45,147.34);

\path[draw=drawColor,line width= 0.6pt,line join=round] ( 28.93,147.34) -- (271.45,147.34);

\path[draw=drawColor,line width= 0.6pt,line join=round] ( 28.93,147.34) -- (271.45,147.34);

\path[draw=drawColor,line width= 0.6pt,line join=round] ( 28.93,147.34) -- (271.45,147.34);

\path[draw=drawColor,line width= 0.6pt,line join=round] ( 28.93,147.34) -- (271.45,147.34);

\path[draw=drawColor,line width= 0.6pt,line join=round] ( 28.93,147.34) -- (271.45,147.34);

\path[draw=drawColor,line width= 0.6pt,line join=round] ( 28.93,147.34) -- (271.45,147.34);

\path[draw=drawColor,line width= 0.6pt,line join=round] ( 28.93,147.34) -- (271.45,147.34);

\path[draw=drawColor,line width= 0.6pt,line join=round] ( 28.93,147.34) -- (271.45,147.34);

\path[draw=drawColor,line width= 0.6pt,line join=round] ( 28.93,147.34) -- (271.45,147.34);

\path[draw=drawColor,line width= 0.6pt,line join=round] ( 28.93,147.34) -- (271.45,147.34);

\path[draw=drawColor,line width= 0.6pt,line join=round] ( 28.93,147.34) -- (271.45,147.34);

\path[draw=drawColor,line width= 0.6pt,line join=round] ( 28.93,147.34) -- (271.45,147.34);

\path[draw=drawColor,line width= 0.6pt,line join=round] ( 28.93,147.34) -- (271.45,147.34);

\path[draw=drawColor,line width= 0.6pt,line join=round] ( 28.93,147.34) -- (271.45,147.34);

\path[draw=drawColor,line width= 0.6pt,line join=round] ( 28.93,147.34) -- (271.45,147.34);

\path[draw=drawColor,line width= 0.6pt,line join=round] ( 28.93,147.34) -- (271.45,147.34);

\path[draw=drawColor,line width= 0.6pt,line join=round] ( 28.93,147.34) -- (271.45,147.34);

\path[draw=drawColor,line width= 0.6pt,line join=round] ( 28.93,147.34) -- (271.45,147.34);

\path[draw=drawColor,line width= 0.6pt,line join=round] ( 28.93,147.34) -- (271.45,147.34);

\path[draw=drawColor,line width= 0.6pt,line join=round] ( 28.93,147.34) -- (271.45,147.34);

\path[draw=drawColor,line width= 0.6pt,line join=round] ( 28.93,147.34) -- (271.45,147.34);

\path[draw=drawColor,line width= 0.6pt,line join=round] ( 28.93,147.34) -- (271.45,147.34);

\path[draw=drawColor,line width= 0.6pt,line join=round] ( 28.93,147.34) -- (271.45,147.34);

\path[draw=drawColor,line width= 0.6pt,line join=round] ( 28.93,147.34) -- (271.45,147.34);

\path[draw=drawColor,line width= 0.6pt,line join=round] ( 28.93,147.34) -- (271.45,147.34);

\path[draw=drawColor,line width= 0.6pt,line join=round] ( 28.93,147.34) -- (271.45,147.34);

\path[draw=drawColor,line width= 0.6pt,line join=round] ( 28.93,147.34) -- (271.45,147.34);

\path[draw=drawColor,line width= 0.6pt,line join=round] ( 28.93,147.34) -- (271.45,147.34);

\path[draw=drawColor,line width= 0.6pt,line join=round] ( 28.93,147.34) -- (271.45,147.34);

\path[draw=drawColor,line width= 0.6pt,line join=round] ( 28.93,147.34) -- (271.45,147.34);

\path[draw=drawColor,line width= 0.6pt,line join=round] ( 28.93,147.34) -- (271.45,147.34);

\path[draw=drawColor,line width= 0.6pt,line join=round] ( 28.93,147.34) -- (271.45,147.34);

\path[draw=drawColor,line width= 0.6pt,line join=round] ( 28.93,147.34) -- (271.45,147.34);

\path[draw=drawColor,line width= 0.6pt,line join=round] ( 28.93,147.34) -- (271.45,147.34);

\path[draw=drawColor,line width= 0.6pt,line join=round] ( 28.93,147.34) -- (271.45,147.34);

\path[draw=drawColor,line width= 0.6pt,line join=round] ( 28.93,147.34) -- (271.45,147.34);

\path[draw=drawColor,line width= 0.6pt,line join=round] ( 28.93,147.34) -- (271.45,147.34);

\path[draw=drawColor,line width= 0.6pt,line join=round] ( 28.93,147.34) -- (271.45,147.34);

\path[draw=drawColor,line width= 0.6pt,line join=round] ( 28.93,147.34) -- (271.45,147.34);

\path[draw=drawColor,line width= 0.6pt,line join=round] ( 28.93,147.34) -- (271.45,147.34);

\path[draw=drawColor,line width= 0.6pt,line join=round] ( 28.93,147.34) -- (271.45,147.34);

\path[draw=drawColor,line width= 0.6pt,line join=round] ( 28.93,147.34) -- (271.45,147.34);

\path[draw=drawColor,line width= 0.6pt,line join=round] ( 28.93,147.34) -- (271.45,147.34);

\path[draw=drawColor,line width= 0.6pt,line join=round] ( 28.93,147.34) -- (271.45,147.34);

\path[draw=drawColor,line width= 0.6pt,line join=round] ( 28.93,147.34) -- (271.45,147.34);

\path[draw=drawColor,line width= 0.6pt,line join=round] ( 28.93,147.34) -- (271.45,147.34);

\path[draw=drawColor,line width= 0.6pt,line join=round] ( 28.93,147.34) -- (271.45,147.34);

\path[draw=drawColor,line width= 0.6pt,line join=round] ( 28.93,147.34) -- (271.45,147.34);

\path[draw=drawColor,line width= 0.6pt,line join=round] ( 28.93,147.34) -- (271.45,147.34);

\path[draw=drawColor,line width= 0.6pt,line join=round] ( 28.93,147.34) -- (271.45,147.34);

\path[draw=drawColor,line width= 0.6pt,line join=round] ( 28.93,147.34) -- (271.45,147.34);

\path[draw=drawColor,line width= 0.6pt,line join=round] ( 28.93,147.34) -- (271.45,147.34);

\path[draw=drawColor,line width= 0.6pt,line join=round] ( 28.93,147.34) -- (271.45,147.34);

\path[draw=drawColor,line width= 0.6pt,line join=round] ( 28.93,147.34) -- (271.45,147.34);

\path[draw=drawColor,line width= 0.6pt,line join=round] ( 28.93,147.34) -- (271.45,147.34);

\path[draw=drawColor,line width= 0.6pt,line join=round] ( 28.93,147.34) -- (271.45,147.34);

\path[draw=drawColor,line width= 0.6pt,line join=round] ( 28.93,147.34) -- (271.45,147.34);

\path[draw=drawColor,line width= 0.6pt,line join=round] ( 28.93,147.34) -- (271.45,147.34);

\path[draw=drawColor,line width= 0.6pt,line join=round] ( 28.93,147.34) -- (271.45,147.34);

\path[draw=drawColor,line width= 0.6pt,line join=round] ( 28.93,147.34) -- (271.45,147.34);

\path[draw=drawColor,line width= 0.6pt,line join=round] ( 28.93,147.34) -- (271.45,147.34);

\path[draw=drawColor,line width= 0.6pt,line join=round] ( 28.93,147.34) -- (271.45,147.34);

\path[draw=drawColor,line width= 0.6pt,line join=round] ( 28.93,147.34) -- (271.45,147.34);

\path[draw=drawColor,line width= 0.6pt,line join=round] ( 28.93,147.34) -- (271.45,147.34);

\path[draw=drawColor,line width= 0.6pt,line join=round] ( 28.93,147.34) -- (271.45,147.34);

\path[draw=drawColor,line width= 0.6pt,line join=round] ( 28.93,147.34) -- (271.45,147.34);

\path[draw=drawColor,line width= 0.6pt,line join=round] ( 28.93,147.34) -- (271.45,147.34);

\path[draw=drawColor,line width= 0.6pt,line join=round] ( 28.93,147.34) -- (271.45,147.34);

\path[draw=drawColor,line width= 0.6pt,line join=round] ( 28.93,147.34) -- (271.45,147.34);

\path[draw=drawColor,line width= 0.6pt,line join=round] ( 28.93,147.34) -- (271.45,147.34);

\path[draw=drawColor,line width= 0.6pt,line join=round] ( 28.93,147.34) -- (271.45,147.34);

\path[draw=drawColor,line width= 0.6pt,line join=round] ( 28.93,147.34) -- (271.45,147.34);

\path[draw=drawColor,line width= 0.6pt,line join=round] ( 28.93,147.34) -- (271.45,147.34);

\path[draw=drawColor,line width= 0.6pt,line join=round] ( 28.93,147.34) -- (271.45,147.34);

\path[draw=drawColor,line width= 0.6pt,line join=round] ( 28.93,147.34) -- (271.45,147.34);

\path[draw=drawColor,line width= 0.6pt,line join=round] ( 28.93,147.34) -- (271.45,147.34);

\path[draw=drawColor,line width= 0.6pt,line join=round] ( 28.93,147.34) -- (271.45,147.34);

\path[draw=drawColor,line width= 0.6pt,line join=round] ( 28.93,147.34) -- (271.45,147.34);

\path[draw=drawColor,line width= 0.6pt,line join=round] ( 28.93,147.34) -- (271.45,147.34);

\path[draw=drawColor,line width= 0.6pt,line join=round] ( 28.93,147.34) -- (271.45,147.34);

\path[draw=drawColor,line width= 0.6pt,line join=round] ( 28.93,147.34) -- (271.45,147.34);

\path[draw=drawColor,line width= 0.6pt,line join=round] ( 28.93,147.34) -- (271.45,147.34);

\path[draw=drawColor,line width= 0.6pt,line join=round] ( 28.93,147.34) -- (271.45,147.34);

\path[draw=drawColor,line width= 0.6pt,line join=round] ( 28.93,147.34) -- (271.45,147.34);

\path[draw=drawColor,line width= 0.6pt,line join=round] ( 28.93,147.34) -- (271.45,147.34);

\path[draw=drawColor,line width= 0.6pt,line join=round] ( 28.93,147.34) -- (271.45,147.34);

\path[draw=drawColor,line width= 0.6pt,line join=round] ( 28.93,147.34) -- (271.45,147.34);

\path[draw=drawColor,line width= 0.6pt,line join=round] ( 28.93,147.34) -- (271.45,147.34);

\path[draw=drawColor,line width= 0.6pt,line join=round] ( 28.93,147.34) -- (271.45,147.34);

\path[draw=drawColor,line width= 0.6pt,line join=round] ( 28.93,147.34) -- (271.45,147.34);

\path[draw=drawColor,line width= 0.6pt,line join=round] ( 28.93,147.34) -- (271.45,147.34);

\path[draw=drawColor,line width= 0.6pt,line join=round] ( 28.93,147.34) -- (271.45,147.34);

\path[draw=drawColor,line width= 0.6pt,line join=round] ( 28.93,147.34) -- (271.45,147.34);

\path[draw=drawColor,line width= 0.6pt,line join=round] ( 28.93,147.34) -- (271.45,147.34);

\path[draw=drawColor,line width= 0.6pt,line join=round] ( 28.93,147.34) -- (271.45,147.34);

\path[draw=drawColor,line width= 0.6pt,line join=round] ( 28.93,147.34) -- (271.45,147.34);

\path[draw=drawColor,line width= 0.6pt,line join=round] ( 28.93,147.34) -- (271.45,147.34);

\path[draw=drawColor,line width= 0.6pt,line join=round] ( 28.93,147.34) -- (271.45,147.34);

\path[draw=drawColor,line width= 0.6pt,line join=round] ( 28.93,147.34) -- (271.45,147.34);

\path[draw=drawColor,line width= 0.6pt,line join=round] ( 28.93,147.34) -- (271.45,147.34);

\path[draw=drawColor,line width= 0.6pt,line join=round] ( 28.93,147.34) -- (271.45,147.34);

\path[draw=drawColor,line width= 0.6pt,line join=round] ( 28.93,147.34) -- (271.45,147.34);

\path[draw=drawColor,line width= 0.6pt,line join=round] ( 28.93,147.34) -- (271.45,147.34);

\path[draw=drawColor,line width= 0.6pt,line join=round] ( 28.93,147.34) -- (271.45,147.34);

\path[draw=drawColor,line width= 0.6pt,line join=round] ( 28.93,147.34) -- (271.45,147.34);

\path[draw=drawColor,line width= 0.6pt,line join=round] ( 28.93,147.34) -- (271.45,147.34);

\path[draw=drawColor,line width= 0.6pt,line join=round] ( 28.93,147.34) -- (271.45,147.34);

\path[draw=drawColor,line width= 0.6pt,line join=round] ( 28.93,147.34) -- (271.45,147.34);

\path[draw=drawColor,line width= 0.6pt,line join=round] ( 28.93,147.34) -- (271.45,147.34);

\path[draw=drawColor,line width= 0.6pt,line join=round] ( 28.93,147.34) -- (271.45,147.34);

\path[draw=drawColor,line width= 0.6pt,line join=round] ( 28.93,147.34) -- (271.45,147.34);

\path[draw=drawColor,line width= 0.6pt,line join=round] ( 28.93,147.34) -- (271.45,147.34);

\path[draw=drawColor,line width= 0.6pt,line join=round] ( 28.93,147.34) -- (271.45,147.34);

\path[draw=drawColor,line width= 0.6pt,line join=round] ( 28.93,147.34) -- (271.45,147.34);

\path[draw=drawColor,line width= 0.6pt,line join=round] ( 28.93,147.34) -- (271.45,147.34);

\path[draw=drawColor,line width= 0.6pt,line join=round] ( 28.93,147.34) -- (271.45,147.34);

\path[draw=drawColor,line width= 0.6pt,line join=round] ( 28.93,147.34) -- (271.45,147.34);

\path[draw=drawColor,line width= 0.6pt,line join=round] ( 28.93,147.34) -- (271.45,147.34);

\path[draw=drawColor,line width= 0.6pt,line join=round] ( 28.93,147.34) -- (271.45,147.34);

\path[draw=drawColor,line width= 0.6pt,line join=round] ( 28.93,147.34) -- (271.45,147.34);

\path[draw=drawColor,line width= 0.6pt,line join=round] ( 28.93,147.34) -- (271.45,147.34);

\path[draw=drawColor,line width= 0.6pt,line join=round] ( 28.93,147.34) -- (271.45,147.34);

\path[draw=drawColor,line width= 0.6pt,line join=round] ( 28.93,147.34) -- (271.45,147.34);

\path[draw=drawColor,line width= 0.6pt,line join=round] ( 28.93,147.34) -- (271.45,147.34);

\path[draw=drawColor,line width= 0.6pt,line join=round] ( 28.93,147.34) -- (271.45,147.34);

\path[draw=drawColor,line width= 0.6pt,line join=round] ( 28.93,147.34) -- (271.45,147.34);

\path[draw=drawColor,line width= 0.6pt,line join=round] ( 28.93,147.34) -- (271.45,147.34);

\path[draw=drawColor,line width= 0.6pt,line join=round] ( 28.93,147.34) -- (271.45,147.34);

\path[draw=drawColor,line width= 0.6pt,line join=round] ( 28.93,147.34) -- (271.45,147.34);

\path[draw=drawColor,line width= 0.6pt,line join=round] ( 28.93,147.34) -- (271.45,147.34);

\path[draw=drawColor,line width= 0.6pt,line join=round] ( 28.93,147.34) -- (271.45,147.34);

\path[draw=drawColor,line width= 0.6pt,line join=round] ( 28.93,147.34) -- (271.45,147.34);

\path[draw=drawColor,line width= 0.6pt,line join=round] ( 28.93,147.34) -- (271.45,147.34);

\path[draw=drawColor,line width= 0.6pt,line join=round] ( 28.93,147.34) -- (271.45,147.34);

\path[draw=drawColor,line width= 0.6pt,line join=round] ( 28.93,147.34) -- (271.45,147.34);

\path[draw=drawColor,line width= 0.6pt,line join=round] ( 28.93,147.34) -- (271.45,147.34);

\path[draw=drawColor,line width= 0.6pt,line join=round] ( 28.93,147.34) -- (271.45,147.34);

\path[draw=drawColor,line width= 0.6pt,line join=round] ( 28.93,147.34) -- (271.45,147.34);

\path[draw=drawColor,line width= 0.6pt,line join=round] ( 28.93,147.34) -- (271.45,147.34);

\path[draw=drawColor,line width= 0.6pt,line join=round] ( 28.93,147.34) -- (271.45,147.34);

\path[draw=drawColor,line width= 0.6pt,line join=round] ( 28.93,147.34) -- (271.45,147.34);

\path[draw=drawColor,line width= 0.6pt,line join=round] ( 28.93,147.34) -- (271.45,147.34);

\path[draw=drawColor,line width= 0.6pt,line join=round] ( 28.93,147.34) -- (271.45,147.34);

\path[draw=drawColor,line width= 0.6pt,line join=round] ( 28.93,147.34) -- (271.45,147.34);

\path[draw=drawColor,line width= 0.6pt,line join=round] ( 28.93,147.34) -- (271.45,147.34);

\path[draw=drawColor,line width= 0.6pt,line join=round] ( 28.93,147.34) -- (271.45,147.34);

\path[draw=drawColor,line width= 0.6pt,line join=round] ( 28.93,147.34) -- (271.45,147.34);

\path[draw=drawColor,line width= 0.6pt,line join=round] ( 28.93,147.34) -- (271.45,147.34);

\path[draw=drawColor,line width= 0.6pt,line join=round] ( 28.93,147.34) -- (271.45,147.34);

\path[draw=drawColor,line width= 0.6pt,line join=round] ( 28.93,147.34) -- (271.45,147.34);

\path[draw=drawColor,line width= 0.6pt,line join=round] ( 28.93,147.34) -- (271.45,147.34);

\path[draw=drawColor,line width= 0.6pt,line join=round] ( 28.93,147.34) -- (271.45,147.34);

\path[draw=drawColor,line width= 0.6pt,line join=round] ( 28.93,147.34) -- (271.45,147.34);

\path[draw=drawColor,line width= 0.6pt,line join=round] ( 28.93,147.34) -- (271.45,147.34);

\path[draw=drawColor,line width= 0.6pt,line join=round] ( 28.93,147.34) -- (271.45,147.34);

\path[draw=drawColor,line width= 0.6pt,line join=round] ( 28.93,147.34) -- (271.45,147.34);

\path[draw=drawColor,line width= 0.6pt,line join=round] ( 28.93,147.34) -- (271.45,147.34);

\path[draw=drawColor,line width= 0.6pt,line join=round] ( 28.93,147.34) -- (271.45,147.34);

\path[draw=drawColor,line width= 0.6pt,line join=round] ( 28.93,147.34) -- (271.45,147.34);

\path[draw=drawColor,line width= 0.6pt,line join=round] ( 28.93,147.34) -- (271.45,147.34);

\path[draw=drawColor,line width= 0.6pt,line join=round] ( 28.93,147.34) -- (271.45,147.34);

\path[draw=drawColor,line width= 0.6pt,line join=round] ( 28.93,147.34) -- (271.45,147.34);

\path[draw=drawColor,line width= 0.6pt,line join=round] ( 28.93,147.34) -- (271.45,147.34);

\path[draw=drawColor,line width= 0.6pt,line join=round] ( 28.93,147.34) -- (271.45,147.34);

\path[draw=drawColor,line width= 0.6pt,line join=round] ( 28.93,147.34) -- (271.45,147.34);

\path[draw=drawColor,line width= 0.6pt,line join=round] ( 28.93,147.34) -- (271.45,147.34);

\path[draw=drawColor,line width= 0.6pt,line join=round] ( 28.93,147.34) -- (271.45,147.34);

\path[draw=drawColor,line width= 0.6pt,line join=round] ( 28.93,147.34) -- (271.45,147.34);

\path[draw=drawColor,line width= 0.6pt,line join=round] ( 28.93,147.34) -- (271.45,147.34);

\path[draw=drawColor,line width= 0.6pt,line join=round] ( 28.93,147.34) -- (271.45,147.34);

\path[draw=drawColor,line width= 0.6pt,line join=round] ( 28.93,147.34) -- (271.45,147.34);

\path[draw=drawColor,line width= 0.6pt,line join=round] ( 28.93,147.34) -- (271.45,147.34);

\path[draw=drawColor,line width= 0.6pt,line join=round] ( 28.93,147.34) -- (271.45,147.34);

\path[draw=drawColor,line width= 0.6pt,line join=round] ( 28.93,147.34) -- (271.45,147.34);

\path[draw=drawColor,line width= 0.6pt,line join=round] ( 28.93,147.34) -- (271.45,147.34);

\path[draw=drawColor,line width= 0.6pt,line join=round] ( 28.93,147.34) -- (271.45,147.34);

\path[draw=drawColor,line width= 0.6pt,line join=round] ( 28.93,147.34) -- (271.45,147.34);

\path[draw=drawColor,line width= 0.6pt,line join=round] ( 28.93,147.34) -- (271.45,147.34);

\path[draw=drawColor,line width= 0.6pt,line join=round] ( 28.93,147.34) -- (271.45,147.34);

\path[draw=drawColor,line width= 0.6pt,line join=round] ( 28.93,147.34) -- (271.45,147.34);

\path[draw=drawColor,line width= 0.6pt,line join=round] ( 28.93,147.34) -- (271.45,147.34);

\path[draw=drawColor,line width= 0.6pt,line join=round] ( 28.93,147.34) -- (271.45,147.34);

\path[draw=drawColor,line width= 0.6pt,line join=round] ( 28.93,147.34) -- (271.45,147.34);

\path[draw=drawColor,line width= 0.6pt,line join=round] ( 28.93,147.34) -- (271.45,147.34);

\path[draw=drawColor,line width= 0.6pt,line join=round] ( 28.93,147.34) -- (271.45,147.34);

\path[draw=drawColor,line width= 0.6pt,line join=round] ( 28.93,147.34) -- (271.45,147.34);

\path[draw=drawColor,line width= 0.6pt,line join=round] ( 28.93,147.34) -- (271.45,147.34);

\path[draw=drawColor,line width= 0.6pt,line join=round] ( 28.93,147.34) -- (271.45,147.34);

\path[draw=drawColor,line width= 0.6pt,line join=round] ( 28.93,147.34) -- (271.45,147.34);

\path[draw=drawColor,line width= 0.6pt,line join=round] ( 28.93,147.34) -- (271.45,147.34);

\path[draw=drawColor,line width= 0.6pt,line join=round] ( 28.93,147.34) -- (271.45,147.34);

\path[draw=drawColor,line width= 0.6pt,line join=round] ( 28.93,147.34) -- (271.45,147.34);

\path[draw=drawColor,line width= 0.6pt,line join=round] ( 28.93,147.34) -- (271.45,147.34);

\path[draw=drawColor,line width= 0.6pt,line join=round] ( 28.93,147.34) -- (271.45,147.34);

\path[draw=drawColor,line width= 0.6pt,line join=round] ( 28.93,147.34) -- (271.45,147.34);

\path[draw=drawColor,line width= 0.6pt,line join=round] ( 28.93,147.34) -- (271.45,147.34);

\path[draw=drawColor,line width= 0.6pt,line join=round] ( 28.93,147.34) -- (271.45,147.34);

\path[draw=drawColor,line width= 0.6pt,line join=round] ( 28.93,147.34) -- (271.45,147.34);

\path[draw=drawColor,line width= 0.6pt,line join=round] ( 28.93,147.34) -- (271.45,147.34);

\path[draw=drawColor,line width= 0.6pt,line join=round] ( 28.93,147.34) -- (271.45,147.34);

\path[draw=drawColor,line width= 0.6pt,line join=round] ( 28.93,147.34) -- (271.45,147.34);

\path[draw=drawColor,line width= 0.6pt,line join=round] ( 28.93,147.34) -- (271.45,147.34);

\path[draw=drawColor,line width= 0.6pt,line join=round] ( 28.93,147.34) -- (271.45,147.34);

\path[draw=drawColor,line width= 0.6pt,line join=round] ( 28.93,147.34) -- (271.45,147.34);

\path[draw=drawColor,line width= 0.6pt,line join=round] ( 28.93,147.34) -- (271.45,147.34);

\path[draw=drawColor,line width= 0.6pt,line join=round] ( 28.93,147.34) -- (271.45,147.34);

\path[draw=drawColor,line width= 0.6pt,line join=round] ( 28.93,147.34) -- (271.45,147.34);

\path[draw=drawColor,line width= 0.6pt,line join=round] ( 28.93,147.34) -- (271.45,147.34);

\path[draw=drawColor,line width= 0.6pt,line join=round] ( 28.93,147.34) -- (271.45,147.34);

\path[draw=drawColor,line width= 0.6pt,line join=round] ( 28.93,147.34) -- (271.45,147.34);

\path[draw=drawColor,line width= 0.6pt,line join=round] ( 28.93,147.34) -- (271.45,147.34);

\path[draw=drawColor,line width= 0.6pt,line join=round] ( 28.93,147.34) -- (271.45,147.34);

\path[draw=drawColor,line width= 0.6pt,line join=round] ( 28.93,147.34) -- (271.45,147.34);

\path[draw=drawColor,line width= 0.6pt,line join=round] ( 28.93,147.34) -- (271.45,147.34);

\path[draw=drawColor,line width= 0.6pt,line join=round] ( 28.93,147.34) -- (271.45,147.34);

\path[draw=drawColor,line width= 0.6pt,line join=round] ( 28.93,147.34) -- (271.45,147.34);

\path[draw=drawColor,line width= 0.6pt,line join=round] ( 28.93,147.34) -- (271.45,147.34);

\path[draw=drawColor,line width= 0.6pt,line join=round] ( 28.93,147.34) -- (271.45,147.34);

\path[draw=drawColor,line width= 0.6pt,line join=round] ( 28.93,147.34) -- (271.45,147.34);

\path[draw=drawColor,line width= 0.6pt,line join=round] ( 28.93,147.34) -- (271.45,147.34);

\path[draw=drawColor,line width= 0.6pt,line join=round] ( 28.93,147.34) -- (271.45,147.34);

\path[draw=drawColor,line width= 0.6pt,line join=round] ( 28.93,147.34) -- (271.45,147.34);

\path[draw=drawColor,line width= 0.6pt,line join=round] ( 28.93,147.34) -- (271.45,147.34);

\path[draw=drawColor,line width= 0.6pt,line join=round] ( 28.93,147.34) -- (271.45,147.34);

\path[draw=drawColor,line width= 0.6pt,line join=round] ( 28.93,147.34) -- (271.45,147.34);

\path[draw=drawColor,line width= 0.6pt,line join=round] ( 28.93,147.34) -- (271.45,147.34);

\path[draw=drawColor,line width= 0.6pt,line join=round] ( 28.93,147.34) -- (271.45,147.34);

\path[draw=drawColor,line width= 0.6pt,line join=round] ( 28.93,147.34) -- (271.45,147.34);

\path[draw=drawColor,line width= 0.6pt,line join=round] ( 28.93,147.34) -- (271.45,147.34);

\path[draw=drawColor,line width= 0.6pt,line join=round] ( 28.93,147.34) -- (271.45,147.34);

\path[draw=drawColor,line width= 0.6pt,line join=round] ( 28.93,147.34) -- (271.45,147.34);

\path[draw=drawColor,line width= 0.6pt,line join=round] ( 28.93,147.34) -- (271.45,147.34);

\path[draw=drawColor,line width= 0.6pt,line join=round] ( 28.93,147.34) -- (271.45,147.34);

\path[draw=drawColor,line width= 0.6pt,line join=round] ( 28.93,147.34) -- (271.45,147.34);

\path[draw=drawColor,line width= 0.6pt,line join=round] ( 28.93,147.34) -- (271.45,147.34);

\path[draw=drawColor,line width= 0.6pt,line join=round] ( 28.93,147.34) -- (271.45,147.34);

\path[draw=drawColor,line width= 0.6pt,line join=round] ( 28.93,147.34) -- (271.45,147.34);

\path[draw=drawColor,line width= 0.6pt,line join=round] ( 28.93,147.34) -- (271.45,147.34);

\path[draw=drawColor,line width= 0.6pt,line join=round] ( 28.93,147.34) -- (271.45,147.34);

\path[draw=drawColor,line width= 0.6pt,line join=round] ( 28.93,147.34) -- (271.45,147.34);

\path[draw=drawColor,line width= 0.6pt,line join=round] ( 28.93,147.34) -- (271.45,147.34);

\path[draw=drawColor,line width= 0.6pt,line join=round] ( 28.93,147.34) -- (271.45,147.34);

\path[draw=drawColor,line width= 0.6pt,line join=round] ( 28.93,147.34) -- (271.45,147.34);

\path[draw=drawColor,line width= 0.6pt,line join=round] ( 28.93,147.34) -- (271.45,147.34);

\path[draw=drawColor,line width= 0.6pt,line join=round] ( 28.93,147.34) -- (271.45,147.34);

\path[draw=drawColor,line width= 0.6pt,line join=round] ( 28.93,147.34) -- (271.45,147.34);

\path[draw=drawColor,line width= 0.6pt,line join=round] ( 28.93,147.34) -- (271.45,147.34);

\path[draw=drawColor,line width= 0.6pt,line join=round] ( 28.93,147.34) -- (271.45,147.34);

\path[draw=drawColor,line width= 0.6pt,line join=round] ( 28.93,147.34) -- (271.45,147.34);

\path[draw=drawColor,line width= 0.6pt,line join=round] ( 28.93,147.34) -- (271.45,147.34);

\path[draw=drawColor,line width= 0.6pt,line join=round] ( 28.93,147.34) -- (271.45,147.34);

\path[draw=drawColor,line width= 0.6pt,line join=round] ( 28.93,147.34) -- (271.45,147.34);

\path[draw=drawColor,line width= 0.6pt,line join=round] ( 28.93,147.34) -- (271.45,147.34);

\path[draw=drawColor,line width= 0.6pt,line join=round] ( 28.93,147.34) -- (271.45,147.34);

\path[draw=drawColor,line width= 0.6pt,line join=round] ( 28.93,147.34) -- (271.45,147.34);

\path[draw=drawColor,line width= 0.6pt,line join=round] ( 28.93,147.34) -- (271.45,147.34);

\path[draw=drawColor,line width= 0.6pt,line join=round] ( 28.93,147.34) -- (271.45,147.34);

\path[draw=drawColor,line width= 0.6pt,line join=round] ( 28.93,147.34) -- (271.45,147.34);

\path[draw=drawColor,line width= 0.6pt,line join=round] ( 28.93,147.34) -- (271.45,147.34);

\path[draw=drawColor,line width= 0.6pt,line join=round] ( 28.93,147.34) -- (271.45,147.34);

\path[draw=drawColor,line width= 0.6pt,line join=round] ( 28.93,147.34) -- (271.45,147.34);

\path[draw=drawColor,line width= 0.6pt,line join=round] ( 28.93,147.34) -- (271.45,147.34);

\path[draw=drawColor,line width= 0.6pt,line join=round] ( 28.93,147.34) -- (271.45,147.34);

\path[draw=drawColor,line width= 0.6pt,line join=round] ( 28.93,147.34) -- (271.45,147.34);

\path[draw=drawColor,line width= 0.6pt,line join=round] ( 28.93,147.34) -- (271.45,147.34);

\path[draw=drawColor,line width= 0.6pt,line join=round] ( 28.93,147.34) -- (271.45,147.34);

\path[draw=drawColor,line width= 0.6pt,line join=round] ( 28.93,147.34) -- (271.45,147.34);

\path[draw=drawColor,line width= 0.6pt,line join=round] ( 28.93,147.34) -- (271.45,147.34);

\path[draw=drawColor,line width= 0.6pt,line join=round] ( 28.93,147.34) -- (271.45,147.34);

\path[draw=drawColor,line width= 0.6pt,line join=round] ( 28.93,147.34) -- (271.45,147.34);

\path[draw=drawColor,line width= 0.6pt,line join=round] ( 28.93,147.34) -- (271.45,147.34);

\path[draw=drawColor,line width= 0.6pt,line join=round] ( 28.93,147.34) -- (271.45,147.34);

\path[draw=drawColor,line width= 0.6pt,line join=round] ( 28.93,147.34) -- (271.45,147.34);

\path[draw=drawColor,line width= 0.6pt,line join=round] ( 28.93,147.34) -- (271.45,147.34);

\path[draw=drawColor,line width= 0.6pt,line join=round] ( 28.93,147.34) -- (271.45,147.34);

\path[draw=drawColor,line width= 0.6pt,line join=round] ( 28.93,147.34) -- (271.45,147.34);

\path[draw=drawColor,line width= 0.6pt,line join=round] ( 28.93,147.34) -- (271.45,147.34);

\path[draw=drawColor,line width= 0.6pt,line join=round] ( 28.93,147.34) -- (271.45,147.34);

\path[draw=drawColor,line width= 0.6pt,line join=round] ( 28.93,147.34) -- (271.45,147.34);

\path[draw=drawColor,line width= 0.6pt,line join=round] ( 28.93,147.34) -- (271.45,147.34);

\path[draw=drawColor,line width= 0.6pt,line join=round] ( 28.93,147.34) -- (271.45,147.34);

\path[draw=drawColor,line width= 0.6pt,line join=round] ( 28.93,147.34) -- (271.45,147.34);

\path[draw=drawColor,line width= 0.6pt,line join=round] ( 28.93,147.34) -- (271.45,147.34);

\path[draw=drawColor,line width= 0.6pt,line join=round] ( 28.93,147.34) -- (271.45,147.34);

\path[draw=drawColor,line width= 0.6pt,line join=round] ( 28.93,147.34) -- (271.45,147.34);

\path[draw=drawColor,line width= 0.6pt,line join=round] ( 28.93,147.34) -- (271.45,147.34);

\path[draw=drawColor,line width= 0.6pt,line join=round] ( 28.93,147.34) -- (271.45,147.34);

\path[draw=drawColor,line width= 0.6pt,line join=round] ( 28.93,147.34) -- (271.45,147.34);

\path[draw=drawColor,line width= 0.6pt,line join=round] ( 28.93,147.34) -- (271.45,147.34);

\path[draw=drawColor,line width= 0.6pt,line join=round] ( 28.93,147.34) -- (271.45,147.34);

\path[draw=drawColor,line width= 0.6pt,line join=round] ( 28.93,147.34) -- (271.45,147.34);

\path[draw=drawColor,line width= 0.6pt,line join=round] ( 28.93,147.34) -- (271.45,147.34);

\path[draw=drawColor,line width= 0.6pt,line join=round] ( 28.93,147.34) -- (271.45,147.34);

\path[draw=drawColor,line width= 0.6pt,line join=round] ( 28.93,147.34) -- (271.45,147.34);

\path[draw=drawColor,line width= 0.6pt,line join=round] ( 28.93,147.34) -- (271.45,147.34);

\path[draw=drawColor,line width= 0.6pt,line join=round] ( 28.93,147.34) -- (271.45,147.34);

\path[draw=drawColor,line width= 0.6pt,line join=round] ( 28.93,147.34) -- (271.45,147.34);

\path[draw=drawColor,line width= 0.6pt,line join=round] ( 28.93,147.34) -- (271.45,147.34);

\path[draw=drawColor,line width= 0.6pt,line join=round] ( 28.93,147.34) -- (271.45,147.34);

\path[draw=drawColor,line width= 0.6pt,line join=round] ( 28.93,147.34) -- (271.45,147.34);

\path[draw=drawColor,line width= 0.6pt,line join=round] ( 28.93,147.34) -- (271.45,147.34);

\path[draw=drawColor,line width= 0.6pt,line join=round] ( 28.93,147.34) -- (271.45,147.34);

\path[draw=drawColor,line width= 0.6pt,line join=round] ( 28.93,147.34) -- (271.45,147.34);

\path[draw=drawColor,line width= 0.6pt,line join=round] ( 28.93,147.34) -- (271.45,147.34);

\path[draw=drawColor,line width= 0.6pt,line join=round] ( 28.93,147.34) -- (271.45,147.34);

\path[draw=drawColor,line width= 0.6pt,line join=round] ( 28.93,147.34) -- (271.45,147.34);

\path[draw=drawColor,line width= 0.6pt,line join=round] ( 28.93,147.34) -- (271.45,147.34);

\path[draw=drawColor,line width= 0.6pt,line join=round] ( 28.93,147.34) -- (271.45,147.34);

\path[draw=drawColor,line width= 0.6pt,line join=round] ( 28.93,147.34) -- (271.45,147.34);

\path[draw=drawColor,line width= 0.6pt,line join=round] ( 28.93,147.34) -- (271.45,147.34);

\path[draw=drawColor,line width= 0.6pt,line join=round] ( 28.93,147.34) -- (271.45,147.34);

\path[draw=drawColor,line width= 0.6pt,line join=round] ( 28.93,147.34) -- (271.45,147.34);

\path[draw=drawColor,line width= 0.6pt,line join=round] ( 28.93,147.34) -- (271.45,147.34);

\path[draw=drawColor,line width= 0.6pt,line join=round] ( 28.93,147.34) -- (271.45,147.34);

\path[draw=drawColor,line width= 0.6pt,line join=round] ( 28.93,147.34) -- (271.45,147.34);

\path[draw=drawColor,line width= 0.6pt,line join=round] ( 28.93,147.34) -- (271.45,147.34);

\path[draw=drawColor,line width= 0.6pt,line join=round] ( 28.93,147.34) -- (271.45,147.34);

\path[draw=drawColor,line width= 0.6pt,line join=round] ( 28.93,147.34) -- (271.45,147.34);

\path[draw=drawColor,line width= 0.6pt,line join=round] ( 28.93,147.34) -- (271.45,147.34);

\path[draw=drawColor,line width= 0.6pt,line join=round] ( 28.93,147.34) -- (271.45,147.34);

\path[draw=drawColor,line width= 0.6pt,line join=round] ( 28.93,147.34) -- (271.45,147.34);

\path[draw=drawColor,line width= 0.6pt,line join=round] ( 28.93,147.34) -- (271.45,147.34);

\path[draw=drawColor,line width= 0.6pt,line join=round] ( 28.93,147.34) -- (271.45,147.34);

\path[draw=drawColor,line width= 0.6pt,line join=round] ( 28.93,147.34) -- (271.45,147.34);

\path[draw=drawColor,line width= 0.6pt,line join=round] ( 28.93,147.34) -- (271.45,147.34);

\path[draw=drawColor,line width= 0.6pt,line join=round] ( 28.93,147.34) -- (271.45,147.34);

\path[draw=drawColor,line width= 0.6pt,line join=round] ( 28.93,147.34) -- (271.45,147.34);

\path[draw=drawColor,line width= 0.6pt,line join=round] ( 28.93,147.34) -- (271.45,147.34);

\path[draw=drawColor,line width= 0.6pt,line join=round] ( 28.93,147.34) -- (271.45,147.34);

\path[draw=drawColor,line width= 0.6pt,line join=round] ( 28.93,147.34) -- (271.45,147.34);

\path[draw=drawColor,line width= 0.6pt,line join=round] ( 28.93,147.34) -- (271.45,147.34);

\path[draw=drawColor,line width= 0.6pt,line join=round] ( 28.93,147.34) -- (271.45,147.34);

\path[draw=drawColor,line width= 0.6pt,line join=round] ( 28.93,147.34) -- (271.45,147.34);

\path[draw=drawColor,line width= 0.6pt,line join=round] ( 28.93,147.34) -- (271.45,147.34);

\path[draw=drawColor,line width= 0.6pt,line join=round] ( 28.93,147.34) -- (271.45,147.34);

\path[draw=drawColor,line width= 0.6pt,line join=round] ( 28.93,147.34) -- (271.45,147.34);

\path[draw=drawColor,line width= 0.6pt,line join=round] ( 28.93,147.34) -- (271.45,147.34);

\path[draw=drawColor,line width= 0.6pt,line join=round] ( 28.93,147.34) -- (271.45,147.34);

\path[draw=drawColor,line width= 0.6pt,line join=round] ( 28.93,147.34) -- (271.45,147.34);

\path[draw=drawColor,line width= 0.6pt,line join=round] ( 28.93,147.34) -- (271.45,147.34);

\path[draw=drawColor,line width= 0.6pt,line join=round] ( 28.93,147.34) -- (271.45,147.34);

\path[draw=drawColor,line width= 0.6pt,line join=round] ( 28.93,147.34) -- (271.45,147.34);

\path[draw=drawColor,line width= 0.6pt,line join=round] ( 28.93,147.34) -- (271.45,147.34);

\path[draw=drawColor,line width= 0.6pt,line join=round] ( 28.93,147.34) -- (271.45,147.34);

\path[draw=drawColor,line width= 0.6pt,line join=round] ( 28.93,147.34) -- (271.45,147.34);

\path[draw=drawColor,line width= 0.6pt,line join=round] ( 28.93,147.34) -- (271.45,147.34);

\path[draw=drawColor,line width= 0.6pt,line join=round] ( 28.93,147.34) -- (271.45,147.34);

\path[draw=drawColor,line width= 0.6pt,line join=round] ( 28.93,147.34) -- (271.45,147.34);

\path[draw=drawColor,line width= 0.6pt,line join=round] ( 28.93,147.34) -- (271.45,147.34);

\path[draw=drawColor,line width= 0.6pt,line join=round] ( 28.93,147.34) -- (271.45,147.34);

\path[draw=drawColor,line width= 0.6pt,line join=round] ( 28.93,147.34) -- (271.45,147.34);

\path[draw=drawColor,line width= 0.6pt,line join=round] ( 28.93,147.34) -- (271.45,147.34);

\path[draw=drawColor,line width= 0.6pt,line join=round] ( 28.93,147.34) -- (271.45,147.34);

\path[draw=drawColor,line width= 0.6pt,line join=round] ( 28.93,147.34) -- (271.45,147.34);

\path[draw=drawColor,line width= 0.6pt,line join=round] ( 28.93,147.34) -- (271.45,147.34);

\path[draw=drawColor,line width= 0.6pt,line join=round] ( 28.93,147.34) -- (271.45,147.34);

\path[draw=drawColor,line width= 0.6pt,line join=round] ( 28.93,147.34) -- (271.45,147.34);

\path[draw=drawColor,line width= 0.6pt,line join=round] ( 28.93,147.34) -- (271.45,147.34);

\path[draw=drawColor,line width= 0.6pt,line join=round] ( 28.93,147.34) -- (271.45,147.34);

\path[draw=drawColor,line width= 0.6pt,line join=round] ( 28.93,147.34) -- (271.45,147.34);

\path[draw=drawColor,line width= 0.6pt,line join=round] ( 28.93,147.34) -- (271.45,147.34);

\path[draw=drawColor,line width= 0.6pt,line join=round] ( 28.93,147.34) -- (271.45,147.34);

\path[draw=drawColor,line width= 0.6pt,line join=round] ( 28.93,147.34) -- (271.45,147.34);

\path[draw=drawColor,line width= 0.6pt,line join=round] ( 28.93,147.34) -- (271.45,147.34);

\path[draw=drawColor,line width= 0.6pt,line join=round] ( 28.93,147.34) -- (271.45,147.34);

\path[draw=drawColor,line width= 0.6pt,line join=round] ( 28.93,147.34) -- (271.45,147.34);

\path[draw=drawColor,line width= 0.6pt,line join=round] ( 28.93,147.34) -- (271.45,147.34);

\path[draw=drawColor,line width= 0.6pt,line join=round] ( 28.93,147.34) -- (271.45,147.34);

\path[draw=drawColor,line width= 0.6pt,line join=round] ( 28.93,147.34) -- (271.45,147.34);

\path[draw=drawColor,line width= 0.6pt,line join=round] ( 28.93,147.34) -- (271.45,147.34);

\path[draw=drawColor,line width= 0.6pt,line join=round] ( 28.93,147.34) -- (271.45,147.34);

\path[draw=drawColor,line width= 0.6pt,line join=round] ( 28.93,147.34) -- (271.45,147.34);

\path[draw=drawColor,line width= 0.6pt,line join=round] ( 28.93,147.34) -- (271.45,147.34);

\path[draw=drawColor,line width= 0.6pt,line join=round] ( 28.93,147.34) -- (271.45,147.34);

\path[draw=drawColor,line width= 0.6pt,line join=round] ( 28.93,147.34) -- (271.45,147.34);

\path[draw=drawColor,line width= 0.6pt,line join=round] ( 28.93,147.34) -- (271.45,147.34);

\path[draw=drawColor,line width= 0.6pt,line join=round] ( 28.93,147.34) -- (271.45,147.34);

\path[draw=drawColor,line width= 0.6pt,line join=round] ( 28.93,147.34) -- (271.45,147.34);

\path[draw=drawColor,line width= 0.6pt,line join=round] ( 28.93,147.34) -- (271.45,147.34);

\path[draw=drawColor,line width= 0.6pt,line join=round] ( 28.93,147.34) -- (271.45,147.34);

\path[draw=drawColor,line width= 0.6pt,line join=round] ( 28.93,147.34) -- (271.45,147.34);

\path[draw=drawColor,line width= 0.6pt,line join=round] ( 28.93,147.34) -- (271.45,147.34);

\path[draw=drawColor,line width= 0.6pt,line join=round] ( 28.93,147.34) -- (271.45,147.34);

\path[draw=drawColor,line width= 0.6pt,line join=round] ( 28.93,147.34) -- (271.45,147.34);

\path[draw=drawColor,line width= 0.6pt,line join=round] ( 28.93,147.34) -- (271.45,147.34);

\path[draw=drawColor,line width= 0.6pt,line join=round] ( 28.93,147.34) -- (271.45,147.34);

\path[draw=drawColor,line width= 0.6pt,line join=round] ( 28.93,147.34) -- (271.45,147.34);

\path[draw=drawColor,line width= 0.6pt,line join=round] ( 28.93,147.34) -- (271.45,147.34);

\path[draw=drawColor,line width= 0.6pt,line join=round] ( 28.93,147.34) -- (271.45,147.34);

\path[draw=drawColor,line width= 0.6pt,line join=round] ( 28.93,147.34) -- (271.45,147.34);

\path[draw=drawColor,line width= 0.6pt,line join=round] ( 28.93,147.34) -- (271.45,147.34);

\path[draw=drawColor,line width= 0.6pt,line join=round] ( 28.93,147.34) -- (271.45,147.34);

\path[draw=drawColor,line width= 0.6pt,line join=round] ( 28.93,147.34) -- (271.45,147.34);

\path[draw=drawColor,line width= 0.6pt,line join=round] ( 28.93,147.34) -- (271.45,147.34);

\path[draw=drawColor,line width= 0.6pt,line join=round] ( 28.93,147.34) -- (271.45,147.34);

\path[draw=drawColor,line width= 0.6pt,line join=round] ( 28.93,147.34) -- (271.45,147.34);

\path[draw=drawColor,line width= 0.6pt,line join=round] ( 28.93,147.34) -- (271.45,147.34);

\path[draw=drawColor,line width= 0.6pt,line join=round] ( 28.93,147.34) -- (271.45,147.34);

\path[draw=drawColor,line width= 0.6pt,line join=round] ( 28.93,147.34) -- (271.45,147.34);

\path[draw=drawColor,line width= 0.6pt,line join=round] ( 28.93,147.34) -- (271.45,147.34);

\path[draw=drawColor,line width= 0.6pt,line join=round] ( 28.93,147.34) -- (271.45,147.34);

\path[draw=drawColor,line width= 0.6pt,line join=round] ( 28.93,147.34) -- (271.45,147.34);

\path[draw=drawColor,line width= 0.6pt,line join=round] ( 28.93,147.34) -- (271.45,147.34);

\path[draw=drawColor,line width= 0.6pt,line join=round] ( 28.93,147.34) -- (271.45,147.34);

\path[draw=drawColor,line width= 0.6pt,line join=round] ( 28.93,147.34) -- (271.45,147.34);

\path[draw=drawColor,line width= 0.6pt,line join=round] ( 28.93,147.34) -- (271.45,147.34);

\path[draw=drawColor,line width= 0.6pt,line join=round] ( 28.93,147.34) -- (271.45,147.34);

\path[draw=drawColor,line width= 0.6pt,line join=round] ( 28.93,147.34) -- (271.45,147.34);

\path[draw=drawColor,line width= 0.6pt,line join=round] ( 28.93,147.34) -- (271.45,147.34);

\path[draw=drawColor,line width= 0.6pt,line join=round] ( 28.93,147.34) -- (271.45,147.34);

\path[draw=drawColor,line width= 0.6pt,line join=round] ( 28.93,147.34) -- (271.45,147.34);

\path[draw=drawColor,line width= 0.6pt,line join=round] ( 28.93,147.34) -- (271.45,147.34);

\path[draw=drawColor,line width= 0.6pt,line join=round] ( 28.93,147.34) -- (271.45,147.34);

\path[draw=drawColor,line width= 0.6pt,line join=round] ( 28.93,147.34) -- (271.45,147.34);

\path[draw=drawColor,line width= 0.6pt,line join=round] ( 28.93,147.34) -- (271.45,147.34);

\path[draw=drawColor,line width= 0.6pt,line join=round] ( 28.93,147.34) -- (271.45,147.34);

\path[draw=drawColor,line width= 0.6pt,line join=round] ( 28.93,147.34) -- (271.45,147.34);

\path[draw=drawColor,line width= 0.6pt,line join=round] ( 28.93,147.34) -- (271.45,147.34);

\path[draw=drawColor,line width= 0.6pt,line join=round] ( 28.93,147.34) -- (271.45,147.34);

\path[draw=drawColor,line width= 0.6pt,line join=round] ( 28.93,147.34) -- (271.45,147.34);

\path[draw=drawColor,line width= 0.6pt,line join=round] ( 28.93,147.34) -- (271.45,147.34);

\path[draw=drawColor,line width= 0.6pt,line join=round] ( 28.93,147.34) -- (271.45,147.34);

\path[draw=drawColor,line width= 0.6pt,line join=round] ( 28.93,147.34) -- (271.45,147.34);

\path[draw=drawColor,line width= 0.6pt,line join=round] ( 28.93,147.34) -- (271.45,147.34);

\path[draw=drawColor,line width= 0.6pt,line join=round] ( 28.93,147.34) -- (271.45,147.34);

\path[draw=drawColor,line width= 0.6pt,line join=round] ( 28.93,147.34) -- (271.45,147.34);

\path[draw=drawColor,line width= 0.6pt,line join=round] ( 28.93,147.34) -- (271.45,147.34);

\path[draw=drawColor,line width= 0.6pt,line join=round] ( 28.93,147.34) -- (271.45,147.34);

\path[draw=drawColor,line width= 0.6pt,line join=round] ( 28.93,147.34) -- (271.45,147.34);

\path[draw=drawColor,line width= 0.6pt,line join=round] ( 28.93,147.34) -- (271.45,147.34);

\path[draw=drawColor,line width= 0.6pt,line join=round] ( 28.93,147.34) -- (271.45,147.34);

\path[draw=drawColor,line width= 0.6pt,line join=round] ( 28.93,147.34) -- (271.45,147.34);

\path[draw=drawColor,line width= 0.6pt,line join=round] ( 28.93,147.34) -- (271.45,147.34);

\path[draw=drawColor,line width= 0.6pt,line join=round] ( 28.93,147.34) -- (271.45,147.34);

\path[draw=drawColor,line width= 0.6pt,line join=round] ( 28.93,147.34) -- (271.45,147.34);

\path[draw=drawColor,line width= 0.6pt,line join=round] ( 28.93,147.34) -- (271.45,147.34);

\path[draw=drawColor,line width= 0.6pt,line join=round] ( 28.93,147.34) -- (271.45,147.34);

\path[draw=drawColor,line width= 0.6pt,line join=round] ( 28.93,147.34) -- (271.45,147.34);

\path[draw=drawColor,line width= 0.6pt,line join=round] ( 28.93,147.34) -- (271.45,147.34);

\path[draw=drawColor,line width= 0.6pt,line join=round] ( 28.93,147.34) -- (271.45,147.34);

\path[draw=drawColor,line width= 0.6pt,line join=round] ( 28.93,147.34) -- (271.45,147.34);

\path[draw=drawColor,line width= 0.6pt,line join=round] ( 28.93,147.34) -- (271.45,147.34);

\path[draw=drawColor,line width= 0.6pt,line join=round] ( 28.93,147.34) -- (271.45,147.34);

\path[draw=drawColor,line width= 0.6pt,line join=round] ( 28.93,147.34) -- (271.45,147.34);

\path[draw=drawColor,line width= 0.6pt,line join=round] ( 28.93,147.34) -- (271.45,147.34);

\path[draw=drawColor,line width= 0.6pt,line join=round] ( 28.93,147.34) -- (271.45,147.34);

\path[draw=drawColor,line width= 0.6pt,line join=round] ( 28.93,147.34) -- (271.45,147.34);

\path[draw=drawColor,line width= 0.6pt,line join=round] ( 28.93,147.34) -- (271.45,147.34);

\path[draw=drawColor,line width= 0.6pt,line join=round] ( 28.93,147.34) -- (271.45,147.34);

\path[draw=drawColor,line width= 0.6pt,line join=round] ( 28.93,147.34) -- (271.45,147.34);

\path[draw=drawColor,line width= 0.6pt,line join=round] ( 28.93,147.34) -- (271.45,147.34);

\path[draw=drawColor,line width= 0.6pt,line join=round] ( 28.93,147.34) -- (271.45,147.34);

\path[draw=drawColor,line width= 0.6pt,line join=round] ( 28.93,147.34) -- (271.45,147.34);

\path[draw=drawColor,line width= 0.6pt,line join=round] ( 28.93,147.34) -- (271.45,147.34);

\path[draw=drawColor,line width= 0.6pt,line join=round] ( 28.93,147.34) -- (271.45,147.34);

\path[draw=drawColor,line width= 0.6pt,line join=round] ( 28.93,147.34) -- (271.45,147.34);

\path[draw=drawColor,line width= 0.6pt,line join=round] ( 28.93,147.34) -- (271.45,147.34);

\path[draw=drawColor,line width= 0.6pt,line join=round] ( 28.93,147.34) -- (271.45,147.34);

\path[draw=drawColor,line width= 0.6pt,line join=round] ( 28.93,147.34) -- (271.45,147.34);

\path[draw=drawColor,line width= 0.6pt,line join=round] ( 28.93,147.34) -- (271.45,147.34);

\path[draw=drawColor,line width= 0.6pt,line join=round] ( 28.93,147.34) -- (271.45,147.34);

\path[draw=drawColor,line width= 0.6pt,line join=round] ( 28.93,147.34) -- (271.45,147.34);

\path[draw=drawColor,line width= 0.6pt,line join=round] ( 28.93,147.34) -- (271.45,147.34);

\path[draw=drawColor,line width= 0.6pt,line join=round] ( 28.93,147.34) -- (271.45,147.34);

\path[draw=drawColor,line width= 0.6pt,line join=round] ( 28.93,147.34) -- (271.45,147.34);

\path[draw=drawColor,line width= 0.6pt,line join=round] ( 28.93,147.34) -- (271.45,147.34);

\path[draw=drawColor,line width= 0.6pt,line join=round] ( 28.93,147.34) -- (271.45,147.34);

\path[draw=drawColor,line width= 0.6pt,line join=round] ( 28.93,147.34) -- (271.45,147.34);

\path[draw=drawColor,line width= 0.6pt,line join=round] ( 28.93,147.34) -- (271.45,147.34);

\path[draw=drawColor,line width= 0.6pt,line join=round] ( 28.93,147.34) -- (271.45,147.34);

\path[draw=drawColor,line width= 0.6pt,line join=round] ( 28.93,147.34) -- (271.45,147.34);

\path[draw=drawColor,line width= 0.6pt,line join=round] ( 28.93,147.34) -- (271.45,147.34);

\path[draw=drawColor,line width= 0.6pt,line join=round] ( 28.93,147.34) -- (271.45,147.34);

\path[draw=drawColor,line width= 0.6pt,line join=round] ( 28.93,147.34) -- (271.45,147.34);

\path[draw=drawColor,line width= 0.6pt,line join=round] ( 28.93,147.34) -- (271.45,147.34);

\path[draw=drawColor,line width= 0.6pt,line join=round] ( 28.93,147.34) -- (271.45,147.34);

\path[draw=drawColor,line width= 0.6pt,line join=round] ( 28.93,147.34) -- (271.45,147.34);

\path[draw=drawColor,line width= 0.6pt,line join=round] ( 28.93,147.34) -- (271.45,147.34);

\path[draw=drawColor,line width= 0.6pt,line join=round] ( 28.93,147.34) -- (271.45,147.34);

\path[draw=drawColor,line width= 0.6pt,line join=round] ( 28.93,147.34) -- (271.45,147.34);

\path[draw=drawColor,line width= 0.6pt,line join=round] ( 28.93,147.34) -- (271.45,147.34);

\path[draw=drawColor,line width= 0.6pt,line join=round] ( 28.93,147.34) -- (271.45,147.34);

\path[draw=drawColor,line width= 0.6pt,line join=round] ( 28.93,147.34) -- (271.45,147.34);

\path[draw=drawColor,line width= 0.6pt,line join=round] ( 28.93,147.34) -- (271.45,147.34);

\path[draw=drawColor,line width= 0.6pt,line join=round] ( 28.93,147.34) -- (271.45,147.34);

\path[draw=drawColor,line width= 0.6pt,line join=round] ( 28.93,147.34) -- (271.45,147.34);

\path[draw=drawColor,line width= 0.6pt,line join=round] ( 28.93,147.34) -- (271.45,147.34);

\path[draw=drawColor,line width= 0.6pt,line join=round] ( 28.93,147.34) -- (271.45,147.34);

\path[draw=drawColor,line width= 0.6pt,line join=round] ( 28.93,147.34) -- (271.45,147.34);

\path[draw=drawColor,line width= 0.6pt,line join=round] ( 28.93,147.34) -- (271.45,147.34);

\path[draw=drawColor,line width= 0.6pt,line join=round] ( 28.93,147.34) -- (271.45,147.34);

\path[draw=drawColor,line width= 0.6pt,line join=round] ( 28.93,147.34) -- (271.45,147.34);

\path[draw=drawColor,line width= 0.6pt,line join=round] ( 28.93,147.34) -- (271.45,147.34);

\path[draw=drawColor,line width= 0.6pt,line join=round] ( 28.93,147.34) -- (271.45,147.34);

\path[draw=drawColor,line width= 0.6pt,line join=round] ( 28.93,147.34) -- (271.45,147.34);

\path[draw=drawColor,line width= 0.6pt,line join=round] ( 28.93,147.34) -- (271.45,147.34);

\path[draw=drawColor,line width= 0.6pt,line join=round] ( 28.93,147.34) -- (271.45,147.34);

\path[draw=drawColor,line width= 0.6pt,line join=round] ( 28.93,147.34) -- (271.45,147.34);

\path[draw=drawColor,line width= 0.6pt,line join=round] ( 28.93,147.34) -- (271.45,147.34);

\path[draw=drawColor,line width= 0.6pt,line join=round] ( 28.93,147.34) -- (271.45,147.34);

\path[draw=drawColor,line width= 0.6pt,line join=round] ( 28.93,147.34) -- (271.45,147.34);

\path[draw=drawColor,line width= 0.6pt,line join=round] ( 28.93,147.34) -- (271.45,147.34);

\path[draw=drawColor,line width= 0.6pt,line join=round] ( 28.93,147.34) -- (271.45,147.34);

\path[draw=drawColor,line width= 0.6pt,line join=round] ( 28.93,147.34) -- (271.45,147.34);

\path[draw=drawColor,line width= 0.6pt,line join=round] ( 28.93,147.34) -- (271.45,147.34);

\path[draw=drawColor,line width= 0.6pt,line join=round] ( 28.93,147.34) -- (271.45,147.34);

\path[draw=drawColor,line width= 0.6pt,line join=round] ( 28.93,147.34) -- (271.45,147.34);

\path[draw=drawColor,line width= 0.6pt,line join=round] ( 28.93,147.34) -- (271.45,147.34);

\path[draw=drawColor,line width= 0.6pt,line join=round] ( 28.93,147.34) -- (271.45,147.34);

\path[draw=drawColor,line width= 0.6pt,line join=round] ( 28.93,147.34) -- (271.45,147.34);

\path[draw=drawColor,line width= 0.6pt,line join=round] ( 28.93,147.34) -- (271.45,147.34);

\path[draw=drawColor,line width= 0.6pt,line join=round] ( 28.93,147.34) -- (271.45,147.34);

\path[draw=drawColor,line width= 0.6pt,line join=round] ( 28.93,147.34) -- (271.45,147.34);

\path[draw=drawColor,line width= 0.6pt,line join=round] ( 28.93,147.34) -- (271.45,147.34);

\path[draw=drawColor,line width= 0.6pt,line join=round] ( 28.93,147.34) -- (271.45,147.34);

\path[draw=drawColor,line width= 0.6pt,line join=round] ( 28.93,147.34) -- (271.45,147.34);

\path[draw=drawColor,line width= 0.6pt,line join=round] ( 28.93,147.34) -- (271.45,147.34);

\path[draw=drawColor,line width= 0.6pt,line join=round] ( 28.93,147.34) -- (271.45,147.34);

\path[draw=drawColor,line width= 0.6pt,line join=round] ( 28.93,147.34) -- (271.45,147.34);

\path[draw=drawColor,line width= 0.6pt,line join=round] ( 28.93,147.34) -- (271.45,147.34);

\path[draw=drawColor,line width= 0.6pt,line join=round] ( 28.93,147.34) -- (271.45,147.34);

\path[draw=drawColor,line width= 0.6pt,line join=round] ( 28.93,147.34) -- (271.45,147.34);

\path[draw=drawColor,line width= 0.6pt,line join=round] ( 28.93,147.34) -- (271.45,147.34);

\path[draw=drawColor,line width= 0.6pt,line join=round] ( 28.93,147.34) -- (271.45,147.34);

\path[draw=drawColor,line width= 0.6pt,line join=round] ( 28.93,147.34) -- (271.45,147.34);

\path[draw=drawColor,line width= 0.6pt,line join=round] ( 28.93,147.34) -- (271.45,147.34);

\path[draw=drawColor,line width= 0.6pt,line join=round] ( 28.93,147.34) -- (271.45,147.34);

\path[draw=drawColor,line width= 0.6pt,line join=round] ( 28.93,147.34) -- (271.45,147.34);

\path[draw=drawColor,line width= 0.6pt,line join=round] ( 28.93,147.34) -- (271.45,147.34);

\path[draw=drawColor,line width= 0.6pt,line join=round] ( 28.93,147.34) -- (271.45,147.34);

\path[draw=drawColor,line width= 0.6pt,line join=round] ( 28.93,147.34) -- (271.45,147.34);

\path[draw=drawColor,line width= 0.6pt,line join=round] ( 28.93,147.34) -- (271.45,147.34);

\path[draw=drawColor,line width= 0.6pt,line join=round] ( 28.93,147.34) -- (271.45,147.34);

\path[draw=drawColor,line width= 0.6pt,line join=round] ( 28.93,147.34) -- (271.45,147.34);

\path[draw=drawColor,line width= 0.6pt,line join=round] ( 28.93,147.34) -- (271.45,147.34);

\path[draw=drawColor,line width= 0.6pt,line join=round] ( 28.93,147.34) -- (271.45,147.34);

\path[draw=drawColor,line width= 0.6pt,line join=round] ( 28.93,147.34) -- (271.45,147.34);

\path[draw=drawColor,line width= 0.6pt,line join=round] ( 28.93,147.34) -- (271.45,147.34);

\path[draw=drawColor,line width= 0.6pt,line join=round] ( 28.93,147.34) -- (271.45,147.34);

\path[draw=drawColor,line width= 0.6pt,line join=round] ( 28.93,147.34) -- (271.45,147.34);

\path[draw=drawColor,line width= 0.6pt,line join=round] ( 28.93,147.34) -- (271.45,147.34);

\path[draw=drawColor,line width= 0.6pt,line join=round] ( 28.93,147.34) -- (271.45,147.34);

\path[draw=drawColor,line width= 0.6pt,line join=round] ( 28.93,147.34) -- (271.45,147.34);

\path[draw=drawColor,line width= 0.6pt,line join=round] ( 28.93,147.34) -- (271.45,147.34);

\path[draw=drawColor,line width= 0.6pt,line join=round] ( 28.93,147.34) -- (271.45,147.34);

\path[draw=drawColor,line width= 0.6pt,line join=round] ( 28.93,147.34) -- (271.45,147.34);

\path[draw=drawColor,line width= 0.6pt,line join=round] ( 28.93,147.34) -- (271.45,147.34);

\path[draw=drawColor,line width= 0.6pt,line join=round] ( 28.93,147.34) -- (271.45,147.34);

\path[draw=drawColor,line width= 0.6pt,line join=round] ( 28.93,147.34) -- (271.45,147.34);

\path[draw=drawColor,line width= 0.6pt,line join=round] ( 28.93,147.34) -- (271.45,147.34);

\path[draw=drawColor,line width= 0.6pt,line join=round] ( 28.93,147.34) -- (271.45,147.34);

\path[draw=drawColor,line width= 0.6pt,line join=round] ( 28.93,147.34) -- (271.45,147.34);

\path[draw=drawColor,line width= 0.6pt,line join=round] ( 28.93,147.34) -- (271.45,147.34);

\path[draw=drawColor,line width= 0.6pt,line join=round] ( 28.93,147.34) -- (271.45,147.34);

\path[draw=drawColor,line width= 0.6pt,line join=round] ( 28.93,147.34) -- (271.45,147.34);

\path[draw=drawColor,line width= 0.6pt,line join=round] ( 28.93,147.34) -- (271.45,147.34);

\path[draw=drawColor,line width= 0.6pt,line join=round] ( 28.93,147.34) -- (271.45,147.34);

\path[draw=drawColor,line width= 0.6pt,line join=round] ( 28.93,147.34) -- (271.45,147.34);

\path[draw=drawColor,line width= 0.6pt,line join=round] ( 28.93,147.34) -- (271.45,147.34);

\path[draw=drawColor,line width= 0.6pt,line join=round] ( 28.93,147.34) -- (271.45,147.34);

\path[draw=drawColor,line width= 0.6pt,line join=round] ( 28.93,147.34) -- (271.45,147.34);

\path[draw=drawColor,line width= 0.6pt,line join=round] ( 28.93,147.34) -- (271.45,147.34);

\path[draw=drawColor,line width= 0.6pt,line join=round] ( 28.93,147.34) -- (271.45,147.34);

\path[draw=drawColor,line width= 0.6pt,line join=round] ( 28.93,147.34) -- (271.45,147.34);

\path[draw=drawColor,line width= 0.6pt,line join=round] ( 28.93,147.34) -- (271.45,147.34);

\path[draw=drawColor,line width= 0.6pt,line join=round] ( 28.93,147.34) -- (271.45,147.34);

\path[draw=drawColor,line width= 0.6pt,line join=round] ( 28.93,147.34) -- (271.45,147.34);

\path[draw=drawColor,line width= 0.6pt,line join=round] ( 28.93,147.34) -- (271.45,147.34);

\path[draw=drawColor,line width= 0.6pt,line join=round] ( 28.93,147.34) -- (271.45,147.34);

\path[draw=drawColor,line width= 0.6pt,line join=round] ( 28.93,147.34) -- (271.45,147.34);

\path[draw=drawColor,line width= 0.6pt,line join=round] ( 28.93,147.34) -- (271.45,147.34);

\path[draw=drawColor,line width= 0.6pt,line join=round] ( 28.93,147.34) -- (271.45,147.34);

\path[draw=drawColor,line width= 0.6pt,line join=round] ( 28.93,147.34) -- (271.45,147.34);

\path[draw=drawColor,line width= 0.6pt,line join=round] ( 28.93,147.34) -- (271.45,147.34);

\path[draw=drawColor,line width= 0.6pt,line join=round] ( 28.93,147.34) -- (271.45,147.34);

\path[draw=drawColor,line width= 0.6pt,line join=round] ( 28.93,147.34) -- (271.45,147.34);

\path[draw=drawColor,line width= 0.6pt,line join=round] ( 28.93,147.34) -- (271.45,147.34);

\path[draw=drawColor,line width= 0.6pt,line join=round] ( 28.93,147.34) -- (271.45,147.34);

\path[draw=drawColor,line width= 0.6pt,line join=round] ( 28.93,147.34) -- (271.45,147.34);

\path[draw=drawColor,line width= 0.6pt,line join=round] ( 28.93,147.34) -- (271.45,147.34);

\path[draw=drawColor,line width= 0.6pt,line join=round] ( 28.93,147.34) -- (271.45,147.34);

\path[draw=drawColor,line width= 0.6pt,line join=round] ( 28.93,147.34) -- (271.45,147.34);

\path[draw=drawColor,line width= 0.6pt,line join=round] ( 28.93,147.34) -- (271.45,147.34);

\path[draw=drawColor,line width= 0.6pt,line join=round] ( 28.93,147.34) -- (271.45,147.34);

\path[draw=drawColor,line width= 0.6pt,line join=round] ( 28.93,147.34) -- (271.45,147.34);

\path[draw=drawColor,line width= 0.6pt,line join=round] ( 28.93,147.34) -- (271.45,147.34);

\path[draw=drawColor,line width= 0.6pt,line join=round] ( 28.93,147.34) -- (271.45,147.34);

\path[draw=drawColor,line width= 0.6pt,line join=round] ( 28.93,147.34) -- (271.45,147.34);

\path[draw=drawColor,line width= 0.6pt,line join=round] ( 28.93,147.34) -- (271.45,147.34);

\path[draw=drawColor,line width= 0.6pt,line join=round] ( 28.93,147.34) -- (271.45,147.34);

\path[draw=drawColor,line width= 0.6pt,line join=round] ( 28.93,147.34) -- (271.45,147.34);

\path[draw=drawColor,line width= 0.6pt,line join=round] ( 28.93,147.34) -- (271.45,147.34);

\path[draw=drawColor,line width= 0.6pt,line join=round] ( 28.93,147.34) -- (271.45,147.34);

\path[draw=drawColor,line width= 0.6pt,line join=round] ( 28.93,147.34) -- (271.45,147.34);

\path[draw=drawColor,line width= 0.6pt,line join=round] ( 28.93,147.34) -- (271.45,147.34);

\path[draw=drawColor,line width= 0.6pt,line join=round] ( 28.93,147.34) -- (271.45,147.34);

\path[draw=drawColor,line width= 0.6pt,line join=round] ( 28.93,147.34) -- (271.45,147.34);

\path[draw=drawColor,line width= 0.6pt,line join=round] ( 28.93,147.34) -- (271.45,147.34);

\path[draw=drawColor,line width= 0.6pt,line join=round] ( 28.93,147.34) -- (271.45,147.34);

\path[draw=drawColor,line width= 0.6pt,line join=round] ( 28.93,147.34) -- (271.45,147.34);

\path[draw=drawColor,line width= 0.6pt,line join=round] ( 28.93,147.34) -- (271.45,147.34);

\path[draw=drawColor,line width= 0.6pt,line join=round] ( 28.93,147.34) -- (271.45,147.34);

\path[draw=drawColor,line width= 0.6pt,line join=round] ( 28.93,147.34) -- (271.45,147.34);

\path[draw=drawColor,line width= 0.6pt,line join=round] ( 28.93,147.34) -- (271.45,147.34);

\path[draw=drawColor,line width= 0.6pt,line join=round] ( 28.93,147.34) -- (271.45,147.34);

\path[draw=drawColor,line width= 0.6pt,line join=round] ( 28.93,147.34) -- (271.45,147.34);

\path[draw=drawColor,line width= 0.6pt,line join=round] ( 28.93,147.34) -- (271.45,147.34);

\path[draw=drawColor,line width= 0.6pt,line join=round] ( 28.93,147.34) -- (271.45,147.34);

\path[draw=drawColor,line width= 0.6pt,line join=round] ( 28.93,147.34) -- (271.45,147.34);

\path[draw=drawColor,line width= 0.6pt,line join=round] ( 28.93,147.34) -- (271.45,147.34);

\path[draw=drawColor,line width= 0.6pt,line join=round] ( 28.93,147.34) -- (271.45,147.34);

\path[draw=drawColor,line width= 0.6pt,line join=round] ( 28.93,147.34) -- (271.45,147.34);

\path[draw=drawColor,line width= 0.6pt,line join=round] ( 28.93,147.34) -- (271.45,147.34);

\path[draw=drawColor,line width= 0.6pt,line join=round] ( 28.93,147.34) -- (271.45,147.34);

\path[draw=drawColor,line width= 0.6pt,line join=round] ( 28.93,147.34) -- (271.45,147.34);

\path[draw=drawColor,line width= 0.6pt,line join=round] ( 28.93,147.34) -- (271.45,147.34);

\path[draw=drawColor,line width= 0.6pt,line join=round] ( 28.93,147.34) -- (271.45,147.34);

\path[draw=drawColor,line width= 0.6pt,line join=round] ( 28.93,147.34) -- (271.45,147.34);

\path[draw=drawColor,line width= 0.6pt,line join=round] ( 28.93,147.34) -- (271.45,147.34);

\path[draw=drawColor,line width= 0.6pt,line join=round] ( 28.93,147.34) -- (271.45,147.34);

\path[draw=drawColor,line width= 0.6pt,line join=round] ( 28.93,147.34) -- (271.45,147.34);

\path[draw=drawColor,line width= 0.6pt,line join=round] ( 28.93,147.34) -- (271.45,147.34);

\path[draw=drawColor,line width= 0.6pt,line join=round] ( 28.93,147.34) -- (271.45,147.34);

\path[draw=drawColor,line width= 0.6pt,line join=round] ( 28.93,147.34) -- (271.45,147.34);

\path[draw=drawColor,line width= 0.6pt,line join=round] ( 28.93,147.34) -- (271.45,147.34);

\path[draw=drawColor,line width= 0.6pt,line join=round] ( 28.93,147.34) -- (271.45,147.34);

\path[draw=drawColor,line width= 0.6pt,line join=round] ( 28.93,147.34) -- (271.45,147.34);

\path[draw=drawColor,line width= 0.6pt,line join=round] (109.77, 32.48) -- (109.77,262.20);

\path[draw=drawColor,line width= 0.6pt,line join=round] (109.77, 32.48) -- (109.77,262.20);

\path[draw=drawColor,line width= 0.6pt,line join=round] (109.77, 32.48) -- (109.77,262.20);

\path[draw=drawColor,line width= 0.6pt,line join=round] (109.77, 32.48) -- (109.77,262.20);

\path[draw=drawColor,line width= 0.6pt,line join=round] (109.77, 32.48) -- (109.77,262.20);

\path[draw=drawColor,line width= 0.6pt,line join=round] (109.77, 32.48) -- (109.77,262.20);

\path[draw=drawColor,line width= 0.6pt,line join=round] (109.77, 32.48) -- (109.77,262.20);

\path[draw=drawColor,line width= 0.6pt,line join=round] (109.77, 32.48) -- (109.77,262.20);

\path[draw=drawColor,line width= 0.6pt,line join=round] (109.77, 32.48) -- (109.77,262.20);

\path[draw=drawColor,line width= 0.6pt,line join=round] (109.77, 32.48) -- (109.77,262.20);

\path[draw=drawColor,line width= 0.6pt,line join=round] (109.77, 32.48) -- (109.77,262.20);

\path[draw=drawColor,line width= 0.6pt,line join=round] (109.77, 32.48) -- (109.77,262.20);

\path[draw=drawColor,line width= 0.6pt,line join=round] (109.77, 32.48) -- (109.77,262.20);

\path[draw=drawColor,line width= 0.6pt,line join=round] (109.77, 32.48) -- (109.77,262.20);

\path[draw=drawColor,line width= 0.6pt,line join=round] (109.77, 32.48) -- (109.77,262.20);

\path[draw=drawColor,line width= 0.6pt,line join=round] (109.77, 32.48) -- (109.77,262.20);

\path[draw=drawColor,line width= 0.6pt,line join=round] (109.77, 32.48) -- (109.77,262.20);

\path[draw=drawColor,line width= 0.6pt,line join=round] (109.77, 32.48) -- (109.77,262.20);

\path[draw=drawColor,line width= 0.6pt,line join=round] (109.77, 32.48) -- (109.77,262.20);

\path[draw=drawColor,line width= 0.6pt,line join=round] (109.77, 32.48) -- (109.77,262.20);

\path[draw=drawColor,line width= 0.6pt,line join=round] (109.77, 32.48) -- (109.77,262.20);

\path[draw=drawColor,line width= 0.6pt,line join=round] (109.77, 32.48) -- (109.77,262.20);

\path[draw=drawColor,line width= 0.6pt,line join=round] (109.77, 32.48) -- (109.77,262.20);

\path[draw=drawColor,line width= 0.6pt,line join=round] (109.77, 32.48) -- (109.77,262.20);

\path[draw=drawColor,line width= 0.6pt,line join=round] (109.77, 32.48) -- (109.77,262.20);

\path[draw=drawColor,line width= 0.6pt,line join=round] (109.77, 32.48) -- (109.77,262.20);

\path[draw=drawColor,line width= 0.6pt,line join=round] (109.77, 32.48) -- (109.77,262.20);

\path[draw=drawColor,line width= 0.6pt,line join=round] (109.77, 32.48) -- (109.77,262.20);

\path[draw=drawColor,line width= 0.6pt,line join=round] (109.77, 32.48) -- (109.77,262.20);

\path[draw=drawColor,line width= 0.6pt,line join=round] (109.77, 32.48) -- (109.77,262.20);

\path[draw=drawColor,line width= 0.6pt,line join=round] (109.77, 32.48) -- (109.77,262.20);

\path[draw=drawColor,line width= 0.6pt,line join=round] (109.77, 32.48) -- (109.77,262.20);

\path[draw=drawColor,line width= 0.6pt,line join=round] (109.77, 32.48) -- (109.77,262.20);

\path[draw=drawColor,line width= 0.6pt,line join=round] (109.77, 32.48) -- (109.77,262.20);

\path[draw=drawColor,line width= 0.6pt,line join=round] (109.77, 32.48) -- (109.77,262.20);

\path[draw=drawColor,line width= 0.6pt,line join=round] (109.77, 32.48) -- (109.77,262.20);

\path[draw=drawColor,line width= 0.6pt,line join=round] (109.77, 32.48) -- (109.77,262.20);

\path[draw=drawColor,line width= 0.6pt,line join=round] (109.77, 32.48) -- (109.77,262.20);

\path[draw=drawColor,line width= 0.6pt,line join=round] (109.77, 32.48) -- (109.77,262.20);

\path[draw=drawColor,line width= 0.6pt,line join=round] (109.77, 32.48) -- (109.77,262.20);

\path[draw=drawColor,line width= 0.6pt,line join=round] (109.77, 32.48) -- (109.77,262.20);

\path[draw=drawColor,line width= 0.6pt,line join=round] (109.77, 32.48) -- (109.77,262.20);

\path[draw=drawColor,line width= 0.6pt,line join=round] (109.77, 32.48) -- (109.77,262.20);

\path[draw=drawColor,line width= 0.6pt,line join=round] (109.77, 32.48) -- (109.77,262.20);

\path[draw=drawColor,line width= 0.6pt,line join=round] (109.77, 32.48) -- (109.77,262.20);

\path[draw=drawColor,line width= 0.6pt,line join=round] (109.77, 32.48) -- (109.77,262.20);

\path[draw=drawColor,line width= 0.6pt,line join=round] (109.77, 32.48) -- (109.77,262.20);

\path[draw=drawColor,line width= 0.6pt,line join=round] (109.77, 32.48) -- (109.77,262.20);

\path[draw=drawColor,line width= 0.6pt,line join=round] (109.77, 32.48) -- (109.77,262.20);

\path[draw=drawColor,line width= 0.6pt,line join=round] (109.77, 32.48) -- (109.77,262.20);

\path[draw=drawColor,line width= 0.6pt,line join=round] (109.77, 32.48) -- (109.77,262.20);

\path[draw=drawColor,line width= 0.6pt,line join=round] (109.77, 32.48) -- (109.77,262.20);

\path[draw=drawColor,line width= 0.6pt,line join=round] (109.77, 32.48) -- (109.77,262.20);

\path[draw=drawColor,line width= 0.6pt,line join=round] (109.77, 32.48) -- (109.77,262.20);

\path[draw=drawColor,line width= 0.6pt,line join=round] (109.77, 32.48) -- (109.77,262.20);

\path[draw=drawColor,line width= 0.6pt,line join=round] (109.77, 32.48) -- (109.77,262.20);

\path[draw=drawColor,line width= 0.6pt,line join=round] (109.77, 32.48) -- (109.77,262.20);

\path[draw=drawColor,line width= 0.6pt,line join=round] (109.77, 32.48) -- (109.77,262.20);

\path[draw=drawColor,line width= 0.6pt,line join=round] (109.77, 32.48) -- (109.77,262.20);

\path[draw=drawColor,line width= 0.6pt,line join=round] (109.77, 32.48) -- (109.77,262.20);

\path[draw=drawColor,line width= 0.6pt,line join=round] (109.77, 32.48) -- (109.77,262.20);

\path[draw=drawColor,line width= 0.6pt,line join=round] (109.77, 32.48) -- (109.77,262.20);

\path[draw=drawColor,line width= 0.6pt,line join=round] (109.77, 32.48) -- (109.77,262.20);

\path[draw=drawColor,line width= 0.6pt,line join=round] (109.77, 32.48) -- (109.77,262.20);

\path[draw=drawColor,line width= 0.6pt,line join=round] (109.77, 32.48) -- (109.77,262.20);

\path[draw=drawColor,line width= 0.6pt,line join=round] (109.77, 32.48) -- (109.77,262.20);

\path[draw=drawColor,line width= 0.6pt,line join=round] (109.77, 32.48) -- (109.77,262.20);

\path[draw=drawColor,line width= 0.6pt,line join=round] (109.77, 32.48) -- (109.77,262.20);

\path[draw=drawColor,line width= 0.6pt,line join=round] (109.77, 32.48) -- (109.77,262.20);

\path[draw=drawColor,line width= 0.6pt,line join=round] (109.77, 32.48) -- (109.77,262.20);

\path[draw=drawColor,line width= 0.6pt,line join=round] (109.77, 32.48) -- (109.77,262.20);

\path[draw=drawColor,line width= 0.6pt,line join=round] (109.77, 32.48) -- (109.77,262.20);

\path[draw=drawColor,line width= 0.6pt,line join=round] (109.77, 32.48) -- (109.77,262.20);

\path[draw=drawColor,line width= 0.6pt,line join=round] (109.77, 32.48) -- (109.77,262.20);

\path[draw=drawColor,line width= 0.6pt,line join=round] (109.77, 32.48) -- (109.77,262.20);

\path[draw=drawColor,line width= 0.6pt,line join=round] (109.77, 32.48) -- (109.77,262.20);

\path[draw=drawColor,line width= 0.6pt,line join=round] (109.77, 32.48) -- (109.77,262.20);

\path[draw=drawColor,line width= 0.6pt,line join=round] (109.77, 32.48) -- (109.77,262.20);

\path[draw=drawColor,line width= 0.6pt,line join=round] (109.77, 32.48) -- (109.77,262.20);

\path[draw=drawColor,line width= 0.6pt,line join=round] (109.77, 32.48) -- (109.77,262.20);

\path[draw=drawColor,line width= 0.6pt,line join=round] (109.77, 32.48) -- (109.77,262.20);

\path[draw=drawColor,line width= 0.6pt,line join=round] (109.77, 32.48) -- (109.77,262.20);

\path[draw=drawColor,line width= 0.6pt,line join=round] (109.77, 32.48) -- (109.77,262.20);

\path[draw=drawColor,line width= 0.6pt,line join=round] (109.77, 32.48) -- (109.77,262.20);

\path[draw=drawColor,line width= 0.6pt,line join=round] (109.77, 32.48) -- (109.77,262.20);

\path[draw=drawColor,line width= 0.6pt,line join=round] (109.77, 32.48) -- (109.77,262.20);

\path[draw=drawColor,line width= 0.6pt,line join=round] (109.77, 32.48) -- (109.77,262.20);

\path[draw=drawColor,line width= 0.6pt,line join=round] (109.77, 32.48) -- (109.77,262.20);

\path[draw=drawColor,line width= 0.6pt,line join=round] (109.77, 32.48) -- (109.77,262.20);

\path[draw=drawColor,line width= 0.6pt,line join=round] (109.77, 32.48) -- (109.77,262.20);

\path[draw=drawColor,line width= 0.6pt,line join=round] (109.77, 32.48) -- (109.77,262.20);

\path[draw=drawColor,line width= 0.6pt,line join=round] (109.77, 32.48) -- (109.77,262.20);

\path[draw=drawColor,line width= 0.6pt,line join=round] (109.77, 32.48) -- (109.77,262.20);

\path[draw=drawColor,line width= 0.6pt,line join=round] (109.77, 32.48) -- (109.77,262.20);

\path[draw=drawColor,line width= 0.6pt,line join=round] (109.77, 32.48) -- (109.77,262.20);

\path[draw=drawColor,line width= 0.6pt,line join=round] (109.77, 32.48) -- (109.77,262.20);

\path[draw=drawColor,line width= 0.6pt,line join=round] (109.77, 32.48) -- (109.77,262.20);

\path[draw=drawColor,line width= 0.6pt,line join=round] (109.77, 32.48) -- (109.77,262.20);

\path[draw=drawColor,line width= 0.6pt,line join=round] (109.77, 32.48) -- (109.77,262.20);

\path[draw=drawColor,line width= 0.6pt,line join=round] (109.77, 32.48) -- (109.77,262.20);

\path[draw=drawColor,line width= 0.6pt,line join=round] (109.77, 32.48) -- (109.77,262.20);

\path[draw=drawColor,line width= 0.6pt,line join=round] (109.77, 32.48) -- (109.77,262.20);

\path[draw=drawColor,line width= 0.6pt,line join=round] (109.77, 32.48) -- (109.77,262.20);

\path[draw=drawColor,line width= 0.6pt,line join=round] (109.77, 32.48) -- (109.77,262.20);

\path[draw=drawColor,line width= 0.6pt,line join=round] (109.77, 32.48) -- (109.77,262.20);

\path[draw=drawColor,line width= 0.6pt,line join=round] (109.77, 32.48) -- (109.77,262.20);

\path[draw=drawColor,line width= 0.6pt,line join=round] (109.77, 32.48) -- (109.77,262.20);

\path[draw=drawColor,line width= 0.6pt,line join=round] (109.77, 32.48) -- (109.77,262.20);

\path[draw=drawColor,line width= 0.6pt,line join=round] (109.77, 32.48) -- (109.77,262.20);

\path[draw=drawColor,line width= 0.6pt,line join=round] (109.77, 32.48) -- (109.77,262.20);

\path[draw=drawColor,line width= 0.6pt,line join=round] (109.77, 32.48) -- (109.77,262.20);

\path[draw=drawColor,line width= 0.6pt,line join=round] (109.77, 32.48) -- (109.77,262.20);

\path[draw=drawColor,line width= 0.6pt,line join=round] (109.77, 32.48) -- (109.77,262.20);

\path[draw=drawColor,line width= 0.6pt,line join=round] (109.77, 32.48) -- (109.77,262.20);

\path[draw=drawColor,line width= 0.6pt,line join=round] (109.77, 32.48) -- (109.77,262.20);

\path[draw=drawColor,line width= 0.6pt,line join=round] (109.77, 32.48) -- (109.77,262.20);

\path[draw=drawColor,line width= 0.6pt,line join=round] (109.77, 32.48) -- (109.77,262.20);

\path[draw=drawColor,line width= 0.6pt,line join=round] (109.77, 32.48) -- (109.77,262.20);

\path[draw=drawColor,line width= 0.6pt,line join=round] (109.77, 32.48) -- (109.77,262.20);

\path[draw=drawColor,line width= 0.6pt,line join=round] (109.77, 32.48) -- (109.77,262.20);

\path[draw=drawColor,line width= 0.6pt,line join=round] (109.77, 32.48) -- (109.77,262.20);

\path[draw=drawColor,line width= 0.6pt,line join=round] (109.77, 32.48) -- (109.77,262.20);

\path[draw=drawColor,line width= 0.6pt,line join=round] (109.77, 32.48) -- (109.77,262.20);

\path[draw=drawColor,line width= 0.6pt,line join=round] (109.77, 32.48) -- (109.77,262.20);

\path[draw=drawColor,line width= 0.6pt,line join=round] (109.77, 32.48) -- (109.77,262.20);

\path[draw=drawColor,line width= 0.6pt,line join=round] (109.77, 32.48) -- (109.77,262.20);

\path[draw=drawColor,line width= 0.6pt,line join=round] (109.77, 32.48) -- (109.77,262.20);

\path[draw=drawColor,line width= 0.6pt,line join=round] (109.77, 32.48) -- (109.77,262.20);

\path[draw=drawColor,line width= 0.6pt,line join=round] (109.77, 32.48) -- (109.77,262.20);

\path[draw=drawColor,line width= 0.6pt,line join=round] (109.77, 32.48) -- (109.77,262.20);

\path[draw=drawColor,line width= 0.6pt,line join=round] (109.77, 32.48) -- (109.77,262.20);

\path[draw=drawColor,line width= 0.6pt,line join=round] (109.77, 32.48) -- (109.77,262.20);

\path[draw=drawColor,line width= 0.6pt,line join=round] (109.77, 32.48) -- (109.77,262.20);

\path[draw=drawColor,line width= 0.6pt,line join=round] (109.77, 32.48) -- (109.77,262.20);

\path[draw=drawColor,line width= 0.6pt,line join=round] (109.77, 32.48) -- (109.77,262.20);

\path[draw=drawColor,line width= 0.6pt,line join=round] (109.77, 32.48) -- (109.77,262.20);

\path[draw=drawColor,line width= 0.6pt,line join=round] (109.77, 32.48) -- (109.77,262.20);

\path[draw=drawColor,line width= 0.6pt,line join=round] (109.77, 32.48) -- (109.77,262.20);

\path[draw=drawColor,line width= 0.6pt,line join=round] (109.77, 32.48) -- (109.77,262.20);

\path[draw=drawColor,line width= 0.6pt,line join=round] (109.77, 32.48) -- (109.77,262.20);

\path[draw=drawColor,line width= 0.6pt,line join=round] (109.77, 32.48) -- (109.77,262.20);

\path[draw=drawColor,line width= 0.6pt,line join=round] (109.77, 32.48) -- (109.77,262.20);

\path[draw=drawColor,line width= 0.6pt,line join=round] (109.77, 32.48) -- (109.77,262.20);

\path[draw=drawColor,line width= 0.6pt,line join=round] (109.77, 32.48) -- (109.77,262.20);

\path[draw=drawColor,line width= 0.6pt,line join=round] (109.77, 32.48) -- (109.77,262.20);

\path[draw=drawColor,line width= 0.6pt,line join=round] (109.77, 32.48) -- (109.77,262.20);

\path[draw=drawColor,line width= 0.6pt,line join=round] (109.77, 32.48) -- (109.77,262.20);

\path[draw=drawColor,line width= 0.6pt,line join=round] (109.77, 32.48) -- (109.77,262.20);

\path[draw=drawColor,line width= 0.6pt,line join=round] (109.77, 32.48) -- (109.77,262.20);

\path[draw=drawColor,line width= 0.6pt,line join=round] (109.77, 32.48) -- (109.77,262.20);

\path[draw=drawColor,line width= 0.6pt,line join=round] (109.77, 32.48) -- (109.77,262.20);

\path[draw=drawColor,line width= 0.6pt,line join=round] (109.77, 32.48) -- (109.77,262.20);

\path[draw=drawColor,line width= 0.6pt,line join=round] (109.77, 32.48) -- (109.77,262.20);

\path[draw=drawColor,line width= 0.6pt,line join=round] (109.77, 32.48) -- (109.77,262.20);

\path[draw=drawColor,line width= 0.6pt,line join=round] (109.77, 32.48) -- (109.77,262.20);

\path[draw=drawColor,line width= 0.6pt,line join=round] (109.77, 32.48) -- (109.77,262.20);

\path[draw=drawColor,line width= 0.6pt,line join=round] (109.77, 32.48) -- (109.77,262.20);

\path[draw=drawColor,line width= 0.6pt,line join=round] (109.77, 32.48) -- (109.77,262.20);

\path[draw=drawColor,line width= 0.6pt,line join=round] (109.77, 32.48) -- (109.77,262.20);

\path[draw=drawColor,line width= 0.6pt,line join=round] (109.77, 32.48) -- (109.77,262.20);

\path[draw=drawColor,line width= 0.6pt,line join=round] (109.77, 32.48) -- (109.77,262.20);

\path[draw=drawColor,line width= 0.6pt,line join=round] (109.77, 32.48) -- (109.77,262.20);

\path[draw=drawColor,line width= 0.6pt,line join=round] (109.77, 32.48) -- (109.77,262.20);

\path[draw=drawColor,line width= 0.6pt,line join=round] (109.77, 32.48) -- (109.77,262.20);

\path[draw=drawColor,line width= 0.6pt,line join=round] (109.77, 32.48) -- (109.77,262.20);

\path[draw=drawColor,line width= 0.6pt,line join=round] (109.77, 32.48) -- (109.77,262.20);

\path[draw=drawColor,line width= 0.6pt,line join=round] (109.77, 32.48) -- (109.77,262.20);

\path[draw=drawColor,line width= 0.6pt,line join=round] (109.77, 32.48) -- (109.77,262.20);

\path[draw=drawColor,line width= 0.6pt,line join=round] (109.77, 32.48) -- (109.77,262.20);

\path[draw=drawColor,line width= 0.6pt,line join=round] (109.77, 32.48) -- (109.77,262.20);

\path[draw=drawColor,line width= 0.6pt,line join=round] (109.77, 32.48) -- (109.77,262.20);

\path[draw=drawColor,line width= 0.6pt,line join=round] (109.77, 32.48) -- (109.77,262.20);

\path[draw=drawColor,line width= 0.6pt,line join=round] (109.77, 32.48) -- (109.77,262.20);

\path[draw=drawColor,line width= 0.6pt,line join=round] (109.77, 32.48) -- (109.77,262.20);

\path[draw=drawColor,line width= 0.6pt,line join=round] (109.77, 32.48) -- (109.77,262.20);

\path[draw=drawColor,line width= 0.6pt,line join=round] (109.77, 32.48) -- (109.77,262.20);

\path[draw=drawColor,line width= 0.6pt,line join=round] (109.77, 32.48) -- (109.77,262.20);

\path[draw=drawColor,line width= 0.6pt,line join=round] (109.77, 32.48) -- (109.77,262.20);

\path[draw=drawColor,line width= 0.6pt,line join=round] (109.77, 32.48) -- (109.77,262.20);

\path[draw=drawColor,line width= 0.6pt,line join=round] (109.77, 32.48) -- (109.77,262.20);

\path[draw=drawColor,line width= 0.6pt,line join=round] (109.77, 32.48) -- (109.77,262.20);

\path[draw=drawColor,line width= 0.6pt,line join=round] (109.77, 32.48) -- (109.77,262.20);

\path[draw=drawColor,line width= 0.6pt,line join=round] (109.77, 32.48) -- (109.77,262.20);

\path[draw=drawColor,line width= 0.6pt,line join=round] (109.77, 32.48) -- (109.77,262.20);

\path[draw=drawColor,line width= 0.6pt,line join=round] (109.77, 32.48) -- (109.77,262.20);

\path[draw=drawColor,line width= 0.6pt,line join=round] (109.77, 32.48) -- (109.77,262.20);

\path[draw=drawColor,line width= 0.6pt,line join=round] (109.77, 32.48) -- (109.77,262.20);

\path[draw=drawColor,line width= 0.6pt,line join=round] (109.77, 32.48) -- (109.77,262.20);

\path[draw=drawColor,line width= 0.6pt,line join=round] (109.77, 32.48) -- (109.77,262.20);

\path[draw=drawColor,line width= 0.6pt,line join=round] (109.77, 32.48) -- (109.77,262.20);

\path[draw=drawColor,line width= 0.6pt,line join=round] (109.77, 32.48) -- (109.77,262.20);

\path[draw=drawColor,line width= 0.6pt,line join=round] (109.77, 32.48) -- (109.77,262.20);

\path[draw=drawColor,line width= 0.6pt,line join=round] (109.77, 32.48) -- (109.77,262.20);

\path[draw=drawColor,line width= 0.6pt,line join=round] (109.77, 32.48) -- (109.77,262.20);

\path[draw=drawColor,line width= 0.6pt,line join=round] (109.77, 32.48) -- (109.77,262.20);

\path[draw=drawColor,line width= 0.6pt,line join=round] (109.77, 32.48) -- (109.77,262.20);

\path[draw=drawColor,line width= 0.6pt,line join=round] (109.77, 32.48) -- (109.77,262.20);

\path[draw=drawColor,line width= 0.6pt,line join=round] (109.77, 32.48) -- (109.77,262.20);

\path[draw=drawColor,line width= 0.6pt,line join=round] (109.77, 32.48) -- (109.77,262.20);

\path[draw=drawColor,line width= 0.6pt,line join=round] (109.77, 32.48) -- (109.77,262.20);

\path[draw=drawColor,line width= 0.6pt,line join=round] (109.77, 32.48) -- (109.77,262.20);

\path[draw=drawColor,line width= 0.6pt,line join=round] (109.77, 32.48) -- (109.77,262.20);

\path[draw=drawColor,line width= 0.6pt,line join=round] (109.77, 32.48) -- (109.77,262.20);

\path[draw=drawColor,line width= 0.6pt,line join=round] (109.77, 32.48) -- (109.77,262.20);

\path[draw=drawColor,line width= 0.6pt,line join=round] (109.77, 32.48) -- (109.77,262.20);

\path[draw=drawColor,line width= 0.6pt,line join=round] (109.77, 32.48) -- (109.77,262.20);

\path[draw=drawColor,line width= 0.6pt,line join=round] (109.77, 32.48) -- (109.77,262.20);

\path[draw=drawColor,line width= 0.6pt,line join=round] (109.77, 32.48) -- (109.77,262.20);

\path[draw=drawColor,line width= 0.6pt,line join=round] (109.77, 32.48) -- (109.77,262.20);

\path[draw=drawColor,line width= 0.6pt,line join=round] (109.77, 32.48) -- (109.77,262.20);

\path[draw=drawColor,line width= 0.6pt,line join=round] (109.77, 32.48) -- (109.77,262.20);

\path[draw=drawColor,line width= 0.6pt,line join=round] (109.77, 32.48) -- (109.77,262.20);

\path[draw=drawColor,line width= 0.6pt,line join=round] (109.77, 32.48) -- (109.77,262.20);

\path[draw=drawColor,line width= 0.6pt,line join=round] (109.77, 32.48) -- (109.77,262.20);

\path[draw=drawColor,line width= 0.6pt,line join=round] (109.77, 32.48) -- (109.77,262.20);

\path[draw=drawColor,line width= 0.6pt,line join=round] (109.77, 32.48) -- (109.77,262.20);

\path[draw=drawColor,line width= 0.6pt,line join=round] (109.77, 32.48) -- (109.77,262.20);

\path[draw=drawColor,line width= 0.6pt,line join=round] (109.77, 32.48) -- (109.77,262.20);

\path[draw=drawColor,line width= 0.6pt,line join=round] (109.77, 32.48) -- (109.77,262.20);

\path[draw=drawColor,line width= 0.6pt,line join=round] (109.77, 32.48) -- (109.77,262.20);

\path[draw=drawColor,line width= 0.6pt,line join=round] (109.77, 32.48) -- (109.77,262.20);

\path[draw=drawColor,line width= 0.6pt,line join=round] (109.77, 32.48) -- (109.77,262.20);

\path[draw=drawColor,line width= 0.6pt,line join=round] (109.77, 32.48) -- (109.77,262.20);

\path[draw=drawColor,line width= 0.6pt,line join=round] (109.77, 32.48) -- (109.77,262.20);

\path[draw=drawColor,line width= 0.6pt,line join=round] (109.77, 32.48) -- (109.77,262.20);

\path[draw=drawColor,line width= 0.6pt,line join=round] (109.77, 32.48) -- (109.77,262.20);

\path[draw=drawColor,line width= 0.6pt,line join=round] (109.77, 32.48) -- (109.77,262.20);

\path[draw=drawColor,line width= 0.6pt,line join=round] (109.77, 32.48) -- (109.77,262.20);

\path[draw=drawColor,line width= 0.6pt,line join=round] (109.77, 32.48) -- (109.77,262.20);

\path[draw=drawColor,line width= 0.6pt,line join=round] (109.77, 32.48) -- (109.77,262.20);

\path[draw=drawColor,line width= 0.6pt,line join=round] (109.77, 32.48) -- (109.77,262.20);

\path[draw=drawColor,line width= 0.6pt,line join=round] (109.77, 32.48) -- (109.77,262.20);

\path[draw=drawColor,line width= 0.6pt,line join=round] (109.77, 32.48) -- (109.77,262.20);

\path[draw=drawColor,line width= 0.6pt,line join=round] (109.77, 32.48) -- (109.77,262.20);

\path[draw=drawColor,line width= 0.6pt,line join=round] (109.77, 32.48) -- (109.77,262.20);

\path[draw=drawColor,line width= 0.6pt,line join=round] (109.77, 32.48) -- (109.77,262.20);

\path[draw=drawColor,line width= 0.6pt,line join=round] (109.77, 32.48) -- (109.77,262.20);

\path[draw=drawColor,line width= 0.6pt,line join=round] (109.77, 32.48) -- (109.77,262.20);

\path[draw=drawColor,line width= 0.6pt,line join=round] (109.77, 32.48) -- (109.77,262.20);

\path[draw=drawColor,line width= 0.6pt,line join=round] (109.77, 32.48) -- (109.77,262.20);

\path[draw=drawColor,line width= 0.6pt,line join=round] (109.77, 32.48) -- (109.77,262.20);

\path[draw=drawColor,line width= 0.6pt,line join=round] (109.77, 32.48) -- (109.77,262.20);

\path[draw=drawColor,line width= 0.6pt,line join=round] (109.77, 32.48) -- (109.77,262.20);

\path[draw=drawColor,line width= 0.6pt,line join=round] (109.77, 32.48) -- (109.77,262.20);

\path[draw=drawColor,line width= 0.6pt,line join=round] (109.77, 32.48) -- (109.77,262.20);

\path[draw=drawColor,line width= 0.6pt,line join=round] (109.77, 32.48) -- (109.77,262.20);

\path[draw=drawColor,line width= 0.6pt,line join=round] (109.77, 32.48) -- (109.77,262.20);

\path[draw=drawColor,line width= 0.6pt,line join=round] (109.77, 32.48) -- (109.77,262.20);

\path[draw=drawColor,line width= 0.6pt,line join=round] (109.77, 32.48) -- (109.77,262.20);

\path[draw=drawColor,line width= 0.6pt,line join=round] (109.77, 32.48) -- (109.77,262.20);

\path[draw=drawColor,line width= 0.6pt,line join=round] (109.77, 32.48) -- (109.77,262.20);

\path[draw=drawColor,line width= 0.6pt,line join=round] (109.77, 32.48) -- (109.77,262.20);

\path[draw=drawColor,line width= 0.6pt,line join=round] (109.77, 32.48) -- (109.77,262.20);

\path[draw=drawColor,line width= 0.6pt,line join=round] (109.77, 32.48) -- (109.77,262.20);

\path[draw=drawColor,line width= 0.6pt,line join=round] (109.77, 32.48) -- (109.77,262.20);

\path[draw=drawColor,line width= 0.6pt,line join=round] (109.77, 32.48) -- (109.77,262.20);

\path[draw=drawColor,line width= 0.6pt,line join=round] (109.77, 32.48) -- (109.77,262.20);

\path[draw=drawColor,line width= 0.6pt,line join=round] (109.77, 32.48) -- (109.77,262.20);

\path[draw=drawColor,line width= 0.6pt,line join=round] (109.77, 32.48) -- (109.77,262.20);

\path[draw=drawColor,line width= 0.6pt,line join=round] (109.77, 32.48) -- (109.77,262.20);

\path[draw=drawColor,line width= 0.6pt,line join=round] (109.77, 32.48) -- (109.77,262.20);

\path[draw=drawColor,line width= 0.6pt,line join=round] (109.77, 32.48) -- (109.77,262.20);

\path[draw=drawColor,line width= 0.6pt,line join=round] (109.77, 32.48) -- (109.77,262.20);

\path[draw=drawColor,line width= 0.6pt,line join=round] (109.77, 32.48) -- (109.77,262.20);

\path[draw=drawColor,line width= 0.6pt,line join=round] (109.77, 32.48) -- (109.77,262.20);

\path[draw=drawColor,line width= 0.6pt,line join=round] (109.77, 32.48) -- (109.77,262.20);

\path[draw=drawColor,line width= 0.6pt,line join=round] (109.77, 32.48) -- (109.77,262.20);

\path[draw=drawColor,line width= 0.6pt,line join=round] (109.77, 32.48) -- (109.77,262.20);

\path[draw=drawColor,line width= 0.6pt,line join=round] (109.77, 32.48) -- (109.77,262.20);

\path[draw=drawColor,line width= 0.6pt,line join=round] (109.77, 32.48) -- (109.77,262.20);

\path[draw=drawColor,line width= 0.6pt,line join=round] (109.77, 32.48) -- (109.77,262.20);

\path[draw=drawColor,line width= 0.6pt,line join=round] (109.77, 32.48) -- (109.77,262.20);

\path[draw=drawColor,line width= 0.6pt,line join=round] (109.77, 32.48) -- (109.77,262.20);

\path[draw=drawColor,line width= 0.6pt,line join=round] (109.77, 32.48) -- (109.77,262.20);

\path[draw=drawColor,line width= 0.6pt,line join=round] (109.77, 32.48) -- (109.77,262.20);

\path[draw=drawColor,line width= 0.6pt,line join=round] (109.77, 32.48) -- (109.77,262.20);

\path[draw=drawColor,line width= 0.6pt,line join=round] (109.77, 32.48) -- (109.77,262.20);

\path[draw=drawColor,line width= 0.6pt,line join=round] (109.77, 32.48) -- (109.77,262.20);

\path[draw=drawColor,line width= 0.6pt,line join=round] (109.77, 32.48) -- (109.77,262.20);

\path[draw=drawColor,line width= 0.6pt,line join=round] (109.77, 32.48) -- (109.77,262.20);

\path[draw=drawColor,line width= 0.6pt,line join=round] (109.77, 32.48) -- (109.77,262.20);

\path[draw=drawColor,line width= 0.6pt,line join=round] (109.77, 32.48) -- (109.77,262.20);

\path[draw=drawColor,line width= 0.6pt,line join=round] (109.77, 32.48) -- (109.77,262.20);

\path[draw=drawColor,line width= 0.6pt,line join=round] (109.77, 32.48) -- (109.77,262.20);

\path[draw=drawColor,line width= 0.6pt,line join=round] (109.77, 32.48) -- (109.77,262.20);

\path[draw=drawColor,line width= 0.6pt,line join=round] (109.77, 32.48) -- (109.77,262.20);

\path[draw=drawColor,line width= 0.6pt,line join=round] (109.77, 32.48) -- (109.77,262.20);

\path[draw=drawColor,line width= 0.6pt,line join=round] (109.77, 32.48) -- (109.77,262.20);

\path[draw=drawColor,line width= 0.6pt,line join=round] (109.77, 32.48) -- (109.77,262.20);

\path[draw=drawColor,line width= 0.6pt,line join=round] (109.77, 32.48) -- (109.77,262.20);

\path[draw=drawColor,line width= 0.6pt,line join=round] (109.77, 32.48) -- (109.77,262.20);

\path[draw=drawColor,line width= 0.6pt,line join=round] (109.77, 32.48) -- (109.77,262.20);

\path[draw=drawColor,line width= 0.6pt,line join=round] (109.77, 32.48) -- (109.77,262.20);

\path[draw=drawColor,line width= 0.6pt,line join=round] (109.77, 32.48) -- (109.77,262.20);

\path[draw=drawColor,line width= 0.6pt,line join=round] (109.77, 32.48) -- (109.77,262.20);

\path[draw=drawColor,line width= 0.6pt,line join=round] (109.77, 32.48) -- (109.77,262.20);

\path[draw=drawColor,line width= 0.6pt,line join=round] (109.77, 32.48) -- (109.77,262.20);

\path[draw=drawColor,line width= 0.6pt,line join=round] (109.77, 32.48) -- (109.77,262.20);

\path[draw=drawColor,line width= 0.6pt,line join=round] (109.77, 32.48) -- (109.77,262.20);

\path[draw=drawColor,line width= 0.6pt,line join=round] (109.77, 32.48) -- (109.77,262.20);

\path[draw=drawColor,line width= 0.6pt,line join=round] (109.77, 32.48) -- (109.77,262.20);

\path[draw=drawColor,line width= 0.6pt,line join=round] (109.77, 32.48) -- (109.77,262.20);

\path[draw=drawColor,line width= 0.6pt,line join=round] (109.77, 32.48) -- (109.77,262.20);

\path[draw=drawColor,line width= 0.6pt,line join=round] (109.77, 32.48) -- (109.77,262.20);

\path[draw=drawColor,line width= 0.6pt,line join=round] (109.77, 32.48) -- (109.77,262.20);

\path[draw=drawColor,line width= 0.6pt,line join=round] (109.77, 32.48) -- (109.77,262.20);

\path[draw=drawColor,line width= 0.6pt,line join=round] (109.77, 32.48) -- (109.77,262.20);

\path[draw=drawColor,line width= 0.6pt,line join=round] (109.77, 32.48) -- (109.77,262.20);

\path[draw=drawColor,line width= 0.6pt,line join=round] (109.77, 32.48) -- (109.77,262.20);

\path[draw=drawColor,line width= 0.6pt,line join=round] (109.77, 32.48) -- (109.77,262.20);

\path[draw=drawColor,line width= 0.6pt,line join=round] (109.77, 32.48) -- (109.77,262.20);

\path[draw=drawColor,line width= 0.6pt,line join=round] (109.77, 32.48) -- (109.77,262.20);

\path[draw=drawColor,line width= 0.6pt,line join=round] (109.77, 32.48) -- (109.77,262.20);

\path[draw=drawColor,line width= 0.6pt,line join=round] (109.77, 32.48) -- (109.77,262.20);

\path[draw=drawColor,line width= 0.6pt,line join=round] (109.77, 32.48) -- (109.77,262.20);

\path[draw=drawColor,line width= 0.6pt,line join=round] (109.77, 32.48) -- (109.77,262.20);

\path[draw=drawColor,line width= 0.6pt,line join=round] (109.77, 32.48) -- (109.77,262.20);

\path[draw=drawColor,line width= 0.6pt,line join=round] (109.77, 32.48) -- (109.77,262.20);

\path[draw=drawColor,line width= 0.6pt,line join=round] (109.77, 32.48) -- (109.77,262.20);

\path[draw=drawColor,line width= 0.6pt,line join=round] (109.77, 32.48) -- (109.77,262.20);

\path[draw=drawColor,line width= 0.6pt,line join=round] (109.77, 32.48) -- (109.77,262.20);

\path[draw=drawColor,line width= 0.6pt,line join=round] (109.77, 32.48) -- (109.77,262.20);

\path[draw=drawColor,line width= 0.6pt,line join=round] (109.77, 32.48) -- (109.77,262.20);

\path[draw=drawColor,line width= 0.6pt,line join=round] (109.77, 32.48) -- (109.77,262.20);

\path[draw=drawColor,line width= 0.6pt,line join=round] (109.77, 32.48) -- (109.77,262.20);

\path[draw=drawColor,line width= 0.6pt,line join=round] (109.77, 32.48) -- (109.77,262.20);

\path[draw=drawColor,line width= 0.6pt,line join=round] (109.77, 32.48) -- (109.77,262.20);

\path[draw=drawColor,line width= 0.6pt,line join=round] (109.77, 32.48) -- (109.77,262.20);

\path[draw=drawColor,line width= 0.6pt,line join=round] (109.77, 32.48) -- (109.77,262.20);

\path[draw=drawColor,line width= 0.6pt,line join=round] (109.77, 32.48) -- (109.77,262.20);

\path[draw=drawColor,line width= 0.6pt,line join=round] (109.77, 32.48) -- (109.77,262.20);

\path[draw=drawColor,line width= 0.6pt,line join=round] (109.77, 32.48) -- (109.77,262.20);

\path[draw=drawColor,line width= 0.6pt,line join=round] (109.77, 32.48) -- (109.77,262.20);

\path[draw=drawColor,line width= 0.6pt,line join=round] (109.77, 32.48) -- (109.77,262.20);

\path[draw=drawColor,line width= 0.6pt,line join=round] (109.77, 32.48) -- (109.77,262.20);

\path[draw=drawColor,line width= 0.6pt,line join=round] (109.77, 32.48) -- (109.77,262.20);

\path[draw=drawColor,line width= 0.6pt,line join=round] (109.77, 32.48) -- (109.77,262.20);

\path[draw=drawColor,line width= 0.6pt,line join=round] (109.77, 32.48) -- (109.77,262.20);

\path[draw=drawColor,line width= 0.6pt,line join=round] (109.77, 32.48) -- (109.77,262.20);

\path[draw=drawColor,line width= 0.6pt,line join=round] (109.77, 32.48) -- (109.77,262.20);

\path[draw=drawColor,line width= 0.6pt,line join=round] (109.77, 32.48) -- (109.77,262.20);

\path[draw=drawColor,line width= 0.6pt,line join=round] (109.77, 32.48) -- (109.77,262.20);

\path[draw=drawColor,line width= 0.6pt,line join=round] (109.77, 32.48) -- (109.77,262.20);

\path[draw=drawColor,line width= 0.6pt,line join=round] (109.77, 32.48) -- (109.77,262.20);

\path[draw=drawColor,line width= 0.6pt,line join=round] (109.77, 32.48) -- (109.77,262.20);

\path[draw=drawColor,line width= 0.6pt,line join=round] (109.77, 32.48) -- (109.77,262.20);

\path[draw=drawColor,line width= 0.6pt,line join=round] (109.77, 32.48) -- (109.77,262.20);

\path[draw=drawColor,line width= 0.6pt,line join=round] (109.77, 32.48) -- (109.77,262.20);

\path[draw=drawColor,line width= 0.6pt,line join=round] (109.77, 32.48) -- (109.77,262.20);

\path[draw=drawColor,line width= 0.6pt,line join=round] (109.77, 32.48) -- (109.77,262.20);

\path[draw=drawColor,line width= 0.6pt,line join=round] (109.77, 32.48) -- (109.77,262.20);

\path[draw=drawColor,line width= 0.6pt,line join=round] (109.77, 32.48) -- (109.77,262.20);

\path[draw=drawColor,line width= 0.6pt,line join=round] (109.77, 32.48) -- (109.77,262.20);

\path[draw=drawColor,line width= 0.6pt,line join=round] (109.77, 32.48) -- (109.77,262.20);

\path[draw=drawColor,line width= 0.6pt,line join=round] (109.77, 32.48) -- (109.77,262.20);

\path[draw=drawColor,line width= 0.6pt,line join=round] (109.77, 32.48) -- (109.77,262.20);

\path[draw=drawColor,line width= 0.6pt,line join=round] (109.77, 32.48) -- (109.77,262.20);

\path[draw=drawColor,line width= 0.6pt,line join=round] (109.77, 32.48) -- (109.77,262.20);

\path[draw=drawColor,line width= 0.6pt,line join=round] (109.77, 32.48) -- (109.77,262.20);

\path[draw=drawColor,line width= 0.6pt,line join=round] (109.77, 32.48) -- (109.77,262.20);

\path[draw=drawColor,line width= 0.6pt,line join=round] (109.77, 32.48) -- (109.77,262.20);

\path[draw=drawColor,line width= 0.6pt,line join=round] (109.77, 32.48) -- (109.77,262.20);

\path[draw=drawColor,line width= 0.6pt,line join=round] (109.77, 32.48) -- (109.77,262.20);

\path[draw=drawColor,line width= 0.6pt,line join=round] (109.77, 32.48) -- (109.77,262.20);

\path[draw=drawColor,line width= 0.6pt,line join=round] (109.77, 32.48) -- (109.77,262.20);

\path[draw=drawColor,line width= 0.6pt,line join=round] (109.77, 32.48) -- (109.77,262.20);

\path[draw=drawColor,line width= 0.6pt,line join=round] (109.77, 32.48) -- (109.77,262.20);

\path[draw=drawColor,line width= 0.6pt,line join=round] (109.77, 32.48) -- (109.77,262.20);

\path[draw=drawColor,line width= 0.6pt,line join=round] (109.77, 32.48) -- (109.77,262.20);

\path[draw=drawColor,line width= 0.6pt,line join=round] (109.77, 32.48) -- (109.77,262.20);

\path[draw=drawColor,line width= 0.6pt,line join=round] (109.77, 32.48) -- (109.77,262.20);

\path[draw=drawColor,line width= 0.6pt,line join=round] (109.77, 32.48) -- (109.77,262.20);

\path[draw=drawColor,line width= 0.6pt,line join=round] (109.77, 32.48) -- (109.77,262.20);

\path[draw=drawColor,line width= 0.6pt,line join=round] (109.77, 32.48) -- (109.77,262.20);

\path[draw=drawColor,line width= 0.6pt,line join=round] (109.77, 32.48) -- (109.77,262.20);

\path[draw=drawColor,line width= 0.6pt,line join=round] (109.77, 32.48) -- (109.77,262.20);

\path[draw=drawColor,line width= 0.6pt,line join=round] (109.77, 32.48) -- (109.77,262.20);

\path[draw=drawColor,line width= 0.6pt,line join=round] (109.77, 32.48) -- (109.77,262.20);

\path[draw=drawColor,line width= 0.6pt,line join=round] (109.77, 32.48) -- (109.77,262.20);

\path[draw=drawColor,line width= 0.6pt,line join=round] (109.77, 32.48) -- (109.77,262.20);

\path[draw=drawColor,line width= 0.6pt,line join=round] (109.77, 32.48) -- (109.77,262.20);

\path[draw=drawColor,line width= 0.6pt,line join=round] (109.77, 32.48) -- (109.77,262.20);

\path[draw=drawColor,line width= 0.6pt,line join=round] (109.77, 32.48) -- (109.77,262.20);

\path[draw=drawColor,line width= 0.6pt,line join=round] (109.77, 32.48) -- (109.77,262.20);

\path[draw=drawColor,line width= 0.6pt,line join=round] (109.77, 32.48) -- (109.77,262.20);

\path[draw=drawColor,line width= 0.6pt,line join=round] (109.77, 32.48) -- (109.77,262.20);

\path[draw=drawColor,line width= 0.6pt,line join=round] (109.77, 32.48) -- (109.77,262.20);

\path[draw=drawColor,line width= 0.6pt,line join=round] (109.77, 32.48) -- (109.77,262.20);

\path[draw=drawColor,line width= 0.6pt,line join=round] (109.77, 32.48) -- (109.77,262.20);

\path[draw=drawColor,line width= 0.6pt,line join=round] (109.77, 32.48) -- (109.77,262.20);

\path[draw=drawColor,line width= 0.6pt,line join=round] (109.77, 32.48) -- (109.77,262.20);

\path[draw=drawColor,line width= 0.6pt,line join=round] (109.77, 32.48) -- (109.77,262.20);

\path[draw=drawColor,line width= 0.6pt,line join=round] (109.77, 32.48) -- (109.77,262.20);

\path[draw=drawColor,line width= 0.6pt,line join=round] (109.77, 32.48) -- (109.77,262.20);

\path[draw=drawColor,line width= 0.6pt,line join=round] (109.77, 32.48) -- (109.77,262.20);

\path[draw=drawColor,line width= 0.6pt,line join=round] (109.77, 32.48) -- (109.77,262.20);

\path[draw=drawColor,line width= 0.6pt,line join=round] (109.77, 32.48) -- (109.77,262.20);

\path[draw=drawColor,line width= 0.6pt,line join=round] (109.77, 32.48) -- (109.77,262.20);

\path[draw=drawColor,line width= 0.6pt,line join=round] (109.77, 32.48) -- (109.77,262.20);

\path[draw=drawColor,line width= 0.6pt,line join=round] (109.77, 32.48) -- (109.77,262.20);

\path[draw=drawColor,line width= 0.6pt,line join=round] (109.77, 32.48) -- (109.77,262.20);

\path[draw=drawColor,line width= 0.6pt,line join=round] (109.77, 32.48) -- (109.77,262.20);

\path[draw=drawColor,line width= 0.6pt,line join=round] (109.77, 32.48) -- (109.77,262.20);

\path[draw=drawColor,line width= 0.6pt,line join=round] (109.77, 32.48) -- (109.77,262.20);

\path[draw=drawColor,line width= 0.6pt,line join=round] (109.77, 32.48) -- (109.77,262.20);

\path[draw=drawColor,line width= 0.6pt,line join=round] (109.77, 32.48) -- (109.77,262.20);

\path[draw=drawColor,line width= 0.6pt,line join=round] (109.77, 32.48) -- (109.77,262.20);

\path[draw=drawColor,line width= 0.6pt,line join=round] (109.77, 32.48) -- (109.77,262.20);

\path[draw=drawColor,line width= 0.6pt,line join=round] (109.77, 32.48) -- (109.77,262.20);

\path[draw=drawColor,line width= 0.6pt,line join=round] (109.77, 32.48) -- (109.77,262.20);

\path[draw=drawColor,line width= 0.6pt,line join=round] (109.77, 32.48) -- (109.77,262.20);

\path[draw=drawColor,line width= 0.6pt,line join=round] (109.77, 32.48) -- (109.77,262.20);

\path[draw=drawColor,line width= 0.6pt,line join=round] (109.77, 32.48) -- (109.77,262.20);

\path[draw=drawColor,line width= 0.6pt,line join=round] (109.77, 32.48) -- (109.77,262.20);

\path[draw=drawColor,line width= 0.6pt,line join=round] (109.77, 32.48) -- (109.77,262.20);

\path[draw=drawColor,line width= 0.6pt,line join=round] (109.77, 32.48) -- (109.77,262.20);

\path[draw=drawColor,line width= 0.6pt,line join=round] (109.77, 32.48) -- (109.77,262.20);

\path[draw=drawColor,line width= 0.6pt,line join=round] (109.77, 32.48) -- (109.77,262.20);

\path[draw=drawColor,line width= 0.6pt,line join=round] (109.77, 32.48) -- (109.77,262.20);

\path[draw=drawColor,line width= 0.6pt,line join=round] (109.77, 32.48) -- (109.77,262.20);

\path[draw=drawColor,line width= 0.6pt,line join=round] (109.77, 32.48) -- (109.77,262.20);

\path[draw=drawColor,line width= 0.6pt,line join=round] (109.77, 32.48) -- (109.77,262.20);

\path[draw=drawColor,line width= 0.6pt,line join=round] (109.77, 32.48) -- (109.77,262.20);

\path[draw=drawColor,line width= 0.6pt,line join=round] (109.77, 32.48) -- (109.77,262.20);

\path[draw=drawColor,line width= 0.6pt,line join=round] (109.77, 32.48) -- (109.77,262.20);

\path[draw=drawColor,line width= 0.6pt,line join=round] (109.77, 32.48) -- (109.77,262.20);

\path[draw=drawColor,line width= 0.6pt,line join=round] (109.77, 32.48) -- (109.77,262.20);

\path[draw=drawColor,line width= 0.6pt,line join=round] (109.77, 32.48) -- (109.77,262.20);

\path[draw=drawColor,line width= 0.6pt,line join=round] (109.77, 32.48) -- (109.77,262.20);

\path[draw=drawColor,line width= 0.6pt,line join=round] (109.77, 32.48) -- (109.77,262.20);

\path[draw=drawColor,line width= 0.6pt,line join=round] (109.77, 32.48) -- (109.77,262.20);

\path[draw=drawColor,line width= 0.6pt,line join=round] (109.77, 32.48) -- (109.77,262.20);

\path[draw=drawColor,line width= 0.6pt,line join=round] (109.77, 32.48) -- (109.77,262.20);

\path[draw=drawColor,line width= 0.6pt,line join=round] (109.77, 32.48) -- (109.77,262.20);

\path[draw=drawColor,line width= 0.6pt,line join=round] (109.77, 32.48) -- (109.77,262.20);

\path[draw=drawColor,line width= 0.6pt,line join=round] (109.77, 32.48) -- (109.77,262.20);

\path[draw=drawColor,line width= 0.6pt,line join=round] (109.77, 32.48) -- (109.77,262.20);

\path[draw=drawColor,line width= 0.6pt,line join=round] (109.77, 32.48) -- (109.77,262.20);

\path[draw=drawColor,line width= 0.6pt,line join=round] (109.77, 32.48) -- (109.77,262.20);

\path[draw=drawColor,line width= 0.6pt,line join=round] (109.77, 32.48) -- (109.77,262.20);

\path[draw=drawColor,line width= 0.6pt,line join=round] (109.77, 32.48) -- (109.77,262.20);

\path[draw=drawColor,line width= 0.6pt,line join=round] (109.77, 32.48) -- (109.77,262.20);

\path[draw=drawColor,line width= 0.6pt,line join=round] (109.77, 32.48) -- (109.77,262.20);

\path[draw=drawColor,line width= 0.6pt,line join=round] (109.77, 32.48) -- (109.77,262.20);

\path[draw=drawColor,line width= 0.6pt,line join=round] (109.77, 32.48) -- (109.77,262.20);

\path[draw=drawColor,line width= 0.6pt,line join=round] (109.77, 32.48) -- (109.77,262.20);

\path[draw=drawColor,line width= 0.6pt,line join=round] (109.77, 32.48) -- (109.77,262.20);

\path[draw=drawColor,line width= 0.6pt,line join=round] (109.77, 32.48) -- (109.77,262.20);

\path[draw=drawColor,line width= 0.6pt,line join=round] (109.77, 32.48) -- (109.77,262.20);

\path[draw=drawColor,line width= 0.6pt,line join=round] (109.77, 32.48) -- (109.77,262.20);

\path[draw=drawColor,line width= 0.6pt,line join=round] (109.77, 32.48) -- (109.77,262.20);

\path[draw=drawColor,line width= 0.6pt,line join=round] (109.77, 32.48) -- (109.77,262.20);

\path[draw=drawColor,line width= 0.6pt,line join=round] (109.77, 32.48) -- (109.77,262.20);

\path[draw=drawColor,line width= 0.6pt,line join=round] (109.77, 32.48) -- (109.77,262.20);

\path[draw=drawColor,line width= 0.6pt,line join=round] (109.77, 32.48) -- (109.77,262.20);

\path[draw=drawColor,line width= 0.6pt,line join=round] (109.77, 32.48) -- (109.77,262.20);

\path[draw=drawColor,line width= 0.6pt,line join=round] (109.77, 32.48) -- (109.77,262.20);

\path[draw=drawColor,line width= 0.6pt,line join=round] (109.77, 32.48) -- (109.77,262.20);

\path[draw=drawColor,line width= 0.6pt,line join=round] (109.77, 32.48) -- (109.77,262.20);

\path[draw=drawColor,line width= 0.6pt,line join=round] (109.77, 32.48) -- (109.77,262.20);

\path[draw=drawColor,line width= 0.6pt,line join=round] (109.77, 32.48) -- (109.77,262.20);

\path[draw=drawColor,line width= 0.6pt,line join=round] (109.77, 32.48) -- (109.77,262.20);

\path[draw=drawColor,line width= 0.6pt,line join=round] (109.77, 32.48) -- (109.77,262.20);

\path[draw=drawColor,line width= 0.6pt,line join=round] (109.77, 32.48) -- (109.77,262.20);

\path[draw=drawColor,line width= 0.6pt,line join=round] (109.77, 32.48) -- (109.77,262.20);

\path[draw=drawColor,line width= 0.6pt,line join=round] (109.77, 32.48) -- (109.77,262.20);

\path[draw=drawColor,line width= 0.6pt,line join=round] (109.77, 32.48) -- (109.77,262.20);

\path[draw=drawColor,line width= 0.6pt,line join=round] (109.77, 32.48) -- (109.77,262.20);

\path[draw=drawColor,line width= 0.6pt,line join=round] (109.77, 32.48) -- (109.77,262.20);

\path[draw=drawColor,line width= 0.6pt,line join=round] (109.77, 32.48) -- (109.77,262.20);

\path[draw=drawColor,line width= 0.6pt,line join=round] (109.77, 32.48) -- (109.77,262.20);

\path[draw=drawColor,line width= 0.6pt,line join=round] (109.77, 32.48) -- (109.77,262.20);

\path[draw=drawColor,line width= 0.6pt,line join=round] (109.77, 32.48) -- (109.77,262.20);

\path[draw=drawColor,line width= 0.6pt,line join=round] (109.77, 32.48) -- (109.77,262.20);

\path[draw=drawColor,line width= 0.6pt,line join=round] (109.77, 32.48) -- (109.77,262.20);

\path[draw=drawColor,line width= 0.6pt,line join=round] (109.77, 32.48) -- (109.77,262.20);

\path[draw=drawColor,line width= 0.6pt,line join=round] (109.77, 32.48) -- (109.77,262.20);

\path[draw=drawColor,line width= 0.6pt,line join=round] (109.77, 32.48) -- (109.77,262.20);

\path[draw=drawColor,line width= 0.6pt,line join=round] (109.77, 32.48) -- (109.77,262.20);

\path[draw=drawColor,line width= 0.6pt,line join=round] (109.77, 32.48) -- (109.77,262.20);

\path[draw=drawColor,line width= 0.6pt,line join=round] (109.77, 32.48) -- (109.77,262.20);

\path[draw=drawColor,line width= 0.6pt,line join=round] (109.77, 32.48) -- (109.77,262.20);

\path[draw=drawColor,line width= 0.6pt,line join=round] (109.77, 32.48) -- (109.77,262.20);

\path[draw=drawColor,line width= 0.6pt,line join=round] (109.77, 32.48) -- (109.77,262.20);

\path[draw=drawColor,line width= 0.6pt,line join=round] (109.77, 32.48) -- (109.77,262.20);

\path[draw=drawColor,line width= 0.6pt,line join=round] (109.77, 32.48) -- (109.77,262.20);

\path[draw=drawColor,line width= 0.6pt,line join=round] (109.77, 32.48) -- (109.77,262.20);

\path[draw=drawColor,line width= 0.6pt,line join=round] (109.77, 32.48) -- (109.77,262.20);

\path[draw=drawColor,line width= 0.6pt,line join=round] (109.77, 32.48) -- (109.77,262.20);

\path[draw=drawColor,line width= 0.6pt,line join=round] (109.77, 32.48) -- (109.77,262.20);

\path[draw=drawColor,line width= 0.6pt,line join=round] (109.77, 32.48) -- (109.77,262.20);

\path[draw=drawColor,line width= 0.6pt,line join=round] (109.77, 32.48) -- (109.77,262.20);

\path[draw=drawColor,line width= 0.6pt,line join=round] (109.77, 32.48) -- (109.77,262.20);

\path[draw=drawColor,line width= 0.6pt,line join=round] (109.77, 32.48) -- (109.77,262.20);

\path[draw=drawColor,line width= 0.6pt,line join=round] (109.77, 32.48) -- (109.77,262.20);

\path[draw=drawColor,line width= 0.6pt,line join=round] (109.77, 32.48) -- (109.77,262.20);

\path[draw=drawColor,line width= 0.6pt,line join=round] (109.77, 32.48) -- (109.77,262.20);

\path[draw=drawColor,line width= 0.6pt,line join=round] (109.77, 32.48) -- (109.77,262.20);

\path[draw=drawColor,line width= 0.6pt,line join=round] (109.77, 32.48) -- (109.77,262.20);

\path[draw=drawColor,line width= 0.6pt,line join=round] (109.77, 32.48) -- (109.77,262.20);

\path[draw=drawColor,line width= 0.6pt,line join=round] (109.77, 32.48) -- (109.77,262.20);

\path[draw=drawColor,line width= 0.6pt,line join=round] (109.77, 32.48) -- (109.77,262.20);

\path[draw=drawColor,line width= 0.6pt,line join=round] (109.77, 32.48) -- (109.77,262.20);

\path[draw=drawColor,line width= 0.6pt,line join=round] (109.77, 32.48) -- (109.77,262.20);

\path[draw=drawColor,line width= 0.6pt,line join=round] (109.77, 32.48) -- (109.77,262.20);

\path[draw=drawColor,line width= 0.6pt,line join=round] (109.77, 32.48) -- (109.77,262.20);

\path[draw=drawColor,line width= 0.6pt,line join=round] (109.77, 32.48) -- (109.77,262.20);

\path[draw=drawColor,line width= 0.6pt,line join=round] (109.77, 32.48) -- (109.77,262.20);

\path[draw=drawColor,line width= 0.6pt,line join=round] (109.77, 32.48) -- (109.77,262.20);

\path[draw=drawColor,line width= 0.6pt,line join=round] (109.77, 32.48) -- (109.77,262.20);

\path[draw=drawColor,line width= 0.6pt,line join=round] (109.77, 32.48) -- (109.77,262.20);

\path[draw=drawColor,line width= 0.6pt,line join=round] (109.77, 32.48) -- (109.77,262.20);

\path[draw=drawColor,line width= 0.6pt,line join=round] (109.77, 32.48) -- (109.77,262.20);

\path[draw=drawColor,line width= 0.6pt,line join=round] (109.77, 32.48) -- (109.77,262.20);

\path[draw=drawColor,line width= 0.6pt,line join=round] (109.77, 32.48) -- (109.77,262.20);

\path[draw=drawColor,line width= 0.6pt,line join=round] (109.77, 32.48) -- (109.77,262.20);

\path[draw=drawColor,line width= 0.6pt,line join=round] (109.77, 32.48) -- (109.77,262.20);

\path[draw=drawColor,line width= 0.6pt,line join=round] (109.77, 32.48) -- (109.77,262.20);

\path[draw=drawColor,line width= 0.6pt,line join=round] (109.77, 32.48) -- (109.77,262.20);

\path[draw=drawColor,line width= 0.6pt,line join=round] (109.77, 32.48) -- (109.77,262.20);

\path[draw=drawColor,line width= 0.6pt,line join=round] (109.77, 32.48) -- (109.77,262.20);

\path[draw=drawColor,line width= 0.6pt,line join=round] (109.77, 32.48) -- (109.77,262.20);

\path[draw=drawColor,line width= 0.6pt,line join=round] (109.77, 32.48) -- (109.77,262.20);

\path[draw=drawColor,line width= 0.6pt,line join=round] (109.77, 32.48) -- (109.77,262.20);

\path[draw=drawColor,line width= 0.6pt,line join=round] (109.77, 32.48) -- (109.77,262.20);

\path[draw=drawColor,line width= 0.6pt,line join=round] (109.77, 32.48) -- (109.77,262.20);

\path[draw=drawColor,line width= 0.6pt,line join=round] (109.77, 32.48) -- (109.77,262.20);

\path[draw=drawColor,line width= 0.6pt,line join=round] (109.77, 32.48) -- (109.77,262.20);

\path[draw=drawColor,line width= 0.6pt,line join=round] (109.77, 32.48) -- (109.77,262.20);

\path[draw=drawColor,line width= 0.6pt,line join=round] (109.77, 32.48) -- (109.77,262.20);

\path[draw=drawColor,line width= 0.6pt,line join=round] (109.77, 32.48) -- (109.77,262.20);

\path[draw=drawColor,line width= 0.6pt,line join=round] (109.77, 32.48) -- (109.77,262.20);

\path[draw=drawColor,line width= 0.6pt,line join=round] (109.77, 32.48) -- (109.77,262.20);

\path[draw=drawColor,line width= 0.6pt,line join=round] (109.77, 32.48) -- (109.77,262.20);

\path[draw=drawColor,line width= 0.6pt,line join=round] (109.77, 32.48) -- (109.77,262.20);

\path[draw=drawColor,line width= 0.6pt,line join=round] (109.77, 32.48) -- (109.77,262.20);

\path[draw=drawColor,line width= 0.6pt,line join=round] (109.77, 32.48) -- (109.77,262.20);

\path[draw=drawColor,line width= 0.6pt,line join=round] (109.77, 32.48) -- (109.77,262.20);

\path[draw=drawColor,line width= 0.6pt,line join=round] (109.77, 32.48) -- (109.77,262.20);

\path[draw=drawColor,line width= 0.6pt,line join=round] (109.77, 32.48) -- (109.77,262.20);

\path[draw=drawColor,line width= 0.6pt,line join=round] (109.77, 32.48) -- (109.77,262.20);

\path[draw=drawColor,line width= 0.6pt,line join=round] (109.77, 32.48) -- (109.77,262.20);

\path[draw=drawColor,line width= 0.6pt,line join=round] (109.77, 32.48) -- (109.77,262.20);

\path[draw=drawColor,line width= 0.6pt,line join=round] (109.77, 32.48) -- (109.77,262.20);

\path[draw=drawColor,line width= 0.6pt,line join=round] (109.77, 32.48) -- (109.77,262.20);

\path[draw=drawColor,line width= 0.6pt,line join=round] (109.77, 32.48) -- (109.77,262.20);

\path[draw=drawColor,line width= 0.6pt,line join=round] (109.77, 32.48) -- (109.77,262.20);

\path[draw=drawColor,line width= 0.6pt,line join=round] (109.77, 32.48) -- (109.77,262.20);

\path[draw=drawColor,line width= 0.6pt,line join=round] (109.77, 32.48) -- (109.77,262.20);

\path[draw=drawColor,line width= 0.6pt,line join=round] (109.77, 32.48) -- (109.77,262.20);

\path[draw=drawColor,line width= 0.6pt,line join=round] (109.77, 32.48) -- (109.77,262.20);

\path[draw=drawColor,line width= 0.6pt,line join=round] (109.77, 32.48) -- (109.77,262.20);

\path[draw=drawColor,line width= 0.6pt,line join=round] (109.77, 32.48) -- (109.77,262.20);

\path[draw=drawColor,line width= 0.6pt,line join=round] (109.77, 32.48) -- (109.77,262.20);

\path[draw=drawColor,line width= 0.6pt,line join=round] (109.77, 32.48) -- (109.77,262.20);

\path[draw=drawColor,line width= 0.6pt,line join=round] (109.77, 32.48) -- (109.77,262.20);

\path[draw=drawColor,line width= 0.6pt,line join=round] (109.77, 32.48) -- (109.77,262.20);

\path[draw=drawColor,line width= 0.6pt,line join=round] (109.77, 32.48) -- (109.77,262.20);

\path[draw=drawColor,line width= 0.6pt,line join=round] (109.77, 32.48) -- (109.77,262.20);

\path[draw=drawColor,line width= 0.6pt,line join=round] (109.77, 32.48) -- (109.77,262.20);

\path[draw=drawColor,line width= 0.6pt,line join=round] (109.77, 32.48) -- (109.77,262.20);

\path[draw=drawColor,line width= 0.6pt,line join=round] (109.77, 32.48) -- (109.77,262.20);

\path[draw=drawColor,line width= 0.6pt,line join=round] (109.77, 32.48) -- (109.77,262.20);

\path[draw=drawColor,line width= 0.6pt,line join=round] (109.77, 32.48) -- (109.77,262.20);

\path[draw=drawColor,line width= 0.6pt,line join=round] (109.77, 32.48) -- (109.77,262.20);

\path[draw=drawColor,line width= 0.6pt,line join=round] (109.77, 32.48) -- (109.77,262.20);

\path[draw=drawColor,line width= 0.6pt,line join=round] (109.77, 32.48) -- (109.77,262.20);

\path[draw=drawColor,line width= 0.6pt,line join=round] (109.77, 32.48) -- (109.77,262.20);

\path[draw=drawColor,line width= 0.6pt,line join=round] (109.77, 32.48) -- (109.77,262.20);

\path[draw=drawColor,line width= 0.6pt,line join=round] (109.77, 32.48) -- (109.77,262.20);

\path[draw=drawColor,line width= 0.6pt,line join=round] (109.77, 32.48) -- (109.77,262.20);

\path[draw=drawColor,line width= 0.6pt,line join=round] (109.77, 32.48) -- (109.77,262.20);

\path[draw=drawColor,line width= 0.6pt,line join=round] (109.77, 32.48) -- (109.77,262.20);

\path[draw=drawColor,line width= 0.6pt,line join=round] (109.77, 32.48) -- (109.77,262.20);

\path[draw=drawColor,line width= 0.6pt,line join=round] (109.77, 32.48) -- (109.77,262.20);

\path[draw=drawColor,line width= 0.6pt,line join=round] (109.77, 32.48) -- (109.77,262.20);

\path[draw=drawColor,line width= 0.6pt,line join=round] (109.77, 32.48) -- (109.77,262.20);

\path[draw=drawColor,line width= 0.6pt,line join=round] (109.77, 32.48) -- (109.77,262.20);

\path[draw=drawColor,line width= 0.6pt,line join=round] (109.77, 32.48) -- (109.77,262.20);

\path[draw=drawColor,line width= 0.6pt,line join=round] (109.77, 32.48) -- (109.77,262.20);

\path[draw=drawColor,line width= 0.6pt,line join=round] (109.77, 32.48) -- (109.77,262.20);

\path[draw=drawColor,line width= 0.6pt,line join=round] (109.77, 32.48) -- (109.77,262.20);

\path[draw=drawColor,line width= 0.6pt,line join=round] (109.77, 32.48) -- (109.77,262.20);

\path[draw=drawColor,line width= 0.6pt,line join=round] (109.77, 32.48) -- (109.77,262.20);

\path[draw=drawColor,line width= 0.6pt,line join=round] (109.77, 32.48) -- (109.77,262.20);

\path[draw=drawColor,line width= 0.6pt,line join=round] (109.77, 32.48) -- (109.77,262.20);

\path[draw=drawColor,line width= 0.6pt,line join=round] (109.77, 32.48) -- (109.77,262.20);

\path[draw=drawColor,line width= 0.6pt,line join=round] (109.77, 32.48) -- (109.77,262.20);

\path[draw=drawColor,line width= 0.6pt,line join=round] (109.77, 32.48) -- (109.77,262.20);

\path[draw=drawColor,line width= 0.6pt,line join=round] (109.77, 32.48) -- (109.77,262.20);

\path[draw=drawColor,line width= 0.6pt,line join=round] (109.77, 32.48) -- (109.77,262.20);

\path[draw=drawColor,line width= 0.6pt,line join=round] (109.77, 32.48) -- (109.77,262.20);

\path[draw=drawColor,line width= 0.6pt,line join=round] (109.77, 32.48) -- (109.77,262.20);

\path[draw=drawColor,line width= 0.6pt,line join=round] (109.77, 32.48) -- (109.77,262.20);

\path[draw=drawColor,line width= 0.6pt,line join=round] (109.77, 32.48) -- (109.77,262.20);

\path[draw=drawColor,line width= 0.6pt,line join=round] (109.77, 32.48) -- (109.77,262.20);

\path[draw=drawColor,line width= 0.6pt,line join=round] (109.77, 32.48) -- (109.77,262.20);

\path[draw=drawColor,line width= 0.6pt,line join=round] (109.77, 32.48) -- (109.77,262.20);

\path[draw=drawColor,line width= 0.6pt,line join=round] (109.77, 32.48) -- (109.77,262.20);

\path[draw=drawColor,line width= 0.6pt,line join=round] (109.77, 32.48) -- (109.77,262.20);

\path[draw=drawColor,line width= 0.6pt,line join=round] (109.77, 32.48) -- (109.77,262.20);

\path[draw=drawColor,line width= 0.6pt,line join=round] (109.77, 32.48) -- (109.77,262.20);

\path[draw=drawColor,line width= 0.6pt,line join=round] (109.77, 32.48) -- (109.77,262.20);

\path[draw=drawColor,line width= 0.6pt,line join=round] (109.77, 32.48) -- (109.77,262.20);

\path[draw=drawColor,line width= 0.6pt,line join=round] (109.77, 32.48) -- (109.77,262.20);

\path[draw=drawColor,line width= 0.6pt,line join=round] (109.77, 32.48) -- (109.77,262.20);

\path[draw=drawColor,line width= 0.6pt,line join=round] (109.77, 32.48) -- (109.77,262.20);

\path[draw=drawColor,line width= 0.6pt,line join=round] (109.77, 32.48) -- (109.77,262.20);

\path[draw=drawColor,line width= 0.6pt,line join=round] (109.77, 32.48) -- (109.77,262.20);

\path[draw=drawColor,line width= 0.6pt,line join=round] (109.77, 32.48) -- (109.77,262.20);

\path[draw=drawColor,line width= 0.6pt,line join=round] (109.77, 32.48) -- (109.77,262.20);

\path[draw=drawColor,line width= 0.6pt,line join=round] (109.77, 32.48) -- (109.77,262.20);

\path[draw=drawColor,line width= 0.6pt,line join=round] (109.77, 32.48) -- (109.77,262.20);

\path[draw=drawColor,line width= 0.6pt,line join=round] (109.77, 32.48) -- (109.77,262.20);

\path[draw=drawColor,line width= 0.6pt,line join=round] (109.77, 32.48) -- (109.77,262.20);

\path[draw=drawColor,line width= 0.6pt,line join=round] (109.77, 32.48) -- (109.77,262.20);

\path[draw=drawColor,line width= 0.6pt,line join=round] (109.77, 32.48) -- (109.77,262.20);

\path[draw=drawColor,line width= 0.6pt,line join=round] (109.77, 32.48) -- (109.77,262.20);

\path[draw=drawColor,line width= 0.6pt,line join=round] (109.77, 32.48) -- (109.77,262.20);

\path[draw=drawColor,line width= 0.6pt,line join=round] (109.77, 32.48) -- (109.77,262.20);

\path[draw=drawColor,line width= 0.6pt,line join=round] (109.77, 32.48) -- (109.77,262.20);

\path[draw=drawColor,line width= 0.6pt,line join=round] (109.77, 32.48) -- (109.77,262.20);

\path[draw=drawColor,line width= 0.6pt,line join=round] (109.77, 32.48) -- (109.77,262.20);

\path[draw=drawColor,line width= 0.6pt,line join=round] (109.77, 32.48) -- (109.77,262.20);

\path[draw=drawColor,line width= 0.6pt,line join=round] (109.77, 32.48) -- (109.77,262.20);

\path[draw=drawColor,line width= 0.6pt,line join=round] (109.77, 32.48) -- (109.77,262.20);

\path[draw=drawColor,line width= 0.6pt,line join=round] (109.77, 32.48) -- (109.77,262.20);

\path[draw=drawColor,line width= 0.6pt,line join=round] (109.77, 32.48) -- (109.77,262.20);

\path[draw=drawColor,line width= 0.6pt,line join=round] (109.77, 32.48) -- (109.77,262.20);

\path[draw=drawColor,line width= 0.6pt,line join=round] (109.77, 32.48) -- (109.77,262.20);

\path[draw=drawColor,line width= 0.6pt,line join=round] (109.77, 32.48) -- (109.77,262.20);

\path[draw=drawColor,line width= 0.6pt,line join=round] (109.77, 32.48) -- (109.77,262.20);

\path[draw=drawColor,line width= 0.6pt,line join=round] (109.77, 32.48) -- (109.77,262.20);

\path[draw=drawColor,line width= 0.6pt,line join=round] (109.77, 32.48) -- (109.77,262.20);

\path[draw=drawColor,line width= 0.6pt,line join=round] (109.77, 32.48) -- (109.77,262.20);

\path[draw=drawColor,line width= 0.6pt,line join=round] (109.77, 32.48) -- (109.77,262.20);

\path[draw=drawColor,line width= 0.6pt,line join=round] (109.77, 32.48) -- (109.77,262.20);

\path[draw=drawColor,line width= 0.6pt,line join=round] (109.77, 32.48) -- (109.77,262.20);

\path[draw=drawColor,line width= 0.6pt,line join=round] (109.77, 32.48) -- (109.77,262.20);

\path[draw=drawColor,line width= 0.6pt,line join=round] (109.77, 32.48) -- (109.77,262.20);

\path[draw=drawColor,line width= 0.6pt,line join=round] (109.77, 32.48) -- (109.77,262.20);

\path[draw=drawColor,line width= 0.6pt,line join=round] (109.77, 32.48) -- (109.77,262.20);

\path[draw=drawColor,line width= 0.6pt,line join=round] (109.77, 32.48) -- (109.77,262.20);

\path[draw=drawColor,line width= 0.6pt,line join=round] (109.77, 32.48) -- (109.77,262.20);

\path[draw=drawColor,line width= 0.6pt,line join=round] (109.77, 32.48) -- (109.77,262.20);

\path[draw=drawColor,line width= 0.6pt,line join=round] (109.77, 32.48) -- (109.77,262.20);

\path[draw=drawColor,line width= 0.6pt,line join=round] (109.77, 32.48) -- (109.77,262.20);

\path[draw=drawColor,line width= 0.6pt,line join=round] (109.77, 32.48) -- (109.77,262.20);

\path[draw=drawColor,line width= 0.6pt,line join=round] (109.77, 32.48) -- (109.77,262.20);

\path[draw=drawColor,line width= 0.6pt,line join=round] (109.77, 32.48) -- (109.77,262.20);

\path[draw=drawColor,line width= 0.6pt,line join=round] (109.77, 32.48) -- (109.77,262.20);

\path[draw=drawColor,line width= 0.6pt,line join=round] (109.77, 32.48) -- (109.77,262.20);

\path[draw=drawColor,line width= 0.6pt,line join=round] (109.77, 32.48) -- (109.77,262.20);

\path[draw=drawColor,line width= 0.6pt,line join=round] (109.77, 32.48) -- (109.77,262.20);

\path[draw=drawColor,line width= 0.6pt,line join=round] (109.77, 32.48) -- (109.77,262.20);

\path[draw=drawColor,line width= 0.6pt,line join=round] (109.77, 32.48) -- (109.77,262.20);

\path[draw=drawColor,line width= 0.6pt,line join=round] (109.77, 32.48) -- (109.77,262.20);

\path[draw=drawColor,line width= 0.6pt,line join=round] (109.77, 32.48) -- (109.77,262.20);

\path[draw=drawColor,line width= 0.6pt,line join=round] (109.77, 32.48) -- (109.77,262.20);

\path[draw=drawColor,line width= 0.6pt,line join=round] (109.77, 32.48) -- (109.77,262.20);

\path[draw=drawColor,line width= 0.6pt,line join=round] (109.77, 32.48) -- (109.77,262.20);

\path[draw=drawColor,line width= 0.6pt,line join=round] (109.77, 32.48) -- (109.77,262.20);

\path[draw=drawColor,line width= 0.6pt,line join=round] (109.77, 32.48) -- (109.77,262.20);

\path[draw=drawColor,line width= 0.6pt,line join=round] (109.77, 32.48) -- (109.77,262.20);

\path[draw=drawColor,line width= 0.6pt,line join=round] (109.77, 32.48) -- (109.77,262.20);

\path[draw=drawColor,line width= 0.6pt,line join=round] (109.77, 32.48) -- (109.77,262.20);

\path[draw=drawColor,line width= 0.6pt,line join=round] (109.77, 32.48) -- (109.77,262.20);

\path[draw=drawColor,line width= 0.6pt,line join=round] (109.77, 32.48) -- (109.77,262.20);

\path[draw=drawColor,line width= 0.6pt,line join=round] (109.77, 32.48) -- (109.77,262.20);

\path[draw=drawColor,line width= 0.6pt,line join=round] (109.77, 32.48) -- (109.77,262.20);

\path[draw=drawColor,line width= 0.6pt,line join=round] (109.77, 32.48) -- (109.77,262.20);

\path[draw=drawColor,line width= 0.6pt,line join=round] (109.77, 32.48) -- (109.77,262.20);

\path[draw=drawColor,line width= 0.6pt,line join=round] (109.77, 32.48) -- (109.77,262.20);

\path[draw=drawColor,line width= 0.6pt,line join=round] (109.77, 32.48) -- (109.77,262.20);

\path[draw=drawColor,line width= 0.6pt,line join=round] (109.77, 32.48) -- (109.77,262.20);

\path[draw=drawColor,line width= 0.6pt,line join=round] (109.77, 32.48) -- (109.77,262.20);

\path[draw=drawColor,line width= 0.6pt,line join=round] (109.77, 32.48) -- (109.77,262.20);

\path[draw=drawColor,line width= 0.6pt,line join=round] (109.77, 32.48) -- (109.77,262.20);

\path[draw=drawColor,line width= 0.6pt,line join=round] (109.77, 32.48) -- (109.77,262.20);

\path[draw=drawColor,line width= 0.6pt,line join=round] (109.77, 32.48) -- (109.77,262.20);

\path[draw=drawColor,line width= 0.6pt,line join=round] (109.77, 32.48) -- (109.77,262.20);

\path[draw=drawColor,line width= 0.6pt,line join=round] (109.77, 32.48) -- (109.77,262.20);

\path[draw=drawColor,line width= 0.6pt,line join=round] (109.77, 32.48) -- (109.77,262.20);

\path[draw=drawColor,line width= 0.6pt,line join=round] (109.77, 32.48) -- (109.77,262.20);

\path[draw=drawColor,line width= 0.6pt,line join=round] (109.77, 32.48) -- (109.77,262.20);

\path[draw=drawColor,line width= 0.6pt,line join=round] (109.77, 32.48) -- (109.77,262.20);

\path[draw=drawColor,line width= 0.6pt,line join=round] (109.77, 32.48) -- (109.77,262.20);

\path[draw=drawColor,line width= 0.6pt,line join=round] (109.77, 32.48) -- (109.77,262.20);

\path[draw=drawColor,line width= 0.6pt,line join=round] (109.77, 32.48) -- (109.77,262.20);

\path[draw=drawColor,line width= 0.6pt,line join=round] (109.77, 32.48) -- (109.77,262.20);

\path[draw=drawColor,line width= 0.6pt,line join=round] (109.77, 32.48) -- (109.77,262.20);

\path[draw=drawColor,line width= 0.6pt,line join=round] (109.77, 32.48) -- (109.77,262.20);

\path[draw=drawColor,line width= 0.6pt,line join=round] (109.77, 32.48) -- (109.77,262.20);

\path[draw=drawColor,line width= 0.6pt,line join=round] (109.77, 32.48) -- (109.77,262.20);

\path[draw=drawColor,line width= 0.6pt,line join=round] (109.77, 32.48) -- (109.77,262.20);

\path[draw=drawColor,line width= 0.6pt,line join=round] (109.77, 32.48) -- (109.77,262.20);

\path[draw=drawColor,line width= 0.6pt,line join=round] (109.77, 32.48) -- (109.77,262.20);

\path[draw=drawColor,line width= 0.6pt,line join=round] (109.77, 32.48) -- (109.77,262.20);

\path[draw=drawColor,line width= 0.6pt,line join=round] (109.77, 32.48) -- (109.77,262.20);

\path[draw=drawColor,line width= 0.6pt,line join=round] (109.77, 32.48) -- (109.77,262.20);

\path[draw=drawColor,line width= 0.6pt,line join=round] (109.77, 32.48) -- (109.77,262.20);

\path[draw=drawColor,line width= 0.6pt,line join=round] (109.77, 32.48) -- (109.77,262.20);

\path[draw=drawColor,line width= 0.6pt,line join=round] (109.77, 32.48) -- (109.77,262.20);

\path[draw=drawColor,line width= 0.6pt,line join=round] (109.77, 32.48) -- (109.77,262.20);

\path[draw=drawColor,line width= 0.6pt,line join=round] (109.77, 32.48) -- (109.77,262.20);

\path[draw=drawColor,line width= 0.6pt,line join=round] (109.77, 32.48) -- (109.77,262.20);

\path[draw=drawColor,line width= 0.6pt,line join=round] (109.77, 32.48) -- (109.77,262.20);

\path[draw=drawColor,line width= 0.6pt,line join=round] (109.77, 32.48) -- (109.77,262.20);

\path[draw=drawColor,line width= 0.6pt,line join=round] (109.77, 32.48) -- (109.77,262.20);

\path[draw=drawColor,line width= 0.6pt,line join=round] (109.77, 32.48) -- (109.77,262.20);

\path[draw=drawColor,line width= 0.6pt,line join=round] (109.77, 32.48) -- (109.77,262.20);

\path[draw=drawColor,line width= 0.6pt,line join=round] (109.77, 32.48) -- (109.77,262.20);

\path[draw=drawColor,line width= 0.6pt,line join=round] (109.77, 32.48) -- (109.77,262.20);

\path[draw=drawColor,line width= 0.6pt,line join=round] (109.77, 32.48) -- (109.77,262.20);

\path[draw=drawColor,line width= 0.6pt,line join=round] (109.77, 32.48) -- (109.77,262.20);

\path[draw=drawColor,line width= 0.6pt,line join=round] (109.77, 32.48) -- (109.77,262.20);

\path[draw=drawColor,line width= 0.6pt,line join=round] (109.77, 32.48) -- (109.77,262.20);

\path[draw=drawColor,line width= 0.6pt,line join=round] (109.77, 32.48) -- (109.77,262.20);

\path[draw=drawColor,line width= 0.6pt,line join=round] (109.77, 32.48) -- (109.77,262.20);

\path[draw=drawColor,line width= 0.6pt,line join=round] (109.77, 32.48) -- (109.77,262.20);

\path[draw=drawColor,line width= 0.6pt,line join=round] (109.77, 32.48) -- (109.77,262.20);

\path[draw=drawColor,line width= 0.6pt,line join=round] (109.77, 32.48) -- (109.77,262.20);

\path[draw=drawColor,line width= 0.6pt,line join=round] (109.77, 32.48) -- (109.77,262.20);

\path[draw=drawColor,line width= 0.6pt,line join=round] (109.77, 32.48) -- (109.77,262.20);

\path[draw=drawColor,line width= 0.6pt,line join=round] (109.77, 32.48) -- (109.77,262.20);

\path[draw=drawColor,line width= 0.6pt,line join=round] (109.77, 32.48) -- (109.77,262.20);

\path[draw=drawColor,line width= 0.6pt,line join=round] (109.77, 32.48) -- (109.77,262.20);

\path[draw=drawColor,line width= 0.6pt,line join=round] (109.77, 32.48) -- (109.77,262.20);

\path[draw=drawColor,line width= 0.6pt,line join=round] (109.77, 32.48) -- (109.77,262.20);

\path[draw=drawColor,line width= 0.6pt,line join=round] (109.77, 32.48) -- (109.77,262.20);

\path[draw=drawColor,line width= 0.6pt,line join=round] (109.77, 32.48) -- (109.77,262.20);

\path[draw=drawColor,line width= 0.6pt,line join=round] (109.77, 32.48) -- (109.77,262.20);

\path[draw=drawColor,line width= 0.6pt,line join=round] (109.77, 32.48) -- (109.77,262.20);

\path[draw=drawColor,line width= 0.6pt,line join=round] (109.77, 32.48) -- (109.77,262.20);

\path[draw=drawColor,line width= 0.6pt,line join=round] (109.77, 32.48) -- (109.77,262.20);

\path[draw=drawColor,line width= 0.6pt,line join=round] (109.77, 32.48) -- (109.77,262.20);

\path[draw=drawColor,line width= 0.6pt,line join=round] (109.77, 32.48) -- (109.77,262.20);

\path[draw=drawColor,line width= 0.6pt,line join=round] (109.77, 32.48) -- (109.77,262.20);

\path[draw=drawColor,line width= 0.6pt,line join=round] (109.77, 32.48) -- (109.77,262.20);

\path[draw=drawColor,line width= 0.6pt,line join=round] (109.77, 32.48) -- (109.77,262.20);

\path[draw=drawColor,line width= 0.6pt,line join=round] (109.77, 32.48) -- (109.77,262.20);

\path[draw=drawColor,line width= 0.6pt,line join=round] (109.77, 32.48) -- (109.77,262.20);

\path[draw=drawColor,line width= 0.6pt,line join=round] (109.77, 32.48) -- (109.77,262.20);

\path[draw=drawColor,line width= 0.6pt,line join=round] (109.77, 32.48) -- (109.77,262.20);

\path[draw=drawColor,line width= 0.6pt,line join=round] (109.77, 32.48) -- (109.77,262.20);

\path[draw=drawColor,line width= 0.6pt,line join=round] (109.77, 32.48) -- (109.77,262.20);

\path[draw=drawColor,line width= 0.6pt,line join=round] (109.77, 32.48) -- (109.77,262.20);

\path[draw=drawColor,line width= 0.6pt,line join=round] (109.77, 32.48) -- (109.77,262.20);

\path[draw=drawColor,line width= 0.6pt,line join=round] (109.77, 32.48) -- (109.77,262.20);

\path[draw=drawColor,line width= 0.6pt,line join=round] (109.77, 32.48) -- (109.77,262.20);

\path[draw=drawColor,line width= 0.6pt,line join=round] (109.77, 32.48) -- (109.77,262.20);

\path[draw=drawColor,line width= 0.6pt,line join=round] (109.77, 32.48) -- (109.77,262.20);

\path[draw=drawColor,line width= 0.6pt,line join=round] (109.77, 32.48) -- (109.77,262.20);

\path[draw=drawColor,line width= 0.6pt,line join=round] (109.77, 32.48) -- (109.77,262.20);

\path[draw=drawColor,line width= 0.6pt,line join=round] (109.77, 32.48) -- (109.77,262.20);

\path[draw=drawColor,line width= 0.6pt,line join=round] (109.77, 32.48) -- (109.77,262.20);

\path[draw=drawColor,line width= 0.6pt,line join=round] (109.77, 32.48) -- (109.77,262.20);

\path[draw=drawColor,line width= 0.6pt,line join=round] (109.77, 32.48) -- (109.77,262.20);

\path[draw=drawColor,line width= 0.6pt,line join=round] (109.77, 32.48) -- (109.77,262.20);

\path[draw=drawColor,line width= 0.6pt,line join=round] (109.77, 32.48) -- (109.77,262.20);

\path[draw=drawColor,line width= 0.6pt,line join=round] (109.77, 32.48) -- (109.77,262.20);

\path[draw=drawColor,line width= 0.6pt,line join=round] (109.77, 32.48) -- (109.77,262.20);

\path[draw=drawColor,line width= 0.6pt,line join=round] (109.77, 32.48) -- (109.77,262.20);

\path[draw=drawColor,line width= 0.6pt,line join=round] (109.77, 32.48) -- (109.77,262.20);

\path[draw=drawColor,line width= 0.6pt,line join=round] (109.77, 32.48) -- (109.77,262.20);

\path[draw=drawColor,line width= 0.6pt,line join=round] (109.77, 32.48) -- (109.77,262.20);

\path[draw=drawColor,line width= 0.6pt,line join=round] (109.77, 32.48) -- (109.77,262.20);

\path[draw=drawColor,line width= 0.6pt,line join=round] (109.77, 32.48) -- (109.77,262.20);

\path[draw=drawColor,line width= 0.6pt,line join=round] (109.77, 32.48) -- (109.77,262.20);

\path[draw=drawColor,line width= 0.6pt,line join=round] (109.77, 32.48) -- (109.77,262.20);

\path[draw=drawColor,line width= 0.6pt,line join=round] (109.77, 32.48) -- (109.77,262.20);

\path[draw=drawColor,line width= 0.6pt,line join=round] (109.77, 32.48) -- (109.77,262.20);

\path[draw=drawColor,line width= 0.6pt,line join=round] (109.77, 32.48) -- (109.77,262.20);

\path[draw=drawColor,line width= 0.6pt,line join=round] (109.77, 32.48) -- (109.77,262.20);

\path[draw=drawColor,line width= 0.6pt,line join=round] (109.77, 32.48) -- (109.77,262.20);

\path[draw=drawColor,line width= 0.6pt,line join=round] (109.77, 32.48) -- (109.77,262.20);

\path[draw=drawColor,line width= 0.6pt,line join=round] (109.77, 32.48) -- (109.77,262.20);

\path[draw=drawColor,line width= 0.6pt,line join=round] (109.77, 32.48) -- (109.77,262.20);

\path[draw=drawColor,line width= 0.6pt,line join=round] (109.77, 32.48) -- (109.77,262.20);

\path[draw=drawColor,line width= 0.6pt,line join=round] (109.77, 32.48) -- (109.77,262.20);

\path[draw=drawColor,line width= 0.6pt,line join=round] (109.77, 32.48) -- (109.77,262.20);

\path[draw=drawColor,line width= 0.6pt,line join=round] (109.77, 32.48) -- (109.77,262.20);

\path[draw=drawColor,line width= 0.6pt,line join=round] (109.77, 32.48) -- (109.77,262.20);

\path[draw=drawColor,line width= 0.6pt,line join=round] (109.77, 32.48) -- (109.77,262.20);

\path[draw=drawColor,line width= 0.6pt,line join=round] (109.77, 32.48) -- (109.77,262.20);

\path[draw=drawColor,line width= 0.6pt,line join=round] (109.77, 32.48) -- (109.77,262.20);

\path[draw=drawColor,line width= 0.6pt,line join=round] (109.77, 32.48) -- (109.77,262.20);

\path[draw=drawColor,line width= 0.6pt,line join=round] (109.77, 32.48) -- (109.77,262.20);

\path[draw=drawColor,line width= 0.6pt,line join=round] (109.77, 32.48) -- (109.77,262.20);

\path[draw=drawColor,line width= 0.6pt,line join=round] (109.77, 32.48) -- (109.77,262.20);

\path[draw=drawColor,line width= 0.6pt,line join=round] (109.77, 32.48) -- (109.77,262.20);

\path[draw=drawColor,line width= 0.6pt,line join=round] (109.77, 32.48) -- (109.77,262.20);

\path[draw=drawColor,line width= 0.6pt,line join=round] (109.77, 32.48) -- (109.77,262.20);

\path[draw=drawColor,line width= 0.6pt,line join=round] (109.77, 32.48) -- (109.77,262.20);

\path[draw=drawColor,line width= 0.6pt,line join=round] (109.77, 32.48) -- (109.77,262.20);

\path[draw=drawColor,line width= 0.6pt,line join=round] (109.77, 32.48) -- (109.77,262.20);

\path[draw=drawColor,line width= 0.6pt,line join=round] (109.77, 32.48) -- (109.77,262.20);

\path[draw=drawColor,line width= 0.6pt,line join=round] (109.77, 32.48) -- (109.77,262.20);

\path[draw=drawColor,line width= 0.6pt,line join=round] (109.77, 32.48) -- (109.77,262.20);

\path[draw=drawColor,line width= 0.6pt,line join=round] (109.77, 32.48) -- (109.77,262.20);

\path[draw=drawColor,line width= 0.6pt,line join=round] (109.77, 32.48) -- (109.77,262.20);

\path[draw=drawColor,line width= 0.6pt,line join=round] (109.77, 32.48) -- (109.77,262.20);

\path[draw=drawColor,line width= 0.6pt,line join=round] (109.77, 32.48) -- (109.77,262.20);

\path[draw=drawColor,line width= 0.6pt,line join=round] (109.77, 32.48) -- (109.77,262.20);

\path[draw=drawColor,line width= 0.6pt,line join=round] (109.77, 32.48) -- (109.77,262.20);

\path[draw=drawColor,line width= 0.6pt,line join=round] (109.77, 32.48) -- (109.77,262.20);

\path[draw=drawColor,line width= 0.6pt,line join=round] (109.77, 32.48) -- (109.77,262.20);

\path[draw=drawColor,line width= 0.6pt,line join=round] (109.77, 32.48) -- (109.77,262.20);

\path[draw=drawColor,line width= 0.6pt,line join=round] (109.77, 32.48) -- (109.77,262.20);

\path[draw=drawColor,line width= 0.6pt,line join=round] (109.77, 32.48) -- (109.77,262.20);

\path[draw=drawColor,line width= 0.6pt,line join=round] (109.77, 32.48) -- (109.77,262.20);

\path[draw=drawColor,line width= 0.6pt,line join=round] (109.77, 32.48) -- (109.77,262.20);

\path[draw=drawColor,line width= 0.6pt,line join=round] (109.77, 32.48) -- (109.77,262.20);

\path[draw=drawColor,line width= 0.6pt,line join=round] (109.77, 32.48) -- (109.77,262.20);

\path[draw=drawColor,line width= 0.6pt,line join=round] (109.77, 32.48) -- (109.77,262.20);

\path[draw=drawColor,line width= 0.6pt,line join=round] (109.77, 32.48) -- (109.77,262.20);

\path[draw=drawColor,line width= 0.6pt,line join=round] (109.77, 32.48) -- (109.77,262.20);

\path[draw=drawColor,line width= 0.6pt,line join=round] (109.77, 32.48) -- (109.77,262.20);

\path[draw=drawColor,line width= 0.6pt,line join=round] (109.77, 32.48) -- (109.77,262.20);

\path[draw=drawColor,line width= 0.6pt,line join=round] (109.77, 32.48) -- (109.77,262.20);

\path[draw=drawColor,line width= 0.6pt,line join=round] (109.77, 32.48) -- (109.77,262.20);

\path[draw=drawColor,line width= 0.6pt,line join=round] (109.77, 32.48) -- (109.77,262.20);

\path[draw=drawColor,line width= 0.6pt,line join=round] (109.77, 32.48) -- (109.77,262.20);

\path[draw=drawColor,line width= 0.6pt,line join=round] (109.77, 32.48) -- (109.77,262.20);

\path[draw=drawColor,line width= 0.6pt,line join=round] (109.77, 32.48) -- (109.77,262.20);

\path[draw=drawColor,line width= 0.6pt,line join=round] (109.77, 32.48) -- (109.77,262.20);

\path[draw=drawColor,line width= 0.6pt,line join=round] (109.77, 32.48) -- (109.77,262.20);

\path[draw=drawColor,line width= 0.6pt,line join=round] (109.77, 32.48) -- (109.77,262.20);

\path[draw=drawColor,line width= 0.6pt,line join=round] (109.77, 32.48) -- (109.77,262.20);

\path[draw=drawColor,line width= 0.6pt,line join=round] (109.77, 32.48) -- (109.77,262.20);

\path[draw=drawColor,line width= 0.6pt,line join=round] (109.77, 32.48) -- (109.77,262.20);

\path[draw=drawColor,line width= 0.6pt,line join=round] (109.77, 32.48) -- (109.77,262.20);

\path[draw=drawColor,line width= 0.6pt,line join=round] (109.77, 32.48) -- (109.77,262.20);

\path[draw=drawColor,line width= 0.6pt,line join=round] (109.77, 32.48) -- (109.77,262.20);

\path[draw=drawColor,line width= 0.6pt,line join=round] (109.77, 32.48) -- (109.77,262.20);

\path[draw=drawColor,line width= 0.6pt,line join=round] (109.77, 32.48) -- (109.77,262.20);

\path[draw=drawColor,line width= 0.6pt,line join=round] (109.77, 32.48) -- (109.77,262.20);

\path[draw=drawColor,line width= 0.6pt,line join=round] (109.77, 32.48) -- (109.77,262.20);

\path[draw=drawColor,line width= 0.6pt,line join=round] (109.77, 32.48) -- (109.77,262.20);

\path[draw=drawColor,line width= 0.6pt,line join=round] (109.77, 32.48) -- (109.77,262.20);

\path[draw=drawColor,line width= 0.6pt,line join=round] (109.77, 32.48) -- (109.77,262.20);

\path[draw=drawColor,line width= 0.6pt,line join=round] (109.77, 32.48) -- (109.77,262.20);

\path[draw=drawColor,line width= 0.6pt,line join=round] (109.77, 32.48) -- (109.77,262.20);

\path[draw=drawColor,line width= 0.6pt,line join=round] (109.77, 32.48) -- (109.77,262.20);

\path[draw=drawColor,line width= 0.6pt,line join=round] (109.77, 32.48) -- (109.77,262.20);

\path[draw=drawColor,line width= 0.6pt,line join=round] (109.77, 32.48) -- (109.77,262.20);

\path[draw=drawColor,line width= 0.6pt,line join=round] (109.77, 32.48) -- (109.77,262.20);

\path[draw=drawColor,line width= 0.6pt,line join=round] (109.77, 32.48) -- (109.77,262.20);

\path[draw=drawColor,line width= 0.6pt,line join=round] (109.77, 32.48) -- (109.77,262.20);

\path[draw=drawColor,line width= 0.6pt,line join=round] (109.77, 32.48) -- (109.77,262.20);

\path[draw=drawColor,line width= 0.6pt,line join=round] (109.77, 32.48) -- (109.77,262.20);

\path[draw=drawColor,line width= 0.6pt,line join=round] (109.77, 32.48) -- (109.77,262.20);

\path[draw=drawColor,line width= 0.6pt,line join=round] (109.77, 32.48) -- (109.77,262.20);

\path[draw=drawColor,line width= 0.6pt,line join=round] (109.77, 32.48) -- (109.77,262.20);

\path[draw=drawColor,line width= 0.6pt,line join=round] (109.77, 32.48) -- (109.77,262.20);

\path[draw=drawColor,line width= 0.6pt,line join=round] (109.77, 32.48) -- (109.77,262.20);

\path[draw=drawColor,line width= 0.6pt,line join=round] (109.77, 32.48) -- (109.77,262.20);

\path[draw=drawColor,line width= 0.6pt,line join=round] (109.77, 32.48) -- (109.77,262.20);

\path[draw=drawColor,line width= 0.6pt,line join=round] (109.77, 32.48) -- (109.77,262.20);

\path[draw=drawColor,line width= 0.6pt,line join=round] (109.77, 32.48) -- (109.77,262.20);

\path[draw=drawColor,line width= 0.6pt,line join=round] (109.77, 32.48) -- (109.77,262.20);

\path[draw=drawColor,line width= 0.6pt,line join=round] (109.77, 32.48) -- (109.77,262.20);

\path[draw=drawColor,line width= 0.6pt,line join=round] (109.77, 32.48) -- (109.77,262.20);

\path[draw=drawColor,line width= 0.6pt,line join=round] (109.77, 32.48) -- (109.77,262.20);

\path[draw=drawColor,line width= 0.6pt,line join=round] (109.77, 32.48) -- (109.77,262.20);

\path[draw=drawColor,line width= 0.6pt,line join=round] (109.77, 32.48) -- (109.77,262.20);

\path[draw=drawColor,line width= 0.6pt,line join=round] (109.77, 32.48) -- (109.77,262.20);

\path[draw=drawColor,line width= 0.6pt,line join=round] (109.77, 32.48) -- (109.77,262.20);

\path[draw=drawColor,line width= 0.6pt,line join=round] (109.77, 32.48) -- (109.77,262.20);

\path[draw=drawColor,line width= 0.6pt,line join=round] (109.77, 32.48) -- (109.77,262.20);

\path[draw=drawColor,line width= 0.6pt,line join=round] (109.77, 32.48) -- (109.77,262.20);

\path[draw=drawColor,line width= 0.6pt,line join=round] (109.77, 32.48) -- (109.77,262.20);

\path[draw=drawColor,line width= 0.6pt,line join=round] (109.77, 32.48) -- (109.77,262.20);

\path[draw=drawColor,line width= 0.6pt,line join=round] (109.77, 32.48) -- (109.77,262.20);

\path[draw=drawColor,line width= 0.6pt,line join=round] (109.77, 32.48) -- (109.77,262.20);

\path[draw=drawColor,line width= 0.6pt,line join=round] (109.77, 32.48) -- (109.77,262.20);

\path[draw=drawColor,line width= 0.6pt,line join=round] (109.77, 32.48) -- (109.77,262.20);

\path[draw=drawColor,line width= 0.6pt,line join=round] (109.77, 32.48) -- (109.77,262.20);

\path[draw=drawColor,line width= 0.6pt,line join=round] (109.77, 32.48) -- (109.77,262.20);

\path[draw=drawColor,line width= 0.6pt,line join=round] (109.77, 32.48) -- (109.77,262.20);

\path[draw=drawColor,line width= 0.6pt,line join=round] (109.77, 32.48) -- (109.77,262.20);

\path[draw=drawColor,line width= 0.6pt,line join=round] (109.77, 32.48) -- (109.77,262.20);

\path[draw=drawColor,line width= 0.6pt,line join=round] (109.77, 32.48) -- (109.77,262.20);

\path[draw=drawColor,line width= 0.6pt,line join=round] (109.77, 32.48) -- (109.77,262.20);

\path[draw=drawColor,line width= 0.6pt,line join=round] (109.77, 32.48) -- (109.77,262.20);

\path[draw=drawColor,line width= 0.6pt,line join=round] (109.77, 32.48) -- (109.77,262.20);

\path[draw=drawColor,line width= 0.6pt,line join=round] (109.77, 32.48) -- (109.77,262.20);

\path[draw=drawColor,line width= 0.6pt,line join=round] (109.77, 32.48) -- (109.77,262.20);

\path[draw=drawColor,line width= 0.6pt,line join=round] (109.77, 32.48) -- (109.77,262.20);

\path[draw=drawColor,line width= 0.6pt,line join=round] (109.77, 32.48) -- (109.77,262.20);

\path[draw=drawColor,line width= 0.6pt,line join=round] (109.77, 32.48) -- (109.77,262.20);

\path[draw=drawColor,line width= 0.6pt,line join=round] (109.77, 32.48) -- (109.77,262.20);

\path[draw=drawColor,line width= 0.6pt,line join=round] (109.77, 32.48) -- (109.77,262.20);

\path[draw=drawColor,line width= 0.6pt,line join=round] (109.77, 32.48) -- (109.77,262.20);

\path[draw=drawColor,line width= 0.6pt,line join=round] (109.77, 32.48) -- (109.77,262.20);

\path[draw=drawColor,line width= 0.6pt,line join=round] (109.77, 32.48) -- (109.77,262.20);

\path[draw=drawColor,line width= 0.6pt,line join=round] (109.77, 32.48) -- (109.77,262.20);

\path[draw=drawColor,line width= 0.6pt,line join=round] (109.77, 32.48) -- (109.77,262.20);

\path[draw=drawColor,line width= 0.6pt,line join=round] (109.77, 32.48) -- (109.77,262.20);

\path[draw=drawColor,line width= 0.6pt,line join=round] (109.77, 32.48) -- (109.77,262.20);

\path[draw=drawColor,line width= 0.6pt,line join=round] (109.77, 32.48) -- (109.77,262.20);

\path[draw=drawColor,line width= 0.6pt,line join=round] (109.77, 32.48) -- (109.77,262.20);

\path[draw=drawColor,line width= 0.6pt,line join=round] (109.77, 32.48) -- (109.77,262.20);

\path[draw=drawColor,line width= 0.6pt,line join=round] (109.77, 32.48) -- (109.77,262.20);

\path[draw=drawColor,line width= 0.6pt,line join=round] (109.77, 32.48) -- (109.77,262.20);

\path[draw=drawColor,line width= 0.6pt,line join=round] (109.77, 32.48) -- (109.77,262.20);

\path[draw=drawColor,line width= 0.6pt,line join=round] (109.77, 32.48) -- (109.77,262.20);

\path[draw=drawColor,line width= 0.6pt,line join=round] (109.77, 32.48) -- (109.77,262.20);

\path[draw=drawColor,line width= 0.6pt,line join=round] (109.77, 32.48) -- (109.77,262.20);

\path[draw=drawColor,line width= 0.6pt,line join=round] (109.77, 32.48) -- (109.77,262.20);

\path[draw=drawColor,line width= 0.6pt,line join=round] (109.77, 32.48) -- (109.77,262.20);

\path[draw=drawColor,line width= 0.6pt,line join=round] (109.77, 32.48) -- (109.77,262.20);

\path[draw=drawColor,line width= 0.6pt,line join=round] (109.77, 32.48) -- (109.77,262.20);

\path[draw=drawColor,line width= 0.6pt,line join=round] (109.77, 32.48) -- (109.77,262.20);

\path[draw=drawColor,line width= 0.6pt,line join=round] (109.77, 32.48) -- (109.77,262.20);

\path[draw=drawColor,line width= 0.6pt,line join=round] (109.77, 32.48) -- (109.77,262.20);

\path[draw=drawColor,line width= 0.6pt,line join=round] (109.77, 32.48) -- (109.77,262.20);

\path[draw=drawColor,line width= 0.6pt,line join=round] (109.77, 32.48) -- (109.77,262.20);

\path[draw=drawColor,line width= 0.6pt,line join=round] (109.77, 32.48) -- (109.77,262.20);

\path[draw=drawColor,line width= 0.6pt,line join=round] (109.77, 32.48) -- (109.77,262.20);

\path[draw=drawColor,line width= 0.6pt,line join=round] (109.77, 32.48) -- (109.77,262.20);

\path[draw=drawColor,line width= 0.6pt,line join=round] (109.77, 32.48) -- (109.77,262.20);

\path[draw=drawColor,line width= 0.6pt,line join=round] (109.77, 32.48) -- (109.77,262.20);

\path[draw=drawColor,line width= 0.6pt,line join=round] (109.77, 32.48) -- (109.77,262.20);

\path[draw=drawColor,line width= 0.6pt,line join=round] (109.77, 32.48) -- (109.77,262.20);

\path[draw=drawColor,line width= 0.6pt,line join=round] (109.77, 32.48) -- (109.77,262.20);

\path[draw=drawColor,line width= 0.6pt,line join=round] (109.77, 32.48) -- (109.77,262.20);

\path[draw=drawColor,line width= 0.6pt,line join=round] (109.77, 32.48) -- (109.77,262.20);

\path[draw=drawColor,line width= 0.6pt,line join=round] (109.77, 32.48) -- (109.77,262.20);

\path[draw=drawColor,line width= 0.6pt,line join=round] (109.77, 32.48) -- (109.77,262.20);

\path[draw=drawColor,line width= 0.6pt,line join=round] (109.77, 32.48) -- (109.77,262.20);

\path[draw=drawColor,line width= 0.6pt,line join=round] (109.77, 32.48) -- (109.77,262.20);

\path[draw=drawColor,line width= 0.6pt,line join=round] (109.77, 32.48) -- (109.77,262.20);

\path[draw=drawColor,line width= 0.6pt,line join=round] (109.77, 32.48) -- (109.77,262.20);

\path[draw=drawColor,line width= 0.6pt,line join=round] (109.77, 32.48) -- (109.77,262.20);

\path[draw=drawColor,line width= 0.6pt,line join=round] (109.77, 32.48) -- (109.77,262.20);

\path[draw=drawColor,line width= 0.6pt,line join=round] (109.77, 32.48) -- (109.77,262.20);

\path[draw=drawColor,line width= 0.6pt,line join=round] (109.77, 32.48) -- (109.77,262.20);

\path[draw=drawColor,line width= 0.6pt,line join=round] (109.77, 32.48) -- (109.77,262.20);

\path[draw=drawColor,line width= 0.6pt,line join=round] (109.77, 32.48) -- (109.77,262.20);

\path[draw=drawColor,line width= 0.6pt,line join=round] (109.77, 32.48) -- (109.77,262.20);

\path[draw=drawColor,line width= 0.6pt,line join=round] (109.77, 32.48) -- (109.77,262.20);

\path[draw=drawColor,line width= 0.6pt,line join=round] (109.77, 32.48) -- (109.77,262.20);

\path[draw=drawColor,line width= 0.6pt,line join=round] (109.77, 32.48) -- (109.77,262.20);

\path[draw=drawColor,line width= 0.6pt,line join=round] (109.77, 32.48) -- (109.77,262.20);

\path[draw=drawColor,line width= 0.6pt,line join=round] (109.77, 32.48) -- (109.77,262.20);

\path[draw=drawColor,line width= 0.6pt,line join=round] (109.77, 32.48) -- (109.77,262.20);

\path[draw=drawColor,line width= 0.6pt,line join=round] (109.77, 32.48) -- (109.77,262.20);

\path[draw=drawColor,line width= 0.6pt,line join=round] (109.77, 32.48) -- (109.77,262.20);

\path[draw=drawColor,line width= 0.6pt,line join=round] (109.77, 32.48) -- (109.77,262.20);

\path[draw=drawColor,line width= 0.6pt,line join=round] (109.77, 32.48) -- (109.77,262.20);

\path[draw=drawColor,line width= 0.6pt,line join=round] (109.77, 32.48) -- (109.77,262.20);

\path[draw=drawColor,line width= 0.6pt,line join=round] (109.77, 32.48) -- (109.77,262.20);

\path[draw=drawColor,line width= 0.6pt,line join=round] (109.77, 32.48) -- (109.77,262.20);

\path[draw=drawColor,line width= 0.6pt,line join=round] (109.77, 32.48) -- (109.77,262.20);

\path[draw=drawColor,line width= 0.6pt,line join=round] (109.77, 32.48) -- (109.77,262.20);

\path[draw=drawColor,line width= 0.6pt,line join=round] (109.77, 32.48) -- (109.77,262.20);

\path[draw=drawColor,line width= 0.6pt,line join=round] (109.77, 32.48) -- (109.77,262.20);

\path[draw=drawColor,line width= 0.6pt,line join=round] (109.77, 32.48) -- (109.77,262.20);

\path[draw=drawColor,line width= 0.6pt,line join=round] (109.77, 32.48) -- (109.77,262.20);

\path[draw=drawColor,line width= 0.6pt,line join=round] (109.77, 32.48) -- (109.77,262.20);

\path[draw=drawColor,line width= 0.6pt,line join=round] (109.77, 32.48) -- (109.77,262.20);

\path[draw=drawColor,line width= 0.6pt,line join=round] (109.77, 32.48) -- (109.77,262.20);

\path[draw=drawColor,line width= 0.6pt,line join=round] (109.77, 32.48) -- (109.77,262.20);

\path[draw=drawColor,line width= 0.6pt,line join=round] (109.77, 32.48) -- (109.77,262.20);

\path[draw=drawColor,line width= 0.6pt,line join=round] (109.77, 32.48) -- (109.77,262.20);

\path[draw=drawColor,line width= 0.6pt,line join=round] (109.77, 32.48) -- (109.77,262.20);

\path[draw=drawColor,line width= 0.6pt,line join=round] (109.77, 32.48) -- (109.77,262.20);

\path[draw=drawColor,line width= 0.6pt,line join=round] (109.77, 32.48) -- (109.77,262.20);

\path[draw=drawColor,line width= 0.6pt,line join=round] (109.77, 32.48) -- (109.77,262.20);

\path[draw=drawColor,line width= 0.6pt,line join=round] (109.77, 32.48) -- (109.77,262.20);

\path[draw=drawColor,line width= 0.6pt,line join=round] (109.77, 32.48) -- (109.77,262.20);

\path[draw=drawColor,line width= 0.6pt,line join=round] (109.77, 32.48) -- (109.77,262.20);

\path[draw=drawColor,line width= 0.6pt,line join=round] (109.77, 32.48) -- (109.77,262.20);

\path[draw=drawColor,line width= 0.6pt,line join=round] (109.77, 32.48) -- (109.77,262.20);

\path[draw=drawColor,line width= 0.6pt,line join=round] (109.77, 32.48) -- (109.77,262.20);

\path[draw=drawColor,line width= 0.6pt,line join=round] (109.77, 32.48) -- (109.77,262.20);

\path[draw=drawColor,line width= 0.6pt,line join=round] (109.77, 32.48) -- (109.77,262.20);

\path[draw=drawColor,line width= 0.6pt,line join=round] (109.77, 32.48) -- (109.77,262.20);

\path[draw=drawColor,line width= 0.6pt,line join=round] (109.77, 32.48) -- (109.77,262.20);

\path[draw=drawColor,line width= 0.6pt,line join=round] (109.77, 32.48) -- (109.77,262.20);

\path[draw=drawColor,line width= 0.6pt,line join=round] (109.77, 32.48) -- (109.77,262.20);

\path[draw=drawColor,line width= 0.6pt,line join=round] (109.77, 32.48) -- (109.77,262.20);

\path[draw=drawColor,line width= 0.6pt,line join=round] (109.77, 32.48) -- (109.77,262.20);

\path[draw=drawColor,line width= 0.6pt,line join=round] (109.77, 32.48) -- (109.77,262.20);

\path[draw=drawColor,line width= 0.6pt,line join=round] (109.77, 32.48) -- (109.77,262.20);

\path[draw=drawColor,line width= 0.6pt,line join=round] (109.77, 32.48) -- (109.77,262.20);

\path[draw=drawColor,line width= 0.6pt,line join=round] (109.77, 32.48) -- (109.77,262.20);

\path[draw=drawColor,line width= 0.6pt,line join=round] (109.77, 32.48) -- (109.77,262.20);

\path[draw=drawColor,line width= 0.6pt,line join=round] (109.77, 32.48) -- (109.77,262.20);

\path[draw=drawColor,line width= 0.6pt,line join=round] (109.77, 32.48) -- (109.77,262.20);

\path[draw=drawColor,line width= 0.6pt,line join=round] (109.77, 32.48) -- (109.77,262.20);

\path[draw=drawColor,line width= 0.6pt,line join=round] (109.77, 32.48) -- (109.77,262.20);

\path[draw=drawColor,line width= 0.6pt,line join=round] (109.77, 32.48) -- (109.77,262.20);

\path[draw=drawColor,line width= 0.6pt,line join=round] (109.77, 32.48) -- (109.77,262.20);

\path[draw=drawColor,line width= 0.6pt,line join=round] (109.77, 32.48) -- (109.77,262.20);

\path[draw=drawColor,line width= 0.6pt,line join=round] (109.77, 32.48) -- (109.77,262.20);

\path[draw=drawColor,line width= 0.6pt,line join=round] (109.77, 32.48) -- (109.77,262.20);

\path[draw=drawColor,line width= 0.6pt,line join=round] (109.77, 32.48) -- (109.77,262.20);

\path[draw=drawColor,line width= 0.6pt,line join=round] (109.77, 32.48) -- (109.77,262.20);

\path[draw=drawColor,line width= 0.6pt,line join=round] (109.77, 32.48) -- (109.77,262.20);

\path[draw=drawColor,line width= 0.6pt,line join=round] (109.77, 32.48) -- (109.77,262.20);

\path[draw=drawColor,line width= 0.6pt,line join=round] (109.77, 32.48) -- (109.77,262.20);

\path[draw=drawColor,line width= 0.6pt,line join=round] (109.77, 32.48) -- (109.77,262.20);

\path[draw=drawColor,line width= 0.6pt,line join=round] (109.77, 32.48) -- (109.77,262.20);

\path[draw=drawColor,line width= 0.6pt,line join=round] (109.77, 32.48) -- (109.77,262.20);

\path[draw=drawColor,line width= 0.6pt,line join=round] (109.77, 32.48) -- (109.77,262.20);

\path[draw=drawColor,line width= 0.6pt,line join=round] (109.77, 32.48) -- (109.77,262.20);

\path[draw=drawColor,line width= 0.6pt,line join=round] (109.77, 32.48) -- (109.77,262.20);

\path[draw=drawColor,line width= 0.6pt,line join=round] (109.77, 32.48) -- (109.77,262.20);

\path[draw=drawColor,line width= 0.6pt,line join=round] (109.77, 32.48) -- (109.77,262.20);

\path[draw=drawColor,line width= 0.6pt,line join=round] (109.77, 32.48) -- (109.77,262.20);

\path[draw=drawColor,line width= 0.6pt,line join=round] (109.77, 32.48) -- (109.77,262.20);

\path[draw=drawColor,line width= 0.6pt,line join=round] (109.77, 32.48) -- (109.77,262.20);

\path[draw=drawColor,line width= 0.6pt,line join=round] (109.77, 32.48) -- (109.77,262.20);

\path[draw=drawColor,line width= 0.6pt,line join=round] (109.77, 32.48) -- (109.77,262.20);

\path[draw=drawColor,line width= 0.6pt,line join=round] (109.77, 32.48) -- (109.77,262.20);

\path[draw=drawColor,line width= 0.6pt,line join=round] (109.77, 32.48) -- (109.77,262.20);

\path[draw=drawColor,line width= 0.6pt,line join=round] (109.77, 32.48) -- (109.77,262.20);

\path[draw=drawColor,line width= 0.6pt,line join=round] (109.77, 32.48) -- (109.77,262.20);

\path[draw=drawColor,line width= 0.6pt,line join=round] (109.77, 32.48) -- (109.77,262.20);

\path[draw=drawColor,line width= 0.6pt,line join=round] (109.77, 32.48) -- (109.77,262.20);

\path[draw=drawColor,line width= 0.6pt,line join=round] (109.77, 32.48) -- (109.77,262.20);

\path[draw=drawColor,line width= 0.6pt,line join=round] (109.77, 32.48) -- (109.77,262.20);

\path[draw=drawColor,line width= 0.6pt,line join=round] (109.77, 32.48) -- (109.77,262.20);

\path[draw=drawColor,line width= 0.6pt,line join=round] (109.77, 32.48) -- (109.77,262.20);

\path[draw=drawColor,line width= 0.6pt,line join=round] (109.77, 32.48) -- (109.77,262.20);

\path[draw=drawColor,line width= 0.6pt,line join=round] (109.77, 32.48) -- (109.77,262.20);

\path[draw=drawColor,line width= 0.6pt,line join=round] (109.77, 32.48) -- (109.77,262.20);

\path[draw=drawColor,line width= 0.6pt,line join=round] (109.77, 32.48) -- (109.77,262.20);

\path[draw=drawColor,line width= 0.6pt,line join=round] (109.77, 32.48) -- (109.77,262.20);

\path[draw=drawColor,line width= 0.6pt,line join=round] (109.77, 32.48) -- (109.77,262.20);

\path[draw=drawColor,line width= 0.6pt,line join=round] (109.77, 32.48) -- (109.77,262.20);

\path[draw=drawColor,line width= 0.6pt,line join=round] (109.77, 32.48) -- (109.77,262.20);

\path[draw=drawColor,line width= 0.6pt,line join=round] (109.77, 32.48) -- (109.77,262.20);

\path[draw=drawColor,line width= 0.6pt,line join=round] (109.77, 32.48) -- (109.77,262.20);

\path[draw=drawColor,line width= 0.6pt,line join=round] (109.77, 32.48) -- (109.77,262.20);

\path[draw=drawColor,line width= 0.6pt,line join=round] (109.77, 32.48) -- (109.77,262.20);

\path[draw=drawColor,line width= 0.6pt,line join=round] (109.77, 32.48) -- (109.77,262.20);

\path[draw=drawColor,line width= 0.6pt,line join=round] (109.77, 32.48) -- (109.77,262.20);

\path[draw=drawColor,line width= 0.6pt,line join=round] (109.77, 32.48) -- (109.77,262.20);

\path[draw=drawColor,line width= 0.6pt,line join=round] (109.77, 32.48) -- (109.77,262.20);

\path[draw=drawColor,line width= 0.6pt,line join=round] (109.77, 32.48) -- (109.77,262.20);

\path[draw=drawColor,line width= 0.6pt,line join=round] (109.77, 32.48) -- (109.77,262.20);

\path[draw=drawColor,line width= 0.6pt,line join=round] (109.77, 32.48) -- (109.77,262.20);

\path[draw=drawColor,line width= 0.6pt,line join=round] (109.77, 32.48) -- (109.77,262.20);

\path[draw=drawColor,line width= 0.6pt,line join=round] (109.77, 32.48) -- (109.77,262.20);

\path[draw=drawColor,line width= 0.6pt,line join=round] (109.77, 32.48) -- (109.77,262.20);

\path[draw=drawColor,line width= 0.6pt,line join=round] (109.77, 32.48) -- (109.77,262.20);

\path[draw=drawColor,line width= 0.6pt,line join=round] (109.77, 32.48) -- (109.77,262.20);

\path[draw=drawColor,line width= 0.6pt,line join=round] (109.77, 32.48) -- (109.77,262.20);

\path[draw=drawColor,line width= 0.6pt,line join=round] (109.77, 32.48) -- (109.77,262.20);

\path[draw=drawColor,line width= 0.6pt,line join=round] (109.77, 32.48) -- (109.77,262.20);

\path[draw=drawColor,line width= 0.6pt,line join=round] (109.77, 32.48) -- (109.77,262.20);

\path[draw=drawColor,line width= 0.6pt,line join=round] (109.77, 32.48) -- (109.77,262.20);

\path[draw=drawColor,line width= 0.6pt,line join=round] (109.77, 32.48) -- (109.77,262.20);

\path[draw=drawColor,line width= 0.6pt,line join=round] (109.77, 32.48) -- (109.77,262.20);

\path[draw=drawColor,line width= 0.6pt,line join=round] (109.77, 32.48) -- (109.77,262.20);

\path[draw=drawColor,line width= 0.6pt,line join=round] (109.77, 32.48) -- (109.77,262.20);

\path[draw=drawColor,line width= 0.6pt,line join=round] (109.77, 32.48) -- (109.77,262.20);

\path[draw=drawColor,line width= 0.6pt,line join=round] (109.77, 32.48) -- (109.77,262.20);

\path[draw=drawColor,line width= 0.6pt,line join=round] (109.77, 32.48) -- (109.77,262.20);

\path[draw=drawColor,line width= 0.6pt,line join=round] (109.77, 32.48) -- (109.77,262.20);

\path[draw=drawColor,line width= 0.6pt,line join=round] (109.77, 32.48) -- (109.77,262.20);

\path[draw=drawColor,line width= 0.6pt,line join=round] (109.77, 32.48) -- (109.77,262.20);

\path[draw=drawColor,line width= 0.6pt,line join=round] (109.77, 32.48) -- (109.77,262.20);

\path[draw=drawColor,line width= 0.6pt,line join=round] (109.77, 32.48) -- (109.77,262.20);

\path[draw=drawColor,line width= 0.6pt,line join=round] (109.77, 32.48) -- (109.77,262.20);

\path[draw=drawColor,line width= 0.6pt,line join=round] (109.77, 32.48) -- (109.77,262.20);

\path[draw=drawColor,line width= 0.6pt,line join=round] (109.77, 32.48) -- (109.77,262.20);

\path[draw=drawColor,line width= 0.6pt,line join=round] (109.77, 32.48) -- (109.77,262.20);

\path[draw=drawColor,line width= 0.6pt,line join=round] (109.77, 32.48) -- (109.77,262.20);

\path[draw=drawColor,line width= 0.6pt,line join=round] (109.77, 32.48) -- (109.77,262.20);

\path[draw=drawColor,line width= 0.6pt,line join=round] (109.77, 32.48) -- (109.77,262.20);

\path[draw=drawColor,line width= 0.6pt,line join=round] (109.77, 32.48) -- (109.77,262.20);

\path[draw=drawColor,line width= 0.6pt,line join=round] (109.77, 32.48) -- (109.77,262.20);

\path[draw=drawColor,line width= 0.6pt,line join=round] (109.77, 32.48) -- (109.77,262.20);

\path[draw=drawColor,line width= 0.6pt,line join=round] (109.77, 32.48) -- (109.77,262.20);

\path[draw=drawColor,line width= 0.6pt,line join=round] (109.77, 32.48) -- (109.77,262.20);

\path[draw=drawColor,line width= 0.6pt,line join=round] (109.77, 32.48) -- (109.77,262.20);

\path[draw=drawColor,line width= 0.6pt,line join=round] (109.77, 32.48) -- (109.77,262.20);

\path[draw=drawColor,line width= 0.6pt,line join=round] (109.77, 32.48) -- (109.77,262.20);

\path[draw=drawColor,line width= 0.6pt,line join=round] (109.77, 32.48) -- (109.77,262.20);

\path[draw=drawColor,line width= 0.6pt,line join=round] (109.77, 32.48) -- (109.77,262.20);

\path[draw=drawColor,line width= 0.6pt,line join=round] (109.77, 32.48) -- (109.77,262.20);

\path[draw=drawColor,line width= 0.6pt,line join=round] (109.77, 32.48) -- (109.77,262.20);

\path[draw=drawColor,line width= 0.6pt,line join=round] (109.77, 32.48) -- (109.77,262.20);

\path[draw=drawColor,line width= 0.6pt,line join=round] (109.77, 32.48) -- (109.77,262.20);

\path[draw=drawColor,line width= 0.6pt,line join=round] (109.77, 32.48) -- (109.77,262.20);

\path[draw=drawColor,line width= 0.6pt,line join=round] (109.77, 32.48) -- (109.77,262.20);

\path[draw=drawColor,line width= 0.6pt,line join=round] (109.77, 32.48) -- (109.77,262.20);

\path[draw=drawColor,line width= 0.6pt,line join=round] (109.77, 32.48) -- (109.77,262.20);

\path[draw=drawColor,line width= 0.6pt,line join=round] (109.77, 32.48) -- (109.77,262.20);

\path[draw=drawColor,line width= 0.6pt,line join=round] (109.77, 32.48) -- (109.77,262.20);

\path[draw=drawColor,line width= 0.6pt,line join=round] (109.77, 32.48) -- (109.77,262.20);

\path[draw=drawColor,line width= 0.6pt,line join=round] (109.77, 32.48) -- (109.77,262.20);

\path[draw=drawColor,line width= 0.6pt,line join=round] (109.77, 32.48) -- (109.77,262.20);

\path[draw=drawColor,line width= 0.6pt,line join=round] (109.77, 32.48) -- (109.77,262.20);

\path[draw=drawColor,line width= 0.6pt,line join=round] (109.77, 32.48) -- (109.77,262.20);

\path[draw=drawColor,line width= 0.6pt,line join=round] (109.77, 32.48) -- (109.77,262.20);

\path[draw=drawColor,line width= 0.6pt,line join=round] (109.77, 32.48) -- (109.77,262.20);

\path[draw=drawColor,line width= 0.6pt,line join=round] (109.77, 32.48) -- (109.77,262.20);

\path[draw=drawColor,line width= 0.6pt,line join=round] (109.77, 32.48) -- (109.77,262.20);

\path[draw=drawColor,line width= 0.6pt,line join=round] (109.77, 32.48) -- (109.77,262.20);

\path[draw=drawColor,line width= 0.6pt,line join=round] (109.77, 32.48) -- (109.77,262.20);

\path[draw=drawColor,line width= 0.6pt,line join=round] (109.77, 32.48) -- (109.77,262.20);

\path[draw=drawColor,line width= 0.6pt,line join=round] (109.77, 32.48) -- (109.77,262.20);

\path[draw=drawColor,line width= 0.6pt,line join=round] (109.77, 32.48) -- (109.77,262.20);

\path[draw=drawColor,line width= 0.6pt,line join=round] (109.77, 32.48) -- (109.77,262.20);

\path[draw=drawColor,line width= 0.6pt,line join=round] (109.77, 32.48) -- (109.77,262.20);

\path[draw=drawColor,line width= 0.6pt,line join=round] (109.77, 32.48) -- (109.77,262.20);

\path[draw=drawColor,line width= 0.6pt,line join=round] (109.77, 32.48) -- (109.77,262.20);

\path[draw=drawColor,line width= 0.6pt,line join=round] (109.77, 32.48) -- (109.77,262.20);

\path[draw=drawColor,line width= 0.6pt,line join=round] (109.77, 32.48) -- (109.77,262.20);

\path[draw=drawColor,line width= 0.6pt,line join=round] (109.77, 32.48) -- (109.77,262.20);

\path[draw=drawColor,line width= 0.6pt,line join=round] (109.77, 32.48) -- (109.77,262.20);

\path[draw=drawColor,line width= 0.6pt,line join=round] (109.77, 32.48) -- (109.77,262.20);

\path[draw=drawColor,line width= 0.6pt,line join=round] (109.77, 32.48) -- (109.77,262.20);

\path[draw=drawColor,line width= 0.6pt,line join=round] (109.77, 32.48) -- (109.77,262.20);

\path[draw=drawColor,line width= 0.6pt,line join=round] (109.77, 32.48) -- (109.77,262.20);

\path[draw=drawColor,line width= 0.6pt,line join=round] (109.77, 32.48) -- (109.77,262.20);

\path[draw=drawColor,line width= 0.6pt,line join=round] (109.77, 32.48) -- (109.77,262.20);

\path[draw=drawColor,line width= 0.6pt,line join=round] (109.77, 32.48) -- (109.77,262.20);

\path[draw=drawColor,line width= 0.6pt,line join=round] (109.77, 32.48) -- (109.77,262.20);

\path[draw=drawColor,line width= 0.6pt,line join=round] (109.77, 32.48) -- (109.77,262.20);

\path[draw=drawColor,line width= 0.6pt,line join=round] (109.77, 32.48) -- (109.77,262.20);

\path[draw=drawColor,line width= 0.6pt,line join=round] (109.77, 32.48) -- (109.77,262.20);

\path[draw=drawColor,line width= 0.6pt,line join=round] (109.77, 32.48) -- (109.77,262.20);

\path[draw=drawColor,line width= 0.6pt,line join=round] (109.77, 32.48) -- (109.77,262.20);

\path[draw=drawColor,line width= 0.6pt,line join=round] (109.77, 32.48) -- (109.77,262.20);

\path[draw=drawColor,line width= 0.6pt,line join=round] (109.77, 32.48) -- (109.77,262.20);

\path[draw=drawColor,line width= 0.6pt,line join=round] (109.77, 32.48) -- (109.77,262.20);

\path[draw=drawColor,line width= 0.6pt,line join=round] (109.77, 32.48) -- (109.77,262.20);

\path[draw=drawColor,line width= 0.6pt,line join=round] (109.77, 32.48) -- (109.77,262.20);

\path[draw=drawColor,line width= 0.6pt,line join=round] (109.77, 32.48) -- (109.77,262.20);

\path[draw=drawColor,line width= 0.6pt,line join=round] (109.77, 32.48) -- (109.77,262.20);

\path[draw=drawColor,line width= 0.6pt,line join=round] (109.77, 32.48) -- (109.77,262.20);

\path[draw=drawColor,line width= 0.6pt,line join=round] (109.77, 32.48) -- (109.77,262.20);

\path[draw=drawColor,line width= 0.6pt,line join=round] (109.77, 32.48) -- (109.77,262.20);

\path[draw=drawColor,line width= 0.6pt,line join=round] (109.77, 32.48) -- (109.77,262.20);

\path[draw=drawColor,line width= 0.6pt,line join=round] (109.77, 32.48) -- (109.77,262.20);

\path[draw=drawColor,line width= 0.6pt,line join=round] (109.77, 32.48) -- (109.77,262.20);

\path[draw=drawColor,line width= 0.6pt,line join=round] (109.77, 32.48) -- (109.77,262.20);

\path[draw=drawColor,line width= 0.6pt,line join=round] (109.77, 32.48) -- (109.77,262.20);

\path[draw=drawColor,line width= 0.6pt,line join=round] (109.77, 32.48) -- (109.77,262.20);

\path[draw=drawColor,line width= 0.6pt,line join=round] (109.77, 32.48) -- (109.77,262.20);

\path[draw=drawColor,line width= 0.6pt,line join=round] (109.77, 32.48) -- (109.77,262.20);

\path[draw=drawColor,line width= 0.6pt,line join=round] (109.77, 32.48) -- (109.77,262.20);

\path[draw=drawColor,line width= 0.6pt,line join=round] (109.77, 32.48) -- (109.77,262.20);

\path[draw=drawColor,line width= 0.6pt,line join=round] (109.77, 32.48) -- (109.77,262.20);

\path[draw=drawColor,line width= 0.6pt,line join=round] (109.77, 32.48) -- (109.77,262.20);

\path[draw=drawColor,line width= 0.6pt,line join=round] (109.77, 32.48) -- (109.77,262.20);

\path[draw=drawColor,line width= 0.6pt,line join=round] (109.77, 32.48) -- (109.77,262.20);

\path[draw=drawColor,line width= 0.6pt,line join=round] (109.77, 32.48) -- (109.77,262.20);

\path[draw=drawColor,line width= 0.6pt,line join=round] (109.77, 32.48) -- (109.77,262.20);

\path[draw=drawColor,line width= 0.6pt,line join=round] (109.77, 32.48) -- (109.77,262.20);

\path[draw=drawColor,line width= 0.6pt,line join=round] (109.77, 32.48) -- (109.77,262.20);

\path[draw=drawColor,line width= 0.6pt,line join=round] (109.77, 32.48) -- (109.77,262.20);

\path[draw=drawColor,line width= 0.6pt,line join=round] (109.77, 32.48) -- (109.77,262.20);

\path[draw=drawColor,line width= 0.6pt,line join=round] (109.77, 32.48) -- (109.77,262.20);

\path[draw=drawColor,line width= 0.6pt,line join=round] (109.77, 32.48) -- (109.77,262.20);

\path[draw=drawColor,line width= 0.6pt,line join=round] (109.77, 32.48) -- (109.77,262.20);

\path[draw=drawColor,line width= 0.6pt,line join=round] (109.77, 32.48) -- (109.77,262.20);

\path[draw=drawColor,line width= 0.6pt,line join=round] (109.77, 32.48) -- (109.77,262.20);

\path[draw=drawColor,line width= 0.6pt,line join=round] (109.77, 32.48) -- (109.77,262.20);

\path[draw=drawColor,line width= 0.6pt,line join=round] (109.77, 32.48) -- (109.77,262.20);

\path[draw=drawColor,line width= 0.6pt,line join=round] (109.77, 32.48) -- (109.77,262.20);

\path[draw=drawColor,line width= 0.6pt,line join=round] (109.77, 32.48) -- (109.77,262.20);

\path[draw=drawColor,line width= 0.6pt,line join=round] (109.77, 32.48) -- (109.77,262.20);

\path[draw=drawColor,line width= 0.6pt,line join=round] (109.77, 32.48) -- (109.77,262.20);

\path[draw=drawColor,line width= 0.6pt,line join=round] (109.77, 32.48) -- (109.77,262.20);

\path[draw=drawColor,line width= 0.6pt,line join=round] (109.77, 32.48) -- (109.77,262.20);

\path[draw=drawColor,line width= 0.6pt,line join=round] (109.77, 32.48) -- (109.77,262.20);

\path[draw=drawColor,line width= 0.6pt,line join=round] (109.77, 32.48) -- (109.77,262.20);

\path[draw=drawColor,line width= 0.6pt,line join=round] (109.77, 32.48) -- (109.77,262.20);

\path[draw=drawColor,line width= 0.6pt,line join=round] (109.77, 32.48) -- (109.77,262.20);

\path[draw=drawColor,line width= 0.6pt,line join=round] (109.77, 32.48) -- (109.77,262.20);

\path[draw=drawColor,line width= 0.6pt,line join=round] (109.77, 32.48) -- (109.77,262.20);

\path[draw=drawColor,line width= 0.6pt,line join=round] (109.77, 32.48) -- (109.77,262.20);

\path[draw=drawColor,line width= 0.6pt,line join=round] (109.77, 32.48) -- (109.77,262.20);

\path[draw=drawColor,line width= 0.6pt,line join=round] (109.77, 32.48) -- (109.77,262.20);

\path[draw=drawColor,line width= 0.6pt,line join=round] (109.77, 32.48) -- (109.77,262.20);

\path[draw=drawColor,line width= 0.6pt,line join=round] (109.77, 32.48) -- (109.77,262.20);

\path[draw=drawColor,line width= 0.6pt,line join=round] (109.77, 32.48) -- (109.77,262.20);

\path[draw=drawColor,line width= 0.6pt,line join=round] (109.77, 32.48) -- (109.77,262.20);

\path[draw=drawColor,line width= 0.6pt,line join=round] (109.77, 32.48) -- (109.77,262.20);

\path[draw=drawColor,line width= 0.6pt,line join=round] (109.77, 32.48) -- (109.77,262.20);

\path[draw=drawColor,line width= 0.6pt,line join=round] (109.77, 32.48) -- (109.77,262.20);

\path[draw=drawColor,line width= 0.6pt,line join=round] (109.77, 32.48) -- (109.77,262.20);

\path[draw=drawColor,line width= 0.6pt,line join=round] (109.77, 32.48) -- (109.77,262.20);

\path[draw=drawColor,line width= 0.6pt,line join=round] (109.77, 32.48) -- (109.77,262.20);

\path[draw=drawColor,line width= 0.6pt,line join=round] (109.77, 32.48) -- (109.77,262.20);

\path[draw=drawColor,line width= 0.6pt,line join=round] (109.77, 32.48) -- (109.77,262.20);

\path[draw=drawColor,line width= 0.6pt,line join=round] (109.77, 32.48) -- (109.77,262.20);

\path[draw=drawColor,line width= 0.6pt,line join=round] (109.77, 32.48) -- (109.77,262.20);

\path[draw=drawColor,line width= 0.6pt,line join=round] (109.77, 32.48) -- (109.77,262.20);

\path[draw=drawColor,line width= 0.6pt,line join=round] (109.77, 32.48) -- (109.77,262.20);

\path[draw=drawColor,line width= 0.6pt,line join=round] (109.77, 32.48) -- (109.77,262.20);

\path[draw=drawColor,line width= 0.6pt,line join=round] (109.77, 32.48) -- (109.77,262.20);

\path[draw=drawColor,line width= 0.6pt,line join=round] (109.77, 32.48) -- (109.77,262.20);

\path[draw=drawColor,line width= 0.6pt,line join=round] (109.77, 32.48) -- (109.77,262.20);

\path[draw=drawColor,line width= 0.6pt,line join=round] (109.77, 32.48) -- (109.77,262.20);

\path[draw=drawColor,line width= 0.6pt,line join=round] (109.77, 32.48) -- (109.77,262.20);

\path[draw=drawColor,line width= 0.6pt,line join=round] (109.77, 32.48) -- (109.77,262.20);

\path[draw=drawColor,line width= 0.6pt,line join=round] (109.77, 32.48) -- (109.77,262.20);

\path[draw=drawColor,line width= 0.6pt,line join=round] (109.77, 32.48) -- (109.77,262.20);

\path[draw=drawColor,line width= 0.6pt,line join=round] (109.77, 32.48) -- (109.77,262.20);

\path[draw=drawColor,line width= 0.6pt,line join=round] (109.77, 32.48) -- (109.77,262.20);

\path[draw=drawColor,line width= 0.6pt,line join=round] (109.77, 32.48) -- (109.77,262.20);

\path[draw=drawColor,line width= 0.6pt,line join=round] (109.77, 32.48) -- (109.77,262.20);

\path[draw=drawColor,line width= 0.6pt,line join=round] (109.77, 32.48) -- (109.77,262.20);

\path[draw=drawColor,line width= 0.6pt,line join=round] (109.77, 32.48) -- (109.77,262.20);

\path[draw=drawColor,line width= 0.6pt,line join=round] (109.77, 32.48) -- (109.77,262.20);

\path[draw=drawColor,line width= 0.6pt,line join=round] (109.77, 32.48) -- (109.77,262.20);

\path[draw=drawColor,line width= 0.6pt,line join=round] (109.77, 32.48) -- (109.77,262.20);

\path[draw=drawColor,line width= 0.6pt,line join=round] (109.77, 32.48) -- (109.77,262.20);

\path[draw=drawColor,line width= 0.6pt,line join=round] (109.77, 32.48) -- (109.77,262.20);

\path[draw=drawColor,line width= 0.6pt,line join=round] (109.77, 32.48) -- (109.77,262.20);

\path[draw=drawColor,line width= 0.6pt,line join=round] (109.77, 32.48) -- (109.77,262.20);

\path[draw=drawColor,line width= 0.6pt,line join=round] (109.77, 32.48) -- (109.77,262.20);

\path[draw=drawColor,line width= 0.6pt,line join=round] (109.77, 32.48) -- (109.77,262.20);

\path[draw=drawColor,line width= 0.6pt,line join=round] (109.77, 32.48) -- (109.77,262.20);

\path[draw=drawColor,line width= 0.6pt,line join=round] (109.77, 32.48) -- (109.77,262.20);

\path[draw=drawColor,line width= 0.6pt,line join=round] (109.77, 32.48) -- (109.77,262.20);

\path[draw=drawColor,line width= 0.6pt,line join=round] (109.77, 32.48) -- (109.77,262.20);

\path[draw=drawColor,line width= 0.6pt,line join=round] (109.77, 32.48) -- (109.77,262.20);

\path[draw=drawColor,line width= 0.6pt,line join=round] (109.77, 32.48) -- (109.77,262.20);

\path[draw=drawColor,line width= 0.6pt,line join=round] (109.77, 32.48) -- (109.77,262.20);

\path[draw=drawColor,line width= 0.6pt,line join=round] (109.77, 32.48) -- (109.77,262.20);

\path[draw=drawColor,line width= 0.6pt,line join=round] (109.77, 32.48) -- (109.77,262.20);

\path[draw=drawColor,line width= 0.6pt,line join=round] (109.77, 32.48) -- (109.77,262.20);

\path[draw=drawColor,line width= 0.6pt,line join=round] (109.77, 32.48) -- (109.77,262.20);

\path[draw=drawColor,line width= 0.6pt,line join=round] (109.77, 32.48) -- (109.77,262.20);

\path[draw=drawColor,line width= 0.6pt,line join=round] (109.77, 32.48) -- (109.77,262.20);

\path[draw=drawColor,line width= 0.6pt,line join=round] (109.77, 32.48) -- (109.77,262.20);

\path[draw=drawColor,line width= 0.6pt,line join=round] (109.77, 32.48) -- (109.77,262.20);

\path[draw=drawColor,line width= 0.6pt,line join=round] (109.77, 32.48) -- (109.77,262.20);

\path[draw=drawColor,line width= 0.6pt,line join=round] (109.77, 32.48) -- (109.77,262.20);

\path[draw=drawColor,line width= 0.6pt,line join=round] (109.77, 32.48) -- (109.77,262.20);

\path[draw=drawColor,line width= 0.6pt,line join=round] (109.77, 32.48) -- (109.77,262.20);

\path[draw=drawColor,line width= 0.6pt,line join=round] (109.77, 32.48) -- (109.77,262.20);

\path[draw=drawColor,line width= 0.6pt,line join=round] (109.77, 32.48) -- (109.77,262.20);

\path[draw=drawColor,line width= 0.6pt,line join=round] (109.77, 32.48) -- (109.77,262.20);

\path[draw=drawColor,line width= 0.6pt,line join=round] (109.77, 32.48) -- (109.77,262.20);

\path[draw=drawColor,line width= 0.6pt,line join=round] (109.77, 32.48) -- (109.77,262.20);

\path[draw=drawColor,line width= 0.6pt,line join=round] (109.77, 32.48) -- (109.77,262.20);

\path[draw=drawColor,line width= 0.6pt,line join=round] (109.77, 32.48) -- (109.77,262.20);

\path[draw=drawColor,line width= 0.6pt,line join=round] (109.77, 32.48) -- (109.77,262.20);

\path[draw=drawColor,line width= 0.6pt,line join=round] (109.77, 32.48) -- (109.77,262.20);

\path[draw=drawColor,line width= 0.6pt,line join=round] (109.77, 32.48) -- (109.77,262.20);

\path[draw=drawColor,line width= 0.6pt,line join=round] (109.77, 32.48) -- (109.77,262.20);

\path[draw=drawColor,line width= 0.6pt,line join=round] (109.77, 32.48) -- (109.77,262.20);

\path[draw=drawColor,line width= 0.6pt,line join=round] (109.77, 32.48) -- (109.77,262.20);

\path[draw=drawColor,line width= 0.6pt,line join=round] (109.77, 32.48) -- (109.77,262.20);

\path[draw=drawColor,line width= 0.6pt,line join=round] (109.77, 32.48) -- (109.77,262.20);

\path[draw=drawColor,line width= 0.6pt,line join=round] (109.77, 32.48) -- (109.77,262.20);

\path[draw=drawColor,line width= 0.6pt,line join=round] (109.77, 32.48) -- (109.77,262.20);

\path[draw=drawColor,line width= 0.6pt,line join=round] (109.77, 32.48) -- (109.77,262.20);

\path[draw=drawColor,line width= 0.6pt,line join=round] (109.77, 32.48) -- (109.77,262.20);

\path[draw=drawColor,line width= 0.6pt,line join=round] (109.77, 32.48) -- (109.77,262.20);

\path[draw=drawColor,line width= 0.6pt,line join=round] (109.77, 32.48) -- (109.77,262.20);

\path[draw=drawColor,line width= 0.6pt,line join=round] (109.77, 32.48) -- (109.77,262.20);

\path[draw=drawColor,line width= 0.6pt,line join=round] (109.77, 32.48) -- (109.77,262.20);

\path[draw=drawColor,line width= 0.6pt,line join=round] (109.77, 32.48) -- (109.77,262.20);

\path[draw=drawColor,line width= 0.6pt,line join=round] (109.77, 32.48) -- (109.77,262.20);

\path[draw=drawColor,line width= 0.6pt,line join=round] (109.77, 32.48) -- (109.77,262.20);

\path[draw=drawColor,line width= 0.6pt,line join=round] (109.77, 32.48) -- (109.77,262.20);

\path[draw=drawColor,line width= 0.6pt,line join=round] (109.77, 32.48) -- (109.77,262.20);

\path[draw=drawColor,line width= 0.6pt,line join=round] (109.77, 32.48) -- (109.77,262.20);

\path[draw=drawColor,line width= 0.6pt,line join=round] (109.77, 32.48) -- (109.77,262.20);

\path[draw=drawColor,line width= 0.6pt,line join=round] (109.77, 32.48) -- (109.77,262.20);

\path[draw=drawColor,line width= 0.6pt,line join=round] (109.77, 32.48) -- (109.77,262.20);

\path[draw=drawColor,line width= 0.6pt,line join=round] (109.77, 32.48) -- (109.77,262.20);

\path[draw=drawColor,line width= 0.6pt,line join=round] (109.77, 32.48) -- (109.77,262.20);

\path[draw=drawColor,line width= 0.6pt,line join=round] (109.77, 32.48) -- (109.77,262.20);

\path[draw=drawColor,line width= 0.6pt,line join=round] (109.77, 32.48) -- (109.77,262.20);

\path[draw=drawColor,line width= 0.6pt,line join=round] (109.77, 32.48) -- (109.77,262.20);

\path[draw=drawColor,line width= 0.6pt,line join=round] (109.77, 32.48) -- (109.77,262.20);

\path[draw=drawColor,line width= 0.6pt,line join=round] (109.77, 32.48) -- (109.77,262.20);

\path[draw=drawColor,line width= 0.6pt,line join=round] (109.77, 32.48) -- (109.77,262.20);

\path[draw=drawColor,line width= 0.6pt,line join=round] (109.77, 32.48) -- (109.77,262.20);

\path[draw=drawColor,line width= 0.6pt,line join=round] (109.77, 32.48) -- (109.77,262.20);

\path[draw=drawColor,line width= 0.6pt,line join=round] (109.77, 32.48) -- (109.77,262.20);

\path[draw=drawColor,line width= 0.6pt,line join=round] (109.77, 32.48) -- (109.77,262.20);

\path[draw=drawColor,line width= 0.6pt,line join=round] (109.77, 32.48) -- (109.77,262.20);

\path[draw=drawColor,line width= 0.6pt,line join=round] (109.77, 32.48) -- (109.77,262.20);

\path[draw=drawColor,line width= 0.6pt,line join=round] (109.77, 32.48) -- (109.77,262.20);

\path[draw=drawColor,line width= 0.6pt,line join=round] (109.77, 32.48) -- (109.77,262.20);

\path[draw=drawColor,line width= 0.6pt,line join=round] (109.77, 32.48) -- (109.77,262.20);

\path[draw=drawColor,line width= 0.6pt,line join=round] (109.77, 32.48) -- (109.77,262.20);

\path[draw=drawColor,line width= 0.6pt,line join=round] (109.77, 32.48) -- (109.77,262.20);

\path[draw=drawColor,line width= 0.6pt,line join=round] (109.77, 32.48) -- (109.77,262.20);

\path[draw=drawColor,line width= 0.6pt,line join=round] (109.77, 32.48) -- (109.77,262.20);

\path[draw=drawColor,line width= 0.6pt,line join=round] (109.77, 32.48) -- (109.77,262.20);

\path[draw=drawColor,line width= 0.6pt,line join=round] (109.77, 32.48) -- (109.77,262.20);

\path[draw=drawColor,line width= 0.6pt,line join=round] (109.77, 32.48) -- (109.77,262.20);

\path[draw=drawColor,line width= 0.6pt,line join=round] (109.77, 32.48) -- (109.77,262.20);

\path[draw=drawColor,line width= 0.6pt,line join=round] (109.77, 32.48) -- (109.77,262.20);

\path[draw=drawColor,line width= 0.6pt,line join=round] (109.77, 32.48) -- (109.77,262.20);

\path[draw=drawColor,line width= 0.6pt,line join=round] (109.77, 32.48) -- (109.77,262.20);

\path[draw=drawColor,line width= 0.6pt,line join=round] (109.77, 32.48) -- (109.77,262.20);

\path[draw=drawColor,line width= 0.6pt,line join=round] (109.77, 32.48) -- (109.77,262.20);

\path[draw=drawColor,line width= 0.6pt,line join=round] (109.77, 32.48) -- (109.77,262.20);

\path[draw=drawColor,line width= 0.6pt,line join=round] (109.77, 32.48) -- (109.77,262.20);

\path[draw=drawColor,line width= 0.6pt,line join=round] (109.77, 32.48) -- (109.77,262.20);

\path[draw=drawColor,line width= 0.6pt,line join=round] (109.77, 32.48) -- (109.77,262.20);

\path[draw=drawColor,line width= 0.6pt,line join=round] (109.77, 32.48) -- (109.77,262.20);

\path[draw=drawColor,line width= 0.6pt,line join=round] (109.77, 32.48) -- (109.77,262.20);

\path[draw=drawColor,line width= 0.6pt,line join=round] (109.77, 32.48) -- (109.77,262.20);

\path[draw=drawColor,line width= 0.6pt,line join=round] (109.77, 32.48) -- (109.77,262.20);

\path[draw=drawColor,line width= 0.6pt,line join=round] (109.77, 32.48) -- (109.77,262.20);

\path[draw=drawColor,line width= 0.6pt,line join=round] (109.77, 32.48) -- (109.77,262.20);

\path[draw=drawColor,line width= 0.6pt,line join=round] (109.77, 32.48) -- (109.77,262.20);

\path[draw=drawColor,line width= 0.6pt,line join=round] (109.77, 32.48) -- (109.77,262.20);

\path[draw=drawColor,line width= 0.6pt,line join=round] (109.77, 32.48) -- (109.77,262.20);

\path[draw=drawColor,line width= 0.6pt,line join=round] (109.77, 32.48) -- (109.77,262.20);

\path[draw=drawColor,line width= 0.6pt,line join=round] (109.77, 32.48) -- (109.77,262.20);

\path[draw=drawColor,line width= 0.6pt,line join=round] (109.77, 32.48) -- (109.77,262.20);

\path[draw=drawColor,line width= 0.6pt,line join=round] (109.77, 32.48) -- (109.77,262.20);

\path[draw=drawColor,line width= 0.6pt,line join=round] (109.77, 32.48) -- (109.77,262.20);

\path[draw=drawColor,line width= 0.6pt,line join=round] (109.77, 32.48) -- (109.77,262.20);

\path[draw=drawColor,line width= 0.6pt,line join=round] (109.77, 32.48) -- (109.77,262.20);

\path[draw=drawColor,line width= 0.6pt,line join=round] (109.77, 32.48) -- (109.77,262.20);

\path[draw=drawColor,line width= 0.6pt,line join=round] (109.77, 32.48) -- (109.77,262.20);

\path[draw=drawColor,line width= 0.6pt,line join=round] (109.77, 32.48) -- (109.77,262.20);

\path[draw=drawColor,line width= 0.6pt,line join=round] (109.77, 32.48) -- (109.77,262.20);

\path[draw=drawColor,line width= 0.6pt,line join=round] (109.77, 32.48) -- (109.77,262.20);

\path[draw=drawColor,line width= 0.6pt,line join=round] (109.77, 32.48) -- (109.77,262.20);

\path[draw=drawColor,line width= 0.6pt,line join=round] (109.77, 32.48) -- (109.77,262.20);

\path[draw=drawColor,line width= 0.6pt,line join=round] (109.77, 32.48) -- (109.77,262.20);

\path[draw=drawColor,line width= 0.6pt,line join=round] (109.77, 32.48) -- (109.77,262.20);

\path[draw=drawColor,line width= 0.6pt,line join=round] (109.77, 32.48) -- (109.77,262.20);

\path[draw=drawColor,line width= 0.6pt,line join=round] (109.77, 32.48) -- (109.77,262.20);

\path[draw=drawColor,line width= 0.6pt,line join=round] (109.77, 32.48) -- (109.77,262.20);

\path[draw=drawColor,line width= 0.6pt,line join=round] (109.77, 32.48) -- (109.77,262.20);

\path[draw=drawColor,line width= 0.6pt,line join=round] (109.77, 32.48) -- (109.77,262.20);

\path[draw=drawColor,line width= 0.6pt,line join=round] (109.77, 32.48) -- (109.77,262.20);

\path[draw=drawColor,line width= 0.6pt,line join=round] (109.77, 32.48) -- (109.77,262.20);

\path[draw=drawColor,line width= 0.6pt,line join=round] (109.77, 32.48) -- (109.77,262.20);

\path[draw=drawColor,line width= 0.6pt,line join=round] (109.77, 32.48) -- (109.77,262.20);

\path[draw=drawColor,line width= 0.6pt,line join=round] (109.77, 32.48) -- (109.77,262.20);

\path[draw=drawColor,line width= 0.6pt,line join=round] (109.77, 32.48) -- (109.77,262.20);

\path[draw=drawColor,line width= 0.6pt,line join=round] (109.77, 32.48) -- (109.77,262.20);

\path[draw=drawColor,line width= 0.6pt,line join=round] (109.77, 32.48) -- (109.77,262.20);

\path[draw=drawColor,line width= 0.6pt,line join=round] (109.77, 32.48) -- (109.77,262.20);

\path[draw=drawColor,line width= 0.6pt,line join=round] (109.77, 32.48) -- (109.77,262.20);

\path[draw=drawColor,line width= 0.6pt,line join=round] (109.77, 32.48) -- (109.77,262.20);

\path[draw=drawColor,line width= 0.6pt,line join=round] (109.77, 32.48) -- (109.77,262.20);

\path[draw=drawColor,line width= 0.6pt,line join=round] (109.77, 32.48) -- (109.77,262.20);

\path[draw=drawColor,line width= 0.6pt,line join=round] (109.77, 32.48) -- (109.77,262.20);

\path[draw=drawColor,line width= 0.6pt,line join=round] (109.77, 32.48) -- (109.77,262.20);

\path[draw=drawColor,line width= 0.6pt,line join=round] (109.77, 32.48) -- (109.77,262.20);

\path[draw=drawColor,line width= 0.6pt,line join=round] (109.77, 32.48) -- (109.77,262.20);

\path[draw=drawColor,line width= 0.6pt,line join=round] (109.77, 32.48) -- (109.77,262.20);

\path[draw=drawColor,line width= 0.6pt,line join=round] (109.77, 32.48) -- (109.77,262.20);

\path[draw=drawColor,line width= 0.6pt,line join=round] (109.77, 32.48) -- (109.77,262.20);

\path[draw=drawColor,line width= 0.6pt,line join=round] (109.77, 32.48) -- (109.77,262.20);

\path[draw=drawColor,line width= 0.6pt,line join=round] (109.77, 32.48) -- (109.77,262.20);

\path[draw=drawColor,line width= 0.6pt,line join=round] (109.77, 32.48) -- (109.77,262.20);

\path[draw=drawColor,line width= 0.6pt,line join=round] (109.77, 32.48) -- (109.77,262.20);

\path[draw=drawColor,line width= 0.6pt,line join=round] (109.77, 32.48) -- (109.77,262.20);

\path[draw=drawColor,line width= 0.6pt,line join=round] (109.77, 32.48) -- (109.77,262.20);

\path[draw=drawColor,line width= 0.6pt,line join=round] (109.77, 32.48) -- (109.77,262.20);

\path[draw=drawColor,line width= 0.6pt,line join=round] (109.77, 32.48) -- (109.77,262.20);

\path[draw=drawColor,line width= 0.6pt,line join=round] (109.77, 32.48) -- (109.77,262.20);

\path[draw=drawColor,line width= 0.6pt,line join=round] (109.77, 32.48) -- (109.77,262.20);

\path[draw=drawColor,line width= 0.6pt,line join=round] (109.77, 32.48) -- (109.77,262.20);

\path[draw=drawColor,line width= 0.6pt,line join=round] (109.77, 32.48) -- (109.77,262.20);

\path[draw=drawColor,line width= 0.6pt,line join=round] (109.77, 32.48) -- (109.77,262.20);

\path[draw=drawColor,line width= 0.6pt,line join=round] (109.77, 32.48) -- (109.77,262.20);

\path[draw=drawColor,line width= 0.6pt,line join=round] (109.77, 32.48) -- (109.77,262.20);

\path[draw=drawColor,line width= 0.6pt,line join=round] (109.77, 32.48) -- (109.77,262.20);

\path[draw=drawColor,line width= 0.6pt,line join=round] (109.77, 32.48) -- (109.77,262.20);

\path[draw=drawColor,line width= 0.6pt,line join=round] (109.77, 32.48) -- (109.77,262.20);

\path[draw=drawColor,line width= 0.6pt,line join=round] (109.77, 32.48) -- (109.77,262.20);

\path[draw=drawColor,line width= 0.6pt,line join=round] (109.77, 32.48) -- (109.77,262.20);

\path[draw=drawColor,line width= 0.6pt,line join=round] (109.77, 32.48) -- (109.77,262.20);

\path[draw=drawColor,line width= 0.6pt,line join=round] (109.77, 32.48) -- (109.77,262.20);

\path[draw=drawColor,line width= 0.6pt,line join=round] (109.77, 32.48) -- (109.77,262.20);

\path[draw=drawColor,line width= 0.6pt,line join=round] (109.77, 32.48) -- (109.77,262.20);

\path[draw=drawColor,line width= 0.6pt,line join=round] (109.77, 32.48) -- (109.77,262.20);

\path[draw=drawColor,line width= 0.6pt,line join=round] (109.77, 32.48) -- (109.77,262.20);

\path[draw=drawColor,line width= 0.6pt,line join=round] (109.77, 32.48) -- (109.77,262.20);

\path[draw=drawColor,line width= 0.6pt,line join=round] (109.77, 32.48) -- (109.77,262.20);

\path[draw=drawColor,line width= 0.6pt,line join=round] (109.77, 32.48) -- (109.77,262.20);

\path[draw=drawColor,line width= 0.6pt,line join=round] (109.77, 32.48) -- (109.77,262.20);

\path[draw=drawColor,line width= 0.6pt,line join=round] (109.77, 32.48) -- (109.77,262.20);

\path[draw=drawColor,line width= 0.6pt,line join=round] (109.77, 32.48) -- (109.77,262.20);

\path[draw=drawColor,line width= 0.6pt,line join=round] (109.77, 32.48) -- (109.77,262.20);

\path[draw=drawColor,line width= 0.6pt,line join=round] (109.77, 32.48) -- (109.77,262.20);

\path[draw=drawColor,line width= 0.6pt,line join=round] (109.77, 32.48) -- (109.77,262.20);

\path[draw=drawColor,line width= 0.6pt,line join=round] (109.77, 32.48) -- (109.77,262.20);

\path[draw=drawColor,line width= 0.6pt,line join=round] (109.77, 32.48) -- (109.77,262.20);

\path[draw=drawColor,line width= 0.6pt,line join=round] (109.77, 32.48) -- (109.77,262.20);

\path[draw=drawColor,line width= 0.6pt,line join=round] (109.77, 32.48) -- (109.77,262.20);

\path[draw=drawColor,line width= 0.6pt,line join=round] (109.77, 32.48) -- (109.77,262.20);

\path[draw=drawColor,line width= 0.6pt,line join=round] (109.77, 32.48) -- (109.77,262.20);

\path[draw=drawColor,line width= 0.6pt,line join=round] (109.77, 32.48) -- (109.77,262.20);

\path[draw=drawColor,line width= 0.6pt,line join=round] (109.77, 32.48) -- (109.77,262.20);

\path[draw=drawColor,line width= 0.6pt,line join=round] (109.77, 32.48) -- (109.77,262.20);

\path[draw=drawColor,line width= 0.6pt,line join=round] (109.77, 32.48) -- (109.77,262.20);

\path[draw=drawColor,line width= 0.6pt,line join=round] (109.77, 32.48) -- (109.77,262.20);

\path[draw=drawColor,line width= 0.6pt,line join=round] (109.77, 32.48) -- (109.77,262.20);

\path[draw=drawColor,line width= 0.6pt,line join=round] (109.77, 32.48) -- (109.77,262.20);

\path[draw=drawColor,line width= 0.6pt,line join=round] (109.77, 32.48) -- (109.77,262.20);

\path[draw=drawColor,line width= 0.6pt,line join=round] (109.77, 32.48) -- (109.77,262.20);

\path[draw=drawColor,line width= 0.6pt,line join=round] (109.77, 32.48) -- (109.77,262.20);

\path[draw=drawColor,line width= 0.6pt,line join=round] (109.77, 32.48) -- (109.77,262.20);

\path[draw=drawColor,line width= 0.6pt,line join=round] (109.77, 32.48) -- (109.77,262.20);

\path[draw=drawColor,line width= 0.6pt,line join=round] (109.77, 32.48) -- (109.77,262.20);

\path[draw=drawColor,line width= 0.6pt,line join=round] (109.77, 32.48) -- (109.77,262.20);

\path[draw=drawColor,line width= 0.6pt,line join=round] (109.77, 32.48) -- (109.77,262.20);

\path[draw=drawColor,line width= 0.6pt,line join=round] (109.77, 32.48) -- (109.77,262.20);

\path[draw=drawColor,line width= 0.6pt,line join=round] (109.77, 32.48) -- (109.77,262.20);

\path[draw=drawColor,line width= 0.6pt,line join=round] (109.77, 32.48) -- (109.77,262.20);

\path[draw=drawColor,line width= 0.6pt,line join=round] (109.77, 32.48) -- (109.77,262.20);

\path[draw=drawColor,line width= 0.6pt,line join=round] (109.77, 32.48) -- (109.77,262.20);

\path[draw=drawColor,line width= 0.6pt,line join=round] (109.77, 32.48) -- (109.77,262.20);

\path[draw=drawColor,line width= 0.6pt,line join=round] (109.77, 32.48) -- (109.77,262.20);

\path[draw=drawColor,line width= 0.6pt,line join=round] (109.77, 32.48) -- (109.77,262.20);

\path[draw=drawColor,line width= 0.6pt,line join=round] (109.77, 32.48) -- (109.77,262.20);

\path[draw=drawColor,line width= 0.6pt,line join=round] (109.77, 32.48) -- (109.77,262.20);

\path[draw=drawColor,line width= 0.6pt,line join=round] (109.77, 32.48) -- (109.77,262.20);

\path[draw=drawColor,line width= 0.6pt,line join=round] (109.77, 32.48) -- (109.77,262.20);

\path[draw=drawColor,line width= 0.6pt,line join=round] (109.77, 32.48) -- (109.77,262.20);

\path[draw=drawColor,line width= 0.6pt,line join=round] (109.77, 32.48) -- (109.77,262.20);

\path[draw=drawColor,line width= 0.6pt,line join=round] (109.77, 32.48) -- (109.77,262.20);

\path[draw=drawColor,line width= 0.6pt,line join=round] (109.77, 32.48) -- (109.77,262.20);

\path[draw=drawColor,line width= 0.6pt,line join=round] (109.77, 32.48) -- (109.77,262.20);

\path[draw=drawColor,line width= 0.6pt,line join=round] (109.77, 32.48) -- (109.77,262.20);

\path[draw=drawColor,line width= 0.6pt,line join=round] (109.77, 32.48) -- (109.77,262.20);

\path[draw=drawColor,line width= 0.6pt,line join=round] (109.77, 32.48) -- (109.77,262.20);

\path[draw=drawColor,line width= 0.6pt,line join=round] (109.77, 32.48) -- (109.77,262.20);

\path[draw=drawColor,line width= 0.6pt,line join=round] (109.77, 32.48) -- (109.77,262.20);

\path[draw=drawColor,line width= 0.6pt,line join=round] (109.77, 32.48) -- (109.77,262.20);

\path[draw=drawColor,line width= 0.6pt,line join=round] (109.77, 32.48) -- (109.77,262.20);

\path[draw=drawColor,line width= 0.6pt,line join=round] (109.77, 32.48) -- (109.77,262.20);

\path[draw=drawColor,line width= 0.6pt,line join=round] (109.77, 32.48) -- (109.77,262.20);

\path[draw=drawColor,line width= 0.6pt,line join=round] (109.77, 32.48) -- (109.77,262.20);

\path[draw=drawColor,line width= 0.6pt,line join=round] (109.77, 32.48) -- (109.77,262.20);

\path[draw=drawColor,line width= 0.6pt,line join=round] (109.77, 32.48) -- (109.77,262.20);

\path[draw=drawColor,line width= 0.6pt,line join=round] (109.77, 32.48) -- (109.77,262.20);

\path[draw=drawColor,line width= 0.6pt,line join=round] (109.77, 32.48) -- (109.77,262.20);

\path[draw=drawColor,line width= 0.6pt,line join=round] (109.77, 32.48) -- (109.77,262.20);

\path[draw=drawColor,line width= 0.6pt,line join=round] (109.77, 32.48) -- (109.77,262.20);

\path[draw=drawColor,line width= 0.6pt,line join=round] (109.77, 32.48) -- (109.77,262.20);

\path[draw=drawColor,line width= 0.6pt,line join=round] (109.77, 32.48) -- (109.77,262.20);

\path[draw=drawColor,line width= 0.6pt,line join=round] (109.77, 32.48) -- (109.77,262.20);

\path[draw=drawColor,line width= 0.6pt,line join=round] (109.77, 32.48) -- (109.77,262.20);

\path[draw=drawColor,line width= 0.6pt,line join=round] (109.77, 32.48) -- (109.77,262.20);

\path[draw=drawColor,line width= 0.6pt,line join=round] (109.77, 32.48) -- (109.77,262.20);

\path[draw=drawColor,line width= 0.6pt,line join=round] (109.77, 32.48) -- (109.77,262.20);

\path[draw=drawColor,line width= 0.6pt,line join=round] (109.77, 32.48) -- (109.77,262.20);

\path[draw=drawColor,line width= 0.6pt,line join=round] (109.77, 32.48) -- (109.77,262.20);

\path[draw=drawColor,line width= 0.6pt,line join=round] (109.77, 32.48) -- (109.77,262.20);

\path[draw=drawColor,line width= 0.6pt,line join=round] (109.77, 32.48) -- (109.77,262.20);

\path[draw=drawColor,line width= 0.6pt,line join=round] (109.77, 32.48) -- (109.77,262.20);

\path[draw=drawColor,line width= 0.6pt,line join=round] (109.77, 32.48) -- (109.77,262.20);

\path[draw=drawColor,line width= 0.6pt,line join=round] (109.77, 32.48) -- (109.77,262.20);

\path[draw=drawColor,line width= 0.6pt,line join=round] (109.77, 32.48) -- (109.77,262.20);

\path[draw=drawColor,line width= 0.6pt,line join=round] (109.77, 32.48) -- (109.77,262.20);

\path[draw=drawColor,line width= 0.6pt,line join=round] (109.77, 32.48) -- (109.77,262.20);

\path[draw=drawColor,line width= 0.6pt,line join=round] (109.77, 32.48) -- (109.77,262.20);

\path[draw=drawColor,line width= 0.6pt,line join=round] (109.77, 32.48) -- (109.77,262.20);

\path[draw=drawColor,line width= 0.6pt,line join=round] (109.77, 32.48) -- (109.77,262.20);

\path[draw=drawColor,line width= 0.6pt,line join=round] (109.77, 32.48) -- (109.77,262.20);

\path[draw=drawColor,line width= 0.6pt,line join=round] (109.77, 32.48) -- (109.77,262.20);

\path[draw=drawColor,line width= 0.6pt,line join=round] (109.77, 32.48) -- (109.77,262.20);

\path[draw=drawColor,line width= 0.6pt,line join=round] (109.77, 32.48) -- (109.77,262.20);

\path[draw=drawColor,line width= 0.6pt,line join=round] (109.77, 32.48) -- (109.77,262.20);

\path[draw=drawColor,line width= 0.6pt,line join=round] (109.77, 32.48) -- (109.77,262.20);

\path[draw=drawColor,line width= 0.6pt,line join=round] (109.77, 32.48) -- (109.77,262.20);

\path[draw=drawColor,line width= 0.6pt,line join=round] (109.77, 32.48) -- (109.77,262.20);

\path[draw=drawColor,line width= 0.6pt,line join=round] (109.77, 32.48) -- (109.77,262.20);

\path[draw=drawColor,line width= 0.6pt,line join=round] (109.77, 32.48) -- (109.77,262.20);

\path[draw=drawColor,line width= 0.6pt,line join=round] (109.77, 32.48) -- (109.77,262.20);

\path[draw=drawColor,line width= 0.6pt,line join=round] (109.77, 32.48) -- (109.77,262.20);

\path[draw=drawColor,line width= 0.6pt,line join=round] (109.77, 32.48) -- (109.77,262.20);

\path[draw=drawColor,line width= 0.6pt,line join=round] (109.77, 32.48) -- (109.77,262.20);

\path[draw=drawColor,line width= 0.6pt,line join=round] (109.77, 32.48) -- (109.77,262.20);

\path[draw=drawColor,line width= 0.6pt,line join=round] (109.77, 32.48) -- (109.77,262.20);

\path[draw=drawColor,line width= 0.6pt,line join=round] (109.77, 32.48) -- (109.77,262.20);

\path[draw=drawColor,line width= 0.6pt,line join=round] (109.77, 32.48) -- (109.77,262.20);

\path[draw=drawColor,line width= 0.6pt,line join=round] (109.77, 32.48) -- (109.77,262.20);

\path[draw=drawColor,line width= 0.6pt,line join=round] (109.77, 32.48) -- (109.77,262.20);

\path[draw=drawColor,line width= 0.6pt,line join=round] (109.77, 32.48) -- (109.77,262.20);

\path[draw=drawColor,line width= 0.6pt,line join=round] (109.77, 32.48) -- (109.77,262.20);

\path[draw=drawColor,line width= 0.6pt,line join=round] (109.77, 32.48) -- (109.77,262.20);

\path[draw=drawColor,line width= 0.6pt,line join=round] (109.77, 32.48) -- (109.77,262.20);

\path[draw=drawColor,line width= 0.6pt,line join=round] (109.77, 32.48) -- (109.77,262.20);

\path[draw=drawColor,line width= 0.6pt,line join=round] (109.77, 32.48) -- (109.77,262.20);

\path[draw=drawColor,line width= 0.6pt,line join=round] (109.77, 32.48) -- (109.77,262.20);

\path[draw=drawColor,line width= 0.6pt,line join=round] (109.77, 32.48) -- (109.77,262.20);

\path[draw=drawColor,line width= 0.6pt,line join=round] (109.77, 32.48) -- (109.77,262.20);

\path[draw=drawColor,line width= 0.6pt,line join=round] (109.77, 32.48) -- (109.77,262.20);

\path[draw=drawColor,line width= 0.6pt,line join=round] (109.77, 32.48) -- (109.77,262.20);

\path[draw=drawColor,line width= 0.6pt,line join=round] (109.77, 32.48) -- (109.77,262.20);

\path[draw=drawColor,line width= 0.6pt,line join=round] (109.77, 32.48) -- (109.77,262.20);

\path[draw=drawColor,line width= 0.6pt,line join=round] (109.77, 32.48) -- (109.77,262.20);

\path[draw=drawColor,line width= 0.6pt,line join=round] (109.77, 32.48) -- (109.77,262.20);

\path[draw=drawColor,line width= 0.6pt,line join=round] (109.77, 32.48) -- (109.77,262.20);

\path[draw=drawColor,line width= 0.6pt,line join=round] (109.77, 32.48) -- (109.77,262.20);

\path[draw=drawColor,line width= 0.6pt,line join=round] (109.77, 32.48) -- (109.77,262.20);

\path[draw=drawColor,line width= 0.6pt,line join=round] (109.77, 32.48) -- (109.77,262.20);

\path[draw=drawColor,line width= 0.6pt,line join=round] (109.77, 32.48) -- (109.77,262.20);

\path[draw=drawColor,line width= 0.6pt,line join=round] (109.77, 32.48) -- (109.77,262.20);

\path[draw=drawColor,line width= 0.6pt,line join=round] (109.77, 32.48) -- (109.77,262.20);

\path[draw=drawColor,line width= 0.6pt,line join=round] (109.77, 32.48) -- (109.77,262.20);

\path[draw=drawColor,line width= 0.6pt,line join=round] (109.77, 32.48) -- (109.77,262.20);

\path[draw=drawColor,line width= 0.6pt,line join=round] (109.77, 32.48) -- (109.77,262.20);

\path[draw=drawColor,line width= 0.6pt,line join=round] (109.77, 32.48) -- (109.77,262.20);

\path[draw=drawColor,line width= 0.6pt,line join=round] (109.77, 32.48) -- (109.77,262.20);

\path[draw=drawColor,line width= 0.6pt,line join=round] (109.77, 32.48) -- (109.77,262.20);

\path[draw=drawColor,line width= 0.6pt,line join=round] (109.77, 32.48) -- (109.77,262.20);

\path[draw=drawColor,line width= 0.6pt,line join=round] (109.77, 32.48) -- (109.77,262.20);

\path[draw=drawColor,line width= 0.6pt,line join=round] (109.77, 32.48) -- (109.77,262.20);

\path[draw=drawColor,line width= 0.6pt,line join=round] (109.77, 32.48) -- (109.77,262.20);

\path[draw=drawColor,line width= 0.6pt,line join=round] (109.77, 32.48) -- (109.77,262.20);

\path[draw=drawColor,line width= 0.6pt,line join=round] (109.77, 32.48) -- (109.77,262.20);

\path[draw=drawColor,line width= 0.6pt,line join=round] (109.77, 32.48) -- (109.77,262.20);

\path[draw=drawColor,line width= 0.6pt,line join=round] (109.77, 32.48) -- (109.77,262.20);

\path[draw=drawColor,line width= 0.6pt,line join=round] (109.77, 32.48) -- (109.77,262.20);

\path[draw=drawColor,line width= 0.6pt,line join=round] (109.77, 32.48) -- (109.77,262.20);

\path[draw=drawColor,line width= 0.6pt,line join=round] (109.77, 32.48) -- (109.77,262.20);

\path[draw=drawColor,line width= 0.6pt,line join=round] (109.77, 32.48) -- (109.77,262.20);

\path[draw=drawColor,line width= 0.6pt,line join=round] (109.77, 32.48) -- (109.77,262.20);

\path[draw=drawColor,line width= 0.6pt,line join=round] (109.77, 32.48) -- (109.77,262.20);

\path[draw=drawColor,line width= 0.6pt,line join=round] (109.77, 32.48) -- (109.77,262.20);

\path[draw=drawColor,line width= 0.6pt,line join=round] (109.77, 32.48) -- (109.77,262.20);

\path[draw=drawColor,line width= 0.6pt,line join=round] (109.77, 32.48) -- (109.77,262.20);

\path[draw=drawColor,line width= 0.6pt,line join=round] (109.77, 32.48) -- (109.77,262.20);

\path[draw=drawColor,line width= 0.6pt,line join=round] (109.77, 32.48) -- (109.77,262.20);

\path[draw=drawColor,line width= 0.6pt,line join=round] (109.77, 32.48) -- (109.77,262.20);

\path[draw=drawColor,line width= 0.6pt,line join=round] (109.77, 32.48) -- (109.77,262.20);

\path[draw=drawColor,line width= 0.6pt,line join=round] (109.77, 32.48) -- (109.77,262.20);

\path[draw=drawColor,line width= 0.6pt,line join=round] (109.77, 32.48) -- (109.77,262.20);

\path[draw=drawColor,line width= 0.6pt,line join=round] (109.77, 32.48) -- (109.77,262.20);

\path[draw=drawColor,line width= 0.6pt,line join=round] (109.77, 32.48) -- (109.77,262.20);

\path[draw=drawColor,line width= 0.6pt,line join=round] (109.77, 32.48) -- (109.77,262.20);

\path[draw=drawColor,line width= 0.6pt,line join=round] (109.77, 32.48) -- (109.77,262.20);

\path[draw=drawColor,line width= 0.6pt,line join=round] (109.77, 32.48) -- (109.77,262.20);

\path[draw=drawColor,line width= 0.6pt,line join=round] (109.77, 32.48) -- (109.77,262.20);

\path[draw=drawColor,line width= 0.6pt,line join=round] (109.77, 32.48) -- (109.77,262.20);

\path[draw=drawColor,line width= 0.6pt,line join=round] (109.77, 32.48) -- (109.77,262.20);

\path[draw=drawColor,line width= 0.6pt,line join=round] (109.77, 32.48) -- (109.77,262.20);

\path[draw=drawColor,line width= 0.6pt,line join=round] (109.77, 32.48) -- (109.77,262.20);

\path[draw=drawColor,line width= 0.6pt,line join=round] (109.77, 32.48) -- (109.77,262.20);

\path[draw=drawColor,line width= 0.6pt,line join=round] (109.77, 32.48) -- (109.77,262.20);

\path[draw=drawColor,line width= 0.6pt,line join=round] (109.77, 32.48) -- (109.77,262.20);

\path[draw=drawColor,line width= 0.6pt,line join=round] (109.77, 32.48) -- (109.77,262.20);

\path[draw=drawColor,line width= 0.6pt,line join=round] (109.77, 32.48) -- (109.77,262.20);

\path[draw=drawColor,line width= 0.6pt,line join=round] (109.77, 32.48) -- (109.77,262.20);

\path[draw=drawColor,line width= 0.6pt,line join=round] (109.77, 32.48) -- (109.77,262.20);

\path[draw=drawColor,line width= 0.6pt,line join=round] (109.77, 32.48) -- (109.77,262.20);

\path[draw=drawColor,line width= 0.6pt,line join=round] (109.77, 32.48) -- (109.77,262.20);

\path[draw=drawColor,line width= 0.6pt,line join=round] (109.77, 32.48) -- (109.77,262.20);

\path[draw=drawColor,line width= 0.6pt,line join=round] (109.77, 32.48) -- (109.77,262.20);

\path[draw=drawColor,line width= 0.6pt,line join=round] (109.77, 32.48) -- (109.77,262.20);

\path[draw=drawColor,line width= 0.6pt,line join=round] (109.77, 32.48) -- (109.77,262.20);

\path[draw=drawColor,line width= 0.6pt,line join=round] (109.77, 32.48) -- (109.77,262.20);

\path[draw=drawColor,line width= 0.6pt,line join=round] (109.77, 32.48) -- (109.77,262.20);

\path[draw=drawColor,line width= 0.6pt,line join=round] (109.77, 32.48) -- (109.77,262.20);

\path[draw=drawColor,line width= 0.6pt,line join=round] (109.77, 32.48) -- (109.77,262.20);

\path[draw=drawColor,line width= 0.6pt,line join=round] (109.77, 32.48) -- (109.77,262.20);

\path[draw=drawColor,line width= 0.6pt,line join=round] (109.77, 32.48) -- (109.77,262.20);

\path[draw=drawColor,line width= 0.6pt,line join=round] (109.77, 32.48) -- (109.77,262.20);

\path[draw=drawColor,line width= 0.6pt,line join=round] (109.77, 32.48) -- (109.77,262.20);

\path[draw=drawColor,line width= 0.6pt,line join=round] (109.77, 32.48) -- (109.77,262.20);

\path[draw=drawColor,line width= 0.6pt,line join=round] (109.77, 32.48) -- (109.77,262.20);

\path[draw=drawColor,line width= 0.6pt,line join=round] (109.77, 32.48) -- (109.77,262.20);

\path[draw=drawColor,line width= 0.6pt,line join=round] (109.77, 32.48) -- (109.77,262.20);

\path[draw=drawColor,line width= 0.6pt,line join=round] (109.77, 32.48) -- (109.77,262.20);

\path[draw=drawColor,line width= 0.6pt,line join=round] (109.77, 32.48) -- (109.77,262.20);

\path[draw=drawColor,line width= 0.6pt,line join=round] (109.77, 32.48) -- (109.77,262.20);

\path[draw=drawColor,line width= 0.6pt,line join=round] (109.77, 32.48) -- (109.77,262.20);

\path[draw=drawColor,line width= 0.6pt,line join=round] (109.77, 32.48) -- (109.77,262.20);

\path[draw=drawColor,line width= 0.6pt,line join=round] (109.77, 32.48) -- (109.77,262.20);

\path[draw=drawColor,line width= 0.6pt,line join=round] (109.77, 32.48) -- (109.77,262.20);

\path[draw=drawColor,line width= 0.6pt,line join=round] (109.77, 32.48) -- (109.77,262.20);

\path[draw=drawColor,line width= 0.6pt,line join=round] (109.77, 32.48) -- (109.77,262.20);

\path[draw=drawColor,line width= 0.6pt,line join=round] (109.77, 32.48) -- (109.77,262.20);

\path[draw=drawColor,line width= 0.6pt,line join=round] (109.77, 32.48) -- (109.77,262.20);

\path[draw=drawColor,line width= 0.6pt,line join=round] (109.77, 32.48) -- (109.77,262.20);

\path[draw=drawColor,line width= 0.6pt,line join=round] (109.77, 32.48) -- (109.77,262.20);

\path[draw=drawColor,line width= 0.6pt,line join=round] (109.77, 32.48) -- (109.77,262.20);

\path[draw=drawColor,line width= 0.6pt,line join=round] (109.77, 32.48) -- (109.77,262.20);

\path[draw=drawColor,line width= 0.6pt,line join=round] (109.77, 32.48) -- (109.77,262.20);

\path[draw=drawColor,line width= 0.6pt,line join=round] (109.77, 32.48) -- (109.77,262.20);

\path[draw=drawColor,line width= 0.6pt,line join=round] (109.77, 32.48) -- (109.77,262.20);

\path[draw=drawColor,line width= 0.6pt,line join=round] (109.77, 32.48) -- (109.77,262.20);

\path[draw=drawColor,line width= 0.6pt,line join=round] (109.77, 32.48) -- (109.77,262.20);

\path[draw=drawColor,line width= 0.6pt,line join=round] (109.77, 32.48) -- (109.77,262.20);

\path[draw=drawColor,line width= 0.6pt,line join=round] (109.77, 32.48) -- (109.77,262.20);

\path[draw=drawColor,line width= 0.6pt,line join=round] (109.77, 32.48) -- (109.77,262.20);

\path[draw=drawColor,line width= 0.6pt,line join=round] (109.77, 32.48) -- (109.77,262.20);

\path[draw=drawColor,line width= 0.6pt,line join=round] (109.77, 32.48) -- (109.77,262.20);

\path[draw=drawColor,line width= 0.6pt,line join=round] (109.77, 32.48) -- (109.77,262.20);

\path[draw=drawColor,line width= 0.6pt,line join=round] (109.77, 32.48) -- (109.77,262.20);

\path[draw=drawColor,line width= 0.6pt,line join=round] (109.77, 32.48) -- (109.77,262.20);

\path[draw=drawColor,line width= 0.6pt,line join=round] (109.77, 32.48) -- (109.77,262.20);

\path[draw=drawColor,line width= 0.6pt,line join=round] (109.77, 32.48) -- (109.77,262.20);

\path[draw=drawColor,line width= 0.6pt,line join=round] (109.77, 32.48) -- (109.77,262.20);

\path[draw=drawColor,line width= 0.6pt,line join=round] (109.77, 32.48) -- (109.77,262.20);

\path[draw=drawColor,line width= 0.6pt,line join=round] (109.77, 32.48) -- (109.77,262.20);

\path[draw=drawColor,line width= 0.6pt,line join=round] (109.77, 32.48) -- (109.77,262.20);

\path[draw=drawColor,line width= 0.6pt,line join=round] (109.77, 32.48) -- (109.77,262.20);

\path[draw=drawColor,line width= 0.6pt,line join=round] (109.77, 32.48) -- (109.77,262.20);

\path[draw=drawColor,line width= 0.6pt,line join=round] (109.77, 32.48) -- (109.77,262.20);

\path[draw=drawColor,line width= 0.6pt,line join=round] (109.77, 32.48) -- (109.77,262.20);

\path[draw=drawColor,line width= 0.6pt,line join=round] (109.77, 32.48) -- (109.77,262.20);

\path[draw=drawColor,line width= 0.6pt,line join=round] (109.77, 32.48) -- (109.77,262.20);

\path[draw=drawColor,line width= 0.6pt,line join=round] (109.77, 32.48) -- (109.77,262.20);

\path[draw=drawColor,line width= 0.6pt,line join=round] (109.77, 32.48) -- (109.77,262.20);

\path[draw=drawColor,line width= 0.6pt,line join=round] (109.77, 32.48) -- (109.77,262.20);

\path[draw=drawColor,line width= 0.6pt,line join=round] (109.77, 32.48) -- (109.77,262.20);

\path[draw=drawColor,line width= 0.6pt,line join=round] (109.77, 32.48) -- (109.77,262.20);

\path[draw=drawColor,line width= 0.6pt,line join=round] (109.77, 32.48) -- (109.77,262.20);

\path[draw=drawColor,line width= 0.6pt,line join=round] (109.77, 32.48) -- (109.77,262.20);

\path[draw=drawColor,line width= 0.6pt,line join=round] (109.77, 32.48) -- (109.77,262.20);

\path[draw=drawColor,line width= 0.6pt,line join=round] (109.77, 32.48) -- (109.77,262.20);

\path[draw=drawColor,line width= 0.6pt,line join=round] (109.77, 32.48) -- (109.77,262.20);

\path[draw=drawColor,line width= 0.6pt,line join=round] (109.77, 32.48) -- (109.77,262.20);

\path[draw=drawColor,line width= 0.6pt,line join=round] (109.77, 32.48) -- (109.77,262.20);

\path[draw=drawColor,line width= 0.6pt,line join=round] (109.77, 32.48) -- (109.77,262.20);

\path[draw=drawColor,line width= 0.6pt,line join=round] (109.77, 32.48) -- (109.77,262.20);

\path[draw=drawColor,line width= 0.6pt,line join=round] (109.77, 32.48) -- (109.77,262.20);

\path[draw=drawColor,line width= 0.6pt,line join=round] (109.77, 32.48) -- (109.77,262.20);

\path[draw=drawColor,line width= 0.6pt,line join=round] (109.77, 32.48) -- (109.77,262.20);

\path[draw=drawColor,line width= 0.6pt,line join=round] (109.77, 32.48) -- (109.77,262.20);

\path[draw=drawColor,line width= 0.6pt,line join=round] (109.77, 32.48) -- (109.77,262.20);

\path[draw=drawColor,line width= 0.6pt,line join=round] (109.77, 32.48) -- (109.77,262.20);

\path[draw=drawColor,line width= 0.6pt,line join=round] (109.77, 32.48) -- (109.77,262.20);

\path[draw=drawColor,line width= 0.6pt,line join=round] (109.77, 32.48) -- (109.77,262.20);

\path[draw=drawColor,line width= 0.6pt,line join=round] (109.77, 32.48) -- (109.77,262.20);

\path[draw=drawColor,line width= 0.6pt,line join=round] (109.77, 32.48) -- (109.77,262.20);

\path[draw=drawColor,line width= 0.6pt,line join=round] (109.77, 32.48) -- (109.77,262.20);

\path[draw=drawColor,line width= 0.6pt,line join=round] (109.77, 32.48) -- (109.77,262.20);

\path[draw=drawColor,line width= 0.6pt,line join=round] (109.77, 32.48) -- (109.77,262.20);

\path[draw=drawColor,line width= 0.6pt,line join=round] (109.77, 32.48) -- (109.77,262.20);

\path[draw=drawColor,line width= 0.6pt,line join=round] (109.77, 32.48) -- (109.77,262.20);

\path[draw=drawColor,line width= 0.6pt,line join=round] (109.77, 32.48) -- (109.77,262.20);

\path[draw=drawColor,line width= 0.6pt,line join=round] (109.77, 32.48) -- (109.77,262.20);

\path[draw=drawColor,line width= 0.6pt,line join=round] (109.77, 32.48) -- (109.77,262.20);

\path[draw=drawColor,line width= 0.6pt,line join=round] (109.77, 32.48) -- (109.77,262.20);

\path[draw=drawColor,line width= 0.6pt,line join=round] (109.77, 32.48) -- (109.77,262.20);

\path[draw=drawColor,line width= 0.6pt,line join=round] (109.77, 32.48) -- (109.77,262.20);

\path[draw=drawColor,line width= 0.6pt,line join=round] (109.77, 32.48) -- (109.77,262.20);

\path[draw=drawColor,line width= 0.6pt,line join=round] (109.77, 32.48) -- (109.77,262.20);

\path[draw=drawColor,line width= 0.6pt,line join=round] (109.77, 32.48) -- (109.77,262.20);

\path[draw=drawColor,line width= 0.6pt,line join=round] (109.77, 32.48) -- (109.77,262.20);

\path[draw=drawColor,line width= 0.6pt,line join=round] (109.77, 32.48) -- (109.77,262.20);

\path[draw=drawColor,line width= 0.6pt,line join=round] (109.77, 32.48) -- (109.77,262.20);

\path[draw=drawColor,line width= 0.6pt,line join=round] (109.77, 32.48) -- (109.77,262.20);

\path[draw=drawColor,line width= 0.6pt,line join=round] (109.77, 32.48) -- (109.77,262.20);

\path[draw=drawColor,line width= 0.6pt,line join=round] (109.77, 32.48) -- (109.77,262.20);

\path[draw=drawColor,line width= 0.6pt,line join=round] (109.77, 32.48) -- (109.77,262.20);

\path[draw=drawColor,line width= 0.6pt,line join=round] (109.77, 32.48) -- (109.77,262.20);

\path[draw=drawColor,line width= 0.6pt,line join=round] (109.77, 32.48) -- (109.77,262.20);

\path[draw=drawColor,line width= 0.6pt,line join=round] (109.77, 32.48) -- (109.77,262.20);

\path[draw=drawColor,line width= 0.6pt,line join=round] (109.77, 32.48) -- (109.77,262.20);

\path[draw=drawColor,line width= 0.6pt,line join=round] (109.77, 32.48) -- (109.77,262.20);

\path[draw=drawColor,line width= 0.6pt,line join=round] (109.77, 32.48) -- (109.77,262.20);

\path[draw=drawColor,line width= 0.6pt,line join=round] (109.77, 32.48) -- (109.77,262.20);

\path[draw=drawColor,line width= 0.6pt,line join=round] (109.77, 32.48) -- (109.77,262.20);

\path[draw=drawColor,line width= 0.6pt,line join=round] (109.77, 32.48) -- (109.77,262.20);

\path[draw=drawColor,line width= 0.6pt,line join=round] (109.77, 32.48) -- (109.77,262.20);

\path[draw=drawColor,line width= 0.6pt,line join=round] (109.77, 32.48) -- (109.77,262.20);

\path[draw=drawColor,line width= 0.6pt,line join=round] (109.77, 32.48) -- (109.77,262.20);

\path[draw=drawColor,line width= 0.6pt,line join=round] (109.77, 32.48) -- (109.77,262.20);

\path[draw=drawColor,line width= 0.6pt,line join=round] (109.77, 32.48) -- (109.77,262.20);

\path[draw=drawColor,line width= 0.6pt,line join=round] (109.77, 32.48) -- (109.77,262.20);

\path[draw=drawColor,line width= 0.6pt,line join=round] (109.77, 32.48) -- (109.77,262.20);

\path[draw=drawColor,line width= 0.6pt,line join=round] (109.77, 32.48) -- (109.77,262.20);

\path[draw=drawColor,line width= 0.6pt,line join=round] (109.77, 32.48) -- (109.77,262.20);

\path[draw=drawColor,line width= 0.6pt,line join=round] (109.77, 32.48) -- (109.77,262.20);

\path[draw=drawColor,line width= 0.6pt,line join=round] (109.77, 32.48) -- (109.77,262.20);

\path[draw=drawColor,line width= 0.6pt,line join=round] (109.77, 32.48) -- (109.77,262.20);

\path[draw=drawColor,line width= 0.6pt,line join=round] (109.77, 32.48) -- (109.77,262.20);

\path[draw=drawColor,line width= 0.6pt,line join=round] (109.77, 32.48) -- (109.77,262.20);

\path[draw=drawColor,line width= 0.6pt,line join=round] (109.77, 32.48) -- (109.77,262.20);

\path[draw=drawColor,line width= 0.6pt,line join=round] (109.77, 32.48) -- (109.77,262.20);

\path[draw=drawColor,line width= 0.6pt,line join=round] (109.77, 32.48) -- (109.77,262.20);

\path[draw=drawColor,line width= 0.6pt,line join=round] (109.77, 32.48) -- (109.77,262.20);

\path[draw=drawColor,line width= 0.6pt,line join=round] (109.77, 32.48) -- (109.77,262.20);

\path[draw=drawColor,line width= 0.6pt,line join=round] (109.77, 32.48) -- (109.77,262.20);

\path[draw=drawColor,line width= 0.6pt,line join=round] (109.77, 32.48) -- (109.77,262.20);

\path[draw=drawColor,line width= 0.6pt,line join=round] (109.77, 32.48) -- (109.77,262.20);

\path[draw=drawColor,line width= 0.6pt,line join=round] (109.77, 32.48) -- (109.77,262.20);

\path[draw=drawColor,line width= 0.6pt,line join=round] (109.77, 32.48) -- (109.77,262.20);

\path[draw=drawColor,line width= 0.6pt,line join=round] (109.77, 32.48) -- (109.77,262.20);

\path[draw=drawColor,line width= 0.6pt,line join=round] (109.77, 32.48) -- (109.77,262.20);

\path[draw=drawColor,line width= 0.6pt,line join=round] (109.77, 32.48) -- (109.77,262.20);

\path[draw=drawColor,line width= 0.6pt,line join=round] (109.77, 32.48) -- (109.77,262.20);

\path[draw=drawColor,line width= 0.6pt,line join=round] (109.77, 32.48) -- (109.77,262.20);

\path[draw=drawColor,line width= 0.6pt,line join=round] (109.77, 32.48) -- (109.77,262.20);

\path[draw=drawColor,line width= 0.6pt,line join=round] (109.77, 32.48) -- (109.77,262.20);

\path[draw=drawColor,line width= 0.6pt,line join=round] (109.77, 32.48) -- (109.77,262.20);

\path[draw=drawColor,line width= 0.6pt,line join=round] (109.77, 32.48) -- (109.77,262.20);

\path[draw=drawColor,line width= 0.6pt,line join=round] (109.77, 32.48) -- (109.77,262.20);

\path[draw=drawColor,line width= 0.6pt,line join=round] (109.77, 32.48) -- (109.77,262.20);

\path[draw=drawColor,line width= 0.6pt,line join=round] (109.77, 32.48) -- (109.77,262.20);

\path[draw=drawColor,line width= 0.6pt,line join=round] (109.77, 32.48) -- (109.77,262.20);

\path[draw=drawColor,line width= 0.6pt,line join=round] (109.77, 32.48) -- (109.77,262.20);

\path[draw=drawColor,line width= 0.6pt,line join=round] (109.77, 32.48) -- (109.77,262.20);

\path[draw=drawColor,line width= 0.6pt,line join=round] (109.77, 32.48) -- (109.77,262.20);

\path[draw=drawColor,line width= 0.6pt,line join=round] (109.77, 32.48) -- (109.77,262.20);

\path[draw=drawColor,line width= 0.6pt,line join=round] (109.77, 32.48) -- (109.77,262.20);

\path[draw=drawColor,line width= 0.6pt,line join=round] (109.77, 32.48) -- (109.77,262.20);

\path[draw=drawColor,line width= 0.6pt,line join=round] (109.77, 32.48) -- (109.77,262.20);

\path[draw=drawColor,line width= 0.6pt,line join=round] (109.77, 32.48) -- (109.77,262.20);

\path[draw=drawColor,line width= 0.6pt,line join=round] (109.77, 32.48) -- (109.77,262.20);

\path[draw=drawColor,line width= 0.6pt,line join=round] (109.77, 32.48) -- (109.77,262.20);

\path[draw=drawColor,line width= 0.6pt,line join=round] (109.77, 32.48) -- (109.77,262.20);

\path[draw=drawColor,line width= 0.6pt,line join=round] (109.77, 32.48) -- (109.77,262.20);

\path[draw=drawColor,line width= 0.6pt,line join=round] (109.77, 32.48) -- (109.77,262.20);

\path[draw=drawColor,line width= 0.6pt,line join=round] (109.77, 32.48) -- (109.77,262.20);

\path[draw=drawColor,line width= 0.6pt,line join=round] (109.77, 32.48) -- (109.77,262.20);

\path[draw=drawColor,line width= 0.6pt,line join=round] (109.77, 32.48) -- (109.77,262.20);

\path[draw=drawColor,line width= 0.6pt,line join=round] (109.77, 32.48) -- (109.77,262.20);

\path[draw=drawColor,line width= 0.6pt,line join=round] (109.77, 32.48) -- (109.77,262.20);

\path[draw=drawColor,line width= 0.6pt,line join=round] (109.77, 32.48) -- (109.77,262.20);

\path[draw=drawColor,line width= 0.6pt,line join=round] (109.77, 32.48) -- (109.77,262.20);

\path[draw=drawColor,line width= 0.6pt,line join=round] (109.77, 32.48) -- (109.77,262.20);

\path[draw=drawColor,line width= 0.6pt,line join=round] (109.77, 32.48) -- (109.77,262.20);

\path[draw=drawColor,line width= 0.6pt,line join=round] (109.77, 32.48) -- (109.77,262.20);

\path[draw=drawColor,line width= 0.6pt,line join=round] (109.77, 32.48) -- (109.77,262.20);

\path[draw=drawColor,line width= 0.6pt,line join=round] (109.77, 32.48) -- (109.77,262.20);

\path[draw=drawColor,line width= 0.6pt,line join=round] (109.77, 32.48) -- (109.77,262.20);

\path[draw=drawColor,line width= 0.6pt,line join=round] (109.77, 32.48) -- (109.77,262.20);

\path[draw=drawColor,line width= 0.6pt,line join=round] (109.77, 32.48) -- (109.77,262.20);

\path[draw=drawColor,line width= 0.6pt,line join=round] (109.77, 32.48) -- (109.77,262.20);

\path[draw=drawColor,line width= 0.6pt,line join=round] (109.77, 32.48) -- (109.77,262.20);

\path[draw=drawColor,line width= 0.6pt,line join=round] (109.77, 32.48) -- (109.77,262.20);

\path[draw=drawColor,line width= 0.6pt,line join=round] (109.77, 32.48) -- (109.77,262.20);

\path[draw=drawColor,line width= 0.6pt,line join=round] (109.77, 32.48) -- (109.77,262.20);

\path[draw=drawColor,line width= 0.6pt,line join=round] (109.77, 32.48) -- (109.77,262.20);

\path[draw=drawColor,line width= 0.6pt,line join=round] (109.77, 32.48) -- (109.77,262.20);

\path[draw=drawColor,line width= 0.6pt,line join=round] (109.77, 32.48) -- (109.77,262.20);

\path[draw=drawColor,line width= 0.6pt,line join=round] (109.77, 32.48) -- (109.77,262.20);

\path[draw=drawColor,line width= 0.6pt,line join=round] (109.77, 32.48) -- (109.77,262.20);

\path[draw=drawColor,line width= 0.6pt,line join=round] (109.77, 32.48) -- (109.77,262.20);

\path[draw=drawColor,line width= 0.6pt,line join=round] (109.77, 32.48) -- (109.77,262.20);

\path[draw=drawColor,line width= 0.6pt,line join=round] (109.77, 32.48) -- (109.77,262.20);

\path[draw=drawColor,line width= 0.6pt,line join=round] (109.77, 32.48) -- (109.77,262.20);

\path[draw=drawColor,line width= 0.6pt,line join=round] (109.77, 32.48) -- (109.77,262.20);

\path[draw=drawColor,line width= 0.6pt,line join=round] (109.77, 32.48) -- (109.77,262.20);

\path[draw=drawColor,line width= 0.6pt,line join=round] (109.77, 32.48) -- (109.77,262.20);

\path[draw=drawColor,line width= 0.6pt,line join=round] (109.77, 32.48) -- (109.77,262.20);

\path[draw=drawColor,line width= 0.6pt,line join=round] (109.77, 32.48) -- (109.77,262.20);

\path[draw=drawColor,line width= 0.6pt,line join=round] (109.77, 32.48) -- (109.77,262.20);

\path[draw=drawColor,line width= 0.6pt,line join=round] (109.77, 32.48) -- (109.77,262.20);

\path[draw=drawColor,line width= 0.6pt,line join=round] (109.77, 32.48) -- (109.77,262.20);

\path[draw=drawColor,line width= 0.6pt,line join=round] (109.77, 32.48) -- (109.77,262.20);

\path[draw=drawColor,line width= 0.6pt,line join=round] (109.77, 32.48) -- (109.77,262.20);

\path[draw=drawColor,line width= 0.6pt,line join=round] (109.77, 32.48) -- (109.77,262.20);

\path[draw=drawColor,line width= 0.6pt,line join=round] (109.77, 32.48) -- (109.77,262.20);

\path[draw=drawColor,line width= 0.6pt,line join=round] (109.77, 32.48) -- (109.77,262.20);

\path[draw=drawColor,line width= 0.6pt,line join=round] (109.77, 32.48) -- (109.77,262.20);

\path[draw=drawColor,line width= 0.6pt,line join=round] (109.77, 32.48) -- (109.77,262.20);

\path[draw=drawColor,line width= 0.6pt,line join=round] (109.77, 32.48) -- (109.77,262.20);

\path[draw=drawColor,line width= 0.6pt,line join=round] (109.77, 32.48) -- (109.77,262.20);

\path[draw=drawColor,line width= 0.6pt,line join=round] (109.77, 32.48) -- (109.77,262.20);

\path[draw=drawColor,line width= 0.6pt,line join=round] (109.77, 32.48) -- (109.77,262.20);

\path[draw=drawColor,line width= 0.6pt,line join=round] (109.77, 32.48) -- (109.77,262.20);

\path[draw=drawColor,line width= 0.6pt,line join=round] (109.77, 32.48) -- (109.77,262.20);

\path[draw=drawColor,line width= 0.6pt,line join=round] (109.77, 32.48) -- (109.77,262.20);

\path[draw=drawColor,line width= 0.6pt,line join=round] (109.77, 32.48) -- (109.77,262.20);

\path[draw=drawColor,line width= 0.6pt,line join=round] (109.77, 32.48) -- (109.77,262.20);

\path[draw=drawColor,line width= 0.6pt,line join=round] (109.77, 32.48) -- (109.77,262.20);

\path[draw=drawColor,line width= 0.6pt,line join=round] (109.77, 32.48) -- (109.77,262.20);

\path[draw=drawColor,line width= 0.6pt,line join=round] (109.77, 32.48) -- (109.77,262.20);

\path[draw=drawColor,line width= 0.6pt,line join=round] (109.77, 32.48) -- (109.77,262.20);

\path[draw=drawColor,line width= 0.6pt,line join=round] (109.77, 32.48) -- (109.77,262.20);

\path[draw=drawColor,line width= 0.6pt,line join=round] (109.77, 32.48) -- (109.77,262.20);

\path[draw=drawColor,line width= 0.6pt,line join=round] (109.77, 32.48) -- (109.77,262.20);

\path[draw=drawColor,line width= 0.6pt,line join=round] (109.77, 32.48) -- (109.77,262.20);

\path[draw=drawColor,line width= 0.6pt,line join=round] (109.77, 32.48) -- (109.77,262.20);

\path[draw=drawColor,line width= 0.6pt,line join=round] (109.77, 32.48) -- (109.77,262.20);

\path[draw=drawColor,line width= 0.6pt,line join=round] (109.77, 32.48) -- (109.77,262.20);

\path[draw=drawColor,line width= 0.6pt,line join=round] (109.77, 32.48) -- (109.77,262.20);

\path[draw=drawColor,line width= 0.6pt,line join=round] (109.77, 32.48) -- (109.77,262.20);

\path[draw=drawColor,line width= 0.6pt,line join=round] (109.77, 32.48) -- (109.77,262.20);

\path[draw=drawColor,line width= 0.6pt,line join=round] (109.77, 32.48) -- (109.77,262.20);

\path[draw=drawColor,line width= 0.6pt,line join=round] (109.77, 32.48) -- (109.77,262.20);

\path[draw=drawColor,line width= 0.6pt,line join=round] (109.77, 32.48) -- (109.77,262.20);

\path[draw=drawColor,line width= 0.6pt,line join=round] (109.77, 32.48) -- (109.77,262.20);

\path[draw=drawColor,line width= 0.6pt,line join=round] (109.77, 32.48) -- (109.77,262.20);

\path[draw=drawColor,line width= 0.6pt,line join=round] (109.77, 32.48) -- (109.77,262.20);

\path[draw=drawColor,line width= 0.6pt,line join=round] (109.77, 32.48) -- (109.77,262.20);

\path[draw=drawColor,line width= 0.6pt,line join=round] (109.77, 32.48) -- (109.77,262.20);

\path[draw=drawColor,line width= 0.6pt,line join=round] (109.77, 32.48) -- (109.77,262.20);

\path[draw=drawColor,line width= 0.6pt,line join=round] (109.77, 32.48) -- (109.77,262.20);

\path[draw=drawColor,line width= 0.6pt,line join=round] (109.77, 32.48) -- (109.77,262.20);

\path[draw=drawColor,line width= 0.6pt,line join=round] (109.77, 32.48) -- (109.77,262.20);

\path[draw=drawColor,line width= 0.6pt,line join=round] (109.77, 32.48) -- (109.77,262.20);

\path[draw=drawColor,line width= 0.6pt,line join=round] (109.77, 32.48) -- (109.77,262.20);

\path[draw=drawColor,line width= 0.6pt,line join=round] (109.77, 32.48) -- (109.77,262.20);

\path[draw=drawColor,line width= 0.6pt,line join=round] (109.77, 32.48) -- (109.77,262.20);

\path[draw=drawColor,line width= 0.6pt,line join=round] (109.77, 32.48) -- (109.77,262.20);

\path[draw=drawColor,line width= 0.6pt,line join=round] (109.77, 32.48) -- (109.77,262.20);

\path[draw=drawColor,line width= 0.6pt,line join=round] (109.77, 32.48) -- (109.77,262.20);

\path[draw=drawColor,line width= 0.6pt,line join=round] (109.77, 32.48) -- (109.77,262.20);

\path[draw=drawColor,line width= 0.6pt,line join=round] (109.77, 32.48) -- (109.77,262.20);

\path[draw=drawColor,line width= 0.6pt,line join=round] (109.77, 32.48) -- (109.77,262.20);

\path[draw=drawColor,line width= 0.6pt,line join=round] (109.77, 32.48) -- (109.77,262.20);

\path[draw=drawColor,line width= 0.6pt,line join=round] (109.77, 32.48) -- (109.77,262.20);

\path[draw=drawColor,line width= 0.6pt,line join=round] (109.77, 32.48) -- (109.77,262.20);

\path[draw=drawColor,line width= 0.6pt,line join=round] (109.77, 32.48) -- (109.77,262.20);

\path[draw=drawColor,line width= 0.6pt,line join=round] (109.77, 32.48) -- (109.77,262.20);

\path[draw=drawColor,line width= 0.6pt,line join=round] (109.77, 32.48) -- (109.77,262.20);

\path[draw=drawColor,line width= 0.6pt,line join=round] (109.77, 32.48) -- (109.77,262.20);

\path[draw=drawColor,line width= 0.6pt,line join=round] (109.77, 32.48) -- (109.77,262.20);

\path[draw=drawColor,line width= 0.6pt,line join=round] (109.77, 32.48) -- (109.77,262.20);

\path[draw=drawColor,line width= 0.6pt,line join=round] (109.77, 32.48) -- (109.77,262.20);

\path[draw=drawColor,line width= 0.6pt,line join=round] (109.77, 32.48) -- (109.77,262.20);

\path[draw=drawColor,line width= 0.6pt,line join=round] (109.77, 32.48) -- (109.77,262.20);

\path[draw=drawColor,line width= 0.6pt,line join=round] (109.77, 32.48) -- (109.77,262.20);

\path[draw=drawColor,line width= 0.6pt,line join=round] (109.77, 32.48) -- (109.77,262.20);

\path[draw=drawColor,line width= 0.6pt,line join=round] (109.77, 32.48) -- (109.77,262.20);

\path[draw=drawColor,line width= 0.6pt,line join=round] (109.77, 32.48) -- (109.77,262.20);

\path[draw=drawColor,line width= 0.6pt,line join=round] (109.77, 32.48) -- (109.77,262.20);

\path[draw=drawColor,line width= 0.6pt,line join=round] (109.77, 32.48) -- (109.77,262.20);

\path[draw=drawColor,line width= 0.6pt,line join=round] (109.77, 32.48) -- (109.77,262.20);

\path[draw=drawColor,line width= 0.6pt,line join=round] (109.77, 32.48) -- (109.77,262.20);

\path[draw=drawColor,line width= 0.6pt,line join=round] (109.77, 32.48) -- (109.77,262.20);

\path[draw=drawColor,line width= 0.6pt,line join=round] (109.77, 32.48) -- (109.77,262.20);

\path[draw=drawColor,line width= 0.6pt,line join=round] (109.77, 32.48) -- (109.77,262.20);

\path[draw=drawColor,line width= 0.6pt,line join=round] (109.77, 32.48) -- (109.77,262.20);

\path[draw=drawColor,line width= 0.6pt,line join=round] (109.77, 32.48) -- (109.77,262.20);

\path[draw=drawColor,line width= 0.6pt,line join=round] (109.77, 32.48) -- (109.77,262.20);

\path[draw=drawColor,line width= 0.6pt,line join=round] (109.77, 32.48) -- (109.77,262.20);

\path[draw=drawColor,line width= 0.6pt,line join=round] (109.77, 32.48) -- (109.77,262.20);

\path[draw=drawColor,line width= 0.6pt,line join=round] (109.77, 32.48) -- (109.77,262.20);

\path[draw=drawColor,line width= 0.6pt,line join=round] (109.77, 32.48) -- (109.77,262.20);

\path[draw=drawColor,line width= 0.6pt,line join=round] (109.77, 32.48) -- (109.77,262.20);

\path[draw=drawColor,line width= 0.6pt,line join=round] (109.77, 32.48) -- (109.77,262.20);

\path[draw=drawColor,line width= 0.6pt,line join=round] (109.77, 32.48) -- (109.77,262.20);

\path[draw=drawColor,line width= 0.6pt,line join=round] (109.77, 32.48) -- (109.77,262.20);

\path[draw=drawColor,line width= 0.6pt,line join=round] (109.77, 32.48) -- (109.77,262.20);

\path[draw=drawColor,line width= 0.6pt,line join=round] (109.77, 32.48) -- (109.77,262.20);

\path[draw=drawColor,line width= 0.6pt,line join=round] (109.77, 32.48) -- (109.77,262.20);

\path[draw=drawColor,line width= 0.6pt,line join=round] (109.77, 32.48) -- (109.77,262.20);

\path[draw=drawColor,line width= 0.6pt,line join=round] (109.77, 32.48) -- (109.77,262.20);

\path[draw=drawColor,line width= 0.6pt,line join=round] (109.77, 32.48) -- (109.77,262.20);

\path[draw=drawColor,line width= 0.6pt,line join=round] (109.77, 32.48) -- (109.77,262.20);

\path[draw=drawColor,line width= 0.6pt,line join=round] (109.77, 32.48) -- (109.77,262.20);

\path[draw=drawColor,line width= 0.6pt,line join=round] (109.77, 32.48) -- (109.77,262.20);

\path[draw=drawColor,line width= 0.6pt,line join=round] (109.77, 32.48) -- (109.77,262.20);

\path[draw=drawColor,line width= 0.6pt,line join=round] (109.77, 32.48) -- (109.77,262.20);

\path[draw=drawColor,line width= 0.6pt,line join=round] (109.77, 32.48) -- (109.77,262.20);

\path[draw=drawColor,line width= 0.6pt,line join=round] (109.77, 32.48) -- (109.77,262.20);

\path[draw=drawColor,line width= 0.6pt,line join=round] (109.77, 32.48) -- (109.77,262.20);

\path[draw=drawColor,line width= 0.6pt,line join=round] (109.77, 32.48) -- (109.77,262.20);

\path[draw=drawColor,line width= 0.6pt,line join=round] (109.77, 32.48) -- (109.77,262.20);

\path[draw=drawColor,line width= 0.6pt,line join=round] (109.77, 32.48) -- (109.77,262.20);

\path[draw=drawColor,line width= 0.6pt,line join=round] (109.77, 32.48) -- (109.77,262.20);

\path[draw=drawColor,line width= 0.6pt,line join=round] (109.77, 32.48) -- (109.77,262.20);

\path[draw=drawColor,line width= 0.6pt,line join=round] (109.77, 32.48) -- (109.77,262.20);

\path[draw=drawColor,line width= 0.6pt,line join=round] (109.77, 32.48) -- (109.77,262.20);

\path[draw=drawColor,line width= 0.6pt,line join=round] (109.77, 32.48) -- (109.77,262.20);

\path[draw=drawColor,line width= 0.6pt,line join=round] (109.77, 32.48) -- (109.77,262.20);

\path[draw=drawColor,line width= 0.6pt,line join=round] (109.77, 32.48) -- (109.77,262.20);

\path[draw=drawColor,line width= 0.6pt,line join=round] (109.77, 32.48) -- (109.77,262.20);

\path[draw=drawColor,line width= 0.6pt,line join=round] (109.77, 32.48) -- (109.77,262.20);

\path[draw=drawColor,line width= 0.6pt,line join=round] (109.77, 32.48) -- (109.77,262.20);

\path[draw=drawColor,line width= 0.6pt,line join=round] (109.77, 32.48) -- (109.77,262.20);

\path[draw=drawColor,line width= 0.6pt,line join=round] (109.77, 32.48) -- (109.77,262.20);

\path[draw=drawColor,line width= 0.6pt,line join=round] (109.77, 32.48) -- (109.77,262.20);

\path[draw=drawColor,line width= 0.6pt,line join=round] (109.77, 32.48) -- (109.77,262.20);

\path[draw=drawColor,line width= 0.6pt,line join=round] (109.77, 32.48) -- (109.77,262.20);

\path[draw=drawColor,line width= 0.6pt,line join=round] (109.77, 32.48) -- (109.77,262.20);

\path[draw=drawColor,line width= 0.6pt,line join=round] (109.77, 32.48) -- (109.77,262.20);

\path[draw=drawColor,line width= 0.6pt,line join=round] (109.77, 32.48) -- (109.77,262.20);

\path[draw=drawColor,line width= 0.6pt,line join=round] (109.77, 32.48) -- (109.77,262.20);

\path[draw=drawColor,line width= 0.6pt,line join=round] (109.77, 32.48) -- (109.77,262.20);

\path[draw=drawColor,line width= 0.6pt,line join=round] (109.77, 32.48) -- (109.77,262.20);

\path[draw=drawColor,line width= 0.6pt,line join=round] (109.77, 32.48) -- (109.77,262.20);

\path[draw=drawColor,line width= 0.6pt,line join=round] (109.77, 32.48) -- (109.77,262.20);

\path[draw=drawColor,line width= 0.6pt,line join=round] (109.77, 32.48) -- (109.77,262.20);

\path[draw=drawColor,line width= 0.6pt,line join=round] (109.77, 32.48) -- (109.77,262.20);

\path[draw=drawColor,line width= 0.6pt,line join=round] (109.77, 32.48) -- (109.77,262.20);

\path[draw=drawColor,line width= 0.6pt,line join=round] (109.77, 32.48) -- (109.77,262.20);

\path[draw=drawColor,line width= 0.6pt,line join=round] (109.77, 32.48) -- (109.77,262.20);

\path[draw=drawColor,line width= 0.6pt,line join=round] (109.77, 32.48) -- (109.77,262.20);

\path[draw=drawColor,line width= 0.6pt,line join=round] (109.77, 32.48) -- (109.77,262.20);

\path[draw=drawColor,line width= 0.6pt,line join=round] (109.77, 32.48) -- (109.77,262.20);

\path[draw=drawColor,line width= 0.6pt,line join=round] (109.77, 32.48) -- (109.77,262.20);

\path[draw=drawColor,line width= 0.6pt,line join=round] (109.77, 32.48) -- (109.77,262.20);

\path[draw=drawColor,line width= 0.6pt,line join=round] (109.77, 32.48) -- (109.77,262.20);

\path[draw=drawColor,line width= 0.6pt,line join=round] (109.77, 32.48) -- (109.77,262.20);

\path[draw=drawColor,line width= 0.6pt,line join=round] (109.77, 32.48) -- (109.77,262.20);

\path[draw=drawColor,line width= 0.6pt,line join=round] (109.77, 32.48) -- (109.77,262.20);

\path[draw=drawColor,line width= 0.6pt,line join=round] (109.77, 32.48) -- (109.77,262.20);

\path[draw=drawColor,line width= 0.6pt,line join=round] (109.77, 32.48) -- (109.77,262.20);

\path[draw=drawColor,line width= 0.6pt,line join=round] (109.77, 32.48) -- (109.77,262.20);

\path[draw=drawColor,line width= 0.6pt,line join=round] (109.77, 32.48) -- (109.77,262.20);

\path[draw=drawColor,line width= 0.6pt,line join=round] (109.77, 32.48) -- (109.77,262.20);

\path[draw=drawColor,line width= 0.6pt,line join=round] (109.77, 32.48) -- (109.77,262.20);

\path[draw=drawColor,line width= 0.6pt,line join=round] (109.77, 32.48) -- (109.77,262.20);

\path[draw=drawColor,line width= 0.6pt,line join=round] (109.77, 32.48) -- (109.77,262.20);

\path[draw=drawColor,line width= 0.6pt,line join=round] (109.77, 32.48) -- (109.77,262.20);

\path[draw=drawColor,line width= 0.6pt,line join=round] (109.77, 32.48) -- (109.77,262.20);

\path[draw=drawColor,line width= 0.6pt,line join=round] (109.77, 32.48) -- (109.77,262.20);

\path[draw=drawColor,line width= 0.6pt,line join=round] (109.77, 32.48) -- (109.77,262.20);

\path[draw=drawColor,line width= 0.6pt,line join=round] (109.77, 32.48) -- (109.77,262.20);

\path[draw=drawColor,line width= 0.6pt,line join=round] (109.77, 32.48) -- (109.77,262.20);

\path[draw=drawColor,line width= 0.6pt,line join=round] (109.77, 32.48) -- (109.77,262.20);

\path[draw=drawColor,line width= 0.6pt,line join=round] (109.77, 32.48) -- (109.77,262.20);

\path[draw=drawColor,line width= 0.6pt,line join=round] (109.77, 32.48) -- (109.77,262.20);

\path[draw=drawColor,line width= 0.6pt,line join=round] (109.77, 32.48) -- (109.77,262.20);

\path[draw=drawColor,line width= 0.6pt,line join=round] (109.77, 32.48) -- (109.77,262.20);

\path[draw=drawColor,line width= 0.6pt,line join=round] (109.77, 32.48) -- (109.77,262.20);

\path[draw=drawColor,line width= 0.6pt,line join=round] (109.77, 32.48) -- (109.77,262.20);

\path[draw=drawColor,line width= 0.6pt,line join=round] (109.77, 32.48) -- (109.77,262.20);

\path[draw=drawColor,line width= 0.6pt,line join=round] (109.77, 32.48) -- (109.77,262.20);

\path[draw=drawColor,line width= 0.6pt,line join=round] (109.77, 32.48) -- (109.77,262.20);

\path[draw=drawColor,line width= 0.6pt,line join=round] (109.77, 32.48) -- (109.77,262.20);

\path[draw=drawColor,line width= 0.6pt,line join=round] (109.77, 32.48) -- (109.77,262.20);

\path[draw=drawColor,line width= 0.6pt,line join=round] (109.77, 32.48) -- (109.77,262.20);

\path[draw=drawColor,line width= 0.6pt,line join=round] (109.77, 32.48) -- (109.77,262.20);

\path[draw=drawColor,line width= 0.6pt,line join=round] (109.77, 32.48) -- (109.77,262.20);

\path[draw=drawColor,line width= 0.6pt,line join=round] (109.77, 32.48) -- (109.77,262.20);

\path[draw=drawColor,line width= 0.6pt,line join=round] (109.77, 32.48) -- (109.77,262.20);

\path[draw=drawColor,line width= 0.6pt,line join=round] (109.77, 32.48) -- (109.77,262.20);

\path[draw=drawColor,line width= 0.6pt,line join=round] (109.77, 32.48) -- (109.77,262.20);

\path[draw=drawColor,line width= 0.6pt,line join=round] (109.77, 32.48) -- (109.77,262.20);

\path[draw=drawColor,line width= 0.6pt,line join=round] (109.77, 32.48) -- (109.77,262.20);

\path[draw=drawColor,line width= 0.6pt,line join=round] (109.77, 32.48) -- (109.77,262.20);

\path[draw=drawColor,line width= 0.6pt,line join=round] (109.77, 32.48) -- (109.77,262.20);

\path[draw=drawColor,line width= 0.6pt,line join=round] (109.77, 32.48) -- (109.77,262.20);

\path[draw=drawColor,line width= 0.6pt,line join=round] (109.77, 32.48) -- (109.77,262.20);

\path[draw=drawColor,line width= 0.6pt,line join=round] (109.77, 32.48) -- (109.77,262.20);

\path[draw=drawColor,line width= 0.6pt,line join=round] (109.77, 32.48) -- (109.77,262.20);

\path[draw=drawColor,line width= 0.6pt,line join=round] (109.77, 32.48) -- (109.77,262.20);

\path[draw=drawColor,line width= 0.6pt,line join=round] (109.77, 32.48) -- (109.77,262.20);

\path[draw=drawColor,line width= 0.6pt,line join=round] (109.77, 32.48) -- (109.77,262.20);

\path[draw=drawColor,line width= 0.6pt,line join=round] (109.77, 32.48) -- (109.77,262.20);

\path[draw=drawColor,line width= 0.6pt,line join=round] (109.77, 32.48) -- (109.77,262.20);

\path[draw=drawColor,line width= 0.6pt,line join=round] (109.77, 32.48) -- (109.77,262.20);

\path[draw=drawColor,line width= 0.6pt,line join=round] (109.77, 32.48) -- (109.77,262.20);

\path[draw=drawColor,line width= 0.6pt,line join=round] (109.77, 32.48) -- (109.77,262.20);

\path[draw=drawColor,line width= 0.6pt,line join=round] (109.77, 32.48) -- (109.77,262.20);

\path[draw=drawColor,line width= 0.6pt,line join=round] (109.77, 32.48) -- (109.77,262.20);

\path[draw=drawColor,line width= 0.6pt,line join=round] (109.77, 32.48) -- (109.77,262.20);

\path[draw=drawColor,line width= 0.6pt,line join=round] (109.77, 32.48) -- (109.77,262.20);

\path[draw=drawColor,line width= 0.6pt,line join=round] (109.77, 32.48) -- (109.77,262.20);

\path[draw=drawColor,line width= 0.6pt,line join=round] (109.77, 32.48) -- (109.77,262.20);

\path[draw=drawColor,line width= 0.6pt,line join=round] (109.77, 32.48) -- (109.77,262.20);

\path[draw=drawColor,line width= 0.6pt,line join=round] (109.77, 32.48) -- (109.77,262.20);

\path[draw=drawColor,line width= 0.6pt,line join=round] (109.77, 32.48) -- (109.77,262.20);

\path[draw=drawColor,line width= 0.6pt,line join=round] (109.77, 32.48) -- (109.77,262.20);

\path[draw=drawColor,line width= 0.6pt,line join=round] (109.77, 32.48) -- (109.77,262.20);

\path[draw=drawColor,line width= 0.6pt,line join=round] (109.77, 32.48) -- (109.77,262.20);

\path[draw=drawColor,line width= 0.6pt,line join=round] (109.77, 32.48) -- (109.77,262.20);

\path[draw=drawColor,line width= 0.6pt,line join=round] (109.77, 32.48) -- (109.77,262.20);

\path[draw=drawColor,line width= 0.6pt,line join=round] (109.77, 32.48) -- (109.77,262.20);

\path[draw=drawColor,line width= 0.6pt,line join=round] (109.77, 32.48) -- (109.77,262.20);

\path[draw=drawColor,line width= 0.6pt,line join=round] (109.77, 32.48) -- (109.77,262.20);

\path[draw=drawColor,line width= 0.6pt,line join=round] (109.77, 32.48) -- (109.77,262.20);

\path[draw=drawColor,line width= 0.6pt,line join=round] (109.77, 32.48) -- (109.77,262.20);

\path[draw=drawColor,line width= 0.6pt,line join=round] (109.77, 32.48) -- (109.77,262.20);

\path[draw=drawColor,line width= 0.6pt,line join=round] (109.77, 32.48) -- (109.77,262.20);

\path[draw=drawColor,line width= 0.6pt,line join=round] (109.77, 32.48) -- (109.77,262.20);

\path[draw=drawColor,line width= 0.6pt,line join=round] (109.77, 32.48) -- (109.77,262.20);

\path[draw=drawColor,line width= 0.6pt,line join=round] (109.77, 32.48) -- (109.77,262.20);

\path[draw=drawColor,line width= 0.6pt,line join=round] (109.77, 32.48) -- (109.77,262.20);

\path[draw=drawColor,line width= 0.6pt,line join=round] (109.77, 32.48) -- (109.77,262.20);

\path[draw=drawColor,line width= 0.6pt,line join=round] (109.77, 32.48) -- (109.77,262.20);

\path[draw=drawColor,line width= 0.6pt,line join=round] (109.77, 32.48) -- (109.77,262.20);

\path[draw=drawColor,line width= 0.6pt,line join=round] (109.77, 32.48) -- (109.77,262.20);

\path[draw=drawColor,line width= 0.6pt,line join=round] (109.77, 32.48) -- (109.77,262.20);

\path[draw=drawColor,line width= 0.6pt,line join=round] (109.77, 32.48) -- (109.77,262.20);

\path[draw=drawColor,line width= 0.6pt,line join=round] (109.77, 32.48) -- (109.77,262.20);

\path[draw=drawColor,line width= 0.6pt,line join=round] (109.77, 32.48) -- (109.77,262.20);

\path[draw=drawColor,line width= 0.6pt,line join=round] (109.77, 32.48) -- (109.77,262.20);

\path[draw=drawColor,line width= 0.6pt,line join=round] (109.77, 32.48) -- (109.77,262.20);

\path[draw=drawColor,line width= 0.6pt,line join=round] (109.77, 32.48) -- (109.77,262.20);

\path[draw=drawColor,line width= 0.6pt,line join=round] (109.77, 32.48) -- (109.77,262.20);

\path[draw=drawColor,line width= 0.6pt,line join=round] (109.77, 32.48) -- (109.77,262.20);

\path[draw=drawColor,line width= 0.6pt,line join=round] (109.77, 32.48) -- (109.77,262.20);

\path[draw=drawColor,line width= 0.6pt,line join=round] (109.77, 32.48) -- (109.77,262.20);

\path[draw=drawColor,line width= 0.6pt,line join=round] (109.77, 32.48) -- (109.77,262.20);

\path[draw=drawColor,line width= 0.6pt,line join=round] (109.77, 32.48) -- (109.77,262.20);

\path[draw=drawColor,line width= 0.6pt,line join=round] (109.77, 32.48) -- (109.77,262.20);

\path[draw=drawColor,line width= 0.6pt,line join=round] (109.77, 32.48) -- (109.77,262.20);

\path[draw=drawColor,line width= 0.6pt,line join=round] (109.77, 32.48) -- (109.77,262.20);

\path[draw=drawColor,line width= 0.6pt,line join=round] (109.77, 32.48) -- (109.77,262.20);

\path[draw=drawColor,line width= 0.6pt,line join=round] (109.77, 32.48) -- (109.77,262.20);

\path[draw=drawColor,line width= 0.6pt,line join=round] (109.77, 32.48) -- (109.77,262.20);

\path[draw=drawColor,line width= 0.6pt,line join=round] (109.77, 32.48) -- (109.77,262.20);

\path[draw=drawColor,line width= 0.6pt,line join=round] (109.77, 32.48) -- (109.77,262.20);

\path[draw=drawColor,line width= 0.6pt,line join=round] (109.77, 32.48) -- (109.77,262.20);

\path[draw=drawColor,line width= 0.6pt,line join=round] (109.77, 32.48) -- (109.77,262.20);

\path[draw=drawColor,line width= 0.6pt,line join=round] (109.77, 32.48) -- (109.77,262.20);

\path[draw=drawColor,line width= 0.6pt,line join=round] (109.77, 32.48) -- (109.77,262.20);

\path[draw=drawColor,line width= 0.6pt,line join=round] (109.77, 32.48) -- (109.77,262.20);

\path[draw=drawColor,line width= 0.6pt,line join=round] (109.77, 32.48) -- (109.77,262.20);

\path[draw=drawColor,line width= 0.6pt,line join=round] (109.77, 32.48) -- (109.77,262.20);

\path[draw=drawColor,line width= 0.6pt,line join=round] (109.77, 32.48) -- (109.77,262.20);

\path[draw=drawColor,line width= 0.6pt,line join=round] (109.77, 32.48) -- (109.77,262.20);

\path[draw=drawColor,line width= 0.6pt,line join=round] (109.77, 32.48) -- (109.77,262.20);

\path[draw=drawColor,line width= 0.6pt,line join=round] (109.77, 32.48) -- (109.77,262.20);

\path[draw=drawColor,line width= 0.6pt,line join=round] (109.77, 32.48) -- (109.77,262.20);

\path[draw=drawColor,line width= 0.6pt,line join=round] (109.77, 32.48) -- (109.77,262.20);

\path[draw=drawColor,line width= 0.6pt,line join=round] (109.77, 32.48) -- (109.77,262.20);

\path[draw=drawColor,line width= 0.6pt,line join=round] (109.77, 32.48) -- (109.77,262.20);

\path[draw=drawColor,line width= 0.6pt,line join=round] (109.77, 32.48) -- (109.77,262.20);

\path[draw=drawColor,line width= 0.6pt,line join=round] (109.77, 32.48) -- (109.77,262.20);

\path[draw=drawColor,line width= 0.6pt,line join=round] (109.77, 32.48) -- (109.77,262.20);

\path[draw=drawColor,line width= 0.6pt,line join=round] (109.77, 32.48) -- (109.77,262.20);

\path[draw=drawColor,line width= 0.6pt,line join=round] (109.77, 32.48) -- (109.77,262.20);

\path[draw=drawColor,line width= 0.6pt,line join=round] (109.77, 32.48) -- (109.77,262.20);

\path[draw=drawColor,line width= 0.6pt,line join=round] (109.77, 32.48) -- (109.77,262.20);

\path[draw=drawColor,line width= 0.6pt,line join=round] (109.77, 32.48) -- (109.77,262.20);

\path[draw=drawColor,line width= 0.6pt,line join=round] (109.77, 32.48) -- (109.77,262.20);

\path[draw=drawColor,line width= 0.6pt,line join=round] (109.77, 32.48) -- (109.77,262.20);

\path[draw=drawColor,line width= 0.6pt,line join=round] (109.77, 32.48) -- (109.77,262.20);

\path[draw=drawColor,line width= 0.6pt,line join=round] (109.77, 32.48) -- (109.77,262.20);

\path[draw=drawColor,line width= 0.6pt,line join=round] (109.77, 32.48) -- (109.77,262.20);

\path[draw=drawColor,line width= 0.6pt,line join=round] (109.77, 32.48) -- (109.77,262.20);

\path[draw=drawColor,line width= 0.6pt,line join=round] (109.77, 32.48) -- (109.77,262.20);

\path[draw=drawColor,line width= 0.6pt,line join=round] (109.77, 32.48) -- (109.77,262.20);

\path[draw=drawColor,line width= 0.6pt,line join=round] (109.77, 32.48) -- (109.77,262.20);

\path[draw=drawColor,line width= 0.6pt,line join=round] (109.77, 32.48) -- (109.77,262.20);

\path[draw=drawColor,line width= 0.6pt,line join=round] (109.77, 32.48) -- (109.77,262.20);

\path[draw=drawColor,line width= 0.6pt,line join=round] (109.77, 32.48) -- (109.77,262.20);

\path[draw=drawColor,line width= 0.6pt,line join=round] (109.77, 32.48) -- (109.77,262.20);

\path[draw=drawColor,line width= 0.6pt,line join=round] (109.77, 32.48) -- (109.77,262.20);

\path[draw=drawColor,line width= 0.6pt,line join=round] (109.77, 32.48) -- (109.77,262.20);

\path[draw=drawColor,line width= 0.6pt,line join=round] (109.77, 32.48) -- (109.77,262.20);

\path[draw=drawColor,line width= 0.6pt,line join=round] (109.77, 32.48) -- (109.77,262.20);

\path[draw=drawColor,line width= 0.6pt,line join=round] (109.77, 32.48) -- (109.77,262.20);

\path[draw=drawColor,line width= 0.6pt,line join=round] (109.77, 32.48) -- (109.77,262.20);

\path[draw=drawColor,line width= 0.6pt,line join=round] (109.77, 32.48) -- (109.77,262.20);

\path[draw=drawColor,line width= 0.6pt,line join=round] (109.77, 32.48) -- (109.77,262.20);

\path[draw=drawColor,line width= 0.6pt,line join=round] (109.77, 32.48) -- (109.77,262.20);

\path[draw=drawColor,line width= 0.6pt,line join=round] (109.77, 32.48) -- (109.77,262.20);

\path[draw=drawColor,line width= 0.6pt,line join=round] (109.77, 32.48) -- (109.77,262.20);

\path[draw=drawColor,line width= 0.6pt,line join=round] (109.77, 32.48) -- (109.77,262.20);

\path[draw=drawColor,line width= 0.6pt,line join=round] (109.77, 32.48) -- (109.77,262.20);

\path[draw=drawColor,line width= 0.6pt,line join=round] (109.77, 32.48) -- (109.77,262.20);

\path[draw=drawColor,line width= 0.6pt,line join=round] (109.77, 32.48) -- (109.77,262.20);

\path[draw=drawColor,line width= 0.6pt,line join=round] (109.77, 32.48) -- (109.77,262.20);

\path[draw=drawColor,line width= 0.6pt,line join=round] (109.77, 32.48) -- (109.77,262.20);

\path[draw=drawColor,line width= 0.6pt,line join=round] (109.77, 32.48) -- (109.77,262.20);

\path[draw=drawColor,line width= 0.6pt,line join=round] (109.77, 32.48) -- (109.77,262.20);

\path[draw=drawColor,line width= 0.6pt,line join=round] (109.77, 32.48) -- (109.77,262.20);

\path[draw=drawColor,line width= 0.6pt,line join=round] (109.77, 32.48) -- (109.77,262.20);

\path[draw=drawColor,line width= 0.6pt,line join=round] (109.77, 32.48) -- (109.77,262.20);

\path[draw=drawColor,line width= 0.6pt,line join=round] (109.77, 32.48) -- (109.77,262.20);

\path[draw=drawColor,line width= 0.6pt,line join=round] (109.77, 32.48) -- (109.77,262.20);

\path[draw=drawColor,line width= 0.6pt,line join=round] (109.77, 32.48) -- (109.77,262.20);

\path[draw=drawColor,line width= 0.6pt,line join=round] (109.77, 32.48) -- (109.77,262.20);

\path[draw=drawColor,line width= 0.6pt,line join=round] (109.77, 32.48) -- (109.77,262.20);

\path[draw=drawColor,line width= 0.6pt,line join=round] (109.77, 32.48) -- (109.77,262.20);

\path[draw=drawColor,line width= 0.6pt,line join=round] (109.77, 32.48) -- (109.77,262.20);

\path[draw=drawColor,line width= 0.6pt,line join=round] (109.77, 32.48) -- (109.77,262.20);

\path[draw=drawColor,line width= 0.6pt,line join=round] (109.77, 32.48) -- (109.77,262.20);

\path[draw=drawColor,line width= 0.6pt,line join=round] (109.77, 32.48) -- (109.77,262.20);

\path[draw=drawColor,line width= 0.6pt,line join=round] (109.77, 32.48) -- (109.77,262.20);

\path[draw=drawColor,line width= 0.6pt,line join=round] (109.77, 32.48) -- (109.77,262.20);

\path[draw=drawColor,line width= 0.6pt,line join=round] (109.77, 32.48) -- (109.77,262.20);

\path[draw=drawColor,line width= 0.6pt,line join=round] (109.77, 32.48) -- (109.77,262.20);

\path[draw=drawColor,line width= 0.6pt,line join=round] (109.77, 32.48) -- (109.77,262.20);

\path[draw=drawColor,line width= 0.6pt,line join=round] (109.77, 32.48) -- (109.77,262.20);

\path[draw=drawColor,line width= 0.6pt,line join=round] (109.77, 32.48) -- (109.77,262.20);

\path[draw=drawColor,line width= 0.6pt,line join=round] (109.77, 32.48) -- (109.77,262.20);

\path[draw=drawColor,line width= 0.6pt,line join=round] (109.77, 32.48) -- (109.77,262.20);

\path[draw=drawColor,line width= 0.6pt,line join=round] (109.77, 32.48) -- (109.77,262.20);

\path[draw=drawColor,line width= 0.6pt,line join=round] (109.77, 32.48) -- (109.77,262.20);

\path[draw=drawColor,line width= 0.6pt,line join=round] (109.77, 32.48) -- (109.77,262.20);

\path[draw=drawColor,line width= 0.6pt,line join=round] (109.77, 32.48) -- (109.77,262.20);

\path[draw=drawColor,line width= 0.6pt,line join=round] (109.77, 32.48) -- (109.77,262.20);

\path[draw=drawColor,line width= 0.6pt,line join=round] (109.77, 32.48) -- (109.77,262.20);

\path[draw=drawColor,line width= 0.6pt,line join=round] (109.77, 32.48) -- (109.77,262.20);

\path[draw=drawColor,line width= 0.6pt,line join=round] (109.77, 32.48) -- (109.77,262.20);

\path[draw=drawColor,line width= 0.6pt,line join=round] (109.77, 32.48) -- (109.77,262.20);

\path[draw=drawColor,line width= 0.6pt,line join=round] (109.77, 32.48) -- (109.77,262.20);

\path[draw=drawColor,line width= 0.6pt,line join=round] (109.77, 32.48) -- (109.77,262.20);

\path[draw=drawColor,line width= 0.6pt,line join=round] (109.77, 32.48) -- (109.77,262.20);

\path[draw=drawColor,line width= 0.6pt,line join=round] (109.77, 32.48) -- (109.77,262.20);

\path[draw=drawColor,line width= 0.6pt,line join=round] (109.77, 32.48) -- (109.77,262.20);

\path[draw=drawColor,line width= 0.6pt,line join=round] (109.77, 32.48) -- (109.77,262.20);

\path[draw=drawColor,line width= 0.6pt,line join=round] (109.77, 32.48) -- (109.77,262.20);

\path[draw=drawColor,line width= 0.6pt,line join=round] (109.77, 32.48) -- (109.77,262.20);

\path[draw=drawColor,line width= 0.6pt,line join=round] (109.77, 32.48) -- (109.77,262.20);

\path[draw=drawColor,line width= 0.6pt,line join=round] (109.77, 32.48) -- (109.77,262.20);

\path[draw=drawColor,line width= 0.6pt,line join=round] (109.77, 32.48) -- (109.77,262.20);

\path[draw=drawColor,line width= 0.6pt,line join=round] (109.77, 32.48) -- (109.77,262.20);

\path[draw=drawColor,line width= 0.6pt,line join=round] (109.77, 32.48) -- (109.77,262.20);

\path[draw=drawColor,line width= 0.6pt,line join=round] (109.77, 32.48) -- (109.77,262.20);

\path[draw=drawColor,line width= 0.6pt,line join=round] (109.77, 32.48) -- (109.77,262.20);

\path[draw=drawColor,line width= 0.6pt,line join=round] (109.77, 32.48) -- (109.77,262.20);

\path[draw=drawColor,line width= 0.6pt,line join=round] (109.77, 32.48) -- (109.77,262.20);

\path[draw=drawColor,line width= 0.6pt,line join=round] (109.77, 32.48) -- (109.77,262.20);

\path[draw=drawColor,line width= 0.6pt,line join=round] (109.77, 32.48) -- (109.77,262.20);

\path[draw=drawColor,line width= 0.6pt,line join=round] (109.77, 32.48) -- (109.77,262.20);

\path[draw=drawColor,line width= 0.6pt,line join=round] (109.77, 32.48) -- (109.77,262.20);

\path[draw=drawColor,line width= 0.6pt,line join=round] (109.77, 32.48) -- (109.77,262.20);

\path[draw=drawColor,line width= 0.6pt,line join=round] (109.77, 32.48) -- (109.77,262.20);

\path[draw=drawColor,line width= 0.6pt,line join=round] (109.77, 32.48) -- (109.77,262.20);

\path[draw=drawColor,line width= 0.6pt,line join=round] (109.77, 32.48) -- (109.77,262.20);

\path[draw=drawColor,line width= 0.6pt,line join=round] (109.77, 32.48) -- (109.77,262.20);

\path[draw=drawColor,line width= 0.6pt,line join=round] (109.77, 32.48) -- (109.77,262.20);

\path[draw=drawColor,line width= 0.6pt,line join=round] (109.77, 32.48) -- (109.77,262.20);

\path[draw=drawColor,line width= 0.6pt,line join=round] (109.77, 32.48) -- (109.77,262.20);

\path[draw=drawColor,line width= 0.6pt,line join=round] (109.77, 32.48) -- (109.77,262.20);

\path[draw=drawColor,line width= 0.6pt,line join=round] (109.77, 32.48) -- (109.77,262.20);

\path[draw=drawColor,line width= 0.6pt,line join=round] (109.77, 32.48) -- (109.77,262.20);

\path[draw=drawColor,line width= 0.6pt,line join=round] (109.77, 32.48) -- (109.77,262.20);

\path[draw=drawColor,line width= 0.6pt,line join=round] (109.77, 32.48) -- (109.77,262.20);

\path[draw=drawColor,line width= 0.6pt,line join=round] (109.77, 32.48) -- (109.77,262.20);

\path[draw=drawColor,line width= 0.6pt,line join=round] (109.77, 32.48) -- (109.77,262.20);

\path[draw=drawColor,line width= 0.6pt,line join=round] (109.77, 32.48) -- (109.77,262.20);

\path[draw=drawColor,line width= 0.6pt,line join=round] (109.77, 32.48) -- (109.77,262.20);

\path[draw=drawColor,line width= 0.6pt,line join=round] (109.77, 32.48) -- (109.77,262.20);

\path[draw=drawColor,line width= 0.6pt,line join=round] (109.77, 32.48) -- (109.77,262.20);

\path[draw=drawColor,line width= 0.6pt,line join=round] (109.77, 32.48) -- (109.77,262.20);

\path[draw=drawColor,line width= 0.6pt,line join=round] (109.77, 32.48) -- (109.77,262.20);

\path[draw=drawColor,line width= 0.6pt,line join=round] (109.77, 32.48) -- (109.77,262.20);

\path[draw=drawColor,line width= 0.6pt,line join=round] (109.77, 32.48) -- (109.77,262.20);

\path[draw=drawColor,line width= 0.6pt,line join=round] (109.77, 32.48) -- (109.77,262.20);

\path[draw=drawColor,line width= 0.6pt,line join=round] (109.77, 32.48) -- (109.77,262.20);

\path[draw=drawColor,line width= 0.6pt,line join=round] (109.77, 32.48) -- (109.77,262.20);

\path[draw=drawColor,line width= 0.6pt,line join=round] (109.77, 32.48) -- (109.77,262.20);

\path[draw=drawColor,line width= 0.6pt,line join=round] (109.77, 32.48) -- (109.77,262.20);

\path[draw=drawColor,line width= 0.6pt,line join=round] (109.77, 32.48) -- (109.77,262.20);

\path[draw=drawColor,line width= 0.6pt,line join=round] (109.77, 32.48) -- (109.77,262.20);

\path[draw=drawColor,line width= 0.6pt,line join=round] (109.77, 32.48) -- (109.77,262.20);

\path[draw=drawColor,line width= 0.6pt,line join=round] (109.77, 32.48) -- (109.77,262.20);

\path[draw=drawColor,line width= 0.6pt,line join=round] (109.77, 32.48) -- (109.77,262.20);

\path[draw=drawColor,line width= 0.6pt,line join=round] (109.77, 32.48) -- (109.77,262.20);

\path[draw=drawColor,line width= 0.6pt,line join=round] (109.77, 32.48) -- (109.77,262.20);

\path[draw=drawColor,line width= 0.6pt,line join=round] (109.77, 32.48) -- (109.77,262.20);

\path[draw=drawColor,line width= 0.6pt,line join=round] (109.77, 32.48) -- (109.77,262.20);

\path[draw=drawColor,line width= 0.6pt,line join=round] (109.77, 32.48) -- (109.77,262.20);

\path[draw=drawColor,line width= 0.6pt,line join=round] (109.77, 32.48) -- (109.77,262.20);

\path[draw=drawColor,line width= 0.6pt,line join=round] (109.77, 32.48) -- (109.77,262.20);

\path[draw=drawColor,line width= 0.6pt,line join=round] (109.77, 32.48) -- (109.77,262.20);

\path[draw=drawColor,line width= 0.6pt,line join=round] (109.77, 32.48) -- (109.77,262.20);

\path[draw=drawColor,line width= 0.6pt,line join=round] (109.77, 32.48) -- (109.77,262.20);

\path[draw=drawColor,line width= 0.6pt,line join=round] (109.77, 32.48) -- (109.77,262.20);

\path[draw=drawColor,line width= 0.6pt,line join=round] (109.77, 32.48) -- (109.77,262.20);

\path[draw=drawColor,line width= 0.6pt,line join=round] (109.77, 32.48) -- (109.77,262.20);

\path[draw=drawColor,line width= 0.6pt,line join=round] (109.77, 32.48) -- (109.77,262.20);

\path[draw=drawColor,line width= 0.6pt,line join=round] (109.77, 32.48) -- (109.77,262.20);

\path[draw=drawColor,line width= 0.6pt,line join=round] (109.77, 32.48) -- (109.77,262.20);

\path[draw=drawColor,line width= 0.6pt,line join=round] (109.77, 32.48) -- (109.77,262.20);

\path[draw=drawColor,line width= 0.6pt,line join=round] (109.77, 32.48) -- (109.77,262.20);

\path[draw=drawColor,line width= 0.6pt,line join=round] (109.77, 32.48) -- (109.77,262.20);

\path[draw=drawColor,line width= 0.6pt,line join=round] (109.77, 32.48) -- (109.77,262.20);

\path[draw=drawColor,line width= 0.6pt,line join=round] (109.77, 32.48) -- (109.77,262.20);

\path[draw=drawColor,line width= 0.6pt,line join=round] (109.77, 32.48) -- (109.77,262.20);

\path[draw=drawColor,line width= 0.6pt,line join=round] (109.77, 32.48) -- (109.77,262.20);

\path[draw=drawColor,line width= 0.6pt,line join=round] (109.77, 32.48) -- (109.77,262.20);

\path[draw=drawColor,line width= 0.6pt,line join=round] (109.77, 32.48) -- (109.77,262.20);

\path[draw=drawColor,line width= 0.6pt,line join=round] (109.77, 32.48) -- (109.77,262.20);

\path[draw=drawColor,line width= 0.6pt,line join=round] (109.77, 32.48) -- (109.77,262.20);

\path[draw=drawColor,line width= 0.6pt,line join=round] (109.77, 32.48) -- (109.77,262.20);

\path[draw=drawColor,line width= 0.6pt,line join=round] (109.77, 32.48) -- (109.77,262.20);

\path[draw=drawColor,line width= 0.6pt,line join=round] (109.77, 32.48) -- (109.77,262.20);

\path[draw=drawColor,line width= 0.6pt,line join=round] (109.77, 32.48) -- (109.77,262.20);

\path[draw=drawColor,line width= 0.6pt,line join=round] (109.77, 32.48) -- (109.77,262.20);

\path[draw=drawColor,line width= 0.6pt,line join=round] (109.77, 32.48) -- (109.77,262.20);

\path[draw=drawColor,line width= 0.6pt,line join=round] (109.77, 32.48) -- (109.77,262.20);

\path[draw=drawColor,line width= 0.6pt,line join=round] (109.77, 32.48) -- (109.77,262.20);

\path[draw=drawColor,line width= 0.6pt,line join=round] (109.77, 32.48) -- (109.77,262.20);

\path[draw=drawColor,line width= 0.6pt,line join=round] (109.77, 32.48) -- (109.77,262.20);

\path[draw=drawColor,line width= 0.6pt,line join=round] (109.77, 32.48) -- (109.77,262.20);

\path[draw=drawColor,line width= 0.6pt,line join=round] (109.77, 32.48) -- (109.77,262.20);

\path[draw=drawColor,line width= 0.6pt,line join=round] (109.77, 32.48) -- (109.77,262.20);

\path[draw=drawColor,line width= 0.6pt,line join=round] (109.77, 32.48) -- (109.77,262.20);

\path[draw=drawColor,line width= 0.6pt,line join=round] (109.77, 32.48) -- (109.77,262.20);

\path[draw=drawColor,line width= 0.6pt,line join=round] (109.77, 32.48) -- (109.77,262.20);

\path[draw=drawColor,line width= 0.6pt,line join=round] (109.77, 32.48) -- (109.77,262.20);

\path[draw=drawColor,line width= 0.6pt,line join=round] (109.77, 32.48) -- (109.77,262.20);

\path[draw=drawColor,line width= 0.6pt,line join=round] (109.77, 32.48) -- (109.77,262.20);

\path[draw=drawColor,line width= 0.6pt,line join=round] (109.77, 32.48) -- (109.77,262.20);

\path[draw=drawColor,line width= 0.6pt,line join=round] (109.77, 32.48) -- (109.77,262.20);

\path[draw=drawColor,line width= 0.6pt,line join=round] (109.77, 32.48) -- (109.77,262.20);

\path[draw=drawColor,line width= 0.6pt,line join=round] (109.77, 32.48) -- (109.77,262.20);

\path[draw=drawColor,line width= 0.6pt,line join=round] (109.77, 32.48) -- (109.77,262.20);

\path[draw=drawColor,line width= 0.6pt,line join=round] (109.77, 32.48) -- (109.77,262.20);

\path[draw=drawColor,line width= 0.6pt,line join=round] (109.77, 32.48) -- (109.77,262.20);

\path[draw=drawColor,line width= 0.6pt,line join=round] (109.77, 32.48) -- (109.77,262.20);

\path[draw=drawColor,line width= 0.6pt,line join=round] (109.77, 32.48) -- (109.77,262.20);

\path[draw=drawColor,line width= 0.6pt,line join=round] (109.77, 32.48) -- (109.77,262.20);

\path[draw=drawColor,line width= 0.6pt,line join=round] (109.77, 32.48) -- (109.77,262.20);

\path[draw=drawColor,line width= 0.6pt,line join=round] (109.77, 32.48) -- (109.77,262.20);

\path[draw=drawColor,line width= 0.6pt,line join=round] (109.77, 32.48) -- (109.77,262.20);

\path[draw=drawColor,line width= 0.6pt,line join=round] (109.77, 32.48) -- (109.77,262.20);

\path[draw=drawColor,line width= 0.6pt,line join=round] (109.77, 32.48) -- (109.77,262.20);

\path[draw=drawColor,line width= 0.6pt,line join=round] (109.77, 32.48) -- (109.77,262.20);

\path[draw=drawColor,line width= 0.6pt,line join=round] (109.77, 32.48) -- (109.77,262.20);

\path[draw=drawColor,line width= 0.6pt,line join=round] (109.77, 32.48) -- (109.77,262.20);

\path[draw=drawColor,line width= 0.6pt,line join=round] (109.77, 32.48) -- (109.77,262.20);

\path[draw=drawColor,line width= 0.6pt,line join=round] (109.77, 32.48) -- (109.77,262.20);

\path[draw=drawColor,line width= 0.6pt,line join=round] (109.77, 32.48) -- (109.77,262.20);

\path[draw=drawColor,line width= 0.6pt,line join=round] (109.77, 32.48) -- (109.77,262.20);

\path[draw=drawColor,line width= 0.6pt,line join=round] (109.77, 32.48) -- (109.77,262.20);

\path[draw=drawColor,line width= 0.6pt,line join=round] (109.77, 32.48) -- (109.77,262.20);

\path[draw=drawColor,line width= 0.6pt,line join=round] (109.77, 32.48) -- (109.77,262.20);

\path[draw=drawColor,line width= 0.6pt,line join=round] (109.77, 32.48) -- (109.77,262.20);

\path[draw=drawColor,line width= 0.6pt,line join=round] (109.77, 32.48) -- (109.77,262.20);

\path[draw=drawColor,line width= 0.6pt,line join=round] (109.77, 32.48) -- (109.77,262.20);

\path[draw=drawColor,line width= 0.6pt,line join=round] (109.77, 32.48) -- (109.77,262.20);

\path[draw=drawColor,line width= 0.6pt,line join=round] (109.77, 32.48) -- (109.77,262.20);

\path[draw=drawColor,line width= 0.6pt,line join=round] (109.77, 32.48) -- (109.77,262.20);

\path[draw=drawColor,line width= 0.6pt,line join=round] (109.77, 32.48) -- (109.77,262.20);

\path[draw=drawColor,line width= 0.6pt,line join=round] (109.77, 32.48) -- (109.77,262.20);

\path[draw=drawColor,line width= 0.6pt,line join=round] (109.77, 32.48) -- (109.77,262.20);

\path[draw=drawColor,line width= 0.6pt,line join=round] (109.77, 32.48) -- (109.77,262.20);

\path[draw=drawColor,line width= 0.6pt,line join=round] (109.77, 32.48) -- (109.77,262.20);

\path[draw=drawColor,line width= 0.6pt,line join=round] (109.77, 32.48) -- (109.77,262.20);

\path[draw=drawColor,line width= 0.6pt,line join=round] (109.77, 32.48) -- (109.77,262.20);

\path[draw=drawColor,line width= 0.6pt,line join=round] (109.77, 32.48) -- (109.77,262.20);

\path[draw=drawColor,line width= 0.6pt,line join=round] (109.77, 32.48) -- (109.77,262.20);

\path[draw=drawColor,line width= 0.6pt,line join=round] (109.77, 32.48) -- (109.77,262.20);

\path[draw=drawColor,line width= 0.6pt,line join=round] (109.77, 32.48) -- (109.77,262.20);

\path[draw=drawColor,line width= 0.6pt,line join=round] (109.77, 32.48) -- (109.77,262.20);

\path[draw=drawColor,line width= 0.6pt,line join=round] (109.77, 32.48) -- (109.77,262.20);

\path[draw=drawColor,line width= 0.6pt,line join=round] (109.77, 32.48) -- (109.77,262.20);

\path[draw=drawColor,line width= 0.6pt,line join=round] (109.77, 32.48) -- (109.77,262.20);

\path[draw=drawColor,line width= 0.6pt,line join=round] (109.77, 32.48) -- (109.77,262.20);

\path[draw=drawColor,line width= 0.6pt,line join=round] (109.77, 32.48) -- (109.77,262.20);

\path[draw=drawColor,line width= 0.6pt,line join=round] (109.77, 32.48) -- (109.77,262.20);

\path[draw=drawColor,line width= 0.6pt,line join=round] (109.77, 32.48) -- (109.77,262.20);

\path[draw=drawColor,line width= 0.6pt,line join=round] (109.77, 32.48) -- (109.77,262.20);

\path[draw=drawColor,line width= 0.6pt,line join=round] (109.77, 32.48) -- (109.77,262.20);

\path[draw=drawColor,line width= 0.6pt,line join=round] (109.77, 32.48) -- (109.77,262.20);

\path[draw=drawColor,line width= 0.6pt,line join=round] (109.77, 32.48) -- (109.77,262.20);

\path[draw=drawColor,line width= 0.6pt,line join=round] (109.77, 32.48) -- (109.77,262.20);

\path[draw=drawColor,line width= 0.6pt,line join=round] (109.77, 32.48) -- (109.77,262.20);

\path[draw=drawColor,line width= 0.6pt,line join=round] (109.77, 32.48) -- (109.77,262.20);

\path[draw=drawColor,line width= 0.6pt,line join=round] (109.77, 32.48) -- (109.77,262.20);

\path[draw=drawColor,line width= 0.6pt,line join=round] (109.77, 32.48) -- (109.77,262.20);

\path[draw=drawColor,line width= 0.6pt,line join=round] (109.77, 32.48) -- (109.77,262.20);

\path[draw=drawColor,line width= 0.6pt,line join=round] (109.77, 32.48) -- (109.77,262.20);

\path[draw=drawColor,line width= 0.6pt,line join=round] (109.77, 32.48) -- (109.77,262.20);

\path[draw=drawColor,line width= 0.6pt,line join=round] (109.77, 32.48) -- (109.77,262.20);

\path[draw=drawColor,line width= 0.6pt,line join=round] (109.77, 32.48) -- (109.77,262.20);

\path[draw=drawColor,line width= 0.6pt,line join=round] (109.77, 32.48) -- (109.77,262.20);

\path[draw=drawColor,line width= 0.6pt,line join=round] (109.77, 32.48) -- (109.77,262.20);

\path[draw=drawColor,line width= 0.6pt,line join=round] (109.77, 32.48) -- (109.77,262.20);

\path[draw=drawColor,line width= 0.6pt,line join=round] (109.77, 32.48) -- (109.77,262.20);

\path[draw=drawColor,line width= 0.6pt,line join=round] (109.77, 32.48) -- (109.77,262.20);

\path[draw=drawColor,line width= 0.6pt,line join=round] (109.77, 32.48) -- (109.77,262.20);

\path[draw=drawColor,line width= 0.6pt,line join=round] (109.77, 32.48) -- (109.77,262.20);

\path[draw=drawColor,line width= 0.6pt,line join=round] (109.77, 32.48) -- (109.77,262.20);

\path[draw=drawColor,line width= 0.6pt,line join=round] (109.77, 32.48) -- (109.77,262.20);

\path[draw=drawColor,line width= 0.6pt,line join=round] (109.77, 32.48) -- (109.77,262.20);

\path[draw=drawColor,line width= 0.6pt,line join=round] (109.77, 32.48) -- (109.77,262.20);

\path[draw=drawColor,line width= 0.6pt,line join=round] (109.77, 32.48) -- (109.77,262.20);

\path[draw=drawColor,line width= 0.6pt,line join=round] (109.77, 32.48) -- (109.77,262.20);

\path[draw=drawColor,line width= 0.6pt,line join=round] (109.77, 32.48) -- (109.77,262.20);

\path[draw=drawColor,line width= 0.6pt,line join=round] (109.77, 32.48) -- (109.77,262.20);

\path[draw=drawColor,line width= 0.6pt,line join=round] (109.77, 32.48) -- (109.77,262.20);

\path[draw=drawColor,line width= 0.6pt,line join=round] (109.77, 32.48) -- (109.77,262.20);

\path[draw=drawColor,line width= 0.6pt,line join=round] (109.77, 32.48) -- (109.77,262.20);

\path[draw=drawColor,line width= 0.6pt,line join=round] (109.77, 32.48) -- (109.77,262.20);

\path[draw=drawColor,line width= 0.6pt,line join=round] (109.77, 32.48) -- (109.77,262.20);

\path[draw=drawColor,line width= 0.6pt,line join=round] (109.77, 32.48) -- (109.77,262.20);

\path[draw=drawColor,line width= 0.6pt,line join=round] (109.77, 32.48) -- (109.77,262.20);

\path[draw=drawColor,line width= 0.6pt,line join=round] (109.77, 32.48) -- (109.77,262.20);

\path[draw=drawColor,line width= 0.6pt,line join=round] (109.77, 32.48) -- (109.77,262.20);

\path[draw=drawColor,line width= 0.6pt,line join=round] (109.77, 32.48) -- (109.77,262.20);

\path[draw=drawColor,line width= 0.6pt,line join=round] (109.77, 32.48) -- (109.77,262.20);

\path[draw=drawColor,line width= 0.6pt,line join=round] (109.77, 32.48) -- (109.77,262.20);

\path[draw=drawColor,line width= 0.6pt,line join=round] (109.77, 32.48) -- (109.77,262.20);

\path[draw=drawColor,line width= 0.6pt,line join=round] (109.77, 32.48) -- (109.77,262.20);

\path[draw=drawColor,line width= 0.6pt,line join=round] (109.77, 32.48) -- (109.77,262.20);

\path[draw=drawColor,line width= 0.6pt,line join=round] (109.77, 32.48) -- (109.77,262.20);

\path[draw=drawColor,line width= 0.6pt,line join=round] (109.77, 32.48) -- (109.77,262.20);

\path[draw=drawColor,line width= 0.6pt,line join=round] (109.77, 32.48) -- (109.77,262.20);

\path[draw=drawColor,line width= 0.6pt,line join=round] (109.77, 32.48) -- (109.77,262.20);

\path[draw=drawColor,line width= 0.6pt,line join=round] (109.77, 32.48) -- (109.77,262.20);

\path[draw=drawColor,line width= 0.6pt,line join=round] (109.77, 32.48) -- (109.77,262.20);

\path[draw=drawColor,line width= 0.6pt,line join=round] (109.77, 32.48) -- (109.77,262.20);

\path[draw=drawColor,line width= 0.6pt,line join=round] (109.77, 32.48) -- (109.77,262.20);

\path[draw=drawColor,line width= 0.6pt,line join=round] (109.77, 32.48) -- (109.77,262.20);

\path[draw=drawColor,line width= 0.6pt,line join=round] (109.77, 32.48) -- (109.77,262.20);

\path[draw=drawColor,line width= 0.6pt,line join=round] (109.77, 32.48) -- (109.77,262.20);

\path[draw=drawColor,line width= 0.6pt,line join=round] (109.77, 32.48) -- (109.77,262.20);

\path[draw=drawColor,line width= 0.6pt,line join=round] (109.77, 32.48) -- (109.77,262.20);

\path[draw=drawColor,line width= 0.6pt,line join=round] (109.77, 32.48) -- (109.77,262.20);

\path[draw=drawColor,line width= 0.6pt,line join=round] (109.77, 32.48) -- (109.77,262.20);

\path[draw=drawColor,line width= 0.6pt,line join=round] (109.77, 32.48) -- (109.77,262.20);

\path[draw=drawColor,line width= 0.6pt,line join=round] (109.77, 32.48) -- (109.77,262.20);

\path[draw=drawColor,line width= 0.6pt,line join=round] (109.77, 32.48) -- (109.77,262.20);

\path[draw=drawColor,line width= 0.6pt,line join=round] (109.77, 32.48) -- (109.77,262.20);

\path[draw=drawColor,line width= 0.6pt,line join=round] (109.77, 32.48) -- (109.77,262.20);

\path[draw=drawColor,line width= 0.6pt,line join=round] (109.77, 32.48) -- (109.77,262.20);

\path[draw=drawColor,line width= 0.6pt,line join=round] (109.77, 32.48) -- (109.77,262.20);

\path[draw=drawColor,line width= 0.6pt,line join=round] (109.77, 32.48) -- (109.77,262.20);

\path[draw=drawColor,line width= 0.6pt,line join=round] (109.77, 32.48) -- (109.77,262.20);

\path[draw=drawColor,line width= 0.6pt,line join=round] (109.77, 32.48) -- (109.77,262.20);

\path[draw=drawColor,line width= 0.6pt,line join=round] (109.77, 32.48) -- (109.77,262.20);

\path[draw=drawColor,line width= 0.6pt,line join=round] (109.77, 32.48) -- (109.77,262.20);

\path[draw=drawColor,line width= 0.6pt,line join=round] (109.77, 32.48) -- (109.77,262.20);

\path[draw=drawColor,line width= 0.6pt,line join=round] (109.77, 32.48) -- (109.77,262.20);

\path[draw=drawColor,line width= 0.6pt,line join=round] (109.77, 32.48) -- (109.77,262.20);

\path[draw=drawColor,line width= 0.6pt,line join=round] (109.77, 32.48) -- (109.77,262.20);

\path[draw=drawColor,line width= 0.6pt,line join=round] (109.77, 32.48) -- (109.77,262.20);

\path[draw=drawColor,line width= 0.6pt,line join=round] (109.77, 32.48) -- (109.77,262.20);

\path[draw=drawColor,line width= 0.6pt,line join=round] (109.77, 32.48) -- (109.77,262.20);

\path[draw=drawColor,line width= 0.6pt,line join=round] (109.77, 32.48) -- (109.77,262.20);

\path[draw=drawColor,line width= 0.6pt,line join=round] (109.77, 32.48) -- (109.77,262.20);

\path[draw=drawColor,line width= 0.6pt,line join=round] (109.77, 32.48) -- (109.77,262.20);

\path[draw=drawColor,line width= 0.6pt,line join=round] (109.77, 32.48) -- (109.77,262.20);

\path[draw=drawColor,line width= 0.6pt,line join=round] (109.77, 32.48) -- (109.77,262.20);

\path[draw=drawColor,line width= 0.6pt,line join=round] (109.77, 32.48) -- (109.77,262.20);

\path[draw=drawColor,line width= 0.6pt,line join=round] (109.77, 32.48) -- (109.77,262.20);

\path[draw=drawColor,line width= 0.6pt,line join=round] (109.77, 32.48) -- (109.77,262.20);

\path[draw=drawColor,line width= 0.6pt,line join=round] (109.77, 32.48) -- (109.77,262.20);

\path[draw=drawColor,line width= 0.6pt,line join=round] (109.77, 32.48) -- (109.77,262.20);

\path[draw=drawColor,line width= 0.6pt,line join=round] (109.77, 32.48) -- (109.77,262.20);

\path[draw=drawColor,line width= 0.6pt,line join=round] (109.77, 32.48) -- (109.77,262.20);

\path[draw=drawColor,line width= 0.6pt,line join=round] (109.77, 32.48) -- (109.77,262.20);

\path[draw=drawColor,line width= 0.6pt,line join=round] (109.77, 32.48) -- (109.77,262.20);

\path[draw=drawColor,line width= 0.6pt,line join=round] (109.77, 32.48) -- (109.77,262.20);

\path[draw=drawColor,line width= 0.6pt,line join=round] (109.77, 32.48) -- (109.77,262.20);

\path[draw=drawColor,line width= 0.6pt,line join=round] (109.77, 32.48) -- (109.77,262.20);

\path[draw=drawColor,line width= 0.6pt,line join=round] (109.77, 32.48) -- (109.77,262.20);

\path[draw=drawColor,line width= 0.6pt,line join=round] (109.77, 32.48) -- (109.77,262.20);

\path[draw=drawColor,line width= 0.6pt,line join=round] (109.77, 32.48) -- (109.77,262.20);

\path[draw=drawColor,line width= 0.6pt,line join=round] (109.77, 32.48) -- (109.77,262.20);

\path[draw=drawColor,line width= 0.6pt,line join=round] (109.77, 32.48) -- (109.77,262.20);

\path[draw=drawColor,line width= 0.6pt,line join=round] (109.77, 32.48) -- (109.77,262.20);

\path[draw=drawColor,line width= 0.6pt,line join=round] (109.77, 32.48) -- (109.77,262.20);

\path[draw=drawColor,line width= 0.6pt,line join=round] (109.77, 32.48) -- (109.77,262.20);

\path[draw=drawColor,line width= 0.6pt,line join=round] (109.77, 32.48) -- (109.77,262.20);

\path[draw=drawColor,line width= 0.6pt,line join=round] (109.77, 32.48) -- (109.77,262.20);

\path[draw=drawColor,line width= 0.6pt,line join=round] (109.77, 32.48) -- (109.77,262.20);

\path[draw=drawColor,line width= 0.6pt,line join=round] (109.77, 32.48) -- (109.77,262.20);

\path[draw=drawColor,line width= 0.6pt,line join=round] (109.77, 32.48) -- (109.77,262.20);

\path[draw=drawColor,line width= 0.6pt,line join=round] (109.77, 32.48) -- (109.77,262.20);

\path[draw=drawColor,line width= 0.6pt,line join=round] (109.77, 32.48) -- (109.77,262.20);

\path[draw=drawColor,line width= 0.6pt,line join=round] (109.77, 32.48) -- (109.77,262.20);

\path[draw=drawColor,line width= 0.6pt,line join=round] (109.77, 32.48) -- (109.77,262.20);

\path[draw=drawColor,line width= 0.6pt,line join=round] (109.77, 32.48) -- (109.77,262.20);

\path[draw=drawColor,line width= 0.6pt,line join=round] (109.77, 32.48) -- (109.77,262.20);

\path[draw=drawColor,line width= 0.6pt,line join=round] (109.77, 32.48) -- (109.77,262.20);

\path[draw=drawColor,line width= 0.6pt,line join=round] (109.77, 32.48) -- (109.77,262.20);

\path[draw=drawColor,line width= 0.6pt,line join=round] (109.77, 32.48) -- (109.77,262.20);

\path[draw=drawColor,line width= 0.6pt,line join=round] (109.77, 32.48) -- (109.77,262.20);

\path[draw=drawColor,line width= 0.6pt,line join=round] (109.77, 32.48) -- (109.77,262.20);

\path[draw=drawColor,line width= 0.6pt,line join=round] (109.77, 32.48) -- (109.77,262.20);

\path[draw=drawColor,line width= 0.6pt,line join=round] (109.77, 32.48) -- (109.77,262.20);

\path[draw=drawColor,line width= 0.6pt,line join=round] (109.77, 32.48) -- (109.77,262.20);

\path[draw=drawColor,line width= 0.6pt,line join=round] (109.77, 32.48) -- (109.77,262.20);

\path[draw=drawColor,line width= 0.6pt,line join=round] (109.77, 32.48) -- (109.77,262.20);

\path[draw=drawColor,line width= 0.6pt,line join=round] (109.77, 32.48) -- (109.77,262.20);

\path[draw=drawColor,line width= 0.6pt,line join=round] (109.77, 32.48) -- (109.77,262.20);

\path[draw=drawColor,line width= 0.6pt,line join=round] (109.77, 32.48) -- (109.77,262.20);

\path[draw=drawColor,line width= 0.6pt,line join=round] (109.77, 32.48) -- (109.77,262.20);

\path[draw=drawColor,line width= 0.6pt,line join=round] (109.77, 32.48) -- (109.77,262.20);

\path[draw=drawColor,line width= 0.6pt,line join=round] (109.77, 32.48) -- (109.77,262.20);

\path[draw=drawColor,line width= 0.6pt,line join=round] (109.77, 32.48) -- (109.77,262.20);

\path[draw=drawColor,line width= 0.6pt,line join=round] (109.77, 32.48) -- (109.77,262.20);

\path[draw=drawColor,line width= 0.6pt,line join=round] (109.77, 32.48) -- (109.77,262.20);

\path[draw=drawColor,line width= 0.6pt,line join=round] (109.77, 32.48) -- (109.77,262.20);

\path[draw=drawColor,line width= 0.6pt,line join=round] (109.77, 32.48) -- (109.77,262.20);

\path[draw=drawColor,line width= 0.6pt,line join=round] (109.77, 32.48) -- (109.77,262.20);

\path[draw=drawColor,line width= 0.6pt,line join=round] (109.77, 32.48) -- (109.77,262.20);

\path[draw=drawColor,line width= 0.6pt,line join=round] (109.77, 32.48) -- (109.77,262.20);

\path[draw=drawColor,line width= 0.6pt,line join=round] (109.77, 32.48) -- (109.77,262.20);

\path[draw=drawColor,line width= 0.6pt,line join=round] (109.77, 32.48) -- (109.77,262.20);

\path[draw=drawColor,line width= 0.6pt,line join=round] (109.77, 32.48) -- (109.77,262.20);

\path[draw=drawColor,line width= 0.6pt,line join=round] (109.77, 32.48) -- (109.77,262.20);

\path[draw=drawColor,line width= 0.6pt,line join=round] (109.77, 32.48) -- (109.77,262.20);

\path[draw=drawColor,line width= 0.6pt,line join=round] (109.77, 32.48) -- (109.77,262.20);

\path[draw=drawColor,line width= 0.6pt,line join=round] (109.77, 32.48) -- (109.77,262.20);

\path[draw=drawColor,line width= 0.6pt,line join=round] (109.77, 32.48) -- (109.77,262.20);

\path[draw=drawColor,line width= 0.6pt,line join=round] (109.77, 32.48) -- (109.77,262.20);

\path[draw=drawColor,line width= 0.6pt,line join=round] (109.77, 32.48) -- (109.77,262.20);

\path[draw=drawColor,line width= 0.6pt,line join=round] (109.77, 32.48) -- (109.77,262.20);

\path[draw=drawColor,line width= 0.6pt,line join=round] (109.77, 32.48) -- (109.77,262.20);

\path[draw=drawColor,line width= 0.6pt,line join=round] (109.77, 32.48) -- (109.77,262.20);

\path[draw=drawColor,line width= 0.6pt,line join=round] (109.77, 32.48) -- (109.77,262.20);

\path[draw=drawColor,line width= 0.6pt,line join=round] (109.77, 32.48) -- (109.77,262.20);

\path[draw=drawColor,line width= 0.6pt,line join=round] (109.77, 32.48) -- (109.77,262.20);

\path[draw=drawColor,line width= 0.6pt,line join=round] (109.77, 32.48) -- (109.77,262.20);

\path[draw=drawColor,line width= 0.6pt,line join=round] (109.77, 32.48) -- (109.77,262.20);

\path[draw=drawColor,line width= 0.6pt,line join=round] (109.77, 32.48) -- (109.77,262.20);

\path[draw=drawColor,line width= 0.6pt,line join=round] (109.77, 32.48) -- (109.77,262.20);

\path[draw=drawColor,line width= 0.6pt,line join=round] (109.77, 32.48) -- (109.77,262.20);

\path[draw=drawColor,line width= 0.6pt,line join=round] (109.77, 32.48) -- (109.77,262.20);

\path[draw=drawColor,line width= 0.6pt,line join=round] (109.77, 32.48) -- (109.77,262.20);

\path[draw=drawColor,line width= 0.6pt,line join=round] (109.77, 32.48) -- (109.77,262.20);

\path[draw=drawColor,line width= 0.6pt,line join=round] (109.77, 32.48) -- (109.77,262.20);

\path[draw=drawColor,line width= 0.6pt,line join=round] (109.77, 32.48) -- (109.77,262.20);

\path[draw=drawColor,line width= 0.6pt,line join=round] (109.77, 32.48) -- (109.77,262.20);

\path[draw=drawColor,line width= 0.6pt,line join=round] (109.77, 32.48) -- (109.77,262.20);

\path[draw=drawColor,line width= 0.6pt,line join=round] (109.77, 32.48) -- (109.77,262.20);

\path[draw=drawColor,line width= 0.6pt,line join=round] (109.77, 32.48) -- (109.77,262.20);

\path[draw=drawColor,line width= 0.6pt,line join=round] (109.77, 32.48) -- (109.77,262.20);

\path[draw=drawColor,line width= 0.6pt,line join=round] (109.77, 32.48) -- (109.77,262.20);

\path[draw=drawColor,line width= 0.6pt,line join=round] (109.77, 32.48) -- (109.77,262.20);

\path[draw=drawColor,line width= 0.6pt,line join=round] (109.77, 32.48) -- (109.77,262.20);

\path[draw=drawColor,line width= 0.6pt,line join=round] (109.77, 32.48) -- (109.77,262.20);

\path[draw=drawColor,line width= 0.6pt,line join=round] (109.77, 32.48) -- (109.77,262.20);

\path[draw=drawColor,line width= 0.6pt,line join=round] (109.77, 32.48) -- (109.77,262.20);

\path[draw=drawColor,line width= 0.6pt,line join=round] (109.77, 32.48) -- (109.77,262.20);

\path[draw=drawColor,line width= 0.6pt,line join=round] (109.77, 32.48) -- (109.77,262.20);

\path[draw=drawColor,line width= 0.6pt,line join=round] (109.77, 32.48) -- (109.77,262.20);

\path[draw=drawColor,line width= 0.6pt,line join=round] (109.77, 32.48) -- (109.77,262.20);

\path[draw=drawColor,line width= 0.6pt,line join=round] (109.77, 32.48) -- (109.77,262.20);

\path[draw=drawColor,line width= 0.6pt,line join=round] (109.77, 32.48) -- (109.77,262.20);

\path[draw=drawColor,line width= 0.6pt,line join=round] (109.77, 32.48) -- (109.77,262.20);

\path[draw=drawColor,line width= 0.6pt,line join=round] (109.77, 32.48) -- (109.77,262.20);

\path[draw=drawColor,line width= 0.6pt,line join=round] (109.77, 32.48) -- (109.77,262.20);

\path[draw=drawColor,line width= 0.6pt,line join=round] (109.77, 32.48) -- (109.77,262.20);

\path[draw=drawColor,line width= 0.6pt,line join=round] (109.77, 32.48) -- (109.77,262.20);

\path[draw=drawColor,line width= 0.6pt,line join=round] (109.77, 32.48) -- (109.77,262.20);

\path[draw=drawColor,line width= 0.6pt,line join=round] (109.77, 32.48) -- (109.77,262.20);

\path[draw=drawColor,line width= 0.6pt,line join=round] (109.77, 32.48) -- (109.77,262.20);

\path[draw=drawColor,line width= 0.6pt,line join=round] (109.77, 32.48) -- (109.77,262.20);

\path[draw=drawColor,line width= 0.6pt,line join=round] (109.77, 32.48) -- (109.77,262.20);

\path[draw=drawColor,line width= 0.6pt,line join=round] (109.77, 32.48) -- (109.77,262.20);

\path[draw=drawColor,line width= 0.6pt,line join=round] (109.77, 32.48) -- (109.77,262.20);

\path[draw=drawColor,line width= 0.6pt,line join=round] (109.77, 32.48) -- (109.77,262.20);

\path[draw=drawColor,line width= 0.6pt,line join=round] (109.77, 32.48) -- (109.77,262.20);

\path[draw=drawColor,line width= 0.6pt,line join=round] (109.77, 32.48) -- (109.77,262.20);

\path[draw=drawColor,line width= 0.6pt,line join=round] (109.77, 32.48) -- (109.77,262.20);

\path[draw=drawColor,line width= 0.6pt,line join=round] (109.77, 32.48) -- (109.77,262.20);

\path[draw=drawColor,line width= 0.6pt,line join=round] (109.77, 32.48) -- (109.77,262.20);

\path[draw=drawColor,line width= 0.6pt,line join=round] (109.77, 32.48) -- (109.77,262.20);

\path[draw=drawColor,line width= 0.6pt,line join=round] (109.77, 32.48) -- (109.77,262.20);

\path[draw=drawColor,line width= 0.6pt,line join=round] (109.77, 32.48) -- (109.77,262.20);

\path[draw=drawColor,line width= 0.6pt,line join=round] (109.77, 32.48) -- (109.77,262.20);

\path[draw=drawColor,line width= 0.6pt,line join=round] (109.77, 32.48) -- (109.77,262.20);

\path[draw=drawColor,line width= 0.6pt,line join=round] (109.77, 32.48) -- (109.77,262.20);

\path[draw=drawColor,line width= 0.6pt,line join=round] (109.77, 32.48) -- (109.77,262.20);

\path[draw=drawColor,line width= 0.6pt,line join=round] (109.77, 32.48) -- (109.77,262.20);

\path[draw=drawColor,line width= 0.6pt,line join=round] (109.77, 32.48) -- (109.77,262.20);

\path[draw=drawColor,line width= 0.6pt,line join=round] (109.77, 32.48) -- (109.77,262.20);

\path[draw=drawColor,line width= 0.6pt,line join=round] (109.77, 32.48) -- (109.77,262.20);

\path[draw=drawColor,line width= 0.6pt,line join=round] (109.77, 32.48) -- (109.77,262.20);

\path[draw=drawColor,line width= 0.6pt,line join=round] (109.77, 32.48) -- (109.77,262.20);

\path[draw=drawColor,line width= 0.6pt,line join=round] (109.77, 32.48) -- (109.77,262.20);

\path[draw=drawColor,line width= 0.6pt,line join=round] (109.77, 32.48) -- (109.77,262.20);

\path[draw=drawColor,line width= 0.6pt,line join=round] (109.77, 32.48) -- (109.77,262.20);

\path[draw=drawColor,line width= 0.6pt,line join=round] (109.77, 32.48) -- (109.77,262.20);

\path[draw=drawColor,line width= 0.6pt,line join=round] (109.77, 32.48) -- (109.77,262.20);

\path[draw=drawColor,line width= 0.6pt,line join=round] (109.77, 32.48) -- (109.77,262.20);

\path[draw=drawColor,line width= 0.6pt,line join=round] (109.77, 32.48) -- (109.77,262.20);

\path[draw=drawColor,line width= 0.6pt,line join=round] (109.77, 32.48) -- (109.77,262.20);

\path[draw=drawColor,line width= 0.6pt,line join=round] (109.77, 32.48) -- (109.77,262.20);

\path[draw=drawColor,line width= 0.6pt,line join=round] (109.77, 32.48) -- (109.77,262.20);

\path[draw=drawColor,line width= 0.6pt,line join=round] (109.77, 32.48) -- (109.77,262.20);

\path[draw=drawColor,line width= 0.6pt,line join=round] (109.77, 32.48) -- (109.77,262.20);

\path[draw=drawColor,line width= 0.6pt,line join=round] (109.77, 32.48) -- (109.77,262.20);

\path[draw=drawColor,line width= 0.6pt,line join=round] (109.77, 32.48) -- (109.77,262.20);

\path[draw=drawColor,line width= 0.6pt,line join=round] (109.77, 32.48) -- (109.77,262.20);

\path[draw=drawColor,line width= 0.6pt,line join=round] (109.77, 32.48) -- (109.77,262.20);

\path[draw=drawColor,line width= 0.6pt,line join=round] (109.77, 32.48) -- (109.77,262.20);

\path[draw=drawColor,line width= 0.6pt,line join=round] (109.77, 32.48) -- (109.77,262.20);

\path[draw=drawColor,line width= 0.6pt,line join=round] (109.77, 32.48) -- (109.77,262.20);

\path[draw=drawColor,line width= 0.6pt,line join=round] (109.77, 32.48) -- (109.77,262.20);

\path[draw=drawColor,line width= 0.6pt,line join=round] (109.77, 32.48) -- (109.77,262.20);

\path[draw=drawColor,line width= 0.6pt,line join=round] (109.77, 32.48) -- (109.77,262.20);

\path[draw=drawColor,line width= 0.6pt,line join=round] (109.77, 32.48) -- (109.77,262.20);

\path[draw=drawColor,line width= 0.6pt,line join=round] (109.77, 32.48) -- (109.77,262.20);

\path[draw=drawColor,line width= 0.6pt,line join=round] (109.77, 32.48) -- (109.77,262.20);

\path[draw=drawColor,line width= 0.6pt,line join=round] (109.77, 32.48) -- (109.77,262.20);

\path[draw=drawColor,line width= 0.6pt,line join=round] (109.77, 32.48) -- (109.77,262.20);

\path[draw=drawColor,line width= 0.6pt,line join=round] (109.77, 32.48) -- (109.77,262.20);

\path[draw=drawColor,line width= 0.6pt,line join=round] (109.77, 32.48) -- (109.77,262.20);

\path[draw=drawColor,line width= 0.6pt,line join=round] (109.77, 32.48) -- (109.77,262.20);

\path[draw=drawColor,line width= 0.6pt,line join=round] (109.77, 32.48) -- (109.77,262.20);

\path[draw=drawColor,line width= 0.6pt,line join=round] (109.77, 32.48) -- (109.77,262.20);

\path[draw=drawColor,line width= 0.6pt,line join=round] (109.77, 32.48) -- (109.77,262.20);

\path[draw=drawColor,line width= 0.6pt,line join=round] (109.77, 32.48) -- (109.77,262.20);

\path[draw=drawColor,line width= 0.6pt,line join=round] (109.77, 32.48) -- (109.77,262.20);

\path[draw=drawColor,line width= 0.6pt,line join=round] (109.77, 32.48) -- (109.77,262.20);

\path[draw=drawColor,line width= 0.6pt,line join=round] (109.77, 32.48) -- (109.77,262.20);

\path[draw=drawColor,line width= 0.6pt,line join=round] (109.77, 32.48) -- (109.77,262.20);

\path[draw=drawColor,line width= 0.6pt,line join=round] (109.77, 32.48) -- (109.77,262.20);

\path[draw=drawColor,line width= 0.6pt,line join=round] (109.77, 32.48) -- (109.77,262.20);

\path[draw=drawColor,line width= 0.6pt,line join=round] (109.77, 32.48) -- (109.77,262.20);

\path[draw=drawColor,line width= 0.6pt,line join=round] (109.77, 32.48) -- (109.77,262.20);

\path[draw=drawColor,line width= 0.6pt,line join=round] (109.77, 32.48) -- (109.77,262.20);

\path[draw=drawColor,line width= 0.6pt,line join=round] (109.77, 32.48) -- (109.77,262.20);

\path[draw=drawColor,line width= 0.6pt,line join=round] (109.77, 32.48) -- (109.77,262.20);

\path[draw=drawColor,line width= 0.6pt,line join=round] (109.77, 32.48) -- (109.77,262.20);

\path[draw=drawColor,line width= 0.6pt,line join=round] (109.77, 32.48) -- (109.77,262.20);

\path[draw=drawColor,line width= 0.6pt,line join=round] (109.77, 32.48) -- (109.77,262.20);

\path[draw=drawColor,line width= 0.6pt,line join=round] (109.77, 32.48) -- (109.77,262.20);

\path[draw=drawColor,line width= 0.6pt,line join=round] (109.77, 32.48) -- (109.77,262.20);

\path[draw=drawColor,line width= 0.6pt,line join=round] (109.77, 32.48) -- (109.77,262.20);

\path[draw=drawColor,line width= 0.6pt,line join=round] (109.77, 32.48) -- (109.77,262.20);

\path[draw=drawColor,line width= 0.6pt,line join=round] (109.77, 32.48) -- (109.77,262.20);

\path[draw=drawColor,line width= 0.6pt,line join=round] (109.77, 32.48) -- (109.77,262.20);

\path[draw=drawColor,line width= 0.6pt,line join=round] (109.77, 32.48) -- (109.77,262.20);

\path[draw=drawColor,line width= 0.6pt,line join=round] (109.77, 32.48) -- (109.77,262.20);

\path[draw=drawColor,line width= 0.6pt,line join=round] (109.77, 32.48) -- (109.77,262.20);

\path[draw=drawColor,line width= 0.6pt,line join=round] (109.77, 32.48) -- (109.77,262.20);

\path[draw=drawColor,line width= 0.6pt,line join=round] (109.77, 32.48) -- (109.77,262.20);

\path[draw=drawColor,line width= 0.6pt,line join=round] (109.77, 32.48) -- (109.77,262.20);

\path[draw=drawColor,line width= 0.6pt,line join=round] (109.77, 32.48) -- (109.77,262.20);

\path[draw=drawColor,line width= 0.6pt,line join=round] (109.77, 32.48) -- (109.77,262.20);

\path[draw=drawColor,line width= 0.6pt,line join=round] (109.77, 32.48) -- (109.77,262.20);

\path[draw=drawColor,line width= 0.6pt,line join=round] (109.77, 32.48) -- (109.77,262.20);

\path[draw=drawColor,line width= 0.6pt,line join=round] (109.77, 32.48) -- (109.77,262.20);

\path[draw=drawColor,line width= 0.6pt,line join=round] (109.77, 32.48) -- (109.77,262.20);

\path[draw=drawColor,line width= 0.6pt,line join=round] (109.77, 32.48) -- (109.77,262.20);

\path[draw=drawColor,line width= 0.6pt,line join=round] (109.77, 32.48) -- (109.77,262.20);

\path[draw=drawColor,line width= 0.6pt,line join=round] (109.77, 32.48) -- (109.77,262.20);

\path[draw=drawColor,line width= 0.6pt,line join=round] (109.77, 32.48) -- (109.77,262.20);

\path[draw=drawColor,line width= 0.6pt,line join=round] (109.77, 32.48) -- (109.77,262.20);

\path[draw=drawColor,line width= 0.6pt,line join=round] (109.77, 32.48) -- (109.77,262.20);

\path[draw=drawColor,line width= 0.6pt,line join=round] (109.77, 32.48) -- (109.77,262.20);

\path[draw=drawColor,line width= 0.6pt,line join=round] (109.77, 32.48) -- (109.77,262.20);

\path[draw=drawColor,line width= 0.6pt,line join=round] (109.77, 32.48) -- (109.77,262.20);

\path[draw=drawColor,line width= 0.6pt,line join=round] (109.77, 32.48) -- (109.77,262.20);

\path[draw=drawColor,line width= 0.6pt,line join=round] (109.77, 32.48) -- (109.77,262.20);

\path[draw=drawColor,line width= 0.6pt,line join=round] (109.77, 32.48) -- (109.77,262.20);

\path[draw=drawColor,line width= 0.6pt,line join=round] (109.77, 32.48) -- (109.77,262.20);

\path[draw=drawColor,line width= 0.6pt,line join=round] (109.77, 32.48) -- (109.77,262.20);

\path[draw=drawColor,line width= 0.6pt,line join=round] (109.77, 32.48) -- (109.77,262.20);

\path[draw=drawColor,line width= 0.6pt,line join=round] (109.77, 32.48) -- (109.77,262.20);

\path[draw=drawColor,line width= 0.6pt,line join=round] (109.77, 32.48) -- (109.77,262.20);

\path[draw=drawColor,line width= 0.6pt,line join=round] (109.77, 32.48) -- (109.77,262.20);

\path[draw=drawColor,line width= 0.6pt,line join=round] (109.77, 32.48) -- (109.77,262.20);

\path[draw=drawColor,line width= 0.6pt,line join=round] (109.77, 32.48) -- (109.77,262.20);

\path[draw=drawColor,line width= 0.6pt,line join=round] (109.77, 32.48) -- (109.77,262.20);

\path[draw=drawColor,line width= 0.6pt,line join=round] (109.77, 32.48) -- (109.77,262.20);

\path[draw=drawColor,line width= 0.6pt,line join=round] (109.77, 32.48) -- (109.77,262.20);

\path[draw=drawColor,line width= 0.6pt,line join=round] (109.77, 32.48) -- (109.77,262.20);

\path[draw=drawColor,line width= 0.6pt,line join=round] (109.77, 32.48) -- (109.77,262.20);

\path[draw=drawColor,line width= 0.6pt,line join=round] (109.77, 32.48) -- (109.77,262.20);

\path[draw=drawColor,line width= 0.6pt,line join=round] (109.77, 32.48) -- (109.77,262.20);

\path[draw=drawColor,line width= 0.6pt,line join=round] (109.77, 32.48) -- (109.77,262.20);

\path[draw=drawColor,line width= 0.6pt,line join=round] (109.77, 32.48) -- (109.77,262.20);

\path[draw=drawColor,line width= 0.6pt,line join=round] (109.77, 32.48) -- (109.77,262.20);

\path[draw=drawColor,line width= 0.6pt,line join=round] (109.77, 32.48) -- (109.77,262.20);

\path[draw=drawColor,line width= 0.6pt,line join=round] (109.77, 32.48) -- (109.77,262.20);

\path[draw=drawColor,line width= 0.6pt,line join=round] (109.77, 32.48) -- (109.77,262.20);

\path[draw=drawColor,line width= 0.6pt,line join=round] (109.77, 32.48) -- (109.77,262.20);

\path[draw=drawColor,line width= 0.6pt,line join=round] (109.77, 32.48) -- (109.77,262.20);

\path[draw=drawColor,line width= 0.6pt,line join=round] (109.77, 32.48) -- (109.77,262.20);

\path[draw=drawColor,line width= 0.6pt,line join=round] (109.77, 32.48) -- (109.77,262.20);

\path[draw=drawColor,line width= 0.6pt,line join=round] (109.77, 32.48) -- (109.77,262.20);

\path[draw=drawColor,line width= 0.6pt,line join=round] (109.77, 32.48) -- (109.77,262.20);

\path[draw=drawColor,line width= 0.6pt,line join=round] (109.77, 32.48) -- (109.77,262.20);

\path[draw=drawColor,line width= 0.6pt,line join=round] (109.77, 32.48) -- (109.77,262.20);

\path[draw=drawColor,line width= 0.6pt,line join=round] (109.77, 32.48) -- (109.77,262.20);

\path[draw=drawColor,line width= 0.6pt,line join=round] (109.77, 32.48) -- (109.77,262.20);

\path[draw=drawColor,line width= 0.6pt,line join=round] (109.77, 32.48) -- (109.77,262.20);

\path[draw=drawColor,line width= 0.6pt,line join=round] (109.77, 32.48) -- (109.77,262.20);

\path[draw=drawColor,line width= 0.6pt,line join=round] (109.77, 32.48) -- (109.77,262.20);

\path[draw=drawColor,line width= 0.6pt,line join=round] (109.77, 32.48) -- (109.77,262.20);

\path[draw=drawColor,line width= 0.6pt,line join=round] (109.77, 32.48) -- (109.77,262.20);

\path[draw=drawColor,line width= 0.6pt,line join=round] (109.77, 32.48) -- (109.77,262.20);

\path[draw=drawColor,line width= 0.6pt,line join=round] (109.77, 32.48) -- (109.77,262.20);

\path[draw=drawColor,line width= 0.6pt,line join=round] (109.77, 32.48) -- (109.77,262.20);

\path[draw=drawColor,line width= 0.6pt,line join=round] (109.77, 32.48) -- (109.77,262.20);

\path[draw=drawColor,line width= 0.6pt,line join=round] (109.77, 32.48) -- (109.77,262.20);

\path[draw=drawColor,line width= 0.6pt,line join=round] (109.77, 32.48) -- (109.77,262.20);

\path[draw=drawColor,line width= 0.6pt,line join=round] (109.77, 32.48) -- (109.77,262.20);

\path[draw=drawColor,line width= 0.6pt,line join=round] (109.77, 32.48) -- (109.77,262.20);

\path[draw=drawColor,line width= 0.6pt,line join=round] (109.77, 32.48) -- (109.77,262.20);

\path[draw=drawColor,line width= 0.6pt,line join=round] (109.77, 32.48) -- (109.77,262.20);

\path[draw=drawColor,line width= 0.6pt,line join=round] (109.77, 32.48) -- (109.77,262.20);

\path[draw=drawColor,line width= 0.6pt,line join=round] (109.77, 32.48) -- (109.77,262.20);

\path[draw=drawColor,line width= 0.6pt,line join=round] (109.77, 32.48) -- (109.77,262.20);

\path[draw=drawColor,line width= 0.6pt,line join=round] (109.77, 32.48) -- (109.77,262.20);

\path[draw=drawColor,line width= 0.6pt,line join=round] (109.77, 32.48) -- (109.77,262.20);

\path[draw=drawColor,line width= 0.6pt,line join=round] (109.77, 32.48) -- (109.77,262.20);

\path[draw=drawColor,line width= 0.6pt,line join=round] (109.77, 32.48) -- (109.77,262.20);

\path[draw=drawColor,line width= 0.6pt,line join=round] (109.77, 32.48) -- (109.77,262.20);

\path[draw=drawColor,line width= 0.6pt,line join=round] (109.77, 32.48) -- (109.77,262.20);

\path[draw=drawColor,line width= 0.6pt,line join=round] (109.77, 32.48) -- (109.77,262.20);

\path[draw=drawColor,line width= 0.6pt,line join=round] (109.77, 32.48) -- (109.77,262.20);

\path[draw=drawColor,line width= 0.6pt,line join=round] (109.77, 32.48) -- (109.77,262.20);

\path[draw=drawColor,line width= 0.6pt,line join=round] (109.77, 32.48) -- (109.77,262.20);

\path[draw=drawColor,line width= 0.6pt,line join=round] (109.77, 32.48) -- (109.77,262.20);

\path[draw=drawColor,line width= 0.6pt,line join=round] (109.77, 32.48) -- (109.77,262.20);

\path[draw=drawColor,line width= 0.6pt,line join=round] (109.77, 32.48) -- (109.77,262.20);

\path[draw=drawColor,line width= 0.6pt,line join=round] (109.77, 32.48) -- (109.77,262.20);

\path[draw=drawColor,line width= 0.6pt,line join=round] (109.77, 32.48) -- (109.77,262.20);

\path[draw=drawColor,line width= 0.6pt,line join=round] (109.77, 32.48) -- (109.77,262.20);

\path[draw=drawColor,line width= 0.6pt,line join=round] (109.77, 32.48) -- (109.77,262.20);

\path[draw=drawColor,line width= 0.6pt,line join=round] (109.77, 32.48) -- (109.77,262.20);

\path[draw=drawColor,line width= 0.6pt,line join=round] (109.77, 32.48) -- (109.77,262.20);

\path[draw=drawColor,line width= 0.6pt,line join=round] (109.77, 32.48) -- (109.77,262.20);

\path[draw=drawColor,line width= 0.6pt,line join=round] (109.77, 32.48) -- (109.77,262.20);

\path[draw=drawColor,line width= 0.6pt,line join=round] (109.77, 32.48) -- (109.77,262.20);

\path[draw=drawColor,line width= 0.6pt,line join=round] (109.77, 32.48) -- (109.77,262.20);

\path[draw=drawColor,line width= 0.6pt,line join=round] (109.77, 32.48) -- (109.77,262.20);

\path[draw=drawColor,line width= 0.6pt,line join=round] (109.77, 32.48) -- (109.77,262.20);

\path[draw=drawColor,line width= 0.6pt,line join=round] (109.77, 32.48) -- (109.77,262.20);

\path[draw=drawColor,line width= 0.6pt,line join=round] (109.77, 32.48) -- (109.77,262.20);

\path[draw=drawColor,line width= 0.6pt,line join=round] (109.77, 32.48) -- (109.77,262.20);

\path[draw=drawColor,line width= 0.6pt,line join=round] (109.77, 32.48) -- (109.77,262.20);

\path[draw=drawColor,line width= 0.6pt,line join=round] (109.77, 32.48) -- (109.77,262.20);

\path[draw=drawColor,line width= 0.6pt,line join=round] (109.77, 32.48) -- (109.77,262.20);

\path[draw=drawColor,line width= 0.6pt,line join=round] (109.77, 32.48) -- (109.77,262.20);

\path[draw=drawColor,line width= 0.6pt,line join=round] (109.77, 32.48) -- (109.77,262.20);

\path[draw=drawColor,line width= 0.6pt,line join=round] (109.77, 32.48) -- (109.77,262.20);

\path[draw=drawColor,line width= 0.6pt,line join=round] (109.77, 32.48) -- (109.77,262.20);

\path[draw=drawColor,line width= 0.6pt,line join=round] (109.77, 32.48) -- (109.77,262.20);

\path[draw=drawColor,line width= 0.6pt,line join=round] (109.77, 32.48) -- (109.77,262.20);

\path[draw=drawColor,line width= 0.6pt,line join=round] (109.77, 32.48) -- (109.77,262.20);

\path[draw=drawColor,line width= 0.6pt,line join=round] (109.77, 32.48) -- (109.77,262.20);

\path[draw=drawColor,line width= 0.6pt,line join=round] (109.77, 32.48) -- (109.77,262.20);

\path[draw=drawColor,line width= 0.6pt,line join=round] (109.77, 32.48) -- (109.77,262.20);

\path[draw=drawColor,line width= 0.6pt,line join=round] (109.77, 32.48) -- (109.77,262.20);

\path[draw=drawColor,line width= 0.6pt,line join=round] (109.77, 32.48) -- (109.77,262.20);

\path[draw=drawColor,line width= 0.6pt,line join=round] (109.77, 32.48) -- (109.77,262.20);

\path[draw=drawColor,line width= 0.6pt,line join=round] (109.77, 32.48) -- (109.77,262.20);

\path[draw=drawColor,line width= 0.6pt,line join=round] (109.77, 32.48) -- (109.77,262.20);

\path[draw=drawColor,line width= 0.6pt,line join=round] (109.77, 32.48) -- (109.77,262.20);

\path[draw=drawColor,line width= 0.6pt,line join=round] (109.77, 32.48) -- (109.77,262.20);

\path[draw=drawColor,line width= 0.6pt,line join=round] (109.77, 32.48) -- (109.77,262.20);

\path[draw=drawColor,line width= 0.6pt,line join=round] (109.77, 32.48) -- (109.77,262.20);

\path[draw=drawColor,line width= 0.6pt,line join=round] (109.77, 32.48) -- (109.77,262.20);

\path[draw=drawColor,line width= 0.6pt,line join=round] (109.77, 32.48) -- (109.77,262.20);

\path[draw=drawColor,line width= 0.6pt,line join=round] (109.77, 32.48) -- (109.77,262.20);

\path[draw=drawColor,line width= 0.6pt,line join=round] (109.77, 32.48) -- (109.77,262.20);

\path[draw=drawColor,line width= 0.6pt,line join=round] (109.77, 32.48) -- (109.77,262.20);

\path[draw=drawColor,line width= 0.6pt,line join=round] (109.77, 32.48) -- (109.77,262.20);

\path[draw=drawColor,line width= 0.6pt,line join=round] (109.77, 32.48) -- (109.77,262.20);

\path[draw=drawColor,line width= 0.6pt,line join=round] (109.77, 32.48) -- (109.77,262.20);

\path[draw=drawColor,line width= 0.6pt,line join=round] (109.77, 32.48) -- (109.77,262.20);

\path[draw=drawColor,line width= 0.6pt,line join=round] (109.77, 32.48) -- (109.77,262.20);

\path[draw=drawColor,line width= 0.6pt,line join=round] (109.77, 32.48) -- (109.77,262.20);

\path[draw=drawColor,line width= 0.6pt,line join=round] (109.77, 32.48) -- (109.77,262.20);

\path[draw=drawColor,line width= 0.6pt,line join=round] (109.77, 32.48) -- (109.77,262.20);

\path[draw=drawColor,line width= 0.6pt,line join=round] (109.77, 32.48) -- (109.77,262.20);

\path[draw=drawColor,line width= 0.6pt,line join=round] (109.77, 32.48) -- (109.77,262.20);

\path[draw=drawColor,line width= 0.6pt,line join=round] (109.77, 32.48) -- (109.77,262.20);

\path[draw=drawColor,line width= 0.6pt,line join=round] (109.77, 32.48) -- (109.77,262.20);

\path[draw=drawColor,line width= 0.6pt,line join=round] (109.77, 32.48) -- (109.77,262.20);

\path[draw=drawColor,line width= 0.6pt,line join=round] (109.77, 32.48) -- (109.77,262.20);

\path[draw=drawColor,line width= 0.6pt,line join=round] (109.77, 32.48) -- (109.77,262.20);

\path[draw=drawColor,line width= 0.6pt,line join=round] (109.77, 32.48) -- (109.77,262.20);

\path[draw=drawColor,line width= 0.6pt,line join=round] (109.77, 32.48) -- (109.77,262.20);

\path[draw=drawColor,line width= 0.6pt,line join=round] (109.77, 32.48) -- (109.77,262.20);

\path[draw=drawColor,line width= 0.6pt,line join=round] (109.77, 32.48) -- (109.77,262.20);

\path[draw=drawColor,line width= 0.6pt,line join=round] (109.77, 32.48) -- (109.77,262.20);

\path[draw=drawColor,line width= 0.6pt,line join=round] (109.77, 32.48) -- (109.77,262.20);

\path[draw=drawColor,line width= 0.6pt,line join=round] (109.77, 32.48) -- (109.77,262.20);

\path[draw=drawColor,line width= 0.6pt,line join=round] (109.77, 32.48) -- (109.77,262.20);

\path[draw=drawColor,line width= 0.6pt,line join=round] (109.77, 32.48) -- (109.77,262.20);

\path[draw=drawColor,line width= 0.6pt,line join=round] (109.77, 32.48) -- (109.77,262.20);

\path[draw=drawColor,line width= 0.6pt,line join=round] (109.77, 32.48) -- (109.77,262.20);

\path[draw=drawColor,line width= 0.6pt,line join=round] (109.77, 32.48) -- (109.77,262.20);

\path[draw=drawColor,line width= 0.6pt,line join=round] (109.77, 32.48) -- (109.77,262.20);

\path[draw=drawColor,line width= 0.6pt,line join=round] (109.77, 32.48) -- (109.77,262.20);

\path[draw=drawColor,line width= 0.6pt,line join=round] (109.77, 32.48) -- (109.77,262.20);

\path[draw=drawColor,line width= 0.6pt,line join=round] (109.77, 32.48) -- (109.77,262.20);

\path[draw=drawColor,line width= 0.6pt,line join=round] (109.77, 32.48) -- (109.77,262.20);

\path[draw=drawColor,line width= 0.6pt,line join=round] (109.77, 32.48) -- (109.77,262.20);

\path[draw=drawColor,line width= 0.6pt,line join=round] (109.77, 32.48) -- (109.77,262.20);

\path[draw=drawColor,line width= 0.6pt,line join=round] (109.77, 32.48) -- (109.77,262.20);

\path[draw=drawColor,line width= 0.6pt,line join=round] (109.77, 32.48) -- (109.77,262.20);

\path[draw=drawColor,line width= 0.6pt,line join=round] (109.77, 32.48) -- (109.77,262.20);

\path[draw=drawColor,line width= 0.6pt,line join=round] (109.77, 32.48) -- (109.77,262.20);

\path[draw=drawColor,line width= 0.6pt,line join=round] (109.77, 32.48) -- (109.77,262.20);

\path[draw=drawColor,line width= 0.6pt,line join=round] (109.77, 32.48) -- (109.77,262.20);

\path[draw=drawColor,line width= 0.6pt,line join=round] (109.77, 32.48) -- (109.77,262.20);

\path[draw=drawColor,line width= 0.6pt,line join=round] (109.77, 32.48) -- (109.77,262.20);

\path[draw=drawColor,line width= 0.6pt,line join=round] (109.77, 32.48) -- (109.77,262.20);

\path[draw=drawColor,line width= 0.6pt,line join=round] (109.77, 32.48) -- (109.77,262.20);

\path[draw=drawColor,line width= 0.6pt,line join=round] (109.77, 32.48) -- (109.77,262.20);

\path[draw=drawColor,line width= 0.6pt,line join=round] (109.77, 32.48) -- (109.77,262.20);

\path[draw=drawColor,line width= 0.6pt,line join=round] (109.77, 32.48) -- (109.77,262.20);

\path[draw=drawColor,line width= 0.6pt,line join=round] (109.77, 32.48) -- (109.77,262.20);

\path[draw=drawColor,line width= 0.6pt,line join=round] (109.77, 32.48) -- (109.77,262.20);

\path[draw=drawColor,line width= 0.6pt,line join=round] (109.77, 32.48) -- (109.77,262.20);

\path[draw=drawColor,line width= 0.6pt,line join=round] (109.77, 32.48) -- (109.77,262.20);

\path[draw=drawColor,line width= 0.6pt,line join=round] (109.77, 32.48) -- (109.77,262.20);

\path[draw=drawColor,line width= 0.6pt,line join=round] (109.77, 32.48) -- (109.77,262.20);

\path[draw=drawColor,line width= 0.6pt,line join=round] (109.77, 32.48) -- (109.77,262.20);

\path[draw=drawColor,line width= 0.6pt,line join=round] (109.77, 32.48) -- (109.77,262.20);

\path[draw=drawColor,line width= 0.6pt,line join=round] (109.77, 32.48) -- (109.77,262.20);

\path[draw=drawColor,line width= 0.6pt,line join=round] (109.77, 32.48) -- (109.77,262.20);

\path[draw=drawColor,line width= 0.6pt,line join=round] (109.77, 32.48) -- (109.77,262.20);

\path[draw=drawColor,line width= 0.6pt,line join=round] (109.77, 32.48) -- (109.77,262.20);

\path[draw=drawColor,line width= 0.6pt,line join=round] (109.77, 32.48) -- (109.77,262.20);

\path[draw=drawColor,line width= 0.6pt,line join=round] (109.77, 32.48) -- (109.77,262.20);

\path[draw=drawColor,line width= 0.6pt,line join=round] (109.77, 32.48) -- (109.77,262.20);

\path[draw=drawColor,line width= 0.6pt,line join=round] (109.77, 32.48) -- (109.77,262.20);

\path[draw=drawColor,line width= 0.6pt,line join=round] (109.77, 32.48) -- (109.77,262.20);

\path[draw=drawColor,line width= 0.6pt,line join=round] (109.77, 32.48) -- (109.77,262.20);

\path[draw=drawColor,line width= 0.6pt,line join=round] (109.77, 32.48) -- (109.77,262.20);

\path[draw=drawColor,line width= 0.6pt,line join=round] (109.77, 32.48) -- (109.77,262.20);

\path[draw=drawColor,line width= 0.6pt,line join=round] (109.77, 32.48) -- (109.77,262.20);

\path[draw=drawColor,line width= 0.6pt,line join=round] (109.77, 32.48) -- (109.77,262.20);

\path[draw=drawColor,line width= 0.6pt,line join=round] (109.77, 32.48) -- (109.77,262.20);

\path[draw=drawColor,line width= 0.6pt,line join=round] (109.77, 32.48) -- (109.77,262.20);

\path[draw=drawColor,line width= 0.6pt,line join=round] (109.77, 32.48) -- (109.77,262.20);

\path[draw=drawColor,line width= 0.6pt,line join=round] (109.77, 32.48) -- (109.77,262.20);

\path[draw=drawColor,line width= 0.6pt,line join=round] (109.77, 32.48) -- (109.77,262.20);

\path[draw=drawColor,line width= 0.6pt,line join=round] (109.77, 32.48) -- (109.77,262.20);

\path[draw=drawColor,line width= 0.6pt,line join=round] (109.77, 32.48) -- (109.77,262.20);

\path[draw=drawColor,line width= 0.6pt,line join=round] (109.77, 32.48) -- (109.77,262.20);

\path[draw=drawColor,line width= 0.6pt,line join=round] (109.77, 32.48) -- (109.77,262.20);

\path[draw=drawColor,line width= 0.6pt,line join=round] (109.77, 32.48) -- (109.77,262.20);

\path[draw=drawColor,line width= 0.6pt,line join=round] (109.77, 32.48) -- (109.77,262.20);

\path[draw=drawColor,line width= 0.6pt,line join=round] (109.77, 32.48) -- (109.77,262.20);

\path[draw=drawColor,line width= 0.6pt,line join=round] (109.77, 32.48) -- (109.77,262.20);

\path[draw=drawColor,line width= 0.6pt,line join=round] (109.77, 32.48) -- (109.77,262.20);

\path[draw=drawColor,line width= 0.6pt,line join=round] (109.77, 32.48) -- (109.77,262.20);

\path[draw=drawColor,line width= 0.6pt,line join=round] (109.77, 32.48) -- (109.77,262.20);

\path[draw=drawColor,line width= 0.6pt,line join=round] (109.77, 32.48) -- (109.77,262.20);

\path[draw=drawColor,line width= 0.6pt,line join=round] (109.77, 32.48) -- (109.77,262.20);

\path[draw=drawColor,line width= 0.6pt,line join=round] (109.77, 32.48) -- (109.77,262.20);

\path[draw=drawColor,line width= 0.6pt,line join=round] (109.77, 32.48) -- (109.77,262.20);

\path[draw=drawColor,line width= 0.6pt,line join=round] (109.77, 32.48) -- (109.77,262.20);

\path[draw=drawColor,line width= 0.6pt,line join=round] (109.77, 32.48) -- (109.77,262.20);

\path[draw=drawColor,line width= 0.6pt,line join=round] (109.77, 32.48) -- (109.77,262.20);

\path[draw=drawColor,line width= 0.6pt,line join=round] (109.77, 32.48) -- (109.77,262.20);

\path[draw=drawColor,line width= 0.6pt,line join=round] (109.77, 32.48) -- (109.77,262.20);

\path[draw=drawColor,line width= 0.6pt,line join=round] (109.77, 32.48) -- (109.77,262.20);

\path[draw=drawColor,line width= 0.6pt,line join=round] (109.77, 32.48) -- (109.77,262.20);

\path[draw=drawColor,line width= 0.6pt,line join=round] (109.77, 32.48) -- (109.77,262.20);

\path[draw=drawColor,line width= 0.6pt,line join=round] (109.77, 32.48) -- (109.77,262.20);

\path[draw=drawColor,line width= 0.6pt,line join=round] (109.77, 32.48) -- (109.77,262.20);

\path[draw=drawColor,line width= 0.6pt,line join=round] (109.77, 32.48) -- (109.77,262.20);

\path[draw=drawColor,line width= 0.6pt,line join=round] (109.77, 32.48) -- (109.77,262.20);

\path[draw=drawColor,line width= 0.6pt,line join=round] (109.77, 32.48) -- (109.77,262.20);

\path[draw=drawColor,line width= 0.6pt,line join=round] (109.77, 32.48) -- (109.77,262.20);

\path[draw=drawColor,line width= 0.6pt,line join=round] (109.77, 32.48) -- (109.77,262.20);

\path[draw=drawColor,line width= 0.6pt,line join=round] (109.77, 32.48) -- (109.77,262.20);

\path[draw=drawColor,line width= 0.6pt,line join=round] (109.77, 32.48) -- (109.77,262.20);

\path[draw=drawColor,line width= 0.6pt,line join=round] (109.77, 32.48) -- (109.77,262.20);

\path[draw=drawColor,line width= 0.6pt,line join=round] (109.77, 32.48) -- (109.77,262.20);

\path[draw=drawColor,line width= 0.6pt,line join=round] (109.77, 32.48) -- (109.77,262.20);

\path[draw=drawColor,line width= 0.6pt,line join=round] (109.77, 32.48) -- (109.77,262.20);

\path[draw=drawColor,line width= 0.6pt,line join=round] (109.77, 32.48) -- (109.77,262.20);

\path[draw=drawColor,line width= 0.6pt,line join=round] (109.77, 32.48) -- (109.77,262.20);

\path[draw=drawColor,line width= 0.6pt,line join=round] (109.77, 32.48) -- (109.77,262.20);

\path[draw=drawColor,line width= 0.6pt,line join=round] (109.77, 32.48) -- (109.77,262.20);

\path[draw=drawColor,line width= 0.6pt,line join=round] (109.77, 32.48) -- (109.77,262.20);

\path[draw=drawColor,line width= 0.6pt,line join=round] (109.77, 32.48) -- (109.77,262.20);

\path[draw=drawColor,line width= 0.6pt,line join=round] (109.77, 32.48) -- (109.77,262.20);

\path[draw=drawColor,line width= 0.6pt,line join=round] (109.77, 32.48) -- (109.77,262.20);

\path[draw=drawColor,line width= 0.6pt,line join=round] (109.77, 32.48) -- (109.77,262.20);

\path[draw=drawColor,line width= 0.6pt,line join=round] (109.77, 32.48) -- (109.77,262.20);

\path[draw=drawColor,line width= 0.6pt,line join=round] (109.77, 32.48) -- (109.77,262.20);

\path[draw=drawColor,line width= 0.6pt,line join=round] (109.77, 32.48) -- (109.77,262.20);

\path[draw=drawColor,line width= 0.6pt,line join=round] (109.77, 32.48) -- (109.77,262.20);

\path[draw=drawColor,line width= 0.6pt,line join=round] (109.77, 32.48) -- (109.77,262.20);

\path[draw=drawColor,line width= 0.6pt,line join=round] (109.77, 32.48) -- (109.77,262.20);

\path[draw=drawColor,line width= 0.6pt,line join=round] (109.77, 32.48) -- (109.77,262.20);

\path[draw=drawColor,line width= 0.6pt,line join=round] (109.77, 32.48) -- (109.77,262.20);

\path[draw=drawColor,line width= 0.6pt,line join=round] (109.77, 32.48) -- (109.77,262.20);

\path[draw=drawColor,line width= 0.6pt,line join=round] (109.77, 32.48) -- (109.77,262.20);

\path[draw=drawColor,line width= 0.6pt,line join=round] (109.77, 32.48) -- (109.77,262.20);

\path[draw=drawColor,line width= 0.6pt,line join=round] (109.77, 32.48) -- (109.77,262.20);

\path[draw=drawColor,line width= 0.6pt,line join=round] (109.77, 32.48) -- (109.77,262.20);

\path[draw=drawColor,line width= 0.6pt,line join=round] (109.77, 32.48) -- (109.77,262.20);

\path[draw=drawColor,line width= 0.6pt,line join=round] (109.77, 32.48) -- (109.77,262.20);

\path[draw=drawColor,line width= 0.6pt,line join=round] (109.77, 32.48) -- (109.77,262.20);

\path[draw=drawColor,line width= 0.6pt,line join=round] (109.77, 32.48) -- (109.77,262.20);

\path[draw=drawColor,line width= 0.6pt,line join=round] (109.77, 32.48) -- (109.77,262.20);

\path[draw=drawColor,line width= 0.6pt,line join=round] (109.77, 32.48) -- (109.77,262.20);

\path[draw=drawColor,line width= 0.6pt,line join=round] (109.77, 32.48) -- (109.77,262.20);

\path[draw=drawColor,line width= 0.6pt,line join=round] (109.77, 32.48) -- (109.77,262.20);

\path[draw=drawColor,line width= 0.6pt,line join=round] (109.77, 32.48) -- (109.77,262.20);

\path[draw=drawColor,line width= 0.6pt,line join=round] (109.77, 32.48) -- (109.77,262.20);

\path[draw=drawColor,line width= 0.6pt,line join=round] (109.77, 32.48) -- (109.77,262.20);

\path[draw=drawColor,line width= 0.6pt,line join=round] (109.77, 32.48) -- (109.77,262.20);

\path[draw=drawColor,line width= 0.6pt,line join=round] (109.77, 32.48) -- (109.77,262.20);

\path[draw=drawColor,line width= 0.6pt,line join=round] (109.77, 32.48) -- (109.77,262.20);

\path[draw=drawColor,line width= 0.6pt,line join=round] (109.77, 32.48) -- (109.77,262.20);

\path[draw=drawColor,line width= 0.6pt,line join=round] (109.77, 32.48) -- (109.77,262.20);

\path[draw=drawColor,line width= 0.6pt,line join=round] (109.77, 32.48) -- (109.77,262.20);

\path[draw=drawColor,line width= 0.6pt,line join=round] (109.77, 32.48) -- (109.77,262.20);

\path[draw=drawColor,line width= 0.6pt,line join=round] (109.77, 32.48) -- (109.77,262.20);

\path[draw=drawColor,line width= 0.6pt,line join=round] (109.77, 32.48) -- (109.77,262.20);

\path[draw=drawColor,line width= 0.6pt,line join=round] (109.77, 32.48) -- (109.77,262.20);

\path[draw=drawColor,line width= 0.6pt,line join=round] (109.77, 32.48) -- (109.77,262.20);

\path[draw=drawColor,line width= 0.6pt,line join=round] (109.77, 32.48) -- (109.77,262.20);

\path[draw=drawColor,line width= 0.6pt,line join=round] (109.77, 32.48) -- (109.77,262.20);

\path[draw=drawColor,line width= 0.6pt,line join=round] (109.77, 32.48) -- (109.77,262.20);

\path[draw=drawColor,line width= 0.6pt,line join=round] (109.77, 32.48) -- (109.77,262.20);

\path[draw=drawColor,line width= 0.6pt,line join=round] (109.77, 32.48) -- (109.77,262.20);

\path[draw=drawColor,line width= 0.6pt,line join=round] (109.77, 32.48) -- (109.77,262.20);

\path[draw=drawColor,line width= 0.6pt,line join=round] (109.77, 32.48) -- (109.77,262.20);

\path[draw=drawColor,line width= 0.6pt,line join=round] (109.77, 32.48) -- (109.77,262.20);

\path[draw=drawColor,line width= 0.6pt,line join=round] (109.77, 32.48) -- (109.77,262.20);

\path[draw=drawColor,line width= 0.6pt,line join=round] (109.77, 32.48) -- (109.77,262.20);

\path[draw=drawColor,line width= 0.6pt,line join=round] (109.77, 32.48) -- (109.77,262.20);

\path[draw=drawColor,line width= 0.6pt,line join=round] (109.77, 32.48) -- (109.77,262.20);

\path[draw=drawColor,line width= 0.6pt,line join=round] (109.77, 32.48) -- (109.77,262.20);

\path[draw=drawColor,line width= 0.6pt,line join=round] (109.77, 32.48) -- (109.77,262.20);

\path[draw=drawColor,line width= 0.6pt,line join=round] (109.77, 32.48) -- (109.77,262.20);

\path[draw=drawColor,line width= 0.6pt,line join=round] (109.77, 32.48) -- (109.77,262.20);

\path[draw=drawColor,line width= 0.6pt,line join=round] (109.77, 32.48) -- (109.77,262.20);

\path[draw=drawColor,line width= 0.6pt,line join=round] (109.77, 32.48) -- (109.77,262.20);

\path[draw=drawColor,line width= 0.6pt,line join=round] (109.77, 32.48) -- (109.77,262.20);

\path[draw=drawColor,line width= 0.6pt,line join=round] (109.77, 32.48) -- (109.77,262.20);

\path[draw=drawColor,line width= 0.6pt,line join=round] (109.77, 32.48) -- (109.77,262.20);

\path[draw=drawColor,line width= 0.6pt,line join=round] (109.77, 32.48) -- (109.77,262.20);

\path[draw=drawColor,line width= 0.6pt,line join=round] (109.77, 32.48) -- (109.77,262.20);

\path[draw=drawColor,line width= 0.6pt,line join=round] (109.77, 32.48) -- (109.77,262.20);

\path[draw=drawColor,line width= 0.6pt,line join=round] (109.77, 32.48) -- (109.77,262.20);

\path[draw=drawColor,line width= 0.6pt,line join=round] (109.77, 32.48) -- (109.77,262.20);

\path[draw=drawColor,line width= 0.6pt,line join=round] (109.77, 32.48) -- (109.77,262.20);

\path[draw=drawColor,line width= 0.6pt,line join=round] (109.77, 32.48) -- (109.77,262.20);

\path[draw=drawColor,line width= 0.6pt,line join=round] (109.77, 32.48) -- (109.77,262.20);

\path[draw=drawColor,line width= 0.6pt,line join=round] (109.77, 32.48) -- (109.77,262.20);

\path[draw=drawColor,line width= 0.6pt,line join=round] (109.77, 32.48) -- (109.77,262.20);

\path[draw=drawColor,line width= 0.6pt,line join=round] (109.77, 32.48) -- (109.77,262.20);

\path[draw=drawColor,line width= 0.6pt,line join=round] (109.77, 32.48) -- (109.77,262.20);

\path[draw=drawColor,line width= 0.6pt,line join=round] (109.77, 32.48) -- (109.77,262.20);

\path[draw=drawColor,line width= 0.6pt,line join=round] (109.77, 32.48) -- (109.77,262.20);

\path[draw=drawColor,line width= 0.6pt,line join=round] (109.77, 32.48) -- (109.77,262.20);

\path[draw=drawColor,line width= 0.6pt,line join=round] (109.77, 32.48) -- (109.77,262.20);

\path[draw=drawColor,line width= 0.6pt,line join=round] (109.77, 32.48) -- (109.77,262.20);

\path[draw=drawColor,line width= 0.6pt,line join=round] (109.77, 32.48) -- (109.77,262.20);

\path[draw=drawColor,line width= 0.6pt,line join=round] (109.77, 32.48) -- (109.77,262.20);

\path[draw=drawColor,line width= 0.6pt,line join=round] (109.77, 32.48) -- (109.77,262.20);

\path[draw=drawColor,line width= 0.6pt,line join=round] (109.77, 32.48) -- (109.77,262.20);

\path[draw=drawColor,line width= 0.6pt,line join=round] (109.77, 32.48) -- (109.77,262.20);

\path[draw=drawColor,line width= 0.6pt,line join=round] (109.77, 32.48) -- (109.77,262.20);

\path[draw=drawColor,line width= 0.6pt,line join=round] (109.77, 32.48) -- (109.77,262.20);

\path[draw=drawColor,line width= 0.6pt,line join=round] (109.77, 32.48) -- (109.77,262.20);

\path[draw=drawColor,line width= 0.6pt,line join=round] (109.77, 32.48) -- (109.77,262.20);

\path[draw=drawColor,line width= 0.6pt,line join=round] (109.77, 32.48) -- (109.77,262.20);

\path[draw=drawColor,line width= 0.6pt,line join=round] (109.77, 32.48) -- (109.77,262.20);

\path[draw=drawColor,line width= 0.6pt,line join=round] (109.77, 32.48) -- (109.77,262.20);

\path[draw=drawColor,line width= 0.6pt,line join=round] (109.77, 32.48) -- (109.77,262.20);

\path[draw=drawColor,line width= 0.6pt,line join=round] (109.77, 32.48) -- (109.77,262.20);

\path[draw=drawColor,line width= 0.6pt,line join=round] (109.77, 32.48) -- (109.77,262.20);

\path[draw=drawColor,line width= 0.6pt,line join=round] (109.77, 32.48) -- (109.77,262.20);

\path[draw=drawColor,line width= 0.6pt,line join=round] (109.77, 32.48) -- (109.77,262.20);

\path[draw=drawColor,line width= 0.6pt,line join=round] (109.77, 32.48) -- (109.77,262.20);

\path[draw=drawColor,line width= 0.6pt,line join=round] (109.77, 32.48) -- (109.77,262.20);

\path[draw=drawColor,line width= 0.6pt,line join=round] (109.77, 32.48) -- (109.77,262.20);

\path[draw=drawColor,line width= 0.6pt,line join=round] (109.77, 32.48) -- (109.77,262.20);

\path[draw=drawColor,line width= 0.6pt,line join=round] (109.77, 32.48) -- (109.77,262.20);

\path[draw=drawColor,line width= 0.6pt,line join=round] (109.77, 32.48) -- (109.77,262.20);

\path[draw=drawColor,line width= 0.6pt,line join=round] (109.77, 32.48) -- (109.77,262.20);

\path[draw=drawColor,line width= 0.6pt,line join=round] (109.77, 32.48) -- (109.77,262.20);

\path[draw=drawColor,line width= 0.6pt,line join=round] (109.77, 32.48) -- (109.77,262.20);

\path[draw=drawColor,line width= 0.6pt,line join=round] (109.77, 32.48) -- (109.77,262.20);

\path[draw=drawColor,line width= 0.6pt,line join=round] (109.77, 32.48) -- (109.77,262.20);

\path[draw=drawColor,line width= 0.6pt,line join=round] (109.77, 32.48) -- (109.77,262.20);

\path[draw=drawColor,line width= 0.6pt,line join=round] (109.77, 32.48) -- (109.77,262.20);

\path[draw=drawColor,line width= 0.6pt,line join=round] (109.77, 32.48) -- (109.77,262.20);

\path[draw=drawColor,line width= 0.6pt,line join=round] (109.77, 32.48) -- (109.77,262.20);

\path[draw=drawColor,line width= 0.6pt,line join=round] (109.77, 32.48) -- (109.77,262.20);

\path[draw=drawColor,line width= 0.6pt,line join=round] (109.77, 32.48) -- (109.77,262.20);

\path[draw=drawColor,line width= 0.6pt,line join=round] (109.77, 32.48) -- (109.77,262.20);

\path[draw=drawColor,line width= 0.6pt,line join=round] (109.77, 32.48) -- (109.77,262.20);

\path[draw=drawColor,line width= 0.6pt,line join=round] (109.77, 32.48) -- (109.77,262.20);

\path[draw=drawColor,line width= 0.6pt,line join=round] (109.77, 32.48) -- (109.77,262.20);

\path[draw=drawColor,line width= 0.6pt,line join=round] (109.77, 32.48) -- (109.77,262.20);

\path[draw=drawColor,line width= 0.6pt,line join=round] (109.77, 32.48) -- (109.77,262.20);

\path[draw=drawColor,line width= 0.6pt,line join=round] (109.77, 32.48) -- (109.77,262.20);

\path[draw=drawColor,line width= 0.6pt,line join=round] (109.77, 32.48) -- (109.77,262.20);

\path[draw=drawColor,line width= 0.6pt,line join=round] (109.77, 32.48) -- (109.77,262.20);

\path[draw=drawColor,line width= 0.6pt,line join=round] (109.77, 32.48) -- (109.77,262.20);

\path[draw=drawColor,line width= 0.6pt,line join=round] (109.77, 32.48) -- (109.77,262.20);

\path[draw=drawColor,line width= 0.6pt,line join=round] (109.77, 32.48) -- (109.77,262.20);

\path[draw=drawColor,line width= 0.6pt,line join=round] (109.77, 32.48) -- (109.77,262.20);

\path[draw=drawColor,line width= 0.6pt,line join=round] (109.77, 32.48) -- (109.77,262.20);

\path[draw=drawColor,line width= 0.6pt,line join=round] (109.77, 32.48) -- (109.77,262.20);

\path[draw=drawColor,line width= 0.6pt,line join=round] (109.77, 32.48) -- (109.77,262.20);

\path[draw=drawColor,line width= 0.6pt,line join=round] (109.77, 32.48) -- (109.77,262.20);

\path[draw=drawColor,line width= 0.6pt,line join=round] (109.77, 32.48) -- (109.77,262.20);

\path[draw=drawColor,line width= 0.6pt,line join=round] (109.77, 32.48) -- (109.77,262.20);

\path[draw=drawColor,line width= 0.6pt,line join=round] (109.77, 32.48) -- (109.77,262.20);

\path[draw=drawColor,line width= 0.6pt,line join=round] (109.77, 32.48) -- (109.77,262.20);

\path[draw=drawColor,line width= 0.6pt,line join=round] (109.77, 32.48) -- (109.77,262.20);

\path[draw=drawColor,line width= 0.6pt,line join=round] (109.77, 32.48) -- (109.77,262.20);

\path[draw=drawColor,line width= 0.6pt,line join=round] (109.77, 32.48) -- (109.77,262.20);

\path[draw=drawColor,line width= 0.6pt,line join=round] (109.77, 32.48) -- (109.77,262.20);

\path[draw=drawColor,line width= 0.6pt,line join=round] (109.77, 32.48) -- (109.77,262.20);

\path[draw=drawColor,line width= 0.6pt,line join=round] (109.77, 32.48) -- (109.77,262.20);

\path[draw=drawColor,line width= 0.6pt,line join=round] (109.77, 32.48) -- (109.77,262.20);

\path[draw=drawColor,line width= 0.6pt,line join=round] (109.77, 32.48) -- (109.77,262.20);

\path[draw=drawColor,line width= 0.6pt,line join=round] (109.77, 32.48) -- (109.77,262.20);

\path[draw=drawColor,line width= 0.6pt,line join=round] (109.77, 32.48) -- (109.77,262.20);

\path[draw=drawColor,line width= 0.6pt,line join=round] (109.77, 32.48) -- (109.77,262.20);

\path[draw=drawColor,line width= 0.6pt,line join=round] (109.77, 32.48) -- (109.77,262.20);

\path[draw=drawColor,line width= 0.6pt,line join=round] (109.77, 32.48) -- (109.77,262.20);

\path[draw=drawColor,line width= 0.6pt,line join=round] (109.77, 32.48) -- (109.77,262.20);

\path[draw=drawColor,line width= 0.6pt,line join=round] (109.77, 32.48) -- (109.77,262.20);

\path[draw=drawColor,line width= 0.6pt,line join=round] (109.77, 32.48) -- (109.77,262.20);

\path[draw=drawColor,line width= 0.6pt,line join=round] (109.77, 32.48) -- (109.77,262.20);

\path[draw=drawColor,line width= 0.6pt,line join=round] (109.77, 32.48) -- (109.77,262.20);

\path[draw=drawColor,line width= 0.6pt,line join=round] (109.77, 32.48) -- (109.77,262.20);

\path[draw=drawColor,line width= 0.6pt,line join=round] (109.77, 32.48) -- (109.77,262.20);

\path[draw=drawColor,line width= 0.6pt,line join=round] (109.77, 32.48) -- (109.77,262.20);

\path[draw=drawColor,line width= 0.6pt,line join=round] (109.77, 32.48) -- (109.77,262.20);

\path[draw=drawColor,line width= 0.6pt,line join=round] (109.77, 32.48) -- (109.77,262.20);

\path[draw=drawColor,line width= 0.6pt,line join=round] (109.77, 32.48) -- (109.77,262.20);

\path[draw=drawColor,line width= 0.6pt,line join=round] (109.77, 32.48) -- (109.77,262.20);

\path[draw=drawColor,line width= 0.6pt,line join=round] (109.77, 32.48) -- (109.77,262.20);

\path[draw=drawColor,line width= 0.6pt,line join=round] (109.77, 32.48) -- (109.77,262.20);

\path[draw=drawColor,line width= 0.6pt,line join=round] (109.77, 32.48) -- (109.77,262.20);

\path[draw=drawColor,line width= 0.6pt,line join=round] (109.77, 32.48) -- (109.77,262.20);

\path[draw=drawColor,line width= 0.6pt,line join=round] (109.77, 32.48) -- (109.77,262.20);

\path[draw=drawColor,line width= 0.6pt,line join=round] (109.77, 32.48) -- (109.77,262.20);

\path[draw=drawColor,line width= 0.6pt,line join=round] (109.77, 32.48) -- (109.77,262.20);

\path[draw=drawColor,line width= 0.6pt,line join=round] (109.77, 32.48) -- (109.77,262.20);

\path[draw=drawColor,line width= 0.6pt,line join=round] (109.77, 32.48) -- (109.77,262.20);

\path[draw=drawColor,line width= 0.6pt,line join=round] (109.77, 32.48) -- (109.77,262.20);

\path[draw=drawColor,line width= 0.6pt,line join=round] (109.77, 32.48) -- (109.77,262.20);

\path[draw=drawColor,line width= 0.6pt,line join=round] (109.77, 32.48) -- (109.77,262.20);

\path[draw=drawColor,line width= 0.6pt,line join=round] (109.77, 32.48) -- (109.77,262.20);

\path[draw=drawColor,line width= 0.6pt,line join=round] (109.77, 32.48) -- (109.77,262.20);

\path[draw=drawColor,line width= 0.6pt,line join=round] (109.77, 32.48) -- (109.77,262.20);

\path[draw=drawColor,line width= 0.6pt,line join=round] (109.77, 32.48) -- (109.77,262.20);

\path[draw=drawColor,line width= 0.6pt,line join=round] (109.77, 32.48) -- (109.77,262.20);

\path[draw=drawColor,line width= 0.6pt,line join=round] (109.77, 32.48) -- (109.77,262.20);

\path[draw=drawColor,line width= 0.6pt,line join=round] (109.77, 32.48) -- (109.77,262.20);

\path[draw=drawColor,line width= 0.6pt,line join=round] (109.77, 32.48) -- (109.77,262.20);

\path[draw=drawColor,line width= 0.6pt,line join=round] (109.77, 32.48) -- (109.77,262.20);

\path[draw=drawColor,line width= 0.6pt,line join=round] (109.77, 32.48) -- (109.77,262.20);

\path[draw=drawColor,line width= 0.6pt,line join=round] (109.77, 32.48) -- (109.77,262.20);

\path[draw=drawColor,line width= 0.6pt,line join=round] (109.77, 32.48) -- (109.77,262.20);

\path[draw=drawColor,line width= 0.6pt,line join=round] (109.77, 32.48) -- (109.77,262.20);

\path[draw=drawColor,line width= 0.6pt,line join=round] (109.77, 32.48) -- (109.77,262.20);

\path[draw=drawColor,line width= 0.6pt,line join=round] (109.77, 32.48) -- (109.77,262.20);

\path[draw=drawColor,line width= 0.6pt,line join=round] (109.77, 32.48) -- (109.77,262.20);

\path[draw=drawColor,line width= 0.6pt,line join=round] (109.77, 32.48) -- (109.77,262.20);

\path[draw=drawColor,line width= 0.6pt,line join=round] (109.77, 32.48) -- (109.77,262.20);

\path[draw=drawColor,line width= 0.6pt,line join=round] (109.77, 32.48) -- (109.77,262.20);

\path[draw=drawColor,line width= 0.6pt,line join=round] (109.77, 32.48) -- (109.77,262.20);

\path[draw=drawColor,line width= 0.6pt,line join=round] (109.77, 32.48) -- (109.77,262.20);

\path[draw=drawColor,line width= 0.6pt,line join=round] (109.77, 32.48) -- (109.77,262.20);

\path[draw=drawColor,line width= 0.6pt,line join=round] (109.77, 32.48) -- (109.77,262.20);

\path[draw=drawColor,line width= 0.6pt,line join=round] (109.77, 32.48) -- (109.77,262.20);

\path[draw=drawColor,line width= 0.6pt,line join=round] (109.77, 32.48) -- (109.77,262.20);

\path[draw=drawColor,line width= 0.6pt,line join=round] (109.77, 32.48) -- (109.77,262.20);

\path[draw=drawColor,line width= 0.6pt,line join=round] (109.77, 32.48) -- (109.77,262.20);

\path[draw=drawColor,line width= 0.6pt,line join=round] (109.77, 32.48) -- (109.77,262.20);

\path[draw=drawColor,line width= 0.6pt,line join=round] (109.77, 32.48) -- (109.77,262.20);

\path[draw=drawColor,line width= 0.6pt,line join=round] (109.77, 32.48) -- (109.77,262.20);

\path[draw=drawColor,line width= 0.6pt,line join=round] (109.77, 32.48) -- (109.77,262.20);

\path[draw=drawColor,line width= 0.6pt,line join=round] (109.77, 32.48) -- (109.77,262.20);

\path[draw=drawColor,line width= 0.6pt,line join=round] (109.77, 32.48) -- (109.77,262.20);

\path[draw=drawColor,line width= 0.6pt,line join=round] (109.77, 32.48) -- (109.77,262.20);

\path[draw=drawColor,line width= 0.6pt,line join=round] (109.77, 32.48) -- (109.77,262.20);

\path[draw=drawColor,line width= 0.6pt,line join=round] (109.77, 32.48) -- (109.77,262.20);

\path[draw=drawColor,line width= 0.6pt,line join=round] (109.77, 32.48) -- (109.77,262.20);

\path[draw=drawColor,line width= 0.6pt,line join=round] (109.77, 32.48) -- (109.77,262.20);

\path[draw=drawColor,line width= 0.6pt,line join=round] (109.77, 32.48) -- (109.77,262.20);

\path[draw=drawColor,line width= 0.6pt,line join=round] (109.77, 32.48) -- (109.77,262.20);

\path[draw=drawColor,line width= 0.6pt,line join=round] (109.77, 32.48) -- (109.77,262.20);

\path[draw=drawColor,line width= 0.6pt,line join=round] (109.77, 32.48) -- (109.77,262.20);

\path[draw=drawColor,line width= 0.6pt,line join=round] (109.77, 32.48) -- (109.77,262.20);

\path[draw=drawColor,line width= 0.6pt,line join=round] (109.77, 32.48) -- (109.77,262.20);

\path[draw=drawColor,line width= 0.6pt,line join=round] (109.77, 32.48) -- (109.77,262.20);

\path[draw=drawColor,line width= 0.6pt,line join=round] (109.77, 32.48) -- (109.77,262.20);

\path[draw=drawColor,line width= 0.6pt,line join=round] (109.77, 32.48) -- (109.77,262.20);

\path[draw=drawColor,line width= 0.6pt,line join=round] (109.77, 32.48) -- (109.77,262.20);

\path[draw=drawColor,line width= 0.6pt,line join=round] (109.77, 32.48) -- (109.77,262.20);

\path[draw=drawColor,line width= 0.6pt,line join=round] (109.77, 32.48) -- (109.77,262.20);

\path[draw=drawColor,line width= 0.6pt,line join=round] (109.77, 32.48) -- (109.77,262.20);

\path[draw=drawColor,line width= 0.6pt,line join=round] (109.77, 32.48) -- (109.77,262.20);

\path[draw=drawColor,line width= 0.6pt,line join=round] (109.77, 32.48) -- (109.77,262.20);

\path[draw=drawColor,line width= 0.6pt,line join=round] (109.77, 32.48) -- (109.77,262.20);

\path[draw=drawColor,line width= 0.6pt,line join=round] (109.77, 32.48) -- (109.77,262.20);

\path[draw=drawColor,line width= 0.6pt,line join=round] (109.77, 32.48) -- (109.77,262.20);

\path[draw=drawColor,line width= 0.6pt,line join=round] (109.77, 32.48) -- (109.77,262.20);

\path[draw=drawColor,line width= 0.6pt,line join=round] (109.77, 32.48) -- (109.77,262.20);

\path[draw=drawColor,line width= 0.6pt,line join=round] (109.77, 32.48) -- (109.77,262.20);

\path[draw=drawColor,line width= 0.6pt,line join=round] (109.77, 32.48) -- (109.77,262.20);

\path[draw=drawColor,line width= 0.6pt,line join=round] (109.77, 32.48) -- (109.77,262.20);

\path[draw=drawColor,line width= 0.6pt,line join=round] (109.77, 32.48) -- (109.77,262.20);

\path[draw=drawColor,line width= 0.6pt,line join=round] (109.77, 32.48) -- (109.77,262.20);

\path[draw=drawColor,line width= 0.6pt,line join=round] (109.77, 32.48) -- (109.77,262.20);

\path[draw=drawColor,line width= 0.6pt,line join=round] (109.77, 32.48) -- (109.77,262.20);

\path[draw=drawColor,line width= 0.6pt,line join=round] (109.77, 32.48) -- (109.77,262.20);

\path[draw=drawColor,line width= 0.6pt,line join=round] (109.77, 32.48) -- (109.77,262.20);

\path[draw=drawColor,line width= 0.6pt,line join=round] (109.77, 32.48) -- (109.77,262.20);

\path[draw=drawColor,line width= 0.6pt,line join=round] (109.77, 32.48) -- (109.77,262.20);

\path[draw=drawColor,line width= 0.6pt,line join=round] (109.77, 32.48) -- (109.77,262.20);

\path[draw=drawColor,line width= 0.6pt,line join=round] (109.77, 32.48) -- (109.77,262.20);

\path[draw=drawColor,line width= 0.6pt,line join=round] (109.77, 32.48) -- (109.77,262.20);

\path[draw=drawColor,line width= 0.6pt,line join=round] (109.77, 32.48) -- (109.77,262.20);

\path[draw=drawColor,line width= 0.6pt,line join=round] (109.77, 32.48) -- (109.77,262.20);

\path[draw=drawColor,line width= 0.6pt,line join=round] (109.77, 32.48) -- (109.77,262.20);

\path[draw=drawColor,line width= 0.6pt,line join=round] (109.77, 32.48) -- (109.77,262.20);

\path[draw=drawColor,line width= 0.6pt,line join=round] (109.77, 32.48) -- (109.77,262.20);

\path[draw=drawColor,line width= 0.6pt,line join=round] (109.77, 32.48) -- (109.77,262.20);

\path[draw=drawColor,line width= 0.6pt,line join=round] (109.77, 32.48) -- (109.77,262.20);

\path[draw=drawColor,line width= 0.6pt,line join=round] (109.77, 32.48) -- (109.77,262.20);

\path[draw=drawColor,line width= 0.6pt,line join=round] (109.77, 32.48) -- (109.77,262.20);

\path[draw=drawColor,line width= 0.6pt,line join=round] (109.77, 32.48) -- (109.77,262.20);

\path[draw=drawColor,line width= 0.6pt,line join=round] (109.77, 32.48) -- (109.77,262.20);

\path[draw=drawColor,line width= 0.6pt,line join=round] (109.77, 32.48) -- (109.77,262.20);

\path[draw=drawColor,line width= 0.6pt,line join=round] (109.77, 32.48) -- (109.77,262.20);

\path[draw=drawColor,line width= 0.6pt,line join=round] (109.77, 32.48) -- (109.77,262.20);

\path[draw=drawColor,line width= 0.6pt,line join=round] (109.77, 32.48) -- (109.77,262.20);

\path[draw=drawColor,line width= 0.6pt,line join=round] (109.77, 32.48) -- (109.77,262.20);

\path[draw=drawColor,line width= 0.6pt,line join=round] (109.77, 32.48) -- (109.77,262.20);

\path[draw=drawColor,line width= 0.6pt,line join=round] (109.77, 32.48) -- (109.77,262.20);

\path[draw=drawColor,line width= 0.6pt,line join=round] (109.77, 32.48) -- (109.77,262.20);

\path[draw=drawColor,line width= 0.6pt,line join=round] (109.77, 32.48) -- (109.77,262.20);

\path[draw=drawColor,line width= 0.6pt,line join=round] (109.77, 32.48) -- (109.77,262.20);

\path[draw=drawColor,line width= 0.6pt,line join=round] (109.77, 32.48) -- (109.77,262.20);

\path[draw=drawColor,line width= 0.6pt,line join=round] (109.77, 32.48) -- (109.77,262.20);

\path[draw=drawColor,line width= 0.6pt,line join=round] (109.77, 32.48) -- (109.77,262.20);

\path[draw=drawColor,line width= 0.6pt,line join=round] (109.77, 32.48) -- (109.77,262.20);

\path[draw=drawColor,line width= 0.6pt,line join=round] (109.77, 32.48) -- (109.77,262.20);

\path[draw=drawColor,line width= 0.6pt,line join=round] (109.77, 32.48) -- (109.77,262.20);

\path[draw=drawColor,line width= 0.6pt,line join=round] (109.77, 32.48) -- (109.77,262.20);

\path[draw=drawColor,line width= 0.6pt,line join=round] (109.77, 32.48) -- (109.77,262.20);

\path[draw=drawColor,line width= 0.6pt,line join=round] (109.77, 32.48) -- (109.77,262.20);

\path[draw=drawColor,line width= 0.6pt,line join=round] (109.77, 32.48) -- (109.77,262.20);

\path[draw=drawColor,line width= 0.6pt,line join=round] (109.77, 32.48) -- (109.77,262.20);

\path[draw=drawColor,line width= 0.6pt,line join=round] (109.77, 32.48) -- (109.77,262.20);

\path[draw=drawColor,line width= 0.6pt,line join=round] (109.77, 32.48) -- (109.77,262.20);

\path[draw=drawColor,line width= 0.6pt,line join=round] (109.77, 32.48) -- (109.77,262.20);

\path[draw=drawColor,line width= 0.6pt,line join=round] (109.77, 32.48) -- (109.77,262.20);

\path[draw=drawColor,line width= 0.6pt,line join=round] (109.77, 32.48) -- (109.77,262.20);

\path[draw=drawColor,line width= 0.6pt,line join=round] (109.77, 32.48) -- (109.77,262.20);

\path[draw=drawColor,line width= 0.6pt,line join=round] (109.77, 32.48) -- (109.77,262.20);

\path[draw=drawColor,line width= 0.6pt,line join=round] (109.77, 32.48) -- (109.77,262.20);

\path[draw=drawColor,line width= 0.6pt,line join=round] (109.77, 32.48) -- (109.77,262.20);

\path[draw=drawColor,line width= 0.6pt,line join=round] (109.77, 32.48) -- (109.77,262.20);

\path[draw=drawColor,line width= 0.6pt,line join=round] (109.77, 32.48) -- (109.77,262.20);

\path[draw=drawColor,line width= 0.6pt,line join=round] (109.77, 32.48) -- (109.77,262.20);

\path[draw=drawColor,line width= 0.6pt,line join=round] (109.77, 32.48) -- (109.77,262.20);

\path[draw=drawColor,line width= 0.6pt,line join=round] (109.77, 32.48) -- (109.77,262.20);

\path[draw=drawColor,line width= 0.6pt,line join=round] (109.77, 32.48) -- (109.77,262.20);

\path[draw=drawColor,line width= 0.6pt,line join=round] (109.77, 32.48) -- (109.77,262.20);

\path[draw=drawColor,line width= 0.6pt,line join=round] (109.77, 32.48) -- (109.77,262.20);

\path[draw=drawColor,line width= 0.6pt,line join=round] (109.77, 32.48) -- (109.77,262.20);

\path[draw=drawColor,line width= 0.6pt,line join=round] (109.77, 32.48) -- (109.77,262.20);

\path[draw=drawColor,line width= 0.6pt,line join=round] (109.77, 32.48) -- (109.77,262.20);

\path[draw=drawColor,line width= 0.6pt,line join=round] (109.77, 32.48) -- (109.77,262.20);

\path[draw=drawColor,line width= 0.6pt,line join=round] (109.77, 32.48) -- (109.77,262.20);

\path[draw=drawColor,line width= 0.6pt,line join=round] (109.77, 32.48) -- (109.77,262.20);

\path[draw=drawColor,line width= 0.6pt,line join=round] (109.77, 32.48) -- (109.77,262.20);

\path[draw=drawColor,line width= 0.6pt,line join=round] (109.77, 32.48) -- (109.77,262.20);

\path[draw=drawColor,line width= 0.6pt,line join=round] (109.77, 32.48) -- (109.77,262.20);

\path[draw=drawColor,line width= 0.6pt,line join=round] (109.77, 32.48) -- (109.77,262.20);

\path[draw=drawColor,line width= 0.6pt,line join=round] (109.77, 32.48) -- (109.77,262.20);

\path[draw=drawColor,line width= 0.6pt,line join=round] (109.77, 32.48) -- (109.77,262.20);

\path[draw=drawColor,line width= 0.6pt,line join=round] (109.77, 32.48) -- (109.77,262.20);

\path[draw=drawColor,line width= 0.6pt,line join=round] (109.77, 32.48) -- (109.77,262.20);

\path[draw=drawColor,line width= 0.6pt,line join=round] (109.77, 32.48) -- (109.77,262.20);

\path[draw=drawColor,line width= 0.6pt,line join=round] (109.77, 32.48) -- (109.77,262.20);

\path[draw=drawColor,line width= 0.6pt,line join=round] (109.77, 32.48) -- (109.77,262.20);

\path[draw=drawColor,line width= 0.6pt,line join=round] (109.77, 32.48) -- (109.77,262.20);

\path[draw=drawColor,line width= 0.6pt,line join=round] (109.77, 32.48) -- (109.77,262.20);

\path[draw=drawColor,line width= 0.6pt,line join=round] (109.77, 32.48) -- (109.77,262.20);

\path[draw=drawColor,line width= 0.6pt,line join=round] (109.77, 32.48) -- (109.77,262.20);

\path[draw=drawColor,line width= 0.6pt,line join=round] (109.77, 32.48) -- (109.77,262.20);

\path[draw=drawColor,line width= 0.6pt,line join=round] (109.77, 32.48) -- (109.77,262.20);

\path[draw=drawColor,line width= 0.6pt,line join=round] (109.77, 32.48) -- (109.77,262.20);

\path[draw=drawColor,line width= 0.6pt,line join=round] (109.77, 32.48) -- (109.77,262.20);

\path[draw=drawColor,line width= 0.6pt,line join=round] (109.77, 32.48) -- (109.77,262.20);

\path[draw=drawColor,line width= 0.6pt,line join=round] (109.77, 32.48) -- (109.77,262.20);

\path[draw=drawColor,line width= 0.6pt,line join=round] (109.77, 32.48) -- (109.77,262.20);

\path[draw=drawColor,line width= 0.6pt,line join=round] (109.77, 32.48) -- (109.77,262.20);

\path[draw=drawColor,line width= 0.6pt,line join=round] (109.77, 32.48) -- (109.77,262.20);

\path[draw=drawColor,line width= 0.6pt,line join=round] (109.77, 32.48) -- (109.77,262.20);

\path[draw=drawColor,line width= 0.6pt,line join=round] (109.77, 32.48) -- (109.77,262.20);

\path[draw=drawColor,line width= 0.6pt,line join=round] (109.77, 32.48) -- (109.77,262.20);

\path[draw=drawColor,line width= 0.6pt,line join=round] (109.77, 32.48) -- (109.77,262.20);

\path[draw=drawColor,line width= 0.6pt,line join=round] (109.77, 32.48) -- (109.77,262.20);

\path[draw=drawColor,line width= 0.6pt,line join=round] (109.77, 32.48) -- (109.77,262.20);

\path[draw=drawColor,line width= 0.6pt,line join=round] (109.77, 32.48) -- (109.77,262.20);

\path[draw=drawColor,line width= 0.6pt,line join=round] (109.77, 32.48) -- (109.77,262.20);

\path[draw=drawColor,line width= 0.6pt,line join=round] (109.77, 32.48) -- (109.77,262.20);

\path[draw=drawColor,line width= 0.6pt,line join=round] (109.77, 32.48) -- (109.77,262.20);

\path[draw=drawColor,line width= 0.6pt,line join=round] (109.77, 32.48) -- (109.77,262.20);

\path[draw=drawColor,line width= 0.6pt,line join=round] (109.77, 32.48) -- (109.77,262.20);

\path[draw=drawColor,line width= 0.6pt,line join=round] (109.77, 32.48) -- (109.77,262.20);

\path[draw=drawColor,line width= 0.6pt,line join=round] (109.77, 32.48) -- (109.77,262.20);

\path[draw=drawColor,line width= 0.6pt,line join=round] (109.77, 32.48) -- (109.77,262.20);

\path[draw=drawColor,line width= 0.6pt,line join=round] (109.77, 32.48) -- (109.77,262.20);

\path[draw=drawColor,line width= 0.6pt,line join=round] (109.77, 32.48) -- (109.77,262.20);

\path[draw=drawColor,line width= 0.6pt,line join=round] (109.77, 32.48) -- (109.77,262.20);

\path[draw=drawColor,line width= 0.6pt,line join=round] (109.77, 32.48) -- (109.77,262.20);

\path[draw=drawColor,line width= 0.6pt,line join=round] (109.77, 32.48) -- (109.77,262.20);

\path[draw=drawColor,line width= 0.6pt,line join=round] (109.77, 32.48) -- (109.77,262.20);

\path[draw=drawColor,line width= 0.6pt,line join=round] (109.77, 32.48) -- (109.77,262.20);

\path[draw=drawColor,line width= 0.6pt,line join=round] (109.77, 32.48) -- (109.77,262.20);

\path[draw=drawColor,line width= 0.6pt,line join=round] (109.77, 32.48) -- (109.77,262.20);

\path[draw=drawColor,line width= 0.6pt,line join=round] (109.77, 32.48) -- (109.77,262.20);

\path[draw=drawColor,line width= 0.6pt,line join=round] (109.77, 32.48) -- (109.77,262.20);

\path[draw=drawColor,line width= 0.6pt,line join=round] (109.77, 32.48) -- (109.77,262.20);

\path[draw=drawColor,line width= 0.6pt,line join=round] (109.77, 32.48) -- (109.77,262.20);

\path[draw=drawColor,line width= 0.6pt,line join=round] (109.77, 32.48) -- (109.77,262.20);

\path[draw=drawColor,line width= 0.6pt,line join=round] (109.77, 32.48) -- (109.77,262.20);

\path[draw=drawColor,line width= 0.6pt,line join=round] (109.77, 32.48) -- (109.77,262.20);

\path[draw=drawColor,line width= 0.6pt,line join=round] (109.77, 32.48) -- (109.77,262.20);

\path[draw=drawColor,line width= 0.6pt,line join=round] (109.77, 32.48) -- (109.77,262.20);

\path[draw=drawColor,line width= 0.6pt,line join=round] (109.77, 32.48) -- (109.77,262.20);

\path[draw=drawColor,line width= 0.6pt,line join=round] (109.77, 32.48) -- (109.77,262.20);

\path[draw=drawColor,line width= 0.6pt,line join=round] (109.77, 32.48) -- (109.77,262.20);

\path[draw=drawColor,line width= 0.6pt,line join=round] (109.77, 32.48) -- (109.77,262.20);

\path[draw=drawColor,line width= 0.6pt,line join=round] (109.77, 32.48) -- (109.77,262.20);

\path[draw=drawColor,line width= 0.6pt,line join=round] (109.77, 32.48) -- (109.77,262.20);

\path[draw=drawColor,line width= 0.6pt,line join=round] (109.77, 32.48) -- (109.77,262.20);

\path[draw=drawColor,line width= 0.6pt,line join=round] (109.77, 32.48) -- (109.77,262.20);

\path[draw=drawColor,line width= 0.6pt,line join=round] (109.77, 32.48) -- (109.77,262.20);

\path[draw=drawColor,line width= 0.6pt,line join=round] (109.77, 32.48) -- (109.77,262.20);

\path[draw=drawColor,line width= 0.6pt,line join=round] (109.77, 32.48) -- (109.77,262.20);

\path[draw=drawColor,line width= 0.6pt,line join=round] (109.77, 32.48) -- (109.77,262.20);

\path[draw=drawColor,line width= 0.6pt,line join=round] (109.77, 32.48) -- (109.77,262.20);

\path[draw=drawColor,line width= 0.6pt,line join=round] (109.77, 32.48) -- (109.77,262.20);

\path[draw=drawColor,line width= 0.6pt,line join=round] (109.77, 32.48) -- (109.77,262.20);

\path[draw=drawColor,line width= 0.6pt,line join=round] (109.77, 32.48) -- (109.77,262.20);

\path[draw=drawColor,line width= 0.6pt,line join=round] (109.77, 32.48) -- (109.77,262.20);

\path[draw=drawColor,line width= 0.6pt,line join=round] (109.77, 32.48) -- (109.77,262.20);

\path[draw=drawColor,line width= 0.6pt,line join=round] (109.77, 32.48) -- (109.77,262.20);

\path[draw=drawColor,line width= 0.6pt,line join=round] (109.77, 32.48) -- (109.77,262.20);

\path[draw=drawColor,line width= 0.6pt,line join=round] (109.77, 32.48) -- (109.77,262.20);

\path[draw=drawColor,line width= 0.6pt,line join=round] (109.77, 32.48) -- (109.77,262.20);

\path[draw=drawColor,line width= 0.6pt,line join=round] (109.77, 32.48) -- (109.77,262.20);

\path[draw=drawColor,line width= 0.6pt,line join=round] (109.77, 32.48) -- (109.77,262.20);

\path[draw=drawColor,line width= 0.6pt,line join=round] (109.77, 32.48) -- (109.77,262.20);

\path[draw=drawColor,line width= 0.6pt,line join=round] (109.77, 32.48) -- (109.77,262.20);

\path[draw=drawColor,line width= 0.6pt,line join=round] (109.77, 32.48) -- (109.77,262.20);

\path[draw=drawColor,line width= 0.6pt,line join=round] (109.77, 32.48) -- (109.77,262.20);

\path[draw=drawColor,line width= 0.6pt,line join=round] (109.77, 32.48) -- (109.77,262.20);

\path[draw=drawColor,line width= 0.6pt,line join=round] (109.77, 32.48) -- (109.77,262.20);

\path[draw=drawColor,line width= 0.6pt,line join=round] (109.77, 32.48) -- (109.77,262.20);

\path[draw=drawColor,line width= 0.6pt,line join=round] (109.77, 32.48) -- (109.77,262.20);

\path[draw=drawColor,line width= 0.6pt,line join=round] (109.77, 32.48) -- (109.77,262.20);

\path[draw=drawColor,line width= 0.6pt,line join=round] (109.77, 32.48) -- (109.77,262.20);

\path[draw=drawColor,line width= 0.6pt,line join=round] (109.77, 32.48) -- (109.77,262.20);

\path[draw=drawColor,line width= 0.6pt,line join=round] (109.77, 32.48) -- (109.77,262.20);

\path[draw=drawColor,line width= 0.6pt,line join=round] (109.77, 32.48) -- (109.77,262.20);

\path[draw=drawColor,line width= 0.6pt,line join=round] (109.77, 32.48) -- (109.77,262.20);

\path[draw=drawColor,line width= 0.6pt,line join=round] (109.77, 32.48) -- (109.77,262.20);

\path[draw=drawColor,line width= 0.6pt,line join=round] (109.77, 32.48) -- (109.77,262.20);

\path[draw=drawColor,line width= 0.6pt,line join=round] (109.77, 32.48) -- (109.77,262.20);

\path[draw=drawColor,line width= 0.6pt,line join=round] (109.77, 32.48) -- (109.77,262.20);

\path[draw=drawColor,line width= 0.6pt,line join=round] (109.77, 32.48) -- (109.77,262.20);

\path[draw=drawColor,line width= 0.6pt,line join=round] (109.77, 32.48) -- (109.77,262.20);

\path[draw=drawColor,line width= 0.6pt,line join=round] (109.77, 32.48) -- (109.77,262.20);

\path[draw=drawColor,line width= 0.6pt,line join=round] (109.77, 32.48) -- (109.77,262.20);

\path[draw=drawColor,line width= 0.6pt,line join=round] (109.77, 32.48) -- (109.77,262.20);

\path[draw=drawColor,line width= 0.6pt,line join=round] (109.77, 32.48) -- (109.77,262.20);

\path[draw=drawColor,line width= 0.6pt,line join=round] (109.77, 32.48) -- (109.77,262.20);

\path[draw=drawColor,line width= 0.6pt,line join=round] (109.77, 32.48) -- (109.77,262.20);

\path[draw=drawColor,line width= 0.6pt,line join=round] (109.77, 32.48) -- (109.77,262.20);

\path[draw=drawColor,line width= 0.6pt,line join=round] (109.77, 32.48) -- (109.77,262.20);

\path[draw=drawColor,line width= 0.6pt,line join=round] (109.77, 32.48) -- (109.77,262.20);

\path[draw=drawColor,line width= 0.6pt,line join=round] (109.77, 32.48) -- (109.77,262.20);

\path[draw=drawColor,line width= 0.6pt,line join=round] (109.77, 32.48) -- (109.77,262.20);

\path[draw=drawColor,line width= 0.6pt,line join=round] (109.77, 32.48) -- (109.77,262.20);

\path[draw=drawColor,line width= 0.6pt,line join=round] (109.77, 32.48) -- (109.77,262.20);

\path[draw=drawColor,line width= 0.6pt,line join=round] (109.77, 32.48) -- (109.77,262.20);

\path[draw=drawColor,line width= 0.6pt,line join=round] (109.77, 32.48) -- (109.77,262.20);

\path[draw=drawColor,line width= 0.6pt,line join=round] (109.77, 32.48) -- (109.77,262.20);

\path[draw=drawColor,line width= 0.6pt,line join=round] (109.77, 32.48) -- (109.77,262.20);

\path[draw=drawColor,line width= 0.6pt,line join=round] (109.77, 32.48) -- (109.77,262.20);

\path[draw=drawColor,line width= 0.6pt,line join=round] (109.77, 32.48) -- (109.77,262.20);

\path[draw=drawColor,line width= 0.6pt,line join=round] (109.77, 32.48) -- (109.77,262.20);

\path[draw=drawColor,line width= 0.6pt,line join=round] (109.77, 32.48) -- (109.77,262.20);

\path[draw=drawColor,line width= 0.6pt,line join=round] (109.77, 32.48) -- (109.77,262.20);

\path[draw=drawColor,line width= 0.6pt,line join=round] (109.77, 32.48) -- (109.77,262.20);

\path[draw=drawColor,line width= 0.6pt,line join=round] (109.77, 32.48) -- (109.77,262.20);

\path[draw=drawColor,line width= 0.6pt,line join=round] (109.77, 32.48) -- (109.77,262.20);

\path[draw=drawColor,line width= 0.6pt,line join=round] (109.77, 32.48) -- (109.77,262.20);

\path[draw=drawColor,line width= 0.6pt,line join=round] (109.77, 32.48) -- (109.77,262.20);

\path[draw=drawColor,line width= 0.6pt,line join=round] (109.77, 32.48) -- (109.77,262.20);

\path[draw=drawColor,line width= 0.6pt,line join=round] (109.77, 32.48) -- (109.77,262.20);

\path[draw=drawColor,line width= 0.6pt,line join=round] (109.77, 32.48) -- (109.77,262.20);

\path[draw=drawColor,line width= 0.6pt,line join=round] (109.77, 32.48) -- (109.77,262.20);

\path[draw=drawColor,line width= 0.6pt,line join=round] (109.77, 32.48) -- (109.77,262.20);

\path[draw=drawColor,line width= 0.6pt,line join=round] (109.77, 32.48) -- (109.77,262.20);

\path[draw=drawColor,line width= 0.6pt,line join=round] (109.77, 32.48) -- (109.77,262.20);

\path[draw=drawColor,line width= 0.6pt,line join=round] (109.77, 32.48) -- (109.77,262.20);

\path[draw=drawColor,line width= 0.6pt,line join=round] (109.77, 32.48) -- (109.77,262.20);

\path[draw=drawColor,line width= 0.6pt,line join=round] (109.77, 32.48) -- (109.77,262.20);

\path[draw=drawColor,line width= 0.6pt,line join=round] (109.77, 32.48) -- (109.77,262.20);

\path[draw=drawColor,line width= 0.6pt,line join=round] (109.77, 32.48) -- (109.77,262.20);

\path[draw=drawColor,line width= 0.6pt,line join=round] (109.77, 32.48) -- (109.77,262.20);

\path[draw=drawColor,line width= 0.6pt,line join=round] (109.77, 32.48) -- (109.77,262.20);

\path[draw=drawColor,line width= 0.6pt,line join=round] (109.77, 32.48) -- (109.77,262.20);

\path[draw=drawColor,line width= 0.6pt,line join=round] (109.77, 32.48) -- (109.77,262.20);

\path[draw=drawColor,line width= 0.6pt,line join=round] (109.77, 32.48) -- (109.77,262.20);

\path[draw=drawColor,line width= 0.6pt,line join=round] (109.77, 32.48) -- (109.77,262.20);

\path[draw=drawColor,line width= 0.6pt,line join=round] (109.77, 32.48) -- (109.77,262.20);

\path[draw=drawColor,line width= 0.6pt,line join=round] (109.77, 32.48) -- (109.77,262.20);

\path[draw=drawColor,line width= 0.6pt,line join=round] (109.77, 32.48) -- (109.77,262.20);

\path[draw=drawColor,line width= 0.6pt,line join=round] (109.77, 32.48) -- (109.77,262.20);

\path[draw=drawColor,line width= 0.6pt,line join=round] (109.77, 32.48) -- (109.77,262.20);

\path[draw=drawColor,line width= 0.6pt,line join=round] (109.77, 32.48) -- (109.77,262.20);

\path[draw=drawColor,line width= 0.6pt,line join=round] (109.77, 32.48) -- (109.77,262.20);

\path[draw=drawColor,line width= 0.6pt,line join=round] (109.77, 32.48) -- (109.77,262.20);

\path[draw=drawColor,line width= 0.6pt,line join=round] (109.77, 32.48) -- (109.77,262.20);

\path[draw=drawColor,line width= 0.6pt,line join=round] (109.77, 32.48) -- (109.77,262.20);

\path[draw=drawColor,line width= 0.6pt,line join=round] (109.77, 32.48) -- (109.77,262.20);

\path[draw=drawColor,line width= 0.6pt,line join=round] (109.77, 32.48) -- (109.77,262.20);

\path[draw=drawColor,line width= 0.6pt,line join=round] (109.77, 32.48) -- (109.77,262.20);

\path[draw=drawColor,line width= 0.6pt,line join=round] (109.77, 32.48) -- (109.77,262.20);

\path[draw=drawColor,line width= 0.6pt,line join=round] (109.77, 32.48) -- (109.77,262.20);

\path[draw=drawColor,line width= 0.6pt,line join=round] (109.77, 32.48) -- (109.77,262.20);

\path[draw=drawColor,line width= 0.6pt,line join=round] (109.77, 32.48) -- (109.77,262.20);

\path[draw=drawColor,line width= 0.6pt,line join=round] (109.77, 32.48) -- (109.77,262.20);

\path[draw=drawColor,line width= 0.6pt,line join=round] (109.77, 32.48) -- (109.77,262.20);

\path[draw=drawColor,line width= 0.6pt,line join=round] (109.77, 32.48) -- (109.77,262.20);

\path[draw=drawColor,line width= 0.6pt,line join=round] (109.77, 32.48) -- (109.77,262.20);

\path[draw=drawColor,line width= 0.6pt,line join=round] (109.77, 32.48) -- (109.77,262.20);

\path[draw=drawColor,line width= 0.6pt,line join=round] (109.77, 32.48) -- (109.77,262.20);

\path[draw=drawColor,line width= 0.6pt,line join=round] (109.77, 32.48) -- (109.77,262.20);

\path[draw=drawColor,line width= 0.6pt,line join=round] (109.77, 32.48) -- (109.77,262.20);

\path[draw=drawColor,line width= 0.6pt,line join=round] (109.77, 32.48) -- (109.77,262.20);

\path[draw=drawColor,line width= 0.6pt,line join=round] (109.77, 32.48) -- (109.77,262.20);

\path[draw=drawColor,line width= 0.6pt,line join=round] (109.77, 32.48) -- (109.77,262.20);

\path[draw=drawColor,line width= 0.6pt,line join=round] (109.77, 32.48) -- (109.77,262.20);

\path[draw=drawColor,line width= 0.6pt,line join=round] (109.77, 32.48) -- (109.77,262.20);

\path[draw=drawColor,line width= 0.6pt,line join=round] (109.77, 32.48) -- (109.77,262.20);

\path[draw=drawColor,line width= 0.6pt,line join=round] (109.77, 32.48) -- (109.77,262.20);

\path[draw=drawColor,line width= 0.6pt,line join=round] (109.77, 32.48) -- (109.77,262.20);

\path[draw=drawColor,line width= 0.6pt,line join=round] (109.77, 32.48) -- (109.77,262.20);

\path[draw=drawColor,line width= 0.6pt,line join=round] (109.77, 32.48) -- (109.77,262.20);

\path[draw=drawColor,line width= 0.6pt,line join=round] (109.77, 32.48) -- (109.77,262.20);

\path[draw=drawColor,line width= 0.6pt,line join=round] (109.77, 32.48) -- (109.77,262.20);

\path[draw=drawColor,line width= 0.6pt,line join=round] (109.77, 32.48) -- (109.77,262.20);

\path[draw=drawColor,line width= 0.6pt,line join=round] (109.77, 32.48) -- (109.77,262.20);

\path[draw=drawColor,line width= 0.6pt,line join=round] (109.77, 32.48) -- (109.77,262.20);

\path[draw=drawColor,line width= 0.6pt,line join=round] (109.77, 32.48) -- (109.77,262.20);

\path[draw=drawColor,line width= 0.6pt,line join=round] (109.77, 32.48) -- (109.77,262.20);

\path[draw=drawColor,line width= 0.6pt,line join=round] (109.77, 32.48) -- (109.77,262.20);

\path[draw=drawColor,line width= 0.6pt,line join=round] (109.77, 32.48) -- (109.77,262.20);

\path[draw=drawColor,line width= 0.6pt,line join=round] (109.77, 32.48) -- (109.77,262.20);

\path[draw=drawColor,line width= 0.6pt,line join=round] (109.77, 32.48) -- (109.77,262.20);

\path[draw=drawColor,line width= 0.6pt,line join=round] (109.77, 32.48) -- (109.77,262.20);

\path[draw=drawColor,line width= 0.6pt,line join=round] (109.77, 32.48) -- (109.77,262.20);

\path[draw=drawColor,line width= 0.6pt,line join=round] (109.77, 32.48) -- (109.77,262.20);

\path[draw=drawColor,line width= 0.6pt,line join=round] (109.77, 32.48) -- (109.77,262.20);

\path[draw=drawColor,line width= 0.6pt,line join=round] (109.77, 32.48) -- (109.77,262.20);

\path[draw=drawColor,line width= 0.6pt,line join=round] (109.77, 32.48) -- (109.77,262.20);

\path[draw=drawColor,line width= 0.6pt,line join=round] (109.77, 32.48) -- (109.77,262.20);

\path[draw=drawColor,line width= 0.6pt,line join=round] (109.77, 32.48) -- (109.77,262.20);

\path[draw=drawColor,line width= 0.6pt,line join=round] (109.77, 32.48) -- (109.77,262.20);

\path[draw=drawColor,line width= 0.6pt,line join=round] (109.77, 32.48) -- (109.77,262.20);

\path[draw=drawColor,line width= 0.6pt,line join=round] (109.77, 32.48) -- (109.77,262.20);

\path[draw=drawColor,line width= 0.6pt,line join=round] (109.77, 32.48) -- (109.77,262.20);

\path[draw=drawColor,line width= 0.6pt,line join=round] (109.77, 32.48) -- (109.77,262.20);

\path[draw=drawColor,line width= 0.6pt,line join=round] (109.77, 32.48) -- (109.77,262.20);

\path[draw=drawColor,line width= 0.6pt,line join=round] (109.77, 32.48) -- (109.77,262.20);

\path[draw=drawColor,line width= 0.6pt,line join=round] (109.77, 32.48) -- (109.77,262.20);

\path[draw=drawColor,line width= 0.6pt,line join=round] (109.77, 32.48) -- (109.77,262.20);

\path[draw=drawColor,line width= 0.6pt,line join=round] (109.77, 32.48) -- (109.77,262.20);

\path[draw=drawColor,line width= 0.6pt,line join=round] (109.77, 32.48) -- (109.77,262.20);

\path[draw=drawColor,line width= 0.6pt,line join=round] (109.77, 32.48) -- (109.77,262.20);

\path[draw=drawColor,line width= 0.6pt,line join=round] (109.77, 32.48) -- (109.77,262.20);

\path[draw=drawColor,line width= 0.6pt,line join=round] (109.77, 32.48) -- (109.77,262.20);

\path[draw=drawColor,line width= 0.6pt,line join=round] (109.77, 32.48) -- (109.77,262.20);

\path[draw=drawColor,line width= 0.6pt,line join=round] (109.77, 32.48) -- (109.77,262.20);

\path[draw=drawColor,line width= 0.6pt,line join=round] (109.77, 32.48) -- (109.77,262.20);

\path[draw=drawColor,line width= 0.6pt,line join=round] (109.77, 32.48) -- (109.77,262.20);

\path[draw=drawColor,line width= 0.6pt,line join=round] (109.77, 32.48) -- (109.77,262.20);

\path[draw=drawColor,line width= 0.6pt,line join=round] (109.77, 32.48) -- (109.77,262.20);

\path[draw=drawColor,line width= 0.6pt,line join=round] (109.77, 32.48) -- (109.77,262.20);

\path[draw=drawColor,line width= 0.6pt,line join=round] (109.77, 32.48) -- (109.77,262.20);

\path[draw=drawColor,line width= 0.6pt,line join=round] (109.77, 32.48) -- (109.77,262.20);

\path[draw=drawColor,line width= 0.6pt,line join=round] (109.77, 32.48) -- (109.77,262.20);

\path[draw=drawColor,line width= 0.6pt,line join=round] (109.77, 32.48) -- (109.77,262.20);

\path[draw=drawColor,line width= 0.6pt,line join=round] (109.77, 32.48) -- (109.77,262.20);

\path[draw=drawColor,line width= 0.6pt,line join=round] (109.77, 32.48) -- (109.77,262.20);

\path[draw=drawColor,line width= 0.6pt,line join=round] (109.77, 32.48) -- (109.77,262.20);

\path[draw=drawColor,line width= 0.6pt,line join=round] (109.77, 32.48) -- (109.77,262.20);

\path[draw=drawColor,line width= 0.6pt,line join=round] (109.77, 32.48) -- (109.77,262.20);

\path[draw=drawColor,line width= 0.6pt,line join=round] (109.77, 32.48) -- (109.77,262.20);

\path[draw=drawColor,line width= 0.6pt,line join=round] (109.77, 32.48) -- (109.77,262.20);

\path[draw=drawColor,line width= 0.6pt,line join=round] (109.77, 32.48) -- (109.77,262.20);

\path[draw=drawColor,line width= 0.6pt,line join=round] (109.77, 32.48) -- (109.77,262.20);

\path[draw=drawColor,line width= 0.6pt,line join=round] (109.77, 32.48) -- (109.77,262.20);

\path[draw=drawColor,line width= 0.6pt,line join=round] (109.77, 32.48) -- (109.77,262.20);

\path[draw=drawColor,line width= 0.6pt,line join=round] (109.77, 32.48) -- (109.77,262.20);

\path[draw=drawColor,line width= 0.6pt,line join=round] (109.77, 32.48) -- (109.77,262.20);

\path[draw=drawColor,line width= 0.6pt,line join=round] (109.77, 32.48) -- (109.77,262.20);

\path[draw=drawColor,line width= 0.6pt,line join=round] (109.77, 32.48) -- (109.77,262.20);

\path[draw=drawColor,line width= 0.6pt,line join=round] (109.77, 32.48) -- (109.77,262.20);

\path[draw=drawColor,line width= 0.6pt,line join=round] (109.77, 32.48) -- (109.77,262.20);

\path[draw=drawColor,line width= 0.6pt,line join=round] (109.77, 32.48) -- (109.77,262.20);

\path[draw=drawColor,line width= 0.6pt,line join=round] (109.77, 32.48) -- (109.77,262.20);

\path[draw=drawColor,line width= 0.6pt,line join=round] (109.77, 32.48) -- (109.77,262.20);

\path[draw=drawColor,line width= 0.6pt,line join=round] (109.77, 32.48) -- (109.77,262.20);

\path[draw=drawColor,line width= 0.6pt,line join=round] (109.77, 32.48) -- (109.77,262.20);

\path[draw=drawColor,line width= 0.6pt,line join=round] (109.77, 32.48) -- (109.77,262.20);

\path[draw=drawColor,line width= 0.6pt,line join=round] (109.77, 32.48) -- (109.77,262.20);

\path[draw=drawColor,line width= 0.6pt,line join=round] (109.77, 32.48) -- (109.77,262.20);

\path[draw=drawColor,line width= 0.6pt,line join=round] (109.77, 32.48) -- (109.77,262.20);

\path[draw=drawColor,line width= 0.6pt,line join=round] (109.77, 32.48) -- (109.77,262.20);

\path[draw=drawColor,line width= 0.6pt,line join=round] (109.77, 32.48) -- (109.77,262.20);

\path[draw=drawColor,line width= 0.6pt,line join=round] (109.77, 32.48) -- (109.77,262.20);

\path[draw=drawColor,line width= 0.6pt,line join=round] (109.77, 32.48) -- (109.77,262.20);

\path[draw=drawColor,line width= 0.6pt,line join=round] (109.77, 32.48) -- (109.77,262.20);

\path[draw=drawColor,line width= 0.6pt,line join=round] (109.77, 32.48) -- (109.77,262.20);

\path[draw=drawColor,line width= 0.6pt,line join=round] (109.77, 32.48) -- (109.77,262.20);

\path[draw=drawColor,line width= 0.6pt,line join=round] (109.77, 32.48) -- (109.77,262.20);

\path[draw=drawColor,line width= 0.6pt,line join=round] (109.77, 32.48) -- (109.77,262.20);

\path[draw=drawColor,line width= 0.6pt,line join=round] (109.77, 32.48) -- (109.77,262.20);

\path[draw=drawColor,line width= 0.6pt,line join=round] (109.77, 32.48) -- (109.77,262.20);

\path[draw=drawColor,line width= 0.6pt,line join=round] (109.77, 32.48) -- (109.77,262.20);

\path[draw=drawColor,line width= 0.6pt,line join=round] (109.77, 32.48) -- (109.77,262.20);

\path[draw=drawColor,line width= 0.6pt,line join=round] (109.77, 32.48) -- (109.77,262.20);

\path[draw=drawColor,line width= 0.6pt,line join=round] (109.77, 32.48) -- (109.77,262.20);

\path[draw=drawColor,line width= 0.6pt,line join=round] (109.77, 32.48) -- (109.77,262.20);

\path[draw=drawColor,line width= 0.6pt,line join=round] (109.77, 32.48) -- (109.77,262.20);

\path[draw=drawColor,line width= 0.6pt,line join=round] (109.77, 32.48) -- (109.77,262.20);

\path[draw=drawColor,line width= 0.6pt,line join=round] (109.77, 32.48) -- (109.77,262.20);

\path[draw=drawColor,line width= 0.6pt,line join=round] (109.77, 32.48) -- (109.77,262.20);

\path[draw=drawColor,line width= 0.6pt,line join=round] (109.77, 32.48) -- (109.77,262.20);

\path[draw=drawColor,line width= 0.6pt,line join=round] (109.77, 32.48) -- (109.77,262.20);

\path[draw=drawColor,line width= 0.6pt,line join=round] (109.77, 32.48) -- (109.77,262.20);

\path[draw=drawColor,line width= 0.6pt,line join=round] (109.77, 32.48) -- (109.77,262.20);

\path[draw=drawColor,line width= 0.6pt,line join=round] (109.77, 32.48) -- (109.77,262.20);

\path[draw=drawColor,line width= 0.6pt,line join=round] (109.77, 32.48) -- (109.77,262.20);

\path[draw=drawColor,line width= 0.6pt,line join=round] (109.77, 32.48) -- (109.77,262.20);

\path[draw=drawColor,line width= 0.6pt,line join=round] (109.77, 32.48) -- (109.77,262.20);

\path[draw=drawColor,line width= 0.6pt,line join=round] (109.77, 32.48) -- (109.77,262.20);

\path[draw=drawColor,line width= 0.6pt,line join=round] (109.77, 32.48) -- (109.77,262.20);

\path[draw=drawColor,line width= 0.6pt,line join=round] (109.77, 32.48) -- (109.77,262.20);

\path[draw=drawColor,line width= 0.6pt,line join=round] (109.77, 32.48) -- (109.77,262.20);

\path[draw=drawColor,line width= 0.6pt,line join=round] (109.77, 32.48) -- (109.77,262.20);

\path[draw=drawColor,line width= 0.6pt,line join=round] (109.77, 32.48) -- (109.77,262.20);

\path[draw=drawColor,line width= 0.6pt,line join=round] (109.77, 32.48) -- (109.77,262.20);

\path[draw=drawColor,line width= 0.6pt,line join=round] (109.77, 32.48) -- (109.77,262.20);

\path[draw=drawColor,line width= 0.6pt,line join=round] (109.77, 32.48) -- (109.77,262.20);

\path[draw=drawColor,line width= 0.6pt,line join=round] (109.77, 32.48) -- (109.77,262.20);

\path[draw=drawColor,line width= 0.6pt,line join=round] (109.77, 32.48) -- (109.77,262.20);

\path[draw=drawColor,line width= 0.6pt,line join=round] (109.77, 32.48) -- (109.77,262.20);

\path[draw=drawColor,line width= 0.6pt,line join=round] (109.77, 32.48) -- (109.77,262.20);

\path[draw=drawColor,line width= 0.6pt,line join=round] (109.77, 32.48) -- (109.77,262.20);

\path[draw=drawColor,line width= 0.6pt,line join=round] (109.77, 32.48) -- (109.77,262.20);

\path[draw=drawColor,line width= 0.6pt,line join=round] (109.77, 32.48) -- (109.77,262.20);

\path[draw=drawColor,line width= 0.6pt,line join=round] (109.77, 32.48) -- (109.77,262.20);

\path[draw=drawColor,line width= 0.6pt,line join=round] (109.77, 32.48) -- (109.77,262.20);

\path[draw=drawColor,line width= 0.6pt,line join=round] (109.77, 32.48) -- (109.77,262.20);

\path[draw=drawColor,line width= 0.6pt,line join=round] (109.77, 32.48) -- (109.77,262.20);

\path[draw=drawColor,line width= 0.6pt,line join=round] (109.77, 32.48) -- (109.77,262.20);

\path[draw=drawColor,line width= 0.6pt,line join=round] (109.77, 32.48) -- (109.77,262.20);

\path[draw=drawColor,line width= 0.6pt,line join=round] (109.77, 32.48) -- (109.77,262.20);

\path[draw=drawColor,line width= 0.6pt,line join=round] (109.77, 32.48) -- (109.77,262.20);

\path[draw=drawColor,line width= 0.6pt,line join=round] (109.77, 32.48) -- (109.77,262.20);

\path[draw=drawColor,line width= 0.6pt,line join=round] (109.77, 32.48) -- (109.77,262.20);

\path[draw=drawColor,line width= 0.6pt,line join=round] (109.77, 32.48) -- (109.77,262.20);

\path[draw=drawColor,line width= 0.6pt,line join=round] (109.77, 32.48) -- (109.77,262.20);

\path[draw=drawColor,line width= 0.6pt,line join=round] (109.77, 32.48) -- (109.77,262.20);

\path[draw=drawColor,line width= 0.6pt,line join=round] (109.77, 32.48) -- (109.77,262.20);

\path[draw=drawColor,line width= 0.6pt,line join=round] (109.77, 32.48) -- (109.77,262.20);

\path[draw=drawColor,line width= 0.6pt,line join=round] (109.77, 32.48) -- (109.77,262.20);

\path[draw=drawColor,line width= 0.6pt,line join=round] (109.77, 32.48) -- (109.77,262.20);

\path[draw=drawColor,line width= 0.6pt,line join=round] (109.77, 32.48) -- (109.77,262.20);

\path[draw=drawColor,line width= 0.6pt,line join=round] (109.77, 32.48) -- (109.77,262.20);

\path[draw=drawColor,line width= 0.6pt,line join=round] (109.77, 32.48) -- (109.77,262.20);

\path[draw=drawColor,line width= 0.6pt,line join=round] (109.77, 32.48) -- (109.77,262.20);

\path[draw=drawColor,line width= 0.6pt,line join=round] (109.77, 32.48) -- (109.77,262.20);

\path[draw=drawColor,line width= 0.6pt,line join=round] (109.77, 32.48) -- (109.77,262.20);

\path[draw=drawColor,line width= 0.6pt,line join=round] (109.77, 32.48) -- (109.77,262.20);

\path[draw=drawColor,line width= 0.6pt,line join=round] (109.77, 32.48) -- (109.77,262.20);

\path[draw=drawColor,line width= 0.6pt,line join=round] (109.77, 32.48) -- (109.77,262.20);

\path[draw=drawColor,line width= 0.6pt,line join=round] (109.77, 32.48) -- (109.77,262.20);

\path[draw=drawColor,line width= 0.6pt,line join=round] (109.77, 32.48) -- (109.77,262.20);

\path[draw=drawColor,line width= 0.6pt,line join=round] (109.77, 32.48) -- (109.77,262.20);

\path[draw=drawColor,line width= 0.6pt,line join=round] (109.77, 32.48) -- (109.77,262.20);

\path[draw=drawColor,line width= 0.6pt,line join=round] (109.77, 32.48) -- (109.77,262.20);

\path[draw=drawColor,line width= 0.6pt,line join=round] (109.77, 32.48) -- (109.77,262.20);

\path[draw=drawColor,line width= 0.6pt,line join=round] (109.77, 32.48) -- (109.77,262.20);

\path[draw=drawColor,line width= 0.6pt,line join=round] (109.77, 32.48) -- (109.77,262.20);

\path[draw=drawColor,line width= 0.6pt,line join=round] (109.77, 32.48) -- (109.77,262.20);

\path[draw=drawColor,line width= 0.6pt,line join=round] (109.77, 32.48) -- (109.77,262.20);

\path[draw=drawColor,line width= 0.6pt,line join=round] (109.77, 32.48) -- (109.77,262.20);

\path[draw=drawColor,line width= 0.6pt,line join=round] (109.77, 32.48) -- (109.77,262.20);

\path[draw=drawColor,line width= 0.6pt,line join=round] (109.77, 32.48) -- (109.77,262.20);

\path[draw=drawColor,line width= 0.6pt,line join=round] (109.77, 32.48) -- (109.77,262.20);

\path[draw=drawColor,line width= 0.6pt,line join=round] (109.77, 32.48) -- (109.77,262.20);

\path[draw=drawColor,line width= 0.6pt,line join=round] (109.77, 32.48) -- (109.77,262.20);

\path[draw=drawColor,line width= 0.6pt,line join=round] (109.77, 32.48) -- (109.77,262.20);

\path[draw=drawColor,line width= 0.6pt,line join=round] (109.77, 32.48) -- (109.77,262.20);

\path[draw=drawColor,line width= 0.6pt,line join=round] (109.77, 32.48) -- (109.77,262.20);

\path[draw=drawColor,line width= 0.6pt,line join=round] (109.77, 32.48) -- (109.77,262.20);

\path[draw=drawColor,line width= 0.6pt,line join=round] (109.77, 32.48) -- (109.77,262.20);

\path[draw=drawColor,line width= 0.6pt,line join=round] (109.77, 32.48) -- (109.77,262.20);

\path[draw=drawColor,line width= 0.6pt,line join=round] (109.77, 32.48) -- (109.77,262.20);

\path[draw=drawColor,line width= 0.6pt,line join=round] (109.77, 32.48) -- (109.77,262.20);

\path[draw=drawColor,line width= 0.6pt,line join=round] (109.77, 32.48) -- (109.77,262.20);

\path[draw=drawColor,line width= 0.6pt,line join=round] (109.77, 32.48) -- (109.77,262.20);

\path[draw=drawColor,line width= 0.6pt,line join=round] (109.77, 32.48) -- (109.77,262.20);

\path[draw=drawColor,line width= 0.6pt,line join=round] (109.77, 32.48) -- (109.77,262.20);

\path[draw=drawColor,line width= 0.6pt,line join=round] (109.77, 32.48) -- (109.77,262.20);

\path[draw=drawColor,line width= 0.6pt,line join=round] (109.77, 32.48) -- (109.77,262.20);

\path[draw=drawColor,line width= 0.6pt,line join=round] (109.77, 32.48) -- (109.77,262.20);

\path[draw=drawColor,line width= 0.6pt,line join=round] (109.77, 32.48) -- (109.77,262.20);

\path[draw=drawColor,line width= 0.6pt,line join=round] (109.77, 32.48) -- (109.77,262.20);

\path[draw=drawColor,line width= 0.6pt,line join=round] (109.77, 32.48) -- (109.77,262.20);

\path[draw=drawColor,line width= 0.6pt,line join=round] (109.77, 32.48) -- (109.77,262.20);

\path[draw=drawColor,line width= 0.6pt,line join=round] (109.77, 32.48) -- (109.77,262.20);

\path[draw=drawColor,line width= 0.6pt,line join=round] (109.77, 32.48) -- (109.77,262.20);

\path[draw=drawColor,line width= 0.6pt,line join=round] (109.77, 32.48) -- (109.77,262.20);

\path[draw=drawColor,line width= 0.6pt,line join=round] (109.77, 32.48) -- (109.77,262.20);

\path[draw=drawColor,line width= 0.6pt,line join=round] (109.77, 32.48) -- (109.77,262.20);

\path[draw=drawColor,line width= 0.6pt,line join=round] (109.77, 32.48) -- (109.77,262.20);

\path[draw=drawColor,line width= 0.6pt,line join=round] (109.77, 32.48) -- (109.77,262.20);

\path[draw=drawColor,line width= 0.6pt,line join=round] (109.77, 32.48) -- (109.77,262.20);

\path[draw=drawColor,line width= 0.6pt,line join=round] (109.77, 32.48) -- (109.77,262.20);

\path[draw=drawColor,line width= 0.6pt,line join=round] (109.77, 32.48) -- (109.77,262.20);

\path[draw=drawColor,line width= 0.6pt,line join=round] (109.77, 32.48) -- (109.77,262.20);

\path[draw=drawColor,line width= 0.6pt,line join=round] (109.77, 32.48) -- (109.77,262.20);

\path[draw=drawColor,line width= 0.6pt,line join=round] (109.77, 32.48) -- (109.77,262.20);

\path[draw=drawColor,line width= 0.6pt,line join=round] (109.77, 32.48) -- (109.77,262.20);

\path[draw=drawColor,line width= 0.6pt,line join=round] (109.77, 32.48) -- (109.77,262.20);

\path[draw=drawColor,line width= 0.6pt,line join=round] (109.77, 32.48) -- (109.77,262.20);

\path[draw=drawColor,line width= 0.6pt,line join=round] (109.77, 32.48) -- (109.77,262.20);

\path[draw=drawColor,line width= 0.6pt,line join=round] (109.77, 32.48) -- (109.77,262.20);

\path[draw=drawColor,line width= 0.6pt,line join=round] (109.77, 32.48) -- (109.77,262.20);

\path[draw=drawColor,line width= 0.6pt,line join=round] (109.77, 32.48) -- (109.77,262.20);

\path[draw=drawColor,line width= 0.6pt,line join=round] (109.77, 32.48) -- (109.77,262.20);

\path[draw=drawColor,line width= 0.6pt,line join=round] (109.77, 32.48) -- (109.77,262.20);

\path[draw=drawColor,line width= 0.6pt,line join=round] (109.77, 32.48) -- (109.77,262.20);

\path[draw=drawColor,line width= 0.6pt,line join=round] (109.77, 32.48) -- (109.77,262.20);

\path[draw=drawColor,line width= 0.6pt,line join=round] (109.77, 32.48) -- (109.77,262.20);

\path[draw=drawColor,line width= 0.6pt,line join=round] (109.77, 32.48) -- (109.77,262.20);

\path[draw=drawColor,line width= 0.6pt,line join=round] (109.77, 32.48) -- (109.77,262.20);

\path[draw=drawColor,line width= 0.6pt,line join=round] (109.77, 32.48) -- (109.77,262.20);

\path[draw=drawColor,line width= 0.6pt,line join=round] (109.77, 32.48) -- (109.77,262.20);

\path[draw=drawColor,line width= 0.6pt,line join=round] (109.77, 32.48) -- (109.77,262.20);

\path[draw=drawColor,line width= 0.6pt,line join=round] (109.77, 32.48) -- (109.77,262.20);

\path[draw=drawColor,line width= 0.6pt,line join=round] (109.77, 32.48) -- (109.77,262.20);

\path[draw=drawColor,line width= 0.6pt,line join=round] (109.77, 32.48) -- (109.77,262.20);

\path[draw=drawColor,line width= 0.6pt,line join=round] (109.77, 32.48) -- (109.77,262.20);

\path[draw=drawColor,line width= 0.6pt,line join=round] (109.77, 32.48) -- (109.77,262.20);

\path[draw=drawColor,line width= 0.6pt,line join=round] (109.77, 32.48) -- (109.77,262.20);

\path[draw=drawColor,line width= 0.6pt,line join=round] (109.77, 32.48) -- (109.77,262.20);

\path[draw=drawColor,line width= 0.6pt,line join=round] (109.77, 32.48) -- (109.77,262.20);

\path[draw=drawColor,line width= 0.6pt,line join=round] (109.77, 32.48) -- (109.77,262.20);

\path[draw=drawColor,line width= 0.6pt,line join=round] (109.77, 32.48) -- (109.77,262.20);

\path[draw=drawColor,line width= 0.6pt,line join=round] (109.77, 32.48) -- (109.77,262.20);

\path[draw=drawColor,line width= 0.6pt,line join=round] (109.77, 32.48) -- (109.77,262.20);

\path[draw=drawColor,line width= 0.6pt,line join=round] (109.77, 32.48) -- (109.77,262.20);

\path[draw=drawColor,line width= 0.6pt,line join=round] (109.77, 32.48) -- (109.77,262.20);

\path[draw=drawColor,line width= 0.6pt,line join=round] (109.77, 32.48) -- (109.77,262.20);

\path[draw=drawColor,line width= 0.6pt,line join=round] (109.77, 32.48) -- (109.77,262.20);

\path[draw=drawColor,line width= 0.6pt,line join=round] (109.77, 32.48) -- (109.77,262.20);

\path[draw=drawColor,line width= 0.6pt,line join=round] (109.77, 32.48) -- (109.77,262.20);

\path[draw=drawColor,line width= 0.6pt,line join=round] (109.77, 32.48) -- (109.77,262.20);

\path[draw=drawColor,line width= 0.6pt,line join=round] (109.77, 32.48) -- (109.77,262.20);

\path[draw=drawColor,line width= 0.6pt,line join=round] (109.77, 32.48) -- (109.77,262.20);

\path[draw=drawColor,line width= 0.6pt,line join=round] (109.77, 32.48) -- (109.77,262.20);

\path[draw=drawColor,line width= 0.6pt,line join=round] (109.77, 32.48) -- (109.77,262.20);

\path[draw=drawColor,line width= 0.6pt,line join=round] (109.77, 32.48) -- (109.77,262.20);

\path[draw=drawColor,line width= 0.6pt,line join=round] (109.77, 32.48) -- (109.77,262.20);

\path[draw=drawColor,line width= 0.6pt,line join=round] (109.77, 32.48) -- (109.77,262.20);

\path[draw=drawColor,line width= 0.6pt,line join=round] (109.77, 32.48) -- (109.77,262.20);

\path[draw=drawColor,line width= 0.6pt,line join=round] (109.77, 32.48) -- (109.77,262.20);

\path[draw=drawColor,line width= 0.6pt,line join=round] (109.77, 32.48) -- (109.77,262.20);

\path[draw=drawColor,line width= 0.6pt,line join=round] (109.77, 32.48) -- (109.77,262.20);

\path[draw=drawColor,line width= 0.6pt,line join=round] (109.77, 32.48) -- (109.77,262.20);

\path[draw=drawColor,line width= 0.6pt,line join=round] (109.77, 32.48) -- (109.77,262.20);

\path[draw=drawColor,line width= 0.6pt,line join=round] (109.77, 32.48) -- (109.77,262.20);

\path[draw=drawColor,line width= 0.6pt,line join=round] (109.77, 32.48) -- (109.77,262.20);

\path[draw=drawColor,line width= 0.6pt,line join=round] (109.77, 32.48) -- (109.77,262.20);

\path[draw=drawColor,line width= 0.6pt,line join=round] (109.77, 32.48) -- (109.77,262.20);

\path[draw=drawColor,line width= 0.6pt,line join=round] (109.77, 32.48) -- (109.77,262.20);

\path[draw=drawColor,line width= 0.6pt,line join=round] (109.77, 32.48) -- (109.77,262.20);

\path[draw=drawColor,line width= 0.6pt,line join=round] (109.77, 32.48) -- (109.77,262.20);

\path[draw=drawColor,line width= 0.6pt,line join=round] (109.77, 32.48) -- (109.77,262.20);

\path[draw=drawColor,line width= 0.6pt,line join=round] (109.77, 32.48) -- (109.77,262.20);

\path[draw=drawColor,line width= 0.6pt,line join=round] (109.77, 32.48) -- (109.77,262.20);

\path[draw=drawColor,line width= 0.6pt,line join=round] (109.77, 32.48) -- (109.77,262.20);

\path[draw=drawColor,line width= 0.6pt,line join=round] (109.77, 32.48) -- (109.77,262.20);

\path[draw=drawColor,line width= 0.6pt,line join=round] (109.77, 32.48) -- (109.77,262.20);

\path[draw=drawColor,line width= 0.6pt,line join=round] (109.77, 32.48) -- (109.77,262.20);

\path[draw=drawColor,line width= 0.6pt,line join=round] (109.77, 32.48) -- (109.77,262.20);

\path[draw=drawColor,line width= 0.6pt,line join=round] (109.77, 32.48) -- (109.77,262.20);

\path[draw=drawColor,line width= 0.6pt,line join=round] (109.77, 32.48) -- (109.77,262.20);

\path[draw=drawColor,line width= 0.6pt,line join=round] (109.77, 32.48) -- (109.77,262.20);

\path[draw=drawColor,line width= 0.6pt,line join=round] (109.77, 32.48) -- (109.77,262.20);

\path[draw=drawColor,line width= 0.6pt,line join=round] (109.77, 32.48) -- (109.77,262.20);

\path[draw=drawColor,line width= 0.6pt,line join=round] (109.77, 32.48) -- (109.77,262.20);

\path[draw=drawColor,line width= 0.6pt,line join=round] (109.77, 32.48) -- (109.77,262.20);

\path[draw=drawColor,line width= 0.6pt,line join=round] (109.77, 32.48) -- (109.77,262.20);

\path[draw=drawColor,line width= 0.6pt,line join=round] (109.77, 32.48) -- (109.77,262.20);

\path[draw=drawColor,line width= 0.6pt,line join=round] (109.77, 32.48) -- (109.77,262.20);

\path[draw=drawColor,line width= 0.6pt,line join=round] (109.77, 32.48) -- (109.77,262.20);

\path[draw=drawColor,line width= 0.6pt,line join=round] (109.77, 32.48) -- (109.77,262.20);

\path[draw=drawColor,line width= 0.6pt,line join=round] (109.77, 32.48) -- (109.77,262.20);

\path[draw=drawColor,line width= 0.6pt,line join=round] (109.77, 32.48) -- (109.77,262.20);

\path[draw=drawColor,line width= 0.6pt,line join=round] (109.77, 32.48) -- (109.77,262.20);

\path[draw=drawColor,line width= 0.6pt,line join=round] (109.77, 32.48) -- (109.77,262.20);

\path[draw=drawColor,line width= 0.6pt,line join=round] (109.77, 32.48) -- (109.77,262.20);

\path[draw=drawColor,line width= 0.6pt,line join=round] (109.77, 32.48) -- (109.77,262.20);

\path[draw=drawColor,line width= 0.6pt,line join=round] (109.77, 32.48) -- (109.77,262.20);

\path[draw=drawColor,line width= 0.6pt,line join=round] (109.77, 32.48) -- (109.77,262.20);

\path[draw=drawColor,line width= 0.6pt,line join=round] (109.77, 32.48) -- (109.77,262.20);

\path[draw=drawColor,line width= 0.6pt,line join=round] (109.77, 32.48) -- (109.77,262.20);

\path[draw=drawColor,line width= 0.6pt,line join=round] (109.77, 32.48) -- (109.77,262.20);

\path[draw=drawColor,line width= 0.6pt,line join=round] (109.77, 32.48) -- (109.77,262.20);

\path[draw=drawColor,line width= 0.6pt,line join=round] (109.77, 32.48) -- (109.77,262.20);

\path[draw=drawColor,line width= 0.6pt,line join=round] (109.77, 32.48) -- (109.77,262.20);

\path[draw=drawColor,line width= 0.6pt,line join=round] (109.77, 32.48) -- (109.77,262.20);

\path[draw=drawColor,line width= 0.6pt,line join=round] (109.77, 32.48) -- (109.77,262.20);

\path[draw=drawColor,line width= 0.6pt,line join=round] (109.77, 32.48) -- (109.77,262.20);

\path[draw=drawColor,line width= 0.6pt,line join=round] (109.77, 32.48) -- (109.77,262.20);

\path[draw=drawColor,line width= 0.6pt,line join=round] (109.77, 32.48) -- (109.77,262.20);

\path[draw=drawColor,line width= 0.6pt,line join=round] (109.77, 32.48) -- (109.77,262.20);

\path[draw=drawColor,line width= 0.6pt,line join=round] (109.77, 32.48) -- (109.77,262.20);

\path[draw=drawColor,line width= 0.6pt,line join=round] (109.77, 32.48) -- (109.77,262.20);

\path[draw=drawColor,line width= 0.6pt,line join=round] (109.77, 32.48) -- (109.77,262.20);

\path[draw=drawColor,line width= 0.6pt,line join=round] (109.77, 32.48) -- (109.77,262.20);

\path[draw=drawColor,line width= 0.6pt,line join=round] (109.77, 32.48) -- (109.77,262.20);

\path[draw=drawColor,line width= 0.6pt,line join=round] (109.77, 32.48) -- (109.77,262.20);

\path[draw=drawColor,line width= 0.6pt,line join=round] (109.77, 32.48) -- (109.77,262.20);

\path[draw=drawColor,line width= 0.6pt,line join=round] (109.77, 32.48) -- (109.77,262.20);

\path[draw=drawColor,line width= 0.6pt,line join=round] (109.77, 32.48) -- (109.77,262.20);

\path[draw=drawColor,line width= 0.6pt,line join=round] (109.77, 32.48) -- (109.77,262.20);

\path[draw=drawColor,line width= 0.6pt,line join=round] (109.77, 32.48) -- (109.77,262.20);

\path[draw=drawColor,line width= 0.6pt,line join=round] (109.77, 32.48) -- (109.77,262.20);

\path[draw=drawColor,line width= 0.6pt,line join=round] (109.77, 32.48) -- (109.77,262.20);

\path[draw=drawColor,line width= 0.6pt,line join=round] (109.77, 32.48) -- (109.77,262.20);

\path[draw=drawColor,line width= 0.6pt,line join=round] (109.77, 32.48) -- (109.77,262.20);

\path[draw=drawColor,line width= 0.6pt,line join=round] (109.77, 32.48) -- (109.77,262.20);

\path[draw=drawColor,line width= 0.6pt,line join=round] (109.77, 32.48) -- (109.77,262.20);

\path[draw=drawColor,line width= 0.6pt,line join=round] (109.77, 32.48) -- (109.77,262.20);

\path[draw=drawColor,line width= 0.6pt,line join=round] (109.77, 32.48) -- (109.77,262.20);

\path[draw=drawColor,line width= 0.6pt,line join=round] (109.77, 32.48) -- (109.77,262.20);

\path[draw=drawColor,line width= 0.6pt,line join=round] (109.77, 32.48) -- (109.77,262.20);

\path[draw=drawColor,line width= 0.6pt,line join=round] (109.77, 32.48) -- (109.77,262.20);

\path[draw=drawColor,line width= 0.6pt,line join=round] (109.77, 32.48) -- (109.77,262.20);

\path[draw=drawColor,line width= 0.6pt,line join=round] (109.77, 32.48) -- (109.77,262.20);

\path[draw=drawColor,line width= 0.6pt,line join=round] (109.77, 32.48) -- (109.77,262.20);

\path[draw=drawColor,line width= 0.6pt,line join=round] (109.77, 32.48) -- (109.77,262.20);

\path[draw=drawColor,line width= 0.6pt,line join=round] (109.77, 32.48) -- (109.77,262.20);

\path[draw=drawColor,line width= 0.6pt,line join=round] (109.77, 32.48) -- (109.77,262.20);

\path[draw=drawColor,line width= 0.6pt,line join=round] (109.77, 32.48) -- (109.77,262.20);

\path[draw=drawColor,line width= 0.6pt,line join=round] (109.77, 32.48) -- (109.77,262.20);

\path[draw=drawColor,line width= 0.6pt,line join=round] (109.77, 32.48) -- (109.77,262.20);

\path[draw=drawColor,line width= 0.6pt,line join=round] (109.77, 32.48) -- (109.77,262.20);

\path[draw=drawColor,line width= 0.6pt,line join=round] (109.77, 32.48) -- (109.77,262.20);

\path[draw=drawColor,line width= 0.6pt,line join=round] (109.77, 32.48) -- (109.77,262.20);

\path[draw=drawColor,line width= 0.6pt,line join=round] (109.77, 32.48) -- (109.77,262.20);

\path[draw=drawColor,line width= 0.6pt,line join=round] (109.77, 32.48) -- (109.77,262.20);

\path[draw=drawColor,line width= 0.6pt,line join=round] (109.77, 32.48) -- (109.77,262.20);

\path[draw=drawColor,line width= 0.6pt,line join=round] (109.77, 32.48) -- (109.77,262.20);

\path[draw=drawColor,line width= 0.6pt,line join=round] (109.77, 32.48) -- (109.77,262.20);

\path[draw=drawColor,line width= 0.6pt,line join=round] (109.77, 32.48) -- (109.77,262.20);

\path[draw=drawColor,line width= 0.6pt,line join=round] (109.77, 32.48) -- (109.77,262.20);

\path[draw=drawColor,line width= 0.6pt,line join=round] (109.77, 32.48) -- (109.77,262.20);

\path[draw=drawColor,line width= 0.6pt,line join=round] (109.77, 32.48) -- (109.77,262.20);

\path[draw=drawColor,line width= 0.6pt,line join=round] (109.77, 32.48) -- (109.77,262.20);

\path[draw=drawColor,line width= 0.6pt,line join=round] (109.77, 32.48) -- (109.77,262.20);

\path[draw=drawColor,line width= 0.6pt,line join=round] (109.77, 32.48) -- (109.77,262.20);

\path[draw=drawColor,line width= 0.6pt,line join=round] (109.77, 32.48) -- (109.77,262.20);

\path[draw=drawColor,line width= 0.6pt,line join=round] (109.77, 32.48) -- (109.77,262.20);

\path[draw=drawColor,line width= 0.6pt,line join=round] (109.77, 32.48) -- (109.77,262.20);

\path[draw=drawColor,line width= 0.6pt,line join=round] (109.77, 32.48) -- (109.77,262.20);

\path[draw=drawColor,line width= 0.6pt,line join=round] (109.77, 32.48) -- (109.77,262.20);

\path[draw=drawColor,line width= 0.6pt,line join=round] (109.77, 32.48) -- (109.77,262.20);

\path[draw=drawColor,line width= 0.6pt,line join=round] (109.77, 32.48) -- (109.77,262.20);

\path[draw=drawColor,line width= 0.6pt,line join=round] (109.77, 32.48) -- (109.77,262.20);

\path[draw=drawColor,line width= 0.6pt,line join=round] (109.77, 32.48) -- (109.77,262.20);

\path[draw=drawColor,line width= 0.6pt,line join=round] (109.77, 32.48) -- (109.77,262.20);

\path[draw=drawColor,line width= 0.6pt,line join=round] (109.77, 32.48) -- (109.77,262.20);

\path[draw=drawColor,line width= 0.6pt,line join=round] (109.77, 32.48) -- (109.77,262.20);

\path[draw=drawColor,line width= 0.6pt,line join=round] (109.77, 32.48) -- (109.77,262.20);

\path[draw=drawColor,line width= 0.6pt,line join=round] (109.77, 32.48) -- (109.77,262.20);

\path[draw=drawColor,line width= 0.6pt,line join=round] (109.77, 32.48) -- (109.77,262.20);

\path[draw=drawColor,line width= 0.6pt,line join=round] (109.77, 32.48) -- (109.77,262.20);

\path[draw=drawColor,line width= 0.6pt,line join=round] (109.77, 32.48) -- (109.77,262.20);

\path[draw=drawColor,line width= 0.6pt,line join=round] (109.77, 32.48) -- (109.77,262.20);

\path[draw=drawColor,line width= 0.6pt,line join=round] (109.77, 32.48) -- (109.77,262.20);

\path[draw=drawColor,line width= 0.6pt,line join=round] (109.77, 32.48) -- (109.77,262.20);

\path[draw=drawColor,line width= 0.6pt,line join=round] (109.77, 32.48) -- (109.77,262.20);

\path[draw=drawColor,line width= 0.6pt,line join=round] (109.77, 32.48) -- (109.77,262.20);

\path[draw=drawColor,line width= 0.6pt,line join=round] (109.77, 32.48) -- (109.77,262.20);

\path[draw=drawColor,line width= 0.6pt,line join=round] (109.77, 32.48) -- (109.77,262.20);

\path[draw=drawColor,line width= 0.6pt,line join=round] (109.77, 32.48) -- (109.77,262.20);

\path[draw=drawColor,line width= 0.6pt,line join=round] (109.77, 32.48) -- (109.77,262.20);

\path[draw=drawColor,line width= 0.6pt,line join=round] (109.77, 32.48) -- (109.77,262.20);

\path[draw=drawColor,line width= 0.6pt,line join=round] (109.77, 32.48) -- (109.77,262.20);

\path[draw=drawColor,line width= 0.6pt,line join=round] (109.77, 32.48) -- (109.77,262.20);

\path[draw=drawColor,line width= 0.6pt,line join=round] (109.77, 32.48) -- (109.77,262.20);

\path[draw=drawColor,line width= 0.6pt,line join=round] (109.77, 32.48) -- (109.77,262.20);

\path[draw=drawColor,line width= 0.6pt,line join=round] (109.77, 32.48) -- (109.77,262.20);

\path[draw=drawColor,line width= 0.6pt,line join=round] (109.77, 32.48) -- (109.77,262.20);

\path[draw=drawColor,line width= 0.6pt,line join=round] (109.77, 32.48) -- (109.77,262.20);

\path[draw=drawColor,line width= 0.6pt,line join=round] (109.77, 32.48) -- (109.77,262.20);

\path[draw=drawColor,line width= 0.6pt,line join=round] (109.77, 32.48) -- (109.77,262.20);

\path[draw=drawColor,line width= 0.6pt,line join=round] (109.77, 32.48) -- (109.77,262.20);

\path[draw=drawColor,line width= 0.6pt,line join=round] (109.77, 32.48) -- (109.77,262.20);

\path[draw=drawColor,line width= 0.6pt,line join=round] (109.77, 32.48) -- (109.77,262.20);

\path[draw=drawColor,line width= 0.6pt,line join=round] (109.77, 32.48) -- (109.77,262.20);

\path[draw=drawColor,line width= 0.6pt,line join=round] (109.77, 32.48) -- (109.77,262.20);

\path[draw=drawColor,line width= 0.6pt,line join=round] (109.77, 32.48) -- (109.77,262.20);

\path[draw=drawColor,line width= 0.6pt,line join=round] (109.77, 32.48) -- (109.77,262.20);

\path[draw=drawColor,line width= 0.6pt,line join=round] (109.77, 32.48) -- (109.77,262.20);

\path[draw=drawColor,line width= 0.6pt,line join=round] (109.77, 32.48) -- (109.77,262.20);

\path[draw=drawColor,line width= 0.6pt,line join=round] (109.77, 32.48) -- (109.77,262.20);

\path[draw=drawColor,line width= 0.6pt,line join=round] (109.77, 32.48) -- (109.77,262.20);

\path[draw=drawColor,line width= 0.6pt,line join=round] (109.77, 32.48) -- (109.77,262.20);

\path[draw=drawColor,line width= 0.6pt,line join=round] (109.77, 32.48) -- (109.77,262.20);

\path[draw=drawColor,line width= 0.6pt,line join=round] (109.77, 32.48) -- (109.77,262.20);

\path[draw=drawColor,line width= 0.6pt,line join=round] (109.77, 32.48) -- (109.77,262.20);

\path[draw=drawColor,line width= 0.6pt,line join=round] (109.77, 32.48) -- (109.77,262.20);

\path[draw=drawColor,line width= 0.6pt,line join=round] (109.77, 32.48) -- (109.77,262.20);

\path[draw=drawColor,line width= 0.6pt,line join=round] (109.77, 32.48) -- (109.77,262.20);

\path[draw=drawColor,line width= 0.6pt,line join=round] (109.77, 32.48) -- (109.77,262.20);

\path[draw=drawColor,line width= 0.6pt,line join=round] (109.77, 32.48) -- (109.77,262.20);

\path[draw=drawColor,line width= 0.6pt,line join=round] (109.77, 32.48) -- (109.77,262.20);

\path[draw=drawColor,line width= 0.6pt,line join=round] (109.77, 32.48) -- (109.77,262.20);

\path[draw=drawColor,line width= 0.6pt,line join=round] (109.77, 32.48) -- (109.77,262.20);

\path[draw=drawColor,line width= 0.6pt,line join=round] (109.77, 32.48) -- (109.77,262.20);

\path[draw=drawColor,line width= 0.6pt,line join=round] (109.77, 32.48) -- (109.77,262.20);

\path[draw=drawColor,line width= 0.6pt,line join=round] (109.77, 32.48) -- (109.77,262.20);

\path[draw=drawColor,line width= 0.6pt,line join=round] (109.77, 32.48) -- (109.77,262.20);

\path[draw=drawColor,line width= 0.6pt,line join=round] (109.77, 32.48) -- (109.77,262.20);

\path[draw=drawColor,line width= 0.6pt,line join=round] (109.77, 32.48) -- (109.77,262.20);

\path[draw=drawColor,line width= 0.6pt,line join=round] (109.77, 32.48) -- (109.77,262.20);

\path[draw=drawColor,line width= 0.6pt,line join=round] (109.77, 32.48) -- (109.77,262.20);

\path[draw=drawColor,line width= 0.6pt,line join=round] (109.77, 32.48) -- (109.77,262.20);

\path[draw=drawColor,line width= 0.6pt,line join=round] (109.77, 32.48) -- (109.77,262.20);

\path[draw=drawColor,line width= 0.6pt,line join=round] (109.77, 32.48) -- (109.77,262.20);

\path[draw=drawColor,line width= 0.6pt,line join=round] (109.77, 32.48) -- (109.77,262.20);

\path[draw=drawColor,line width= 0.6pt,line join=round] (109.77, 32.48) -- (109.77,262.20);

\path[draw=drawColor,line width= 0.6pt,line join=round] (109.77, 32.48) -- (109.77,262.20);

\path[draw=drawColor,line width= 0.6pt,line join=round] (109.77, 32.48) -- (109.77,262.20);

\path[draw=drawColor,line width= 0.6pt,line join=round] (109.77, 32.48) -- (109.77,262.20);

\path[draw=drawColor,line width= 0.6pt,line join=round] (109.77, 32.48) -- (109.77,262.20);

\path[draw=drawColor,line width= 0.6pt,line join=round] (109.77, 32.48) -- (109.77,262.20);

\path[draw=drawColor,line width= 0.6pt,line join=round] (109.77, 32.48) -- (109.77,262.20);

\path[draw=drawColor,line width= 0.6pt,line join=round] (109.77, 32.48) -- (109.77,262.20);

\path[draw=drawColor,line width= 0.6pt,line join=round] (109.77, 32.48) -- (109.77,262.20);

\path[draw=drawColor,line width= 0.6pt,line join=round] (109.77, 32.48) -- (109.77,262.20);

\path[draw=drawColor,line width= 0.6pt,line join=round] (109.77, 32.48) -- (109.77,262.20);

\path[draw=drawColor,line width= 0.6pt,line join=round] (109.77, 32.48) -- (109.77,262.20);

\path[draw=drawColor,line width= 0.6pt,line join=round] (109.77, 32.48) -- (109.77,262.20);

\path[draw=drawColor,line width= 0.6pt,line join=round] (109.77, 32.48) -- (109.77,262.20);

\path[draw=drawColor,line width= 0.6pt,line join=round] (109.77, 32.48) -- (109.77,262.20);

\path[draw=drawColor,line width= 0.6pt,line join=round] (109.77, 32.48) -- (109.77,262.20);

\path[draw=drawColor,line width= 0.6pt,line join=round] (109.77, 32.48) -- (109.77,262.20);

\path[draw=drawColor,line width= 0.6pt,line join=round] (109.77, 32.48) -- (109.77,262.20);

\path[draw=drawColor,line width= 0.6pt,line join=round] (109.77, 32.48) -- (109.77,262.20);

\path[draw=drawColor,line width= 0.6pt,line join=round] (109.77, 32.48) -- (109.77,262.20);

\path[draw=drawColor,line width= 0.6pt,line join=round] (109.77, 32.48) -- (109.77,262.20);

\path[draw=drawColor,line width= 0.6pt,line join=round] (109.77, 32.48) -- (109.77,262.20);

\path[draw=drawColor,line width= 0.6pt,line join=round] (109.77, 32.48) -- (109.77,262.20);

\path[draw=drawColor,line width= 0.6pt,line join=round] (109.77, 32.48) -- (109.77,262.20);

\path[draw=drawColor,line width= 0.6pt,line join=round] (109.77, 32.48) -- (109.77,262.20);

\path[draw=drawColor,line width= 0.6pt,line join=round] (109.77, 32.48) -- (109.77,262.20);

\path[draw=drawColor,line width= 0.6pt,line join=round] (109.77, 32.48) -- (109.77,262.20);

\path[draw=drawColor,line width= 0.6pt,line join=round] (109.77, 32.48) -- (109.77,262.20);

\path[draw=drawColor,line width= 0.6pt,line join=round] (109.77, 32.48) -- (109.77,262.20);

\path[draw=drawColor,line width= 0.6pt,line join=round] (109.77, 32.48) -- (109.77,262.20);

\path[draw=drawColor,line width= 0.6pt,line join=round] (109.77, 32.48) -- (109.77,262.20);

\path[draw=drawColor,line width= 0.6pt,line join=round] (109.77, 32.48) -- (109.77,262.20);

\path[draw=drawColor,line width= 0.6pt,line join=round] (109.77, 32.48) -- (109.77,262.20);

\path[draw=drawColor,line width= 0.6pt,line join=round] (109.77, 32.48) -- (109.77,262.20);

\path[draw=drawColor,line width= 0.6pt,line join=round] (109.77, 32.48) -- (109.77,262.20);

\path[draw=drawColor,line width= 0.6pt,line join=round] (109.77, 32.48) -- (109.77,262.20);

\path[draw=drawColor,line width= 0.6pt,line join=round] (109.77, 32.48) -- (109.77,262.20);

\path[draw=drawColor,line width= 0.6pt,line join=round] (109.77, 32.48) -- (109.77,262.20);

\path[draw=drawColor,line width= 0.6pt,line join=round] (109.77, 32.48) -- (109.77,262.20);

\path[draw=drawColor,line width= 0.6pt,line join=round] (109.77, 32.48) -- (109.77,262.20);

\path[draw=drawColor,line width= 0.6pt,line join=round] (109.77, 32.48) -- (109.77,262.20);

\path[draw=drawColor,line width= 0.6pt,line join=round] (109.77, 32.48) -- (109.77,262.20);

\path[draw=drawColor,line width= 0.6pt,line join=round] (109.77, 32.48) -- (109.77,262.20);

\path[draw=drawColor,line width= 0.6pt,line join=round] (109.77, 32.48) -- (109.77,262.20);

\path[draw=drawColor,line width= 0.6pt,line join=round] (109.77, 32.48) -- (109.77,262.20);

\path[draw=drawColor,line width= 0.6pt,line join=round] (109.77, 32.48) -- (109.77,262.20);

\path[draw=drawColor,line width= 0.6pt,line join=round] (109.77, 32.48) -- (109.77,262.20);

\path[draw=drawColor,line width= 0.6pt,line join=round] (109.77, 32.48) -- (109.77,262.20);

\path[draw=drawColor,line width= 0.6pt,line join=round] (109.77, 32.48) -- (109.77,262.20);

\path[draw=drawColor,line width= 0.6pt,line join=round] (109.77, 32.48) -- (109.77,262.20);

\path[draw=drawColor,line width= 0.6pt,line join=round] (109.77, 32.48) -- (109.77,262.20);

\path[draw=drawColor,line width= 0.6pt,line join=round] (109.77, 32.48) -- (109.77,262.20);

\path[draw=drawColor,line width= 0.6pt,line join=round] (109.77, 32.48) -- (109.77,262.20);

\path[draw=drawColor,line width= 0.6pt,line join=round] (109.77, 32.48) -- (109.77,262.20);

\path[draw=drawColor,line width= 0.6pt,line join=round] (109.77, 32.48) -- (109.77,262.20);

\path[draw=drawColor,line width= 0.6pt,line join=round] (109.77, 32.48) -- (109.77,262.20);

\path[draw=drawColor,line width= 0.6pt,line join=round] (109.77, 32.48) -- (109.77,262.20);

\path[draw=drawColor,line width= 0.6pt,line join=round] (109.77, 32.48) -- (109.77,262.20);

\path[draw=drawColor,line width= 0.6pt,line join=round] (109.77, 32.48) -- (109.77,262.20);

\path[draw=drawColor,line width= 0.6pt,line join=round] (109.77, 32.48) -- (109.77,262.20);

\path[draw=drawColor,line width= 0.6pt,line join=round] (109.77, 32.48) -- (109.77,262.20);

\path[draw=drawColor,line width= 0.6pt,line join=round] (109.77, 32.48) -- (109.77,262.20);

\path[draw=drawColor,line width= 0.6pt,line join=round] (109.77, 32.48) -- (109.77,262.20);

\path[draw=drawColor,line width= 0.6pt,line join=round] (109.77, 32.48) -- (109.77,262.20);

\path[draw=drawColor,line width= 0.6pt,line join=round] (109.77, 32.48) -- (109.77,262.20);

\path[draw=drawColor,line width= 0.6pt,line join=round] (109.77, 32.48) -- (109.77,262.20);

\path[draw=drawColor,line width= 0.6pt,line join=round] (109.77, 32.48) -- (109.77,262.20);

\path[draw=drawColor,line width= 0.6pt,line join=round] (109.77, 32.48) -- (109.77,262.20);

\path[draw=drawColor,line width= 0.6pt,line join=round] (109.77, 32.48) -- (109.77,262.20);

\path[draw=drawColor,line width= 0.6pt,line join=round] (109.77, 32.48) -- (109.77,262.20);

\path[draw=drawColor,line width= 0.6pt,line join=round] (109.77, 32.48) -- (109.77,262.20);

\path[draw=drawColor,line width= 0.6pt,line join=round] (109.77, 32.48) -- (109.77,262.20);

\path[draw=drawColor,line width= 0.6pt,line join=round] (109.77, 32.48) -- (109.77,262.20);

\path[draw=drawColor,line width= 0.6pt,line join=round] (109.77, 32.48) -- (109.77,262.20);

\path[draw=drawColor,line width= 0.6pt,line join=round] (109.77, 32.48) -- (109.77,262.20);

\path[draw=drawColor,line width= 0.6pt,line join=round] (109.77, 32.48) -- (109.77,262.20);

\path[draw=drawColor,line width= 0.6pt,line join=round] (109.77, 32.48) -- (109.77,262.20);

\path[draw=drawColor,line width= 0.6pt,line join=round] (109.77, 32.48) -- (109.77,262.20);

\path[draw=drawColor,line width= 0.6pt,line join=round] (109.77, 32.48) -- (109.77,262.20);

\path[draw=drawColor,line width= 0.6pt,line join=round] (109.77, 32.48) -- (109.77,262.20);

\path[draw=drawColor,line width= 0.6pt,line join=round] (109.77, 32.48) -- (109.77,262.20);

\path[draw=drawColor,line width= 0.6pt,line join=round] (109.77, 32.48) -- (109.77,262.20);

\path[draw=drawColor,line width= 0.6pt,line join=round] (109.77, 32.48) -- (109.77,262.20);

\path[draw=drawColor,line width= 0.6pt,line join=round] (109.77, 32.48) -- (109.77,262.20);

\path[draw=drawColor,line width= 0.6pt,line join=round] (109.77, 32.48) -- (109.77,262.20);

\path[draw=drawColor,line width= 0.6pt,line join=round] (109.77, 32.48) -- (109.77,262.20);

\path[draw=drawColor,line width= 0.6pt,line join=round] (109.77, 32.48) -- (109.77,262.20);

\path[draw=drawColor,line width= 0.6pt,line join=round] (109.77, 32.48) -- (109.77,262.20);

\path[draw=drawColor,line width= 0.6pt,line join=round] (109.77, 32.48) -- (109.77,262.20);

\path[draw=drawColor,line width= 0.6pt,line join=round] (109.77, 32.48) -- (109.77,262.20);

\path[draw=drawColor,line width= 0.6pt,line join=round] (109.77, 32.48) -- (109.77,262.20);

\path[draw=drawColor,line width= 0.6pt,line join=round] (109.77, 32.48) -- (109.77,262.20);

\path[draw=drawColor,line width= 0.6pt,line join=round] (109.77, 32.48) -- (109.77,262.20);

\path[draw=drawColor,line width= 0.6pt,line join=round] (109.77, 32.48) -- (109.77,262.20);

\path[draw=drawColor,line width= 0.6pt,line join=round] (109.77, 32.48) -- (109.77,262.20);

\path[draw=drawColor,line width= 0.6pt,line join=round] (109.77, 32.48) -- (109.77,262.20);

\path[draw=drawColor,line width= 0.6pt,line join=round] (109.77, 32.48) -- (109.77,262.20);

\path[draw=drawColor,line width= 0.6pt,line join=round] (109.77, 32.48) -- (109.77,262.20);

\path[draw=drawColor,line width= 0.6pt,line join=round] (109.77, 32.48) -- (109.77,262.20);

\path[draw=drawColor,line width= 0.6pt,line join=round] (109.77, 32.48) -- (109.77,262.20);

\path[draw=drawColor,line width= 0.6pt,line join=round] (109.77, 32.48) -- (109.77,262.20);

\path[draw=drawColor,line width= 0.6pt,line join=round] (109.77, 32.48) -- (109.77,262.20);

\path[draw=drawColor,line width= 0.6pt,line join=round] (109.77, 32.48) -- (109.77,262.20);

\path[draw=drawColor,line width= 0.6pt,line join=round] (109.77, 32.48) -- (109.77,262.20);

\path[draw=drawColor,line width= 0.6pt,line join=round] (109.77, 32.48) -- (109.77,262.20);

\path[draw=drawColor,line width= 0.6pt,line join=round] (109.77, 32.48) -- (109.77,262.20);

\path[draw=drawColor,line width= 0.6pt,line join=round] (109.77, 32.48) -- (109.77,262.20);

\path[draw=drawColor,line width= 0.6pt,line join=round] (109.77, 32.48) -- (109.77,262.20);

\path[draw=drawColor,line width= 0.6pt,line join=round] (109.77, 32.48) -- (109.77,262.20);

\path[draw=drawColor,line width= 0.6pt,line join=round] (109.77, 32.48) -- (109.77,262.20);

\path[draw=drawColor,line width= 0.6pt,line join=round] (109.77, 32.48) -- (109.77,262.20);

\path[draw=drawColor,line width= 0.6pt,line join=round] (109.77, 32.48) -- (109.77,262.20);

\path[draw=drawColor,line width= 0.6pt,line join=round] (109.77, 32.48) -- (109.77,262.20);

\path[draw=drawColor,line width= 0.6pt,line join=round] (109.77, 32.48) -- (109.77,262.20);

\path[draw=drawColor,line width= 0.6pt,line join=round] (109.77, 32.48) -- (109.77,262.20);

\path[draw=drawColor,line width= 0.6pt,line join=round] (109.77, 32.48) -- (109.77,262.20);

\path[draw=drawColor,line width= 0.6pt,line join=round] (109.77, 32.48) -- (109.77,262.20);

\path[draw=drawColor,line width= 0.6pt,line join=round] (109.77, 32.48) -- (109.77,262.20);

\path[draw=drawColor,line width= 0.6pt,line join=round] (109.77, 32.48) -- (109.77,262.20);

\path[draw=drawColor,line width= 0.6pt,line join=round] (109.77, 32.48) -- (109.77,262.20);

\path[draw=drawColor,line width= 0.6pt,line join=round] (109.77, 32.48) -- (109.77,262.20);

\path[draw=drawColor,line width= 0.6pt,line join=round] (109.77, 32.48) -- (109.77,262.20);

\path[draw=drawColor,line width= 0.6pt,line join=round] (109.77, 32.48) -- (109.77,262.20);

\path[draw=drawColor,line width= 0.6pt,line join=round] (109.77, 32.48) -- (109.77,262.20);

\path[draw=drawColor,line width= 0.6pt,line join=round] (109.77, 32.48) -- (109.77,262.20);

\path[draw=drawColor,line width= 0.6pt,line join=round] (109.77, 32.48) -- (109.77,262.20);

\path[draw=drawColor,line width= 0.6pt,line join=round] (109.77, 32.48) -- (109.77,262.20);

\path[draw=drawColor,line width= 0.6pt,line join=round] (109.77, 32.48) -- (109.77,262.20);

\path[draw=drawColor,line width= 0.6pt,line join=round] (109.77, 32.48) -- (109.77,262.20);

\path[draw=drawColor,line width= 0.6pt,line join=round] (109.77, 32.48) -- (109.77,262.20);

\path[draw=drawColor,line width= 0.6pt,line join=round] (109.77, 32.48) -- (109.77,262.20);

\path[draw=drawColor,line width= 0.6pt,line join=round] (109.77, 32.48) -- (109.77,262.20);

\path[draw=drawColor,line width= 0.6pt,line join=round] (109.77, 32.48) -- (109.77,262.20);

\path[draw=drawColor,line width= 0.6pt,line join=round] (109.77, 32.48) -- (109.77,262.20);

\path[draw=drawColor,line width= 0.6pt,line join=round] (109.77, 32.48) -- (109.77,262.20);

\path[draw=drawColor,line width= 0.6pt,line join=round] (109.77, 32.48) -- (109.77,262.20);

\path[draw=drawColor,line width= 0.6pt,line join=round] (109.77, 32.48) -- (109.77,262.20);

\path[draw=drawColor,line width= 0.6pt,line join=round] (150.19,151.48) -- (150.19,143.20);

\path[draw=drawColor,line width= 0.6pt,line join=round] (150.19,151.48) -- (150.19,143.20);

\path[draw=drawColor,line width= 0.6pt,line join=round] (150.19,151.48) -- (150.19,143.20);

\path[draw=drawColor,line width= 0.6pt,line join=round] (150.19,151.48) -- (150.19,143.20);

\path[draw=drawColor,line width= 0.6pt,line join=round] (150.19,151.48) -- (150.19,143.20);

\path[draw=drawColor,line width= 0.6pt,line join=round] (150.19,151.48) -- (150.19,143.20);

\path[draw=drawColor,line width= 0.6pt,line join=round] (150.19,151.48) -- (150.19,143.20);

\path[draw=drawColor,line width= 0.6pt,line join=round] (150.19,151.48) -- (150.19,143.20);

\path[draw=drawColor,line width= 0.6pt,line join=round] (150.19,151.48) -- (150.19,143.20);

\path[draw=drawColor,line width= 0.6pt,line join=round] (150.19,151.48) -- (150.19,143.20);

\path[draw=drawColor,line width= 0.6pt,line join=round] (150.19,151.48) -- (150.19,143.20);

\path[draw=drawColor,line width= 0.6pt,line join=round] (150.19,151.48) -- (150.19,143.20);

\path[draw=drawColor,line width= 0.6pt,line join=round] (150.19,151.48) -- (150.19,143.20);

\path[draw=drawColor,line width= 0.6pt,line join=round] (150.19,151.48) -- (150.19,143.20);

\path[draw=drawColor,line width= 0.6pt,line join=round] (150.19,151.48) -- (150.19,143.20);

\path[draw=drawColor,line width= 0.6pt,line join=round] (150.19,151.48) -- (150.19,143.20);

\path[draw=drawColor,line width= 0.6pt,line join=round] (150.19,151.48) -- (150.19,143.20);

\path[draw=drawColor,line width= 0.6pt,line join=round] (150.19,151.48) -- (150.19,143.20);

\path[draw=drawColor,line width= 0.6pt,line join=round] (150.19,151.48) -- (150.19,143.20);

\path[draw=drawColor,line width= 0.6pt,line join=round] (150.19,151.48) -- (150.19,143.20);

\path[draw=drawColor,line width= 0.6pt,line join=round] (150.19,151.48) -- (150.19,143.20);

\path[draw=drawColor,line width= 0.6pt,line join=round] (150.19,151.48) -- (150.19,143.20);

\path[draw=drawColor,line width= 0.6pt,line join=round] (150.19,151.48) -- (150.19,143.20);

\path[draw=drawColor,line width= 0.6pt,line join=round] (150.19,151.48) -- (150.19,143.20);

\path[draw=drawColor,line width= 0.6pt,line join=round] (150.19,151.48) -- (150.19,143.20);

\path[draw=drawColor,line width= 0.6pt,line join=round] (150.19,151.48) -- (150.19,143.20);

\path[draw=drawColor,line width= 0.6pt,line join=round] (150.19,151.48) -- (150.19,143.20);

\path[draw=drawColor,line width= 0.6pt,line join=round] (150.19,151.48) -- (150.19,143.20);

\path[draw=drawColor,line width= 0.6pt,line join=round] (150.19,151.48) -- (150.19,143.20);

\path[draw=drawColor,line width= 0.6pt,line join=round] (150.19,151.48) -- (150.19,143.20);

\path[draw=drawColor,line width= 0.6pt,line join=round] (150.19,151.48) -- (150.19,143.20);

\path[draw=drawColor,line width= 0.6pt,line join=round] (150.19,151.48) -- (150.19,143.20);

\path[draw=drawColor,line width= 0.6pt,line join=round] (150.19,151.48) -- (150.19,143.20);

\path[draw=drawColor,line width= 0.6pt,line join=round] (150.19,151.48) -- (150.19,143.20);

\path[draw=drawColor,line width= 0.6pt,line join=round] (150.19,151.48) -- (150.19,143.20);

\path[draw=drawColor,line width= 0.6pt,line join=round] (150.19,151.48) -- (150.19,143.20);

\path[draw=drawColor,line width= 0.6pt,line join=round] (150.19,151.48) -- (150.19,143.20);

\path[draw=drawColor,line width= 0.6pt,line join=round] (150.19,151.48) -- (150.19,143.20);

\path[draw=drawColor,line width= 0.6pt,line join=round] (150.19,151.48) -- (150.19,143.20);

\path[draw=drawColor,line width= 0.6pt,line join=round] (150.19,151.48) -- (150.19,143.20);

\path[draw=drawColor,line width= 0.6pt,line join=round] (150.19,151.48) -- (150.19,143.20);

\path[draw=drawColor,line width= 0.6pt,line join=round] (150.19,151.48) -- (150.19,143.20);

\path[draw=drawColor,line width= 0.6pt,line join=round] (150.19,151.48) -- (150.19,143.20);

\path[draw=drawColor,line width= 0.6pt,line join=round] (150.19,151.48) -- (150.19,143.20);

\path[draw=drawColor,line width= 0.6pt,line join=round] (150.19,151.48) -- (150.19,143.20);

\path[draw=drawColor,line width= 0.6pt,line join=round] (150.19,151.48) -- (150.19,143.20);

\path[draw=drawColor,line width= 0.6pt,line join=round] (150.19,151.48) -- (150.19,143.20);

\path[draw=drawColor,line width= 0.6pt,line join=round] (150.19,151.48) -- (150.19,143.20);

\path[draw=drawColor,line width= 0.6pt,line join=round] (150.19,151.48) -- (150.19,143.20);

\path[draw=drawColor,line width= 0.6pt,line join=round] (150.19,151.48) -- (150.19,143.20);

\path[draw=drawColor,line width= 0.6pt,line join=round] (150.19,151.48) -- (150.19,143.20);

\path[draw=drawColor,line width= 0.6pt,line join=round] (150.19,151.48) -- (150.19,143.20);

\path[draw=drawColor,line width= 0.6pt,line join=round] (150.19,151.48) -- (150.19,143.20);

\path[draw=drawColor,line width= 0.6pt,line join=round] (150.19,151.48) -- (150.19,143.20);

\path[draw=drawColor,line width= 0.6pt,line join=round] (150.19,151.48) -- (150.19,143.20);

\path[draw=drawColor,line width= 0.6pt,line join=round] (150.19,151.48) -- (150.19,143.20);

\path[draw=drawColor,line width= 0.6pt,line join=round] (150.19,151.48) -- (150.19,143.20);

\path[draw=drawColor,line width= 0.6pt,line join=round] (150.19,151.48) -- (150.19,143.20);

\path[draw=drawColor,line width= 0.6pt,line join=round] (150.19,151.48) -- (150.19,143.20);

\path[draw=drawColor,line width= 0.6pt,line join=round] (150.19,151.48) -- (150.19,143.20);

\path[draw=drawColor,line width= 0.6pt,line join=round] (150.19,151.48) -- (150.19,143.20);

\path[draw=drawColor,line width= 0.6pt,line join=round] (150.19,151.48) -- (150.19,143.20);

\path[draw=drawColor,line width= 0.6pt,line join=round] (150.19,151.48) -- (150.19,143.20);

\path[draw=drawColor,line width= 0.6pt,line join=round] (150.19,151.48) -- (150.19,143.20);

\path[draw=drawColor,line width= 0.6pt,line join=round] (150.19,151.48) -- (150.19,143.20);

\path[draw=drawColor,line width= 0.6pt,line join=round] (150.19,151.48) -- (150.19,143.20);

\path[draw=drawColor,line width= 0.6pt,line join=round] (150.19,151.48) -- (150.19,143.20);

\path[draw=drawColor,line width= 0.6pt,line join=round] (150.19,151.48) -- (150.19,143.20);

\path[draw=drawColor,line width= 0.6pt,line join=round] (150.19,151.48) -- (150.19,143.20);

\path[draw=drawColor,line width= 0.6pt,line join=round] (150.19,151.48) -- (150.19,143.20);

\path[draw=drawColor,line width= 0.6pt,line join=round] (150.19,151.48) -- (150.19,143.20);

\path[draw=drawColor,line width= 0.6pt,line join=round] (150.19,151.48) -- (150.19,143.20);

\path[draw=drawColor,line width= 0.6pt,line join=round] (150.19,151.48) -- (150.19,143.20);

\path[draw=drawColor,line width= 0.6pt,line join=round] (150.19,151.48) -- (150.19,143.20);

\path[draw=drawColor,line width= 0.6pt,line join=round] (150.19,151.48) -- (150.19,143.20);

\path[draw=drawColor,line width= 0.6pt,line join=round] (150.19,151.48) -- (150.19,143.20);

\path[draw=drawColor,line width= 0.6pt,line join=round] (150.19,151.48) -- (150.19,143.20);

\path[draw=drawColor,line width= 0.6pt,line join=round] (150.19,151.48) -- (150.19,143.20);

\path[draw=drawColor,line width= 0.6pt,line join=round] (150.19,151.48) -- (150.19,143.20);

\path[draw=drawColor,line width= 0.6pt,line join=round] (150.19,151.48) -- (150.19,143.20);

\path[draw=drawColor,line width= 0.6pt,line join=round] (150.19,151.48) -- (150.19,143.20);

\path[draw=drawColor,line width= 0.6pt,line join=round] (150.19,151.48) -- (150.19,143.20);

\path[draw=drawColor,line width= 0.6pt,line join=round] (150.19,151.48) -- (150.19,143.20);

\path[draw=drawColor,line width= 0.6pt,line join=round] (150.19,151.48) -- (150.19,143.20);

\path[draw=drawColor,line width= 0.6pt,line join=round] (150.19,151.48) -- (150.19,143.20);

\path[draw=drawColor,line width= 0.6pt,line join=round] (150.19,151.48) -- (150.19,143.20);

\path[draw=drawColor,line width= 0.6pt,line join=round] (150.19,151.48) -- (150.19,143.20);

\path[draw=drawColor,line width= 0.6pt,line join=round] (150.19,151.48) -- (150.19,143.20);

\path[draw=drawColor,line width= 0.6pt,line join=round] (150.19,151.48) -- (150.19,143.20);

\path[draw=drawColor,line width= 0.6pt,line join=round] (150.19,151.48) -- (150.19,143.20);

\path[draw=drawColor,line width= 0.6pt,line join=round] (150.19,151.48) -- (150.19,143.20);

\path[draw=drawColor,line width= 0.6pt,line join=round] (150.19,151.48) -- (150.19,143.20);

\path[draw=drawColor,line width= 0.6pt,line join=round] (150.19,151.48) -- (150.19,143.20);

\path[draw=drawColor,line width= 0.6pt,line join=round] (150.19,151.48) -- (150.19,143.20);

\path[draw=drawColor,line width= 0.6pt,line join=round] (150.19,151.48) -- (150.19,143.20);

\path[draw=drawColor,line width= 0.6pt,line join=round] (150.19,151.48) -- (150.19,143.20);

\path[draw=drawColor,line width= 0.6pt,line join=round] (150.19,151.48) -- (150.19,143.20);

\path[draw=drawColor,line width= 0.6pt,line join=round] (150.19,151.48) -- (150.19,143.20);

\path[draw=drawColor,line width= 0.6pt,line join=round] (150.19,151.48) -- (150.19,143.20);

\path[draw=drawColor,line width= 0.6pt,line join=round] (150.19,151.48) -- (150.19,143.20);

\path[draw=drawColor,line width= 0.6pt,line join=round] (150.19,151.48) -- (150.19,143.20);

\path[draw=drawColor,line width= 0.6pt,line join=round] (150.19,151.48) -- (150.19,143.20);

\path[draw=drawColor,line width= 0.6pt,line join=round] (150.19,151.48) -- (150.19,143.20);

\path[draw=drawColor,line width= 0.6pt,line join=round] (150.19,151.48) -- (150.19,143.20);

\path[draw=drawColor,line width= 0.6pt,line join=round] (150.19,151.48) -- (150.19,143.20);

\path[draw=drawColor,line width= 0.6pt,line join=round] (150.19,151.48) -- (150.19,143.20);

\path[draw=drawColor,line width= 0.6pt,line join=round] (150.19,151.48) -- (150.19,143.20);

\path[draw=drawColor,line width= 0.6pt,line join=round] (150.19,151.48) -- (150.19,143.20);

\path[draw=drawColor,line width= 0.6pt,line join=round] (150.19,151.48) -- (150.19,143.20);

\path[draw=drawColor,line width= 0.6pt,line join=round] (150.19,151.48) -- (150.19,143.20);

\path[draw=drawColor,line width= 0.6pt,line join=round] (150.19,151.48) -- (150.19,143.20);

\path[draw=drawColor,line width= 0.6pt,line join=round] (150.19,151.48) -- (150.19,143.20);

\path[draw=drawColor,line width= 0.6pt,line join=round] (150.19,151.48) -- (150.19,143.20);

\path[draw=drawColor,line width= 0.6pt,line join=round] (150.19,151.48) -- (150.19,143.20);

\path[draw=drawColor,line width= 0.6pt,line join=round] (150.19,151.48) -- (150.19,143.20);

\path[draw=drawColor,line width= 0.6pt,line join=round] (150.19,151.48) -- (150.19,143.20);

\path[draw=drawColor,line width= 0.6pt,line join=round] (150.19,151.48) -- (150.19,143.20);

\path[draw=drawColor,line width= 0.6pt,line join=round] (150.19,151.48) -- (150.19,143.20);

\path[draw=drawColor,line width= 0.6pt,line join=round] (150.19,151.48) -- (150.19,143.20);

\path[draw=drawColor,line width= 0.6pt,line join=round] (150.19,151.48) -- (150.19,143.20);

\path[draw=drawColor,line width= 0.6pt,line join=round] (150.19,151.48) -- (150.19,143.20);

\path[draw=drawColor,line width= 0.6pt,line join=round] (150.19,151.48) -- (150.19,143.20);

\path[draw=drawColor,line width= 0.6pt,line join=round] (150.19,151.48) -- (150.19,143.20);

\path[draw=drawColor,line width= 0.6pt,line join=round] (150.19,151.48) -- (150.19,143.20);

\path[draw=drawColor,line width= 0.6pt,line join=round] (150.19,151.48) -- (150.19,143.20);

\path[draw=drawColor,line width= 0.6pt,line join=round] (150.19,151.48) -- (150.19,143.20);

\path[draw=drawColor,line width= 0.6pt,line join=round] (150.19,151.48) -- (150.19,143.20);

\path[draw=drawColor,line width= 0.6pt,line join=round] (150.19,151.48) -- (150.19,143.20);

\path[draw=drawColor,line width= 0.6pt,line join=round] (150.19,151.48) -- (150.19,143.20);

\path[draw=drawColor,line width= 0.6pt,line join=round] (150.19,151.48) -- (150.19,143.20);

\path[draw=drawColor,line width= 0.6pt,line join=round] (150.19,151.48) -- (150.19,143.20);

\path[draw=drawColor,line width= 0.6pt,line join=round] (150.19,151.48) -- (150.19,143.20);

\path[draw=drawColor,line width= 0.6pt,line join=round] (150.19,151.48) -- (150.19,143.20);

\path[draw=drawColor,line width= 0.6pt,line join=round] (150.19,151.48) -- (150.19,143.20);

\path[draw=drawColor,line width= 0.6pt,line join=round] (150.19,151.48) -- (150.19,143.20);

\path[draw=drawColor,line width= 0.6pt,line join=round] (150.19,151.48) -- (150.19,143.20);

\path[draw=drawColor,line width= 0.6pt,line join=round] (150.19,151.48) -- (150.19,143.20);

\path[draw=drawColor,line width= 0.6pt,line join=round] (150.19,151.48) -- (150.19,143.20);

\path[draw=drawColor,line width= 0.6pt,line join=round] (150.19,151.48) -- (150.19,143.20);

\path[draw=drawColor,line width= 0.6pt,line join=round] (150.19,151.48) -- (150.19,143.20);

\path[draw=drawColor,line width= 0.6pt,line join=round] (150.19,151.48) -- (150.19,143.20);

\path[draw=drawColor,line width= 0.6pt,line join=round] (150.19,151.48) -- (150.19,143.20);

\path[draw=drawColor,line width= 0.6pt,line join=round] (150.19,151.48) -- (150.19,143.20);

\path[draw=drawColor,line width= 0.6pt,line join=round] (150.19,151.48) -- (150.19,143.20);

\path[draw=drawColor,line width= 0.6pt,line join=round] (150.19,151.48) -- (150.19,143.20);

\path[draw=drawColor,line width= 0.6pt,line join=round] (150.19,151.48) -- (150.19,143.20);

\path[draw=drawColor,line width= 0.6pt,line join=round] (150.19,151.48) -- (150.19,143.20);

\path[draw=drawColor,line width= 0.6pt,line join=round] (150.19,151.48) -- (150.19,143.20);

\path[draw=drawColor,line width= 0.6pt,line join=round] (150.19,151.48) -- (150.19,143.20);

\path[draw=drawColor,line width= 0.6pt,line join=round] (150.19,151.48) -- (150.19,143.20);

\path[draw=drawColor,line width= 0.6pt,line join=round] (150.19,151.48) -- (150.19,143.20);

\path[draw=drawColor,line width= 0.6pt,line join=round] (150.19,151.48) -- (150.19,143.20);

\path[draw=drawColor,line width= 0.6pt,line join=round] (150.19,151.48) -- (150.19,143.20);

\path[draw=drawColor,line width= 0.6pt,line join=round] (150.19,151.48) -- (150.19,143.20);

\path[draw=drawColor,line width= 0.6pt,line join=round] (150.19,151.48) -- (150.19,143.20);

\path[draw=drawColor,line width= 0.6pt,line join=round] (150.19,151.48) -- (150.19,143.20);

\path[draw=drawColor,line width= 0.6pt,line join=round] (150.19,151.48) -- (150.19,143.20);

\path[draw=drawColor,line width= 0.6pt,line join=round] (150.19,151.48) -- (150.19,143.20);

\path[draw=drawColor,line width= 0.6pt,line join=round] (150.19,151.48) -- (150.19,143.20);

\path[draw=drawColor,line width= 0.6pt,line join=round] (150.19,151.48) -- (150.19,143.20);

\path[draw=drawColor,line width= 0.6pt,line join=round] (150.19,151.48) -- (150.19,143.20);

\path[draw=drawColor,line width= 0.6pt,line join=round] (150.19,151.48) -- (150.19,143.20);

\path[draw=drawColor,line width= 0.6pt,line join=round] (150.19,151.48) -- (150.19,143.20);

\path[draw=drawColor,line width= 0.6pt,line join=round] (150.19,151.48) -- (150.19,143.20);

\path[draw=drawColor,line width= 0.6pt,line join=round] (150.19,151.48) -- (150.19,143.20);

\path[draw=drawColor,line width= 0.6pt,line join=round] (150.19,151.48) -- (150.19,143.20);

\path[draw=drawColor,line width= 0.6pt,line join=round] (150.19,151.48) -- (150.19,143.20);

\path[draw=drawColor,line width= 0.6pt,line join=round] (150.19,151.48) -- (150.19,143.20);

\path[draw=drawColor,line width= 0.6pt,line join=round] (150.19,151.48) -- (150.19,143.20);

\path[draw=drawColor,line width= 0.6pt,line join=round] (150.19,151.48) -- (150.19,143.20);

\path[draw=drawColor,line width= 0.6pt,line join=round] (150.19,151.48) -- (150.19,143.20);

\path[draw=drawColor,line width= 0.6pt,line join=round] (150.19,151.48) -- (150.19,143.20);

\path[draw=drawColor,line width= 0.6pt,line join=round] (150.19,151.48) -- (150.19,143.20);

\path[draw=drawColor,line width= 0.6pt,line join=round] (150.19,151.48) -- (150.19,143.20);

\path[draw=drawColor,line width= 0.6pt,line join=round] (150.19,151.48) -- (150.19,143.20);

\path[draw=drawColor,line width= 0.6pt,line join=round] (150.19,151.48) -- (150.19,143.20);

\path[draw=drawColor,line width= 0.6pt,line join=round] (150.19,151.48) -- (150.19,143.20);

\path[draw=drawColor,line width= 0.6pt,line join=round] (150.19,151.48) -- (150.19,143.20);

\path[draw=drawColor,line width= 0.6pt,line join=round] (150.19,151.48) -- (150.19,143.20);

\path[draw=drawColor,line width= 0.6pt,line join=round] (150.19,151.48) -- (150.19,143.20);

\path[draw=drawColor,line width= 0.6pt,line join=round] (150.19,151.48) -- (150.19,143.20);

\path[draw=drawColor,line width= 0.6pt,line join=round] (150.19,151.48) -- (150.19,143.20);

\path[draw=drawColor,line width= 0.6pt,line join=round] (150.19,151.48) -- (150.19,143.20);

\path[draw=drawColor,line width= 0.6pt,line join=round] (150.19,151.48) -- (150.19,143.20);

\path[draw=drawColor,line width= 0.6pt,line join=round] (150.19,151.48) -- (150.19,143.20);

\path[draw=drawColor,line width= 0.6pt,line join=round] (150.19,151.48) -- (150.19,143.20);

\path[draw=drawColor,line width= 0.6pt,line join=round] (150.19,151.48) -- (150.19,143.20);

\path[draw=drawColor,line width= 0.6pt,line join=round] (150.19,151.48) -- (150.19,143.20);

\path[draw=drawColor,line width= 0.6pt,line join=round] (150.19,151.48) -- (150.19,143.20);

\path[draw=drawColor,line width= 0.6pt,line join=round] (150.19,151.48) -- (150.19,143.20);

\path[draw=drawColor,line width= 0.6pt,line join=round] (150.19,151.48) -- (150.19,143.20);

\path[draw=drawColor,line width= 0.6pt,line join=round] (150.19,151.48) -- (150.19,143.20);

\path[draw=drawColor,line width= 0.6pt,line join=round] (150.19,151.48) -- (150.19,143.20);

\path[draw=drawColor,line width= 0.6pt,line join=round] (150.19,151.48) -- (150.19,143.20);

\path[draw=drawColor,line width= 0.6pt,line join=round] (150.19,151.48) -- (150.19,143.20);

\path[draw=drawColor,line width= 0.6pt,line join=round] (150.19,151.48) -- (150.19,143.20);

\path[draw=drawColor,line width= 0.6pt,line join=round] (150.19,151.48) -- (150.19,143.20);

\path[draw=drawColor,line width= 0.6pt,line join=round] (150.19,151.48) -- (150.19,143.20);

\path[draw=drawColor,line width= 0.6pt,line join=round] (150.19,151.48) -- (150.19,143.20);

\path[draw=drawColor,line width= 0.6pt,line join=round] (150.19,151.48) -- (150.19,143.20);

\path[draw=drawColor,line width= 0.6pt,line join=round] (150.19,151.48) -- (150.19,143.20);

\path[draw=drawColor,line width= 0.6pt,line join=round] (150.19,151.48) -- (150.19,143.20);

\path[draw=drawColor,line width= 0.6pt,line join=round] (150.19,151.48) -- (150.19,143.20);

\path[draw=drawColor,line width= 0.6pt,line join=round] (150.19,151.48) -- (150.19,143.20);

\path[draw=drawColor,line width= 0.6pt,line join=round] (150.19,151.48) -- (150.19,143.20);

\path[draw=drawColor,line width= 0.6pt,line join=round] (150.19,151.48) -- (150.19,143.20);

\path[draw=drawColor,line width= 0.6pt,line join=round] (150.19,151.48) -- (150.19,143.20);

\path[draw=drawColor,line width= 0.6pt,line join=round] (150.19,151.48) -- (150.19,143.20);

\path[draw=drawColor,line width= 0.6pt,line join=round] (150.19,151.48) -- (150.19,143.20);

\path[draw=drawColor,line width= 0.6pt,line join=round] (150.19,151.48) -- (150.19,143.20);

\path[draw=drawColor,line width= 0.6pt,line join=round] (150.19,151.48) -- (150.19,143.20);

\path[draw=drawColor,line width= 0.6pt,line join=round] (150.19,151.48) -- (150.19,143.20);

\path[draw=drawColor,line width= 0.6pt,line join=round] (150.19,151.48) -- (150.19,143.20);

\path[draw=drawColor,line width= 0.6pt,line join=round] (150.19,151.48) -- (150.19,143.20);

\path[draw=drawColor,line width= 0.6pt,line join=round] (150.19,151.48) -- (150.19,143.20);

\path[draw=drawColor,line width= 0.6pt,line join=round] (150.19,151.48) -- (150.19,143.20);

\path[draw=drawColor,line width= 0.6pt,line join=round] (150.19,151.48) -- (150.19,143.20);

\path[draw=drawColor,line width= 0.6pt,line join=round] (150.19,151.48) -- (150.19,143.20);

\path[draw=drawColor,line width= 0.6pt,line join=round] (150.19,151.48) -- (150.19,143.20);

\path[draw=drawColor,line width= 0.6pt,line join=round] (150.19,151.48) -- (150.19,143.20);

\path[draw=drawColor,line width= 0.6pt,line join=round] (150.19,151.48) -- (150.19,143.20);

\path[draw=drawColor,line width= 0.6pt,line join=round] (150.19,151.48) -- (150.19,143.20);

\path[draw=drawColor,line width= 0.6pt,line join=round] (150.19,151.48) -- (150.19,143.20);

\path[draw=drawColor,line width= 0.6pt,line join=round] (150.19,151.48) -- (150.19,143.20);

\path[draw=drawColor,line width= 0.6pt,line join=round] (150.19,151.48) -- (150.19,143.20);

\path[draw=drawColor,line width= 0.6pt,line join=round] (150.19,151.48) -- (150.19,143.20);

\path[draw=drawColor,line width= 0.6pt,line join=round] (150.19,151.48) -- (150.19,143.20);

\path[draw=drawColor,line width= 0.6pt,line join=round] (150.19,151.48) -- (150.19,143.20);

\path[draw=drawColor,line width= 0.6pt,line join=round] (150.19,151.48) -- (150.19,143.20);

\path[draw=drawColor,line width= 0.6pt,line join=round] (150.19,151.48) -- (150.19,143.20);

\path[draw=drawColor,line width= 0.6pt,line join=round] (150.19,151.48) -- (150.19,143.20);

\path[draw=drawColor,line width= 0.6pt,line join=round] (150.19,151.48) -- (150.19,143.20);

\path[draw=drawColor,line width= 0.6pt,line join=round] (150.19,151.48) -- (150.19,143.20);

\path[draw=drawColor,line width= 0.6pt,line join=round] (150.19,151.48) -- (150.19,143.20);

\path[draw=drawColor,line width= 0.6pt,line join=round] (150.19,151.48) -- (150.19,143.20);

\path[draw=drawColor,line width= 0.6pt,line join=round] (150.19,151.48) -- (150.19,143.20);

\path[draw=drawColor,line width= 0.6pt,line join=round] (150.19,151.48) -- (150.19,143.20);

\path[draw=drawColor,line width= 0.6pt,line join=round] (150.19,151.48) -- (150.19,143.20);

\path[draw=drawColor,line width= 0.6pt,line join=round] (150.19,151.48) -- (150.19,143.20);

\path[draw=drawColor,line width= 0.6pt,line join=round] (150.19,151.48) -- (150.19,143.20);

\path[draw=drawColor,line width= 0.6pt,line join=round] (150.19,151.48) -- (150.19,143.20);

\path[draw=drawColor,line width= 0.6pt,line join=round] (150.19,151.48) -- (150.19,143.20);

\path[draw=drawColor,line width= 0.6pt,line join=round] (150.19,151.48) -- (150.19,143.20);

\path[draw=drawColor,line width= 0.6pt,line join=round] (150.19,151.48) -- (150.19,143.20);

\path[draw=drawColor,line width= 0.6pt,line join=round] (150.19,151.48) -- (150.19,143.20);

\path[draw=drawColor,line width= 0.6pt,line join=round] (150.19,151.48) -- (150.19,143.20);

\path[draw=drawColor,line width= 0.6pt,line join=round] (150.19,151.48) -- (150.19,143.20);

\path[draw=drawColor,line width= 0.6pt,line join=round] (150.19,151.48) -- (150.19,143.20);

\path[draw=drawColor,line width= 0.6pt,line join=round] (150.19,151.48) -- (150.19,143.20);

\path[draw=drawColor,line width= 0.6pt,line join=round] (150.19,151.48) -- (150.19,143.20);

\path[draw=drawColor,line width= 0.6pt,line join=round] (150.19,151.48) -- (150.19,143.20);

\path[draw=drawColor,line width= 0.6pt,line join=round] (150.19,151.48) -- (150.19,143.20);

\path[draw=drawColor,line width= 0.6pt,line join=round] (150.19,151.48) -- (150.19,143.20);

\path[draw=drawColor,line width= 0.6pt,line join=round] (150.19,151.48) -- (150.19,143.20);

\path[draw=drawColor,line width= 0.6pt,line join=round] (150.19,151.48) -- (150.19,143.20);

\path[draw=drawColor,line width= 0.6pt,line join=round] (150.19,151.48) -- (150.19,143.20);

\path[draw=drawColor,line width= 0.6pt,line join=round] (150.19,151.48) -- (150.19,143.20);

\path[draw=drawColor,line width= 0.6pt,line join=round] (150.19,151.48) -- (150.19,143.20);

\path[draw=drawColor,line width= 0.6pt,line join=round] (150.19,151.48) -- (150.19,143.20);

\path[draw=drawColor,line width= 0.6pt,line join=round] (150.19,151.48) -- (150.19,143.20);

\path[draw=drawColor,line width= 0.6pt,line join=round] (150.19,151.48) -- (150.19,143.20);

\path[draw=drawColor,line width= 0.6pt,line join=round] (150.19,151.48) -- (150.19,143.20);

\path[draw=drawColor,line width= 0.6pt,line join=round] (150.19,151.48) -- (150.19,143.20);

\path[draw=drawColor,line width= 0.6pt,line join=round] (150.19,151.48) -- (150.19,143.20);

\path[draw=drawColor,line width= 0.6pt,line join=round] (150.19,151.48) -- (150.19,143.20);

\path[draw=drawColor,line width= 0.6pt,line join=round] (150.19,151.48) -- (150.19,143.20);

\path[draw=drawColor,line width= 0.6pt,line join=round] (150.19,151.48) -- (150.19,143.20);

\path[draw=drawColor,line width= 0.6pt,line join=round] (150.19,151.48) -- (150.19,143.20);

\path[draw=drawColor,line width= 0.6pt,line join=round] (150.19,151.48) -- (150.19,143.20);

\path[draw=drawColor,line width= 0.6pt,line join=round] (150.19,151.48) -- (150.19,143.20);

\path[draw=drawColor,line width= 0.6pt,line join=round] (150.19,151.48) -- (150.19,143.20);

\path[draw=drawColor,line width= 0.6pt,line join=round] (150.19,151.48) -- (150.19,143.20);

\path[draw=drawColor,line width= 0.6pt,line join=round] (150.19,151.48) -- (150.19,143.20);

\path[draw=drawColor,line width= 0.6pt,line join=round] (150.19,151.48) -- (150.19,143.20);

\path[draw=drawColor,line width= 0.6pt,line join=round] (150.19,151.48) -- (150.19,143.20);

\path[draw=drawColor,line width= 0.6pt,line join=round] (150.19,151.48) -- (150.19,143.20);

\path[draw=drawColor,line width= 0.6pt,line join=round] (150.19,151.48) -- (150.19,143.20);

\path[draw=drawColor,line width= 0.6pt,line join=round] (150.19,151.48) -- (150.19,143.20);

\path[draw=drawColor,line width= 0.6pt,line join=round] (150.19,151.48) -- (150.19,143.20);

\path[draw=drawColor,line width= 0.6pt,line join=round] (150.19,151.48) -- (150.19,143.20);

\path[draw=drawColor,line width= 0.6pt,line join=round] (150.19,151.48) -- (150.19,143.20);

\path[draw=drawColor,line width= 0.6pt,line join=round] (150.19,151.48) -- (150.19,143.20);

\path[draw=drawColor,line width= 0.6pt,line join=round] (150.19,151.48) -- (150.19,143.20);

\path[draw=drawColor,line width= 0.6pt,line join=round] (150.19,151.48) -- (150.19,143.20);

\path[draw=drawColor,line width= 0.6pt,line join=round] (150.19,151.48) -- (150.19,143.20);

\path[draw=drawColor,line width= 0.6pt,line join=round] (150.19,151.48) -- (150.19,143.20);

\path[draw=drawColor,line width= 0.6pt,line join=round] (150.19,151.48) -- (150.19,143.20);

\path[draw=drawColor,line width= 0.6pt,line join=round] (150.19,151.48) -- (150.19,143.20);

\path[draw=drawColor,line width= 0.6pt,line join=round] (150.19,151.48) -- (150.19,143.20);

\path[draw=drawColor,line width= 0.6pt,line join=round] (150.19,151.48) -- (150.19,143.20);

\path[draw=drawColor,line width= 0.6pt,line join=round] (150.19,151.48) -- (150.19,143.20);

\path[draw=drawColor,line width= 0.6pt,line join=round] (150.19,151.48) -- (150.19,143.20);

\path[draw=drawColor,line width= 0.6pt,line join=round] (150.19,151.48) -- (150.19,143.20);

\path[draw=drawColor,line width= 0.6pt,line join=round] (150.19,151.48) -- (150.19,143.20);

\path[draw=drawColor,line width= 0.6pt,line join=round] (150.19,151.48) -- (150.19,143.20);

\path[draw=drawColor,line width= 0.6pt,line join=round] (150.19,151.48) -- (150.19,143.20);

\path[draw=drawColor,line width= 0.6pt,line join=round] (150.19,151.48) -- (150.19,143.20);

\path[draw=drawColor,line width= 0.6pt,line join=round] (150.19,151.48) -- (150.19,143.20);

\path[draw=drawColor,line width= 0.6pt,line join=round] (150.19,151.48) -- (150.19,143.20);

\path[draw=drawColor,line width= 0.6pt,line join=round] (150.19,151.48) -- (150.19,143.20);

\path[draw=drawColor,line width= 0.6pt,line join=round] (150.19,151.48) -- (150.19,143.20);

\path[draw=drawColor,line width= 0.6pt,line join=round] (150.19,151.48) -- (150.19,143.20);

\path[draw=drawColor,line width= 0.6pt,line join=round] (150.19,151.48) -- (150.19,143.20);

\path[draw=drawColor,line width= 0.6pt,line join=round] (150.19,151.48) -- (150.19,143.20);

\path[draw=drawColor,line width= 0.6pt,line join=round] (150.19,151.48) -- (150.19,143.20);

\path[draw=drawColor,line width= 0.6pt,line join=round] (150.19,151.48) -- (150.19,143.20);

\path[draw=drawColor,line width= 0.6pt,line join=round] (150.19,151.48) -- (150.19,143.20);

\path[draw=drawColor,line width= 0.6pt,line join=round] (150.19,151.48) -- (150.19,143.20);

\path[draw=drawColor,line width= 0.6pt,line join=round] (150.19,151.48) -- (150.19,143.20);

\path[draw=drawColor,line width= 0.6pt,line join=round] (150.19,151.48) -- (150.19,143.20);

\path[draw=drawColor,line width= 0.6pt,line join=round] (150.19,151.48) -- (150.19,143.20);

\path[draw=drawColor,line width= 0.6pt,line join=round] (150.19,151.48) -- (150.19,143.20);

\path[draw=drawColor,line width= 0.6pt,line join=round] (150.19,151.48) -- (150.19,143.20);

\path[draw=drawColor,line width= 0.6pt,line join=round] (150.19,151.48) -- (150.19,143.20);

\path[draw=drawColor,line width= 0.6pt,line join=round] (150.19,151.48) -- (150.19,143.20);

\path[draw=drawColor,line width= 0.6pt,line join=round] (150.19,151.48) -- (150.19,143.20);

\path[draw=drawColor,line width= 0.6pt,line join=round] (150.19,151.48) -- (150.19,143.20);

\path[draw=drawColor,line width= 0.6pt,line join=round] (150.19,151.48) -- (150.19,143.20);

\path[draw=drawColor,line width= 0.6pt,line join=round] (150.19,151.48) -- (150.19,143.20);

\path[draw=drawColor,line width= 0.6pt,line join=round] (150.19,151.48) -- (150.19,143.20);

\path[draw=drawColor,line width= 0.6pt,line join=round] (150.19,151.48) -- (150.19,143.20);

\path[draw=drawColor,line width= 0.6pt,line join=round] (150.19,151.48) -- (150.19,143.20);

\path[draw=drawColor,line width= 0.6pt,line join=round] (150.19,151.48) -- (150.19,143.20);

\path[draw=drawColor,line width= 0.6pt,line join=round] (150.19,151.48) -- (150.19,143.20);

\path[draw=drawColor,line width= 0.6pt,line join=round] (150.19,151.48) -- (150.19,143.20);

\path[draw=drawColor,line width= 0.6pt,line join=round] (150.19,151.48) -- (150.19,143.20);

\path[draw=drawColor,line width= 0.6pt,line join=round] (150.19,151.48) -- (150.19,143.20);

\path[draw=drawColor,line width= 0.6pt,line join=round] (150.19,151.48) -- (150.19,143.20);

\path[draw=drawColor,line width= 0.6pt,line join=round] (150.19,151.48) -- (150.19,143.20);

\path[draw=drawColor,line width= 0.6pt,line join=round] (150.19,151.48) -- (150.19,143.20);

\path[draw=drawColor,line width= 0.6pt,line join=round] (150.19,151.48) -- (150.19,143.20);

\path[draw=drawColor,line width= 0.6pt,line join=round] (150.19,151.48) -- (150.19,143.20);

\path[draw=drawColor,line width= 0.6pt,line join=round] (150.19,151.48) -- (150.19,143.20);

\path[draw=drawColor,line width= 0.6pt,line join=round] (150.19,151.48) -- (150.19,143.20);

\path[draw=drawColor,line width= 0.6pt,line join=round] (150.19,151.48) -- (150.19,143.20);

\path[draw=drawColor,line width= 0.6pt,line join=round] (150.19,151.48) -- (150.19,143.20);

\path[draw=drawColor,line width= 0.6pt,line join=round] (150.19,151.48) -- (150.19,143.20);

\path[draw=drawColor,line width= 0.6pt,line join=round] (150.19,151.48) -- (150.19,143.20);

\path[draw=drawColor,line width= 0.6pt,line join=round] (150.19,151.48) -- (150.19,143.20);

\path[draw=drawColor,line width= 0.6pt,line join=round] (150.19,151.48) -- (150.19,143.20);

\path[draw=drawColor,line width= 0.6pt,line join=round] (150.19,151.48) -- (150.19,143.20);

\path[draw=drawColor,line width= 0.6pt,line join=round] (150.19,151.48) -- (150.19,143.20);

\path[draw=drawColor,line width= 0.6pt,line join=round] (150.19,151.48) -- (150.19,143.20);

\path[draw=drawColor,line width= 0.6pt,line join=round] (150.19,151.48) -- (150.19,143.20);

\path[draw=drawColor,line width= 0.6pt,line join=round] (150.19,151.48) -- (150.19,143.20);

\path[draw=drawColor,line width= 0.6pt,line join=round] (150.19,151.48) -- (150.19,143.20);

\path[draw=drawColor,line width= 0.6pt,line join=round] (150.19,151.48) -- (150.19,143.20);

\path[draw=drawColor,line width= 0.6pt,line join=round] (150.19,151.48) -- (150.19,143.20);

\path[draw=drawColor,line width= 0.6pt,line join=round] (150.19,151.48) -- (150.19,143.20);

\path[draw=drawColor,line width= 0.6pt,line join=round] (150.19,151.48) -- (150.19,143.20);

\path[draw=drawColor,line width= 0.6pt,line join=round] (150.19,151.48) -- (150.19,143.20);

\path[draw=drawColor,line width= 0.6pt,line join=round] (150.19,151.48) -- (150.19,143.20);

\path[draw=drawColor,line width= 0.6pt,line join=round] (150.19,151.48) -- (150.19,143.20);

\path[draw=drawColor,line width= 0.6pt,line join=round] (150.19,151.48) -- (150.19,143.20);

\path[draw=drawColor,line width= 0.6pt,line join=round] (150.19,151.48) -- (150.19,143.20);

\path[draw=drawColor,line width= 0.6pt,line join=round] (150.19,151.48) -- (150.19,143.20);

\path[draw=drawColor,line width= 0.6pt,line join=round] (150.19,151.48) -- (150.19,143.20);

\path[draw=drawColor,line width= 0.6pt,line join=round] (150.19,151.48) -- (150.19,143.20);

\path[draw=drawColor,line width= 0.6pt,line join=round] (150.19,151.48) -- (150.19,143.20);

\path[draw=drawColor,line width= 0.6pt,line join=round] (150.19,151.48) -- (150.19,143.20);

\path[draw=drawColor,line width= 0.6pt,line join=round] (150.19,151.48) -- (150.19,143.20);

\path[draw=drawColor,line width= 0.6pt,line join=round] (150.19,151.48) -- (150.19,143.20);

\path[draw=drawColor,line width= 0.6pt,line join=round] (150.19,151.48) -- (150.19,143.20);

\path[draw=drawColor,line width= 0.6pt,line join=round] (150.19,151.48) -- (150.19,143.20);

\path[draw=drawColor,line width= 0.6pt,line join=round] (150.19,151.48) -- (150.19,143.20);

\path[draw=drawColor,line width= 0.6pt,line join=round] (150.19,151.48) -- (150.19,143.20);

\path[draw=drawColor,line width= 0.6pt,line join=round] (150.19,151.48) -- (150.19,143.20);

\path[draw=drawColor,line width= 0.6pt,line join=round] (150.19,151.48) -- (150.19,143.20);

\path[draw=drawColor,line width= 0.6pt,line join=round] (150.19,151.48) -- (150.19,143.20);

\path[draw=drawColor,line width= 0.6pt,line join=round] (150.19,151.48) -- (150.19,143.20);

\path[draw=drawColor,line width= 0.6pt,line join=round] (150.19,151.48) -- (150.19,143.20);

\path[draw=drawColor,line width= 0.6pt,line join=round] (150.19,151.48) -- (150.19,143.20);

\path[draw=drawColor,line width= 0.6pt,line join=round] (150.19,151.48) -- (150.19,143.20);

\path[draw=drawColor,line width= 0.6pt,line join=round] (150.19,151.48) -- (150.19,143.20);

\path[draw=drawColor,line width= 0.6pt,line join=round] (150.19,151.48) -- (150.19,143.20);

\path[draw=drawColor,line width= 0.6pt,line join=round] (150.19,151.48) -- (150.19,143.20);

\path[draw=drawColor,line width= 0.6pt,line join=round] (150.19,151.48) -- (150.19,143.20);

\path[draw=drawColor,line width= 0.6pt,line join=round] (150.19,151.48) -- (150.19,143.20);

\path[draw=drawColor,line width= 0.6pt,line join=round] (150.19,151.48) -- (150.19,143.20);

\path[draw=drawColor,line width= 0.6pt,line join=round] (150.19,151.48) -- (150.19,143.20);

\path[draw=drawColor,line width= 0.6pt,line join=round] (150.19,151.48) -- (150.19,143.20);

\path[draw=drawColor,line width= 0.6pt,line join=round] (150.19,151.48) -- (150.19,143.20);

\path[draw=drawColor,line width= 0.6pt,line join=round] (150.19,151.48) -- (150.19,143.20);

\path[draw=drawColor,line width= 0.6pt,line join=round] (150.19,151.48) -- (150.19,143.20);

\path[draw=drawColor,line width= 0.6pt,line join=round] (150.19,151.48) -- (150.19,143.20);

\path[draw=drawColor,line width= 0.6pt,line join=round] (150.19,151.48) -- (150.19,143.20);

\path[draw=drawColor,line width= 0.6pt,line join=round] (150.19,151.48) -- (150.19,143.20);

\path[draw=drawColor,line width= 0.6pt,line join=round] (150.19,151.48) -- (150.19,143.20);

\path[draw=drawColor,line width= 0.6pt,line join=round] (150.19,151.48) -- (150.19,143.20);

\path[draw=drawColor,line width= 0.6pt,line join=round] (150.19,151.48) -- (150.19,143.20);

\path[draw=drawColor,line width= 0.6pt,line join=round] (150.19,151.48) -- (150.19,143.20);

\path[draw=drawColor,line width= 0.6pt,line join=round] (150.19,151.48) -- (150.19,143.20);

\path[draw=drawColor,line width= 0.6pt,line join=round] (150.19,151.48) -- (150.19,143.20);

\path[draw=drawColor,line width= 0.6pt,line join=round] (150.19,151.48) -- (150.19,143.20);

\path[draw=drawColor,line width= 0.6pt,line join=round] (150.19,151.48) -- (150.19,143.20);

\path[draw=drawColor,line width= 0.6pt,line join=round] (150.19,151.48) -- (150.19,143.20);

\path[draw=drawColor,line width= 0.6pt,line join=round] (150.19,151.48) -- (150.19,143.20);

\path[draw=drawColor,line width= 0.6pt,line join=round] (150.19,151.48) -- (150.19,143.20);

\path[draw=drawColor,line width= 0.6pt,line join=round] (150.19,151.48) -- (150.19,143.20);

\path[draw=drawColor,line width= 0.6pt,line join=round] (150.19,151.48) -- (150.19,143.20);

\path[draw=drawColor,line width= 0.6pt,line join=round] (150.19,151.48) -- (150.19,143.20);

\path[draw=drawColor,line width= 0.6pt,line join=round] (150.19,151.48) -- (150.19,143.20);

\path[draw=drawColor,line width= 0.6pt,line join=round] (150.19,151.48) -- (150.19,143.20);

\path[draw=drawColor,line width= 0.6pt,line join=round] (150.19,151.48) -- (150.19,143.20);

\path[draw=drawColor,line width= 0.6pt,line join=round] (150.19,151.48) -- (150.19,143.20);

\path[draw=drawColor,line width= 0.6pt,line join=round] (150.19,151.48) -- (150.19,143.20);

\path[draw=drawColor,line width= 0.6pt,line join=round] (150.19,151.48) -- (150.19,143.20);

\path[draw=drawColor,line width= 0.6pt,line join=round] (150.19,151.48) -- (150.19,143.20);

\path[draw=drawColor,line width= 0.6pt,line join=round] (150.19,151.48) -- (150.19,143.20);

\path[draw=drawColor,line width= 0.6pt,line join=round] (150.19,151.48) -- (150.19,143.20);

\path[draw=drawColor,line width= 0.6pt,line join=round] (150.19,151.48) -- (150.19,143.20);

\path[draw=drawColor,line width= 0.6pt,line join=round] (150.19,151.48) -- (150.19,143.20);

\path[draw=drawColor,line width= 0.6pt,line join=round] (150.19,151.48) -- (150.19,143.20);

\path[draw=drawColor,line width= 0.6pt,line join=round] (150.19,151.48) -- (150.19,143.20);

\path[draw=drawColor,line width= 0.6pt,line join=round] (150.19,151.48) -- (150.19,143.20);

\path[draw=drawColor,line width= 0.6pt,line join=round] (150.19,151.48) -- (150.19,143.20);

\path[draw=drawColor,line width= 0.6pt,line join=round] (150.19,151.48) -- (150.19,143.20);

\path[draw=drawColor,line width= 0.6pt,line join=round] (150.19,151.48) -- (150.19,143.20);

\path[draw=drawColor,line width= 0.6pt,line join=round] (150.19,151.48) -- (150.19,143.20);

\path[draw=drawColor,line width= 0.6pt,line join=round] (150.19,151.48) -- (150.19,143.20);

\path[draw=drawColor,line width= 0.6pt,line join=round] (150.19,151.48) -- (150.19,143.20);

\path[draw=drawColor,line width= 0.6pt,line join=round] (150.19,151.48) -- (150.19,143.20);

\path[draw=drawColor,line width= 0.6pt,line join=round] (150.19,151.48) -- (150.19,143.20);

\path[draw=drawColor,line width= 0.6pt,line join=round] (150.19,151.48) -- (150.19,143.20);

\path[draw=drawColor,line width= 0.6pt,line join=round] (150.19,151.48) -- (150.19,143.20);

\path[draw=drawColor,line width= 0.6pt,line join=round] (150.19,151.48) -- (150.19,143.20);

\path[draw=drawColor,line width= 0.6pt,line join=round] (150.19,151.48) -- (150.19,143.20);

\path[draw=drawColor,line width= 0.6pt,line join=round] (150.19,151.48) -- (150.19,143.20);

\path[draw=drawColor,line width= 0.6pt,line join=round] (150.19,151.48) -- (150.19,143.20);

\path[draw=drawColor,line width= 0.6pt,line join=round] (150.19,151.48) -- (150.19,143.20);

\path[draw=drawColor,line width= 0.6pt,line join=round] (150.19,151.48) -- (150.19,143.20);

\path[draw=drawColor,line width= 0.6pt,line join=round] (150.19,151.48) -- (150.19,143.20);

\path[draw=drawColor,line width= 0.6pt,line join=round] (150.19,151.48) -- (150.19,143.20);

\path[draw=drawColor,line width= 0.6pt,line join=round] (150.19,151.48) -- (150.19,143.20);

\path[draw=drawColor,line width= 0.6pt,line join=round] (150.19,151.48) -- (150.19,143.20);

\path[draw=drawColor,line width= 0.6pt,line join=round] (150.19,151.48) -- (150.19,143.20);

\path[draw=drawColor,line width= 0.6pt,line join=round] (150.19,151.48) -- (150.19,143.20);

\path[draw=drawColor,line width= 0.6pt,line join=round] (150.19,151.48) -- (150.19,143.20);

\path[draw=drawColor,line width= 0.6pt,line join=round] (150.19,151.48) -- (150.19,143.20);

\path[draw=drawColor,line width= 0.6pt,line join=round] (150.19,151.48) -- (150.19,143.20);

\path[draw=drawColor,line width= 0.6pt,line join=round] (150.19,151.48) -- (150.19,143.20);

\path[draw=drawColor,line width= 0.6pt,line join=round] (150.19,151.48) -- (150.19,143.20);

\path[draw=drawColor,line width= 0.6pt,line join=round] (150.19,151.48) -- (150.19,143.20);

\path[draw=drawColor,line width= 0.6pt,line join=round] (150.19,151.48) -- (150.19,143.20);

\path[draw=drawColor,line width= 0.6pt,line join=round] (150.19,151.48) -- (150.19,143.20);

\path[draw=drawColor,line width= 0.6pt,line join=round] (150.19,151.48) -- (150.19,143.20);

\path[draw=drawColor,line width= 0.6pt,line join=round] (150.19,151.48) -- (150.19,143.20);

\path[draw=drawColor,line width= 0.6pt,line join=round] (150.19,151.48) -- (150.19,143.20);

\path[draw=drawColor,line width= 0.6pt,line join=round] (150.19,151.48) -- (150.19,143.20);

\path[draw=drawColor,line width= 0.6pt,line join=round] (150.19,151.48) -- (150.19,143.20);

\path[draw=drawColor,line width= 0.6pt,line join=round] (150.19,151.48) -- (150.19,143.20);

\path[draw=drawColor,line width= 0.6pt,line join=round] (150.19,151.48) -- (150.19,143.20);

\path[draw=drawColor,line width= 0.6pt,line join=round] (150.19,151.48) -- (150.19,143.20);

\path[draw=drawColor,line width= 0.6pt,line join=round] (150.19,151.48) -- (150.19,143.20);

\path[draw=drawColor,line width= 0.6pt,line join=round] (150.19,151.48) -- (150.19,143.20);

\path[draw=drawColor,line width= 0.6pt,line join=round] (150.19,151.48) -- (150.19,143.20);

\path[draw=drawColor,line width= 0.6pt,line join=round] (150.19,151.48) -- (150.19,143.20);

\path[draw=drawColor,line width= 0.6pt,line join=round] (150.19,151.48) -- (150.19,143.20);

\path[draw=drawColor,line width= 0.6pt,line join=round] (150.19,151.48) -- (150.19,143.20);

\path[draw=drawColor,line width= 0.6pt,line join=round] (150.19,151.48) -- (150.19,143.20);

\path[draw=drawColor,line width= 0.6pt,line join=round] (150.19,151.48) -- (150.19,143.20);

\path[draw=drawColor,line width= 0.6pt,line join=round] (150.19,151.48) -- (150.19,143.20);

\path[draw=drawColor,line width= 0.6pt,line join=round] (150.19,151.48) -- (150.19,143.20);

\path[draw=drawColor,line width= 0.6pt,line join=round] (150.19,151.48) -- (150.19,143.20);

\path[draw=drawColor,line width= 0.6pt,line join=round] (150.19,151.48) -- (150.19,143.20);

\path[draw=drawColor,line width= 0.6pt,line join=round] (150.19,151.48) -- (150.19,143.20);

\path[draw=drawColor,line width= 0.6pt,line join=round] (150.19,151.48) -- (150.19,143.20);

\path[draw=drawColor,line width= 0.6pt,line join=round] (150.19,151.48) -- (150.19,143.20);

\path[draw=drawColor,line width= 0.6pt,line join=round] (150.19,151.48) -- (150.19,143.20);

\path[draw=drawColor,line width= 0.6pt,line join=round] (150.19,151.48) -- (150.19,143.20);

\path[draw=drawColor,line width= 0.6pt,line join=round] (150.19,151.48) -- (150.19,143.20);

\path[draw=drawColor,line width= 0.6pt,line join=round] (150.19,151.48) -- (150.19,143.20);

\path[draw=drawColor,line width= 0.6pt,line join=round] (150.19,151.48) -- (150.19,143.20);

\path[draw=drawColor,line width= 0.6pt,line join=round] (150.19,151.48) -- (150.19,143.20);

\path[draw=drawColor,line width= 0.6pt,line join=round] (150.19,151.48) -- (150.19,143.20);

\path[draw=drawColor,line width= 0.6pt,line join=round] (150.19,151.48) -- (150.19,143.20);

\path[draw=drawColor,line width= 0.6pt,line join=round] (150.19,151.48) -- (150.19,143.20);

\path[draw=drawColor,line width= 0.6pt,line join=round] (150.19,151.48) -- (150.19,143.20);

\path[draw=drawColor,line width= 0.6pt,line join=round] (150.19,151.48) -- (150.19,143.20);

\path[draw=drawColor,line width= 0.6pt,line join=round] (150.19,151.48) -- (150.19,143.20);

\path[draw=drawColor,line width= 0.6pt,line join=round] (150.19,151.48) -- (150.19,143.20);

\path[draw=drawColor,line width= 0.6pt,line join=round] (150.19,151.48) -- (150.19,143.20);

\path[draw=drawColor,line width= 0.6pt,line join=round] (150.19,151.48) -- (150.19,143.20);

\path[draw=drawColor,line width= 0.6pt,line join=round] (150.19,151.48) -- (150.19,143.20);

\path[draw=drawColor,line width= 0.6pt,line join=round] (150.19,151.48) -- (150.19,143.20);

\path[draw=drawColor,line width= 0.6pt,line join=round] (150.19,151.48) -- (150.19,143.20);

\path[draw=drawColor,line width= 0.6pt,line join=round] (150.19,151.48) -- (150.19,143.20);

\path[draw=drawColor,line width= 0.6pt,line join=round] (150.19,151.48) -- (150.19,143.20);

\path[draw=drawColor,line width= 0.6pt,line join=round] (150.19,151.48) -- (150.19,143.20);

\path[draw=drawColor,line width= 0.6pt,line join=round] (150.19,151.48) -- (150.19,143.20);

\path[draw=drawColor,line width= 0.6pt,line join=round] (150.19,151.48) -- (150.19,143.20);

\path[draw=drawColor,line width= 0.6pt,line join=round] (150.19,151.48) -- (150.19,143.20);

\path[draw=drawColor,line width= 0.6pt,line join=round] (150.19,151.48) -- (150.19,143.20);

\path[draw=drawColor,line width= 0.6pt,line join=round] (150.19,151.48) -- (150.19,143.20);

\path[draw=drawColor,line width= 0.6pt,line join=round] (150.19,151.48) -- (150.19,143.20);

\path[draw=drawColor,line width= 0.6pt,line join=round] (150.19,151.48) -- (150.19,143.20);

\path[draw=drawColor,line width= 0.6pt,line join=round] (150.19,151.48) -- (150.19,143.20);

\path[draw=drawColor,line width= 0.6pt,line join=round] (150.19,151.48) -- (150.19,143.20);

\path[draw=drawColor,line width= 0.6pt,line join=round] (150.19,151.48) -- (150.19,143.20);

\path[draw=drawColor,line width= 0.6pt,line join=round] (150.19,151.48) -- (150.19,143.20);

\path[draw=drawColor,line width= 0.6pt,line join=round] (150.19,151.48) -- (150.19,143.20);

\path[draw=drawColor,line width= 0.6pt,line join=round] (150.19,151.48) -- (150.19,143.20);

\path[draw=drawColor,line width= 0.6pt,line join=round] (150.19,151.48) -- (150.19,143.20);

\path[draw=drawColor,line width= 0.6pt,line join=round] (150.19,151.48) -- (150.19,143.20);

\path[draw=drawColor,line width= 0.6pt,line join=round] (150.19,151.48) -- (150.19,143.20);

\path[draw=drawColor,line width= 0.6pt,line join=round] (150.19,151.48) -- (150.19,143.20);

\path[draw=drawColor,line width= 0.6pt,line join=round] (150.19,151.48) -- (150.19,143.20);

\path[draw=drawColor,line width= 0.6pt,line join=round] (150.19,151.48) -- (150.19,143.20);

\path[draw=drawColor,line width= 0.6pt,line join=round] (150.19,151.48) -- (150.19,143.20);

\path[draw=drawColor,line width= 0.6pt,line join=round] (150.19,151.48) -- (150.19,143.20);

\path[draw=drawColor,line width= 0.6pt,line join=round] (150.19,151.48) -- (150.19,143.20);

\path[draw=drawColor,line width= 0.6pt,line join=round] (150.19,151.48) -- (150.19,143.20);

\path[draw=drawColor,line width= 0.6pt,line join=round] (150.19,151.48) -- (150.19,143.20);

\path[draw=drawColor,line width= 0.6pt,line join=round] (150.19,151.48) -- (150.19,143.20);

\path[draw=drawColor,line width= 0.6pt,line join=round] (150.19,151.48) -- (150.19,143.20);

\path[draw=drawColor,line width= 0.6pt,line join=round] (150.19,151.48) -- (150.19,143.20);

\path[draw=drawColor,line width= 0.6pt,line join=round] (150.19,151.48) -- (150.19,143.20);

\path[draw=drawColor,line width= 0.6pt,line join=round] (150.19,151.48) -- (150.19,143.20);

\path[draw=drawColor,line width= 0.6pt,line join=round] (150.19,151.48) -- (150.19,143.20);

\path[draw=drawColor,line width= 0.6pt,line join=round] (150.19,151.48) -- (150.19,143.20);

\path[draw=drawColor,line width= 0.6pt,line join=round] (150.19,151.48) -- (150.19,143.20);

\path[draw=drawColor,line width= 0.6pt,line join=round] (150.19,151.48) -- (150.19,143.20);

\path[draw=drawColor,line width= 0.6pt,line join=round] (150.19,151.48) -- (150.19,143.20);

\path[draw=drawColor,line width= 0.6pt,line join=round] (150.19,151.48) -- (150.19,143.20);

\path[draw=drawColor,line width= 0.6pt,line join=round] (150.19,151.48) -- (150.19,143.20);

\path[draw=drawColor,line width= 0.6pt,line join=round] (150.19,151.48) -- (150.19,143.20);

\path[draw=drawColor,line width= 0.6pt,line join=round] (150.19,151.48) -- (150.19,143.20);

\path[draw=drawColor,line width= 0.6pt,line join=round] (150.19,151.48) -- (150.19,143.20);

\path[draw=drawColor,line width= 0.6pt,line join=round] (150.19,151.48) -- (150.19,143.20);

\path[draw=drawColor,line width= 0.6pt,line join=round] (150.19,151.48) -- (150.19,143.20);

\path[draw=drawColor,line width= 0.6pt,line join=round] (150.19,151.48) -- (150.19,143.20);

\path[draw=drawColor,line width= 0.6pt,line join=round] (150.19,151.48) -- (150.19,143.20);

\path[draw=drawColor,line width= 0.6pt,line join=round] (150.19,151.48) -- (150.19,143.20);

\path[draw=drawColor,line width= 0.6pt,line join=round] (150.19,151.48) -- (150.19,143.20);

\path[draw=drawColor,line width= 0.6pt,line join=round] (150.19,151.48) -- (150.19,143.20);

\path[draw=drawColor,line width= 0.6pt,line join=round] (150.19,151.48) -- (150.19,143.20);

\path[draw=drawColor,line width= 0.6pt,line join=round] (150.19,151.48) -- (150.19,143.20);

\path[draw=drawColor,line width= 0.6pt,line join=round] (150.19,151.48) -- (150.19,143.20);

\path[draw=drawColor,line width= 0.6pt,line join=round] (150.19,151.48) -- (150.19,143.20);

\path[draw=drawColor,line width= 0.6pt,line join=round] (150.19,151.48) -- (150.19,143.20);

\path[draw=drawColor,line width= 0.6pt,line join=round] (150.19,151.48) -- (150.19,143.20);

\path[draw=drawColor,line width= 0.6pt,line join=round] (150.19,151.48) -- (150.19,143.20);

\path[draw=drawColor,line width= 0.6pt,line join=round] (150.19,151.48) -- (150.19,143.20);

\path[draw=drawColor,line width= 0.6pt,line join=round] (150.19,151.48) -- (150.19,143.20);

\path[draw=drawColor,line width= 0.6pt,line join=round] (150.19,151.48) -- (150.19,143.20);

\path[draw=drawColor,line width= 0.6pt,line join=round] (150.19,151.48) -- (150.19,143.20);

\path[draw=drawColor,line width= 0.6pt,line join=round] (150.19,151.48) -- (150.19,143.20);

\path[draw=drawColor,line width= 0.6pt,line join=round] (150.19,151.48) -- (150.19,143.20);

\path[draw=drawColor,line width= 0.6pt,line join=round] (150.19,151.48) -- (150.19,143.20);

\path[draw=drawColor,line width= 0.6pt,line join=round] (150.19,151.48) -- (150.19,143.20);

\path[draw=drawColor,line width= 0.6pt,line join=round] (150.19,151.48) -- (150.19,143.20);

\path[draw=drawColor,line width= 0.6pt,line join=round] (150.19,151.48) -- (150.19,143.20);

\path[draw=drawColor,line width= 0.6pt,line join=round] (150.19,151.48) -- (150.19,143.20);

\path[draw=drawColor,line width= 0.6pt,line join=round] (150.19,151.48) -- (150.19,143.20);

\path[draw=drawColor,line width= 0.6pt,line join=round] (150.19,151.48) -- (150.19,143.20);

\path[draw=drawColor,line width= 0.6pt,line join=round] (150.19,151.48) -- (150.19,143.20);

\path[draw=drawColor,line width= 0.6pt,line join=round] (150.19,151.48) -- (150.19,143.20);

\path[draw=drawColor,line width= 0.6pt,line join=round] (150.19,151.48) -- (150.19,143.20);

\path[draw=drawColor,line width= 0.6pt,line join=round] (150.19,151.48) -- (150.19,143.20);

\path[draw=drawColor,line width= 0.6pt,line join=round] (150.19,151.48) -- (150.19,143.20);

\path[draw=drawColor,line width= 0.6pt,line join=round] (150.19,151.48) -- (150.19,143.20);

\path[draw=drawColor,line width= 0.6pt,line join=round] (150.19,151.48) -- (150.19,143.20);

\path[draw=drawColor,line width= 0.6pt,line join=round] (150.19,151.48) -- (150.19,143.20);

\path[draw=drawColor,line width= 0.6pt,line join=round] (150.19,151.48) -- (150.19,143.20);

\path[draw=drawColor,line width= 0.6pt,line join=round] (150.19,151.48) -- (150.19,143.20);

\path[draw=drawColor,line width= 0.6pt,line join=round] (150.19,151.48) -- (150.19,143.20);

\path[draw=drawColor,line width= 0.6pt,line join=round] (150.19,151.48) -- (150.19,143.20);

\path[draw=drawColor,line width= 0.6pt,line join=round] (150.19,151.48) -- (150.19,143.20);

\path[draw=drawColor,line width= 0.6pt,line join=round] (150.19,151.48) -- (150.19,143.20);

\path[draw=drawColor,line width= 0.6pt,line join=round] (150.19,151.48) -- (150.19,143.20);

\path[draw=drawColor,line width= 0.6pt,line join=round] (150.19,151.48) -- (150.19,143.20);

\path[draw=drawColor,line width= 0.6pt,line join=round] (150.19,151.48) -- (150.19,143.20);

\path[draw=drawColor,line width= 0.6pt,line join=round] (150.19,151.48) -- (150.19,143.20);

\path[draw=drawColor,line width= 0.6pt,line join=round] (150.19,151.48) -- (150.19,143.20);

\path[draw=drawColor,line width= 0.6pt,line join=round] (150.19,151.48) -- (150.19,143.20);

\path[draw=drawColor,line width= 0.6pt,line join=round] (150.19,151.48) -- (150.19,143.20);

\path[draw=drawColor,line width= 0.6pt,line join=round] (150.19,151.48) -- (150.19,143.20);

\path[draw=drawColor,line width= 0.6pt,line join=round] (150.19,151.48) -- (150.19,143.20);

\path[draw=drawColor,line width= 0.6pt,line join=round] (150.19,151.48) -- (150.19,143.20);

\path[draw=drawColor,line width= 0.6pt,line join=round] (150.19,151.48) -- (150.19,143.20);

\path[draw=drawColor,line width= 0.6pt,line join=round] (150.19,151.48) -- (150.19,143.20);

\path[draw=drawColor,line width= 0.6pt,line join=round] (150.19,151.48) -- (150.19,143.20);

\path[draw=drawColor,line width= 0.6pt,line join=round] (150.19,151.48) -- (150.19,143.20);

\path[draw=drawColor,line width= 0.6pt,line join=round] (150.19,151.48) -- (150.19,143.20);

\path[draw=drawColor,line width= 0.6pt,line join=round] (150.19,151.48) -- (150.19,143.20);

\path[draw=drawColor,line width= 0.6pt,line join=round] (150.19,151.48) -- (150.19,143.20);

\path[draw=drawColor,line width= 0.6pt,line join=round] (150.19,151.48) -- (150.19,143.20);

\path[draw=drawColor,line width= 0.6pt,line join=round] (150.19,151.48) -- (150.19,143.20);

\path[draw=drawColor,line width= 0.6pt,line join=round] (150.19,151.48) -- (150.19,143.20);

\path[draw=drawColor,line width= 0.6pt,line join=round] (150.19,151.48) -- (150.19,143.20);

\path[draw=drawColor,line width= 0.6pt,line join=round] (150.19,151.48) -- (150.19,143.20);

\path[draw=drawColor,line width= 0.6pt,line join=round] (150.19,151.48) -- (150.19,143.20);

\path[draw=drawColor,line width= 0.6pt,line join=round] (150.19,151.48) -- (150.19,143.20);

\path[draw=drawColor,line width= 0.6pt,line join=round] (150.19,151.48) -- (150.19,143.20);

\path[draw=drawColor,line width= 0.6pt,line join=round] (150.19,151.48) -- (150.19,143.20);

\path[draw=drawColor,line width= 0.6pt,line join=round] (150.19,151.48) -- (150.19,143.20);

\path[draw=drawColor,line width= 0.6pt,line join=round] (150.19,151.48) -- (150.19,143.20);

\path[draw=drawColor,line width= 0.6pt,line join=round] (150.19,151.48) -- (150.19,143.20);

\path[draw=drawColor,line width= 0.6pt,line join=round] (150.19,151.48) -- (150.19,143.20);

\path[draw=drawColor,line width= 0.6pt,line join=round] (150.19,151.48) -- (150.19,143.20);

\path[draw=drawColor,line width= 0.6pt,line join=round] (150.19,151.48) -- (150.19,143.20);

\path[draw=drawColor,line width= 0.6pt,line join=round] (150.19,151.48) -- (150.19,143.20);

\path[draw=drawColor,line width= 0.6pt,line join=round] (150.19,151.48) -- (150.19,143.20);

\path[draw=drawColor,line width= 0.6pt,line join=round] (150.19,151.48) -- (150.19,143.20);

\path[draw=drawColor,line width= 0.6pt,line join=round] (150.19,151.48) -- (150.19,143.20);

\path[draw=drawColor,line width= 0.6pt,line join=round] (150.19,151.48) -- (150.19,143.20);

\path[draw=drawColor,line width= 0.6pt,line join=round] (150.19,151.48) -- (150.19,143.20);

\path[draw=drawColor,line width= 0.6pt,line join=round] (150.19,151.48) -- (150.19,143.20);

\path[draw=drawColor,line width= 0.6pt,line join=round] (150.19,151.48) -- (150.19,143.20);

\path[draw=drawColor,line width= 0.6pt,line join=round] (150.19,151.48) -- (150.19,143.20);

\path[draw=drawColor,line width= 0.6pt,line join=round] (150.19,151.48) -- (150.19,143.20);

\path[draw=drawColor,line width= 0.6pt,line join=round] (150.19,151.48) -- (150.19,143.20);

\path[draw=drawColor,line width= 0.6pt,line join=round] (150.19,151.48) -- (150.19,143.20);

\path[draw=drawColor,line width= 0.6pt,line join=round] (150.19,151.48) -- (150.19,143.20);

\path[draw=drawColor,line width= 0.6pt,line join=round] (150.19,151.48) -- (150.19,143.20);

\path[draw=drawColor,line width= 0.6pt,line join=round] (150.19,151.48) -- (150.19,143.20);

\path[draw=drawColor,line width= 0.6pt,line join=round] (150.19,151.48) -- (150.19,143.20);

\path[draw=drawColor,line width= 0.6pt,line join=round] (150.19,151.48) -- (150.19,143.20);

\path[draw=drawColor,line width= 0.6pt,line join=round] (150.19,151.48) -- (150.19,143.20);

\path[draw=drawColor,line width= 0.6pt,line join=round] (150.19,151.48) -- (150.19,143.20);

\path[draw=drawColor,line width= 0.6pt,line join=round] (150.19,151.48) -- (150.19,143.20);

\path[draw=drawColor,line width= 0.6pt,line join=round] (150.19,151.48) -- (150.19,143.20);

\path[draw=drawColor,line width= 0.6pt,line join=round] (150.19,151.48) -- (150.19,143.20);

\path[draw=drawColor,line width= 0.6pt,line join=round] (150.19,151.48) -- (150.19,143.20);

\path[draw=drawColor,line width= 0.6pt,line join=round] (150.19,151.48) -- (150.19,143.20);

\path[draw=drawColor,line width= 0.6pt,line join=round] (150.19,151.48) -- (150.19,143.20);

\path[draw=drawColor,line width= 0.6pt,line join=round] (150.19,151.48) -- (150.19,143.20);

\path[draw=drawColor,line width= 0.6pt,line join=round] (150.19,151.48) -- (150.19,143.20);

\path[draw=drawColor,line width= 0.6pt,line join=round] (150.19,151.48) -- (150.19,143.20);

\path[draw=drawColor,line width= 0.6pt,line join=round] (150.19,151.48) -- (150.19,143.20);

\path[draw=drawColor,line width= 0.6pt,line join=round] (150.19,151.48) -- (150.19,143.20);

\path[draw=drawColor,line width= 0.6pt,line join=round] (150.19,151.48) -- (150.19,143.20);

\path[draw=drawColor,line width= 0.6pt,line join=round] (150.19,151.48) -- (150.19,143.20);

\path[draw=drawColor,line width= 0.6pt,line join=round] (150.19,151.48) -- (150.19,143.20);

\path[draw=drawColor,line width= 0.6pt,line join=round] (150.19,151.48) -- (150.19,143.20);

\path[draw=drawColor,line width= 0.6pt,line join=round] (150.19,151.48) -- (150.19,143.20);

\path[draw=drawColor,line width= 0.6pt,line join=round] (150.19,151.48) -- (150.19,143.20);

\path[draw=drawColor,line width= 0.6pt,line join=round] (150.19,151.48) -- (150.19,143.20);

\path[draw=drawColor,line width= 0.6pt,line join=round] (150.19,151.48) -- (150.19,143.20);

\path[draw=drawColor,line width= 0.6pt,line join=round] (150.19,151.48) -- (150.19,143.20);

\path[draw=drawColor,line width= 0.6pt,line join=round] (150.19,151.48) -- (150.19,143.20);

\path[draw=drawColor,line width= 0.6pt,line join=round] (150.19,151.48) -- (150.19,143.20);

\path[draw=drawColor,line width= 0.6pt,line join=round] (150.19,151.48) -- (150.19,143.20);

\path[draw=drawColor,line width= 0.6pt,line join=round] (150.19,151.48) -- (150.19,143.20);

\path[draw=drawColor,line width= 0.6pt,line join=round] (150.19,151.48) -- (150.19,143.20);

\path[draw=drawColor,line width= 0.6pt,line join=round] (150.19,151.48) -- (150.19,143.20);

\path[draw=drawColor,line width= 0.6pt,line join=round] (150.19,151.48) -- (150.19,143.20);

\path[draw=drawColor,line width= 0.6pt,line join=round] (150.19,151.48) -- (150.19,143.20);

\path[draw=drawColor,line width= 0.6pt,line join=round] (150.19,151.48) -- (150.19,143.20);

\path[draw=drawColor,line width= 0.6pt,line join=round] (150.19,151.48) -- (150.19,143.20);

\path[draw=drawColor,line width= 0.6pt,line join=round] (150.19,151.48) -- (150.19,143.20);

\path[draw=drawColor,line width= 0.6pt,line join=round] (150.19,151.48) -- (150.19,143.20);

\path[draw=drawColor,line width= 0.6pt,line join=round] (150.19,151.48) -- (150.19,143.20);

\path[draw=drawColor,line width= 0.6pt,line join=round] (150.19,151.48) -- (150.19,143.20);

\path[draw=drawColor,line width= 0.6pt,line join=round] (150.19,151.48) -- (150.19,143.20);

\path[draw=drawColor,line width= 0.6pt,line join=round] (150.19,151.48) -- (150.19,143.20);

\path[draw=drawColor,line width= 0.6pt,line join=round] (150.19,151.48) -- (150.19,143.20);

\path[draw=drawColor,line width= 0.6pt,line join=round] (150.19,151.48) -- (150.19,143.20);

\path[draw=drawColor,line width= 0.6pt,line join=round] (150.19,151.48) -- (150.19,143.20);

\path[draw=drawColor,line width= 0.6pt,line join=round] (150.19,151.48) -- (150.19,143.20);

\path[draw=drawColor,line width= 0.6pt,line join=round] (150.19,151.48) -- (150.19,143.20);

\path[draw=drawColor,line width= 0.6pt,line join=round] (150.19,151.48) -- (150.19,143.20);

\path[draw=drawColor,line width= 0.6pt,line join=round] (150.19,151.48) -- (150.19,143.20);

\path[draw=drawColor,line width= 0.6pt,line join=round] (150.19,151.48) -- (150.19,143.20);

\path[draw=drawColor,line width= 0.6pt,line join=round] (150.19,151.48) -- (150.19,143.20);

\path[draw=drawColor,line width= 0.6pt,line join=round] (150.19,151.48) -- (150.19,143.20);

\path[draw=drawColor,line width= 0.6pt,line join=round] (150.19,151.48) -- (150.19,143.20);

\path[draw=drawColor,line width= 0.6pt,line join=round] (150.19,151.48) -- (150.19,143.20);

\path[draw=drawColor,line width= 0.6pt,line join=round] (150.19,151.48) -- (150.19,143.20);

\path[draw=drawColor,line width= 0.6pt,line join=round] (150.19,151.48) -- (150.19,143.20);

\path[draw=drawColor,line width= 0.6pt,line join=round] (150.19,151.48) -- (150.19,143.20);

\path[draw=drawColor,line width= 0.6pt,line join=round] (150.19,151.48) -- (150.19,143.20);

\path[draw=drawColor,line width= 0.6pt,line join=round] (150.19,151.48) -- (150.19,143.20);

\path[draw=drawColor,line width= 0.6pt,line join=round] (150.19,151.48) -- (150.19,143.20);

\path[draw=drawColor,line width= 0.6pt,line join=round] (150.19,151.48) -- (150.19,143.20);

\path[draw=drawColor,line width= 0.6pt,line join=round] (150.19,151.48) -- (150.19,143.20);

\path[draw=drawColor,line width= 0.6pt,line join=round] (150.19,151.48) -- (150.19,143.20);

\path[draw=drawColor,line width= 0.6pt,line join=round] (150.19,151.48) -- (150.19,143.20);

\path[draw=drawColor,line width= 0.6pt,line join=round] (150.19,151.48) -- (150.19,143.20);

\path[draw=drawColor,line width= 0.6pt,line join=round] (150.19,151.48) -- (150.19,143.20);

\path[draw=drawColor,line width= 0.6pt,line join=round] (150.19,151.48) -- (150.19,143.20);

\path[draw=drawColor,line width= 0.6pt,line join=round] (150.19,151.48) -- (150.19,143.20);

\path[draw=drawColor,line width= 0.6pt,line join=round] (150.19,151.48) -- (150.19,143.20);

\path[draw=drawColor,line width= 0.6pt,line join=round] (150.19,151.48) -- (150.19,143.20);

\path[draw=drawColor,line width= 0.6pt,line join=round] (150.19,151.48) -- (150.19,143.20);

\path[draw=drawColor,line width= 0.6pt,line join=round] (150.19,151.48) -- (150.19,143.20);

\path[draw=drawColor,line width= 0.6pt,line join=round] (150.19,151.48) -- (150.19,143.20);

\path[draw=drawColor,line width= 0.6pt,line join=round] (150.19,151.48) -- (150.19,143.20);

\path[draw=drawColor,line width= 0.6pt,line join=round] (150.19,151.48) -- (150.19,143.20);

\path[draw=drawColor,line width= 0.6pt,line join=round] (150.19,151.48) -- (150.19,143.20);

\path[draw=drawColor,line width= 0.6pt,line join=round] (150.19,151.48) -- (150.19,143.20);

\path[draw=drawColor,line width= 0.6pt,line join=round] (150.19,151.48) -- (150.19,143.20);

\path[draw=drawColor,line width= 0.6pt,line join=round] (150.19,151.48) -- (150.19,143.20);

\path[draw=drawColor,line width= 0.6pt,line join=round] (150.19,151.48) -- (150.19,143.20);

\path[draw=drawColor,line width= 0.6pt,line join=round] (150.19,151.48) -- (150.19,143.20);

\path[draw=drawColor,line width= 0.6pt,line join=round] (150.19,151.48) -- (150.19,143.20);

\path[draw=drawColor,line width= 0.6pt,line join=round] (150.19,151.48) -- (150.19,143.20);

\path[draw=drawColor,line width= 0.6pt,line join=round] (150.19,151.48) -- (150.19,143.20);

\path[draw=drawColor,line width= 0.6pt,line join=round] (150.19,151.48) -- (150.19,143.20);

\path[draw=drawColor,line width= 0.6pt,line join=round] (150.19,151.48) -- (150.19,143.20);

\path[draw=drawColor,line width= 0.6pt,line join=round] (150.19,151.48) -- (150.19,143.20);

\path[draw=drawColor,line width= 0.6pt,line join=round] (150.19,151.48) -- (150.19,143.20);

\path[draw=drawColor,line width= 0.6pt,line join=round] (150.19,151.48) -- (150.19,143.20);

\path[draw=drawColor,line width= 0.6pt,line join=round] (150.19,151.48) -- (150.19,143.20);

\path[draw=drawColor,line width= 0.6pt,line join=round] (150.19,151.48) -- (150.19,143.20);

\path[draw=drawColor,line width= 0.6pt,line join=round] (150.19,151.48) -- (150.19,143.20);

\path[draw=drawColor,line width= 0.6pt,line join=round] (150.19,151.48) -- (150.19,143.20);

\path[draw=drawColor,line width= 0.6pt,line join=round] (150.19,151.48) -- (150.19,143.20);

\path[draw=drawColor,line width= 0.6pt,line join=round] (150.19,151.48) -- (150.19,143.20);

\path[draw=drawColor,line width= 0.6pt,line join=round] (150.19,151.48) -- (150.19,143.20);

\path[draw=drawColor,line width= 0.6pt,line join=round] (150.19,151.48) -- (150.19,143.20);

\path[draw=drawColor,line width= 0.6pt,line join=round] (150.19,151.48) -- (150.19,143.20);

\path[draw=drawColor,line width= 0.6pt,line join=round] (150.19,151.48) -- (150.19,143.20);

\path[draw=drawColor,line width= 0.6pt,line join=round] (150.19,151.48) -- (150.19,143.20);

\path[draw=drawColor,line width= 0.6pt,line join=round] (150.19,151.48) -- (150.19,143.20);

\path[draw=drawColor,line width= 0.6pt,line join=round] (150.19,151.48) -- (150.19,143.20);

\path[draw=drawColor,line width= 0.6pt,line join=round] (150.19,151.48) -- (150.19,143.20);

\path[draw=drawColor,line width= 0.6pt,line join=round] (150.19,151.48) -- (150.19,143.20);

\path[draw=drawColor,line width= 0.6pt,line join=round] (150.19,151.48) -- (150.19,143.20);

\path[draw=drawColor,line width= 0.6pt,line join=round] (150.19,151.48) -- (150.19,143.20);

\path[draw=drawColor,line width= 0.6pt,line join=round] (150.19,151.48) -- (150.19,143.20);

\path[draw=drawColor,line width= 0.6pt,line join=round] (150.19,151.48) -- (150.19,143.20);

\path[draw=drawColor,line width= 0.6pt,line join=round] (150.19,151.48) -- (150.19,143.20);

\path[draw=drawColor,line width= 0.6pt,line join=round] (150.19,151.48) -- (150.19,143.20);

\path[draw=drawColor,line width= 0.6pt,line join=round] (150.19,151.48) -- (150.19,143.20);

\path[draw=drawColor,line width= 0.6pt,line join=round] (150.19,151.48) -- (150.19,143.20);

\path[draw=drawColor,line width= 0.6pt,line join=round] (150.19,151.48) -- (150.19,143.20);

\path[draw=drawColor,line width= 0.6pt,line join=round] (150.19,151.48) -- (150.19,143.20);

\path[draw=drawColor,line width= 0.6pt,line join=round] (150.19,151.48) -- (150.19,143.20);

\path[draw=drawColor,line width= 0.6pt,line join=round] (150.19,151.48) -- (150.19,143.20);

\path[draw=drawColor,line width= 0.6pt,line join=round] (150.19,151.48) -- (150.19,143.20);

\path[draw=drawColor,line width= 0.6pt,line join=round] (150.19,151.48) -- (150.19,143.20);

\path[draw=drawColor,line width= 0.6pt,line join=round] (150.19,151.48) -- (150.19,143.20);

\path[draw=drawColor,line width= 0.6pt,line join=round] (150.19,151.48) -- (150.19,143.20);

\path[draw=drawColor,line width= 0.6pt,line join=round] (150.19,151.48) -- (150.19,143.20);

\path[draw=drawColor,line width= 0.6pt,line join=round] (150.19,151.48) -- (150.19,143.20);

\path[draw=drawColor,line width= 0.6pt,line join=round] (150.19,151.48) -- (150.19,143.20);

\path[draw=drawColor,line width= 0.6pt,line join=round] (150.19,151.48) -- (150.19,143.20);

\path[draw=drawColor,line width= 0.6pt,line join=round] (150.19,151.48) -- (150.19,143.20);

\path[draw=drawColor,line width= 0.6pt,line join=round] (150.19,151.48) -- (150.19,143.20);

\path[draw=drawColor,line width= 0.6pt,line join=round] (150.19,151.48) -- (150.19,143.20);

\path[draw=drawColor,line width= 0.6pt,line join=round] (150.19,151.48) -- (150.19,143.20);

\path[draw=drawColor,line width= 0.6pt,line join=round] (150.19,151.48) -- (150.19,143.20);

\path[draw=drawColor,line width= 0.6pt,line join=round] (150.19,151.48) -- (150.19,143.20);

\path[draw=drawColor,line width= 0.6pt,line join=round] (150.19,151.48) -- (150.19,143.20);

\path[draw=drawColor,line width= 0.6pt,line join=round] (150.19,151.48) -- (150.19,143.20);

\path[draw=drawColor,line width= 0.6pt,line join=round] (150.19,151.48) -- (150.19,143.20);

\path[draw=drawColor,line width= 0.6pt,line join=round] (150.19,151.48) -- (150.19,143.20);

\path[draw=drawColor,line width= 0.6pt,line join=round] (150.19,151.48) -- (150.19,143.20);

\path[draw=drawColor,line width= 0.6pt,line join=round] (150.19,151.48) -- (150.19,143.20);

\path[draw=drawColor,line width= 0.6pt,line join=round] (150.19,151.48) -- (150.19,143.20);

\path[draw=drawColor,line width= 0.6pt,line join=round] (150.19,151.48) -- (150.19,143.20);

\path[draw=drawColor,line width= 0.6pt,line join=round] (150.19,151.48) -- (150.19,143.20);

\path[draw=drawColor,line width= 0.6pt,line join=round] (150.19,151.48) -- (150.19,143.20);

\path[draw=drawColor,line width= 0.6pt,line join=round] (150.19,151.48) -- (150.19,143.20);

\path[draw=drawColor,line width= 0.6pt,line join=round] (150.19,151.48) -- (150.19,143.20);

\path[draw=drawColor,line width= 0.6pt,line join=round] (150.19,151.48) -- (150.19,143.20);

\path[draw=drawColor,line width= 0.6pt,line join=round] (150.19,151.48) -- (150.19,143.20);

\path[draw=drawColor,line width= 0.6pt,line join=round] (150.19,151.48) -- (150.19,143.20);

\path[draw=drawColor,line width= 0.6pt,line join=round] (150.19,151.48) -- (150.19,143.20);

\path[draw=drawColor,line width= 0.6pt,line join=round] (150.19,151.48) -- (150.19,143.20);

\path[draw=drawColor,line width= 0.6pt,line join=round] (150.19,151.48) -- (150.19,143.20);

\path[draw=drawColor,line width= 0.6pt,line join=round] (150.19,151.48) -- (150.19,143.20);

\path[draw=drawColor,line width= 0.6pt,line join=round] (150.19,151.48) -- (150.19,143.20);

\path[draw=drawColor,line width= 0.6pt,line join=round] (150.19,151.48) -- (150.19,143.20);

\path[draw=drawColor,line width= 0.6pt,line join=round] (150.19,151.48) -- (150.19,143.20);

\path[draw=drawColor,line width= 0.6pt,line join=round] (150.19,151.48) -- (150.19,143.20);

\path[draw=drawColor,line width= 0.6pt,line join=round] (150.19,151.48) -- (150.19,143.20);

\path[draw=drawColor,line width= 0.6pt,line join=round] (150.19,151.48) -- (150.19,143.20);

\path[draw=drawColor,line width= 0.6pt,line join=round] (150.19,151.48) -- (150.19,143.20);

\path[draw=drawColor,line width= 0.6pt,line join=round] (150.19,151.48) -- (150.19,143.20);

\path[draw=drawColor,line width= 0.6pt,line join=round] (150.19,151.48) -- (150.19,143.20);

\path[draw=drawColor,line width= 0.6pt,line join=round] (150.19,151.48) -- (150.19,143.20);

\path[draw=drawColor,line width= 0.6pt,line join=round] (150.19,151.48) -- (150.19,143.20);

\path[draw=drawColor,line width= 0.6pt,line join=round] (150.19,151.48) -- (150.19,143.20);

\path[draw=drawColor,line width= 0.6pt,line join=round] (150.19,151.48) -- (150.19,143.20);

\path[draw=drawColor,line width= 0.6pt,line join=round] (150.19,151.48) -- (150.19,143.20);

\path[draw=drawColor,line width= 0.6pt,line join=round] (150.19,151.48) -- (150.19,143.20);

\path[draw=drawColor,line width= 0.6pt,line join=round] (150.19,151.48) -- (150.19,143.20);

\path[draw=drawColor,line width= 0.6pt,line join=round] (150.19,151.48) -- (150.19,143.20);

\path[draw=drawColor,line width= 0.6pt,line join=round] (150.19,151.48) -- (150.19,143.20);

\path[draw=drawColor,line width= 0.6pt,line join=round] (150.19,151.48) -- (150.19,143.20);

\path[draw=drawColor,line width= 0.6pt,line join=round] (150.19,151.48) -- (150.19,143.20);

\path[draw=drawColor,line width= 0.6pt,line join=round] (150.19,151.48) -- (150.19,143.20);

\path[draw=drawColor,line width= 0.6pt,line join=round] (150.19,151.48) -- (150.19,143.20);

\path[draw=drawColor,line width= 0.6pt,line join=round] (150.19,151.48) -- (150.19,143.20);

\path[draw=drawColor,line width= 0.6pt,line join=round] (150.19,151.48) -- (150.19,143.20);

\path[draw=drawColor,line width= 0.6pt,line join=round] (150.19,151.48) -- (150.19,143.20);

\path[draw=drawColor,line width= 0.6pt,line join=round] (150.19,151.48) -- (150.19,143.20);

\path[draw=drawColor,line width= 0.6pt,line join=round] (150.19,151.48) -- (150.19,143.20);

\path[draw=drawColor,line width= 0.6pt,line join=round] (150.19,151.48) -- (150.19,143.20);

\path[draw=drawColor,line width= 0.6pt,line join=round] (150.19,151.48) -- (150.19,143.20);

\path[draw=drawColor,line width= 0.6pt,line join=round] (150.19,151.48) -- (150.19,143.20);

\path[draw=drawColor,line width= 0.6pt,line join=round] (150.19,151.48) -- (150.19,143.20);

\path[draw=drawColor,line width= 0.6pt,line join=round] (150.19,151.48) -- (150.19,143.20);

\path[draw=drawColor,line width= 0.6pt,line join=round] (150.19,151.48) -- (150.19,143.20);

\path[draw=drawColor,line width= 0.6pt,line join=round] (150.19,151.48) -- (150.19,143.20);

\path[draw=drawColor,line width= 0.6pt,line join=round] (150.19,151.48) -- (150.19,143.20);

\path[draw=drawColor,line width= 0.6pt,line join=round] (150.19,151.48) -- (150.19,143.20);

\path[draw=drawColor,line width= 0.6pt,line join=round] (150.19,151.48) -- (150.19,143.20);

\path[draw=drawColor,line width= 0.6pt,line join=round] (150.19,151.48) -- (150.19,143.20);

\path[draw=drawColor,line width= 0.6pt,line join=round] (150.19,151.48) -- (150.19,143.20);

\path[draw=drawColor,line width= 0.6pt,line join=round] (150.19,151.48) -- (150.19,143.20);

\path[draw=drawColor,line width= 0.6pt,line join=round] (150.19,151.48) -- (150.19,143.20);

\path[draw=drawColor,line width= 0.6pt,line join=round] (150.19,151.48) -- (150.19,143.20);

\path[draw=drawColor,line width= 0.6pt,line join=round] (150.19,151.48) -- (150.19,143.20);

\path[draw=drawColor,line width= 0.6pt,line join=round] (150.19,151.48) -- (150.19,143.20);

\path[draw=drawColor,line width= 0.6pt,line join=round] (150.19,151.48) -- (150.19,143.20);

\path[draw=drawColor,line width= 0.6pt,line join=round] (150.19,151.48) -- (150.19,143.20);

\path[draw=drawColor,line width= 0.6pt,line join=round] (150.19,151.48) -- (150.19,143.20);

\path[draw=drawColor,line width= 0.6pt,line join=round] (150.19,151.48) -- (150.19,143.20);

\path[draw=drawColor,line width= 0.6pt,line join=round] (150.19,151.48) -- (150.19,143.20);

\path[draw=drawColor,line width= 0.6pt,line join=round] (150.19,151.48) -- (150.19,143.20);

\path[draw=drawColor,line width= 0.6pt,line join=round] (150.19,151.48) -- (150.19,143.20);

\path[draw=drawColor,line width= 0.6pt,line join=round] (150.19,151.48) -- (150.19,143.20);

\path[draw=drawColor,line width= 0.6pt,line join=round] (150.19,151.48) -- (150.19,143.20);

\path[draw=drawColor,line width= 0.6pt,line join=round] (150.19,151.48) -- (150.19,143.20);

\path[draw=drawColor,line width= 0.6pt,line join=round] (150.19,151.48) -- (150.19,143.20);

\path[draw=drawColor,line width= 0.6pt,line join=round] (150.19,151.48) -- (150.19,143.20);

\path[draw=drawColor,line width= 0.6pt,line join=round] (150.19,151.48) -- (150.19,143.20);

\path[draw=drawColor,line width= 0.6pt,line join=round] (150.19,151.48) -- (150.19,143.20);

\path[draw=drawColor,line width= 0.6pt,line join=round] (150.19,151.48) -- (150.19,143.20);

\path[draw=drawColor,line width= 0.6pt,line join=round] (150.19,151.48) -- (150.19,143.20);

\path[draw=drawColor,line width= 0.6pt,line join=round] (150.19,151.48) -- (150.19,143.20);

\path[draw=drawColor,line width= 0.6pt,line join=round] (150.19,151.48) -- (150.19,143.20);

\path[draw=drawColor,line width= 0.6pt,line join=round] (150.19,151.48) -- (150.19,143.20);

\path[draw=drawColor,line width= 0.6pt,line join=round] (150.19,151.48) -- (150.19,143.20);

\path[draw=drawColor,line width= 0.6pt,line join=round] (150.19,151.48) -- (150.19,143.20);

\path[draw=drawColor,line width= 0.6pt,line join=round] (150.19,151.48) -- (150.19,143.20);

\path[draw=drawColor,line width= 0.6pt,line join=round] (150.19,151.48) -- (150.19,143.20);

\path[draw=drawColor,line width= 0.6pt,line join=round] (150.19,151.48) -- (150.19,143.20);

\path[draw=drawColor,line width= 0.6pt,line join=round] (150.19,151.48) -- (150.19,143.20);

\path[draw=drawColor,line width= 0.6pt,line join=round] (150.19,151.48) -- (150.19,143.20);

\path[draw=drawColor,line width= 0.6pt,line join=round] (150.19,151.48) -- (150.19,143.20);

\path[draw=drawColor,line width= 0.6pt,line join=round] (150.19,151.48) -- (150.19,143.20);

\path[draw=drawColor,line width= 0.6pt,line join=round] (150.19,151.48) -- (150.19,143.20);

\path[draw=drawColor,line width= 0.6pt,line join=round] (150.19,151.48) -- (150.19,143.20);

\path[draw=drawColor,line width= 0.6pt,line join=round] (150.19,151.48) -- (150.19,143.20);

\path[draw=drawColor,line width= 0.6pt,line join=round] (150.19,151.48) -- (150.19,143.20);

\path[draw=drawColor,line width= 0.6pt,line join=round] (150.19,151.48) -- (150.19,143.20);

\path[draw=drawColor,line width= 0.6pt,line join=round] (150.19,151.48) -- (150.19,143.20);

\path[draw=drawColor,line width= 0.6pt,line join=round] (150.19,151.48) -- (150.19,143.20);

\path[draw=drawColor,line width= 0.6pt,line join=round] (150.19,151.48) -- (150.19,143.20);

\path[draw=drawColor,line width= 0.6pt,line join=round] (150.19,151.48) -- (150.19,143.20);

\path[draw=drawColor,line width= 0.6pt,line join=round] (150.19,151.48) -- (150.19,143.20);

\path[draw=drawColor,line width= 0.6pt,line join=round] (150.19,151.48) -- (150.19,143.20);

\path[draw=drawColor,line width= 0.6pt,line join=round] (150.19,151.48) -- (150.19,143.20);

\path[draw=drawColor,line width= 0.6pt,line join=round] (150.19,151.48) -- (150.19,143.20);

\path[draw=drawColor,line width= 0.6pt,line join=round] (150.19,151.48) -- (150.19,143.20);

\path[draw=drawColor,line width= 0.6pt,line join=round] (150.19,151.48) -- (150.19,143.20);

\path[draw=drawColor,line width= 0.6pt,line join=round] (150.19,151.48) -- (150.19,143.20);

\path[draw=drawColor,line width= 0.6pt,line join=round] (150.19,151.48) -- (150.19,143.20);

\path[draw=drawColor,line width= 0.6pt,line join=round] (150.19,151.48) -- (150.19,143.20);

\path[draw=drawColor,line width= 0.6pt,line join=round] (150.19,151.48) -- (150.19,143.20);

\path[draw=drawColor,line width= 0.6pt,line join=round] (150.19,151.48) -- (150.19,143.20);

\path[draw=drawColor,line width= 0.6pt,line join=round] (150.19,151.48) -- (150.19,143.20);

\path[draw=drawColor,line width= 0.6pt,line join=round] (150.19,151.48) -- (150.19,143.20);

\path[draw=drawColor,line width= 0.6pt,line join=round] (150.19,151.48) -- (150.19,143.20);

\path[draw=drawColor,line width= 0.6pt,line join=round] (150.19,151.48) -- (150.19,143.20);

\path[draw=drawColor,line width= 0.6pt,line join=round] (150.19,151.48) -- (150.19,143.20);

\path[draw=drawColor,line width= 0.6pt,line join=round] (150.19,151.48) -- (150.19,143.20);

\path[draw=drawColor,line width= 0.6pt,line join=round] (150.19,151.48) -- (150.19,143.20);

\path[draw=drawColor,line width= 0.6pt,line join=round] (150.19,151.48) -- (150.19,143.20);

\path[draw=drawColor,line width= 0.6pt,line join=round] (150.19,151.48) -- (150.19,143.20);

\path[draw=drawColor,line width= 0.6pt,line join=round] (150.19,151.48) -- (150.19,143.20);

\path[draw=drawColor,line width= 0.6pt,line join=round] (150.19,151.48) -- (150.19,143.20);

\path[draw=drawColor,line width= 0.6pt,line join=round] (150.19,151.48) -- (150.19,143.20);

\path[draw=drawColor,line width= 0.6pt,line join=round] (150.19,151.48) -- (150.19,143.20);

\path[draw=drawColor,line width= 0.6pt,line join=round] (150.19,151.48) -- (150.19,143.20);

\path[draw=drawColor,line width= 0.6pt,line join=round] (150.19,151.48) -- (150.19,143.20);

\path[draw=drawColor,line width= 0.6pt,line join=round] (150.19,151.48) -- (150.19,143.20);

\path[draw=drawColor,line width= 0.6pt,line join=round] (150.19,151.48) -- (150.19,143.20);

\path[draw=drawColor,line width= 0.6pt,line join=round] (150.19,151.48) -- (150.19,143.20);

\path[draw=drawColor,line width= 0.6pt,line join=round] (150.19,151.48) -- (150.19,143.20);

\path[draw=drawColor,line width= 0.6pt,line join=round] (150.19,151.48) -- (150.19,143.20);

\path[draw=drawColor,line width= 0.6pt,line join=round] (150.19,151.48) -- (150.19,143.20);

\path[draw=drawColor,line width= 0.6pt,line join=round] (150.19,151.48) -- (150.19,143.20);

\path[draw=drawColor,line width= 0.6pt,line join=round] (150.19,151.48) -- (150.19,143.20);

\path[draw=drawColor,line width= 0.6pt,line join=round] (150.19,151.48) -- (150.19,143.20);

\path[draw=drawColor,line width= 0.6pt,line join=round] (150.19,151.48) -- (150.19,143.20);

\path[draw=drawColor,line width= 0.6pt,line join=round] (150.19,151.48) -- (150.19,143.20);

\path[draw=drawColor,line width= 0.6pt,line join=round] (150.19,151.48) -- (150.19,143.20);

\path[draw=drawColor,line width= 0.6pt,line join=round] (150.19,151.48) -- (150.19,143.20);

\path[draw=drawColor,line width= 0.6pt,line join=round] (150.19,151.48) -- (150.19,143.20);

\path[draw=drawColor,line width= 0.6pt,line join=round] (150.19,151.48) -- (150.19,143.20);

\path[draw=drawColor,line width= 0.6pt,line join=round] (150.19,151.48) -- (150.19,143.20);

\path[draw=drawColor,line width= 0.6pt,line join=round] (150.19,151.48) -- (150.19,143.20);

\path[draw=drawColor,line width= 0.6pt,line join=round] (150.19,151.48) -- (150.19,143.20);

\path[draw=drawColor,line width= 0.6pt,line join=round] (150.19,151.48) -- (150.19,143.20);

\path[draw=drawColor,line width= 0.6pt,line join=round] (150.19,151.48) -- (150.19,143.20);

\path[draw=drawColor,line width= 0.6pt,line join=round] (150.19,151.48) -- (150.19,143.20);

\path[draw=drawColor,line width= 0.6pt,line join=round] (150.19,151.48) -- (150.19,143.20);

\path[draw=drawColor,line width= 0.6pt,line join=round] (150.19,151.48) -- (150.19,143.20);

\path[draw=drawColor,line width= 0.6pt,line join=round] (150.19,151.48) -- (150.19,143.20);

\path[draw=drawColor,line width= 0.6pt,line join=round] (150.19,151.48) -- (150.19,143.20);

\path[draw=drawColor,line width= 0.6pt,line join=round] (150.19,151.48) -- (150.19,143.20);

\path[draw=drawColor,line width= 0.6pt,line join=round] (150.19,151.48) -- (150.19,143.20);

\path[draw=drawColor,line width= 0.6pt,line join=round] (150.19,151.48) -- (150.19,143.20);

\path[draw=drawColor,line width= 0.6pt,line join=round] (150.19,151.48) -- (150.19,143.20);

\path[draw=drawColor,line width= 0.6pt,line join=round] (150.19,151.48) -- (150.19,143.20);

\path[draw=drawColor,line width= 0.6pt,line join=round] (150.19,151.48) -- (150.19,143.20);

\path[draw=drawColor,line width= 0.6pt,line join=round] (150.19,151.48) -- (150.19,143.20);

\path[draw=drawColor,line width= 0.6pt,line join=round] (150.19,151.48) -- (150.19,143.20);

\path[draw=drawColor,line width= 0.6pt,line join=round] (150.19,151.48) -- (150.19,143.20);

\path[draw=drawColor,line width= 0.6pt,line join=round] (150.19,151.48) -- (150.19,143.20);

\path[draw=drawColor,line width= 0.6pt,line join=round] (150.19,151.48) -- (150.19,143.20);

\path[draw=drawColor,line width= 0.6pt,line join=round] (150.19,151.48) -- (150.19,143.20);

\path[draw=drawColor,line width= 0.6pt,line join=round] (150.19,151.48) -- (150.19,143.20);

\path[draw=drawColor,line width= 0.6pt,line join=round] (150.19,151.48) -- (150.19,143.20);

\path[draw=drawColor,line width= 0.6pt,line join=round] (150.19,151.48) -- (150.19,143.20);

\path[draw=drawColor,line width= 0.6pt,line join=round] (150.19,151.48) -- (150.19,143.20);

\path[draw=drawColor,line width= 0.6pt,line join=round] (150.19,151.48) -- (150.19,143.20);

\path[draw=drawColor,line width= 0.6pt,line join=round] (150.19,151.48) -- (150.19,143.20);

\path[draw=drawColor,line width= 0.6pt,line join=round] (150.19,151.48) -- (150.19,143.20);

\path[draw=drawColor,line width= 0.6pt,line join=round] (150.19,151.48) -- (150.19,143.20);

\path[draw=drawColor,line width= 0.6pt,line join=round] (150.19,151.48) -- (150.19,143.20);

\path[draw=drawColor,line width= 0.6pt,line join=round] (150.19,151.48) -- (150.19,143.20);

\path[draw=drawColor,line width= 0.6pt,line join=round] (150.19,151.48) -- (150.19,143.20);

\path[draw=drawColor,line width= 0.6pt,line join=round] (150.19,151.48) -- (150.19,143.20);

\path[draw=drawColor,line width= 0.6pt,line join=round] (150.19,151.48) -- (150.19,143.20);

\path[draw=drawColor,line width= 0.6pt,line join=round] (150.19,151.48) -- (150.19,143.20);

\path[draw=drawColor,line width= 0.6pt,line join=round] (150.19,151.48) -- (150.19,143.20);

\path[draw=drawColor,line width= 0.6pt,line join=round] (150.19,151.48) -- (150.19,143.20);

\path[draw=drawColor,line width= 0.6pt,line join=round] (150.19,151.48) -- (150.19,143.20);

\path[draw=drawColor,line width= 0.6pt,line join=round] (150.19,151.48) -- (150.19,143.20);

\path[draw=drawColor,line width= 0.6pt,line join=round] (150.19,151.48) -- (150.19,143.20);

\path[draw=drawColor,line width= 0.6pt,line join=round] (150.19,151.48) -- (150.19,143.20);

\path[draw=drawColor,line width= 0.6pt,line join=round] (150.19,151.48) -- (150.19,143.20);

\path[draw=drawColor,line width= 0.6pt,line join=round] (150.19,151.48) -- (150.19,143.20);

\path[draw=drawColor,line width= 0.6pt,line join=round] (150.19,151.48) -- (150.19,143.20);

\path[draw=drawColor,line width= 0.6pt,line join=round] (150.19,151.48) -- (150.19,143.20);

\path[draw=drawColor,line width= 0.6pt,line join=round] (150.19,151.48) -- (150.19,143.20);

\path[draw=drawColor,line width= 0.6pt,line join=round] (150.19,151.48) -- (150.19,143.20);

\path[draw=drawColor,line width= 0.6pt,line join=round] (150.19,151.48) -- (150.19,143.20);

\path[draw=drawColor,line width= 0.6pt,line join=round] (150.19,151.48) -- (150.19,143.20);

\path[draw=drawColor,line width= 0.6pt,line join=round] (150.19,151.48) -- (150.19,143.20);

\path[draw=drawColor,line width= 0.6pt,line join=round] (150.19,151.48) -- (150.19,143.20);

\path[draw=drawColor,line width= 0.6pt,line join=round] (150.19,151.48) -- (150.19,143.20);

\path[draw=drawColor,line width= 0.6pt,line join=round] (150.19,151.48) -- (150.19,143.20);

\path[draw=drawColor,line width= 0.6pt,line join=round] (150.19,151.48) -- (150.19,143.20);

\path[draw=drawColor,line width= 0.6pt,line join=round] (150.19,151.48) -- (150.19,143.20);

\path[draw=drawColor,line width= 0.6pt,line join=round] (150.19,151.48) -- (150.19,143.20);

\path[draw=drawColor,line width= 0.6pt,line join=round] (150.19,151.48) -- (150.19,143.20);

\path[draw=drawColor,line width= 0.6pt,line join=round] (150.19,151.48) -- (150.19,143.20);

\path[draw=drawColor,line width= 0.6pt,line join=round] (150.19,151.48) -- (150.19,143.20);

\path[draw=drawColor,line width= 0.6pt,line join=round] (150.19,151.48) -- (150.19,143.20);

\path[draw=drawColor,line width= 0.6pt,line join=round] (150.19,151.48) -- (150.19,143.20);

\path[draw=drawColor,line width= 0.6pt,line join=round] (150.19,151.48) -- (150.19,143.20);

\path[draw=drawColor,line width= 0.6pt,line join=round] (150.19,151.48) -- (150.19,143.20);

\path[draw=drawColor,line width= 0.6pt,line join=round] (150.19,151.48) -- (150.19,143.20);

\path[draw=drawColor,line width= 0.6pt,line join=round] (150.19,151.48) -- (150.19,143.20);

\path[draw=drawColor,line width= 0.6pt,line join=round] (150.19,151.48) -- (150.19,143.20);

\path[draw=drawColor,line width= 0.6pt,line join=round] (150.19,151.48) -- (150.19,143.20);

\path[draw=drawColor,line width= 0.6pt,line join=round] (150.19,151.48) -- (150.19,143.20);

\path[draw=drawColor,line width= 0.6pt,line join=round] (150.19,151.48) -- (150.19,143.20);

\path[draw=drawColor,line width= 0.6pt,line join=round] (150.19,151.48) -- (150.19,143.20);

\path[draw=drawColor,line width= 0.6pt,line join=round] (150.19,151.48) -- (150.19,143.20);

\path[draw=drawColor,line width= 0.6pt,line join=round] (150.19,151.48) -- (150.19,143.20);

\path[draw=drawColor,line width= 0.6pt,line join=round] (150.19,151.48) -- (150.19,143.20);

\path[draw=drawColor,line width= 0.6pt,line join=round] (150.19,151.48) -- (150.19,143.20);

\path[draw=drawColor,line width= 0.6pt,line join=round] (150.19,151.48) -- (150.19,143.20);

\path[draw=drawColor,line width= 0.6pt,line join=round] (150.19,151.48) -- (150.19,143.20);

\path[draw=drawColor,line width= 0.6pt,line join=round] (150.19,151.48) -- (150.19,143.20);

\path[draw=drawColor,line width= 0.6pt,line join=round] (150.19,151.48) -- (150.19,143.20);

\path[draw=drawColor,line width= 0.6pt,line join=round] (150.19,151.48) -- (150.19,143.20);

\path[draw=drawColor,line width= 0.6pt,line join=round] (150.19,151.48) -- (150.19,143.20);

\path[draw=drawColor,line width= 0.6pt,line join=round] (150.19,151.48) -- (150.19,143.20);

\path[draw=drawColor,line width= 0.6pt,line join=round] (150.19,151.48) -- (150.19,143.20);

\path[draw=drawColor,line width= 0.6pt,line join=round] (150.19,151.48) -- (150.19,143.20);

\path[draw=drawColor,line width= 0.6pt,line join=round] (150.19,151.48) -- (150.19,143.20);

\path[draw=drawColor,line width= 0.6pt,line join=round] (150.19,151.48) -- (150.19,143.20);

\path[draw=drawColor,line width= 0.6pt,line join=round] (150.19,151.48) -- (150.19,143.20);

\path[draw=drawColor,line width= 0.6pt,line join=round] (150.19,151.48) -- (150.19,143.20);

\path[draw=drawColor,line width= 0.6pt,line join=round] (150.19,151.48) -- (150.19,143.20);

\path[draw=drawColor,line width= 0.6pt,line join=round] (150.19,151.48) -- (150.19,143.20);

\path[draw=drawColor,line width= 0.6pt,line join=round] (150.19,151.48) -- (150.19,143.20);

\path[draw=drawColor,line width= 0.6pt,line join=round] (150.19,151.48) -- (150.19,143.20);

\path[draw=drawColor,line width= 0.6pt,line join=round] (150.19,151.48) -- (150.19,143.20);

\path[draw=drawColor,line width= 0.6pt,line join=round] (150.19,151.48) -- (150.19,143.20);

\path[draw=drawColor,line width= 0.6pt,line join=round] (150.19,151.48) -- (150.19,143.20);

\path[draw=drawColor,line width= 0.6pt,line join=round] (150.19,151.48) -- (150.19,143.20);

\path[draw=drawColor,line width= 0.6pt,line join=round] (150.19,151.48) -- (150.19,143.20);

\path[draw=drawColor,line width= 0.6pt,line join=round] (150.19,151.48) -- (150.19,143.20);

\path[draw=drawColor,line width= 0.6pt,line join=round] (150.19,151.48) -- (150.19,143.20);

\path[draw=drawColor,line width= 0.6pt,line join=round] (150.19,151.48) -- (150.19,143.20);

\path[draw=drawColor,line width= 0.6pt,line join=round] (150.19,151.48) -- (150.19,143.20);

\path[draw=drawColor,line width= 0.6pt,line join=round] (150.19,151.48) -- (150.19,143.20);

\path[draw=drawColor,line width= 0.6pt,line join=round] (150.19,151.48) -- (150.19,143.20);

\path[draw=drawColor,line width= 0.6pt,line join=round] (150.19,151.48) -- (150.19,143.20);

\path[draw=drawColor,line width= 0.6pt,line join=round] (150.19,151.48) -- (150.19,143.20);

\path[draw=drawColor,line width= 0.6pt,line join=round] (150.19,151.48) -- (150.19,143.20);

\path[draw=drawColor,line width= 0.6pt,line join=round] (150.19,151.48) -- (150.19,143.20);

\path[draw=drawColor,line width= 0.6pt,line join=round] (150.19,151.48) -- (150.19,143.20);

\path[draw=drawColor,line width= 0.6pt,line join=round] (150.19,151.48) -- (150.19,143.20);

\path[draw=drawColor,line width= 0.6pt,line join=round] (150.19,151.48) -- (150.19,143.20);

\path[draw=drawColor,line width= 0.6pt,line join=round] (150.19,151.48) -- (150.19,143.20);

\path[draw=drawColor,line width= 0.6pt,line join=round] (150.19,151.48) -- (150.19,143.20);

\path[draw=drawColor,line width= 0.6pt,line join=round] (150.19,151.48) -- (150.19,143.20);

\path[draw=drawColor,line width= 0.6pt,line join=round] (150.19,151.48) -- (150.19,143.20);

\path[draw=drawColor,line width= 0.6pt,line join=round] (150.19,151.48) -- (150.19,143.20);

\path[draw=drawColor,line width= 0.6pt,line join=round] (150.19,151.48) -- (150.19,143.20);

\path[draw=drawColor,line width= 0.6pt,line join=round] (150.19,151.48) -- (150.19,143.20);

\path[draw=drawColor,line width= 0.6pt,line join=round] (150.19,151.48) -- (150.19,143.20);

\path[draw=drawColor,line width= 0.6pt,line join=round] (150.19,151.48) -- (150.19,143.20);

\path[draw=drawColor,line width= 0.6pt,line join=round] (150.19,151.48) -- (150.19,143.20);

\path[draw=drawColor,line width= 0.6pt,line join=round] (150.19,151.48) -- (150.19,143.20);

\path[draw=drawColor,line width= 0.6pt,line join=round] (150.19,151.48) -- (150.19,143.20);

\path[draw=drawColor,line width= 0.6pt,line join=round] (150.19,151.48) -- (150.19,143.20);

\path[draw=drawColor,line width= 0.6pt,line join=round] (150.19,151.48) -- (150.19,143.20);

\path[draw=drawColor,line width= 0.6pt,line join=round] (150.19,151.48) -- (150.19,143.20);

\path[draw=drawColor,line width= 0.6pt,line join=round] (150.19,151.48) -- (150.19,143.20);

\path[draw=drawColor,line width= 0.6pt,line join=round] (150.19,151.48) -- (150.19,143.20);

\path[draw=drawColor,line width= 0.6pt,line join=round] (150.19,151.48) -- (150.19,143.20);

\path[draw=drawColor,line width= 0.6pt,line join=round] (150.19,151.48) -- (150.19,143.20);

\path[draw=drawColor,line width= 0.6pt,line join=round] (150.19,151.48) -- (150.19,143.20);

\path[draw=drawColor,line width= 0.6pt,line join=round] (150.19,151.48) -- (150.19,143.20);

\path[draw=drawColor,line width= 0.6pt,line join=round] (150.19,151.48) -- (150.19,143.20);

\path[draw=drawColor,line width= 0.6pt,line join=round] (150.19,151.48) -- (150.19,143.20);

\path[draw=drawColor,line width= 0.6pt,line join=round] (150.19,151.48) -- (150.19,143.20);

\path[draw=drawColor,line width= 0.6pt,line join=round] (150.19,151.48) -- (150.19,143.20);

\path[draw=drawColor,line width= 0.6pt,line join=round] (150.19,151.48) -- (150.19,143.20);

\path[draw=drawColor,line width= 0.6pt,line join=round] (150.19,151.48) -- (150.19,143.20);

\path[draw=drawColor,line width= 0.6pt,line join=round] (150.19,151.48) -- (150.19,143.20);

\path[draw=drawColor,line width= 0.6pt,line join=round] (150.19,151.48) -- (150.19,143.20);

\path[draw=drawColor,line width= 0.6pt,line join=round] (150.19,151.48) -- (150.19,143.20);

\path[draw=drawColor,line width= 0.6pt,line join=round] (150.19,151.48) -- (150.19,143.20);

\path[draw=drawColor,line width= 0.6pt,line join=round] (150.19,151.48) -- (150.19,143.20);

\path[draw=drawColor,line width= 0.6pt,line join=round] (150.19,151.48) -- (150.19,143.20);

\path[draw=drawColor,line width= 0.6pt,line join=round] (150.19,151.48) -- (150.19,143.20);

\path[draw=drawColor,line width= 0.6pt,line join=round] (150.19,151.48) -- (150.19,143.20);

\path[draw=drawColor,line width= 0.6pt,line join=round] (150.19,151.48) -- (150.19,143.20);

\path[draw=drawColor,line width= 0.6pt,line join=round] (150.19,151.48) -- (150.19,143.20);

\path[draw=drawColor,line width= 0.6pt,line join=round] (150.19,151.48) -- (150.19,143.20);

\path[draw=drawColor,line width= 0.6pt,line join=round] (150.19,151.48) -- (150.19,143.20);

\path[draw=drawColor,line width= 0.6pt,line join=round] (150.19,151.48) -- (150.19,143.20);

\path[draw=drawColor,line width= 0.6pt,line join=round] (150.19,151.48) -- (150.19,143.20);

\path[draw=drawColor,line width= 0.6pt,line join=round] (150.19,151.48) -- (150.19,143.20);

\path[draw=drawColor,line width= 0.6pt,line join=round] (150.19,151.48) -- (150.19,143.20);

\path[draw=drawColor,line width= 0.6pt,line join=round] (150.19,151.48) -- (150.19,143.20);

\path[draw=drawColor,line width= 0.6pt,line join=round] (150.19,151.48) -- (150.19,143.20);

\path[draw=drawColor,line width= 0.6pt,line join=round] (150.19,151.48) -- (150.19,143.20);

\path[draw=drawColor,line width= 0.6pt,line join=round] (150.19,151.48) -- (150.19,143.20);

\path[draw=drawColor,line width= 0.6pt,line join=round] (150.19,151.48) -- (150.19,143.20);

\path[draw=drawColor,line width= 0.6pt,line join=round] (150.19,151.48) -- (150.19,143.20);

\path[draw=drawColor,line width= 0.6pt,line join=round] (150.19,151.48) -- (150.19,143.20);

\path[draw=drawColor,line width= 0.6pt,line join=round] (150.19,151.48) -- (150.19,143.20);

\path[draw=drawColor,line width= 0.6pt,line join=round] (150.19,151.48) -- (150.19,143.20);

\path[draw=drawColor,line width= 0.6pt,line join=round] (150.19,151.48) -- (150.19,143.20);

\path[draw=drawColor,line width= 0.6pt,line join=round] (150.19,151.48) -- (150.19,143.20);

\path[draw=drawColor,line width= 0.6pt,line join=round] (150.19,151.48) -- (150.19,143.20);

\path[draw=drawColor,line width= 0.6pt,line join=round] (150.19,151.48) -- (150.19,143.20);

\path[draw=drawColor,line width= 0.6pt,line join=round] (150.19,151.48) -- (150.19,143.20);

\path[draw=drawColor,line width= 0.6pt,line join=round] (150.19,151.48) -- (150.19,143.20);

\path[draw=drawColor,line width= 0.6pt,line join=round] (150.19,151.48) -- (150.19,143.20);

\path[draw=drawColor,line width= 0.6pt,line join=round] (150.19,151.48) -- (150.19,143.20);

\path[draw=drawColor,line width= 0.6pt,line join=round] (150.19,151.48) -- (150.19,143.20);

\path[draw=drawColor,line width= 0.6pt,line join=round] (150.19,151.48) -- (150.19,143.20);

\path[draw=drawColor,line width= 0.6pt,line join=round] (150.19,151.48) -- (150.19,143.20);

\path[draw=drawColor,line width= 0.6pt,line join=round] (150.19,151.48) -- (150.19,143.20);

\path[draw=drawColor,line width= 0.6pt,line join=round] (150.19,151.48) -- (150.19,143.20);

\path[draw=drawColor,line width= 0.6pt,line join=round] (150.19,151.48) -- (150.19,143.20);

\path[draw=drawColor,line width= 0.6pt,line join=round] (150.19,151.48) -- (150.19,143.20);

\path[draw=drawColor,line width= 0.6pt,line join=round] (150.19,151.48) -- (150.19,143.20);

\path[draw=drawColor,line width= 0.6pt,line join=round] (150.19,151.48) -- (150.19,143.20);

\path[draw=drawColor,line width= 0.6pt,line join=round] (150.19,151.48) -- (150.19,143.20);

\path[draw=drawColor,line width= 0.6pt,line join=round] (150.19,151.48) -- (150.19,143.20);

\path[draw=drawColor,line width= 0.6pt,line join=round] (150.19,151.48) -- (150.19,143.20);

\path[draw=drawColor,line width= 0.6pt,line join=round] (150.19,151.48) -- (150.19,143.20);

\path[draw=drawColor,line width= 0.6pt,line join=round] (150.19,151.48) -- (150.19,143.20);

\path[draw=drawColor,line width= 0.6pt,line join=round] (150.19,151.48) -- (150.19,143.20);

\path[draw=drawColor,line width= 0.6pt,line join=round] (150.19,151.48) -- (150.19,143.20);

\path[draw=drawColor,line width= 0.6pt,line join=round] (150.19,151.48) -- (150.19,143.20);

\path[draw=drawColor,line width= 0.6pt,line join=round] (150.19,151.48) -- (150.19,143.20);

\path[draw=drawColor,line width= 0.6pt,line join=round] (150.19,151.48) -- (150.19,143.20);

\path[draw=drawColor,line width= 0.6pt,line join=round] (150.19,151.48) -- (150.19,143.20);

\path[draw=drawColor,line width= 0.6pt,line join=round] (150.19,151.48) -- (150.19,143.20);

\path[draw=drawColor,line width= 0.6pt,line join=round] (150.19,151.48) -- (150.19,143.20);

\path[draw=drawColor,line width= 0.6pt,line join=round] (150.19,151.48) -- (150.19,143.20);

\path[draw=drawColor,line width= 0.6pt,line join=round] (150.19,151.48) -- (150.19,143.20);

\path[draw=drawColor,line width= 0.6pt,line join=round] (150.19,151.48) -- (150.19,143.20);

\path[draw=drawColor,line width= 0.6pt,line join=round] (150.19,151.48) -- (150.19,143.20);

\path[draw=drawColor,line width= 0.6pt,line join=round] (150.19,151.48) -- (150.19,143.20);

\path[draw=drawColor,line width= 0.6pt,line join=round] (150.19,151.48) -- (150.19,143.20);

\path[draw=drawColor,line width= 0.6pt,line join=round] (150.19,151.48) -- (150.19,143.20);

\path[draw=drawColor,line width= 0.6pt,line join=round] (150.19,151.48) -- (150.19,143.20);

\path[draw=drawColor,line width= 0.6pt,line join=round] (150.19,151.48) -- (150.19,143.20);

\path[draw=drawColor,line width= 0.6pt,line join=round] (150.19,151.48) -- (150.19,143.20);

\path[draw=drawColor,line width= 0.6pt,line join=round] (150.19,151.48) -- (150.19,143.20);

\path[draw=drawColor,line width= 0.6pt,line join=round] (150.19,151.48) -- (150.19,143.20);

\path[draw=drawColor,line width= 0.6pt,line join=round] (150.19,151.48) -- (150.19,143.20);

\path[draw=drawColor,line width= 0.6pt,line join=round] (150.19,151.48) -- (150.19,143.20);

\path[draw=drawColor,line width= 0.6pt,line join=round] (150.19,151.48) -- (150.19,143.20);

\path[draw=drawColor,line width= 0.6pt,line join=round] (150.19,151.48) -- (150.19,143.20);

\path[draw=drawColor,line width= 0.6pt,line join=round] (150.19,151.48) -- (150.19,143.20);

\path[draw=drawColor,line width= 0.6pt,line join=round] (150.19,151.48) -- (150.19,143.20);

\path[draw=drawColor,line width= 0.6pt,line join=round] (150.19,151.48) -- (150.19,143.20);

\path[draw=drawColor,line width= 0.6pt,line join=round] (150.19,151.48) -- (150.19,143.20);

\path[draw=drawColor,line width= 0.6pt,line join=round] (150.19,151.48) -- (150.19,143.20);

\path[draw=drawColor,line width= 0.6pt,line join=round] (150.19,151.48) -- (150.19,143.20);

\path[draw=drawColor,line width= 0.6pt,line join=round] (150.19,151.48) -- (150.19,143.20);

\path[draw=drawColor,line width= 0.6pt,line join=round] (150.19,151.48) -- (150.19,143.20);

\path[draw=drawColor,line width= 0.6pt,line join=round] (150.19,151.48) -- (150.19,143.20);

\path[draw=drawColor,line width= 0.6pt,line join=round] (150.19,151.48) -- (150.19,143.20);

\path[draw=drawColor,line width= 0.6pt,line join=round] (150.19,151.48) -- (150.19,143.20);

\path[draw=drawColor,line width= 0.6pt,line join=round] (150.19,151.48) -- (150.19,143.20);

\path[draw=drawColor,line width= 0.6pt,line join=round] (150.19,151.48) -- (150.19,143.20);

\path[draw=drawColor,line width= 0.6pt,line join=round] (150.19,151.48) -- (150.19,143.20);

\path[draw=drawColor,line width= 0.6pt,line join=round] (150.19,151.48) -- (150.19,143.20);

\path[draw=drawColor,line width= 0.6pt,line join=round] (150.19,151.48) -- (150.19,143.20);

\path[draw=drawColor,line width= 0.6pt,line join=round] (150.19,151.48) -- (150.19,143.20);

\path[draw=drawColor,line width= 0.6pt,line join=round] (150.19,151.48) -- (150.19,143.20);

\path[draw=drawColor,line width= 0.6pt,line join=round] (150.19,151.48) -- (150.19,143.20);

\path[draw=drawColor,line width= 0.6pt,line join=round] (150.19,151.48) -- (150.19,143.20);

\path[draw=drawColor,line width= 0.6pt,line join=round] (150.19,151.48) -- (150.19,143.20);

\path[draw=drawColor,line width= 0.6pt,line join=round] (150.19,151.48) -- (150.19,143.20);

\path[draw=drawColor,line width= 0.6pt,line join=round] (150.19,151.48) -- (150.19,143.20);

\path[draw=drawColor,line width= 0.6pt,line join=round] (150.19,151.48) -- (150.19,143.20);

\path[draw=drawColor,line width= 0.6pt,line join=round] (150.19,151.48) -- (150.19,143.20);

\path[draw=drawColor,line width= 0.6pt,line join=round] (150.19,151.48) -- (150.19,143.20);

\path[draw=drawColor,line width= 0.6pt,line join=round] (150.19,151.48) -- (150.19,143.20);

\path[draw=drawColor,line width= 0.6pt,line join=round] (150.19,151.48) -- (150.19,143.20);

\path[draw=drawColor,line width= 0.6pt,line join=round] (150.19,151.48) -- (150.19,143.20);

\path[draw=drawColor,line width= 0.6pt,line join=round] (150.19,151.48) -- (150.19,143.20);

\path[draw=drawColor,line width= 0.6pt,line join=round] (150.19,151.48) -- (150.19,143.20);

\path[draw=drawColor,line width= 0.6pt,line join=round] (150.19,151.48) -- (150.19,143.20);

\path[draw=drawColor,line width= 0.6pt,line join=round] (150.19,151.48) -- (150.19,143.20);

\path[draw=drawColor,line width= 0.6pt,line join=round] (150.19,151.48) -- (150.19,143.20);

\path[draw=drawColor,line width= 0.6pt,line join=round] (150.19,151.48) -- (150.19,143.20);

\path[draw=drawColor,line width= 0.6pt,line join=round] (150.19,151.48) -- (150.19,143.20);

\path[draw=drawColor,line width= 0.6pt,line join=round] (150.19,151.48) -- (150.19,143.20);

\path[draw=drawColor,line width= 0.6pt,line join=round] (150.19,151.48) -- (150.19,143.20);

\path[draw=drawColor,line width= 0.6pt,line join=round] (150.19,151.48) -- (150.19,143.20);

\path[draw=drawColor,line width= 0.6pt,line join=round] (150.19,151.48) -- (150.19,143.20);

\path[draw=drawColor,line width= 0.6pt,line join=round] (150.19,151.48) -- (150.19,143.20);

\path[draw=drawColor,line width= 0.6pt,line join=round] (150.19,151.48) -- (150.19,143.20);

\path[draw=drawColor,line width= 0.6pt,line join=round] (150.19,151.48) -- (150.19,143.20);

\path[draw=drawColor,line width= 0.6pt,line join=round] (150.19,151.48) -- (150.19,143.20);

\path[draw=drawColor,line width= 0.6pt,line join=round] (150.19,151.48) -- (150.19,143.20);

\path[draw=drawColor,line width= 0.6pt,line join=round] (150.19,151.48) -- (150.19,143.20);

\path[draw=drawColor,line width= 0.6pt,line join=round] (150.19,151.48) -- (150.19,143.20);

\path[draw=drawColor,line width= 0.6pt,line join=round] (150.19,151.48) -- (150.19,143.20);

\path[draw=drawColor,line width= 0.6pt,line join=round] (150.19,151.48) -- (150.19,143.20);

\path[draw=drawColor,line width= 0.6pt,line join=round] (150.19,151.48) -- (150.19,143.20);

\path[draw=drawColor,line width= 0.6pt,line join=round] (150.19,151.48) -- (150.19,143.20);

\path[draw=drawColor,line width= 0.6pt,line join=round] (150.19,151.48) -- (150.19,143.20);

\path[draw=drawColor,line width= 0.6pt,line join=round] (150.19,151.48) -- (150.19,143.20);

\path[draw=drawColor,line width= 0.6pt,line join=round] (150.19,151.48) -- (150.19,143.20);

\path[draw=drawColor,line width= 0.6pt,line join=round] (150.19,151.48) -- (150.19,143.20);

\path[draw=drawColor,line width= 0.6pt,line join=round] (150.19,151.48) -- (150.19,143.20);

\path[draw=drawColor,line width= 0.6pt,line join=round] (150.19,151.48) -- (150.19,143.20);

\path[draw=drawColor,line width= 0.6pt,line join=round] (150.19,151.48) -- (150.19,143.20);

\path[draw=drawColor,line width= 0.6pt,line join=round] (150.19,151.48) -- (150.19,143.20);

\path[draw=drawColor,line width= 0.6pt,line join=round] (150.19,151.48) -- (150.19,143.20);

\path[draw=drawColor,line width= 0.6pt,line join=round] (150.19,151.48) -- (150.19,143.20);

\path[draw=drawColor,line width= 0.6pt,line join=round] (150.19,151.48) -- (150.19,143.20);

\path[draw=drawColor,line width= 0.6pt,line join=round] (150.19,151.48) -- (150.19,143.20);

\path[draw=drawColor,line width= 0.6pt,line join=round] (150.19,151.48) -- (150.19,143.20);

\path[draw=drawColor,line width= 0.6pt,line join=round] (150.19,151.48) -- (150.19,143.20);

\path[draw=drawColor,line width= 0.6pt,line join=round] (150.19,151.48) -- (150.19,143.20);

\path[draw=drawColor,line width= 0.6pt,line join=round] (150.19,151.48) -- (150.19,143.20);

\path[draw=drawColor,line width= 0.6pt,line join=round] (150.19,151.48) -- (150.19,143.20);

\path[draw=drawColor,line width= 0.6pt,line join=round] (150.19,151.48) -- (150.19,143.20);

\path[draw=drawColor,line width= 0.6pt,line join=round] (150.19,151.48) -- (150.19,143.20);

\path[draw=drawColor,line width= 0.6pt,line join=round] (150.19,151.48) -- (150.19,143.20);

\path[draw=drawColor,line width= 0.6pt,line join=round] (150.19,151.48) -- (150.19,143.20);

\path[draw=drawColor,line width= 0.6pt,line join=round] (150.19,151.48) -- (150.19,143.20);

\path[draw=drawColor,line width= 0.6pt,line join=round] (150.19,151.48) -- (150.19,143.20);

\path[draw=drawColor,line width= 0.6pt,line join=round] (150.19,151.48) -- (150.19,143.20);

\path[draw=drawColor,line width= 0.6pt,line join=round] (150.19,151.48) -- (150.19,143.20);

\path[draw=drawColor,line width= 0.6pt,line join=round] (150.19,151.48) -- (150.19,143.20);

\path[draw=drawColor,line width= 0.6pt,line join=round] (150.19,151.48) -- (150.19,143.20);

\path[draw=drawColor,line width= 0.6pt,line join=round] (150.19,151.48) -- (150.19,143.20);

\path[draw=drawColor,line width= 0.6pt,line join=round] (150.19,151.48) -- (150.19,143.20);

\path[draw=drawColor,line width= 0.6pt,line join=round] (150.19,151.48) -- (150.19,143.20);

\path[draw=drawColor,line width= 0.6pt,line join=round] (150.19,151.48) -- (150.19,143.20);

\path[draw=drawColor,line width= 0.6pt,line join=round] (150.19,151.48) -- (150.19,143.20);

\path[draw=drawColor,line width= 0.6pt,line join=round] (150.19,151.48) -- (150.19,143.20);

\path[draw=drawColor,line width= 0.6pt,line join=round] (150.19,151.48) -- (150.19,143.20);

\path[draw=drawColor,line width= 0.6pt,line join=round] (150.19,151.48) -- (150.19,143.20);

\path[draw=drawColor,line width= 0.6pt,line join=round] (150.19,151.48) -- (150.19,143.20);

\path[draw=drawColor,line width= 0.6pt,line join=round] (150.19,151.48) -- (150.19,143.20);

\path[draw=drawColor,line width= 0.6pt,line join=round] (150.19,151.48) -- (150.19,143.20);

\path[draw=drawColor,line width= 0.6pt,line join=round] (150.19,151.48) -- (150.19,143.20);

\path[draw=drawColor,line width= 0.6pt,line join=round] (150.19,151.48) -- (150.19,143.20);

\path[draw=drawColor,line width= 0.6pt,line join=round] (150.19,151.48) -- (150.19,143.20);

\path[draw=drawColor,line width= 0.6pt,line join=round] (150.19,151.48) -- (150.19,143.20);

\path[draw=drawColor,line width= 0.6pt,line join=round] (150.19,151.48) -- (150.19,143.20);

\path[draw=drawColor,line width= 0.6pt,line join=round] (150.19,151.48) -- (150.19,143.20);

\path[draw=drawColor,line width= 0.6pt,line join=round] (150.19,151.48) -- (150.19,143.20);

\path[draw=drawColor,line width= 0.6pt,line join=round] (150.19,151.48) -- (150.19,143.20);

\path[draw=drawColor,line width= 0.6pt,line join=round] (150.19,151.48) -- (150.19,143.20);

\path[draw=drawColor,line width= 0.6pt,line join=round] (150.19,151.48) -- (150.19,143.20);

\path[draw=drawColor,line width= 0.6pt,line join=round] (150.19,151.48) -- (150.19,143.20);

\path[draw=drawColor,line width= 0.6pt,line join=round] (150.19,151.48) -- (150.19,143.20);

\path[draw=drawColor,line width= 0.6pt,line join=round] (150.19,151.48) -- (150.19,143.20);

\path[draw=drawColor,line width= 0.6pt,line join=round] (150.19,151.48) -- (150.19,143.20);

\path[draw=drawColor,line width= 0.6pt,line join=round] (150.19,151.48) -- (150.19,143.20);

\path[draw=drawColor,line width= 0.6pt,line join=round] (150.19,151.48) -- (150.19,143.20);

\path[draw=drawColor,line width= 0.6pt,line join=round] (150.19,151.48) -- (150.19,143.20);

\path[draw=drawColor,line width= 0.6pt,line join=round] (150.19,151.48) -- (150.19,143.20);

\path[draw=drawColor,line width= 0.6pt,line join=round] (150.19,151.48) -- (150.19,143.20);

\path[draw=drawColor,line width= 0.6pt,line join=round] (150.19,151.48) -- (150.19,143.20);

\path[draw=drawColor,line width= 0.6pt,line join=round] (150.19,151.48) -- (150.19,143.20);

\path[draw=drawColor,line width= 0.6pt,line join=round] (150.19,151.48) -- (150.19,143.20);

\path[draw=drawColor,line width= 0.6pt,line join=round] (150.19,151.48) -- (150.19,143.20);

\path[draw=drawColor,line width= 0.6pt,line join=round] (150.19,151.48) -- (150.19,143.20);

\path[draw=drawColor,line width= 0.6pt,line join=round] (150.19,151.48) -- (150.19,143.20);

\path[draw=drawColor,line width= 0.6pt,line join=round] (150.19,151.48) -- (150.19,143.20);

\path[draw=drawColor,line width= 0.6pt,line join=round] (150.19,151.48) -- (150.19,143.20);

\path[draw=drawColor,line width= 0.6pt,line join=round] (150.19,151.48) -- (150.19,143.20);

\path[draw=drawColor,line width= 0.6pt,line join=round] (150.19,151.48) -- (150.19,143.20);

\path[draw=drawColor,line width= 0.6pt,line join=round] (150.19,151.48) -- (150.19,143.20);

\path[draw=drawColor,line width= 0.6pt,line join=round] (150.19,151.48) -- (150.19,143.20);

\path[draw=drawColor,line width= 0.6pt,line join=round] (150.19,151.48) -- (150.19,143.20);

\path[draw=drawColor,line width= 0.6pt,line join=round] (150.19,151.48) -- (150.19,143.20);

\path[draw=drawColor,line width= 0.6pt,line join=round] (150.19,151.48) -- (150.19,143.20);

\path[draw=drawColor,line width= 0.6pt,line join=round] (150.19,151.48) -- (150.19,143.20);

\path[draw=drawColor,line width= 0.6pt,line join=round] (150.19,151.48) -- (150.19,143.20);

\path[draw=drawColor,line width= 0.6pt,line join=round] (150.19,151.48) -- (150.19,143.20);

\path[draw=drawColor,line width= 0.6pt,line join=round] (150.19,151.48) -- (150.19,143.20);

\path[draw=drawColor,line width= 0.6pt,line join=round] (150.19,151.48) -- (150.19,143.20);

\path[draw=drawColor,line width= 0.6pt,line join=round] (150.19,151.48) -- (150.19,143.20);

\path[draw=drawColor,line width= 0.6pt,line join=round] (150.19,151.48) -- (150.19,143.20);

\path[draw=drawColor,line width= 0.6pt,line join=round] (150.19,151.48) -- (150.19,143.20);

\path[draw=drawColor,line width= 0.6pt,line join=round] (150.19,151.48) -- (150.19,143.20);

\path[draw=drawColor,line width= 0.6pt,line join=round] (150.19,151.48) -- (150.19,143.20);

\path[draw=drawColor,line width= 0.6pt,line join=round] (150.19,151.48) -- (150.19,143.20);

\path[draw=drawColor,line width= 0.6pt,line join=round] (150.19,151.48) -- (150.19,143.20);

\path[draw=drawColor,line width= 0.6pt,line join=round] (150.19,151.48) -- (150.19,143.20);

\path[draw=drawColor,line width= 0.6pt,line join=round] (150.19,151.48) -- (150.19,143.20);

\path[draw=drawColor,line width= 0.6pt,line join=round] (150.19,151.48) -- (150.19,143.20);

\path[draw=drawColor,line width= 0.6pt,line join=round] (150.19,151.48) -- (150.19,143.20);

\path[draw=drawColor,line width= 0.6pt,line join=round] (150.19,151.48) -- (150.19,143.20);

\path[draw=drawColor,line width= 0.6pt,line join=round] (150.19,151.48) -- (150.19,143.20);

\path[draw=drawColor,line width= 0.6pt,line join=round] (150.19,151.48) -- (150.19,143.20);

\path[draw=drawColor,line width= 0.6pt,line join=round] (150.19,151.48) -- (150.19,143.20);

\path[draw=drawColor,line width= 0.6pt,line join=round] (150.19,151.48) -- (150.19,143.20);

\path[draw=drawColor,line width= 0.6pt,line join=round] (150.19,151.48) -- (150.19,143.20);

\path[draw=drawColor,line width= 0.6pt,line join=round] (150.19,151.48) -- (150.19,143.20);

\path[draw=drawColor,line width= 0.6pt,line join=round] (150.19,151.48) -- (150.19,143.20);

\path[draw=drawColor,line width= 0.6pt,line join=round] (150.19,151.48) -- (150.19,143.20);

\path[draw=drawColor,line width= 0.6pt,line join=round] (150.19,151.48) -- (150.19,143.20);

\path[draw=drawColor,line width= 0.6pt,line join=round] (150.19,151.48) -- (150.19,143.20);

\path[draw=drawColor,line width= 0.6pt,line join=round] (150.19,151.48) -- (150.19,143.20);

\path[draw=drawColor,line width= 0.6pt,line join=round] (150.19,151.48) -- (150.19,143.20);

\path[draw=drawColor,line width= 0.6pt,line join=round] (150.19,151.48) -- (150.19,143.20);

\path[draw=drawColor,line width= 0.6pt,line join=round] (150.19,151.48) -- (150.19,143.20);

\path[draw=drawColor,line width= 0.6pt,line join=round] (150.19,151.48) -- (150.19,143.20);

\path[draw=drawColor,line width= 0.6pt,line join=round] (150.19,151.48) -- (150.19,143.20);

\path[draw=drawColor,line width= 0.6pt,line join=round] (150.19,151.48) -- (150.19,143.20);

\path[draw=drawColor,line width= 0.6pt,line join=round] (150.19,151.48) -- (150.19,143.20);

\path[draw=drawColor,line width= 0.6pt,line join=round] (150.19,151.48) -- (150.19,143.20);

\path[draw=drawColor,line width= 0.6pt,line join=round] (150.19,151.48) -- (150.19,143.20);

\path[draw=drawColor,line width= 0.6pt,line join=round] (150.19,151.48) -- (150.19,143.20);

\path[draw=drawColor,line width= 0.6pt,line join=round] (150.19,151.48) -- (150.19,143.20);

\path[draw=drawColor,line width= 0.6pt,line join=round] (150.19,151.48) -- (150.19,143.20);

\path[draw=drawColor,line width= 0.6pt,line join=round] (150.19,151.48) -- (150.19,143.20);

\path[draw=drawColor,line width= 0.6pt,line join=round] (150.19,151.48) -- (150.19,143.20);

\path[draw=drawColor,line width= 0.6pt,line join=round] (150.19,151.48) -- (150.19,143.20);

\path[draw=drawColor,line width= 0.6pt,line join=round] (150.19,151.48) -- (150.19,143.20);

\path[draw=drawColor,line width= 0.6pt,line join=round] (150.19,151.48) -- (150.19,143.20);

\path[draw=drawColor,line width= 0.6pt,line join=round] (150.19,151.48) -- (150.19,143.20);

\path[draw=drawColor,line width= 0.6pt,line join=round] (150.19,151.48) -- (150.19,143.20);

\path[draw=drawColor,line width= 0.6pt,line join=round] (150.19,151.48) -- (150.19,143.20);

\path[draw=drawColor,line width= 0.6pt,line join=round] (150.19,151.48) -- (150.19,143.20);

\path[draw=drawColor,line width= 0.6pt,line join=round] (150.19,151.48) -- (150.19,143.20);

\path[draw=drawColor,line width= 0.6pt,line join=round] (150.19,151.48) -- (150.19,143.20);

\path[draw=drawColor,line width= 0.6pt,line join=round] (150.19,151.48) -- (150.19,143.20);

\path[draw=drawColor,line width= 0.6pt,line join=round] (150.19,151.48) -- (150.19,143.20);

\path[draw=drawColor,line width= 0.6pt,line join=round] (150.19,151.48) -- (150.19,143.20);

\path[draw=drawColor,line width= 0.6pt,line join=round] (150.19,151.48) -- (150.19,143.20);

\path[draw=drawColor,line width= 0.6pt,line join=round] (150.19,151.48) -- (150.19,143.20);

\path[draw=drawColor,line width= 0.6pt,line join=round] (150.19,151.48) -- (150.19,143.20);

\path[draw=drawColor,line width= 0.6pt,line join=round] (150.19,151.48) -- (150.19,143.20);

\path[draw=drawColor,line width= 0.6pt,line join=round] (150.19,151.48) -- (150.19,143.20);

\path[draw=drawColor,line width= 0.6pt,line join=round] (150.19,151.48) -- (150.19,143.20);

\path[draw=drawColor,line width= 0.6pt,line join=round] (150.19,151.48) -- (150.19,143.20);

\path[draw=drawColor,line width= 0.6pt,line join=round] (150.19,151.48) -- (150.19,143.20);

\path[draw=drawColor,line width= 0.6pt,line join=round] (150.19,151.48) -- (150.19,143.20);

\path[draw=drawColor,line width= 0.6pt,line join=round] (150.19,151.48) -- (150.19,143.20);

\path[draw=drawColor,line width= 0.6pt,line join=round] (150.19,151.48) -- (150.19,143.20);

\path[draw=drawColor,line width= 0.6pt,line join=round] (150.19,151.48) -- (150.19,143.20);

\path[draw=drawColor,line width= 0.6pt,line join=round] (150.19,151.48) -- (150.19,143.20);

\path[draw=drawColor,line width= 0.6pt,line join=round] (150.19,151.48) -- (150.19,143.20);

\path[draw=drawColor,line width= 0.6pt,line join=round] (150.19,151.48) -- (150.19,143.20);

\path[draw=drawColor,line width= 0.6pt,line join=round] (150.19,151.48) -- (150.19,143.20);

\path[draw=drawColor,line width= 0.6pt,line join=round] (150.19,151.48) -- (150.19,143.20);

\path[draw=drawColor,line width= 0.6pt,line join=round] (150.19,151.48) -- (150.19,143.20);

\path[draw=drawColor,line width= 0.6pt,line join=round] (150.19,151.48) -- (150.19,143.20);

\path[draw=drawColor,line width= 0.6pt,line join=round] (150.19,151.48) -- (150.19,143.20);

\path[draw=drawColor,line width= 0.6pt,line join=round] (150.19,151.48) -- (150.19,143.20);

\path[draw=drawColor,line width= 0.6pt,line join=round] (150.19,151.48) -- (150.19,143.20);

\path[draw=drawColor,line width= 0.6pt,line join=round] (150.19,151.48) -- (150.19,143.20);

\path[draw=drawColor,line width= 0.6pt,line join=round] (150.19,151.48) -- (150.19,143.20);

\path[draw=drawColor,line width= 0.6pt,line join=round] (150.19,151.48) -- (150.19,143.20);

\path[draw=drawColor,line width= 0.6pt,line join=round] (150.19,151.48) -- (150.19,143.20);

\path[draw=drawColor,line width= 0.6pt,line join=round] (150.19,151.48) -- (150.19,143.20);

\path[draw=drawColor,line width= 0.6pt,line join=round] (150.19,151.48) -- (150.19,143.20);

\path[draw=drawColor,line width= 0.6pt,line join=round] (150.19,151.48) -- (150.19,143.20);

\path[draw=drawColor,line width= 0.6pt,line join=round] (150.19,151.48) -- (150.19,143.20);

\path[draw=drawColor,line width= 0.6pt,line join=round] (150.19,151.48) -- (150.19,143.20);

\path[draw=drawColor,line width= 0.6pt,line join=round] (150.19,151.48) -- (150.19,143.20);

\path[draw=drawColor,line width= 0.6pt,line join=round] (150.19,151.48) -- (150.19,143.20);

\path[draw=drawColor,line width= 0.6pt,line join=round] (150.19,151.48) -- (150.19,143.20);

\path[draw=drawColor,line width= 0.6pt,line join=round] (150.19,151.48) -- (150.19,143.20);

\path[draw=drawColor,line width= 0.6pt,line join=round] (150.19,151.48) -- (150.19,143.20);

\path[draw=drawColor,line width= 0.6pt,line join=round] (150.19,151.48) -- (150.19,143.20);

\path[draw=drawColor,line width= 0.6pt,line join=round] (150.19,151.48) -- (150.19,143.20);

\path[draw=drawColor,line width= 0.6pt,line join=round] (150.19,151.48) -- (150.19,143.20);

\path[draw=drawColor,line width= 0.6pt,line join=round] (150.19,151.48) -- (150.19,143.20);

\path[draw=drawColor,line width= 0.6pt,line join=round] (150.19,151.48) -- (150.19,143.20);

\path[draw=drawColor,line width= 0.6pt,line join=round] (150.19,151.48) -- (150.19,143.20);

\path[draw=drawColor,line width= 0.6pt,line join=round] (150.19,151.48) -- (150.19,143.20);

\path[draw=drawColor,line width= 0.6pt,line join=round] (150.19,151.48) -- (150.19,143.20);

\path[draw=drawColor,line width= 0.6pt,line join=round] (150.19,151.48) -- (150.19,143.20);

\path[draw=drawColor,line width= 0.6pt,line join=round] (150.19,151.48) -- (150.19,143.20);

\path[draw=drawColor,line width= 0.6pt,line join=round] (150.19,151.48) -- (150.19,143.20);

\path[draw=drawColor,line width= 0.6pt,line join=round] (150.19,151.48) -- (150.19,143.20);

\path[draw=drawColor,line width= 0.6pt,line join=round] (150.19,151.48) -- (150.19,143.20);

\path[draw=drawColor,line width= 0.6pt,line join=round] (150.19,151.48) -- (150.19,143.20);

\path[draw=drawColor,line width= 0.6pt,line join=round] (150.19,151.48) -- (150.19,143.20);

\path[draw=drawColor,line width= 0.6pt,line join=round] (150.19,151.48) -- (150.19,143.20);

\path[draw=drawColor,line width= 0.6pt,line join=round] (150.19,151.48) -- (150.19,143.20);

\path[draw=drawColor,line width= 0.6pt,line join=round] (150.19,151.48) -- (150.19,143.20);

\path[draw=drawColor,line width= 0.6pt,line join=round] (150.19,151.48) -- (150.19,143.20);

\path[draw=drawColor,line width= 0.6pt,line join=round] (150.19,151.48) -- (150.19,143.20);

\path[draw=drawColor,line width= 0.6pt,line join=round] (150.19,151.48) -- (150.19,143.20);

\path[draw=drawColor,line width= 0.6pt,line join=round] (150.19,151.48) -- (150.19,143.20);

\path[draw=drawColor,line width= 0.6pt,line join=round] (150.19,151.48) -- (150.19,143.20);

\path[draw=drawColor,line width= 0.6pt,line join=round] (150.19,151.48) -- (150.19,143.20);

\path[draw=drawColor,line width= 0.6pt,line join=round] (150.19,151.48) -- (150.19,143.20);

\path[draw=drawColor,line width= 0.6pt,line join=round] (150.19,151.48) -- (150.19,143.20);

\path[draw=drawColor,line width= 0.6pt,line join=round] (150.19,151.48) -- (150.19,143.20);

\path[draw=drawColor,line width= 0.6pt,line join=round] (150.19,151.48) -- (150.19,143.20);

\path[draw=drawColor,line width= 0.6pt,line join=round] (150.19,151.48) -- (150.19,143.20);

\path[draw=drawColor,line width= 0.6pt,line join=round] (150.19,151.48) -- (150.19,143.20);

\path[draw=drawColor,line width= 0.6pt,line join=round] (150.19,151.48) -- (150.19,143.20);

\path[draw=drawColor,line width= 0.6pt,line join=round] (150.19,151.48) -- (150.19,143.20);

\path[draw=drawColor,line width= 0.6pt,line join=round] (150.19,151.48) -- (150.19,143.20);

\path[draw=drawColor,line width= 0.6pt,line join=round] (150.19,151.48) -- (150.19,143.20);

\path[draw=drawColor,line width= 0.6pt,line join=round] (150.19,151.48) -- (150.19,143.20);

\path[draw=drawColor,line width= 0.6pt,line join=round] (150.19,151.48) -- (150.19,143.20);

\path[draw=drawColor,line width= 0.6pt,line join=round] (150.19,151.48) -- (150.19,143.20);

\path[draw=drawColor,line width= 0.6pt,line join=round] (150.19,151.48) -- (150.19,143.20);

\path[draw=drawColor,line width= 0.6pt,line join=round] (150.19,151.48) -- (150.19,143.20);

\path[draw=drawColor,line width= 0.6pt,line join=round] (150.19,151.48) -- (150.19,143.20);

\path[draw=drawColor,line width= 0.6pt,line join=round] (150.19,151.48) -- (150.19,143.20);

\path[draw=drawColor,line width= 0.6pt,line join=round] (150.19,151.48) -- (150.19,143.20);

\path[draw=drawColor,line width= 0.6pt,line join=round] (150.19,151.48) -- (150.19,143.20);

\path[draw=drawColor,line width= 0.6pt,line join=round] (150.19,151.48) -- (150.19,143.20);

\path[draw=drawColor,line width= 0.6pt,line join=round] (150.19,151.48) -- (150.19,143.20);

\path[draw=drawColor,line width= 0.6pt,line join=round] (150.19,151.48) -- (150.19,143.20);

\path[draw=drawColor,line width= 0.6pt,line join=round] (150.19,151.48) -- (150.19,143.20);

\path[draw=drawColor,line width= 0.6pt,line join=round] (150.19,151.48) -- (150.19,143.20);

\path[draw=drawColor,line width= 0.6pt,line join=round] (150.19,151.48) -- (150.19,143.20);

\path[draw=drawColor,line width= 0.6pt,line join=round] (150.19,151.48) -- (150.19,143.20);

\path[draw=drawColor,line width= 0.6pt,line join=round] (150.19,151.48) -- (150.19,143.20);

\path[draw=drawColor,line width= 0.6pt,line join=round] (150.19,151.48) -- (150.19,143.20);

\path[draw=drawColor,line width= 0.6pt,line join=round] (150.19,151.48) -- (150.19,143.20);

\path[draw=drawColor,line width= 0.6pt,line join=round] (150.19,151.48) -- (150.19,143.20);

\path[draw=drawColor,line width= 0.6pt,line join=round] (150.19,151.48) -- (150.19,143.20);

\path[draw=drawColor,line width= 0.6pt,line join=round] (150.19,151.48) -- (150.19,143.20);

\path[draw=drawColor,line width= 0.6pt,line join=round] (150.19,151.48) -- (150.19,143.20);

\path[draw=drawColor,line width= 0.6pt,line join=round] (150.19,151.48) -- (150.19,143.20);

\path[draw=drawColor,line width= 0.6pt,line join=round] (150.19,151.48) -- (150.19,143.20);

\path[draw=drawColor,line width= 0.6pt,line join=round] (150.19,151.48) -- (150.19,143.20);

\path[draw=drawColor,line width= 0.6pt,line join=round] (150.19,151.48) -- (150.19,143.20);

\path[draw=drawColor,line width= 0.6pt,line join=round] (150.19,151.48) -- (150.19,143.20);

\path[draw=drawColor,line width= 0.6pt,line join=round] (150.19,151.48) -- (150.19,143.20);

\path[draw=drawColor,line width= 0.6pt,line join=round] (150.19,151.48) -- (150.19,143.20);

\path[draw=drawColor,line width= 0.6pt,line join=round] (150.19,151.48) -- (150.19,143.20);

\path[draw=drawColor,line width= 0.6pt,line join=round] (150.19,151.48) -- (150.19,143.20);

\path[draw=drawColor,line width= 0.6pt,line join=round] (150.19,151.48) -- (150.19,143.20);

\path[draw=drawColor,line width= 0.6pt,line join=round] (150.19,151.48) -- (150.19,143.20);

\path[draw=drawColor,line width= 0.6pt,line join=round] (150.19,151.48) -- (150.19,143.20);

\path[draw=drawColor,line width= 0.6pt,line join=round] (150.19,151.48) -- (150.19,143.20);

\path[draw=drawColor,line width= 0.6pt,line join=round] (150.19,151.48) -- (150.19,143.20);

\path[draw=drawColor,line width= 0.6pt,line join=round] (150.19,151.48) -- (150.19,143.20);

\path[draw=drawColor,line width= 0.6pt,line join=round] (150.19,151.48) -- (150.19,143.20);

\path[draw=drawColor,line width= 0.6pt,line join=round] (150.19,151.48) -- (150.19,143.20);

\path[draw=drawColor,line width= 0.6pt,line join=round] (150.19,151.48) -- (150.19,143.20);

\path[draw=drawColor,line width= 0.6pt,line join=round] (150.19,151.48) -- (150.19,143.20);

\path[draw=drawColor,line width= 0.6pt,line join=round] (150.19,151.48) -- (150.19,143.20);

\path[draw=drawColor,line width= 0.6pt,line join=round] (150.19,151.48) -- (150.19,143.20);

\path[draw=drawColor,line width= 0.6pt,line join=round] (150.19,151.48) -- (150.19,143.20);

\path[draw=drawColor,line width= 0.6pt,line join=round] (150.19,151.48) -- (150.19,143.20);

\path[draw=drawColor,line width= 0.6pt,line join=round] (150.19,151.48) -- (150.19,143.20);

\path[draw=drawColor,line width= 0.6pt,line join=round] (150.19,151.48) -- (150.19,143.20);

\path[draw=drawColor,line width= 0.6pt,line join=round] (150.19,151.48) -- (150.19,143.20);

\path[draw=drawColor,line width= 0.6pt,line join=round] (150.19,151.48) -- (150.19,143.20);

\path[draw=drawColor,line width= 0.6pt,line join=round] (150.19,151.48) -- (150.19,143.20);

\path[draw=drawColor,line width= 0.6pt,line join=round] (150.19,151.48) -- (150.19,143.20);

\path[draw=drawColor,line width= 0.6pt,line join=round] (150.19,151.48) -- (150.19,143.20);

\path[draw=drawColor,line width= 0.6pt,line join=round] (150.19,151.48) -- (150.19,143.20);

\path[draw=drawColor,line width= 0.6pt,line join=round] (150.19,151.48) -- (150.19,143.20);

\path[draw=drawColor,line width= 0.6pt,line join=round] (150.19,151.48) -- (150.19,143.20);

\path[draw=drawColor,line width= 0.6pt,line join=round] (150.19,151.48) -- (150.19,143.20);

\path[draw=drawColor,line width= 0.6pt,line join=round] (150.19,151.48) -- (150.19,143.20);

\path[draw=drawColor,line width= 0.6pt,line join=round] (150.19,151.48) -- (150.19,143.20);

\path[draw=drawColor,line width= 0.6pt,line join=round] (150.19,151.48) -- (150.19,143.20);

\path[draw=drawColor,line width= 0.6pt,line join=round] (150.19,151.48) -- (150.19,143.20);

\path[draw=drawColor,line width= 0.6pt,line join=round] (150.19,151.48) -- (150.19,143.20);

\path[draw=drawColor,line width= 0.6pt,line join=round] (150.19,151.48) -- (150.19,143.20);

\path[draw=drawColor,line width= 0.6pt,line join=round] (150.19,151.48) -- (150.19,143.20);

\path[draw=drawColor,line width= 0.6pt,line join=round] (150.19,151.48) -- (150.19,143.20);

\path[draw=drawColor,line width= 0.6pt,line join=round] (150.19,151.48) -- (150.19,143.20);

\path[draw=drawColor,line width= 0.6pt,line join=round] (150.19,151.48) -- (150.19,143.20);

\path[draw=drawColor,line width= 0.6pt,line join=round] (150.19,151.48) -- (150.19,143.20);

\path[draw=drawColor,line width= 0.6pt,line join=round] (150.19,151.48) -- (150.19,143.20);

\path[draw=drawColor,line width= 0.6pt,line join=round] (150.19,151.48) -- (150.19,143.20);

\path[draw=drawColor,line width= 0.6pt,line join=round] (150.19,151.48) -- (150.19,143.20);

\path[draw=drawColor,line width= 0.6pt,line join=round] (150.19,151.48) -- (150.19,143.20);

\path[draw=drawColor,line width= 0.6pt,line join=round] (150.19,151.48) -- (150.19,143.20);

\path[draw=drawColor,line width= 0.6pt,line join=round] (150.19,151.48) -- (150.19,143.20);

\path[draw=drawColor,line width= 0.6pt,line join=round] (150.19,151.48) -- (150.19,143.20);

\path[draw=drawColor,line width= 0.6pt,line join=round] (150.19,151.48) -- (150.19,143.20);

\path[draw=drawColor,line width= 0.6pt,line join=round] (150.19,151.48) -- (150.19,143.20);

\path[draw=drawColor,line width= 0.6pt,line join=round] (150.19,151.48) -- (150.19,143.20);

\path[draw=drawColor,line width= 0.6pt,line join=round] (150.19,151.48) -- (150.19,143.20);

\path[draw=drawColor,line width= 0.6pt,line join=round] (150.19,151.48) -- (150.19,143.20);

\path[draw=drawColor,line width= 0.6pt,line join=round] (150.19,151.48) -- (150.19,143.20);

\path[draw=drawColor,line width= 0.6pt,line join=round] (150.19,151.48) -- (150.19,143.20);

\path[draw=drawColor,line width= 0.6pt,line join=round] (150.19,151.48) -- (150.19,143.20);

\path[draw=drawColor,line width= 0.6pt,line join=round] (150.19,151.48) -- (150.19,143.20);

\path[draw=drawColor,line width= 0.6pt,line join=round] (150.19,151.48) -- (150.19,143.20);

\path[draw=drawColor,line width= 0.6pt,line join=round] (150.19,151.48) -- (150.19,143.20);

\path[draw=drawColor,line width= 0.6pt,line join=round] (150.19,151.48) -- (150.19,143.20);

\path[draw=drawColor,line width= 0.6pt,line join=round] (150.19,151.48) -- (150.19,143.20);

\path[draw=drawColor,line width= 0.6pt,line join=round] (150.19,151.48) -- (150.19,143.20);

\path[draw=drawColor,line width= 0.6pt,line join=round] (150.19,151.48) -- (150.19,143.20);

\path[draw=drawColor,line width= 0.6pt,line join=round] (150.19,151.48) -- (150.19,143.20);

\path[draw=drawColor,line width= 0.6pt,line join=round] (150.19,151.48) -- (150.19,143.20);

\path[draw=drawColor,line width= 0.6pt,line join=round] (150.19,151.48) -- (150.19,143.20);

\path[draw=drawColor,line width= 0.6pt,line join=round] (150.19,151.48) -- (150.19,143.20);

\path[draw=drawColor,line width= 0.6pt,line join=round] (150.19,151.48) -- (150.19,143.20);

\path[draw=drawColor,line width= 0.6pt,line join=round] (150.19,151.48) -- (150.19,143.20);

\path[draw=drawColor,line width= 0.6pt,line join=round] (150.19,151.48) -- (150.19,143.20);

\path[draw=drawColor,line width= 0.6pt,line join=round] (150.19,151.48) -- (150.19,143.20);

\path[draw=drawColor,line width= 0.6pt,line join=round] (150.19,151.48) -- (150.19,143.20);

\path[draw=drawColor,line width= 0.6pt,line join=round] (150.19,151.48) -- (150.19,143.20);

\path[draw=drawColor,line width= 0.6pt,line join=round] (150.19,151.48) -- (150.19,143.20);

\path[draw=drawColor,line width= 0.6pt,line join=round] (150.19,151.48) -- (150.19,143.20);

\path[draw=drawColor,line width= 0.6pt,line join=round] (150.19,151.48) -- (150.19,143.20);

\path[draw=drawColor,line width= 0.6pt,line join=round] (150.19,151.48) -- (150.19,143.20);

\path[draw=drawColor,line width= 0.6pt,line join=round] (150.19,151.48) -- (150.19,143.20);

\path[draw=drawColor,line width= 0.6pt,line join=round] (150.19,151.48) -- (150.19,143.20);

\path[draw=drawColor,line width= 0.6pt,line join=round] (150.19,151.48) -- (150.19,143.20);

\path[draw=drawColor,line width= 0.6pt,line join=round] (150.19,151.48) -- (150.19,143.20);

\path[draw=drawColor,line width= 0.6pt,line join=round] (150.19,151.48) -- (150.19,143.20);

\path[draw=drawColor,line width= 0.6pt,line join=round] (150.19,151.48) -- (150.19,143.20);

\path[draw=drawColor,line width= 0.6pt,line join=round] (150.19,151.48) -- (150.19,143.20);

\path[draw=drawColor,line width= 0.6pt,line join=round] (150.19,151.48) -- (150.19,143.20);

\path[draw=drawColor,line width= 0.6pt,line join=round] (150.19,151.48) -- (150.19,143.20);

\path[draw=drawColor,line width= 0.6pt,line join=round] (150.19,151.48) -- (150.19,143.20);

\path[draw=drawColor,line width= 0.6pt,line join=round] (150.19,151.48) -- (150.19,143.20);

\path[draw=drawColor,line width= 0.6pt,line join=round] (150.19,151.48) -- (150.19,143.20);

\path[draw=drawColor,line width= 0.6pt,line join=round] (150.19,151.48) -- (150.19,143.20);

\path[draw=drawColor,line width= 0.6pt,line join=round] (150.19,151.48) -- (150.19,143.20);

\path[draw=drawColor,line width= 0.6pt,line join=round] (150.19,151.48) -- (150.19,143.20);

\path[draw=drawColor,line width= 0.6pt,line join=round] (150.19,151.48) -- (150.19,143.20);

\path[draw=drawColor,line width= 0.6pt,line join=round] (150.19,151.48) -- (150.19,143.20);

\path[draw=drawColor,line width= 0.6pt,line join=round] (150.19,151.48) -- (150.19,143.20);

\path[draw=drawColor,line width= 0.6pt,line join=round] (150.19,151.48) -- (150.19,143.20);

\path[draw=drawColor,line width= 0.6pt,line join=round] (150.19,151.48) -- (150.19,143.20);

\path[draw=drawColor,line width= 0.6pt,line join=round] (150.19,151.48) -- (150.19,143.20);

\path[draw=drawColor,line width= 0.6pt,line join=round] (150.19,151.48) -- (150.19,143.20);

\path[draw=drawColor,line width= 0.6pt,line join=round] (150.19,151.48) -- (150.19,143.20);

\path[draw=drawColor,line width= 0.6pt,line join=round] (150.19,151.48) -- (150.19,143.20);

\path[draw=drawColor,line width= 0.6pt,line join=round] (150.19,151.48) -- (150.19,143.20);

\path[draw=drawColor,line width= 0.6pt,line join=round] (150.19,151.48) -- (150.19,143.20);

\path[draw=drawColor,line width= 0.6pt,line join=round] (150.19,151.48) -- (150.19,143.20);

\path[draw=drawColor,line width= 0.6pt,line join=round] (150.19,151.48) -- (150.19,143.20);

\path[draw=drawColor,line width= 0.6pt,line join=round] (150.19,151.48) -- (150.19,143.20);

\path[draw=drawColor,line width= 0.6pt,line join=round] (150.19,151.48) -- (150.19,143.20);

\path[draw=drawColor,line width= 0.6pt,line join=round] (150.19,151.48) -- (150.19,143.20);

\path[draw=drawColor,line width= 0.6pt,line join=round] (150.19,151.48) -- (150.19,143.20);

\path[draw=drawColor,line width= 0.6pt,line join=round] (150.19,151.48) -- (150.19,143.20);

\path[draw=drawColor,line width= 0.6pt,line join=round] (150.19,151.48) -- (150.19,143.20);

\path[draw=drawColor,line width= 0.6pt,line join=round] (150.19,151.48) -- (150.19,143.20);

\path[draw=drawColor,line width= 0.6pt,line join=round] (150.19,151.48) -- (150.19,143.20);

\path[draw=drawColor,line width= 0.6pt,line join=round] (150.19,151.48) -- (150.19,143.20);

\path[draw=drawColor,line width= 0.6pt,line join=round] (150.19,151.48) -- (150.19,143.20);

\path[draw=drawColor,line width= 0.6pt,line join=round] (150.19,151.48) -- (150.19,143.20);

\path[draw=drawColor,line width= 0.6pt,line join=round] (150.19,151.48) -- (150.19,143.20);

\path[draw=drawColor,line width= 0.6pt,line join=round] (150.19,151.48) -- (150.19,143.20);

\path[draw=drawColor,line width= 0.6pt,line join=round] (150.19,151.48) -- (150.19,143.20);

\path[draw=drawColor,line width= 0.6pt,line join=round] (150.19,151.48) -- (150.19,143.20);

\path[draw=drawColor,line width= 0.6pt,line join=round] (150.19,151.48) -- (150.19,143.20);

\path[draw=drawColor,line width= 0.6pt,line join=round] (150.19,151.48) -- (150.19,143.20);

\path[draw=drawColor,line width= 0.6pt,line join=round] (150.19,151.48) -- (150.19,143.20);

\path[draw=drawColor,line width= 0.6pt,line join=round] (150.19,151.48) -- (150.19,143.20);

\path[draw=drawColor,line width= 0.6pt,line join=round] (150.19,151.48) -- (150.19,143.20);

\path[draw=drawColor,line width= 0.6pt,line join=round] (150.19,151.48) -- (150.19,143.20);

\path[draw=drawColor,line width= 0.6pt,line join=round] (150.19,151.48) -- (150.19,143.20);

\path[draw=drawColor,line width= 0.6pt,line join=round] (150.19,151.48) -- (150.19,143.20);

\path[draw=drawColor,line width= 0.6pt,line join=round] (150.19,151.48) -- (150.19,143.20);

\path[draw=drawColor,line width= 0.6pt,line join=round] (150.19,151.48) -- (150.19,143.20);

\path[draw=drawColor,line width= 0.6pt,line join=round] (150.19,151.48) -- (150.19,143.20);

\path[draw=drawColor,line width= 0.6pt,line join=round] (150.19,151.48) -- (150.19,143.20);

\path[draw=drawColor,line width= 0.6pt,line join=round] (150.19,151.48) -- (150.19,143.20);

\path[draw=drawColor,line width= 0.6pt,line join=round] (150.19,151.48) -- (150.19,143.20);

\path[draw=drawColor,line width= 0.6pt,line join=round] (150.19,151.48) -- (150.19,143.20);

\path[draw=drawColor,line width= 0.6pt,line join=round] (150.19,151.48) -- (150.19,143.20);

\path[draw=drawColor,line width= 0.6pt,line join=round] (150.19,151.48) -- (150.19,143.20);

\path[draw=drawColor,line width= 0.6pt,line join=round] (150.19,151.48) -- (150.19,143.20);

\path[draw=drawColor,line width= 0.6pt,line join=round] (150.19,151.48) -- (150.19,143.20);

\path[draw=drawColor,line width= 0.6pt,line join=round] (150.19,151.48) -- (150.19,143.20);

\path[draw=drawColor,line width= 0.6pt,line join=round] (150.19,151.48) -- (150.19,143.20);

\path[draw=drawColor,line width= 0.6pt,line join=round] (150.19,151.48) -- (150.19,143.20);

\path[draw=drawColor,line width= 0.6pt,line join=round] (150.19,151.48) -- (150.19,143.20);

\path[draw=drawColor,line width= 0.6pt,line join=round] (150.19,151.48) -- (150.19,143.20);

\path[draw=drawColor,line width= 0.6pt,line join=round] (150.19,151.48) -- (150.19,143.20);

\path[draw=drawColor,line width= 0.6pt,line join=round] (150.19,151.48) -- (150.19,143.20);

\path[draw=drawColor,line width= 0.6pt,line join=round] (150.19,151.48) -- (150.19,143.20);

\path[draw=drawColor,line width= 0.6pt,line join=round] (150.19,151.48) -- (150.19,143.20);

\path[draw=drawColor,line width= 0.6pt,line join=round] (150.19,151.48) -- (150.19,143.20);

\path[draw=drawColor,line width= 0.6pt,line join=round] (150.19,151.48) -- (150.19,143.20);

\path[draw=drawColor,line width= 0.6pt,line join=round] (150.19,151.48) -- (150.19,143.20);

\path[draw=drawColor,line width= 0.6pt,line join=round] (150.19,151.48) -- (150.19,143.20);

\path[draw=drawColor,line width= 0.6pt,line join=round] (150.19,151.48) -- (150.19,143.20);

\path[draw=drawColor,line width= 0.6pt,line join=round] (150.19,151.48) -- (150.19,143.20);

\path[draw=drawColor,line width= 0.6pt,line join=round] (150.19,151.48) -- (150.19,143.20);

\path[draw=drawColor,line width= 0.6pt,line join=round] (150.19,151.48) -- (150.19,143.20);

\path[draw=drawColor,line width= 0.6pt,line join=round] (150.19,151.48) -- (150.19,143.20);

\path[draw=drawColor,line width= 0.6pt,line join=round] (150.19,151.48) -- (150.19,143.20);

\path[draw=drawColor,line width= 0.6pt,line join=round] (150.19,151.48) -- (150.19,143.20);

\path[draw=drawColor,line width= 0.6pt,line join=round] (150.19,151.48) -- (150.19,143.20);

\path[draw=drawColor,line width= 0.6pt,line join=round] (150.19,151.48) -- (150.19,143.20);

\path[draw=drawColor,line width= 0.6pt,line join=round] (150.19,151.48) -- (150.19,143.20);

\path[draw=drawColor,line width= 0.6pt,line join=round] (150.19,151.48) -- (150.19,143.20);

\path[draw=drawColor,line width= 0.6pt,line join=round] (150.19,151.48) -- (150.19,143.20);

\path[draw=drawColor,line width= 0.6pt,line join=round] (150.19,151.48) -- (150.19,143.20);

\path[draw=drawColor,line width= 0.6pt,line join=round] (150.19,151.48) -- (150.19,143.20);

\path[draw=drawColor,line width= 0.6pt,line join=round] (150.19,151.48) -- (150.19,143.20);

\path[draw=drawColor,line width= 0.6pt,line join=round] (150.19,151.48) -- (150.19,143.20);

\path[draw=drawColor,line width= 0.6pt,line join=round] (150.19,151.48) -- (150.19,143.20);

\path[draw=drawColor,line width= 0.6pt,line join=round] (150.19,151.48) -- (150.19,143.20);

\path[draw=drawColor,line width= 0.6pt,line join=round] (150.19,151.48) -- (150.19,143.20);

\path[draw=drawColor,line width= 0.6pt,line join=round] (150.19,151.48) -- (150.19,143.20);

\path[draw=drawColor,line width= 0.6pt,line join=round] (150.19,151.48) -- (150.19,143.20);

\path[draw=drawColor,line width= 0.6pt,line join=round] (150.19,151.48) -- (150.19,143.20);

\path[draw=drawColor,line width= 0.6pt,line join=round] (150.19,151.48) -- (150.19,143.20);

\path[draw=drawColor,line width= 0.6pt,line join=round] (150.19,151.48) -- (150.19,143.20);

\path[draw=drawColor,line width= 0.6pt,line join=round] (150.19,151.48) -- (150.19,143.20);

\path[draw=drawColor,line width= 0.6pt,line join=round] (150.19,151.48) -- (150.19,143.20);

\path[draw=drawColor,line width= 0.6pt,line join=round] (150.19,151.48) -- (150.19,143.20);

\path[draw=drawColor,line width= 0.6pt,line join=round] (150.19,151.48) -- (150.19,143.20);

\path[draw=drawColor,line width= 0.6pt,line join=round] (150.19,151.48) -- (150.19,143.20);

\path[draw=drawColor,line width= 0.6pt,line join=round] (150.19,151.48) -- (150.19,143.20);

\path[draw=drawColor,line width= 0.6pt,line join=round] (150.19,151.48) -- (150.19,143.20);

\path[draw=drawColor,line width= 0.6pt,line join=round] (150.19,151.48) -- (150.19,143.20);

\path[draw=drawColor,line width= 0.6pt,line join=round] (150.19,151.48) -- (150.19,143.20);

\path[draw=drawColor,line width= 0.6pt,line join=round] (150.19,151.48) -- (150.19,143.20);

\path[draw=drawColor,line width= 0.6pt,line join=round] (150.19,151.48) -- (150.19,143.20);

\path[draw=drawColor,line width= 0.6pt,line join=round] (150.19,151.48) -- (150.19,143.20);

\path[draw=drawColor,line width= 0.6pt,line join=round] (150.19,151.48) -- (150.19,143.20);

\path[draw=drawColor,line width= 0.6pt,line join=round] (150.19,151.48) -- (150.19,143.20);

\path[draw=drawColor,line width= 0.6pt,line join=round] (150.19,151.48) -- (150.19,143.20);

\path[draw=drawColor,line width= 0.6pt,line join=round] (150.19,151.48) -- (150.19,143.20);

\path[draw=drawColor,line width= 0.6pt,line join=round] (150.19,151.48) -- (150.19,143.20);

\path[draw=drawColor,line width= 0.6pt,line join=round] (150.19,151.48) -- (150.19,143.20);

\path[draw=drawColor,line width= 0.6pt,line join=round] (150.19,151.48) -- (150.19,143.20);

\path[draw=drawColor,line width= 0.6pt,line join=round] (150.19,151.48) -- (150.19,143.20);

\path[draw=drawColor,line width= 0.6pt,line join=round] (150.19,151.48) -- (150.19,143.20);

\path[draw=drawColor,line width= 0.6pt,line join=round] (150.19,151.48) -- (150.19,143.20);

\path[draw=drawColor,line width= 0.6pt,line join=round] (150.19,151.48) -- (150.19,143.20);

\path[draw=drawColor,line width= 0.6pt,line join=round] (150.19,151.48) -- (150.19,143.20);

\path[draw=drawColor,line width= 0.6pt,line join=round] (150.19,151.48) -- (150.19,143.20);

\path[draw=drawColor,line width= 0.6pt,line join=round] (150.19,151.48) -- (150.19,143.20);

\path[draw=drawColor,line width= 0.6pt,line join=round] (150.19,151.48) -- (150.19,143.20);

\path[draw=drawColor,line width= 0.6pt,line join=round] (150.19,151.48) -- (150.19,143.20);

\path[draw=drawColor,line width= 0.6pt,line join=round] (150.19,151.48) -- (150.19,143.20);

\path[draw=drawColor,line width= 0.6pt,line join=round] (150.19,151.48) -- (150.19,143.20);

\path[draw=drawColor,line width= 0.6pt,line join=round] (150.19,151.48) -- (150.19,143.20);

\path[draw=drawColor,line width= 0.6pt,line join=round] (150.19,151.48) -- (150.19,143.20);

\path[draw=drawColor,line width= 0.6pt,line join=round] (150.19,151.48) -- (150.19,143.20);

\path[draw=drawColor,line width= 0.6pt,line join=round] (150.19,151.48) -- (150.19,143.20);

\path[draw=drawColor,line width= 0.6pt,line join=round] (150.19,151.48) -- (150.19,143.20);

\path[draw=drawColor,line width= 0.6pt,line join=round] (150.19,151.48) -- (150.19,143.20);

\path[draw=drawColor,line width= 0.6pt,line join=round] (150.19,151.48) -- (150.19,143.20);

\path[draw=drawColor,line width= 0.6pt,line join=round] (150.19,151.48) -- (150.19,143.20);

\path[draw=drawColor,line width= 0.6pt,line join=round] (150.19,151.48) -- (150.19,143.20);

\path[draw=drawColor,line width= 0.6pt,line join=round] (150.19,151.48) -- (150.19,143.20);

\path[draw=drawColor,line width= 0.6pt,line join=round] (150.19,151.48) -- (150.19,143.20);

\path[draw=drawColor,line width= 0.6pt,line join=round] (150.19,151.48) -- (150.19,143.20);

\path[draw=drawColor,line width= 0.6pt,line join=round] (150.19,151.48) -- (150.19,143.20);

\path[draw=drawColor,line width= 0.6pt,line join=round] (150.19,151.48) -- (150.19,143.20);

\path[draw=drawColor,line width= 0.6pt,line join=round] (150.19,151.48) -- (150.19,143.20);

\path[draw=drawColor,line width= 0.6pt,line join=round] (150.19,151.48) -- (150.19,143.20);

\path[draw=drawColor,line width= 0.6pt,line join=round] (150.19,151.48) -- (150.19,143.20);

\path[draw=drawColor,line width= 0.6pt,line join=round] (150.19,151.48) -- (150.19,143.20);

\path[draw=drawColor,line width= 0.6pt,line join=round] (150.19,151.48) -- (150.19,143.20);

\path[draw=drawColor,line width= 0.6pt,line join=round] (150.19,151.48) -- (150.19,143.20);

\path[draw=drawColor,line width= 0.6pt,line join=round] (150.19,151.48) -- (150.19,143.20);

\path[draw=drawColor,line width= 0.6pt,line join=round] (150.19,151.48) -- (150.19,143.20);

\path[draw=drawColor,line width= 0.6pt,line join=round] (150.19,151.48) -- (150.19,143.20);

\path[draw=drawColor,line width= 0.6pt,line join=round] (150.19,151.48) -- (150.19,143.20);

\path[draw=drawColor,line width= 0.6pt,line join=round] (150.19,151.48) -- (150.19,143.20);

\path[draw=drawColor,line width= 0.6pt,line join=round] (150.19,151.48) -- (150.19,143.20);

\path[draw=drawColor,line width= 0.6pt,line join=round] (150.19,151.48) -- (150.19,143.20);

\path[draw=drawColor,line width= 0.6pt,line join=round] (150.19,151.48) -- (150.19,143.20);

\path[draw=drawColor,line width= 0.6pt,line join=round] (150.19,151.48) -- (150.19,143.20);

\path[draw=drawColor,line width= 0.6pt,line join=round] (150.19,151.48) -- (150.19,143.20);

\path[draw=drawColor,line width= 0.6pt,line join=round] (150.19,151.48) -- (150.19,143.20);

\path[draw=drawColor,line width= 0.6pt,line join=round] (150.19,151.48) -- (150.19,143.20);

\path[draw=drawColor,line width= 0.6pt,line join=round] (150.19,151.48) -- (150.19,143.20);

\path[draw=drawColor,line width= 0.6pt,line join=round] (150.19,151.48) -- (150.19,143.20);

\path[draw=drawColor,line width= 0.6pt,line join=round] (150.19,151.48) -- (150.19,143.20);

\path[draw=drawColor,line width= 0.6pt,line join=round] (150.19,151.48) -- (150.19,143.20);

\path[draw=drawColor,line width= 0.6pt,line join=round] (150.19,151.48) -- (150.19,143.20);

\path[draw=drawColor,line width= 0.6pt,line join=round] (150.19,151.48) -- (150.19,143.20);

\path[draw=drawColor,line width= 0.6pt,line join=round] (150.19,151.48) -- (150.19,143.20);

\path[draw=drawColor,line width= 0.6pt,line join=round] (150.19,151.48) -- (150.19,143.20);

\path[draw=drawColor,line width= 0.6pt,line join=round] (150.19,151.48) -- (150.19,143.20);

\path[draw=drawColor,line width= 0.6pt,line join=round] (150.19,151.48) -- (150.19,143.20);

\path[draw=drawColor,line width= 0.6pt,line join=round] (150.19,151.48) -- (150.19,143.20);

\path[draw=drawColor,line width= 0.6pt,line join=round] (150.19,151.48) -- (150.19,143.20);

\path[draw=drawColor,line width= 0.6pt,line join=round] (150.19,151.48) -- (150.19,143.20);

\path[draw=drawColor,line width= 0.6pt,line join=round] (150.19,151.48) -- (150.19,143.20);

\path[draw=drawColor,line width= 0.6pt,line join=round] (150.19,151.48) -- (150.19,143.20);

\path[draw=drawColor,line width= 0.6pt,line join=round] (150.19,151.48) -- (150.19,143.20);

\path[draw=drawColor,line width= 0.6pt,line join=round] (150.19,151.48) -- (150.19,143.20);

\path[draw=drawColor,line width= 0.6pt,line join=round] (150.19,151.48) -- (150.19,143.20);

\path[draw=drawColor,line width= 0.6pt,line join=round] (150.19,151.48) -- (150.19,143.20);

\path[draw=drawColor,line width= 0.6pt,line join=round] (150.19,151.48) -- (150.19,143.20);

\path[draw=drawColor,line width= 0.6pt,line join=round] (150.19,151.48) -- (150.19,143.20);

\path[draw=drawColor,line width= 0.6pt,line join=round] (150.19,151.48) -- (150.19,143.20);

\path[draw=drawColor,line width= 0.6pt,line join=round] (150.19,151.48) -- (150.19,143.20);

\path[draw=drawColor,line width= 0.6pt,line join=round] (150.19,151.48) -- (150.19,143.20);

\path[draw=drawColor,line width= 0.6pt,line join=round] (150.19,151.48) -- (150.19,143.20);

\path[draw=drawColor,line width= 0.6pt,line join=round] (150.19,151.48) -- (150.19,143.20);

\path[draw=drawColor,line width= 0.6pt,line join=round] (150.19,151.48) -- (150.19,143.20);

\path[draw=drawColor,line width= 0.6pt,line join=round] (150.19,151.48) -- (150.19,143.20);

\path[draw=drawColor,line width= 0.6pt,line join=round] (150.19,151.48) -- (150.19,143.20);

\path[draw=drawColor,line width= 0.6pt,line join=round] (150.19,151.48) -- (150.19,143.20);

\path[draw=drawColor,line width= 0.6pt,line join=round] (150.19,151.48) -- (150.19,143.20);

\path[draw=drawColor,line width= 0.6pt,line join=round] (150.19,151.48) -- (150.19,143.20);

\path[draw=drawColor,line width= 0.6pt,line join=round] (150.19,151.48) -- (150.19,143.20);

\path[draw=drawColor,line width= 0.6pt,line join=round] (150.19,151.48) -- (150.19,143.20);

\path[draw=drawColor,line width= 0.6pt,line join=round] (150.19,151.48) -- (150.19,143.20);

\path[draw=drawColor,line width= 0.6pt,line join=round] (150.19,151.48) -- (150.19,143.20);

\path[draw=drawColor,line width= 0.6pt,line join=round] (150.19,151.48) -- (150.19,143.20);

\path[draw=drawColor,line width= 0.6pt,line join=round] (150.19,151.48) -- (150.19,143.20);

\path[draw=drawColor,line width= 0.6pt,line join=round] (150.19,151.48) -- (150.19,143.20);

\path[draw=drawColor,line width= 0.6pt,line join=round] (150.19,151.48) -- (150.19,143.20);

\path[draw=drawColor,line width= 0.6pt,line join=round] (150.19,151.48) -- (150.19,143.20);

\path[draw=drawColor,line width= 0.6pt,line join=round] (150.19,151.48) -- (150.19,143.20);

\path[draw=drawColor,line width= 0.6pt,line join=round] (150.19,151.48) -- (150.19,143.20);

\path[draw=drawColor,line width= 0.6pt,line join=round] (150.19,151.48) -- (150.19,143.20);

\path[draw=drawColor,line width= 0.6pt,line join=round] (150.19,151.48) -- (150.19,143.20);

\path[draw=drawColor,line width= 0.6pt,line join=round] (150.19,151.48) -- (150.19,143.20);

\path[draw=drawColor,line width= 0.6pt,line join=round] (150.19,151.48) -- (150.19,143.20);

\path[draw=drawColor,line width= 0.6pt,line join=round] (150.19,151.48) -- (150.19,143.20);

\path[draw=drawColor,line width= 0.6pt,line join=round] (150.19,151.48) -- (150.19,143.20);

\path[draw=drawColor,line width= 0.6pt,line join=round] (150.19,151.48) -- (150.19,143.20);

\path[draw=drawColor,line width= 0.6pt,line join=round] (150.19,151.48) -- (150.19,143.20);

\path[draw=drawColor,line width= 0.6pt,line join=round] (150.19,151.48) -- (150.19,143.20);

\path[draw=drawColor,line width= 0.6pt,line join=round] (150.19,151.48) -- (150.19,143.20);

\path[draw=drawColor,line width= 0.6pt,line join=round] (150.19,151.48) -- (150.19,143.20);

\path[draw=drawColor,line width= 0.6pt,line join=round] (150.19,151.48) -- (150.19,143.20);

\path[draw=drawColor,line width= 0.6pt,line join=round] (150.19,151.48) -- (150.19,143.20);

\path[draw=drawColor,line width= 0.6pt,line join=round] (150.19,151.48) -- (150.19,143.20);

\path[draw=drawColor,line width= 0.6pt,line join=round] (150.19,151.48) -- (150.19,143.20);

\path[draw=drawColor,line width= 0.6pt,line join=round] (150.19,151.48) -- (150.19,143.20);

\path[draw=drawColor,line width= 0.6pt,line join=round] (150.19,151.48) -- (150.19,143.20);

\path[draw=drawColor,line width= 0.6pt,line join=round] (150.19,151.48) -- (150.19,143.20);

\path[draw=drawColor,line width= 0.6pt,line join=round] (150.19,151.48) -- (150.19,143.20);

\path[draw=drawColor,line width= 0.6pt,line join=round] (150.19,151.48) -- (150.19,143.20);

\path[draw=drawColor,line width= 0.6pt,line join=round] (150.19,151.48) -- (150.19,143.20);

\path[draw=drawColor,line width= 0.6pt,line join=round] (150.19,151.48) -- (150.19,143.20);

\path[draw=drawColor,line width= 0.6pt,line join=round] (150.19,151.48) -- (150.19,143.20);

\path[draw=drawColor,line width= 0.6pt,line join=round] (150.19,151.48) -- (150.19,143.20);

\path[draw=drawColor,line width= 0.6pt,line join=round] (150.19,151.48) -- (150.19,143.20);

\path[draw=drawColor,line width= 0.6pt,line join=round] (150.19,151.48) -- (150.19,143.20);

\path[draw=drawColor,line width= 0.6pt,line join=round] (150.19,151.48) -- (150.19,143.20);

\path[draw=drawColor,line width= 0.6pt,line join=round] (150.19,151.48) -- (150.19,143.20);

\path[draw=drawColor,line width= 0.6pt,line join=round] (150.19,151.48) -- (150.19,143.20);

\path[draw=drawColor,line width= 0.6pt,line join=round] (150.19,151.48) -- (150.19,143.20);

\path[draw=drawColor,line width= 0.6pt,line join=round] (150.19,151.48) -- (150.19,143.20);

\path[draw=drawColor,line width= 0.6pt,line join=round] (150.19,151.48) -- (150.19,143.20);

\path[draw=drawColor,line width= 0.6pt,line join=round] (150.19,151.48) -- (150.19,143.20);

\path[draw=drawColor,line width= 0.6pt,line join=round] (150.19,151.48) -- (150.19,143.20);

\path[draw=drawColor,line width= 0.6pt,line join=round] (150.19,151.48) -- (150.19,143.20);

\path[draw=drawColor,line width= 0.6pt,line join=round] (150.19,151.48) -- (150.19,143.20);

\path[draw=drawColor,line width= 0.6pt,line join=round] (150.19,151.48) -- (150.19,143.20);

\path[draw=drawColor,line width= 0.6pt,line join=round] (150.19,151.48) -- (150.19,143.20);

\path[draw=drawColor,line width= 0.6pt,line join=round] (150.19,151.48) -- (150.19,143.20);

\path[draw=drawColor,line width= 0.6pt,line join=round] (150.19,151.48) -- (150.19,143.20);

\path[draw=drawColor,line width= 0.6pt,line join=round] (150.19,151.48) -- (150.19,143.20);

\path[draw=drawColor,line width= 0.6pt,line join=round] (150.19,151.48) -- (150.19,143.20);

\path[draw=drawColor,line width= 0.6pt,line join=round] (150.19,151.48) -- (150.19,143.20);

\path[draw=drawColor,line width= 0.6pt,line join=round] (150.19,151.48) -- (150.19,143.20);

\path[draw=drawColor,line width= 0.6pt,line join=round] (150.19,151.48) -- (150.19,143.20);

\path[draw=drawColor,line width= 0.6pt,line join=round] (150.19,151.48) -- (150.19,143.20);

\path[draw=drawColor,line width= 0.6pt,line join=round] (150.19,151.48) -- (150.19,143.20);

\path[draw=drawColor,line width= 0.6pt,line join=round] (150.19,151.48) -- (150.19,143.20);

\path[draw=drawColor,line width= 0.6pt,line join=round] (150.19,151.48) -- (150.19,143.20);

\path[draw=drawColor,line width= 0.6pt,line join=round] (150.19,151.48) -- (150.19,143.20);

\path[draw=drawColor,line width= 0.6pt,line join=round] (150.19,151.48) -- (150.19,143.20);

\path[draw=drawColor,line width= 0.6pt,line join=round] (150.19,151.48) -- (150.19,143.20);

\path[draw=drawColor,line width= 0.6pt,line join=round] (150.19,151.48) -- (150.19,143.20);

\path[draw=drawColor,line width= 0.6pt,line join=round] (150.19,151.48) -- (150.19,143.20);

\path[draw=drawColor,line width= 0.6pt,line join=round] (150.19,151.48) -- (150.19,143.20);

\path[draw=drawColor,line width= 0.6pt,line join=round] (150.19,151.48) -- (150.19,143.20);

\path[draw=drawColor,line width= 0.6pt,line join=round] (150.19,151.48) -- (150.19,143.20);

\path[draw=drawColor,line width= 0.6pt,line join=round] (150.19,151.48) -- (150.19,143.20);

\path[draw=drawColor,line width= 0.6pt,line join=round] (150.19,151.48) -- (150.19,143.20);

\path[draw=drawColor,line width= 0.6pt,line join=round] (150.19,151.48) -- (150.19,143.20);

\path[draw=drawColor,line width= 0.6pt,line join=round] (150.19,151.48) -- (150.19,143.20);

\path[draw=drawColor,line width= 0.6pt,line join=round] (150.19,151.48) -- (150.19,143.20);

\path[draw=drawColor,line width= 0.6pt,line join=round] (150.19,151.48) -- (150.19,143.20);

\path[draw=drawColor,line width= 0.6pt,line join=round] (150.19,151.48) -- (150.19,143.20);

\path[draw=drawColor,line width= 0.6pt,line join=round] (150.19,151.48) -- (150.19,143.20);

\path[draw=drawColor,line width= 0.6pt,line join=round] (150.19,151.48) -- (150.19,143.20);

\path[draw=drawColor,line width= 0.6pt,line join=round] (150.19,151.48) -- (150.19,143.20);

\path[draw=drawColor,line width= 0.6pt,line join=round] (150.19,151.48) -- (150.19,143.20);

\path[draw=drawColor,line width= 0.6pt,line join=round] (150.19,151.48) -- (150.19,143.20);

\path[draw=drawColor,line width= 0.6pt,line join=round] (150.19,151.48) -- (150.19,143.20);

\path[draw=drawColor,line width= 0.6pt,line join=round] (150.19,151.48) -- (150.19,143.20);

\path[draw=drawColor,line width= 0.6pt,line join=round] (150.19,151.48) -- (150.19,143.20);

\path[draw=drawColor,line width= 0.6pt,line join=round] (150.19,151.48) -- (150.19,143.20);

\path[draw=drawColor,line width= 0.6pt,line join=round] (150.19,151.48) -- (150.19,143.20);

\path[draw=drawColor,line width= 0.6pt,line join=round] (150.19,151.48) -- (150.19,143.20);

\path[draw=drawColor,line width= 0.6pt,line join=round] (150.19,151.48) -- (150.19,143.20);

\path[draw=drawColor,line width= 0.6pt,line join=round] (150.19,151.48) -- (150.19,143.20);

\path[draw=drawColor,line width= 0.6pt,line join=round] (150.19,151.48) -- (150.19,143.20);

\path[draw=drawColor,line width= 0.6pt,line join=round] (150.19,151.48) -- (150.19,143.20);

\path[draw=drawColor,line width= 0.6pt,line join=round] (150.19,151.48) -- (150.19,143.20);

\path[draw=drawColor,line width= 0.6pt,line join=round] (150.19,151.48) -- (150.19,143.20);

\path[draw=drawColor,line width= 0.6pt,line join=round] (150.19,151.48) -- (150.19,143.20);

\path[draw=drawColor,line width= 0.6pt,line join=round] (150.19,151.48) -- (150.19,143.20);

\path[draw=drawColor,line width= 0.6pt,line join=round] (150.19,151.48) -- (150.19,143.20);

\path[draw=drawColor,line width= 0.6pt,line join=round] (150.19,151.48) -- (150.19,143.20);

\path[draw=drawColor,line width= 0.6pt,line join=round] (150.19,151.48) -- (150.19,143.20);

\path[draw=drawColor,line width= 0.6pt,line join=round] (150.19,151.48) -- (150.19,143.20);

\path[draw=drawColor,line width= 0.6pt,line join=round] (150.19,151.48) -- (150.19,143.20);

\path[draw=drawColor,line width= 0.6pt,line join=round] (150.19,151.48) -- (150.19,143.20);

\path[draw=drawColor,line width= 0.6pt,line join=round] (150.19,151.48) -- (150.19,143.20);

\path[draw=drawColor,line width= 0.6pt,line join=round] (150.19,151.48) -- (150.19,143.20);

\path[draw=drawColor,line width= 0.6pt,line join=round] (150.19,151.48) -- (150.19,143.20);

\path[draw=drawColor,line width= 0.6pt,line join=round] (150.19,151.48) -- (150.19,143.20);

\path[draw=drawColor,line width= 0.6pt,line join=round] (150.19,151.48) -- (150.19,143.20);

\path[draw=drawColor,line width= 0.6pt,line join=round] (150.19,151.48) -- (150.19,143.20);

\path[draw=drawColor,line width= 0.6pt,line join=round] (150.19,151.48) -- (150.19,143.20);

\path[draw=drawColor,line width= 0.6pt,line join=round] (150.19,151.48) -- (150.19,143.20);

\path[draw=drawColor,line width= 0.6pt,line join=round] (150.19,151.48) -- (150.19,143.20);

\path[draw=drawColor,line width= 0.6pt,line join=round] (150.19,151.48) -- (150.19,143.20);

\path[draw=drawColor,line width= 0.6pt,line join=round] (150.19,151.48) -- (150.19,143.20);

\path[draw=drawColor,line width= 0.6pt,line join=round] (150.19,151.48) -- (150.19,143.20);

\path[draw=drawColor,line width= 0.6pt,line join=round] (150.19,151.48) -- (150.19,143.20);

\path[draw=drawColor,line width= 0.6pt,line join=round] (150.19,151.48) -- (150.19,143.20);

\path[draw=drawColor,line width= 0.6pt,line join=round] (150.19,151.48) -- (150.19,143.20);

\path[draw=drawColor,line width= 0.6pt,line join=round] (150.19,151.48) -- (150.19,143.20);

\path[draw=drawColor,line width= 0.6pt,line join=round] (150.19,151.48) -- (150.19,143.20);

\path[draw=drawColor,line width= 0.6pt,line join=round] (150.19,151.48) -- (150.19,143.20);

\path[draw=drawColor,line width= 0.6pt,line join=round] (150.19,151.48) -- (150.19,143.20);

\path[draw=drawColor,line width= 0.6pt,line join=round] (150.19,151.48) -- (150.19,143.20);

\path[draw=drawColor,line width= 0.6pt,line join=round] (150.19,151.48) -- (150.19,143.20);

\path[draw=drawColor,line width= 0.6pt,line join=round] (150.19,151.48) -- (150.19,143.20);

\path[draw=drawColor,line width= 0.6pt,line join=round] (150.19,151.48) -- (150.19,143.20);

\path[draw=drawColor,line width= 0.6pt,line join=round] (150.19,151.48) -- (150.19,143.20);

\path[draw=drawColor,line width= 0.6pt,line join=round] (150.19,151.48) -- (150.19,143.20);

\path[draw=drawColor,line width= 0.6pt,line join=round] (150.19,151.48) -- (150.19,143.20);

\path[draw=drawColor,line width= 0.6pt,line join=round] (150.19,151.48) -- (150.19,143.20);

\path[draw=drawColor,line width= 0.6pt,line join=round] (150.19,151.48) -- (150.19,143.20);

\path[draw=drawColor,line width= 0.6pt,line join=round] (150.19,151.48) -- (150.19,143.20);

\path[draw=drawColor,line width= 0.6pt,line join=round] (150.19,151.48) -- (150.19,143.20);

\path[draw=drawColor,line width= 0.6pt,line join=round] (150.19,151.48) -- (150.19,143.20);

\path[draw=drawColor,line width= 0.6pt,line join=round] (150.19,151.48) -- (150.19,143.20);

\path[draw=drawColor,line width= 0.6pt,line join=round] (150.19,151.48) -- (150.19,143.20);

\path[draw=drawColor,line width= 0.6pt,line join=round] (150.19,151.48) -- (150.19,143.20);

\path[draw=drawColor,line width= 0.6pt,line join=round] (150.19,151.48) -- (150.19,143.20);

\path[draw=drawColor,line width= 0.6pt,line join=round] (150.19,151.48) -- (150.19,143.20);

\path[draw=drawColor,line width= 0.6pt,line join=round] (150.19,151.48) -- (150.19,143.20);

\path[draw=drawColor,line width= 0.6pt,line join=round] (150.19,151.48) -- (150.19,143.20);

\path[draw=drawColor,line width= 0.6pt,line join=round] (150.19,151.48) -- (150.19,143.20);

\path[draw=drawColor,line width= 0.6pt,line join=round] (150.19,151.48) -- (150.19,143.20);

\path[draw=drawColor,line width= 0.6pt,line join=round] (150.19,151.48) -- (150.19,143.20);

\path[draw=drawColor,line width= 0.6pt,line join=round] (150.19,151.48) -- (150.19,143.20);

\path[draw=drawColor,line width= 0.6pt,line join=round] (150.19,151.48) -- (150.19,143.20);

\path[draw=drawColor,line width= 0.6pt,line join=round] (150.19,151.48) -- (150.19,143.20);

\path[draw=drawColor,line width= 0.6pt,line join=round] (150.19,151.48) -- (150.19,143.20);

\path[draw=drawColor,line width= 0.6pt,line join=round] (150.19,151.48) -- (150.19,143.20);

\path[draw=drawColor,line width= 0.6pt,line join=round] (150.19,151.48) -- (150.19,143.20);

\path[draw=drawColor,line width= 0.6pt,line join=round] (150.19,151.48) -- (150.19,143.20);

\path[draw=drawColor,line width= 0.6pt,line join=round] (150.19,151.48) -- (150.19,143.20);

\path[draw=drawColor,line width= 0.6pt,line join=round] (150.19,151.48) -- (150.19,143.20);

\path[draw=drawColor,line width= 0.6pt,line join=round] (150.19,151.48) -- (150.19,143.20);

\path[draw=drawColor,line width= 0.6pt,line join=round] (150.19,151.48) -- (150.19,143.20);

\path[draw=drawColor,line width= 0.6pt,line join=round] (150.19,151.48) -- (150.19,143.20);

\path[draw=drawColor,line width= 0.6pt,line join=round] (150.19,151.48) -- (150.19,143.20);

\path[draw=drawColor,line width= 0.6pt,line join=round] (150.19,151.48) -- (150.19,143.20);

\path[draw=drawColor,line width= 0.6pt,line join=round] (150.19,151.48) -- (150.19,143.20);

\path[draw=drawColor,line width= 0.6pt,line join=round] (150.19,151.48) -- (150.19,143.20);

\path[draw=drawColor,line width= 0.6pt,line join=round] (150.19,151.48) -- (150.19,143.20);

\path[draw=drawColor,line width= 0.6pt,line join=round] (150.19,151.48) -- (150.19,143.20);

\path[draw=drawColor,line width= 0.6pt,line join=round] (150.19,151.48) -- (150.19,143.20);

\path[draw=drawColor,line width= 0.6pt,line join=round] (150.19,151.48) -- (150.19,143.20);

\path[draw=drawColor,line width= 0.6pt,line join=round] (150.19,151.48) -- (150.19,143.20);

\path[draw=drawColor,line width= 0.6pt,line join=round] (150.19,151.48) -- (150.19,143.20);

\path[draw=drawColor,line width= 0.6pt,line join=round] (150.19,151.48) -- (150.19,143.20);

\path[draw=drawColor,line width= 0.6pt,line join=round] (150.19,151.48) -- (150.19,143.20);

\path[draw=drawColor,line width= 0.6pt,line join=round] (150.19,151.48) -- (150.19,143.20);

\path[draw=drawColor,line width= 0.6pt,line join=round] (150.19,151.48) -- (150.19,143.20);

\path[draw=drawColor,line width= 0.6pt,line join=round] (150.19,151.48) -- (150.19,143.20);

\path[draw=drawColor,line width= 0.6pt,line join=round] (150.19,151.48) -- (150.19,143.20);

\path[draw=drawColor,line width= 0.6pt,line join=round] (150.19,151.48) -- (150.19,143.20);

\path[draw=drawColor,line width= 0.6pt,line join=round] (150.19,151.48) -- (150.19,143.20);

\path[draw=drawColor,line width= 0.6pt,line join=round] (150.19,151.48) -- (150.19,143.20);

\path[draw=drawColor,line width= 0.6pt,line join=round] (150.19,151.48) -- (150.19,143.20);

\path[draw=drawColor,line width= 0.6pt,line join=round] (150.19,151.48) -- (150.19,143.20);

\path[draw=drawColor,line width= 0.6pt,line join=round] (150.19,151.48) -- (150.19,143.20);

\path[draw=drawColor,line width= 0.6pt,line join=round] (150.19,151.48) -- (150.19,143.20);

\path[draw=drawColor,line width= 0.6pt,line join=round] (150.19,151.48) -- (150.19,143.20);

\path[draw=drawColor,line width= 0.6pt,line join=round] (150.19,151.48) -- (150.19,143.20);

\path[draw=drawColor,line width= 0.6pt,line join=round] (150.19,151.48) -- (150.19,143.20);

\path[draw=drawColor,line width= 0.6pt,line join=round] (150.19,151.48) -- (150.19,143.20);

\path[draw=drawColor,line width= 0.6pt,line join=round] (150.19,151.48) -- (150.19,143.20);

\path[draw=drawColor,line width= 0.6pt,line join=round] (150.19,151.48) -- (150.19,143.20);

\path[draw=drawColor,line width= 0.6pt,line join=round] (150.19,151.48) -- (150.19,143.20);

\path[draw=drawColor,line width= 0.6pt,line join=round] (150.19,151.48) -- (150.19,143.20);

\path[draw=drawColor,line width= 0.6pt,line join=round] (150.19,151.48) -- (150.19,143.20);

\path[draw=drawColor,line width= 0.6pt,line join=round] (150.19,151.48) -- (150.19,143.20);

\path[draw=drawColor,line width= 0.6pt,line join=round] (150.19,151.48) -- (150.19,143.20);

\path[draw=drawColor,line width= 0.6pt,line join=round] (150.19,151.48) -- (150.19,143.20);

\path[draw=drawColor,line width= 0.6pt,line join=round] (150.19,151.48) -- (150.19,143.20);

\path[draw=drawColor,line width= 0.6pt,line join=round] (150.19,151.48) -- (150.19,143.20);

\path[draw=drawColor,line width= 0.6pt,line join=round] (150.19,151.48) -- (150.19,143.20);

\path[draw=drawColor,line width= 0.6pt,line join=round] (150.19,151.48) -- (150.19,143.20);

\path[draw=drawColor,line width= 0.6pt,line join=round] (150.19,151.48) -- (150.19,143.20);

\path[draw=drawColor,line width= 0.6pt,line join=round] (150.19,151.48) -- (150.19,143.20);

\path[draw=drawColor,line width= 0.6pt,line join=round] (150.19,151.48) -- (150.19,143.20);

\path[draw=drawColor,line width= 0.6pt,line join=round] (150.19,151.48) -- (150.19,143.20);

\path[draw=drawColor,line width= 0.6pt,line join=round] (150.19,151.48) -- (150.19,143.20);

\path[draw=drawColor,line width= 0.6pt,line join=round] (150.19,151.48) -- (150.19,143.20);

\path[draw=drawColor,line width= 0.6pt,line join=round] (150.19,151.48) -- (150.19,143.20);

\path[draw=drawColor,line width= 0.6pt,line join=round] (150.19,151.48) -- (150.19,143.20);

\path[draw=drawColor,line width= 0.6pt,line join=round] (150.19,151.48) -- (150.19,143.20);

\path[draw=drawColor,line width= 0.6pt,line join=round] (150.19,151.48) -- (150.19,143.20);

\path[draw=drawColor,line width= 0.6pt,line join=round] (150.19,151.48) -- (150.19,143.20);

\path[draw=drawColor,line width= 0.6pt,line join=round] (150.19,151.48) -- (150.19,143.20);

\path[draw=drawColor,line width= 0.6pt,line join=round] (150.19,151.48) -- (150.19,143.20);

\path[draw=drawColor,line width= 0.6pt,line join=round] (150.19,151.48) -- (150.19,143.20);

\path[draw=drawColor,line width= 0.6pt,line join=round] (150.19,151.48) -- (150.19,143.20);

\path[draw=drawColor,line width= 0.6pt,line join=round] (150.19,151.48) -- (150.19,143.20);

\path[draw=drawColor,line width= 0.6pt,line join=round] (150.19,151.48) -- (150.19,143.20);

\path[draw=drawColor,line width= 0.6pt,line join=round] (150.19,151.48) -- (150.19,143.20);

\path[draw=drawColor,line width= 0.6pt,line join=round] (150.19,151.48) -- (150.19,143.20);

\path[draw=drawColor,line width= 0.6pt,line join=round] (150.19,151.48) -- (150.19,143.20);

\path[draw=drawColor,line width= 0.6pt,line join=round] (150.19,151.48) -- (150.19,143.20);

\path[draw=drawColor,line width= 0.6pt,line join=round] (150.19,151.48) -- (150.19,143.20);

\path[draw=drawColor,line width= 0.6pt,line join=round] (150.19,151.48) -- (150.19,143.20);

\path[draw=drawColor,line width= 0.6pt,line join=round] (150.19,151.48) -- (150.19,143.20);

\path[draw=drawColor,line width= 0.6pt,line join=round] (150.19,151.48) -- (150.19,143.20);

\path[draw=drawColor,line width= 0.6pt,line join=round] (150.19,151.48) -- (150.19,143.20);

\path[draw=drawColor,line width= 0.6pt,line join=round] (150.19,151.48) -- (150.19,143.20);

\path[draw=drawColor,line width= 0.6pt,line join=round] (150.19,151.48) -- (150.19,143.20);

\path[draw=drawColor,line width= 0.6pt,line join=round] (150.19,151.48) -- (150.19,143.20);

\path[draw=drawColor,line width= 0.6pt,line join=round] (150.19,151.48) -- (150.19,143.20);

\path[draw=drawColor,line width= 0.6pt,line join=round] (150.19,151.48) -- (150.19,143.20);

\path[draw=drawColor,line width= 0.6pt,line join=round] (150.19,151.48) -- (150.19,143.20);

\path[draw=drawColor,line width= 0.6pt,line join=round] (150.19,151.48) -- (150.19,143.20);

\path[draw=drawColor,line width= 0.6pt,line join=round] (150.19,151.48) -- (150.19,143.20);

\path[draw=drawColor,line width= 0.6pt,line join=round] (150.19,151.48) -- (150.19,143.20);

\path[draw=drawColor,line width= 0.6pt,line join=round] (150.19,151.48) -- (150.19,143.20);

\path[draw=drawColor,line width= 0.6pt,line join=round] (150.19,151.48) -- (150.19,143.20);

\path[draw=drawColor,line width= 0.6pt,line join=round] (150.19,151.48) -- (150.19,143.20);

\path[draw=drawColor,line width= 0.6pt,line join=round] (150.19,151.48) -- (150.19,143.20);

\path[draw=drawColor,line width= 0.6pt,line join=round] (150.19,151.48) -- (150.19,143.20);

\path[draw=drawColor,line width= 0.6pt,line join=round] (150.19,151.48) -- (150.19,143.20);

\path[draw=drawColor,line width= 0.6pt,line join=round] (150.19,151.48) -- (150.19,143.20);

\path[draw=drawColor,line width= 0.6pt,line join=round] (150.19,151.48) -- (150.19,143.20);

\path[draw=drawColor,line width= 0.6pt,line join=round] (150.19,151.48) -- (150.19,143.20);

\path[draw=drawColor,line width= 0.6pt,line join=round] (150.19,151.48) -- (150.19,143.20);

\path[draw=drawColor,line width= 0.6pt,line join=round] (150.19,151.48) -- (150.19,143.20);

\path[draw=drawColor,line width= 0.6pt,line join=round] (150.19,151.48) -- (150.19,143.20);

\path[draw=drawColor,line width= 0.6pt,line join=round] (150.19,151.48) -- (150.19,143.20);

\path[draw=drawColor,line width= 0.6pt,line join=round] (150.19,151.48) -- (150.19,143.20);

\path[draw=drawColor,line width= 0.6pt,line join=round] (150.19,151.48) -- (150.19,143.20);

\path[draw=drawColor,line width= 0.6pt,line join=round] (150.19,151.48) -- (150.19,143.20);

\path[draw=drawColor,line width= 0.6pt,line join=round] (150.19,151.48) -- (150.19,143.20);

\path[draw=drawColor,line width= 0.6pt,line join=round] (150.19,151.48) -- (150.19,143.20);

\path[draw=drawColor,line width= 0.6pt,line join=round] (150.19,151.48) -- (150.19,143.20);

\path[draw=drawColor,line width= 0.6pt,line join=round] (150.19,151.48) -- (150.19,143.20);

\path[draw=drawColor,line width= 0.6pt,line join=round] (150.19,151.48) -- (150.19,143.20);

\path[draw=drawColor,line width= 0.6pt,line join=round] (150.19,151.48) -- (150.19,143.20);

\path[draw=drawColor,line width= 0.6pt,line join=round] (150.19,151.48) -- (150.19,143.20);

\path[draw=drawColor,line width= 0.6pt,line join=round] (150.19,151.48) -- (150.19,143.20);

\path[draw=drawColor,line width= 0.6pt,line join=round] (150.19,151.48) -- (150.19,143.20);

\path[draw=drawColor,line width= 0.6pt,line join=round] (150.19,151.48) -- (150.19,143.20);

\path[draw=drawColor,line width= 0.6pt,line join=round] (150.19,151.48) -- (150.19,143.20);

\path[draw=drawColor,line width= 0.6pt,line join=round] (150.19,151.48) -- (150.19,143.20);

\path[draw=drawColor,line width= 0.6pt,line join=round] (150.19,151.48) -- (150.19,143.20);

\path[draw=drawColor,line width= 0.6pt,line join=round] (150.19,151.48) -- (150.19,143.20);

\path[draw=drawColor,line width= 0.6pt,line join=round] (150.19,151.48) -- (150.19,143.20);

\path[draw=drawColor,line width= 0.6pt,line join=round] (150.19,151.48) -- (150.19,143.20);

\path[draw=drawColor,line width= 0.6pt,line join=round] (150.19,151.48) -- (150.19,143.20);

\path[draw=drawColor,line width= 0.6pt,line join=round] (150.19,151.48) -- (150.19,143.20);

\path[draw=drawColor,line width= 0.6pt,line join=round] (150.19,151.48) -- (150.19,143.20);

\path[draw=drawColor,line width= 0.6pt,line join=round] (150.19,151.48) -- (150.19,143.20);

\path[draw=drawColor,line width= 0.6pt,line join=round] (150.19,151.48) -- (150.19,143.20);

\path[draw=drawColor,line width= 0.6pt,line join=round] (150.19,151.48) -- (150.19,143.20);

\path[draw=drawColor,line width= 0.6pt,line join=round] (150.19,151.48) -- (150.19,143.20);

\path[draw=drawColor,line width= 0.6pt,line join=round] (150.19,151.48) -- (150.19,143.20);

\path[draw=drawColor,line width= 0.6pt,line join=round] (150.19,151.48) -- (150.19,143.20);

\path[draw=drawColor,line width= 0.6pt,line join=round] (150.19,151.48) -- (150.19,143.20);

\path[draw=drawColor,line width= 0.6pt,line join=round] (150.19,151.48) -- (150.19,143.20);

\path[draw=drawColor,line width= 0.6pt,line join=round] (150.19,151.48) -- (150.19,143.20);

\path[draw=drawColor,line width= 0.6pt,line join=round] (150.19,151.48) -- (150.19,143.20);

\path[draw=drawColor,line width= 0.6pt,line join=round] (150.19,151.48) -- (150.19,143.20);

\path[draw=drawColor,line width= 0.6pt,line join=round] (150.19,151.48) -- (150.19,143.20);

\path[draw=drawColor,line width= 0.6pt,line join=round] (150.19,151.48) -- (150.19,143.20);

\path[draw=drawColor,line width= 0.6pt,line join=round] (150.19,151.48) -- (150.19,143.20);

\path[draw=drawColor,line width= 0.6pt,line join=round] (150.19,151.48) -- (150.19,143.20);

\path[draw=drawColor,line width= 0.6pt,line join=round] (150.19,151.48) -- (150.19,143.20);

\path[draw=drawColor,line width= 0.6pt,line join=round] (150.19,151.48) -- (150.19,143.20);

\path[draw=drawColor,line width= 0.6pt,line join=round] (150.19,151.48) -- (150.19,143.20);

\path[draw=drawColor,line width= 0.6pt,line join=round] (150.19,151.48) -- (150.19,143.20);

\path[draw=drawColor,line width= 0.6pt,line join=round] (150.19,151.48) -- (150.19,143.20);

\path[draw=drawColor,line width= 0.6pt,line join=round] (150.19,151.48) -- (150.19,143.20);

\path[draw=drawColor,line width= 0.6pt,line join=round] (150.19,151.48) -- (150.19,143.20);

\path[draw=drawColor,line width= 0.6pt,line join=round] (150.19,151.48) -- (150.19,143.20);

\path[draw=drawColor,line width= 0.6pt,line join=round] (150.19,151.48) -- (150.19,143.20);

\path[draw=drawColor,line width= 0.6pt,line join=round] (150.19,151.48) -- (150.19,143.20);

\path[draw=drawColor,line width= 0.6pt,line join=round] (150.19,151.48) -- (150.19,143.20);

\path[draw=drawColor,line width= 0.6pt,line join=round] (150.19,151.48) -- (150.19,143.20);

\path[draw=drawColor,line width= 0.6pt,line join=round] (150.19,151.48) -- (150.19,143.20);

\path[draw=drawColor,line width= 0.6pt,line join=round] (150.19,151.48) -- (150.19,143.20);

\path[draw=drawColor,line width= 0.6pt,line join=round] (150.19,151.48) -- (150.19,143.20);

\path[draw=drawColor,line width= 0.6pt,line join=round] (150.19,151.48) -- (150.19,143.20);

\path[draw=drawColor,line width= 0.6pt,line join=round] (150.19,151.48) -- (150.19,143.20);

\path[draw=drawColor,line width= 0.6pt,line join=round] (150.19,151.48) -- (150.19,143.20);

\path[draw=drawColor,line width= 0.6pt,line join=round] (150.19,151.48) -- (150.19,143.20);

\path[draw=drawColor,line width= 0.6pt,line join=round] (150.19,151.48) -- (150.19,143.20);

\path[draw=drawColor,line width= 0.6pt,line join=round] (150.19,151.48) -- (150.19,143.20);

\path[draw=drawColor,line width= 0.6pt,line join=round] (150.19,151.48) -- (150.19,143.20);

\path[draw=drawColor,line width= 0.6pt,line join=round] (150.19,151.48) -- (150.19,143.20);

\path[draw=drawColor,line width= 0.6pt,line join=round] (150.19,151.48) -- (150.19,143.20);

\path[draw=drawColor,line width= 0.6pt,line join=round] (150.19,151.48) -- (150.19,143.20);

\path[draw=drawColor,line width= 0.6pt,line join=round] (150.19,151.48) -- (150.19,143.20);

\path[draw=drawColor,line width= 0.6pt,line join=round] (150.19,151.48) -- (150.19,143.20);

\path[draw=drawColor,line width= 0.6pt,line join=round] (150.19,151.48) -- (150.19,143.20);

\path[draw=drawColor,line width= 0.6pt,line join=round] (150.19,151.48) -- (150.19,143.20);

\path[draw=drawColor,line width= 0.6pt,line join=round] (150.19,151.48) -- (150.19,143.20);

\path[draw=drawColor,line width= 0.6pt,line join=round] (150.19,151.48) -- (150.19,143.20);

\path[draw=drawColor,line width= 0.6pt,line join=round] (150.19,151.48) -- (150.19,143.20);

\path[draw=drawColor,line width= 0.6pt,line join=round] (150.19,151.48) -- (150.19,143.20);

\path[draw=drawColor,line width= 0.6pt,line join=round] (150.19,151.48) -- (150.19,143.20);

\path[draw=drawColor,line width= 0.6pt,line join=round] (150.19,151.48) -- (150.19,143.20);

\path[draw=drawColor,line width= 0.6pt,line join=round] (150.19,151.48) -- (150.19,143.20);

\path[draw=drawColor,line width= 0.6pt,line join=round] (150.19,151.48) -- (150.19,143.20);

\path[draw=drawColor,line width= 0.6pt,line join=round] (150.19,151.48) -- (150.19,143.20);

\path[draw=drawColor,line width= 0.6pt,line join=round] (150.19,151.48) -- (150.19,143.20);

\path[draw=drawColor,line width= 0.6pt,line join=round] (150.19,151.48) -- (150.19,143.20);

\path[draw=drawColor,line width= 0.6pt,line join=round] (150.19,151.48) -- (150.19,143.20);

\path[draw=drawColor,line width= 0.6pt,line join=round] (150.19,151.48) -- (150.19,143.20);

\path[draw=drawColor,line width= 0.6pt,line join=round] (150.19,151.48) -- (150.19,143.20);

\path[draw=drawColor,line width= 0.6pt,line join=round] (150.19,151.48) -- (150.19,143.20);

\path[draw=drawColor,line width= 0.6pt,line join=round] (150.19,151.48) -- (150.19,143.20);

\path[draw=drawColor,line width= 0.6pt,line join=round] (150.19,151.48) -- (150.19,143.20);

\path[draw=drawColor,line width= 0.6pt,line join=round] (150.19,151.48) -- (150.19,143.20);

\path[draw=drawColor,line width= 0.6pt,line join=round] (150.19,151.48) -- (150.19,143.20);

\path[draw=drawColor,line width= 0.6pt,line join=round] (150.19,151.48) -- (150.19,143.20);

\path[draw=drawColor,line width= 0.6pt,line join=round] (150.19,151.48) -- (150.19,143.20);

\path[draw=drawColor,line width= 0.6pt,line join=round] (150.19,151.48) -- (150.19,143.20);

\path[draw=drawColor,line width= 0.6pt,line join=round] (150.19,151.48) -- (150.19,143.20);

\path[draw=drawColor,line width= 0.6pt,line join=round] (150.19,151.48) -- (150.19,143.20);

\path[draw=drawColor,line width= 0.6pt,line join=round] (150.19,151.48) -- (150.19,143.20);

\path[draw=drawColor,line width= 0.6pt,line join=round] (150.19,151.48) -- (150.19,143.20);

\path[draw=drawColor,line width= 0.6pt,line join=round] (150.19,151.48) -- (150.19,143.20);

\path[draw=drawColor,line width= 0.6pt,line join=round] (150.19,151.48) -- (150.19,143.20);

\path[draw=drawColor,line width= 0.6pt,line join=round] (150.19,151.48) -- (150.19,143.20);

\path[draw=drawColor,line width= 0.6pt,line join=round] (150.19,151.48) -- (150.19,143.20);

\path[draw=drawColor,line width= 0.6pt,line join=round] (150.19,151.48) -- (150.19,143.20);

\path[draw=drawColor,line width= 0.6pt,line join=round] (150.19,151.48) -- (150.19,143.20);

\path[draw=drawColor,line width= 0.6pt,line join=round] (150.19,151.48) -- (150.19,143.20);

\path[draw=drawColor,line width= 0.6pt,line join=round] (150.19,151.48) -- (150.19,143.20);

\path[draw=drawColor,line width= 0.6pt,line join=round] (150.19,151.48) -- (150.19,143.20);

\path[draw=drawColor,line width= 0.6pt,line join=round] (150.19,151.48) -- (150.19,143.20);

\path[draw=drawColor,line width= 0.6pt,line join=round] (150.19,151.48) -- (150.19,143.20);

\path[draw=drawColor,line width= 0.6pt,line join=round] (150.19,151.48) -- (150.19,143.20);

\path[draw=drawColor,line width= 0.6pt,line join=round] (150.19,151.48) -- (150.19,143.20);

\path[draw=drawColor,line width= 0.6pt,line join=round] (150.19,151.48) -- (150.19,143.20);

\path[draw=drawColor,line width= 0.6pt,line join=round] (150.19,151.48) -- (150.19,143.20);

\path[draw=drawColor,line width= 0.6pt,line join=round] (150.19,151.48) -- (150.19,143.20);

\path[draw=drawColor,line width= 0.6pt,line join=round] (150.19,151.48) -- (150.19,143.20);

\path[draw=drawColor,line width= 0.6pt,line join=round] (150.19,151.48) -- (150.19,143.20);

\path[draw=drawColor,line width= 0.6pt,line join=round] (150.19,151.48) -- (150.19,143.20);

\path[draw=drawColor,line width= 0.6pt,line join=round] (150.19,151.48) -- (150.19,143.20);

\path[draw=drawColor,line width= 0.6pt,line join=round] (150.19,151.48) -- (150.19,143.20);

\path[draw=drawColor,line width= 0.6pt,line join=round] (150.19,151.48) -- (150.19,143.20);

\path[draw=drawColor,line width= 0.6pt,line join=round] (150.19,151.48) -- (150.19,143.20);

\path[draw=drawColor,line width= 0.6pt,line join=round] (150.19,151.48) -- (150.19,143.20);

\path[draw=drawColor,line width= 0.6pt,line join=round] (150.19,151.48) -- (150.19,143.20);

\path[draw=drawColor,line width= 0.6pt,line join=round] (150.19,151.48) -- (150.19,143.20);

\path[draw=drawColor,line width= 0.6pt,line join=round] (150.19,151.48) -- (150.19,143.20);

\path[draw=drawColor,line width= 0.6pt,line join=round] (150.19,151.48) -- (150.19,143.20);

\path[draw=drawColor,line width= 0.6pt,line join=round] (150.19,151.48) -- (150.19,143.20);

\path[draw=drawColor,line width= 0.6pt,line join=round] (150.19,151.48) -- (150.19,143.20);

\path[draw=drawColor,line width= 0.6pt,line join=round] (150.19,151.48) -- (150.19,143.20);

\path[draw=drawColor,line width= 0.6pt,line join=round] (150.19,151.48) -- (150.19,143.20);

\path[draw=drawColor,line width= 0.6pt,line join=round] (150.19,151.48) -- (150.19,143.20);

\path[draw=drawColor,line width= 0.6pt,line join=round] (150.19,151.48) -- (150.19,143.20);

\path[draw=drawColor,line width= 0.6pt,line join=round] (150.19,151.48) -- (150.19,143.20);

\path[draw=drawColor,line width= 0.6pt,line join=round] (150.19,151.48) -- (150.19,143.20);

\path[draw=drawColor,line width= 0.6pt,line join=round] (150.19,151.48) -- (150.19,143.20);

\path[draw=drawColor,line width= 0.6pt,line join=round] (150.19,151.48) -- (150.19,143.20);

\path[draw=drawColor,line width= 0.6pt,line join=round] (150.19,151.48) -- (150.19,143.20);

\path[draw=drawColor,line width= 0.6pt,line join=round] (150.19,151.48) -- (150.19,143.20);

\path[draw=drawColor,line width= 0.6pt,line join=round] (150.19,151.48) -- (150.19,143.20);

\path[draw=drawColor,line width= 0.6pt,line join=round] (150.19,151.48) -- (150.19,143.20);

\path[draw=drawColor,line width= 0.6pt,line join=round] (150.19,151.48) -- (150.19,143.20);

\path[draw=drawColor,line width= 0.6pt,line join=round] (150.19,151.48) -- (150.19,143.20);

\path[draw=drawColor,line width= 0.6pt,line join=round] (150.19,151.48) -- (150.19,143.20);

\path[draw=drawColor,line width= 0.6pt,line join=round] (150.19,151.48) -- (150.19,143.20);

\path[draw=drawColor,line width= 0.6pt,line join=round] (150.19,151.48) -- (150.19,143.20);

\path[draw=drawColor,line width= 0.6pt,line join=round] (150.19,151.48) -- (150.19,143.20);

\path[draw=drawColor,line width= 0.6pt,line join=round] (150.19,151.48) -- (150.19,143.20);

\path[draw=drawColor,line width= 0.6pt,line join=round] (150.19,151.48) -- (150.19,143.20);

\path[draw=drawColor,line width= 0.6pt,line join=round] (150.19,151.48) -- (150.19,143.20);

\path[draw=drawColor,line width= 0.6pt,line join=round] (150.19,151.48) -- (150.19,143.20);

\path[draw=drawColor,line width= 0.6pt,line join=round] (150.19,151.48) -- (150.19,143.20);

\path[draw=drawColor,line width= 0.6pt,line join=round] (150.19,151.48) -- (150.19,143.20);

\path[draw=drawColor,line width= 0.6pt,line join=round] (150.19,151.48) -- (150.19,143.20);

\path[draw=drawColor,line width= 0.6pt,line join=round] (150.19,151.48) -- (150.19,143.20);

\path[draw=drawColor,line width= 0.6pt,line join=round] (150.19,151.48) -- (150.19,143.20);

\path[draw=drawColor,line width= 0.6pt,line join=round] (150.19,151.48) -- (150.19,143.20);

\path[draw=drawColor,line width= 0.6pt,line join=round] (150.19,151.48) -- (150.19,143.20);

\path[draw=drawColor,line width= 0.6pt,line join=round] (150.19,151.48) -- (150.19,143.20);

\path[draw=drawColor,line width= 0.6pt,line join=round] (150.19,151.48) -- (150.19,143.20);

\path[draw=drawColor,line width= 0.6pt,line join=round] (150.19,151.48) -- (150.19,143.20);

\path[draw=drawColor,line width= 0.6pt,line join=round] (150.19,151.48) -- (150.19,143.20);

\path[draw=drawColor,line width= 0.6pt,line join=round] (150.19,151.48) -- (150.19,143.20);

\path[draw=drawColor,line width= 0.6pt,line join=round] (150.19,151.48) -- (150.19,143.20);

\path[draw=drawColor,line width= 0.6pt,line join=round] (150.19,151.48) -- (150.19,143.20);

\path[draw=drawColor,line width= 0.6pt,line join=round] (150.19,151.48) -- (150.19,143.20);

\path[draw=drawColor,line width= 0.6pt,line join=round] (150.19,151.48) -- (150.19,143.20);

\path[draw=drawColor,line width= 0.6pt,line join=round] (150.19,151.48) -- (150.19,143.20);

\path[draw=drawColor,line width= 0.6pt,line join=round] (150.19,151.48) -- (150.19,143.20);

\path[draw=drawColor,line width= 0.6pt,line join=round] (150.19,151.48) -- (150.19,143.20);

\path[draw=drawColor,line width= 0.6pt,line join=round] (150.19,151.48) -- (150.19,143.20);

\path[draw=drawColor,line width= 0.6pt,line join=round] (150.19,151.48) -- (150.19,143.20);

\path[draw=drawColor,line width= 0.6pt,line join=round] (150.19,151.48) -- (150.19,143.20);

\path[draw=drawColor,line width= 0.6pt,line join=round] (150.19,151.48) -- (150.19,143.20);

\path[draw=drawColor,line width= 0.6pt,line join=round] (150.19,151.48) -- (150.19,143.20);

\path[draw=drawColor,line width= 0.6pt,line join=round] (150.19,151.48) -- (150.19,143.20);

\path[draw=drawColor,line width= 0.6pt,line join=round] (150.19,151.48) -- (150.19,143.20);

\path[draw=drawColor,line width= 0.6pt,line join=round] (150.19,151.48) -- (150.19,143.20);

\path[draw=drawColor,line width= 0.6pt,line join=round] (150.19,151.48) -- (150.19,143.20);

\path[draw=drawColor,line width= 0.6pt,line join=round] (150.19,151.48) -- (150.19,143.20);

\path[draw=drawColor,line width= 0.6pt,line join=round] (150.19,151.48) -- (150.19,143.20);

\path[draw=drawColor,line width= 0.6pt,line join=round] (150.19,151.48) -- (150.19,143.20);

\path[draw=drawColor,line width= 0.6pt,line join=round] (150.19,151.48) -- (150.19,143.20);

\path[draw=drawColor,line width= 0.6pt,line join=round] (150.19,151.48) -- (150.19,143.20);

\path[draw=drawColor,line width= 0.6pt,line join=round] (150.19,151.48) -- (150.19,143.20);

\path[draw=drawColor,line width= 0.6pt,line join=round] (150.19,151.48) -- (150.19,143.20);

\path[draw=drawColor,line width= 0.6pt,line join=round] (150.19,151.48) -- (150.19,143.20);

\path[draw=drawColor,line width= 0.6pt,line join=round] (150.19,151.48) -- (150.19,143.20);

\path[draw=drawColor,line width= 0.6pt,line join=round] (150.19,151.48) -- (150.19,143.20);

\path[draw=drawColor,line width= 0.6pt,line join=round] (150.19,151.48) -- (150.19,143.20);

\path[draw=drawColor,line width= 0.6pt,line join=round] (150.19,151.48) -- (150.19,143.20);

\path[draw=drawColor,line width= 0.6pt,line join=round] (150.19,151.48) -- (150.19,143.20);

\path[draw=drawColor,line width= 0.6pt,line join=round] (150.19,151.48) -- (150.19,143.20);

\path[draw=drawColor,line width= 0.6pt,line join=round] (150.19,151.48) -- (150.19,143.20);

\path[draw=drawColor,line width= 0.6pt,line join=round] (150.19,151.48) -- (150.19,143.20);

\path[draw=drawColor,line width= 0.6pt,line join=round] (150.19,151.48) -- (150.19,143.20);

\path[draw=drawColor,line width= 0.6pt,line join=round] (150.19,151.48) -- (150.19,143.20);

\path[draw=drawColor,line width= 0.6pt,line join=round] (150.19,151.48) -- (150.19,143.20);

\path[draw=drawColor,line width= 0.6pt,line join=round] (150.19,151.48) -- (150.19,143.20);

\path[draw=drawColor,line width= 0.6pt,line join=round] (150.19,151.48) -- (150.19,143.20);

\path[draw=drawColor,line width= 0.6pt,line join=round] (150.19,151.48) -- (150.19,143.20);

\path[draw=drawColor,line width= 0.6pt,line join=round] (150.19,151.48) -- (150.19,143.20);

\path[draw=drawColor,line width= 0.6pt,line join=round] (150.19,151.48) -- (150.19,143.20);

\path[draw=drawColor,line width= 0.6pt,line join=round] (150.19,151.48) -- (150.19,143.20);

\path[draw=drawColor,line width= 0.6pt,line join=round] (150.19,151.48) -- (150.19,143.20);

\path[draw=drawColor,line width= 0.6pt,line join=round] (150.19,151.48) -- (150.19,143.20);

\path[draw=drawColor,line width= 0.6pt,line join=round] (150.19,151.48) -- (150.19,143.20);

\path[draw=drawColor,line width= 0.6pt,line join=round] (150.19,151.48) -- (150.19,143.20);

\path[draw=drawColor,line width= 0.6pt,line join=round] (150.19,151.48) -- (150.19,143.20);

\path[draw=drawColor,line width= 0.6pt,line join=round] (150.19,151.48) -- (150.19,143.20);

\path[draw=drawColor,line width= 0.6pt,line join=round] (150.19,151.48) -- (150.19,143.20);

\path[draw=drawColor,line width= 0.6pt,line join=round] (150.19,151.48) -- (150.19,143.20);

\path[draw=drawColor,line width= 0.6pt,line join=round] (150.19,151.48) -- (150.19,143.20);

\path[draw=drawColor,line width= 0.6pt,line join=round] (150.19,151.48) -- (150.19,143.20);

\path[draw=drawColor,line width= 0.6pt,line join=round] (150.19,151.48) -- (150.19,143.20);

\path[draw=drawColor,line width= 0.6pt,line join=round] (150.19,151.48) -- (150.19,143.20);

\path[draw=drawColor,line width= 0.6pt,line join=round] (150.19,151.48) -- (150.19,143.20);

\path[draw=drawColor,line width= 0.6pt,line join=round] (150.19,151.48) -- (150.19,143.20);

\path[draw=drawColor,line width= 0.6pt,line join=round] (150.19,151.48) -- (150.19,143.20);

\path[draw=drawColor,line width= 0.6pt,line join=round] (150.19,151.48) -- (150.19,143.20);

\path[draw=drawColor,line width= 0.6pt,line join=round] (150.19,151.48) -- (150.19,143.20);

\path[draw=drawColor,line width= 0.6pt,line join=round] (150.19,151.48) -- (150.19,143.20);

\path[draw=drawColor,line width= 0.6pt,line join=round] (150.19,151.48) -- (150.19,143.20);

\path[draw=drawColor,line width= 0.6pt,line join=round] (150.19,151.48) -- (150.19,143.20);

\path[draw=drawColor,line width= 0.6pt,line join=round] (150.19,151.48) -- (150.19,143.20);

\path[draw=drawColor,line width= 0.6pt,line join=round] (150.19,151.48) -- (150.19,143.20);

\path[draw=drawColor,line width= 0.6pt,line join=round] (150.19,151.48) -- (150.19,143.20);

\path[draw=drawColor,line width= 0.6pt,line join=round] (150.19,151.48) -- (150.19,143.20);

\path[draw=drawColor,line width= 0.6pt,line join=round] (150.19,151.48) -- (150.19,143.20);

\path[draw=drawColor,line width= 0.6pt,line join=round] (150.19,151.48) -- (150.19,143.20);

\path[draw=drawColor,line width= 0.6pt,line join=round] (150.19,151.48) -- (150.19,143.20);

\path[draw=drawColor,line width= 0.6pt,line join=round] (150.19,151.48) -- (150.19,143.20);

\path[draw=drawColor,line width= 0.6pt,line join=round] (150.19,151.48) -- (150.19,143.20);

\path[draw=drawColor,line width= 0.6pt,line join=round] (150.19,151.48) -- (150.19,143.20);

\path[draw=drawColor,line width= 0.6pt,line join=round] (150.19,151.48) -- (150.19,143.20);

\path[draw=drawColor,line width= 0.6pt,line join=round] (150.19,151.48) -- (150.19,143.20);

\path[draw=drawColor,line width= 0.6pt,line join=round] (150.19,151.48) -- (150.19,143.20);

\path[draw=drawColor,line width= 0.6pt,line join=round] (150.19,151.48) -- (150.19,143.20);

\path[draw=drawColor,line width= 0.6pt,line join=round] (150.19,151.48) -- (150.19,143.20);

\path[draw=drawColor,line width= 0.6pt,line join=round] (150.19,151.48) -- (150.19,143.20);

\path[draw=drawColor,line width= 0.6pt,line join=round] (150.19,151.48) -- (150.19,143.20);

\path[draw=drawColor,line width= 0.6pt,line join=round] (150.19,151.48) -- (150.19,143.20);

\path[draw=drawColor,line width= 0.6pt,line join=round] (150.19,151.48) -- (150.19,143.20);

\path[draw=drawColor,line width= 0.6pt,line join=round] (150.19,151.48) -- (150.19,143.20);

\path[draw=drawColor,line width= 0.6pt,line join=round] (150.19,151.48) -- (150.19,143.20);

\path[draw=drawColor,line width= 0.6pt,line join=round] (150.19,151.48) -- (150.19,143.20);

\path[draw=drawColor,line width= 0.6pt,line join=round] (150.19,151.48) -- (150.19,143.20);

\path[draw=drawColor,line width= 0.6pt,line join=round] (150.19,151.48) -- (150.19,143.20);

\path[draw=drawColor,line width= 0.6pt,line join=round] (150.19,151.48) -- (150.19,143.20);

\path[draw=drawColor,line width= 0.6pt,line join=round] (150.19,151.48) -- (150.19,143.20);

\path[draw=drawColor,line width= 0.6pt,line join=round] (150.19,151.48) -- (150.19,143.20);

\path[draw=drawColor,line width= 0.6pt,line join=round] (150.19,151.48) -- (150.19,143.20);

\path[draw=drawColor,line width= 0.6pt,line join=round] (150.19,151.48) -- (150.19,143.20);

\path[draw=drawColor,line width= 0.6pt,line join=round] (150.19,151.48) -- (150.19,143.20);

\path[draw=drawColor,line width= 0.6pt,line join=round] (150.19,151.48) -- (150.19,143.20);

\path[draw=drawColor,line width= 0.6pt,line join=round] (150.19,151.48) -- (150.19,143.20);

\path[draw=drawColor,line width= 0.6pt,line join=round] (150.19,151.48) -- (150.19,143.20);

\path[draw=drawColor,line width= 0.6pt,line join=round] (150.19,151.48) -- (150.19,143.20);

\path[draw=drawColor,line width= 0.6pt,line join=round] (150.19,151.48) -- (150.19,143.20);

\path[draw=drawColor,line width= 0.6pt,line join=round] (150.19,151.48) -- (150.19,143.20);

\path[draw=drawColor,line width= 0.6pt,line join=round] (150.19,151.48) -- (150.19,143.20);

\path[draw=drawColor,line width= 0.6pt,line join=round] (150.19,151.48) -- (150.19,143.20);

\path[draw=drawColor,line width= 0.6pt,line join=round] (150.19,151.48) -- (150.19,143.20);

\path[draw=drawColor,line width= 0.6pt,line join=round] (150.19,151.48) -- (150.19,143.20);

\path[draw=drawColor,line width= 0.6pt,line join=round] (150.19,151.48) -- (150.19,143.20);

\path[draw=drawColor,line width= 0.6pt,line join=round] (150.19,151.48) -- (150.19,143.20);

\path[draw=drawColor,line width= 0.6pt,line join=round] (150.19,151.48) -- (150.19,143.20);

\path[draw=drawColor,line width= 0.6pt,line join=round] (150.19,151.48) -- (150.19,143.20);

\path[draw=drawColor,line width= 0.6pt,line join=round] (150.19,151.48) -- (150.19,143.20);

\path[draw=drawColor,line width= 0.6pt,line join=round] (150.19,151.48) -- (150.19,143.20);

\path[draw=drawColor,line width= 0.6pt,line join=round] (150.19,151.48) -- (150.19,143.20);

\path[draw=drawColor,line width= 0.6pt,line join=round] (150.19,151.48) -- (150.19,143.20);

\path[draw=drawColor,line width= 0.6pt,line join=round] (150.19,151.48) -- (150.19,143.20);

\path[draw=drawColor,line width= 0.6pt,line join=round] (150.19,151.48) -- (150.19,143.20);

\path[draw=drawColor,line width= 0.6pt,line join=round] (150.19,151.48) -- (150.19,143.20);

\path[draw=drawColor,line width= 0.6pt,line join=round] (150.19,151.48) -- (150.19,143.20);

\path[draw=drawColor,line width= 0.6pt,line join=round] (150.19,151.48) -- (150.19,143.20);

\path[draw=drawColor,line width= 0.6pt,line join=round] (150.19,151.48) -- (150.19,143.20);

\path[draw=drawColor,line width= 0.6pt,line join=round] (150.19,151.48) -- (150.19,143.20);

\path[draw=drawColor,line width= 0.6pt,line join=round] (150.19,151.48) -- (150.19,143.20);

\path[draw=drawColor,line width= 0.6pt,line join=round] (150.19,151.48) -- (150.19,143.20);

\path[draw=drawColor,line width= 0.6pt,line join=round] (150.19,151.48) -- (150.19,143.20);

\path[draw=drawColor,line width= 0.6pt,line join=round] (150.19,151.48) -- (150.19,143.20);

\path[draw=drawColor,line width= 0.6pt,line join=round] (150.19,151.48) -- (150.19,143.20);

\path[draw=drawColor,line width= 0.6pt,line join=round] (150.19,151.48) -- (150.19,143.20);

\path[draw=drawColor,line width= 0.6pt,line join=round] (150.19,151.48) -- (150.19,143.20);

\path[draw=drawColor,line width= 0.6pt,line join=round] (150.19,151.48) -- (150.19,143.20);

\path[draw=drawColor,line width= 0.6pt,line join=round] (150.19,151.48) -- (150.19,143.20);

\path[draw=drawColor,line width= 0.6pt,line join=round] (150.19,151.48) -- (150.19,143.20);

\path[draw=drawColor,line width= 0.6pt,line join=round] (150.19,151.48) -- (150.19,143.20);

\path[draw=drawColor,line width= 0.6pt,line join=round] (150.19,151.48) -- (150.19,143.20);

\path[draw=drawColor,line width= 0.6pt,line join=round] (150.19,151.48) -- (150.19,143.20);

\path[draw=drawColor,line width= 0.6pt,line join=round] (150.19,151.48) -- (150.19,143.20);

\path[draw=drawColor,line width= 0.6pt,line join=round] (150.19,151.48) -- (150.19,143.20);

\path[draw=drawColor,line width= 0.6pt,line join=round] (150.19,151.48) -- (150.19,143.20);

\path[draw=drawColor,line width= 0.6pt,line join=round] (150.19,151.48) -- (150.19,143.20);

\path[draw=drawColor,line width= 0.6pt,line join=round] (150.19,151.48) -- (150.19,143.20);

\path[draw=drawColor,line width= 0.6pt,line join=round] (150.19,151.48) -- (150.19,143.20);

\path[draw=drawColor,line width= 0.6pt,line join=round] (150.19,151.48) -- (150.19,143.20);

\path[draw=drawColor,line width= 0.6pt,line join=round] (150.19,151.48) -- (150.19,143.20);

\path[draw=drawColor,line width= 0.6pt,line join=round] (150.19,151.48) -- (150.19,143.20);

\path[draw=drawColor,line width= 0.6pt,line join=round] (150.19,151.48) -- (150.19,143.20);

\path[draw=drawColor,line width= 0.6pt,line join=round] (150.19,151.48) -- (150.19,143.20);

\path[draw=drawColor,line width= 0.6pt,line join=round] (150.19,151.48) -- (150.19,143.20);

\path[draw=drawColor,line width= 0.6pt,line join=round] (150.19,151.48) -- (150.19,143.20);

\path[draw=drawColor,line width= 0.6pt,line join=round] (150.19,151.48) -- (150.19,143.20);

\path[draw=drawColor,line width= 0.6pt,line join=round] (150.19,151.48) -- (150.19,143.20);

\path[draw=drawColor,line width= 0.6pt,line join=round] (150.19,151.48) -- (150.19,143.20);

\path[draw=drawColor,line width= 0.6pt,line join=round] (150.19,151.48) -- (150.19,143.20);

\path[draw=drawColor,line width= 0.6pt,line join=round] (150.19,151.48) -- (150.19,143.20);

\path[draw=drawColor,line width= 0.6pt,line join=round] (150.19,151.48) -- (150.19,143.20);

\path[draw=drawColor,line width= 0.6pt,line join=round] (150.19,151.48) -- (150.19,143.20);

\path[draw=drawColor,line width= 0.6pt,line join=round] (150.19,151.48) -- (150.19,143.20);

\path[draw=drawColor,line width= 0.6pt,line join=round] (150.19,151.48) -- (150.19,143.20);

\path[draw=drawColor,line width= 0.6pt,line join=round] (150.19,151.48) -- (150.19,143.20);

\path[draw=drawColor,line width= 0.6pt,line join=round] (150.19,151.48) -- (150.19,143.20);

\path[draw=drawColor,line width= 0.6pt,line join=round] (150.19,151.48) -- (150.19,143.20);

\path[draw=drawColor,line width= 0.6pt,line join=round] (150.19,151.48) -- (150.19,143.20);

\path[draw=drawColor,line width= 0.6pt,line join=round] (150.19,151.48) -- (150.19,143.20);

\path[draw=drawColor,line width= 0.6pt,line join=round] (150.19,151.48) -- (150.19,143.20);

\path[draw=drawColor,line width= 0.6pt,line join=round] (150.19,151.48) -- (150.19,143.20);

\path[draw=drawColor,line width= 0.6pt,line join=round] (150.19,151.48) -- (150.19,143.20);

\path[draw=drawColor,line width= 0.6pt,line join=round] (150.19,151.48) -- (150.19,143.20);

\path[draw=drawColor,line width= 0.6pt,line join=round] (150.19,151.48) -- (150.19,143.20);

\path[draw=drawColor,line width= 0.6pt,line join=round] (150.19,151.48) -- (150.19,143.20);

\path[draw=drawColor,line width= 0.6pt,line join=round] (150.19,151.48) -- (150.19,143.20);

\path[draw=drawColor,line width= 0.6pt,line join=round] (150.19,151.48) -- (150.19,143.20);

\path[draw=drawColor,line width= 0.6pt,line join=round] (150.19,151.48) -- (150.19,143.20);

\path[draw=drawColor,line width= 0.6pt,line join=round] (150.19,151.48) -- (150.19,143.20);

\path[draw=drawColor,line width= 0.6pt,line join=round] (150.19,151.48) -- (150.19,143.20);

\path[draw=drawColor,line width= 0.6pt,line join=round] (150.19,151.48) -- (150.19,143.20);

\path[draw=drawColor,line width= 0.6pt,line join=round] (150.19,151.48) -- (150.19,143.20);

\path[draw=drawColor,line width= 0.6pt,line join=round] (150.19,151.48) -- (150.19,143.20);

\path[draw=drawColor,line width= 0.6pt,line join=round] (150.19,151.48) -- (150.19,143.20);

\path[draw=drawColor,line width= 0.6pt,line join=round] (150.19,151.48) -- (150.19,143.20);

\path[draw=drawColor,line width= 0.6pt,line join=round] (150.19,151.48) -- (150.19,143.20);

\path[draw=drawColor,line width= 0.6pt,line join=round] (150.19,151.48) -- (150.19,143.20);

\path[draw=drawColor,line width= 0.6pt,line join=round] (150.19,151.48) -- (150.19,143.20);

\path[draw=drawColor,line width= 0.6pt,line join=round] (150.19,151.48) -- (150.19,143.20);

\path[draw=drawColor,line width= 0.6pt,line join=round] (150.19,151.48) -- (150.19,143.20);

\path[draw=drawColor,line width= 0.6pt,line join=round] (150.19,151.48) -- (150.19,143.20);

\path[draw=drawColor,line width= 0.6pt,line join=round] (150.19,151.48) -- (150.19,143.20);

\path[draw=drawColor,line width= 0.6pt,line join=round] (150.19,151.48) -- (150.19,143.20);

\path[draw=drawColor,line width= 0.6pt,line join=round] (150.19,151.48) -- (150.19,143.20);

\path[draw=drawColor,line width= 0.6pt,line join=round] (150.19,151.48) -- (150.19,143.20);

\path[draw=drawColor,line width= 0.6pt,line join=round] (150.19,151.48) -- (150.19,143.20);

\path[draw=drawColor,line width= 0.6pt,line join=round] (150.19,151.48) -- (150.19,143.20);

\path[draw=drawColor,line width= 0.6pt,line join=round] (150.19,151.48) -- (150.19,143.20);

\path[draw=drawColor,line width= 0.6pt,line join=round] (150.19,151.48) -- (150.19,143.20);

\path[draw=drawColor,line width= 0.6pt,line join=round] (150.19,151.48) -- (150.19,143.20);

\path[draw=drawColor,line width= 0.6pt,line join=round] (150.19,151.48) -- (150.19,143.20);

\path[draw=drawColor,line width= 0.6pt,line join=round] (150.19,151.48) -- (150.19,143.20);

\path[draw=drawColor,line width= 0.6pt,line join=round] (150.19,151.48) -- (150.19,143.20);

\path[draw=drawColor,line width= 0.6pt,line join=round] (150.19,151.48) -- (150.19,143.20);

\path[draw=drawColor,line width= 0.6pt,line join=round] (150.19,151.48) -- (150.19,143.20);

\path[draw=drawColor,line width= 0.6pt,line join=round] (150.19,151.48) -- (150.19,143.20);

\path[draw=drawColor,line width= 0.6pt,line join=round] (150.19,151.48) -- (150.19,143.20);

\path[draw=drawColor,line width= 0.6pt,line join=round] (150.19,151.48) -- (150.19,143.20);

\path[draw=drawColor,line width= 0.6pt,line join=round] (150.19,151.48) -- (150.19,143.20);

\path[draw=drawColor,line width= 0.6pt,line join=round] (150.19,151.48) -- (150.19,143.20);

\path[draw=drawColor,line width= 0.6pt,line join=round] (150.19,151.48) -- (150.19,143.20);

\path[draw=drawColor,line width= 0.6pt,line join=round] (150.19,151.48) -- (150.19,143.20);

\path[draw=drawColor,line width= 0.6pt,line join=round] (150.19,151.48) -- (150.19,143.20);

\path[draw=drawColor,line width= 0.6pt,line join=round] (150.19,151.48) -- (150.19,143.20);

\path[draw=drawColor,line width= 0.6pt,line join=round] (150.19,151.48) -- (150.19,143.20);

\path[draw=drawColor,line width= 0.6pt,line join=round] (150.19,151.48) -- (150.19,143.20);

\path[draw=drawColor,line width= 0.6pt,line join=round] (150.19,151.48) -- (150.19,143.20);

\path[draw=drawColor,line width= 0.6pt,line join=round] (150.19,151.48) -- (150.19,143.20);

\path[draw=drawColor,line width= 0.6pt,line join=round] (150.19,151.48) -- (150.19,143.20);

\path[draw=drawColor,line width= 0.6pt,line join=round] (150.19,151.48) -- (150.19,143.20);

\path[draw=drawColor,line width= 0.6pt,line join=round] (150.19,151.48) -- (150.19,143.20);

\path[draw=drawColor,line width= 0.6pt,line join=round] (150.19,151.48) -- (150.19,143.20);

\path[draw=drawColor,line width= 0.6pt,line join=round] (150.19,151.48) -- (150.19,143.20);

\path[draw=drawColor,line width= 0.6pt,line join=round] (150.19,151.48) -- (150.19,143.20);

\path[draw=drawColor,line width= 0.6pt,line join=round] (150.19,151.48) -- (150.19,143.20);

\path[draw=drawColor,line width= 0.6pt,line join=round] (150.19,151.48) -- (150.19,143.20);

\path[draw=drawColor,line width= 0.6pt,line join=round] (150.19,151.48) -- (150.19,143.20);

\path[draw=drawColor,line width= 0.6pt,line join=round] (150.19,151.48) -- (150.19,143.20);

\path[draw=drawColor,line width= 0.6pt,line join=round] (150.19,151.48) -- (150.19,143.20);

\path[draw=drawColor,line width= 0.6pt,line join=round] (150.19,151.48) -- (150.19,143.20);

\path[draw=drawColor,line width= 0.6pt,line join=round] (150.19,151.48) -- (150.19,143.20);

\path[draw=drawColor,line width= 0.6pt,line join=round] (150.19,151.48) -- (150.19,143.20);

\path[draw=drawColor,line width= 0.6pt,line join=round] (150.19,151.48) -- (150.19,143.20);

\path[draw=drawColor,line width= 0.6pt,line join=round] (150.19,151.48) -- (150.19,143.20);

\path[draw=drawColor,line width= 0.6pt,line join=round] (150.19,151.48) -- (150.19,143.20);

\path[draw=drawColor,line width= 0.6pt,line join=round] (150.19,151.48) -- (150.19,143.20);

\path[draw=drawColor,line width= 0.6pt,line join=round] (150.19,151.48) -- (150.19,143.20);

\path[draw=drawColor,line width= 0.6pt,line join=round] (150.19,151.48) -- (150.19,143.20);

\path[draw=drawColor,line width= 0.6pt,line join=round] (150.19,151.48) -- (150.19,143.20);

\path[draw=drawColor,line width= 0.6pt,line join=round] (150.19,151.48) -- (150.19,143.20);

\path[draw=drawColor,line width= 0.6pt,line join=round] (150.19,151.48) -- (150.19,143.20);

\path[draw=drawColor,line width= 0.6pt,line join=round] (150.19,151.48) -- (150.19,143.20);

\path[draw=drawColor,line width= 0.6pt,line join=round] (150.19,151.48) -- (150.19,143.20);

\path[draw=drawColor,line width= 0.6pt,line join=round] (150.19,151.48) -- (150.19,143.20);

\path[draw=drawColor,line width= 0.6pt,line join=round] (150.19,151.48) -- (150.19,143.20);

\path[draw=drawColor,line width= 0.6pt,line join=round] (150.19,151.48) -- (150.19,143.20);

\path[draw=drawColor,line width= 0.6pt,line join=round] (150.19,151.48) -- (150.19,143.20);

\path[draw=drawColor,line width= 0.6pt,line join=round] (150.19,151.48) -- (150.19,143.20);

\path[draw=drawColor,line width= 0.6pt,line join=round] (150.19,151.48) -- (150.19,143.20);

\path[draw=drawColor,line width= 0.6pt,line join=round] (150.19,151.48) -- (150.19,143.20);

\path[draw=drawColor,line width= 0.6pt,line join=round] (150.19,151.48) -- (150.19,143.20);

\path[draw=drawColor,line width= 0.6pt,line join=round] (150.19,151.48) -- (150.19,143.20);

\path[draw=drawColor,line width= 0.6pt,line join=round] (150.19,151.48) -- (150.19,143.20);

\path[draw=drawColor,line width= 0.6pt,line join=round] (150.19,151.48) -- (150.19,143.20);

\path[draw=drawColor,line width= 0.6pt,line join=round] (150.19,151.48) -- (150.19,143.20);

\path[draw=drawColor,line width= 0.6pt,line join=round] (150.19,151.48) -- (150.19,143.20);

\path[draw=drawColor,line width= 0.6pt,line join=round] (150.19,151.48) -- (150.19,143.20);

\path[draw=drawColor,line width= 0.6pt,line join=round] (150.19,151.48) -- (150.19,143.20);

\path[draw=drawColor,line width= 0.6pt,line join=round] (150.19,151.48) -- (150.19,143.20);

\path[draw=drawColor,line width= 0.6pt,line join=round] (150.19,151.48) -- (150.19,143.20);

\path[draw=drawColor,line width= 0.6pt,line join=round] (150.19,151.48) -- (150.19,143.20);

\path[draw=drawColor,line width= 0.6pt,line join=round] (150.19,151.48) -- (150.19,143.20);

\path[draw=drawColor,line width= 0.6pt,line join=round] (150.19,151.48) -- (150.19,143.20);

\path[draw=drawColor,line width= 0.6pt,line join=round] (150.19,151.48) -- (150.19,143.20);

\path[draw=drawColor,line width= 0.6pt,line join=round] (150.19,151.48) -- (150.19,143.20);

\path[draw=drawColor,line width= 0.6pt,line join=round] (150.19,151.48) -- (150.19,143.20);

\path[draw=drawColor,line width= 0.6pt,line join=round] (150.19,151.48) -- (150.19,143.20);

\path[draw=drawColor,line width= 0.6pt,line join=round] (150.19,151.48) -- (150.19,143.20);

\path[draw=drawColor,line width= 0.6pt,line join=round] (150.19,151.48) -- (150.19,143.20);

\path[draw=drawColor,line width= 0.6pt,line join=round] (150.19,151.48) -- (150.19,143.20);

\path[draw=drawColor,line width= 0.6pt,line join=round] (150.19,151.48) -- (150.19,143.20);

\path[draw=drawColor,line width= 0.6pt,line join=round] (150.19,151.48) -- (150.19,143.20);

\path[draw=drawColor,line width= 0.6pt,line join=round] (150.19,151.48) -- (150.19,143.20);

\path[draw=drawColor,line width= 0.6pt,line join=round] (150.19,151.48) -- (150.19,143.20);

\path[draw=drawColor,line width= 0.6pt,line join=round] (150.19,151.48) -- (150.19,143.20);

\path[draw=drawColor,line width= 0.6pt,line join=round] (150.19,151.48) -- (150.19,143.20);

\path[draw=drawColor,line width= 0.6pt,line join=round] (150.19,151.48) -- (150.19,143.20);

\path[draw=drawColor,line width= 0.6pt,line join=round] (150.19,151.48) -- (150.19,143.20);

\path[draw=drawColor,line width= 0.6pt,line join=round] (150.19,151.48) -- (150.19,143.20);

\path[draw=drawColor,line width= 0.6pt,line join=round] (150.19,151.48) -- (150.19,143.20);

\path[draw=drawColor,line width= 0.6pt,line join=round] (150.19,151.48) -- (150.19,143.20);

\path[draw=drawColor,line width= 0.6pt,line join=round] (150.19,151.48) -- (150.19,143.20);

\path[draw=drawColor,line width= 0.6pt,line join=round] (150.19,151.48) -- (150.19,143.20);

\path[draw=drawColor,line width= 0.6pt,line join=round] (150.19,151.48) -- (150.19,143.20);

\path[draw=drawColor,line width= 0.6pt,line join=round] (150.19,151.48) -- (150.19,143.20);

\path[draw=drawColor,line width= 0.6pt,line join=round] (150.19,151.48) -- (150.19,143.20);

\path[draw=drawColor,line width= 0.6pt,line join=round] (150.19,151.48) -- (150.19,143.20);

\path[draw=drawColor,line width= 0.6pt,line join=round] (150.19,151.48) -- (150.19,143.20);

\path[draw=drawColor,line width= 0.6pt,line join=round] (150.19,151.48) -- (150.19,143.20);

\path[draw=drawColor,line width= 0.6pt,line join=round] (150.19,151.48) -- (150.19,143.20);

\path[draw=drawColor,line width= 0.6pt,line join=round] (150.19,151.48) -- (150.19,143.20);

\path[draw=drawColor,line width= 0.6pt,line join=round] (150.19,151.48) -- (150.19,143.20);

\path[draw=drawColor,line width= 0.6pt,line join=round] (150.19,151.48) -- (150.19,143.20);

\path[draw=drawColor,line width= 0.6pt,line join=round] (150.19,151.48) -- (150.19,143.20);

\path[draw=drawColor,line width= 0.6pt,line join=round] (150.19,151.48) -- (150.19,143.20);

\path[draw=drawColor,line width= 0.6pt,line join=round] (150.19,151.48) -- (150.19,143.20);

\path[draw=drawColor,line width= 0.6pt,line join=round] (150.19,151.48) -- (150.19,143.20);

\path[draw=drawColor,line width= 0.6pt,line join=round] (150.19,151.48) -- (150.19,143.20);

\path[draw=drawColor,line width= 0.6pt,line join=round] (150.19,151.48) -- (150.19,143.20);

\path[draw=drawColor,line width= 0.6pt,line join=round] (150.19,151.48) -- (150.19,143.20);

\path[draw=drawColor,line width= 0.6pt,line join=round] (150.19,151.48) -- (150.19,143.20);

\path[draw=drawColor,line width= 0.6pt,line join=round] (150.19,151.48) -- (150.19,143.20);

\path[draw=drawColor,line width= 0.6pt,line join=round] (150.19,151.48) -- (150.19,143.20);

\path[draw=drawColor,line width= 0.6pt,line join=round] (150.19,151.48) -- (150.19,143.20);

\path[draw=drawColor,line width= 0.6pt,line join=round] (150.19,151.48) -- (150.19,143.20);

\path[draw=drawColor,line width= 0.6pt,line join=round] (150.19,151.48) -- (150.19,143.20);

\path[draw=drawColor,line width= 0.6pt,line join=round] (150.19,151.48) -- (150.19,143.20);

\path[draw=drawColor,line width= 0.6pt,line join=round] (150.19,151.48) -- (150.19,143.20);

\path[draw=drawColor,line width= 0.6pt,line join=round] (150.19,151.48) -- (150.19,143.20);

\path[draw=drawColor,line width= 0.6pt,line join=round] (150.19,151.48) -- (150.19,143.20);

\path[draw=drawColor,line width= 0.6pt,line join=round] (150.19,151.48) -- (150.19,143.20);

\path[draw=drawColor,line width= 0.6pt,line join=round] (150.19,151.48) -- (150.19,143.20);

\path[draw=drawColor,line width= 0.6pt,line join=round] (150.19,151.48) -- (150.19,143.20);

\path[draw=drawColor,line width= 0.6pt,line join=round] (150.19,151.48) -- (150.19,143.20);

\path[draw=drawColor,line width= 0.6pt,line join=round] (150.19,151.48) -- (150.19,143.20);

\path[draw=drawColor,line width= 0.6pt,line join=round] (150.19,151.48) -- (150.19,143.20);

\path[draw=drawColor,line width= 0.6pt,line join=round] (150.19,151.48) -- (150.19,143.20);

\path[draw=drawColor,line width= 0.6pt,line join=round] (150.19,151.48) -- (150.19,143.20);

\path[draw=drawColor,line width= 0.6pt,line join=round] (150.19,151.48) -- (150.19,143.20);

\path[draw=drawColor,line width= 0.6pt,line join=round] (150.19,151.48) -- (150.19,143.20);

\path[draw=drawColor,line width= 0.6pt,line join=round] (150.19,151.48) -- (150.19,143.20);

\path[draw=drawColor,line width= 0.6pt,line join=round] (150.19,151.48) -- (150.19,143.20);

\path[draw=drawColor,line width= 0.6pt,line join=round] (150.19,151.48) -- (150.19,143.20);

\path[draw=drawColor,line width= 0.6pt,line join=round] (150.19,151.48) -- (150.19,143.20);

\path[draw=drawColor,line width= 0.6pt,line join=round] (150.19,151.48) -- (150.19,143.20);

\path[draw=drawColor,line width= 0.6pt,line join=round] (150.19,151.48) -- (150.19,143.20);

\path[draw=drawColor,line width= 0.6pt,line join=round] (150.19,151.48) -- (150.19,143.20);

\path[draw=drawColor,line width= 0.6pt,line join=round] (150.19,151.48) -- (150.19,143.20);

\path[draw=drawColor,line width= 0.6pt,line join=round] (150.19,151.48) -- (150.19,143.20);

\path[draw=drawColor,line width= 0.6pt,line join=round] (150.19,151.48) -- (150.19,143.20);

\path[draw=drawColor,line width= 0.6pt,line join=round] (150.19,151.48) -- (150.19,143.20);

\path[draw=drawColor,line width= 0.6pt,line join=round] (150.19,151.48) -- (150.19,143.20);

\path[draw=drawColor,line width= 0.6pt,line join=round] (150.19,151.48) -- (150.19,143.20);

\path[draw=drawColor,line width= 0.6pt,line join=round] (150.19,151.48) -- (150.19,143.20);

\path[draw=drawColor,line width= 0.6pt,line join=round] (150.19,151.48) -- (150.19,143.20);

\path[draw=drawColor,line width= 0.6pt,line join=round] (150.19,151.48) -- (150.19,143.20);

\path[draw=drawColor,line width= 0.6pt,line join=round] (150.19,151.48) -- (150.19,143.20);

\path[draw=drawColor,line width= 0.6pt,line join=round] (150.19,151.48) -- (150.19,143.20);

\path[draw=drawColor,line width= 0.6pt,line join=round] (150.19,151.48) -- (150.19,143.20);

\path[draw=drawColor,line width= 0.6pt,line join=round] (150.19,151.48) -- (150.19,143.20);

\path[draw=drawColor,line width= 0.6pt,line join=round] (150.19,151.48) -- (150.19,143.20);

\path[draw=drawColor,line width= 0.6pt,line join=round] (150.19,151.48) -- (150.19,143.20);

\path[draw=drawColor,line width= 0.6pt,line join=round] (150.19,151.48) -- (150.19,143.20);

\path[draw=drawColor,line width= 0.6pt,line join=round] (150.19,151.48) -- (150.19,143.20);

\path[draw=drawColor,line width= 0.6pt,line join=round] (150.19,151.48) -- (150.19,143.20);

\path[draw=drawColor,line width= 0.6pt,line join=round] (150.19,151.48) -- (150.19,143.20);

\path[draw=drawColor,line width= 0.6pt,line join=round] (150.19,151.48) -- (150.19,143.20);

\path[draw=drawColor,line width= 0.6pt,line join=round] (150.19,151.48) -- (150.19,143.20);

\path[draw=drawColor,line width= 0.6pt,line join=round] (150.19,151.48) -- (150.19,143.20);

\path[draw=drawColor,line width= 0.6pt,line join=round] (150.19,151.48) -- (150.19,143.20);

\path[draw=drawColor,line width= 0.6pt,line join=round] (150.19,151.48) -- (150.19,143.20);

\path[draw=drawColor,line width= 0.6pt,line join=round] (150.19,151.48) -- (150.19,143.20);

\path[draw=drawColor,line width= 0.6pt,line join=round] (150.19,151.48) -- (150.19,143.20);

\path[draw=drawColor,line width= 0.6pt,line join=round] (150.19,151.48) -- (150.19,143.20);

\path[draw=drawColor,line width= 0.6pt,line join=round] (150.19,151.48) -- (150.19,143.20);

\path[draw=drawColor,line width= 0.6pt,line join=round] (150.19,151.48) -- (150.19,143.20);

\path[draw=drawColor,line width= 0.6pt,line join=round] (150.19,151.48) -- (150.19,143.20);

\path[draw=drawColor,line width= 0.6pt,line join=round] (150.19,151.48) -- (150.19,143.20);

\path[draw=drawColor,line width= 0.6pt,line join=round] (150.19,151.48) -- (150.19,143.20);

\path[draw=drawColor,line width= 0.6pt,line join=round] (150.19,151.48) -- (150.19,143.20);

\path[draw=drawColor,line width= 0.6pt,line join=round] (150.19,151.48) -- (150.19,143.20);

\path[draw=drawColor,line width= 0.6pt,line join=round] (150.19,151.48) -- (150.19,143.20);

\path[draw=drawColor,line width= 0.6pt,line join=round] (150.19,151.48) -- (150.19,143.20);

\path[draw=drawColor,line width= 0.6pt,line join=round] (150.19,151.48) -- (150.19,143.20);

\path[draw=drawColor,line width= 0.6pt,line join=round] (150.19,151.48) -- (150.19,143.20);

\path[draw=drawColor,line width= 0.6pt,line join=round] (150.19,151.48) -- (150.19,143.20);

\path[draw=drawColor,line width= 0.6pt,line join=round] (150.19,151.48) -- (150.19,143.20);

\path[draw=drawColor,line width= 0.6pt,line join=round] (150.19,151.48) -- (150.19,143.20);

\path[draw=drawColor,line width= 0.6pt,line join=round] (150.19,151.48) -- (150.19,143.20);

\path[draw=drawColor,line width= 0.6pt,line join=round] (150.19,151.48) -- (150.19,143.20);

\path[draw=drawColor,line width= 0.6pt,line join=round] (150.19,151.48) -- (150.19,143.20);

\path[draw=drawColor,line width= 0.6pt,line join=round] (150.19,151.48) -- (150.19,143.20);

\path[draw=drawColor,line width= 0.6pt,line join=round] (150.19,151.48) -- (150.19,143.20);

\path[draw=drawColor,line width= 0.6pt,line join=round] (150.19,151.48) -- (150.19,143.20);

\path[draw=drawColor,line width= 0.6pt,line join=round] (150.19,151.48) -- (150.19,143.20);

\path[draw=drawColor,line width= 0.6pt,line join=round] (150.19,151.48) -- (150.19,143.20);

\path[draw=drawColor,line width= 0.6pt,line join=round] (150.19,151.48) -- (150.19,143.20);

\path[draw=drawColor,line width= 0.6pt,line join=round] (150.19,151.48) -- (150.19,143.20);

\path[draw=drawColor,line width= 0.6pt,line join=round] (150.19,151.48) -- (150.19,143.20);

\path[draw=drawColor,line width= 0.6pt,line join=round] (150.19,151.48) -- (150.19,143.20);

\path[draw=drawColor,line width= 0.6pt,line join=round] (150.19,151.48) -- (150.19,143.20);

\path[draw=drawColor,line width= 0.6pt,line join=round] (150.19,151.48) -- (150.19,143.20);

\path[draw=drawColor,line width= 0.6pt,line join=round] (150.19,151.48) -- (150.19,143.20);

\path[draw=drawColor,line width= 0.6pt,line join=round] (150.19,151.48) -- (150.19,143.20);

\path[draw=drawColor,line width= 0.6pt,line join=round] (150.19,151.48) -- (150.19,143.20);

\path[draw=drawColor,line width= 0.6pt,line join=round] (150.19,151.48) -- (150.19,143.20);

\path[draw=drawColor,line width= 0.6pt,line join=round] (150.19,151.48) -- (150.19,143.20);

\path[draw=drawColor,line width= 0.6pt,line join=round] (150.19,151.48) -- (150.19,143.20);

\path[draw=drawColor,line width= 0.6pt,line join=round] (150.19,151.48) -- (150.19,143.20);

\path[draw=drawColor,line width= 0.6pt,line join=round] (150.19,151.48) -- (150.19,143.20);

\path[draw=drawColor,line width= 0.6pt,line join=round] (150.19,151.48) -- (150.19,143.20);

\path[draw=drawColor,line width= 0.6pt,line join=round] (150.19,151.48) -- (150.19,143.20);

\path[draw=drawColor,line width= 0.6pt,line join=round] (150.19,151.48) -- (150.19,143.20);

\path[draw=drawColor,line width= 0.6pt,line join=round] (150.19,151.48) -- (150.19,143.20);

\path[draw=drawColor,line width= 0.6pt,line join=round] (150.19,151.48) -- (150.19,143.20);

\path[draw=drawColor,line width= 0.6pt,line join=round] (150.19,151.48) -- (150.19,143.20);

\path[draw=drawColor,line width= 0.6pt,line join=round] (150.19,151.48) -- (150.19,143.20);

\path[draw=drawColor,line width= 0.6pt,line join=round] (150.19,151.48) -- (150.19,143.20);

\path[draw=drawColor,line width= 0.6pt,line join=round] (150.19,151.48) -- (150.19,143.20);

\path[draw=drawColor,line width= 0.6pt,line join=round] (150.19,151.48) -- (150.19,143.20);

\path[draw=drawColor,line width= 0.6pt,line join=round] (150.19,151.48) -- (150.19,143.20);

\path[draw=drawColor,line width= 0.6pt,line join=round] (150.19,151.48) -- (150.19,143.20);

\path[draw=drawColor,line width= 0.6pt,line join=round] (150.19,151.48) -- (150.19,143.20);

\path[draw=drawColor,line width= 0.6pt,line join=round] (150.19,151.48) -- (150.19,143.20);

\path[draw=drawColor,line width= 0.6pt,line join=round] (150.19,151.48) -- (150.19,143.20);

\path[draw=drawColor,line width= 0.6pt,line join=round] (150.19,151.48) -- (150.19,143.20);

\path[draw=drawColor,line width= 0.6pt,line join=round] (150.19,151.48) -- (150.19,143.20);

\path[draw=drawColor,line width= 0.6pt,line join=round] (150.19,151.48) -- (150.19,143.20);

\path[draw=drawColor,line width= 0.6pt,line join=round] (150.19,151.48) -- (150.19,143.20);

\path[draw=drawColor,line width= 0.6pt,line join=round] (150.19,151.48) -- (150.19,143.20);

\path[draw=drawColor,line width= 0.6pt,line join=round] (150.19,151.48) -- (150.19,143.20);

\path[draw=drawColor,line width= 0.6pt,line join=round] (150.19,151.48) -- (150.19,143.20);

\path[draw=drawColor,line width= 0.6pt,line join=round] (150.19,151.48) -- (150.19,143.20);

\path[draw=drawColor,line width= 0.6pt,line join=round] (150.19,151.48) -- (150.19,143.20);

\path[draw=drawColor,line width= 0.6pt,line join=round] (150.19,151.48) -- (150.19,143.20);

\path[draw=drawColor,line width= 0.6pt,line join=round] (150.19,151.48) -- (150.19,143.20);

\path[draw=drawColor,line width= 0.6pt,line join=round] (150.19,151.48) -- (150.19,143.20);

\path[draw=drawColor,line width= 0.6pt,line join=round] (150.19,151.48) -- (150.19,143.20);

\path[draw=drawColor,line width= 0.6pt,line join=round] (150.19,151.48) -- (150.19,143.20);

\path[draw=drawColor,line width= 0.6pt,line join=round] (150.19,151.48) -- (150.19,143.20);

\path[draw=drawColor,line width= 0.6pt,line join=round] (150.19,151.48) -- (150.19,143.20);

\path[draw=drawColor,line width= 0.6pt,line join=round] (150.19,151.48) -- (150.19,143.20);

\path[draw=drawColor,line width= 0.6pt,line join=round] (150.19,151.48) -- (150.19,143.20);

\path[draw=drawColor,line width= 0.6pt,line join=round] (150.19,151.48) -- (150.19,143.20);

\path[draw=drawColor,line width= 0.6pt,line join=round] (150.19,151.48) -- (150.19,143.20);

\path[draw=drawColor,line width= 0.6pt,line join=round] (150.19,151.48) -- (150.19,143.20);

\path[draw=drawColor,line width= 0.6pt,line join=round] (150.19,151.48) -- (150.19,143.20);

\path[draw=drawColor,line width= 0.6pt,line join=round] (150.19,151.48) -- (150.19,143.20);

\path[draw=drawColor,line width= 0.6pt,line join=round] (150.19,151.48) -- (150.19,143.20);

\path[draw=drawColor,line width= 0.6pt,line join=round] (150.19,151.48) -- (150.19,143.20);

\path[draw=drawColor,line width= 0.6pt,line join=round] (150.19,151.48) -- (150.19,143.20);

\path[draw=drawColor,line width= 0.6pt,line join=round] (150.19,151.48) -- (150.19,143.20);

\path[draw=drawColor,line width= 0.6pt,line join=round] (150.19,151.48) -- (150.19,143.20);

\path[draw=drawColor,line width= 0.6pt,line join=round] (150.19,151.48) -- (150.19,143.20);

\path[draw=drawColor,line width= 0.6pt,line join=round] (150.19,151.48) -- (150.19,143.20);

\path[draw=drawColor,line width= 0.6pt,line join=round] (150.19,151.48) -- (150.19,143.20);

\path[draw=drawColor,line width= 0.6pt,line join=round] (150.19,151.48) -- (150.19,143.20);

\path[draw=drawColor,line width= 0.6pt,line join=round] (150.19,151.48) -- (150.19,143.20);

\path[draw=drawColor,line width= 0.6pt,line join=round] (150.19,151.48) -- (150.19,143.20);

\path[draw=drawColor,line width= 0.6pt,line join=round] (150.19,151.48) -- (150.19,143.20);

\path[draw=drawColor,line width= 0.6pt,line join=round] (150.19,151.48) -- (150.19,143.20);

\path[draw=drawColor,line width= 0.6pt,line join=round] (150.19,151.48) -- (150.19,143.20);

\path[draw=drawColor,line width= 0.6pt,line join=round] (150.19,151.48) -- (150.19,143.20);

\path[draw=drawColor,line width= 0.6pt,line join=round] (150.19,151.48) -- (150.19,143.20);

\path[draw=drawColor,line width= 0.6pt,line join=round] (150.19,151.48) -- (150.19,143.20);

\path[draw=drawColor,line width= 0.6pt,line join=round] (150.19,151.48) -- (150.19,143.20);

\path[draw=drawColor,line width= 0.6pt,line join=round] (150.19,151.48) -- (150.19,143.20);

\path[draw=drawColor,line width= 0.6pt,line join=round] (150.19,151.48) -- (150.19,143.20);

\path[draw=drawColor,line width= 0.6pt,line join=round] (150.19,151.48) -- (150.19,143.20);

\path[draw=drawColor,line width= 0.6pt,line join=round] (150.19,151.48) -- (150.19,143.20);

\path[draw=drawColor,line width= 0.6pt,line join=round] (150.19,151.48) -- (150.19,143.20);

\path[draw=drawColor,line width= 0.6pt,line join=round] (150.19,151.48) -- (150.19,143.20);

\path[draw=drawColor,line width= 0.6pt,line join=round] (150.19,151.48) -- (150.19,143.20);

\path[draw=drawColor,line width= 0.6pt,line join=round] (150.19,151.48) -- (150.19,143.20);

\path[draw=drawColor,line width= 0.6pt,line join=round] (150.19,151.48) -- (150.19,143.20);

\path[draw=drawColor,line width= 0.6pt,line join=round] (150.19,151.48) -- (150.19,143.20);

\path[draw=drawColor,line width= 0.6pt,line join=round] (150.19,151.48) -- (150.19,143.20);

\path[draw=drawColor,line width= 0.6pt,line join=round] (150.19,151.48) -- (150.19,143.20);

\path[draw=drawColor,line width= 0.6pt,line join=round] (150.19,151.48) -- (150.19,143.20);

\path[draw=drawColor,line width= 0.6pt,line join=round] (150.19,151.48) -- (150.19,143.20);

\path[draw=drawColor,line width= 0.6pt,line join=round] (150.19,151.48) -- (150.19,143.20);

\path[draw=drawColor,line width= 0.6pt,line join=round] (150.19,151.48) -- (150.19,143.20);

\path[draw=drawColor,line width= 0.6pt,line join=round] (150.19,151.48) -- (150.19,143.20);

\path[draw=drawColor,line width= 0.6pt,line join=round] (150.19,151.48) -- (150.19,143.20);

\path[draw=drawColor,line width= 0.6pt,line join=round] (150.19,151.48) -- (150.19,143.20);

\path[draw=drawColor,line width= 0.6pt,line join=round] (150.19,151.48) -- (150.19,143.20);

\path[draw=drawColor,line width= 0.6pt,line join=round] (150.19,151.48) -- (150.19,143.20);

\path[draw=drawColor,line width= 0.6pt,line join=round] (150.19,151.48) -- (150.19,143.20);

\path[draw=drawColor,line width= 0.6pt,line join=round] (150.19,151.48) -- (150.19,143.20);

\path[draw=drawColor,line width= 0.6pt,line join=round] (150.19,151.48) -- (150.19,143.20);

\path[draw=drawColor,line width= 0.6pt,line join=round] (150.19,151.48) -- (150.19,143.20);

\path[draw=drawColor,line width= 0.6pt,line join=round] (150.19,151.48) -- (150.19,143.20);

\path[draw=drawColor,line width= 0.6pt,line join=round] (150.19,151.48) -- (150.19,143.20);

\path[draw=drawColor,line width= 0.6pt,line join=round] (150.19,151.48) -- (150.19,143.20);

\path[draw=drawColor,line width= 0.6pt,line join=round] (150.19,151.48) -- (150.19,143.20);

\path[draw=drawColor,line width= 0.6pt,line join=round] (150.19,151.48) -- (150.19,143.20);

\path[draw=drawColor,line width= 0.6pt,line join=round] (150.19,151.48) -- (150.19,143.20);

\path[draw=drawColor,line width= 0.6pt,line join=round] (150.19,151.48) -- (150.19,143.20);

\path[draw=drawColor,line width= 0.6pt,line join=round] (150.19,151.48) -- (150.19,143.20);

\path[draw=drawColor,line width= 0.6pt,line join=round] (150.19,151.48) -- (150.19,143.20);

\path[draw=drawColor,line width= 0.6pt,line join=round] (150.19,151.48) -- (150.19,143.20);

\path[draw=drawColor,line width= 0.6pt,line join=round] (150.19,151.48) -- (150.19,143.20);

\path[draw=drawColor,line width= 0.6pt,line join=round] (150.19,151.48) -- (150.19,143.20);

\path[draw=drawColor,line width= 0.6pt,line join=round] (150.19,151.48) -- (150.19,143.20);

\path[draw=drawColor,line width= 0.6pt,line join=round] (150.19,151.48) -- (150.19,143.20);

\path[draw=drawColor,line width= 0.6pt,line join=round] (150.19,151.48) -- (150.19,143.20);

\path[draw=drawColor,line width= 0.6pt,line join=round] (150.19,151.48) -- (150.19,143.20);

\path[draw=drawColor,line width= 0.6pt,line join=round] (150.19,151.48) -- (150.19,143.20);

\path[draw=drawColor,line width= 0.6pt,line join=round] (150.19,151.48) -- (150.19,143.20);

\path[draw=drawColor,line width= 0.6pt,line join=round] (150.19,151.48) -- (150.19,143.20);

\path[draw=drawColor,line width= 0.6pt,line join=round] (150.19,151.48) -- (150.19,143.20);

\path[draw=drawColor,line width= 0.6pt,line join=round] (150.19,151.48) -- (150.19,143.20);

\path[draw=drawColor,line width= 0.6pt,line join=round] (150.19,151.48) -- (150.19,143.20);

\path[draw=drawColor,line width= 0.6pt,line join=round] (150.19,151.48) -- (150.19,143.20);

\path[draw=drawColor,line width= 0.6pt,line join=round] (150.19,151.48) -- (150.19,143.20);

\path[draw=drawColor,line width= 0.6pt,line join=round] (150.19,151.48) -- (150.19,143.20);

\path[draw=drawColor,line width= 0.6pt,line join=round] (150.19,151.48) -- (150.19,143.20);

\path[draw=drawColor,line width= 0.6pt,line join=round] (150.19,151.48) -- (150.19,143.20);

\path[draw=drawColor,line width= 0.6pt,line join=round] (150.19,151.48) -- (150.19,143.20);

\path[draw=drawColor,line width= 0.6pt,line join=round] (150.19,151.48) -- (150.19,143.20);

\path[draw=drawColor,line width= 0.6pt,line join=round] (150.19,151.48) -- (150.19,143.20);

\path[draw=drawColor,line width= 0.6pt,line join=round] (150.19,151.48) -- (150.19,143.20);

\path[draw=drawColor,line width= 0.6pt,line join=round] (150.19,151.48) -- (150.19,143.20);

\path[draw=drawColor,line width= 0.6pt,line join=round] (150.19,151.48) -- (150.19,143.20);

\path[draw=drawColor,line width= 0.6pt,line join=round] (150.19,151.48) -- (150.19,143.20);

\path[draw=drawColor,line width= 0.6pt,line join=round] (150.19,151.48) -- (150.19,143.20);

\path[draw=drawColor,line width= 0.6pt,line join=round] (150.19,151.48) -- (150.19,143.20);

\path[draw=drawColor,line width= 0.6pt,line join=round] (150.19,151.48) -- (150.19,143.20);

\path[draw=drawColor,line width= 0.6pt,line join=round] (150.19,151.48) -- (150.19,143.20);

\path[draw=drawColor,line width= 0.6pt,line join=round] (150.19,151.48) -- (150.19,143.20);

\path[draw=drawColor,line width= 0.6pt,line join=round] (150.19,151.48) -- (150.19,143.20);

\path[draw=drawColor,line width= 0.6pt,line join=round] (150.19,151.48) -- (150.19,143.20);

\path[draw=drawColor,line width= 0.6pt,line join=round] (150.19,151.48) -- (150.19,143.20);

\path[draw=drawColor,line width= 0.6pt,line join=round] (150.19,151.48) -- (150.19,143.20);

\path[draw=drawColor,line width= 0.6pt,line join=round] (150.19,151.48) -- (150.19,143.20);

\path[draw=drawColor,line width= 0.6pt,line join=round] (150.19,151.48) -- (150.19,143.20);

\path[draw=drawColor,line width= 0.6pt,line join=round] (150.19,151.48) -- (150.19,143.20);

\path[draw=drawColor,line width= 0.6pt,line join=round] (150.19,151.48) -- (150.19,143.20);

\path[draw=drawColor,line width= 0.6pt,line join=round] (150.19,151.48) -- (150.19,143.20);

\path[draw=drawColor,line width= 0.6pt,line join=round] (150.19,151.48) -- (150.19,143.20);

\path[draw=drawColor,line width= 0.6pt,line join=round] (150.19,151.48) -- (150.19,143.20);

\path[draw=drawColor,line width= 0.6pt,line join=round] (150.19,151.48) -- (150.19,143.20);

\path[draw=drawColor,line width= 0.6pt,line join=round] (150.19,151.48) -- (150.19,143.20);

\path[draw=drawColor,line width= 0.6pt,line join=round] (150.19,151.48) -- (150.19,143.20);

\path[draw=drawColor,line width= 0.6pt,line join=round] (150.19,151.48) -- (150.19,143.20);

\path[draw=drawColor,line width= 0.6pt,line join=round] (150.19,151.48) -- (150.19,143.20);

\path[draw=drawColor,line width= 0.6pt,line join=round] (150.19,151.48) -- (150.19,143.20);

\path[draw=drawColor,line width= 0.6pt,line join=round] (150.19,151.48) -- (150.19,143.20);

\path[draw=drawColor,line width= 0.6pt,line join=round] (150.19,151.48) -- (150.19,143.20);

\path[draw=drawColor,line width= 0.6pt,line join=round] (150.19,151.48) -- (150.19,143.20);

\path[draw=drawColor,line width= 0.6pt,line join=round] (150.19,151.48) -- (150.19,143.20);

\path[draw=drawColor,line width= 0.6pt,line join=round] (150.19,151.48) -- (150.19,143.20);

\path[draw=drawColor,line width= 0.6pt,line join=round] (150.19,151.48) -- (150.19,143.20);

\path[draw=drawColor,line width= 0.6pt,line join=round] (150.19,151.48) -- (150.19,143.20);

\path[draw=drawColor,line width= 0.6pt,line join=round] (150.19,151.48) -- (150.19,143.20);

\path[draw=drawColor,line width= 0.6pt,line join=round] (150.19,151.48) -- (150.19,143.20);

\path[draw=drawColor,line width= 0.6pt,line join=round] (150.19,151.48) -- (150.19,143.20);

\path[draw=drawColor,line width= 0.6pt,line join=round] (150.19,151.48) -- (150.19,143.20);

\path[draw=drawColor,line width= 0.6pt,line join=round] (150.19,151.48) -- (150.19,143.20);

\path[draw=drawColor,line width= 0.6pt,line join=round] (150.19,151.48) -- (150.19,143.20);

\path[draw=drawColor,line width= 0.6pt,line join=round] (150.19,151.48) -- (150.19,143.20);

\path[draw=drawColor,line width= 0.6pt,line join=round] (150.19,151.48) -- (150.19,143.20);

\path[draw=drawColor,line width= 0.6pt,line join=round] (150.19,151.48) -- (150.19,143.20);

\path[draw=drawColor,line width= 0.6pt,line join=round] (150.19,151.48) -- (150.19,143.20);

\path[draw=drawColor,line width= 0.6pt,line join=round] (150.19,151.48) -- (150.19,143.20);

\path[draw=drawColor,line width= 0.6pt,line join=round] (150.19,151.48) -- (150.19,143.20);

\path[draw=drawColor,line width= 0.6pt,line join=round] (150.19,151.48) -- (150.19,143.20);

\path[draw=drawColor,line width= 0.6pt,line join=round] (150.19,151.48) -- (150.19,143.20);

\path[draw=drawColor,line width= 0.6pt,line join=round] (150.19,151.48) -- (150.19,143.20);

\path[draw=drawColor,line width= 0.6pt,line join=round] (150.19,151.48) -- (150.19,143.20);

\path[draw=drawColor,line width= 0.6pt,line join=round] (150.19,151.48) -- (150.19,143.20);

\path[draw=drawColor,line width= 0.6pt,line join=round] (150.19,151.48) -- (150.19,143.20);

\path[draw=drawColor,line width= 0.6pt,line join=round] (150.19,151.48) -- (150.19,143.20);

\path[draw=drawColor,line width= 0.6pt,line join=round] (150.19,151.48) -- (150.19,143.20);

\path[draw=drawColor,line width= 0.6pt,line join=round] (150.19,151.48) -- (150.19,143.20);

\path[draw=drawColor,line width= 0.6pt,line join=round] (150.19,151.48) -- (150.19,143.20);

\path[draw=drawColor,line width= 0.6pt,line join=round] (150.19,151.48) -- (150.19,143.20);

\path[draw=drawColor,line width= 0.6pt,line join=round] (150.19,151.48) -- (150.19,143.20);

\path[draw=drawColor,line width= 0.6pt,line join=round] (150.19,151.48) -- (150.19,143.20);

\path[draw=drawColor,line width= 0.6pt,line join=round] (150.19,151.48) -- (150.19,143.20);

\path[draw=drawColor,line width= 0.6pt,line join=round] (150.19,151.48) -- (150.19,143.20);

\path[draw=drawColor,line width= 0.6pt,line join=round] (150.19,151.48) -- (150.19,143.20);

\path[draw=drawColor,line width= 0.6pt,line join=round] (150.19,151.48) -- (150.19,143.20);

\path[draw=drawColor,line width= 0.6pt,line join=round] (150.19,151.48) -- (150.19,143.20);

\path[draw=drawColor,line width= 0.6pt,line join=round] (150.19,151.48) -- (150.19,143.20);

\path[draw=drawColor,line width= 0.6pt,line join=round] (150.19,151.48) -- (150.19,143.20);

\path[draw=drawColor,line width= 0.6pt,line join=round] (150.19,151.48) -- (150.19,143.20);

\path[draw=drawColor,line width= 0.6pt,line join=round] (150.19,151.48) -- (150.19,143.20);

\path[draw=drawColor,line width= 0.6pt,line join=round] (150.19,151.48) -- (150.19,143.20);

\path[draw=drawColor,line width= 0.6pt,line join=round] (150.19,151.48) -- (150.19,143.20);

\path[draw=drawColor,line width= 0.6pt,line join=round] (150.19,151.48) -- (150.19,143.20);

\path[draw=drawColor,line width= 0.6pt,line join=round] (150.19,151.48) -- (150.19,143.20);

\path[draw=drawColor,line width= 0.6pt,line join=round] (150.19,151.48) -- (150.19,143.20);

\path[draw=drawColor,line width= 0.6pt,line join=round] (150.19,151.48) -- (150.19,143.20);

\path[draw=drawColor,line width= 0.6pt,line join=round] (150.19,151.48) -- (150.19,143.20);

\path[draw=drawColor,line width= 0.6pt,line join=round] (150.19,151.48) -- (150.19,143.20);

\path[draw=drawColor,line width= 0.6pt,line join=round] (150.19,151.48) -- (150.19,143.20);

\path[draw=drawColor,line width= 0.6pt,line join=round] (150.19,151.48) -- (150.19,143.20);

\path[draw=drawColor,line width= 0.6pt,line join=round] (150.19,151.48) -- (150.19,143.20);

\path[draw=drawColor,line width= 0.6pt,line join=round] (150.19,151.48) -- (150.19,143.20);

\path[draw=drawColor,line width= 0.6pt,line join=round] (150.19,151.48) -- (150.19,143.20);

\path[draw=drawColor,line width= 0.6pt,line join=round] (150.19,151.48) -- (150.19,143.20);

\path[draw=drawColor,line width= 0.6pt,line join=round] (150.19,151.48) -- (150.19,143.20);

\path[draw=drawColor,line width= 0.6pt,line join=round] (150.19,151.48) -- (150.19,143.20);

\path[draw=drawColor,line width= 0.6pt,line join=round] (150.19,151.48) -- (150.19,143.20);

\path[draw=drawColor,line width= 0.6pt,line join=round] (150.19,151.48) -- (150.19,143.20);

\path[draw=drawColor,line width= 0.6pt,line join=round] (150.19,151.48) -- (150.19,143.20);

\path[draw=drawColor,line width= 0.6pt,line join=round] (150.19,151.48) -- (150.19,143.20);

\path[draw=drawColor,line width= 0.6pt,line join=round] (150.19,151.48) -- (150.19,143.20);

\path[draw=drawColor,line width= 0.6pt,line join=round] (150.19,151.48) -- (150.19,143.20);

\path[draw=drawColor,line width= 0.6pt,line join=round] (150.19,151.48) -- (150.19,143.20);

\path[draw=drawColor,line width= 0.6pt,line join=round] (150.19,151.48) -- (150.19,143.20);

\path[draw=drawColor,line width= 0.6pt,line join=round] (150.19,151.48) -- (150.19,143.20);

\path[draw=drawColor,line width= 0.6pt,line join=round] (150.19,151.48) -- (150.19,143.20);

\path[draw=drawColor,line width= 0.6pt,line join=round] (150.19,151.48) -- (150.19,143.20);

\path[draw=drawColor,line width= 0.6pt,line join=round] (150.19,151.48) -- (150.19,143.20);

\path[draw=drawColor,line width= 0.6pt,line join=round] (150.19,151.48) -- (150.19,143.20);

\path[draw=drawColor,line width= 0.6pt,line join=round] (150.19,151.48) -- (150.19,143.20);

\path[draw=drawColor,line width= 0.6pt,line join=round] (150.19,151.48) -- (150.19,143.20);

\path[draw=drawColor,line width= 0.6pt,line join=round] (150.19,151.48) -- (150.19,143.20);

\path[draw=drawColor,line width= 0.6pt,line join=round] (150.19,151.48) -- (150.19,143.20);

\path[draw=drawColor,line width= 0.6pt,line join=round] (150.19,151.48) -- (150.19,143.20);

\path[draw=drawColor,line width= 0.6pt,line join=round] (150.19,151.48) -- (150.19,143.20);

\path[draw=drawColor,line width= 0.6pt,line join=round] (150.19,151.48) -- (150.19,143.20);

\path[draw=drawColor,line width= 0.6pt,line join=round] (150.19,151.48) -- (150.19,143.20);

\path[draw=drawColor,line width= 0.6pt,line join=round] (150.19,151.48) -- (150.19,143.20);

\path[draw=drawColor,line width= 0.6pt,line join=round] (150.19,151.48) -- (150.19,143.20);

\path[draw=drawColor,line width= 0.6pt,line join=round] (150.19,151.48) -- (150.19,143.20);

\path[draw=drawColor,line width= 0.6pt,line join=round] (150.19,151.48) -- (150.19,143.20);

\path[draw=drawColor,line width= 0.6pt,line join=round] (150.19,151.48) -- (150.19,143.20);

\path[draw=drawColor,line width= 0.6pt,line join=round] (150.19,151.48) -- (150.19,143.20);

\path[draw=drawColor,line width= 0.6pt,line join=round] (150.19,151.48) -- (150.19,143.20);

\path[draw=drawColor,line width= 0.6pt,line join=round] (150.19,151.48) -- (150.19,143.20);

\path[draw=drawColor,line width= 0.6pt,line join=round] (150.19,151.48) -- (150.19,143.20);

\path[draw=drawColor,line width= 0.6pt,line join=round] (150.19,151.48) -- (150.19,143.20);

\path[draw=drawColor,line width= 0.6pt,line join=round] (150.19,151.48) -- (150.19,143.20);

\path[draw=drawColor,line width= 0.6pt,line join=round] (150.19,151.48) -- (150.19,143.20);

\path[draw=drawColor,line width= 0.6pt,line join=round] (150.19,151.48) -- (150.19,143.20);

\path[draw=drawColor,line width= 0.6pt,line join=round] (150.19,151.48) -- (150.19,143.20);

\path[draw=drawColor,line width= 0.6pt,line join=round] (150.19,151.48) -- (150.19,143.20);

\path[draw=drawColor,line width= 0.6pt,line join=round] (150.19,151.48) -- (150.19,143.20);

\path[draw=drawColor,line width= 0.6pt,line join=round] (150.19,151.48) -- (150.19,143.20);

\path[draw=drawColor,line width= 0.6pt,line join=round] (150.19,151.48) -- (150.19,143.20);

\path[draw=drawColor,line width= 0.6pt,line join=round] (150.19,151.48) -- (150.19,143.20);

\path[draw=drawColor,line width= 0.6pt,line join=round] (150.19,151.48) -- (150.19,143.20);

\path[draw=drawColor,line width= 0.6pt,line join=round] (150.19,151.48) -- (150.19,143.20);

\path[draw=drawColor,line width= 0.6pt,line join=round] (150.19,151.48) -- (150.19,143.20);

\path[draw=drawColor,line width= 0.6pt,line join=round] (150.19,151.48) -- (150.19,143.20);

\path[draw=drawColor,line width= 0.6pt,line join=round] (150.19,151.48) -- (150.19,143.20);

\path[draw=drawColor,line width= 0.6pt,line join=round] (150.19,151.48) -- (150.19,143.20);

\path[draw=drawColor,line width= 0.6pt,line join=round] (150.19,151.48) -- (150.19,143.20);

\path[draw=drawColor,line width= 0.6pt,line join=round] (150.19,151.48) -- (150.19,143.20);

\path[draw=drawColor,line width= 0.6pt,line join=round] (150.19,151.48) -- (150.19,143.20);

\path[draw=drawColor,line width= 0.6pt,line join=round] (150.19,151.48) -- (150.19,143.20);

\path[draw=drawColor,line width= 0.6pt,line join=round] (150.19,151.48) -- (150.19,143.20);

\path[draw=drawColor,line width= 0.6pt,line join=round] (150.19,151.48) -- (150.19,143.20);

\path[draw=drawColor,line width= 0.6pt,line join=round] (150.19,151.48) -- (150.19,143.20);

\path[draw=drawColor,line width= 0.6pt,line join=round] (150.19,151.48) -- (150.19,143.20);

\path[draw=drawColor,line width= 0.6pt,line join=round] (150.19,151.48) -- (150.19,143.20);

\path[draw=drawColor,line width= 0.6pt,line join=round] (150.19,151.48) -- (150.19,143.20);

\path[draw=drawColor,line width= 0.6pt,line join=round] (150.19,151.48) -- (150.19,143.20);

\path[draw=drawColor,line width= 0.6pt,line join=round] (150.19,151.48) -- (150.19,143.20);

\path[draw=drawColor,line width= 0.6pt,line join=round] (150.19,151.48) -- (150.19,143.20);

\path[draw=drawColor,line width= 0.6pt,line join=round] (150.19,151.48) -- (150.19,143.20);

\path[draw=drawColor,line width= 0.6pt,line join=round] (150.19,151.48) -- (150.19,143.20);

\path[draw=drawColor,line width= 0.6pt,line join=round] (150.19,151.48) -- (150.19,143.20);

\path[draw=drawColor,line width= 0.6pt,line join=round] (150.19,151.48) -- (150.19,143.20);

\path[draw=drawColor,line width= 0.6pt,line join=round] (150.19,151.48) -- (150.19,143.20);

\path[draw=drawColor,line width= 0.6pt,line join=round] (150.19,151.48) -- (150.19,143.20);

\path[draw=drawColor,line width= 0.6pt,line join=round] (150.19,151.48) -- (150.19,143.20);

\path[draw=drawColor,line width= 0.6pt,line join=round] (150.19,151.48) -- (150.19,143.20);

\path[draw=drawColor,line width= 0.6pt,line join=round] (150.19,151.48) -- (150.19,143.20);

\path[draw=drawColor,line width= 0.6pt,line join=round] (150.19,151.48) -- (150.19,143.20);

\path[draw=drawColor,line width= 0.6pt,line join=round] (150.19,151.48) -- (150.19,143.20);

\path[draw=drawColor,line width= 0.6pt,line join=round] (150.19,151.48) -- (150.19,143.20);

\path[draw=drawColor,line width= 0.6pt,line join=round] (150.19,151.48) -- (150.19,143.20);

\path[draw=drawColor,line width= 0.6pt,line join=round] (150.19,151.48) -- (150.19,143.20);

\path[draw=drawColor,line width= 0.6pt,line join=round] (150.19,151.48) -- (150.19,143.20);

\path[draw=drawColor,line width= 0.6pt,line join=round] (150.19,151.48) -- (150.19,143.20);

\path[draw=drawColor,line width= 0.6pt,line join=round] (150.19,151.48) -- (150.19,143.20);

\path[draw=drawColor,line width= 0.6pt,line join=round] (150.19,151.48) -- (150.19,143.20);

\path[draw=drawColor,line width= 0.6pt,line join=round] (150.19,151.48) -- (150.19,143.20);

\path[draw=drawColor,line width= 0.6pt,line join=round] (150.19,151.48) -- (150.19,143.20);

\path[draw=drawColor,line width= 0.6pt,line join=round] (150.19,151.48) -- (150.19,143.20);

\path[draw=drawColor,line width= 0.6pt,line join=round] (150.19,151.48) -- (150.19,143.20);

\path[draw=drawColor,line width= 0.6pt,line join=round] (150.19,151.48) -- (150.19,143.20);

\path[draw=drawColor,line width= 0.6pt,line join=round] (150.19,151.48) -- (150.19,143.20);

\path[draw=drawColor,line width= 0.6pt,line join=round] (150.19,151.48) -- (150.19,143.20);

\path[draw=drawColor,line width= 0.6pt,line join=round] (150.19,151.48) -- (150.19,143.20);

\path[draw=drawColor,line width= 0.6pt,line join=round] (150.19,151.48) -- (150.19,143.20);

\path[draw=drawColor,line width= 0.6pt,line join=round] (150.19,151.48) -- (150.19,143.20);

\path[draw=drawColor,line width= 0.6pt,line join=round] (150.19,151.48) -- (150.19,143.20);

\path[draw=drawColor,line width= 0.6pt,line join=round] (150.19,151.48) -- (150.19,143.20);

\path[draw=drawColor,line width= 0.6pt,line join=round] (150.19,151.48) -- (150.19,143.20);

\path[draw=drawColor,line width= 0.6pt,line join=round] (150.19,151.48) -- (150.19,143.20);

\path[draw=drawColor,line width= 0.6pt,line join=round] (150.19,151.48) -- (150.19,143.20);

\path[draw=drawColor,line width= 0.6pt,line join=round] (150.19,151.48) -- (150.19,143.20);

\path[draw=drawColor,line width= 0.6pt,line join=round] (150.19,151.48) -- (150.19,143.20);

\path[draw=drawColor,line width= 0.6pt,line join=round] (150.19,151.48) -- (150.19,143.20);

\path[draw=drawColor,line width= 0.6pt,line join=round] (150.19,151.48) -- (150.19,143.20);

\path[draw=drawColor,line width= 0.6pt,line join=round] (150.19,151.48) -- (150.19,143.20);

\path[draw=drawColor,line width= 0.6pt,line join=round] (150.19,151.48) -- (150.19,143.20);

\path[draw=drawColor,line width= 0.6pt,line join=round] (150.19,151.48) -- (150.19,143.20);

\path[draw=drawColor,line width= 0.6pt,line join=round] (150.19,151.48) -- (150.19,143.20);

\path[draw=drawColor,line width= 0.6pt,line join=round] (150.19,151.48) -- (150.19,143.20);

\path[draw=drawColor,line width= 0.6pt,line join=round] (150.19,151.48) -- (150.19,143.20);

\path[draw=drawColor,line width= 0.6pt,line join=round] (150.19,151.48) -- (150.19,143.20);

\path[draw=drawColor,line width= 0.6pt,line join=round] (150.19,151.48) -- (150.19,143.20);

\path[draw=drawColor,line width= 0.6pt,line join=round] (150.19,151.48) -- (150.19,143.20);

\path[draw=drawColor,line width= 0.6pt,line join=round] (150.19,151.48) -- (150.19,143.20);

\path[draw=drawColor,line width= 0.6pt,line join=round] (150.19,151.48) -- (150.19,143.20);

\path[draw=drawColor,line width= 0.6pt,line join=round] (150.19,151.48) -- (150.19,143.20);

\path[draw=drawColor,line width= 0.6pt,line join=round] (150.19,151.48) -- (150.19,143.20);

\path[draw=drawColor,line width= 0.6pt,line join=round] (150.19,151.48) -- (150.19,143.20);

\path[draw=drawColor,line width= 0.6pt,line join=round] (150.19,151.48) -- (150.19,143.20);

\path[draw=drawColor,line width= 0.6pt,line join=round] (150.19,151.48) -- (150.19,143.20);

\path[draw=drawColor,line width= 0.6pt,line join=round] (150.19,151.48) -- (150.19,143.20);

\path[draw=drawColor,line width= 0.6pt,line join=round] (150.19,151.48) -- (150.19,143.20);

\path[draw=drawColor,line width= 0.6pt,line join=round] (150.19,151.48) -- (150.19,143.20);

\path[draw=drawColor,line width= 0.6pt,line join=round] (150.19,151.48) -- (150.19,143.20);

\path[draw=drawColor,line width= 0.6pt,line join=round] (150.19,151.48) -- (150.19,143.20);

\path[draw=drawColor,line width= 0.6pt,line join=round] (150.19,151.48) -- (150.19,143.20);

\path[draw=drawColor,line width= 0.6pt,line join=round] (150.19,151.48) -- (150.19,143.20);

\path[draw=drawColor,line width= 0.6pt,line join=round] (150.19,151.48) -- (150.19,143.20);

\path[draw=drawColor,line width= 0.6pt,line join=round] (150.19,151.48) -- (150.19,143.20);

\path[draw=drawColor,line width= 0.6pt,line join=round] (150.19,151.48) -- (150.19,143.20);

\path[draw=drawColor,line width= 0.6pt,line join=round] (150.19,151.48) -- (150.19,143.20);

\path[draw=drawColor,line width= 0.6pt,line join=round] (150.19,151.48) -- (150.19,143.20);

\path[draw=drawColor,line width= 0.6pt,line join=round] (150.19,151.48) -- (150.19,143.20);

\path[draw=drawColor,line width= 0.6pt,line join=round] (150.19,151.48) -- (150.19,143.20);

\path[draw=drawColor,line width= 0.6pt,line join=round] (150.19,151.48) -- (150.19,143.20);

\path[draw=drawColor,line width= 0.6pt,line join=round] (150.19,151.48) -- (150.19,143.20);

\path[draw=drawColor,line width= 0.6pt,line join=round] (150.19,151.48) -- (150.19,143.20);

\path[draw=drawColor,line width= 0.6pt,line join=round] (150.19,151.48) -- (150.19,143.20);

\path[draw=drawColor,line width= 0.6pt,line join=round] (150.19,151.48) -- (150.19,143.20);

\path[draw=drawColor,line width= 0.6pt,line join=round] (150.19,151.48) -- (150.19,143.20);

\path[draw=drawColor,line width= 0.6pt,line join=round] (150.19,151.48) -- (150.19,143.20);

\path[draw=drawColor,line width= 0.6pt,line join=round] (150.19,151.48) -- (150.19,143.20);

\path[draw=drawColor,line width= 0.6pt,line join=round] (150.19,151.48) -- (150.19,143.20);

\path[draw=drawColor,line width= 0.6pt,line join=round] (150.19,151.48) -- (150.19,143.20);

\path[draw=drawColor,line width= 0.6pt,line join=round] (150.19,151.48) -- (150.19,143.20);

\path[draw=drawColor,line width= 0.6pt,line join=round] (150.19,151.48) -- (150.19,143.20);

\path[draw=drawColor,line width= 0.6pt,line join=round] (150.19,151.48) -- (150.19,143.20);

\path[draw=drawColor,line width= 0.6pt,line join=round] (150.19,151.48) -- (150.19,143.20);

\path[draw=drawColor,line width= 0.6pt,line join=round] (150.19,151.48) -- (150.19,143.20);

\path[draw=drawColor,line width= 0.6pt,line join=round] (150.19,151.48) -- (150.19,143.20);

\path[draw=drawColor,line width= 0.6pt,line join=round] (150.19,151.48) -- (150.19,143.20);

\path[draw=drawColor,line width= 0.6pt,line join=round] (150.19,151.48) -- (150.19,143.20);

\path[draw=drawColor,line width= 0.6pt,line join=round] (150.19,151.48) -- (150.19,143.20);

\path[draw=drawColor,line width= 0.6pt,line join=round] (150.19,151.48) -- (150.19,143.20);

\path[draw=drawColor,line width= 0.6pt,line join=round] (150.19,151.48) -- (150.19,143.20);

\path[draw=drawColor,line width= 0.6pt,line join=round] (150.19,151.48) -- (150.19,143.20);

\path[draw=drawColor,line width= 0.6pt,line join=round] (150.19,151.48) -- (150.19,143.20);

\path[draw=drawColor,line width= 0.6pt,line join=round] (150.19,151.48) -- (150.19,143.20);

\path[draw=drawColor,line width= 0.6pt,line join=round] (150.19,151.48) -- (150.19,143.20);

\path[draw=drawColor,line width= 0.6pt,line join=round] (150.19,151.48) -- (150.19,143.20);

\path[draw=drawColor,line width= 0.6pt,line join=round] (150.19,151.48) -- (150.19,143.20);

\path[draw=drawColor,line width= 0.6pt,line join=round] (150.19,151.48) -- (150.19,143.20);

\path[draw=drawColor,line width= 0.6pt,line join=round] (150.19,151.48) -- (150.19,143.20);

\path[draw=drawColor,line width= 0.6pt,line join=round] (150.19,151.48) -- (150.19,143.20);

\path[draw=drawColor,line width= 0.6pt,line join=round] (150.19,151.48) -- (150.19,143.20);

\path[draw=drawColor,line width= 0.6pt,line join=round] (150.19,151.48) -- (150.19,143.20);

\path[draw=drawColor,line width= 0.6pt,line join=round] (150.19,151.48) -- (150.19,143.20);

\path[draw=drawColor,line width= 0.6pt,line join=round] (150.19,151.48) -- (150.19,143.20);

\path[draw=drawColor,line width= 0.6pt,line join=round] (150.19,151.48) -- (150.19,143.20);

\path[draw=drawColor,line width= 0.6pt,line join=round] (150.19,151.48) -- (150.19,143.20);

\path[draw=drawColor,line width= 0.6pt,line join=round] (150.19,151.48) -- (150.19,143.20);

\path[draw=drawColor,line width= 0.6pt,line join=round] (150.19,151.48) -- (150.19,143.20);

\path[draw=drawColor,line width= 0.6pt,line join=round] (150.19,151.48) -- (150.19,143.20);

\path[draw=drawColor,line width= 0.6pt,line join=round] (150.19,151.48) -- (150.19,143.20);

\path[draw=drawColor,line width= 0.6pt,line join=round] (150.19,151.48) -- (150.19,143.20);

\path[draw=drawColor,line width= 0.6pt,line join=round] (150.19,151.48) -- (150.19,143.20);

\path[draw=drawColor,line width= 0.6pt,line join=round] (150.19,151.48) -- (150.19,143.20);

\path[draw=drawColor,line width= 0.6pt,line join=round] (150.19,151.48) -- (150.19,143.20);

\path[draw=drawColor,line width= 0.6pt,line join=round] (150.19,151.48) -- (150.19,143.20);

\path[draw=drawColor,line width= 0.6pt,line join=round] (150.19,151.48) -- (150.19,143.20);

\path[draw=drawColor,line width= 0.6pt,line join=round] (150.19,151.48) -- (150.19,143.20);

\path[draw=drawColor,line width= 0.6pt,line join=round] (150.19,151.48) -- (150.19,143.20);

\path[draw=drawColor,line width= 0.6pt,line join=round] (150.19,151.48) -- (150.19,143.20);

\path[draw=drawColor,line width= 0.6pt,line join=round] (150.19,151.48) -- (150.19,143.20);

\path[draw=drawColor,line width= 0.6pt,line join=round] (150.19,151.48) -- (150.19,143.20);

\path[draw=drawColor,line width= 0.6pt,line join=round] (150.19,151.48) -- (150.19,143.20);

\path[draw=drawColor,line width= 0.6pt,line join=round] (150.19,151.48) -- (150.19,143.20);

\path[draw=drawColor,line width= 0.6pt,line join=round] (150.19,151.48) -- (150.19,143.20);

\path[draw=drawColor,line width= 0.6pt,line join=round] (150.19,151.48) -- (150.19,143.20);

\path[draw=drawColor,line width= 0.6pt,line join=round] (150.19,151.48) -- (150.19,143.20);

\path[draw=drawColor,line width= 0.6pt,line join=round] (150.19,151.48) -- (150.19,143.20);

\path[draw=drawColor,line width= 0.6pt,line join=round] (150.19,151.48) -- (150.19,143.20);

\path[draw=drawColor,line width= 0.6pt,line join=round] (150.19,151.48) -- (150.19,143.20);

\path[draw=drawColor,line width= 0.6pt,line join=round] (150.19,151.48) -- (150.19,143.20);

\path[draw=drawColor,line width= 0.6pt,line join=round] (150.19,151.48) -- (150.19,143.20);

\path[draw=drawColor,line width= 0.6pt,line join=round] (150.19,151.48) -- (150.19,143.20);

\path[draw=drawColor,line width= 0.6pt,line join=round] (150.19,151.48) -- (150.19,143.20);

\path[draw=drawColor,line width= 0.6pt,line join=round] (150.19,151.48) -- (150.19,143.20);

\path[draw=drawColor,line width= 0.6pt,line join=round] (150.19,151.48) -- (150.19,143.20);

\path[draw=drawColor,line width= 0.6pt,line join=round] (150.19,151.48) -- (150.19,143.20);

\path[draw=drawColor,line width= 0.6pt,line join=round] (150.19,151.48) -- (150.19,143.20);

\path[draw=drawColor,line width= 0.6pt,line join=round] (150.19,151.48) -- (150.19,143.20);

\path[draw=drawColor,line width= 0.6pt,line join=round] (150.19,151.48) -- (150.19,143.20);

\path[draw=drawColor,line width= 0.6pt,line join=round] (150.19,151.48) -- (150.19,143.20);

\path[draw=drawColor,line width= 0.6pt,line join=round] (150.19,151.48) -- (150.19,143.20);

\path[draw=drawColor,line width= 0.6pt,line join=round] (150.19,151.48) -- (150.19,143.20);

\path[draw=drawColor,line width= 0.6pt,line join=round] (150.19,151.48) -- (150.19,143.20);

\path[draw=drawColor,line width= 0.6pt,line join=round] (150.19,151.48) -- (150.19,143.20);

\path[draw=drawColor,line width= 0.6pt,line join=round] (150.19,151.48) -- (150.19,143.20);

\path[draw=drawColor,line width= 0.6pt,line join=round] (150.19,151.48) -- (150.19,143.20);

\path[draw=drawColor,line width= 0.6pt,line join=round] (150.19,151.48) -- (150.19,143.20);

\path[draw=drawColor,line width= 0.6pt,line join=round] (150.19,151.48) -- (150.19,143.20);

\path[draw=drawColor,line width= 0.6pt,line join=round] (150.19,151.48) -- (150.19,143.20);

\path[draw=drawColor,line width= 0.6pt,line join=round] (150.19,151.48) -- (150.19,143.20);

\path[draw=drawColor,line width= 0.6pt,line join=round] (150.19,151.48) -- (150.19,143.20);

\path[draw=drawColor,line width= 0.6pt,line join=round] (150.19,151.48) -- (150.19,143.20);

\path[draw=drawColor,line width= 0.6pt,line join=round] (150.19,151.48) -- (150.19,143.20);

\path[draw=drawColor,line width= 0.6pt,line join=round] (150.19,151.48) -- (150.19,143.20);

\path[draw=drawColor,line width= 0.6pt,line join=round] (150.19,151.48) -- (150.19,143.20);

\path[draw=drawColor,line width= 0.6pt,line join=round] (150.19,151.48) -- (150.19,143.20);

\path[draw=drawColor,line width= 0.6pt,line join=round] (150.19,151.48) -- (150.19,143.20);

\path[draw=drawColor,line width= 0.6pt,line join=round] (150.19,151.48) -- (150.19,143.20);

\path[draw=drawColor,line width= 0.6pt,line join=round] (150.19,151.48) -- (150.19,143.20);

\path[draw=drawColor,line width= 0.6pt,line join=round] (150.19,151.48) -- (150.19,143.20);

\path[draw=drawColor,line width= 0.6pt,line join=round] (150.19,151.48) -- (150.19,143.20);

\path[draw=drawColor,line width= 0.6pt,line join=round] (150.19,151.48) -- (150.19,143.20);

\path[draw=drawColor,line width= 0.6pt,line join=round] (150.19,151.48) -- (150.19,143.20);

\path[draw=drawColor,line width= 0.6pt,line join=round] (150.19,151.48) -- (150.19,143.20);

\path[draw=drawColor,line width= 0.6pt,line join=round] (150.19,151.48) -- (150.19,143.20);

\path[draw=drawColor,line width= 0.6pt,line join=round] (150.19,151.48) -- (150.19,143.20);

\path[draw=drawColor,line width= 0.6pt,line join=round] (150.19,151.48) -- (150.19,143.20);

\path[draw=drawColor,line width= 0.6pt,line join=round] (150.19,151.48) -- (150.19,143.20);

\path[draw=drawColor,line width= 0.6pt,line join=round] (150.19,151.48) -- (150.19,143.20);

\path[draw=drawColor,line width= 0.6pt,line join=round] (150.19,151.48) -- (150.19,143.20);

\path[draw=drawColor,line width= 0.6pt,line join=round] (150.19,151.48) -- (150.19,143.20);

\path[draw=drawColor,line width= 0.6pt,line join=round] (150.19,151.48) -- (150.19,143.20);

\path[draw=drawColor,line width= 0.6pt,line join=round] (150.19,151.48) -- (150.19,143.20);

\path[draw=drawColor,line width= 0.6pt,line join=round] (150.19,151.48) -- (150.19,143.20);

\path[draw=drawColor,line width= 0.6pt,line join=round] (150.19,151.48) -- (150.19,143.20);

\path[draw=drawColor,line width= 0.6pt,line join=round] (150.19,151.48) -- (150.19,143.20);

\path[draw=drawColor,line width= 0.6pt,line join=round] (150.19,151.48) -- (150.19,143.20);

\path[draw=drawColor,line width= 0.6pt,line join=round] (150.19,151.48) -- (150.19,143.20);

\path[draw=drawColor,line width= 0.6pt,line join=round] (150.19,151.48) -- (150.19,143.20);

\path[draw=drawColor,line width= 0.6pt,line join=round] (150.19,151.48) -- (150.19,143.20);

\path[draw=drawColor,line width= 0.6pt,line join=round] (150.19,151.48) -- (150.19,143.20);

\path[draw=drawColor,line width= 0.6pt,line join=round] (150.19,151.48) -- (150.19,143.20);

\path[draw=drawColor,line width= 0.6pt,line join=round] (150.19,151.48) -- (150.19,143.20);

\path[draw=drawColor,line width= 0.6pt,line join=round] (150.19,151.48) -- (150.19,143.20);

\path[draw=drawColor,line width= 0.6pt,line join=round] (150.19,151.48) -- (150.19,143.20);

\path[draw=drawColor,line width= 0.6pt,line join=round] (150.19,151.48) -- (150.19,143.20);

\path[draw=drawColor,line width= 0.6pt,line join=round] (150.19,151.48) -- (150.19,143.20);

\path[draw=drawColor,line width= 0.6pt,line join=round] (150.19,151.48) -- (150.19,143.20);

\path[draw=drawColor,line width= 0.6pt,line join=round] (150.19,151.48) -- (150.19,143.20);

\path[draw=drawColor,line width= 0.6pt,line join=round] (150.19,151.48) -- (150.19,143.20);

\path[draw=drawColor,line width= 0.6pt,line join=round] (150.19,151.48) -- (150.19,143.20);

\path[draw=drawColor,line width= 0.6pt,line join=round] (150.19,151.48) -- (150.19,143.20);

\path[draw=drawColor,line width= 0.6pt,line join=round] (150.19,151.48) -- (150.19,143.20);

\path[draw=drawColor,line width= 0.6pt,line join=round] (150.19,151.48) -- (150.19,143.20);

\path[draw=drawColor,line width= 0.6pt,line join=round] (150.19,151.48) -- (150.19,143.20);

\path[draw=drawColor,line width= 0.6pt,line join=round] (150.19,151.48) -- (150.19,143.20);

\path[draw=drawColor,line width= 0.6pt,line join=round] (150.19,151.48) -- (150.19,143.20);

\path[draw=drawColor,line width= 0.6pt,line join=round] (150.19,151.48) -- (150.19,143.20);

\path[draw=drawColor,line width= 0.6pt,line join=round] (150.19,151.48) -- (150.19,143.20);

\path[draw=drawColor,line width= 0.6pt,line join=round] (150.19,151.48) -- (150.19,143.20);

\path[draw=drawColor,line width= 0.6pt,line join=round] (150.19,151.48) -- (150.19,143.20);

\path[draw=drawColor,line width= 0.6pt,line join=round] (150.19,151.48) -- (150.19,143.20);

\path[draw=drawColor,line width= 0.6pt,line join=round] (150.19,151.48) -- (150.19,143.20);

\path[draw=drawColor,line width= 0.6pt,line join=round] (150.19,151.48) -- (150.19,143.20);

\path[draw=drawColor,line width= 0.6pt,line join=round] (150.19,151.48) -- (150.19,143.20);

\path[draw=drawColor,line width= 0.6pt,line join=round] (150.19,151.48) -- (150.19,143.20);

\path[draw=drawColor,line width= 0.6pt,line join=round] (150.19,151.48) -- (150.19,143.20);

\path[draw=drawColor,line width= 0.6pt,line join=round] (150.19,151.48) -- (150.19,143.20);

\path[draw=drawColor,line width= 0.6pt,line join=round] (150.19,151.48) -- (150.19,143.20);

\path[draw=drawColor,line width= 0.6pt,line join=round] (150.19,151.48) -- (150.19,143.20);

\path[draw=drawColor,line width= 0.6pt,line join=round] (150.19,151.48) -- (150.19,143.20);

\path[draw=drawColor,line width= 0.6pt,line join=round] (150.19,151.48) -- (150.19,143.20);

\path[draw=drawColor,line width= 0.6pt,line join=round] (150.19,151.48) -- (150.19,143.20);

\path[draw=drawColor,line width= 0.6pt,line join=round] (150.19,151.48) -- (150.19,143.20);

\path[draw=drawColor,line width= 0.6pt,line join=round] (150.19,151.48) -- (150.19,143.20);

\path[draw=drawColor,line width= 0.6pt,line join=round] (150.19,151.48) -- (150.19,143.20);

\path[draw=drawColor,line width= 0.6pt,line join=round] (150.19,151.48) -- (150.19,143.20);

\path[draw=drawColor,line width= 0.6pt,line join=round] (150.19,151.48) -- (150.19,143.20);

\path[draw=drawColor,line width= 0.6pt,line join=round] (150.19,151.48) -- (150.19,143.20);

\path[draw=drawColor,line width= 0.6pt,line join=round] (150.19,151.48) -- (150.19,143.20);

\path[draw=drawColor,line width= 0.6pt,line join=round] (150.19,151.48) -- (150.19,143.20);

\path[draw=drawColor,line width= 0.6pt,line join=round] (150.19,151.48) -- (150.19,143.20);

\path[draw=drawColor,line width= 0.6pt,line join=round] (150.19,151.48) -- (150.19,143.20);

\path[draw=drawColor,line width= 0.6pt,line join=round] (150.19,151.48) -- (150.19,143.20);

\path[draw=drawColor,line width= 0.6pt,line join=round] (150.19,151.48) -- (150.19,143.20);

\path[draw=drawColor,line width= 0.6pt,line join=round] (150.19,151.48) -- (150.19,143.20);

\path[draw=drawColor,line width= 0.6pt,line join=round] (150.19,151.48) -- (150.19,143.20);

\path[draw=drawColor,line width= 0.6pt,line join=round] (150.19,151.48) -- (150.19,143.20);

\path[draw=drawColor,line width= 0.6pt,line join=round] (150.19,151.48) -- (150.19,143.20);

\path[draw=drawColor,line width= 0.6pt,line join=round] (150.19,151.48) -- (150.19,143.20);

\path[draw=drawColor,line width= 0.6pt,line join=round] (150.19,151.48) -- (150.19,143.20);

\path[draw=drawColor,line width= 0.6pt,line join=round] (150.19,151.48) -- (150.19,143.20);

\path[draw=drawColor,line width= 0.6pt,line join=round] (150.19,151.48) -- (150.19,143.20);

\path[draw=drawColor,line width= 0.6pt,line join=round] (150.19,151.48) -- (150.19,143.20);

\path[draw=drawColor,line width= 0.6pt,line join=round] (150.19,151.48) -- (150.19,143.20);

\path[draw=drawColor,line width= 0.6pt,line join=round] (150.19,151.48) -- (150.19,143.20);

\path[draw=drawColor,line width= 0.6pt,line join=round] (150.19,151.48) -- (150.19,143.20);

\path[draw=drawColor,line width= 0.6pt,line join=round] (150.19,151.48) -- (150.19,143.20);

\path[draw=drawColor,line width= 0.6pt,line join=round] (150.19,151.48) -- (150.19,143.20);

\path[draw=drawColor,line width= 0.6pt,line join=round] (150.19,151.48) -- (150.19,143.20);

\path[draw=drawColor,line width= 0.6pt,line join=round] (150.19,151.48) -- (150.19,143.20);

\path[draw=drawColor,line width= 0.6pt,line join=round] (150.19,151.48) -- (150.19,143.20);

\path[draw=drawColor,line width= 0.6pt,line join=round] (150.19,151.48) -- (150.19,143.20);

\path[draw=drawColor,line width= 0.6pt,line join=round] (150.19,151.48) -- (150.19,143.20);

\path[draw=drawColor,line width= 0.6pt,line join=round] (150.19,151.48) -- (150.19,143.20);

\path[draw=drawColor,line width= 0.6pt,line join=round] (150.19,151.48) -- (150.19,143.20);

\path[draw=drawColor,line width= 0.6pt,line join=round] (150.19,151.48) -- (150.19,143.20);

\path[draw=drawColor,line width= 0.6pt,line join=round] (150.19,151.48) -- (150.19,143.20);

\path[draw=drawColor,line width= 0.6pt,line join=round] (150.19,151.48) -- (150.19,143.20);

\path[draw=drawColor,line width= 0.6pt,line join=round] (150.19,151.48) -- (150.19,143.20);

\path[draw=drawColor,line width= 0.6pt,line join=round] (150.19,151.48) -- (150.19,143.20);

\path[draw=drawColor,line width= 0.6pt,line join=round] (150.19,151.48) -- (150.19,143.20);

\path[draw=drawColor,line width= 0.6pt,line join=round] (150.19,151.48) -- (150.19,143.20);

\path[draw=drawColor,line width= 0.6pt,line join=round] (150.19,151.48) -- (150.19,143.20);

\path[draw=drawColor,line width= 0.6pt,line join=round] (150.19,151.48) -- (150.19,143.20);

\path[draw=drawColor,line width= 0.6pt,line join=round] (150.19,151.48) -- (150.19,143.20);

\path[draw=drawColor,line width= 0.6pt,line join=round] (150.19,151.48) -- (150.19,143.20);

\path[draw=drawColor,line width= 0.6pt,line join=round] (150.19,151.48) -- (150.19,143.20);

\path[draw=drawColor,line width= 0.6pt,line join=round] (150.19,151.48) -- (150.19,143.20);

\path[draw=drawColor,line width= 0.6pt,line join=round] (150.19,151.48) -- (150.19,143.20);

\path[draw=drawColor,line width= 0.6pt,line join=round] (150.19,151.48) -- (150.19,143.20);

\path[draw=drawColor,line width= 0.6pt,line join=round] (150.19,151.48) -- (150.19,143.20);

\path[draw=drawColor,line width= 0.6pt,line join=round] (150.19,151.48) -- (150.19,143.20);

\path[draw=drawColor,line width= 0.6pt,line join=round] (150.19,151.48) -- (150.19,143.20);

\path[draw=drawColor,line width= 0.6pt,line join=round] (150.19,151.48) -- (150.19,143.20);

\path[draw=drawColor,line width= 0.6pt,line join=round] (150.19,151.48) -- (150.19,143.20);

\path[draw=drawColor,line width= 0.6pt,line join=round] (150.19,151.48) -- (150.19,143.20);

\path[draw=drawColor,line width= 0.6pt,line join=round] (150.19,151.48) -- (150.19,143.20);

\path[draw=drawColor,line width= 0.6pt,line join=round] (150.19,151.48) -- (150.19,143.20);

\path[draw=drawColor,line width= 0.6pt,line join=round] (150.19,151.48) -- (150.19,143.20);

\path[draw=drawColor,line width= 0.6pt,line join=round] (150.19,151.48) -- (150.19,143.20);

\path[draw=drawColor,line width= 0.6pt,line join=round] (150.19,151.48) -- (150.19,143.20);

\path[draw=drawColor,line width= 0.6pt,line join=round] (150.19,151.48) -- (150.19,143.20);

\path[draw=drawColor,line width= 0.6pt,line join=round] (150.19,151.48) -- (150.19,143.20);

\path[draw=drawColor,line width= 0.6pt,line join=round] (150.19,151.48) -- (150.19,143.20);

\path[draw=drawColor,line width= 0.6pt,line join=round] (150.19,151.48) -- (150.19,143.20);

\path[draw=drawColor,line width= 0.6pt,line join=round] (150.19,151.48) -- (150.19,143.20);

\path[draw=drawColor,line width= 0.6pt,line join=round] (150.19,151.48) -- (150.19,143.20);

\path[draw=drawColor,line width= 0.6pt,line join=round] (150.19,151.48) -- (150.19,143.20);

\path[draw=drawColor,line width= 0.6pt,line join=round] (150.19,151.48) -- (150.19,143.20);

\path[draw=drawColor,line width= 0.6pt,line join=round] (150.19,151.48) -- (150.19,143.20);

\path[draw=drawColor,line width= 0.6pt,line join=round] (150.19,151.48) -- (150.19,143.20);

\path[draw=drawColor,line width= 0.6pt,line join=round] (150.19,151.48) -- (150.19,143.20);

\path[draw=drawColor,line width= 0.6pt,line join=round] (150.19,151.48) -- (150.19,143.20);

\path[draw=drawColor,line width= 0.6pt,line join=round] (150.19,151.48) -- (150.19,143.20);

\path[draw=drawColor,line width= 0.6pt,line join=round] (150.19,151.48) -- (150.19,143.20);

\path[draw=drawColor,line width= 0.6pt,line join=round] (150.19,151.48) -- (150.19,143.20);

\path[draw=drawColor,line width= 0.6pt,line join=round] (150.19,151.48) -- (150.19,143.20);

\path[draw=drawColor,line width= 0.6pt,line join=round] (150.19,151.48) -- (150.19,143.20);

\path[draw=drawColor,line width= 0.6pt,line join=round] (150.19,151.48) -- (150.19,143.20);

\path[draw=drawColor,line width= 0.6pt,line join=round] (150.19,151.48) -- (150.19,143.20);

\path[draw=drawColor,line width= 0.6pt,line join=round] (150.19,151.48) -- (150.19,143.20);

\path[draw=drawColor,line width= 0.6pt,line join=round] (150.19,151.48) -- (150.19,143.20);

\path[draw=drawColor,line width= 0.6pt,line join=round] (150.19,151.48) -- (150.19,143.20);

\path[draw=drawColor,line width= 0.6pt,line join=round] (150.19,151.48) -- (150.19,143.20);

\path[draw=drawColor,line width= 0.6pt,line join=round] (150.19,151.48) -- (150.19,143.20);

\path[draw=drawColor,line width= 0.6pt,line join=round] (150.19,151.48) -- (150.19,143.20);

\path[draw=drawColor,line width= 0.6pt,line join=round] (150.19,151.48) -- (150.19,143.20);

\path[draw=drawColor,line width= 0.6pt,line join=round] (150.19,151.48) -- (150.19,143.20);

\path[draw=drawColor,line width= 0.6pt,line join=round] (150.19,151.48) -- (150.19,143.20);

\path[draw=drawColor,line width= 0.6pt,line join=round] (150.19,151.48) -- (150.19,143.20);

\path[draw=drawColor,line width= 0.6pt,line join=round] (150.19,151.48) -- (150.19,143.20);

\path[draw=drawColor,line width= 0.6pt,line join=round] (150.19,151.48) -- (150.19,143.20);

\path[draw=drawColor,line width= 0.6pt,line join=round] (150.19,151.48) -- (150.19,143.20);

\path[draw=drawColor,line width= 0.6pt,line join=round] (150.19,151.48) -- (150.19,143.20);

\path[draw=drawColor,line width= 0.6pt,line join=round] (150.19,151.48) -- (150.19,143.20);

\path[draw=drawColor,line width= 0.6pt,line join=round] (150.19,151.48) -- (150.19,143.20);

\path[draw=drawColor,line width= 0.6pt,line join=round] (150.19,151.48) -- (150.19,143.20);

\path[draw=drawColor,line width= 0.6pt,line join=round] (150.19,151.48) -- (150.19,143.20);

\path[draw=drawColor,line width= 0.6pt,line join=round] (150.19,151.48) -- (150.19,143.20);

\path[draw=drawColor,line width= 0.6pt,line join=round] (150.19,151.48) -- (150.19,143.20);

\path[draw=drawColor,line width= 0.6pt,line join=round] (150.19,151.48) -- (150.19,143.20);

\path[draw=drawColor,line width= 0.6pt,line join=round] (150.19,151.48) -- (150.19,143.20);

\path[draw=drawColor,line width= 0.6pt,line join=round] (150.19,151.48) -- (150.19,143.20);

\path[draw=drawColor,line width= 0.6pt,line join=round] (150.19,151.48) -- (150.19,143.20);

\path[draw=drawColor,line width= 0.6pt,line join=round] (150.19,151.48) -- (150.19,143.20);

\path[draw=drawColor,line width= 0.6pt,line join=round] (150.19,151.48) -- (150.19,143.20);

\path[draw=drawColor,line width= 0.6pt,line join=round] (150.19,151.48) -- (150.19,143.20);

\path[draw=drawColor,line width= 0.6pt,line join=round] (150.19,151.48) -- (150.19,143.20);

\path[draw=drawColor,line width= 0.6pt,line join=round] (150.19,151.48) -- (150.19,143.20);

\path[draw=drawColor,line width= 0.6pt,line join=round] (150.19,151.48) -- (150.19,143.20);

\path[draw=drawColor,line width= 0.6pt,line join=round] (150.19,151.48) -- (150.19,143.20);

\path[draw=drawColor,line width= 0.6pt,line join=round] (150.19,151.48) -- (150.19,143.20);

\path[draw=drawColor,line width= 0.6pt,line join=round] (150.19,151.48) -- (150.19,143.20);

\path[draw=drawColor,line width= 0.6pt,line join=round] (150.19,151.48) -- (150.19,143.20);

\path[draw=drawColor,line width= 0.6pt,line join=round] (150.19,151.48) -- (150.19,143.20);

\path[draw=drawColor,line width= 0.6pt,line join=round] (150.19,151.48) -- (150.19,143.20);

\path[draw=drawColor,line width= 0.6pt,line join=round] (150.19,151.48) -- (150.19,143.20);

\path[draw=drawColor,line width= 0.6pt,line join=round] (150.19,151.48) -- (150.19,143.20);

\path[draw=drawColor,line width= 0.6pt,line join=round] (150.19,151.48) -- (150.19,143.20);

\path[draw=drawColor,line width= 0.6pt,line join=round] (150.19,151.48) -- (150.19,143.20);

\path[draw=drawColor,line width= 0.6pt,line join=round] (150.19,151.48) -- (150.19,143.20);

\path[draw=drawColor,line width= 0.6pt,line join=round] (150.19,151.48) -- (150.19,143.20);

\path[draw=drawColor,line width= 0.6pt,line join=round] (150.19,151.48) -- (150.19,143.20);

\path[draw=drawColor,line width= 0.6pt,line join=round] (150.19,151.48) -- (150.19,143.20);

\path[draw=drawColor,line width= 0.6pt,line join=round] (150.19,151.48) -- (150.19,143.20);

\path[draw=drawColor,line width= 0.6pt,line join=round] (150.19,151.48) -- (150.19,143.20);

\path[draw=drawColor,line width= 0.6pt,line join=round] (150.19,151.48) -- (150.19,143.20);

\path[draw=drawColor,line width= 0.6pt,line join=round] (150.19,151.48) -- (150.19,143.20);

\path[draw=drawColor,line width= 0.6pt,line join=round] (150.19,151.48) -- (150.19,143.20);

\path[draw=drawColor,line width= 0.6pt,line join=round] (150.19,151.48) -- (150.19,143.20);

\path[draw=drawColor,line width= 0.6pt,line join=round] (150.19,151.48) -- (150.19,143.20);

\path[draw=drawColor,line width= 0.6pt,line join=round] (150.19,151.48) -- (150.19,143.20);

\path[draw=drawColor,line width= 0.6pt,line join=round] (150.19,151.48) -- (150.19,143.20);

\path[draw=drawColor,line width= 0.6pt,line join=round] (150.19,151.48) -- (150.19,143.20);

\path[draw=drawColor,line width= 0.6pt,line join=round] (150.19,151.48) -- (150.19,143.20);

\path[draw=drawColor,line width= 0.6pt,line join=round] (150.19,151.48) -- (150.19,143.20);

\path[draw=drawColor,line width= 0.6pt,line join=round] (150.19,151.48) -- (150.19,143.20);

\path[draw=drawColor,line width= 0.6pt,line join=round] (150.19,151.48) -- (150.19,143.20);

\path[draw=drawColor,line width= 0.6pt,line join=round] (150.19,151.48) -- (150.19,143.20);

\path[draw=drawColor,line width= 0.6pt,line join=round] (150.19,151.48) -- (150.19,143.20);

\path[draw=drawColor,line width= 0.6pt,line join=round] (150.19,151.48) -- (150.19,143.20);

\path[draw=drawColor,line width= 0.6pt,line join=round] (150.19,151.48) -- (150.19,143.20);

\path[draw=drawColor,line width= 0.6pt,line join=round] (150.19,151.48) -- (150.19,143.20);

\path[draw=drawColor,line width= 0.6pt,line join=round] (150.19,151.48) -- (150.19,143.20);

\path[draw=drawColor,line width= 0.6pt,line join=round] (150.19,151.48) -- (150.19,143.20);

\path[draw=drawColor,line width= 0.6pt,line join=round] (150.19,151.48) -- (150.19,143.20);

\path[draw=drawColor,line width= 0.6pt,line join=round] (150.19,151.48) -- (150.19,143.20);

\path[draw=drawColor,line width= 0.6pt,line join=round] (150.19,151.48) -- (150.19,143.20);

\path[draw=drawColor,line width= 0.6pt,line join=round] (150.19,151.48) -- (150.19,143.20);

\path[draw=drawColor,line width= 0.6pt,line join=round] (150.19,151.48) -- (150.19,143.20);

\path[draw=drawColor,line width= 0.6pt,line join=round] (150.19,151.48) -- (150.19,143.20);

\path[draw=drawColor,line width= 0.6pt,line join=round] (150.19,151.48) -- (150.19,143.20);

\path[draw=drawColor,line width= 0.6pt,line join=round] (150.19,151.48) -- (150.19,143.20);

\path[draw=drawColor,line width= 0.6pt,line join=round] (150.19,151.48) -- (150.19,143.20);

\path[draw=drawColor,line width= 0.6pt,line join=round] (150.19,151.48) -- (150.19,143.20);

\path[draw=drawColor,line width= 0.6pt,line join=round] (150.19,151.48) -- (150.19,143.20);

\path[draw=drawColor,line width= 0.6pt,line join=round] (150.19,151.48) -- (150.19,143.20);

\path[draw=drawColor,line width= 0.6pt,line join=round] (150.19,151.48) -- (150.19,143.20);

\path[draw=drawColor,line width= 0.6pt,line join=round] (150.19,151.48) -- (150.19,143.20);

\path[draw=drawColor,line width= 0.6pt,line join=round] (150.19,151.48) -- (150.19,143.20);

\path[draw=drawColor,line width= 0.6pt,line join=round] (150.19,151.48) -- (150.19,143.20);

\path[draw=drawColor,line width= 0.6pt,line join=round] (150.19,151.48) -- (150.19,143.20);

\path[draw=drawColor,line width= 0.6pt,line join=round] (150.19,151.48) -- (150.19,143.20);

\path[draw=drawColor,line width= 0.6pt,line join=round] (150.19,151.48) -- (150.19,143.20);

\path[draw=drawColor,line width= 0.6pt,line join=round] (150.19,151.48) -- (150.19,143.20);

\path[draw=drawColor,line width= 0.6pt,line join=round] (150.19,151.48) -- (150.19,143.20);

\path[draw=drawColor,line width= 0.6pt,line join=round] (150.19,151.48) -- (150.19,143.20);

\path[draw=drawColor,line width= 0.6pt,line join=round] (150.19,151.48) -- (150.19,143.20);

\path[draw=drawColor,line width= 0.6pt,line join=round] (150.19,151.48) -- (150.19,143.20);

\path[draw=drawColor,line width= 0.6pt,line join=round] (150.19,151.48) -- (150.19,143.20);

\path[draw=drawColor,line width= 0.6pt,line join=round] (150.19,151.48) -- (150.19,143.20);

\path[draw=drawColor,line width= 0.6pt,line join=round] (150.19,151.48) -- (150.19,143.20);

\path[draw=drawColor,line width= 0.6pt,line join=round] (150.19,151.48) -- (150.19,143.20);

\path[draw=drawColor,line width= 0.6pt,line join=round] (150.19,151.48) -- (150.19,143.20);

\path[draw=drawColor,line width= 0.6pt,line join=round] (150.19,151.48) -- (150.19,143.20);

\path[draw=drawColor,line width= 0.6pt,line join=round] (150.19,151.48) -- (150.19,143.20);

\path[draw=drawColor,line width= 0.6pt,line join=round] (150.19,151.48) -- (150.19,143.20);

\path[draw=drawColor,line width= 0.6pt,line join=round] (150.19,151.48) -- (150.19,143.20);

\path[draw=drawColor,line width= 0.6pt,line join=round] (150.19,151.48) -- (150.19,143.20);

\path[draw=drawColor,line width= 0.6pt,line join=round] (150.19,151.48) -- (150.19,143.20);

\path[draw=drawColor,line width= 0.6pt,line join=round] (150.19,151.48) -- (150.19,143.20);

\path[draw=drawColor,line width= 0.6pt,line join=round] (150.19,151.48) -- (150.19,143.20);

\path[draw=drawColor,line width= 0.6pt,line join=round] (150.19,151.48) -- (150.19,143.20);

\path[draw=drawColor,line width= 0.6pt,line join=round] (150.19,151.48) -- (150.19,143.20);

\path[draw=drawColor,line width= 0.6pt,line join=round] (150.19,151.48) -- (150.19,143.20);

\path[draw=drawColor,line width= 0.6pt,line join=round] (150.19,151.48) -- (150.19,143.20);

\path[draw=drawColor,line width= 0.6pt,line join=round] (150.19,151.48) -- (150.19,143.20);

\path[draw=drawColor,line width= 0.6pt,line join=round] (150.19,151.48) -- (150.19,143.20);

\path[draw=drawColor,line width= 0.6pt,line join=round] (150.19,151.48) -- (150.19,143.20);

\path[draw=drawColor,line width= 0.6pt,line join=round] (150.19,151.48) -- (150.19,143.20);

\path[draw=drawColor,line width= 0.6pt,line join=round] (150.19,151.48) -- (150.19,143.20);

\path[draw=drawColor,line width= 0.6pt,line join=round] (150.19,151.48) -- (150.19,143.20);

\path[draw=drawColor,line width= 0.6pt,line join=round] (150.19,151.48) -- (150.19,143.20);

\path[draw=drawColor,line width= 0.6pt,line join=round] (150.19,151.48) -- (150.19,143.20);

\path[draw=drawColor,line width= 0.6pt,line join=round] (150.19,151.48) -- (150.19,143.20);

\path[draw=drawColor,line width= 0.6pt,line join=round] (150.19,151.48) -- (150.19,143.20);

\path[draw=drawColor,line width= 0.6pt,line join=round] (150.19,151.48) -- (150.19,143.20);

\path[draw=drawColor,line width= 0.6pt,line join=round] (150.19,151.48) -- (150.19,143.20);

\path[draw=drawColor,line width= 0.6pt,line join=round] (150.19,151.48) -- (150.19,143.20);

\path[draw=drawColor,line width= 0.6pt,line join=round] (150.19,151.48) -- (150.19,143.20);

\path[draw=drawColor,line width= 0.6pt,line join=round] (150.19,151.48) -- (150.19,143.20);

\path[draw=drawColor,line width= 0.6pt,line join=round] (150.19,151.48) -- (150.19,143.20);

\path[draw=drawColor,line width= 0.6pt,line join=round] (150.19,151.48) -- (150.19,143.20);

\path[draw=drawColor,line width= 0.6pt,line join=round] (150.19,151.48) -- (150.19,143.20);

\path[draw=drawColor,line width= 0.6pt,line join=round] (150.19,151.48) -- (150.19,143.20);

\path[draw=drawColor,line width= 0.6pt,line join=round] (150.19,151.48) -- (150.19,143.20);

\path[draw=drawColor,line width= 0.6pt,line join=round] (150.19,151.48) -- (150.19,143.20);

\path[draw=drawColor,line width= 0.6pt,line join=round] (150.19,151.48) -- (150.19,143.20);

\path[draw=drawColor,line width= 0.6pt,line join=round] (150.19,151.48) -- (150.19,143.20);

\path[draw=drawColor,line width= 0.6pt,line join=round] (150.19,151.48) -- (150.19,143.20);

\path[draw=drawColor,line width= 0.6pt,line join=round] (150.19,151.48) -- (150.19,143.20);

\path[draw=drawColor,line width= 0.6pt,line join=round] (150.19,151.48) -- (150.19,143.20);

\path[draw=drawColor,line width= 0.6pt,line join=round] (150.19,151.48) -- (150.19,143.20);

\path[draw=drawColor,line width= 0.6pt,line join=round] (150.19,151.48) -- (150.19,143.20);

\path[draw=drawColor,line width= 0.6pt,line join=round] (150.19,151.48) -- (150.19,143.20);

\path[draw=drawColor,line width= 0.6pt,line join=round] (150.19,151.48) -- (150.19,143.20);

\path[draw=drawColor,line width= 0.6pt,line join=round] (150.19,151.48) -- (150.19,143.20);

\path[draw=drawColor,line width= 0.6pt,line join=round] (150.19,151.48) -- (150.19,143.20);

\path[draw=drawColor,line width= 0.6pt,line join=round] (150.19,151.48) -- (150.19,143.20);

\path[draw=drawColor,line width= 0.6pt,line join=round] (150.19,151.48) -- (150.19,143.20);

\path[draw=drawColor,line width= 0.6pt,line join=round] (150.19,151.48) -- (150.19,143.20);

\path[draw=drawColor,line width= 0.6pt,line join=round] (150.19,151.48) -- (150.19,143.20);

\path[draw=drawColor,line width= 0.6pt,line join=round] (150.19,151.48) -- (150.19,143.20);

\path[draw=drawColor,line width= 0.6pt,line join=round] (150.19,151.48) -- (150.19,143.20);

\path[draw=drawColor,line width= 0.6pt,line join=round] (150.19,151.48) -- (150.19,143.20);

\path[draw=drawColor,line width= 0.6pt,line join=round] (150.19,151.48) -- (150.19,143.20);

\path[draw=drawColor,line width= 0.6pt,line join=round] (150.19,151.48) -- (150.19,143.20);

\path[draw=drawColor,line width= 0.6pt,line join=round] (150.19,151.48) -- (150.19,143.20);

\path[draw=drawColor,line width= 0.6pt,line join=round] (150.19,151.48) -- (150.19,143.20);

\path[draw=drawColor,line width= 0.6pt,line join=round] (150.19,151.48) -- (150.19,143.20);

\path[draw=drawColor,line width= 0.6pt,line join=round] (150.19,151.48) -- (150.19,143.20);

\path[draw=drawColor,line width= 0.6pt,line join=round] (150.19,151.48) -- (150.19,143.20);

\path[draw=drawColor,line width= 0.6pt,line join=round] (150.19,151.48) -- (150.19,143.20);

\path[draw=drawColor,line width= 0.6pt,line join=round] (150.19,151.48) -- (150.19,143.20);

\path[draw=drawColor,line width= 0.6pt,line join=round] (150.19,151.48) -- (150.19,143.20);

\path[draw=drawColor,line width= 0.6pt,line join=round] (150.19,151.48) -- (150.19,143.20);

\path[draw=drawColor,line width= 0.6pt,line join=round] (150.19,151.48) -- (150.19,143.20);

\path[draw=drawColor,line width= 0.6pt,line join=round] (150.19,151.48) -- (150.19,143.20);

\path[draw=drawColor,line width= 0.6pt,line join=round] (150.19,151.48) -- (150.19,143.20);

\path[draw=drawColor,line width= 0.6pt,line join=round] (150.19,151.48) -- (150.19,143.20);

\path[draw=drawColor,line width= 0.6pt,line join=round] (150.19,151.48) -- (150.19,143.20);

\path[draw=drawColor,line width= 0.6pt,line join=round] (150.19,151.48) -- (150.19,143.20);

\path[draw=drawColor,line width= 0.6pt,line join=round] (150.19,151.48) -- (150.19,143.20);

\path[draw=drawColor,line width= 0.6pt,line join=round] (150.19,151.48) -- (150.19,143.20);

\path[draw=drawColor,line width= 0.6pt,line join=round] (150.19,151.48) -- (150.19,143.20);

\path[draw=drawColor,line width= 0.6pt,line join=round] (150.19,151.48) -- (150.19,143.20);

\path[draw=drawColor,line width= 0.6pt,line join=round] (150.19,151.48) -- (150.19,143.20);

\path[draw=drawColor,line width= 0.6pt,line join=round] (150.19,151.48) -- (150.19,143.20);

\path[draw=drawColor,line width= 0.6pt,line join=round] (150.19,151.48) -- (150.19,143.20);

\path[draw=drawColor,line width= 0.6pt,line join=round] (150.19,151.48) -- (150.19,143.20);

\path[draw=drawColor,line width= 0.6pt,line join=round] (150.19,151.48) -- (150.19,143.20);

\path[draw=drawColor,line width= 0.6pt,line join=round] (150.19,151.48) -- (150.19,143.20);

\path[draw=drawColor,line width= 0.6pt,line join=round] (150.19,151.48) -- (150.19,143.20);

\path[draw=drawColor,line width= 0.6pt,line join=round] (150.19,151.48) -- (150.19,143.20);

\path[draw=drawColor,line width= 0.6pt,line join=round] (150.19,151.48) -- (150.19,143.20);

\path[draw=drawColor,line width= 0.6pt,line join=round] (150.19,151.48) -- (150.19,143.20);

\path[draw=drawColor,line width= 0.6pt,line join=round] (150.19,151.48) -- (150.19,143.20);

\path[draw=drawColor,line width= 0.6pt,line join=round] (150.19,151.48) -- (150.19,143.20);

\path[draw=drawColor,line width= 0.6pt,line join=round] (150.19,151.48) -- (150.19,143.20);

\path[draw=drawColor,line width= 0.6pt,line join=round] (150.19,151.48) -- (150.19,143.20);

\path[draw=drawColor,line width= 0.6pt,line join=round] (150.19,151.48) -- (150.19,143.20);

\path[draw=drawColor,line width= 0.6pt,line join=round] (150.19,151.48) -- (150.19,143.20);

\path[draw=drawColor,line width= 0.6pt,line join=round] (150.19,151.48) -- (150.19,143.20);

\path[draw=drawColor,line width= 0.6pt,line join=round] (150.19,151.48) -- (150.19,143.20);

\path[draw=drawColor,line width= 0.6pt,line join=round] (150.19,151.48) -- (150.19,143.20);

\path[draw=drawColor,line width= 0.6pt,line join=round] (150.19,151.48) -- (150.19,143.20);

\path[draw=drawColor,line width= 0.6pt,line join=round] (150.19,151.48) -- (150.19,143.20);

\path[draw=drawColor,line width= 0.6pt,line join=round] (150.19,151.48) -- (150.19,143.20);

\path[draw=drawColor,line width= 0.6pt,line join=round] (150.19,151.48) -- (150.19,143.20);

\path[draw=drawColor,line width= 0.6pt,line join=round] (150.19,151.48) -- (150.19,143.20);

\path[draw=drawColor,line width= 0.6pt,line join=round] (150.19,151.48) -- (150.19,143.20);

\path[draw=drawColor,line width= 0.6pt,line join=round] (150.19,151.48) -- (150.19,143.20);

\path[draw=drawColor,line width= 0.6pt,line join=round] (150.19,151.48) -- (150.19,143.20);

\path[draw=drawColor,line width= 0.6pt,line join=round] (150.19,151.48) -- (150.19,143.20);

\path[draw=drawColor,line width= 0.6pt,line join=round] (150.19,151.48) -- (150.19,143.20);

\path[draw=drawColor,line width= 0.6pt,line join=round] (150.19,151.48) -- (150.19,143.20);

\path[draw=drawColor,line width= 0.6pt,line join=round] (150.19,151.48) -- (150.19,143.20);

\path[draw=drawColor,line width= 0.6pt,line join=round] (150.19,151.48) -- (150.19,143.20);

\path[draw=drawColor,line width= 0.6pt,line join=round] (150.19,151.48) -- (150.19,143.20);

\path[draw=drawColor,line width= 0.6pt,line join=round] (150.19,151.48) -- (150.19,143.20);

\path[draw=drawColor,line width= 0.6pt,line join=round] (150.19,151.48) -- (150.19,143.20);

\path[draw=drawColor,line width= 0.6pt,line join=round] (150.19,151.48) -- (150.19,143.20);

\path[draw=drawColor,line width= 0.6pt,line join=round] (150.19,151.48) -- (150.19,143.20);

\path[draw=drawColor,line width= 0.6pt,line join=round] (150.19,151.48) -- (150.19,143.20);

\path[draw=drawColor,line width= 0.6pt,line join=round] (150.19,151.48) -- (150.19,143.20);

\path[draw=drawColor,line width= 0.6pt,line join=round] (150.19,151.48) -- (150.19,143.20);

\path[draw=drawColor,line width= 0.6pt,line join=round] (150.19,151.48) -- (150.19,143.20);

\path[draw=drawColor,line width= 0.6pt,line join=round] (150.19,151.48) -- (150.19,143.20);

\path[draw=drawColor,line width= 0.6pt,line join=round] (150.19,151.48) -- (150.19,143.20);

\path[draw=drawColor,line width= 0.6pt,line join=round] (150.19,151.48) -- (150.19,143.20);

\path[draw=drawColor,line width= 0.6pt,line join=round] (150.19,151.48) -- (150.19,143.20);

\path[draw=drawColor,line width= 0.6pt,line join=round] (150.19,151.48) -- (150.19,143.20);

\path[draw=drawColor,line width= 0.6pt,line join=round] (150.19,151.48) -- (150.19,143.20);

\path[draw=drawColor,line width= 0.6pt,line join=round] (150.19,151.48) -- (150.19,143.20);

\path[draw=drawColor,line width= 0.6pt,line join=round] (150.19,151.48) -- (150.19,143.20);

\path[draw=drawColor,line width= 0.6pt,line join=round] (150.19,151.48) -- (150.19,143.20);

\path[draw=drawColor,line width= 0.6pt,line join=round] (150.19,151.48) -- (150.19,143.20);

\path[draw=drawColor,line width= 0.6pt,line join=round] (150.19,151.48) -- (150.19,143.20);

\path[draw=drawColor,line width= 0.6pt,line join=round] (150.19,151.48) -- (150.19,143.20);

\path[draw=drawColor,line width= 0.6pt,line join=round] (150.19,151.48) -- (150.19,143.20);

\path[draw=drawColor,line width= 0.6pt,line join=round] (150.19,151.48) -- (150.19,143.20);

\path[draw=drawColor,line width= 0.6pt,line join=round] (150.19,151.48) -- (150.19,143.20);

\path[draw=drawColor,line width= 0.6pt,line join=round] (150.19,151.48) -- (150.19,143.20);

\path[draw=drawColor,line width= 0.6pt,line join=round] (150.19,151.48) -- (150.19,143.20);

\path[draw=drawColor,line width= 0.6pt,line join=round] (150.19,151.48) -- (150.19,143.20);

\path[draw=drawColor,line width= 0.6pt,line join=round] (150.19,151.48) -- (150.19,143.20);

\path[draw=drawColor,line width= 0.6pt,line join=round] (150.19,151.48) -- (150.19,143.20);

\path[draw=drawColor,line width= 0.6pt,line join=round] (150.19,151.48) -- (150.19,143.20);

\path[draw=drawColor,line width= 0.6pt,line join=round] (150.19,151.48) -- (150.19,143.20);

\path[draw=drawColor,line width= 0.6pt,line join=round] (150.19,151.48) -- (150.19,143.20);

\path[draw=drawColor,line width= 0.6pt,line join=round] (150.19,151.48) -- (150.19,143.20);

\path[draw=drawColor,line width= 0.6pt,line join=round] (150.19,151.48) -- (150.19,143.20);

\path[draw=drawColor,line width= 0.6pt,line join=round] (150.19,151.48) -- (150.19,143.20);

\path[draw=drawColor,line width= 0.6pt,line join=round] (150.19,151.48) -- (150.19,143.20);

\path[draw=drawColor,line width= 0.6pt,line join=round] (150.19,151.48) -- (150.19,143.20);

\path[draw=drawColor,line width= 0.6pt,line join=round] (150.19,151.48) -- (150.19,143.20);

\path[draw=drawColor,line width= 0.6pt,line join=round] (150.19,151.48) -- (150.19,143.20);

\path[draw=drawColor,line width= 0.6pt,line join=round] (150.19,151.48) -- (150.19,143.20);

\path[draw=drawColor,line width= 0.6pt,line join=round] (150.19,151.48) -- (150.19,143.20);

\path[draw=drawColor,line width= 0.6pt,line join=round] (150.19,151.48) -- (150.19,143.20);

\path[draw=drawColor,line width= 0.6pt,line join=round] (150.19,151.48) -- (150.19,143.20);

\path[draw=drawColor,line width= 0.6pt,line join=round] (150.19,151.48) -- (150.19,143.20);

\path[draw=drawColor,line width= 0.6pt,line join=round] (150.19,151.48) -- (150.19,143.20);

\path[draw=drawColor,line width= 0.6pt,line join=round] (150.19,151.48) -- (150.19,143.20);

\path[draw=drawColor,line width= 0.6pt,line join=round] (150.19,151.48) -- (150.19,143.20);

\path[draw=drawColor,line width= 0.6pt,line join=round] (150.19,151.48) -- (150.19,143.20);

\path[draw=drawColor,line width= 0.6pt,line join=round] (150.19,151.48) -- (150.19,143.20);

\path[draw=drawColor,line width= 0.6pt,line join=round] (150.19,151.48) -- (150.19,143.20);

\path[draw=drawColor,line width= 0.6pt,line join=round] (150.19,151.48) -- (150.19,143.20);

\path[draw=drawColor,line width= 0.6pt,line join=round] (150.19,151.48) -- (150.19,143.20);

\path[draw=drawColor,line width= 0.6pt,line join=round] (150.19,151.48) -- (150.19,143.20);

\path[draw=drawColor,line width= 0.6pt,line join=round] (150.19,151.48) -- (150.19,143.20);

\path[draw=drawColor,line width= 0.6pt,line join=round] (150.19,151.48) -- (150.19,143.20);

\path[draw=drawColor,line width= 0.6pt,line join=round] (150.19,151.48) -- (150.19,143.20);

\path[draw=drawColor,line width= 0.6pt,line join=round] (150.19,151.48) -- (150.19,143.20);

\path[draw=drawColor,line width= 0.6pt,line join=round] (150.19,151.48) -- (150.19,143.20);

\path[draw=drawColor,line width= 0.6pt,line join=round] (150.19,151.48) -- (150.19,143.20);

\path[draw=drawColor,line width= 0.6pt,line join=round] (150.19,151.48) -- (150.19,143.20);

\path[draw=drawColor,line width= 0.6pt,line join=round] (150.19,151.48) -- (150.19,143.20);

\path[draw=drawColor,line width= 0.6pt,line join=round] (150.19,151.48) -- (150.19,143.20);

\path[draw=drawColor,line width= 0.6pt,line join=round] (150.19,151.48) -- (150.19,143.20);

\path[draw=drawColor,line width= 0.6pt,line join=round] (150.19,151.48) -- (150.19,143.20);

\path[draw=drawColor,line width= 0.6pt,line join=round] (150.19,151.48) -- (150.19,143.20);

\path[draw=drawColor,line width= 0.6pt,line join=round] (150.19,151.48) -- (150.19,143.20);

\path[draw=drawColor,line width= 0.6pt,line join=round] (150.19,151.48) -- (150.19,143.20);

\path[draw=drawColor,line width= 0.6pt,line join=round] (150.19,151.48) -- (150.19,143.20);

\path[draw=drawColor,line width= 0.6pt,line join=round] (150.19,151.48) -- (150.19,143.20);

\path[draw=drawColor,line width= 0.6pt,line join=round] (150.19,151.48) -- (150.19,143.20);

\path[draw=drawColor,line width= 0.6pt,line join=round] (150.19,151.48) -- (150.19,143.20);

\path[draw=drawColor,line width= 0.6pt,line join=round] (150.19,151.48) -- (150.19,143.20);

\path[draw=drawColor,line width= 0.6pt,line join=round] (150.19,151.48) -- (150.19,143.20);

\path[draw=drawColor,line width= 0.6pt,line join=round] (150.19,151.48) -- (150.19,143.20);

\path[draw=drawColor,line width= 0.6pt,line join=round] (150.19,151.48) -- (150.19,143.20);

\path[draw=drawColor,line width= 0.6pt,line join=round] (150.19,151.48) -- (150.19,143.20);

\path[draw=drawColor,line width= 0.6pt,line join=round] (150.19,151.48) -- (150.19,143.20);

\path[draw=drawColor,line width= 0.6pt,line join=round] (150.19,151.48) -- (150.19,143.20);

\path[draw=drawColor,line width= 0.6pt,line join=round] (150.19,151.48) -- (150.19,143.20);

\path[draw=drawColor,line width= 0.6pt,line join=round] (150.19,151.48) -- (150.19,143.20);

\path[draw=drawColor,line width= 0.6pt,line join=round] (150.19,151.48) -- (150.19,143.20);

\path[draw=drawColor,line width= 0.6pt,line join=round] (150.19,151.48) -- (150.19,143.20);

\path[draw=drawColor,line width= 0.6pt,line join=round] (150.19,151.48) -- (150.19,143.20);

\path[draw=drawColor,line width= 0.6pt,line join=round] (150.19,151.48) -- (150.19,143.20);

\path[draw=drawColor,line width= 0.6pt,line join=round] (150.19,151.48) -- (150.19,143.20);

\path[draw=drawColor,line width= 0.6pt,line join=round] (150.19,151.48) -- (150.19,143.20);

\path[draw=drawColor,line width= 0.6pt,line join=round] (150.19,151.48) -- (150.19,143.20);

\path[draw=drawColor,line width= 0.6pt,line join=round] (150.19,151.48) -- (150.19,143.20);

\path[draw=drawColor,line width= 0.6pt,line join=round] (150.19,151.48) -- (150.19,143.20);

\path[draw=drawColor,line width= 0.6pt,line join=round] (150.19,151.48) -- (150.19,143.20);

\path[draw=drawColor,line width= 0.6pt,line join=round] (150.19,151.48) -- (150.19,143.20);

\path[draw=drawColor,line width= 0.6pt,line join=round] (150.19,151.48) -- (150.19,143.20);

\path[draw=drawColor,line width= 0.6pt,line join=round] (150.19,151.48) -- (150.19,143.20);

\path[draw=drawColor,line width= 0.6pt,line join=round] (150.19,151.48) -- (150.19,143.20);

\path[draw=drawColor,line width= 0.6pt,line join=round] (150.19,151.48) -- (150.19,143.20);

\path[draw=drawColor,line width= 0.6pt,line join=round] (150.19,151.48) -- (150.19,143.20);

\path[draw=drawColor,line width= 0.6pt,line join=round] (150.19,151.48) -- (150.19,143.20);

\path[draw=drawColor,line width= 0.6pt,line join=round] (150.19,151.48) -- (150.19,143.20);

\path[draw=drawColor,line width= 0.6pt,line join=round] (150.19,151.48) -- (150.19,143.20);

\path[draw=drawColor,line width= 0.6pt,line join=round] (150.19,151.48) -- (150.19,143.20);

\path[draw=drawColor,line width= 0.6pt,line join=round] (150.19,151.48) -- (150.19,143.20);

\path[draw=drawColor,line width= 0.6pt,line join=round] (150.19,151.48) -- (150.19,143.20);

\path[draw=drawColor,line width= 0.6pt,line join=round] (150.19,151.48) -- (150.19,143.20);

\path[draw=drawColor,line width= 0.6pt,line join=round] (150.19,151.48) -- (150.19,143.20);

\path[draw=drawColor,line width= 0.6pt,line join=round] (150.19,151.48) -- (150.19,143.20);

\path[draw=drawColor,line width= 0.6pt,line join=round] (150.19,151.48) -- (150.19,143.20);

\path[draw=drawColor,line width= 0.6pt,line join=round] (150.19,151.48) -- (150.19,143.20);

\path[draw=drawColor,line width= 0.6pt,line join=round] (150.19,151.48) -- (150.19,143.20);

\path[draw=drawColor,line width= 0.6pt,line join=round] (150.19,151.48) -- (150.19,143.20);

\path[draw=drawColor,line width= 0.6pt,line join=round] (150.19,151.48) -- (150.19,143.20);

\path[draw=drawColor,line width= 0.6pt,line join=round] (150.19,151.48) -- (150.19,143.20);

\path[draw=drawColor,line width= 0.6pt,line join=round] (150.19,151.48) -- (150.19,143.20);

\path[draw=drawColor,line width= 0.6pt,line join=round] (150.19,151.48) -- (150.19,143.20);

\path[draw=drawColor,line width= 0.6pt,line join=round] (150.19,151.48) -- (150.19,143.20);

\path[draw=drawColor,line width= 0.6pt,line join=round] (150.19,151.48) -- (150.19,143.20);

\path[draw=drawColor,line width= 0.6pt,line join=round] (150.19,151.48) -- (150.19,143.20);

\path[draw=drawColor,line width= 0.6pt,line join=round] (150.19,151.48) -- (150.19,143.20);

\path[draw=drawColor,line width= 0.6pt,line join=round] (150.19,151.48) -- (150.19,143.20);

\path[draw=drawColor,line width= 0.6pt,line join=round] (150.19,151.48) -- (150.19,143.20);

\path[draw=drawColor,line width= 0.6pt,line join=round] (150.19,151.48) -- (150.19,143.20);

\path[draw=drawColor,line width= 0.6pt,line join=round] (150.19,151.48) -- (150.19,143.20);

\path[draw=drawColor,line width= 0.6pt,line join=round] (150.19,151.48) -- (150.19,143.20);

\path[draw=drawColor,line width= 0.6pt,line join=round] (150.19,151.48) -- (150.19,143.20);

\path[draw=drawColor,line width= 0.6pt,line join=round] (150.19,151.48) -- (150.19,143.20);

\path[draw=drawColor,line width= 0.6pt,line join=round] (150.19,151.48) -- (150.19,143.20);

\path[draw=drawColor,line width= 0.6pt,line join=round] (150.19,151.48) -- (150.19,143.20);

\path[draw=drawColor,line width= 0.6pt,line join=round] (150.19,151.48) -- (150.19,143.20);

\path[draw=drawColor,line width= 0.6pt,line join=round] (150.19,151.48) -- (150.19,143.20);

\path[draw=drawColor,line width= 0.6pt,line join=round] (150.19,151.48) -- (150.19,143.20);

\path[draw=drawColor,line width= 0.6pt,line join=round] (150.19,151.48) -- (150.19,143.20);

\path[draw=drawColor,line width= 0.6pt,line join=round] (150.19,151.48) -- (150.19,143.20);

\path[draw=drawColor,line width= 0.6pt,line join=round] (150.19,151.48) -- (150.19,143.20);

\path[draw=drawColor,line width= 0.6pt,line join=round] (150.19,151.48) -- (150.19,143.20);

\path[draw=drawColor,line width= 0.6pt,line join=round] (150.19,151.48) -- (150.19,143.20);

\path[draw=drawColor,line width= 0.6pt,line join=round] (150.19,151.48) -- (150.19,143.20);

\path[draw=drawColor,line width= 0.6pt,line join=round] (150.19,151.48) -- (150.19,143.20);

\path[draw=drawColor,line width= 0.6pt,line join=round] (150.19,151.48) -- (150.19,143.20);

\path[draw=drawColor,line width= 0.6pt,line join=round] (150.19,151.48) -- (150.19,143.20);

\path[draw=drawColor,line width= 0.6pt,line join=round] (150.19,151.48) -- (150.19,143.20);

\path[draw=drawColor,line width= 0.6pt,line join=round] (150.19,151.48) -- (150.19,143.20);

\path[draw=drawColor,line width= 0.6pt,line join=round] (150.19,151.48) -- (150.19,143.20);

\path[draw=drawColor,line width= 0.6pt,line join=round] (150.19,151.48) -- (150.19,143.20);

\path[draw=drawColor,line width= 0.6pt,line join=round] (150.19,151.48) -- (150.19,143.20);

\path[draw=drawColor,line width= 0.6pt,line join=round] (150.19,151.48) -- (150.19,143.20);

\path[draw=drawColor,line width= 0.6pt,line join=round] (150.19,151.48) -- (150.19,143.20);

\path[draw=drawColor,line width= 0.6pt,line join=round] (150.19,151.48) -- (150.19,143.20);

\path[draw=drawColor,line width= 0.6pt,line join=round] (150.19,151.48) -- (150.19,143.20);

\path[draw=drawColor,line width= 0.6pt,line join=round] (150.19,151.48) -- (150.19,143.20);

\path[draw=drawColor,line width= 0.6pt,line join=round] (150.19,151.48) -- (150.19,143.20);

\path[draw=drawColor,line width= 0.6pt,line join=round] (150.19,151.48) -- (150.19,143.20);

\path[draw=drawColor,line width= 0.6pt,line join=round] (150.19,151.48) -- (150.19,143.20);

\path[draw=drawColor,line width= 0.6pt,line join=round] (150.19,151.48) -- (150.19,143.20);

\path[draw=drawColor,line width= 0.6pt,line join=round] (150.19,151.48) -- (150.19,143.20);

\path[draw=drawColor,line width= 0.6pt,line join=round] (150.19,151.48) -- (150.19,143.20);

\path[draw=drawColor,line width= 0.6pt,line join=round] (150.19,151.48) -- (150.19,143.20);

\path[draw=drawColor,line width= 0.6pt,line join=round] (150.19,151.48) -- (150.19,143.20);

\path[draw=drawColor,line width= 0.6pt,line join=round] (150.19,151.48) -- (150.19,143.20);

\path[draw=drawColor,line width= 0.6pt,line join=round] (150.19,151.48) -- (150.19,143.20);

\path[draw=drawColor,line width= 0.6pt,line join=round] (150.19,151.48) -- (150.19,143.20);

\path[draw=drawColor,line width= 0.6pt,line join=round] (150.19,151.48) -- (150.19,143.20);

\path[draw=drawColor,line width= 0.6pt,line join=round] (150.19,151.48) -- (150.19,143.20);

\path[draw=drawColor,line width= 0.6pt,line join=round] (150.19,151.48) -- (150.19,143.20);

\path[draw=drawColor,line width= 0.6pt,line join=round] (150.19,151.48) -- (150.19,143.20);

\path[draw=drawColor,line width= 0.6pt,line join=round] (150.19,151.48) -- (150.19,143.20);

\path[draw=drawColor,line width= 0.6pt,line join=round] (150.19,151.48) -- (150.19,143.20);

\path[draw=drawColor,line width= 0.6pt,line join=round] (150.19,151.48) -- (150.19,143.20);

\path[draw=drawColor,line width= 0.6pt,line join=round] (150.19,151.48) -- (150.19,143.20);

\path[draw=drawColor,line width= 0.6pt,line join=round] (150.19,151.48) -- (150.19,143.20);

\path[draw=drawColor,line width= 0.6pt,line join=round] (150.19,151.48) -- (150.19,143.20);

\path[draw=drawColor,line width= 0.6pt,line join=round] (150.19,151.48) -- (150.19,143.20);

\path[draw=drawColor,line width= 0.6pt,line join=round] (150.19,151.48) -- (150.19,143.20);

\path[draw=drawColor,line width= 0.6pt,line join=round] (150.19,151.48) -- (150.19,143.20);

\path[draw=drawColor,line width= 0.6pt,line join=round] (150.19,151.48) -- (150.19,143.20);

\path[draw=drawColor,line width= 0.6pt,line join=round] (150.19,151.48) -- (150.19,143.20);

\path[draw=drawColor,line width= 0.6pt,line join=round] (150.19,151.48) -- (150.19,143.20);

\path[draw=drawColor,line width= 0.6pt,line join=round] (150.19,151.48) -- (150.19,143.20);

\path[draw=drawColor,line width= 0.6pt,line join=round] (150.19,151.48) -- (150.19,143.20);

\path[draw=drawColor,line width= 0.6pt,line join=round] (150.19,151.48) -- (150.19,143.20);

\path[draw=drawColor,line width= 0.6pt,line join=round] (150.19,151.48) -- (150.19,143.20);

\path[draw=drawColor,line width= 0.6pt,line join=round] (150.19,151.48) -- (150.19,143.20);

\path[draw=drawColor,line width= 0.6pt,line join=round] (150.19,151.48) -- (150.19,143.20);

\path[draw=drawColor,line width= 0.6pt,line join=round] (150.19,151.48) -- (150.19,143.20);

\path[draw=drawColor,line width= 0.6pt,line join=round] (150.19,151.48) -- (150.19,143.20);

\path[draw=drawColor,line width= 0.6pt,line join=round] (150.19,151.48) -- (150.19,143.20);

\path[draw=drawColor,line width= 0.6pt,line join=round] (150.19,151.48) -- (150.19,143.20);

\path[draw=drawColor,line width= 0.6pt,line join=round] (150.19,151.48) -- (150.19,143.20);

\path[draw=drawColor,line width= 0.6pt,line join=round] (150.19,151.48) -- (150.19,143.20);

\path[draw=drawColor,line width= 0.6pt,line join=round] (150.19,151.48) -- (150.19,143.20);

\path[draw=drawColor,line width= 0.6pt,line join=round] (150.19,151.48) -- (150.19,143.20);

\path[draw=drawColor,line width= 0.6pt,line join=round] (150.19,151.48) -- (150.19,143.20);

\path[draw=drawColor,line width= 0.6pt,line join=round] (150.19,151.48) -- (150.19,143.20);

\path[draw=drawColor,line width= 0.6pt,line join=round] (150.19,151.48) -- (150.19,143.20);

\path[draw=drawColor,line width= 0.6pt,line join=round] (150.19,151.48) -- (150.19,143.20);

\path[draw=drawColor,line width= 0.6pt,line join=round] (150.19,151.48) -- (150.19,143.20);

\path[draw=drawColor,line width= 0.6pt,line join=round] (150.19,151.48) -- (150.19,143.20);

\path[draw=drawColor,line width= 0.6pt,line join=round] (150.19,151.48) -- (150.19,143.20);

\path[draw=drawColor,line width= 0.6pt,line join=round] (150.19,151.48) -- (150.19,143.20);

\path[draw=drawColor,line width= 0.6pt,line join=round] (150.19,151.48) -- (150.19,143.20);

\path[draw=drawColor,line width= 0.6pt,line join=round] (150.19,151.48) -- (150.19,143.20);

\path[draw=drawColor,line width= 0.6pt,line join=round] (150.19,151.48) -- (150.19,143.20);

\path[draw=drawColor,line width= 0.6pt,line join=round] (150.19,151.48) -- (150.19,143.20);

\path[draw=drawColor,line width= 0.6pt,line join=round] (150.19,151.48) -- (150.19,143.20);

\path[draw=drawColor,line width= 0.6pt,line join=round] (150.19,151.48) -- (150.19,143.20);

\path[draw=drawColor,line width= 0.6pt,line join=round] (150.19,151.48) -- (150.19,143.20);

\path[draw=drawColor,line width= 0.6pt,line join=round] (150.19,151.48) -- (150.19,143.20);

\path[draw=drawColor,line width= 0.6pt,line join=round] (150.19,151.48) -- (150.19,143.20);

\path[draw=drawColor,line width= 0.6pt,line join=round] (150.19,151.48) -- (150.19,143.20);

\path[draw=drawColor,line width= 0.6pt,line join=round] (150.19,151.48) -- (150.19,143.20);

\path[draw=drawColor,line width= 0.6pt,line join=round] (150.19,151.48) -- (150.19,143.20);

\path[draw=drawColor,line width= 0.6pt,line join=round] (150.19,151.48) -- (150.19,143.20);

\path[draw=drawColor,line width= 0.6pt,line join=round] (150.19,151.48) -- (150.19,143.20);

\path[draw=drawColor,line width= 0.6pt,line join=round] (150.19,151.48) -- (150.19,143.20);

\path[draw=drawColor,line width= 0.6pt,line join=round] (150.19,151.48) -- (150.19,143.20);

\path[draw=drawColor,line width= 0.6pt,line join=round] (150.19,151.48) -- (150.19,143.20);

\path[draw=drawColor,line width= 0.6pt,line join=round] (150.19,151.48) -- (150.19,143.20);

\path[draw=drawColor,line width= 0.6pt,line join=round] (150.19,151.48) -- (150.19,143.20);

\path[draw=drawColor,line width= 0.6pt,line join=round] (150.19,151.48) -- (150.19,143.20);

\path[draw=drawColor,line width= 0.6pt,line join=round] (150.19,151.48) -- (150.19,143.20);

\path[draw=drawColor,line width= 0.6pt,line join=round] (150.19,151.48) -- (150.19,143.20);

\path[draw=drawColor,line width= 0.6pt,line join=round] (150.19,151.48) -- (150.19,143.20);

\path[draw=drawColor,line width= 0.6pt,line join=round] (150.19,151.48) -- (150.19,143.20);

\path[draw=drawColor,line width= 0.6pt,line join=round] (150.19,151.48) -- (150.19,143.20);

\path[draw=drawColor,line width= 0.6pt,line join=round] (150.19,151.48) -- (150.19,143.20);

\path[draw=drawColor,line width= 0.6pt,line join=round] (150.19,151.48) -- (150.19,143.20);

\path[draw=drawColor,line width= 0.6pt,line join=round] (150.19,151.48) -- (150.19,143.20);

\path[draw=drawColor,line width= 0.6pt,line join=round] (150.19,151.48) -- (150.19,143.20);

\path[draw=drawColor,line width= 0.6pt,line join=round] (150.19,151.48) -- (150.19,143.20);

\path[draw=drawColor,line width= 0.6pt,line join=round] (150.19,151.48) -- (150.19,143.20);

\path[draw=drawColor,line width= 0.6pt,line join=round] (150.19,151.48) -- (150.19,143.20);

\path[draw=drawColor,line width= 0.6pt,line join=round] (150.19,151.48) -- (150.19,143.20);

\path[draw=drawColor,line width= 0.6pt,line join=round] (150.19,151.48) -- (150.19,143.20);

\path[draw=drawColor,line width= 0.6pt,line join=round] (150.19,151.48) -- (150.19,143.20);

\path[draw=drawColor,line width= 0.6pt,line join=round] (150.19,151.48) -- (150.19,143.20);

\path[draw=drawColor,line width= 0.6pt,line join=round] (150.19,151.48) -- (150.19,143.20);

\path[draw=drawColor,line width= 0.6pt,line join=round] (150.19,151.48) -- (150.19,143.20);

\path[draw=drawColor,line width= 0.6pt,line join=round] (150.19,151.48) -- (150.19,143.20);

\path[draw=drawColor,line width= 0.6pt,line join=round] (150.19,151.48) -- (150.19,143.20);

\path[draw=drawColor,line width= 0.6pt,line join=round] (150.19,151.48) -- (150.19,143.20);

\path[draw=drawColor,line width= 0.6pt,line join=round] (150.19,151.48) -- (150.19,143.20);

\path[draw=drawColor,line width= 0.6pt,line join=round] (150.19,151.48) -- (150.19,143.20);

\path[draw=drawColor,line width= 0.6pt,line join=round] (150.19,151.48) -- (150.19,143.20);

\path[draw=drawColor,line width= 0.6pt,line join=round] (150.19,151.48) -- (150.19,143.20);

\path[draw=drawColor,line width= 0.6pt,line join=round] (150.19,151.48) -- (150.19,143.20);

\path[draw=drawColor,line width= 0.6pt,line join=round] (150.19,151.48) -- (150.19,143.20);

\path[draw=drawColor,line width= 0.6pt,line join=round] (150.19,151.48) -- (150.19,143.20);

\path[draw=drawColor,line width= 0.6pt,line join=round] (150.19,151.48) -- (150.19,143.20);

\path[draw=drawColor,line width= 0.6pt,line join=round] (150.19,151.48) -- (150.19,143.20);

\path[draw=drawColor,line width= 0.6pt,line join=round] (150.19,151.48) -- (150.19,143.20);

\path[draw=drawColor,line width= 0.6pt,line join=round] (150.19,151.48) -- (150.19,143.20);

\path[draw=drawColor,line width= 0.6pt,line join=round] (150.19,151.48) -- (150.19,143.20);

\path[draw=drawColor,line width= 0.6pt,line join=round] (150.19,151.48) -- (150.19,143.20);

\path[draw=drawColor,line width= 0.6pt,line join=round] (150.19,151.48) -- (150.19,143.20);

\path[draw=drawColor,line width= 0.6pt,line join=round] (150.19,151.48) -- (150.19,143.20);

\path[draw=drawColor,line width= 0.6pt,line join=round] (150.19,151.48) -- (150.19,143.20);

\path[draw=drawColor,line width= 0.6pt,line join=round] (150.19,151.48) -- (150.19,143.20);

\path[draw=drawColor,line width= 0.6pt,line join=round] (150.19,151.48) -- (150.19,143.20);

\path[draw=drawColor,line width= 0.6pt,line join=round] (150.19,151.48) -- (150.19,143.20);

\path[draw=drawColor,line width= 0.6pt,line join=round] (150.19,151.48) -- (150.19,143.20);

\path[draw=drawColor,line width= 0.6pt,line join=round] (150.19,151.48) -- (150.19,143.20);

\path[draw=drawColor,line width= 0.6pt,line join=round] (150.19,151.48) -- (150.19,143.20);

\path[draw=drawColor,line width= 0.6pt,line join=round] (150.19,151.48) -- (150.19,143.20);

\path[draw=drawColor,line width= 0.6pt,line join=round] (150.19,151.48) -- (150.19,143.20);

\path[draw=drawColor,line width= 0.6pt,line join=round] (150.19,151.48) -- (150.19,143.20);

\path[draw=drawColor,line width= 0.6pt,line join=round] (150.19,151.48) -- (150.19,143.20);

\path[draw=drawColor,line width= 0.6pt,line join=round] (150.19,151.48) -- (150.19,143.20);

\path[draw=drawColor,line width= 0.6pt,line join=round] (150.19,151.48) -- (150.19,143.20);

\path[draw=drawColor,line width= 0.6pt,line join=round] (150.19,151.48) -- (150.19,143.20);

\path[draw=drawColor,line width= 0.6pt,line join=round] (150.19,151.48) -- (150.19,143.20);

\path[draw=drawColor,line width= 0.6pt,line join=round] (150.19,151.48) -- (150.19,143.20);

\path[draw=drawColor,line width= 0.6pt,line join=round] (150.19,151.48) -- (150.19,143.20);

\path[draw=drawColor,line width= 0.6pt,line join=round] (150.19,151.48) -- (150.19,143.20);

\path[draw=drawColor,line width= 0.6pt,line join=round] (150.19,151.48) -- (150.19,143.20);

\path[draw=drawColor,line width= 0.6pt,line join=round] (150.19,151.48) -- (150.19,143.20);

\path[draw=drawColor,line width= 0.6pt,line join=round] (150.19,151.48) -- (150.19,143.20);

\path[draw=drawColor,line width= 0.6pt,line join=round] (150.19,151.48) -- (150.19,143.20);

\path[draw=drawColor,line width= 0.6pt,line join=round] (150.19,151.48) -- (150.19,143.20);

\path[draw=drawColor,line width= 0.6pt,line join=round] (150.19,151.48) -- (150.19,143.20);

\path[draw=drawColor,line width= 0.6pt,line join=round] (150.19,151.48) -- (150.19,143.20);

\path[draw=drawColor,line width= 0.6pt,line join=round] (150.19,151.48) -- (150.19,143.20);

\path[draw=drawColor,line width= 0.6pt,line join=round] (150.19,151.48) -- (150.19,143.20);

\path[draw=drawColor,line width= 0.6pt,line join=round] (150.19,151.48) -- (150.19,143.20);

\path[draw=drawColor,line width= 0.6pt,line join=round] (150.19,151.48) -- (150.19,143.20);

\path[draw=drawColor,line width= 0.6pt,line join=round] (150.19,151.48) -- (150.19,143.20);

\path[draw=drawColor,line width= 0.6pt,line join=round] (150.19,151.48) -- (150.19,143.20);

\path[draw=drawColor,line width= 0.6pt,line join=round] (150.19,151.48) -- (150.19,143.20);

\path[draw=drawColor,line width= 0.6pt,line join=round] (150.19,151.48) -- (150.19,143.20);

\path[draw=drawColor,line width= 0.6pt,line join=round] (150.19,151.48) -- (150.19,143.20);

\path[draw=drawColor,line width= 0.6pt,line join=round] (150.19,151.48) -- (150.19,143.20);

\path[draw=drawColor,line width= 0.6pt,line join=round] (150.19,151.48) -- (150.19,143.20);

\path[draw=drawColor,line width= 0.6pt,line join=round] (150.19,151.48) -- (150.19,143.20);

\path[draw=drawColor,line width= 0.6pt,line join=round] (150.19,151.48) -- (150.19,143.20);

\path[draw=drawColor,line width= 0.6pt,line join=round] (150.19,151.48) -- (150.19,143.20);

\path[draw=drawColor,line width= 0.6pt,line join=round] (150.19,151.48) -- (150.19,143.20);

\path[draw=drawColor,line width= 0.6pt,line join=round] (150.19,151.48) -- (150.19,143.20);

\path[draw=drawColor,line width= 0.6pt,line join=round] (150.19,151.48) -- (150.19,143.20);

\path[draw=drawColor,line width= 0.6pt,line join=round] (150.19,151.48) -- (150.19,143.20);

\path[draw=drawColor,line width= 0.6pt,line join=round] (150.19,151.48) -- (150.19,143.20);

\path[draw=drawColor,line width= 0.6pt,line join=round] (150.19,151.48) -- (150.19,143.20);

\path[draw=drawColor,line width= 0.6pt,line join=round] (150.19,151.48) -- (150.19,143.20);

\path[draw=drawColor,line width= 0.6pt,line join=round] (150.19,151.48) -- (150.19,143.20);

\path[draw=drawColor,line width= 0.6pt,line join=round] (150.19,151.48) -- (150.19,143.20);

\path[draw=drawColor,line width= 0.6pt,line join=round] (150.19,151.48) -- (150.19,143.20);

\path[draw=drawColor,line width= 0.6pt,line join=round] (150.19,151.48) -- (150.19,143.20);

\path[draw=drawColor,line width= 0.6pt,line join=round] (150.19,151.48) -- (150.19,143.20);

\path[draw=drawColor,line width= 0.6pt,line join=round] (150.19,151.48) -- (150.19,143.20);

\path[draw=drawColor,line width= 0.6pt,line join=round] (150.19,151.48) -- (150.19,143.20);

\path[draw=drawColor,line width= 0.6pt,line join=round] (150.19,151.48) -- (150.19,143.20);

\path[draw=drawColor,line width= 0.6pt,line join=round] (150.19,151.48) -- (150.19,143.20);

\path[draw=drawColor,line width= 0.6pt,line join=round] (150.19,151.48) -- (150.19,143.20);

\path[draw=drawColor,line width= 0.6pt,line join=round] (150.19,151.48) -- (150.19,143.20);

\path[draw=drawColor,line width= 0.6pt,line join=round] (150.19,151.48) -- (150.19,143.20);

\path[draw=drawColor,line width= 0.6pt,line join=round] (150.19,151.48) -- (150.19,143.20);

\path[draw=drawColor,line width= 0.6pt,line join=round] (150.19,151.48) -- (150.19,143.20);

\path[draw=drawColor,line width= 0.6pt,line join=round] (150.19,151.48) -- (150.19,143.20);

\path[draw=drawColor,line width= 0.6pt,line join=round] (150.19,151.48) -- (150.19,143.20);

\path[draw=drawColor,line width= 0.6pt,line join=round] (150.19,151.48) -- (150.19,143.20);

\path[draw=drawColor,line width= 0.6pt,line join=round] (150.19,151.48) -- (150.19,143.20);

\path[draw=drawColor,line width= 0.6pt,line join=round] (150.19,151.48) -- (150.19,143.20);

\path[draw=drawColor,line width= 0.6pt,line join=round] (150.19,151.48) -- (150.19,143.20);

\path[draw=drawColor,line width= 0.6pt,line join=round] (150.19,151.48) -- (150.19,143.20);

\path[draw=drawColor,line width= 0.6pt,line join=round] (150.19,151.48) -- (150.19,143.20);

\path[draw=drawColor,line width= 0.6pt,line join=round] (150.19,151.48) -- (150.19,143.20);

\path[draw=drawColor,line width= 0.6pt,line join=round] (150.19,151.48) -- (150.19,143.20);

\path[draw=drawColor,line width= 0.6pt,line join=round] (150.19,151.48) -- (150.19,143.20);

\path[draw=drawColor,line width= 0.6pt,line join=round] (150.19,151.48) -- (150.19,143.20);

\path[draw=drawColor,line width= 0.6pt,line join=round] (150.19,151.48) -- (150.19,143.20);

\path[draw=drawColor,line width= 0.6pt,line join=round] (150.19,151.48) -- (150.19,143.20);

\path[draw=drawColor,line width= 0.6pt,line join=round] (150.19,151.48) -- (150.19,143.20);

\path[draw=drawColor,line width= 0.6pt,line join=round] (150.19,151.48) -- (150.19,143.20);

\path[draw=drawColor,line width= 0.6pt,line join=round] (150.19,151.48) -- (150.19,143.20);

\path[draw=drawColor,line width= 0.6pt,line join=round] (150.19,151.48) -- (150.19,143.20);

\path[draw=drawColor,line width= 0.6pt,line join=round] (150.19,151.48) -- (150.19,143.20);

\path[draw=drawColor,line width= 0.6pt,line join=round] (150.19,151.48) -- (150.19,143.20);

\path[draw=drawColor,line width= 0.6pt,line join=round] (150.19,151.48) -- (150.19,143.20);

\path[draw=drawColor,line width= 0.6pt,line join=round] (150.19,151.48) -- (150.19,143.20);

\path[draw=drawColor,line width= 0.6pt,line join=round] (150.19,151.48) -- (150.19,143.20);

\path[draw=drawColor,line width= 0.6pt,line join=round] (150.19,151.48) -- (150.19,143.20);

\path[draw=drawColor,line width= 0.6pt,line join=round] (150.19,151.48) -- (150.19,143.20);

\path[draw=drawColor,line width= 0.6pt,line join=round] (150.19,151.48) -- (150.19,143.20);

\path[draw=drawColor,line width= 0.6pt,line join=round] (150.19,151.48) -- (150.19,143.20);

\path[draw=drawColor,line width= 0.6pt,line join=round] (150.19,151.48) -- (150.19,143.20);

\path[draw=drawColor,line width= 0.6pt,line join=round] (150.19,151.48) -- (150.19,143.20);

\path[draw=drawColor,line width= 0.6pt,line join=round] (150.19,151.48) -- (150.19,143.20);

\path[draw=drawColor,line width= 0.6pt,line join=round] (150.19,151.48) -- (150.19,143.20);

\path[draw=drawColor,line width= 0.6pt,line join=round] (150.19,151.48) -- (150.19,143.20);

\path[draw=drawColor,line width= 0.6pt,line join=round] (150.19,151.48) -- (150.19,143.20);

\path[draw=drawColor,line width= 0.6pt,line join=round] (150.19,151.48) -- (150.19,143.20);

\path[draw=drawColor,line width= 0.6pt,line join=round] (150.19,151.48) -- (150.19,143.20);

\path[draw=drawColor,line width= 0.6pt,line join=round] (150.19,151.48) -- (150.19,143.20);

\path[draw=drawColor,line width= 0.6pt,line join=round] (150.19,151.48) -- (150.19,143.20);

\path[draw=drawColor,line width= 0.6pt,line join=round] (150.19,151.48) -- (150.19,143.20);

\path[draw=drawColor,line width= 0.6pt,line join=round] (150.19,151.48) -- (150.19,143.20);

\path[draw=drawColor,line width= 0.6pt,line join=round] (150.19,151.48) -- (150.19,143.20);

\path[draw=drawColor,line width= 0.6pt,line join=round] (150.19,151.48) -- (150.19,143.20);

\path[draw=drawColor,line width= 0.6pt,line join=round] (150.19,151.48) -- (150.19,143.20);

\path[draw=drawColor,line width= 0.6pt,line join=round] (150.19,151.48) -- (150.19,143.20);

\path[draw=drawColor,line width= 0.6pt,line join=round] (150.19,151.48) -- (150.19,143.20);

\path[draw=drawColor,line width= 0.6pt,line join=round] (150.19,151.48) -- (150.19,143.20);

\path[draw=drawColor,line width= 0.6pt,line join=round] (150.19,151.48) -- (150.19,143.20);

\path[draw=drawColor,line width= 0.6pt,line join=round] (150.19,151.48) -- (150.19,143.20);

\path[draw=drawColor,line width= 0.6pt,line join=round] (150.19,151.48) -- (150.19,143.20);

\path[draw=drawColor,line width= 0.6pt,line join=round] (150.19,151.48) -- (150.19,143.20);

\path[draw=drawColor,line width= 0.6pt,line join=round] (150.19,151.48) -- (150.19,143.20);

\path[draw=drawColor,line width= 0.6pt,line join=round] (150.19,151.48) -- (150.19,143.20);

\path[draw=drawColor,line width= 0.6pt,line join=round] (150.19,151.48) -- (150.19,143.20);

\path[draw=drawColor,line width= 0.6pt,line join=round] (150.19,151.48) -- (150.19,143.20);

\path[draw=drawColor,line width= 0.6pt,line join=round] (150.19,151.48) -- (150.19,143.20);

\path[draw=drawColor,line width= 0.6pt,line join=round] (150.19,151.48) -- (150.19,143.20);

\path[draw=drawColor,line width= 0.6pt,line join=round] (150.19,151.48) -- (150.19,143.20);

\path[draw=drawColor,line width= 0.6pt,line join=round] (150.19,151.48) -- (150.19,143.20);

\path[draw=drawColor,line width= 0.6pt,line join=round] (150.19,151.48) -- (150.19,143.20);

\path[draw=drawColor,line width= 0.6pt,line join=round] (150.19,151.48) -- (150.19,143.20);

\path[draw=drawColor,line width= 0.6pt,line join=round] (150.19,151.48) -- (150.19,143.20);

\path[draw=drawColor,line width= 0.6pt,line join=round] (150.19,151.48) -- (150.19,143.20);

\path[draw=drawColor,line width= 0.6pt,line join=round] (150.19,151.48) -- (150.19,143.20);

\path[draw=drawColor,line width= 0.6pt,line join=round] (150.19,151.48) -- (150.19,143.20);

\path[draw=drawColor,line width= 0.6pt,line join=round] (150.19,151.48) -- (150.19,143.20);

\path[draw=drawColor,line width= 0.6pt,line join=round] (150.19,151.48) -- (150.19,143.20);

\path[draw=drawColor,line width= 0.6pt,line join=round] (150.19,151.48) -- (150.19,143.20);

\path[draw=drawColor,line width= 0.6pt,line join=round] (150.19,151.48) -- (150.19,143.20);

\path[draw=drawColor,line width= 0.6pt,line join=round] (150.19,151.48) -- (150.19,143.20);

\path[draw=drawColor,line width= 0.6pt,line join=round] (150.19,151.48) -- (150.19,143.20);

\path[draw=drawColor,line width= 0.6pt,line join=round] (150.19,151.48) -- (150.19,143.20);

\path[draw=drawColor,line width= 0.6pt,line join=round] (150.19,151.48) -- (150.19,143.20);

\path[draw=drawColor,line width= 0.6pt,line join=round] (150.19,151.48) -- (150.19,143.20);

\path[draw=drawColor,line width= 0.6pt,line join=round] (150.19,151.48) -- (150.19,143.20);

\path[draw=drawColor,line width= 0.6pt,line join=round] (150.19,151.48) -- (150.19,143.20);

\path[draw=drawColor,line width= 0.6pt,line join=round] (150.19,151.48) -- (150.19,143.20);

\path[draw=drawColor,line width= 0.6pt,line join=round] (150.19,151.48) -- (150.19,143.20);

\path[draw=drawColor,line width= 0.6pt,line join=round] (150.19,151.48) -- (150.19,143.20);

\path[draw=drawColor,line width= 0.6pt,line join=round] (150.19,151.48) -- (150.19,143.20);

\path[draw=drawColor,line width= 0.6pt,line join=round] (150.19,151.48) -- (150.19,143.20);

\path[draw=drawColor,line width= 0.6pt,line join=round] (150.19,151.48) -- (150.19,143.20);

\path[draw=drawColor,line width= 0.6pt,line join=round] (150.19,151.48) -- (150.19,143.20);

\path[draw=drawColor,line width= 0.6pt,line join=round] (150.19,151.48) -- (150.19,143.20);

\path[draw=drawColor,line width= 0.6pt,line join=round] (150.19,151.48) -- (150.19,143.20);

\path[draw=drawColor,line width= 0.6pt,line join=round] (150.19,151.48) -- (150.19,143.20);

\path[draw=drawColor,line width= 0.6pt,line join=round] (150.19,151.48) -- (150.19,143.20);

\path[draw=drawColor,line width= 0.6pt,line join=round] (150.19,151.48) -- (150.19,143.20);

\path[draw=drawColor,line width= 0.6pt,line join=round] (150.19,151.48) -- (150.19,143.20);

\path[draw=drawColor,line width= 0.6pt,line join=round] (150.19,151.48) -- (150.19,143.20);

\path[draw=drawColor,line width= 0.6pt,line join=round] (150.19,151.48) -- (150.19,143.20);

\path[draw=drawColor,line width= 0.6pt,line join=round] (150.19,151.48) -- (150.19,143.20);

\path[draw=drawColor,line width= 0.6pt,line join=round] (150.19,151.48) -- (150.19,143.20);

\path[draw=drawColor,line width= 0.6pt,line join=round] (150.19,151.48) -- (150.19,143.20);

\path[draw=drawColor,line width= 0.6pt,line join=round] (150.19,151.48) -- (150.19,143.20);

\path[draw=drawColor,line width= 0.6pt,line join=round] (150.19,151.48) -- (150.19,143.20);

\path[draw=drawColor,line width= 0.6pt,line join=round] (150.19,151.48) -- (150.19,143.20);

\path[draw=drawColor,line width= 0.6pt,line join=round] (150.19,151.48) -- (150.19,143.20);

\path[draw=drawColor,line width= 0.6pt,line join=round] (150.19,151.48) -- (150.19,143.20);

\path[draw=drawColor,line width= 0.6pt,line join=round] (150.19,151.48) -- (150.19,143.20);

\path[draw=drawColor,line width= 0.6pt,line join=round] (150.19,151.48) -- (150.19,143.20);

\path[draw=drawColor,line width= 0.6pt,line join=round] (150.19,151.48) -- (150.19,143.20);

\path[draw=drawColor,line width= 0.6pt,line join=round] (150.19,151.48) -- (150.19,143.20);

\path[draw=drawColor,line width= 0.6pt,line join=round] (150.19,151.48) -- (150.19,143.20);

\path[draw=drawColor,line width= 0.6pt,line join=round] (150.19,151.48) -- (150.19,143.20);

\path[draw=drawColor,line width= 0.6pt,line join=round] (150.19,151.48) -- (150.19,143.20);

\path[draw=drawColor,line width= 0.6pt,line join=round] (150.19,151.48) -- (150.19,143.20);

\path[draw=drawColor,line width= 0.6pt,line join=round] (150.19,151.48) -- (150.19,143.20);

\path[draw=drawColor,line width= 0.6pt,line join=round] (150.19,151.48) -- (150.19,143.20);

\path[draw=drawColor,line width= 0.6pt,line join=round] (150.19,151.48) -- (150.19,143.20);

\path[draw=drawColor,line width= 0.6pt,line join=round] (150.19,151.48) -- (150.19,143.20);

\path[draw=drawColor,line width= 0.6pt,line join=round] (150.19,151.48) -- (150.19,143.20);

\path[draw=drawColor,line width= 0.6pt,line join=round] (150.19,151.48) -- (150.19,143.20);

\path[draw=drawColor,line width= 0.6pt,line join=round] (150.19,151.48) -- (150.19,143.20);

\path[draw=drawColor,line width= 0.6pt,line join=round] (150.19,151.48) -- (150.19,143.20);

\path[draw=drawColor,line width= 0.6pt,line join=round] (150.19,151.48) -- (150.19,143.20);

\path[draw=drawColor,line width= 0.6pt,line join=round] (150.19,151.48) -- (150.19,143.20);

\path[draw=drawColor,line width= 0.6pt,line join=round] (150.19,151.48) -- (150.19,143.20);

\path[draw=drawColor,line width= 0.6pt,line join=round] (150.19,151.48) -- (150.19,143.20);

\path[draw=drawColor,line width= 0.6pt,line join=round] (150.19,151.48) -- (150.19,143.20);

\path[draw=drawColor,line width= 0.6pt,line join=round] (150.19,151.48) -- (150.19,143.20);

\path[draw=drawColor,line width= 0.6pt,line join=round] (150.19,151.48) -- (150.19,143.20);

\path[draw=drawColor,line width= 0.6pt,line join=round] (150.19,151.48) -- (150.19,143.20);

\path[draw=drawColor,line width= 0.6pt,line join=round] (150.19,151.48) -- (150.19,143.20);

\path[draw=drawColor,line width= 0.6pt,line join=round] (150.19,151.48) -- (150.19,143.20);

\path[draw=drawColor,line width= 0.6pt,line join=round] (150.19,151.48) -- (150.19,143.20);

\path[draw=drawColor,line width= 0.6pt,line join=round] (150.19,151.48) -- (150.19,143.20);

\path[draw=drawColor,line width= 0.6pt,line join=round] (150.19,151.48) -- (150.19,143.20);

\path[draw=drawColor,line width= 0.6pt,line join=round] (150.19,151.48) -- (150.19,143.20);

\path[draw=drawColor,line width= 0.6pt,line join=round] (150.19,151.48) -- (150.19,143.20);

\path[draw=drawColor,line width= 0.6pt,line join=round] (150.19,151.48) -- (150.19,143.20);

\path[draw=drawColor,line width= 0.6pt,line join=round] (150.19,151.48) -- (150.19,143.20);

\path[draw=drawColor,line width= 0.6pt,line join=round] (150.19,151.48) -- (150.19,143.20);

\path[draw=drawColor,line width= 0.6pt,line join=round] (150.19,151.48) -- (150.19,143.20);

\path[draw=drawColor,line width= 0.6pt,line join=round] (150.19,151.48) -- (150.19,143.20);

\path[draw=drawColor,line width= 0.6pt,line join=round] (150.19,151.48) -- (150.19,143.20);

\path[draw=drawColor,line width= 0.6pt,line join=round] (150.19,151.48) -- (150.19,143.20);

\path[draw=drawColor,line width= 0.6pt,line join=round] (150.19,151.48) -- (150.19,143.20);

\path[draw=drawColor,line width= 0.6pt,line join=round] (150.19,151.48) -- (150.19,143.20);

\path[draw=drawColor,line width= 0.6pt,line join=round] (150.19,151.48) -- (150.19,143.20);

\path[draw=drawColor,line width= 0.6pt,line join=round] (150.19,151.48) -- (150.19,143.20);

\path[draw=drawColor,line width= 0.6pt,line join=round] (150.19,151.48) -- (150.19,143.20);

\path[draw=drawColor,line width= 0.6pt,line join=round] (150.19,151.48) -- (150.19,143.20);

\path[draw=drawColor,line width= 0.6pt,line join=round] (150.19,151.48) -- (150.19,143.20);

\path[draw=drawColor,line width= 0.6pt,line join=round] (150.19,151.48) -- (150.19,143.20);

\path[draw=drawColor,line width= 0.6pt,line join=round] (150.19,151.48) -- (150.19,143.20);

\path[draw=drawColor,line width= 0.6pt,line join=round] (150.19,151.48) -- (150.19,143.20);

\path[draw=drawColor,line width= 0.6pt,line join=round] (150.19,151.48) -- (150.19,143.20);

\path[draw=drawColor,line width= 0.6pt,line join=round] (150.19,151.48) -- (150.19,143.20);

\path[draw=drawColor,line width= 0.6pt,line join=round] (150.19,151.48) -- (150.19,143.20);

\path[draw=drawColor,line width= 0.6pt,line join=round] (150.19,151.48) -- (150.19,143.20);

\path[draw=drawColor,line width= 0.6pt,line join=round] (150.19,151.48) -- (150.19,143.20);

\path[draw=drawColor,line width= 0.6pt,line join=round] (150.19,151.48) -- (150.19,143.20);

\path[draw=drawColor,line width= 0.6pt,line join=round] (150.19,151.48) -- (150.19,143.20);

\path[draw=drawColor,line width= 0.6pt,line join=round] (150.19,151.48) -- (150.19,143.20);

\path[draw=drawColor,line width= 0.6pt,line join=round] (150.19,151.48) -- (150.19,143.20);

\path[draw=drawColor,line width= 0.6pt,line join=round] (150.19,151.48) -- (150.19,143.20);

\path[draw=drawColor,line width= 0.6pt,line join=round] (150.19,151.48) -- (150.19,143.20);

\path[draw=drawColor,line width= 0.6pt,line join=round] (150.19,151.48) -- (150.19,143.20);

\path[draw=drawColor,line width= 0.6pt,line join=round] (150.19,151.48) -- (150.19,143.20);

\path[draw=drawColor,line width= 0.6pt,line join=round] (150.19,151.48) -- (150.19,143.20);

\path[draw=drawColor,line width= 0.6pt,line join=round] (150.19,151.48) -- (150.19,143.20);

\path[draw=drawColor,line width= 0.6pt,line join=round] (150.19,151.48) -- (150.19,143.20);

\path[draw=drawColor,line width= 0.6pt,line join=round] (150.19,151.48) -- (150.19,143.20);

\path[draw=drawColor,line width= 0.6pt,line join=round] (150.19,151.48) -- (150.19,143.20);

\path[draw=drawColor,line width= 0.6pt,line join=round] (150.19,151.48) -- (150.19,143.20);

\path[draw=drawColor,line width= 0.6pt,line join=round] (150.19,151.48) -- (150.19,143.20);

\path[draw=drawColor,line width= 0.6pt,line join=round] (150.19,151.48) -- (150.19,143.20);

\path[draw=drawColor,line width= 0.6pt,line join=round] (150.19,151.48) -- (150.19,143.20);

\path[draw=drawColor,line width= 0.6pt,line join=round] (150.19,151.48) -- (150.19,143.20);

\path[draw=drawColor,line width= 0.6pt,line join=round] (150.19,151.48) -- (150.19,143.20);

\path[draw=drawColor,line width= 0.6pt,line join=round] (150.19,151.48) -- (150.19,143.20);

\path[draw=drawColor,line width= 0.6pt,line join=round] (150.19,151.48) -- (150.19,143.20);

\path[draw=drawColor,line width= 0.6pt,line join=round] (150.19,151.48) -- (150.19,143.20);

\path[draw=drawColor,line width= 0.6pt,line join=round] (150.19,151.48) -- (150.19,143.20);

\path[draw=drawColor,line width= 0.6pt,line join=round] (150.19,151.48) -- (150.19,143.20);

\path[draw=drawColor,line width= 0.6pt,line join=round] (150.19,151.48) -- (150.19,143.20);

\path[draw=drawColor,line width= 0.6pt,line join=round] (150.19,151.48) -- (150.19,143.20);

\path[draw=drawColor,line width= 0.6pt,line join=round] (150.19,151.48) -- (150.19,143.20);

\path[draw=drawColor,line width= 0.6pt,line join=round] (150.19,151.48) -- (150.19,143.20);

\path[draw=drawColor,line width= 0.6pt,line join=round] (150.19,151.48) -- (150.19,143.20);

\path[draw=drawColor,line width= 0.6pt,line join=round] (150.19,151.48) -- (150.19,143.20);

\path[draw=drawColor,line width= 0.6pt,line join=round] (150.19,151.48) -- (150.19,143.20);

\path[draw=drawColor,line width= 0.6pt,line join=round] (150.19,151.48) -- (150.19,143.20);

\path[draw=drawColor,line width= 0.6pt,line join=round] (150.19,151.48) -- (150.19,143.20);

\path[draw=drawColor,line width= 0.6pt,line join=round] (150.19,151.48) -- (150.19,143.20);

\path[draw=drawColor,line width= 0.6pt,line join=round] (150.19,151.48) -- (150.19,143.20);

\path[draw=drawColor,line width= 0.6pt,line join=round] (150.19,151.48) -- (150.19,143.20);

\path[draw=drawColor,line width= 0.6pt,line join=round] (150.19,151.48) -- (150.19,143.20);

\path[draw=drawColor,line width= 0.6pt,line join=round] (150.19,151.48) -- (150.19,143.20);

\path[draw=drawColor,line width= 0.6pt,line join=round] (150.19,151.48) -- (150.19,143.20);

\path[draw=drawColor,line width= 0.6pt,line join=round] (150.19,151.48) -- (150.19,143.20);

\path[draw=drawColor,line width= 0.6pt,line join=round] (150.19,151.48) -- (150.19,143.20);

\path[draw=drawColor,line width= 0.6pt,line join=round] (150.19,151.48) -- (150.19,143.20);

\path[draw=drawColor,line width= 0.6pt,line join=round] (150.19,151.48) -- (150.19,143.20);

\path[draw=drawColor,line width= 0.6pt,line join=round] (150.19,151.48) -- (150.19,143.20);

\path[draw=drawColor,line width= 0.6pt,line join=round] (150.19,151.48) -- (150.19,143.20);

\path[draw=drawColor,line width= 0.6pt,line join=round] (150.19,151.48) -- (150.19,143.20);

\path[draw=drawColor,line width= 0.6pt,line join=round] (150.19,151.48) -- (150.19,143.20);

\path[draw=drawColor,line width= 0.6pt,line join=round] (150.19,151.48) -- (150.19,143.20);

\path[draw=drawColor,line width= 0.6pt,line join=round] (150.19,151.48) -- (150.19,143.20);

\path[draw=drawColor,line width= 0.6pt,line join=round] (150.19,151.48) -- (150.19,143.20);

\path[draw=drawColor,line width= 0.6pt,line join=round] (150.19,151.48) -- (150.19,143.20);

\path[draw=drawColor,line width= 0.6pt,line join=round] (150.19,151.48) -- (150.19,143.20);

\path[draw=drawColor,line width= 0.6pt,line join=round] (150.19,151.48) -- (150.19,143.20);

\path[draw=drawColor,line width= 0.6pt,line join=round] (150.19,151.48) -- (150.19,143.20);

\path[draw=drawColor,line width= 0.6pt,line join=round] (150.19,151.48) -- (150.19,143.20);

\path[draw=drawColor,line width= 0.6pt,line join=round] (150.19,151.48) -- (150.19,143.20);

\path[draw=drawColor,line width= 0.6pt,line join=round] (150.19,151.48) -- (150.19,143.20);

\path[draw=drawColor,line width= 0.6pt,line join=round] (150.19,151.48) -- (150.19,143.20);

\path[draw=drawColor,line width= 0.6pt,line join=round] (150.19,151.48) -- (150.19,143.20);

\path[draw=drawColor,line width= 0.6pt,line join=round] (150.19,151.48) -- (150.19,143.20);

\path[draw=drawColor,line width= 0.6pt,line join=round] (150.19,151.48) -- (150.19,143.20);

\path[draw=drawColor,line width= 0.6pt,line join=round] (150.19,151.48) -- (150.19,143.20);

\path[draw=drawColor,line width= 0.6pt,line join=round] (150.19,151.48) -- (150.19,143.20);

\path[draw=drawColor,line width= 0.6pt,line join=round] (150.19,151.48) -- (150.19,143.20);

\path[draw=drawColor,line width= 0.6pt,line join=round] (150.19,151.48) -- (150.19,143.20);

\path[draw=drawColor,line width= 0.6pt,line join=round] (150.19,151.48) -- (150.19,143.20);

\path[draw=drawColor,line width= 0.6pt,line join=round] (150.19,151.48) -- (150.19,143.20);

\path[draw=drawColor,line width= 0.6pt,line join=round] (150.19,151.48) -- (150.19,143.20);

\path[draw=drawColor,line width= 0.6pt,line join=round] (150.19,151.48) -- (150.19,143.20);

\path[draw=drawColor,line width= 0.6pt,line join=round] (150.19,151.48) -- (150.19,143.20);

\path[draw=drawColor,line width= 0.6pt,line join=round] (150.19,151.48) -- (150.19,143.20);

\path[draw=drawColor,line width= 0.6pt,line join=round] (150.19,151.48) -- (150.19,143.20);

\path[draw=drawColor,line width= 0.6pt,line join=round] (150.19,151.48) -- (150.19,143.20);

\path[draw=drawColor,line width= 0.6pt,line join=round] (150.19,151.48) -- (150.19,143.20);

\path[draw=drawColor,line width= 0.6pt,line join=round] (150.19,151.48) -- (150.19,143.20);

\path[draw=drawColor,line width= 0.6pt,line join=round] (150.19,151.48) -- (150.19,143.20);

\path[draw=drawColor,line width= 0.6pt,line join=round] (150.19,151.48) -- (150.19,143.20);

\path[draw=drawColor,line width= 0.6pt,line join=round] (150.19,151.48) -- (150.19,143.20);

\path[draw=drawColor,line width= 0.6pt,line join=round] (150.19,151.48) -- (150.19,143.20);

\path[draw=drawColor,line width= 0.6pt,line join=round] (150.19,151.48) -- (150.19,143.20);

\path[draw=drawColor,line width= 0.6pt,line join=round] (150.19,151.48) -- (150.19,143.20);

\path[draw=drawColor,line width= 0.6pt,line join=round] (150.19,151.48) -- (150.19,143.20);

\path[draw=drawColor,line width= 0.6pt,line join=round] (150.19,151.48) -- (150.19,143.20);

\path[draw=drawColor,line width= 0.6pt,line join=round] (150.19,151.48) -- (150.19,143.20);

\path[draw=drawColor,line width= 0.6pt,line join=round] (150.19,151.48) -- (150.19,143.20);

\path[draw=drawColor,line width= 0.6pt,line join=round] (150.19,151.48) -- (150.19,143.20);

\path[draw=drawColor,line width= 0.6pt,line join=round] (150.19,151.48) -- (150.19,143.20);

\path[draw=drawColor,line width= 0.6pt,line join=round] (150.19,151.48) -- (150.19,143.20);

\path[draw=drawColor,line width= 0.6pt,line join=round] (150.19,151.48) -- (150.19,143.20);

\path[draw=drawColor,line width= 0.6pt,line join=round] (150.19,151.48) -- (150.19,143.20);

\path[draw=drawColor,line width= 0.6pt,line join=round] (150.19,151.48) -- (150.19,143.20);

\path[draw=drawColor,line width= 0.6pt,line join=round] (150.19,151.48) -- (150.19,143.20);

\path[draw=drawColor,line width= 0.6pt,line join=round] (150.19,151.48) -- (150.19,143.20);

\path[draw=drawColor,line width= 0.6pt,line join=round] (150.19,151.48) -- (150.19,143.20);

\path[draw=drawColor,line width= 0.6pt,line join=round] (150.19,151.48) -- (150.19,143.20);

\path[draw=drawColor,line width= 0.6pt,line join=round] (150.19,151.48) -- (150.19,143.20);

\path[draw=drawColor,line width= 0.6pt,line join=round] (150.19,151.48) -- (150.19,143.20);

\path[draw=drawColor,line width= 0.6pt,line join=round] (150.19,151.48) -- (150.19,143.20);

\path[draw=drawColor,line width= 0.6pt,line join=round] (150.19,151.48) -- (150.19,143.20);

\path[draw=drawColor,line width= 0.6pt,line join=round] (150.19,151.48) -- (150.19,143.20);

\path[draw=drawColor,line width= 0.6pt,line join=round] (150.19,151.48) -- (150.19,143.20);

\path[draw=drawColor,line width= 0.6pt,line join=round] (150.19,151.48) -- (150.19,143.20);

\path[draw=drawColor,line width= 0.6pt,line join=round] (150.19,151.48) -- (150.19,143.20);

\path[draw=drawColor,line width= 0.6pt,line join=round] (150.19,151.48) -- (150.19,143.20);

\path[draw=drawColor,line width= 0.6pt,line join=round] (150.19,151.48) -- (150.19,143.20);

\path[draw=drawColor,line width= 0.6pt,line join=round] (150.19,151.48) -- (150.19,143.20);

\path[draw=drawColor,line width= 0.6pt,line join=round] (150.19,151.48) -- (150.19,143.20);

\path[draw=drawColor,line width= 0.6pt,line join=round] (150.19,151.48) -- (150.19,143.20);

\path[draw=drawColor,line width= 0.6pt,line join=round] (150.19,151.48) -- (150.19,143.20);

\path[draw=drawColor,line width= 0.6pt,line join=round] (150.19,151.48) -- (150.19,143.20);

\path[draw=drawColor,line width= 0.6pt,line join=round] (150.19,151.48) -- (150.19,143.20);

\path[draw=drawColor,line width= 0.6pt,line join=round] (150.19,151.48) -- (150.19,143.20);

\path[draw=drawColor,line width= 0.6pt,line join=round] (150.19,151.48) -- (150.19,143.20);

\path[draw=drawColor,line width= 0.6pt,line join=round] (150.19,151.48) -- (150.19,143.20);

\path[draw=drawColor,line width= 0.6pt,line join=round] (150.19,151.48) -- (150.19,143.20);

\path[draw=drawColor,line width= 0.6pt,line join=round] (150.19,151.48) -- (150.19,143.20);

\path[draw=drawColor,line width= 0.6pt,line join=round] (150.19,151.48) -- (150.19,143.20);

\path[draw=drawColor,line width= 0.6pt,line join=round] (150.19,151.48) -- (150.19,143.20);

\path[draw=drawColor,line width= 0.6pt,line join=round] (150.19,151.48) -- (150.19,143.20);

\path[draw=drawColor,line width= 0.6pt,line join=round] (150.19,151.48) -- (150.19,143.20);

\path[draw=drawColor,line width= 0.6pt,line join=round] (150.19,151.48) -- (150.19,143.20);

\path[draw=drawColor,line width= 0.6pt,line join=round] (150.19,151.48) -- (150.19,143.20);

\path[draw=drawColor,line width= 0.6pt,line join=round] (150.19,151.48) -- (150.19,143.20);

\path[draw=drawColor,line width= 0.6pt,line join=round] (150.19,151.48) -- (150.19,143.20);

\path[draw=drawColor,line width= 0.6pt,line join=round] (150.19,151.48) -- (150.19,143.20);

\path[draw=drawColor,line width= 0.6pt,line join=round] (150.19,151.48) -- (150.19,143.20);

\path[draw=drawColor,line width= 0.6pt,line join=round] (150.19,151.48) -- (150.19,143.20);

\path[draw=drawColor,line width= 0.6pt,line join=round] (150.19,151.48) -- (150.19,143.20);

\path[draw=drawColor,line width= 0.6pt,line join=round] (150.19,151.48) -- (150.19,143.20);

\path[draw=drawColor,line width= 0.6pt,line join=round] (150.19,151.48) -- (150.19,143.20);

\path[draw=drawColor,line width= 0.6pt,line join=round] (150.19,151.48) -- (150.19,143.20);

\path[draw=drawColor,line width= 0.6pt,line join=round] (150.19,151.48) -- (150.19,143.20);

\path[draw=drawColor,line width= 0.6pt,line join=round] (150.19,151.48) -- (150.19,143.20);

\path[draw=drawColor,line width= 0.6pt,line join=round] (150.19,151.48) -- (150.19,143.20);

\path[draw=drawColor,line width= 0.6pt,line join=round] (150.19,151.48) -- (150.19,143.20);

\path[draw=drawColor,line width= 0.6pt,line join=round] (150.19,151.48) -- (150.19,143.20);

\path[draw=drawColor,line width= 0.6pt,line join=round] (150.19,151.48) -- (150.19,143.20);

\path[draw=drawColor,line width= 0.6pt,line join=round] (150.19,151.48) -- (150.19,143.20);

\path[draw=drawColor,line width= 0.6pt,line join=round] (150.19,151.48) -- (150.19,143.20);

\node[text=drawColor,anchor=base,inner sep=0pt, outer sep=0pt, scale=  1.10] at (180.51,135.25) {$\text{log}_e1=0$};

\node[text=drawColor,anchor=base,inner sep=0pt, outer sep=0pt, scale=  1.10] at (251.24,267.82) {$\text{log}_e=x$};
\end{scope}
\begin{scope}
\path[clip] (  0.00,  0.00) rectangle (289.08,289.08);
\definecolor{drawColor}{RGB}{0,0,0}

\node[text=drawColor,anchor=base,inner sep=0pt, outer sep=0pt, scale=  1.10] at (110.05,  16.44) {x};
\end{scope}
\begin{scope}
\path[clip] (  0.00,  0.00) rectangle (289.08,289.08);
\definecolor{drawColor}{RGB}{0,0,0}

\node[text=drawColor,anchor=base west,inner sep=0pt, outer sep=0pt, scale=  1.10] at (  9.50,145.26) {y};
\end{scope}
\end{tikzpicture}

    }
    \item Note: $log_e(x) \equiv ln(x)$, pronounced the `natural logarithm' 
\end{itemize}