\section{Calculus IV}

\subsection{Fundamental Theorem of Calculus}

This brings us to the grandiose titled fundamental theorem of calculus: 

\vspace{-1em}
\begin{equation*}
    \int_a^b f(x)dx = F(b) - F(a)
\end{equation*}

\noindent Where the difference between the antiderivative at two points gives us the area under a curve between these two values. This is given such importance because it links derivatives and integrals. 

\noindent Ex:

\begin{equation*}
    \int_1^3 x^2dx = \frac{1}{3}x^3 \Big\rvert_1^3 = \frac{27}{3} - \frac{1}{3} = \frac{26}{3}
\end{equation*}

\subsection{Rules for definite integrals}

\begin{enumerate}
    \item \begin{equation*}
        \int_a^b f(x)d(x) = - \int_b^a f(x)dx
    \end{equation*}
    
    \item \begin{equation*}
        \int_a^d f(x)dx = \int_a^b f(x)dx + \int_b^c f(x)dx + \int_c^d f(x)dx \, \, \, \text{ where } \, \, \, a \leq b \leq c \leq d
    \end{equation*}

    \item \begin{equation*}
        \int_a^b - f(x)dx = - \int_a^b f(x)dx
    \end{equation*}

    \item \begin{equation*}
        \int_a^b k f(x)dx = k \int_a^b f(x)dx
    \end{equation*}

    \item \begin{equation*}
        \int_a^b \left[f(x) \pm g(x)\right]dx = \int_a^b f(x)dx \pm \int_a^b g(x)dx
    \end{equation*}
\end{enumerate}

\subsection{Integration by substitution}

\begin{enumerate}
    \item Define a new variable, $u = G(x)$ such that when written in terms of $u$ the integrand is simpler.
    \item Integrate the function, changing the limits of the integral to be from $a$ and $b$ to $g(a)$ or $g(b)$.
    \item Rewrite in terms of $x$.
\end{enumerate}

Ex:

\begin{align*}
    \int_{x=a}^{x=b} 8x(x^2 + 1)^3dx
\end{align*}

\begin{itemize}
    \item Let $u = x^2 + 1$
    \item Then, $\frac{du}{dx} = 2x$
    \item And: $\frac{1}{2x}du = dx$. So we can rewrite the above function as:
\end{itemize}

\vspace{-1em}

\begin{equation*}
    \int_a^b 8xu^3 \cdot \frac{1}{2x}
    du
\end{equation*}

We simplify to change the limits to reflect $g(u)$.

\vspace{-1em}

\begin{align*}
    & \int_{a^2 + 1}^{b^2 + 1} 4u^3 du \\
    & =  \int_{a^2 + 1}^{b^2 + 1} \frac{4}{4}u^4 = u^4 \Big\vert_{a^2 + 1}^{b^2 + 1}
\end{align*}

Then we rewrite in terms of x:

\vspace{-1em}

\begin{equation*}
    (x^2 + 1)^4 \Big\vert_a^b
\end{equation*}  

\subsubsection{Some examples}

\begin{itemize}
    \item 
    \begin{align*}
        & \int_a^b 3x^2 \sqrt{x^3 + 1}dx \\
        & \text{Let } u = x^3 + 1 \\
        & \frac{du}{dx} = 3x^2, \, \, \frac{1}{3x^2}du = dx \\
        & = \int_{a^3+1}^{b^3 + 1} 3x^2 \sqrt{u}\frac{1}{3x^2}du \\
        & = \int_{a^3 + 1}^{b^3 + 1} \sqrt{u}\,du = \int_{a^3 + 1}^{b^3 + 1} u^{\frac{1}{2}}du \\
        & = \frac{2}{3}u^{\frac{3}{2}}\Big\vert_{a^3 + 1}^{b^3 + 1} = \frac{2}{3}(x^3 + 1)^{\frac{3}{2}} \Big\vert_a^b
    \end{align*}

    \item 
    \begin{align*}
        & \int_a^b x^3e^{x^3}dx \\
        & \text{Let } u = x^3, \frac{du}{dx} = 3x^2, dx = \frac{du}{3x^2} \\
        & \int_{a^3}^{b^3} x^2e^u \frac{du}{3x^2} = \frac{1}{3}\int_{a^3}^{b^3} e^u du \\
        & = \frac{1}{3} e^u \Big\vert_{a^3}^{b^3} = \frac{1}{3}e^{x^3}\Big\vert_{a}^{b}
    \end{align*}

    \item 
    \begin{align*}
        & \int_a^b \frac{2 - x}{\sqrt{2x^2 - 8x + 1}}dx \\
        & \text{Let } u = 2x^2 - 8x + 1 \\
        & \frac{du}{dx} = 4x - 8, dx = \frac{du}{4x - 8} \\
        & \int (2 - x)u^{-\frac{1}{2}}\frac{du}{-4(2-x)} = - \frac{1}{4}\int u^{-\frac{1}{2}}du \\
        & = -\frac{1}{4} \cdot \frac{u^{\frac{1}{2}}}{\frac{1}{2}} \\
        & -\frac{1}{2}u^{\frac{1}{2}}\Big\vert_a^b
    \end{align*}
\end{itemize}

\subsection{Integration by parts}

Sometimes we get functions where integrating by substitution does not gain us any leverage. For these, we can \emph{integrate by parts}. Say we have two functions $u$ and $v$, which are differentiable. We can use the product rule to show that:

\begin{equation*}
    \frac{d}{dx}(uv) = v \frac{du}{dx} + u \frac{dv}{dx} 
\end{equation*}

And rearrange: 

\begin{equation*}
    u \frac{dv}{dx} = \frac{d}{dx}(uv) - v \frac{du}{dx}
\end{equation*}

Integrate with respect to $x$, and we get the formula for integration by parts:

\begin{equation*}
    \int u dv = uv - \int v du
\end{equation*}

An example: 

\begin{align*}
    &\int_a^b x^2 \text{ln}x dx \\
    &u = \text{ln}x \Rightarrow du = \frac{1}{x}dx \\
    &dv = x^2 dx \text{ Integrate } \Rightarrow v = \frac{x^3}{3} \\
    &\text{So the original equation equals:} \\
    &\frac{1}{3}x^3 \text{ln}x - \int\frac{1}{3}x^3 \cdot\frac{1}{x}dx \\
    & = \frac{1}{3}x^3 \text{ln}x - \frac{1}{3}\int x^2 dx = \left[\frac{1}{3}x^3 \text{ln}x - \frac{1}{3}\cdot\frac{1}{3}x^3\right]_a^b
\end{align*}

One more: 

\begin{align*}
    & \int_1^3 xe^x dx \\
    & u = x, du = dx, dv = e^xdx, v = e^x \\
    & \int_1^3 xe^x dx = xe^x - \int e^xdx \\ 
    & = \left[xe^x - e^x\right]_1^3
\end{align*}


\subsection{Improper integrals}

What if we integrate to $\infty$ instead of a integer? We take the limit when we get to plugging our bounds in.

\begin{align*}
    & \int_1^\infty \frac{1}{\lambda}e^{-\lambda x}dx \\
    & = \frac{1}{\lambda}\int_1^\infty -\frac{1}{\lambda} e^{\lambda x} = \left[-\frac{1}{\lambda^2}e^{-\lambda x}\right]_1^3 \\ 
    & = \left[\lim_{x \rightarrow \infty} -\frac{1}{\lambda^2} e^{-\lambda x}\right] - \left[-\frac{1}{\lambda^2} e^{-\lambda 1}\right] \\
    & = 0 + \left[\frac{1}{\lambda^2} e^{-\lambda}\right]
\end{align*}